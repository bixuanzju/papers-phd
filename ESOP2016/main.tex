% Springer template
\documentclass[a4paper]{llncs}

%% -- Packages Imports --

% AMS stuff
\usepackage{amsmath}
\usepackage{amssymb}
\let\proof\relax
\let\endproof\relax
\usepackage{amsthm}
\usepackage{mathtools}
\usepackage{xspace}
\usepackage{setspace}
\usepackage{comment}
\usepackage{cite}

% Language
\usepackage{csquotes}
\usepackage[british]{babel}
\MakeOuterQuote{"}


\newcommand{\name}{{\bf $\lambda_{\star}^{\mu}$}\xspace}
\newcommand{\coc}{{\bf $\lambda C$}\xspace}
\newcommand{\ecore}{$\lambda_{\star}$\xspace}
\newcommand{\cc}{$\lambda C$\xspace}
\newcommand{\sufcc}{{\bf Fun}\xspace}
\newcommand{\fold}{{\bf $\mathsf{fold}$}\xspace}
\newcommand{\unfold}{{\bf $\mathsf{unfold}$}\xspace}

\newcommand{\authornote}[3]{{\color{#2} {\sc #1}: #3}}
\newcommand\bruno[1]{\authornote{bruno}{red}{#1}}
\newcommand\jeremy[1]{\authornote{jeremy}{blue}{#1}}
\newcommand\linus[1]{\authornote{linus}{green}{#1}}
\newcommand{\fixme}[1]{{\color{red} #1}}

% Hyper links
\usepackage{hyperref}
\hypersetup{
   colorlinks,
   citecolor=black,
   filecolor=black,
   linkcolor=black,
   urlcolor=black
}

% Compact list
\usepackage{paralist}

% Figure import
\usepackage{graphicx}
\usepackage{float}

% Code highlighting
\usepackage{listings}
\usepackage{xcolor}
\newcommand{\hl}[2][gray!40]{\colorbox{#1}{#2}}
\newcommand{\hlmath}[2][gray!40]{%
  \colorbox{#1}{$\displaystyle#2$}}

% Theorem
% \usepackage{amsthm}
% \newtheorem{thm}{Theorem}[section]
% \newtheorem{lem}[thm]{Lemma}
% \newtheorem{dfn}[thm]{Definition}

%% Typesetting inference rules
\usepackage{styles/mathpartir}  % by Didier Rémy (http://gallium.inria.fr/~remy/latex/mathpartir.html

% Ott includes
\input{sections/expcore.ott.tex}
\input{sections/otthelper.tex}

% lhs2tex
\usepackage{mylhs2tex}

% Subsection style
% \usepackage{titlesec}
% \titleformat{\subsubsection}[runin]{\bfseries\itshape\normalsize}{}{0em}{}[$\;$]

% Table
\usepackage{booktabs}
\usepackage{tabularx}
\usepackage[para,online,flushleft]{threeparttable}

%% -- Packages Imports --

% Main
\begin{document}

\mainmatter

% Title
\title{A Simple Yet Expressive Calculus for Modern Functional Languages}
%\title{A Dependently-typed Intermediate Language with General Recursion}
%\subtitle{Or: Decidable Type-Checking in the presence of
%  Type-Level General Recursion}

\author{Bi Xuan \and Yang Yanpeng \and Bruno C. d. S. Oliveira}

\institute{The University of Hong Kong\\
\email{\{xbihku, yangyp\}@connect.hku.hk} \\
\email{bruno@cs.hku.hk}}

% \authorinfo{Foo \and Bar \and Baz}
%            {The University of Foo}
%            {\{foo,bar,baz\}@foo.edu}

\maketitle

% Abstract
\begin{abstract}
Typed core (or intermediate) languages for modern 
functional languages, such as Haskell or ML, 
are becoming more and more complex. This is a natural tendency.
Programmers and language designers wish for more expressive and
powerful source-language constructs. In turn this requires new, more
powerful constructs in core languages. Unfortunately, the added
complexity means that the meta-theory and implementation of such core
languages becomes significantly harder.

This paper proposes a simple yet expressive core calculus (\name),
which has a fraction of the language constructs of existing core
languages. The key to simplicity is the combination of two ideas. The
first idea is to use a Pure Type Systems (PTS) style of syntax that
unifies the various syntactic levels of the language. However, this
creates an immediate challenge: with types and terms unified, the
\emph{decidability} of type checking requires type-level computation
to terminate, but with general recursion it is hard to have such
guarantee. The second idea, inspired by the traditional treatment of
iso-recursive types, is to solve this challenge by making each
type-level computation step explicit. The usefulness of \name is
illustrated by a light surface language built on top of \name, which
supports many advanced programming language features of
state-of-the-art functional languages. 
%The main limitation of \name is
%the absence of a more expressive form of type-equality, which is left
%for future work.
\end{abstract}

% Category, terms & keywords
% \category{D.3.1}{Programming Languages}{Formal Definitions and Theory}
% \terms Languages, Design
% \keywords Dependent types, Intermediate language

%% -- Starting Point -- 

\section{Introduction}

Modern statically typed functional languages (such as ML, Haskell,
Scala or OCaml) have increasingly expressive type systems. Often these
large source languages are translated into a much smaller typed core
language. The choice of the core language is essential to ensure that
all the features of the source language can be encoded. For a simple
polymorphic functional language it is possible to pick a
variant of System $F$~\cite{systemfw,Reynolds:1974} as a core
language. However, the desire for more expressive type system features
puts pressure on the core languages, often requiring them to be
extended to support new features.  For example, if the source language
supports \emph{higher-kinded types} or \emph{type-level functions}
then System $F$ is not expressive enough and can no longer be used as
the core language. Instead another core language that does provide
support for higher-kinded types, such as
System~$F_{\omega}$~\cite{systemfw}, needs to be used. Of course the
drive to add more and more advanced type-level features means that
eventually the core language needs to be extended again. Indeed modern
functional languages like Haskell use specially crafted core
languages, such as System $F_{C}$~\cite{fc}, that provide support for all
modern features of Haskell. Although \emph{extensions} of System
$F_{C}$~\cite{fc:pro,Eisenberg:2014} satisfy the current needs of
modern Haskell, it is very likely to be extended again in the
future~\cite{fc:kind}. Moreover System $F_{C}$ has grown to be a relatively
large and complex language, with multiple syntactic levels, and dozens
of language constructs.

\begin{comment}
However System~$F_{\omega}$ is
significantly more complex than System F and thus harder to
maintain. If later a new feature, such as \emph{kind polymorphism}, is
desired the core language may need to be changed again to account for
the new feature, introducing at the same time new sources of
complexity. Indeed the core language for modern versions of 
functional languages are quite complex, having multiple syntactic 
sorts (such as terms, types and kinds), as well as dozens of 
language constructs~\cite{}\bruno{$F_{C}$}. 
\end{comment}

The more expressive type (and kind) systems become, the more types become similar
to the terms. Therefore a natural idea is to unify terms and
types. There are obvious benefits in this approach: only one syntactic
level (terms) is needed; and there are much less language constructs,
making the core language easier to reason, implement and maintain. At the same
time the core language becomes more expressive, giving us for free
many useful language features. Moreover, due to the inherent
expressiveness, extensions are less likely to be required.
\emph{Pure type systems} (PTS)~\cite{handbook} build
on such observations and show how a whole family of type systems
(including System $F$ and System $F_{\omega}$) can be implemented
using just a single syntactic form. With the added expressiveness it
is even possible to have type-level programs expressed using the same
syntax as terms, as well as dependently typed programs~\cite{coc}.
Because the idea of using a unified syntax is so appealing several
researchers have in the past considered such an
option for implementing functional languages~\cite{cayenne, typeintype, pts:henk}.

However having the same syntax for types and terms can also be
problematic. Usually type systems based on PTS have a conversion rule
to support type-level computation.  In such type systems ensuring the
\emph{decidability} of type checking requires type-level computation
to terminate. When the syntax of types and terms is the same, the
decidability of type checking is usually dependent on the strong
normalization of the calculus. An example is the proof of decidability
of type checking for the \emph{calculus of constructions}~\cite{coc}
(and other normalizing PTS), which depends on strong normalization
~\cite{pts:normalize}. Modern dependently
typed languages such as Idris~\cite{idris} and Agda~\cite{agda}, which are also
built on a unified syntax for types and terms, require strong
normalization as well: all recursive programs must pass a termination
checker.  An unfortunate consequence of coupling
decidability of type checking and strong normalization is that adding
(unrestricted) general recursion to such calculi is difficult. Indeed
past work on using a simple PTS-like calculi to model functional languages
with unrestricted general recursion, had to give up on decidability of
type-checking~\cite{cayenne, typeintype}.
%There
%is a clear tension between decidability of type checking and allowing
%general recursion in calculi with unified syntax.

This paper proposes \name: a simple yet expressive call-by-name
variant of the calculus of constructions, which has a fraction of the
language constructs of existing core languages. The key challenge
solved in this work is how to define a calculus comparable in
simplicity to the calculus of constructions, while featuring both
general recursion and decidable type checking. The main idea, 
inspired by the traditional treatment of \emph{iso-recursive
  types}~\cite{tapl}, is to recover decidable type-checking by making each
type-level computation step explicit, i.e., each beta reduction or
expansion at the type level is controlled by a \emph{type-safe}
cast. Since single computation steps are trivially terminating, decidability
of type checking is possible even in the presence of non-terminating
programs at the type level.  At the same time term-level programs
using general recursion work as in any conventional functional
languages, and can even be non-terminating.

\begin{comment}
For example, if a type-level program requires two beta reductions to
reach normal form, then two casts are needed in the program. If a
non-terminating program is used at the type level, it is not possible
to cause non-termination in the type checker, because that would
require a program with an infinite number of casts. Therefore, since
single beta-steps are trivially terminating, decidability of type
checking is possible even in the presence of non-terminating programs
at the type level.  At the same time term-level programs using general
recursion work as in any conventional functional languages, and can
even be non-terminating.
\end{comment}

Our motivation to develop \name is to use it as a simpler alternative
to existing core languages for functional programming. We focus on traditional
functional languages like ML or Haskell extended with many interesting
type-level features, but perhaps not the \emph{full power} of
dependent types.  The paper shows how many of programming language
features of Haskell, including some of the latest extensions, can be
encoded in \name via a surface language. The surface
language supports \emph{algebraic datatypes}, \emph{higher-kinded
  types}, \emph{nested datatypes}~\cite{nesteddt}, \emph{kind
  polymorphism}~\cite{fc:pro} and \emph{datatype
  promotion}~\cite{fc:pro}.  This result is interesting because \name
is a minimal calculus with only 8 language constructs and a single
syntactic sort. In contrast the latest versions of System $F_{C}$
(Haskell's core language) have multiple syntactic sorts and dozens of
language constructs.
%Even if support for equality and
%coercions, which constitutes a significant part of System $F_{C}$,
%would be removed the resulting language would still be significantly
%larger and more complex than \name.

It is worth emphasizing that \name does sacrifice having an expressive form
of type equality to gain the ability of doing arbitrary general
recursion at the term level.  Nevertheless, 
the core language (System $F_{C}$) of Haskell also comes with a similarly weak
notion of type equality.  In both System $F_{C}$ and \name, type
equality in \name is purely syntactic (modulo alpha-conversion).

A non-goal of the current work (although a worthy avenue for future
work) is to use \name as a core language for modern dependently typed
languages like Agda or Idris. In contrast to \name, those languages
use a more powerful notion of equality. In particular \name
currently lacks full-reduction and it is unable to exploit injectivity 
properties when comparing two types for equality. Moreover,
\name (and also System $F_{C}$) lack \emph{logical consistency}:
that is ensuring the soundness of proofs written as programs.
This is in contrast to dependently typed languages, where logical
consistency is typically ensured.
Various researchers~\cite{zombie:popl14,zombie:thesis,Swamy2011} have been investigating how to combine logical
consistency, general recursion and dependent types. However, this is
usually done by having the type system carefully control the total and
partial parts of computation, making those calculi significantly more
complex than \name or the calculus of constructions. In
\name, logical consistency is traded by the simplicity of the system.

\begin{comment}
In particular
the treatment of type-level computation in \name shares similar ideas
with Haskell. Although Haskell's surface language provides a rich set
of mechanisms to do type-level computation~\cite{}, the core language
lacks fundamental mechanisms todo type-level computation. Type
equality in System $F_{C}$ is, like in \name, purely syntactic (modulo
alpha-conversion).
\end{comment}

\begin{comment}
 and there is no type-level
abstraction. In other words in Haskell, mechanisms such as type
classes and type families

Although it may seem that forcing each step of computation 
at the type-level to be explicit will prevent convinient use of 
type-level computation.

Point about the treatment of type-level computation in Haskell. Haskell's
core language has type applications, but no type-level lambda. Equality 
is syntactic modulo alpha-conversion. This design choice was rooted in the 
desire to support Hindley-Milner type-inference... 
\end{comment}

In summary, the contributions of this work are:

\begin{itemize}
\item {\bf The \name calculus:} A simple core calculus for functional programming, that collapses terms, types and
  kinds into the same hierarchy and supports general recursion. \name
  is type-safe and the type system is decidable.

\item {\bf One-step casts and a generalization of iso-recursive types:} \name 
 generalizes iso-recursive types by making all type-level computation
 steps explicit via \emph{one-step casts}. In \name the combination of
  one-step casts and recursion subsumes iso-recursive types.

\item {\bf An expressive surface language}, built on top of \name,
  that supports datatypes, pattern matching and various advanced
  language extensions of Haskell. The type safety of the type-directed
  translation to \name is proved.

\item {\bf A prototype implementation:} The implementation of \name is
  available\footnote{\url{https://github.com/bixuanzju/full-version}}.
\end{itemize}

\begin{comment}
\begin{enumerate}[a)]
\item Motivations:

\begin{itemize}

\item Because of the reluctance to introduce dependent
  types\footnote{This might be changed in the near future. See
    \url{https://ghc.haskell.org/trac/ghc/wiki/DependentHaskell/Phase1}.},
  the current intermediate language of Haskell, namely System $F_C$
  \cite{fc}, separates expressions as terms, types and kinds, which
  brings complexity to the implementation as well as further
  extensions \cite{fc:pro,fc:kind}.

\item Popular full-spectrum dependently typed languages, like Agda,
  Coq, Idris, have to ensure the termination of functions for the
  decidability of proofs. No general recursion and the limitation of
  enforcing termination checking make such languages impractical for
  general-purpose programming.

\item We would like to introduce a simple and compiler-friendly
  dependently typed core language with only one hierarchy, which
  supports general recursion at the same time.

\end{itemize}

\item Contribution:

\begin{itemize}

\item A core language based on Calculus of Constructions (CoC) that
  collapses terms, types and kinds into the same hierarchy.

\item General recursion by introducing recursive types for both terms
  and types by the same $\mu$ primitive.

\item Decidable type checking and managed type-level computation by
  replacing implicit conversion rule of CoC with generalized
  \textsf{fold}/\textsf{unfold} semantics.

\item First-class equality by coercion, which is used for encoding
  GADTs or newtypes without runtime overhead.

\item Surface language that supports datatypes, pattern matching and
  other language extensions for Haskell, and can be encoded into the
  core language.

\end{itemize}


\end{enumerate}
\end{comment}



% \section{Overview}


% - Shallow embedding in Haskell (2 interpretations);
% 
% - How to compose? possible but lots of boilerplate;
% 
% - Finally tagless solves some problems, but how about dependencies?
% Still some boilerplate needed. 
% 
% - Introduce the solution in our calculus. Show that we can do
% everything finally tagless can + more because we have 
% distributivity in the type system.

%-------------------------------------------------------------------------------
\section{Compositional Programming}
\label{sec:overview}

% \bruno{Do we need something about easily adding new cases? Although
% this is a solved problem, people may wonder about this? Perhaps
% we need some text (1 or 2 sentences) at least. }


To demonstrate the compositional properties of \fnamee we use Gibbons and Wu's shallow embeddings of
parallel prefix circuits~\cite{DBLP:conf/icfp/GibbonsW14}. By means of several different shallow
embeddings, we first illustrate the short-comings of a state-of-the-art
compositional approach, popularly known as a \emph{finally tagless}
encoding~\cite{CARETTE_2009}, in Haskell.
Next we show how parametric polymorphism and distributive intersection types provide
a more elegant and compact solution in \sedel~\cite{bi_et_al:LIPIcs:2018:9214}, a source language built on top of
our \fnamee calculus.


%- - - - - - - - - - - - - - - - - - - - - - - - - - - - - - - - - - - - - - - - 
\subsection{A Finally Tagless Encoding in Haskell}

The circuit DSL represents networks that map a number of inputs (known as the width) of some type $A$ onto
the same number of outputs of the same type. The outputs combine (with repetitions) one or more
inputs using a binary associative operator $\oplus : A \times A \to A$.
A particularly interesting class of circuits that can be expressed in the DSL are
\emph{parallel prefix circuits}. These represent computations that take $n > 0$
inputs $x_1, \ldots, x_n$ and produce $n$ outputs $y_1, \ldots, y_n$, where
$y_i = x_1 \oplus x_2 \oplus \ldots \oplus x_i$.

The DSL features 5 language primitives: two basic circuit constructors and
three circuit combinators. These are captured in the Haskell type class \lstinline[language=haskell]{Circuit}:
\lstinputlisting[language=haskell,linerange=5-10]{./examples/Scan.hs}% APPLY:linerange=DSL_DEF
An \lstinline[language=haskell]{identity} circuit with $n$ inputs $x_i$,  has
$n$ outputs $y_i = x_i$. A \lstinline[language=haskell]{fan} circuit has $n$ inputs $x_i$ and $n$
outputs $y_i$, where $y_1 = x_1$ and $y_j = x_1 \oplus x_j$ ($j > 1)$.
The binary \lstinline[language=haskell]{beside} combinator puts two circuits in parallel; the combined circuit
takes the inputs of both circuits to the outputs of both circuits.
The binary \lstinline[language=haskell]{above} combinator connects the outputs of the first circuit to
the inputs of the second; the width of both circuits has to be same.
Finally,
\lstinline[language=haskell]{stretch ws c} interleaves the wires of circuit \lstinline[language=haskell]{c} with
bundles of additional wires that map their input straight on their output.
The \lstinline[language=haskell]{ws} parameter specifies the width of the consecutive bundles;
the $i$th wire of \lstinline[language=haskell]{c} is preceded by a bundle of width $\textit{ws}_i - 1$.

%- - - - - - - - - - - - - - - - - - - - - - - - - - - - - - - - - - - - - - - -

\begin{figure}[t]
  \begin{subfigure}[b]{0.44\textwidth}
    \lstinputlisting[language=haskell,linerange=15-22]{./examples/Scan.hs}% APPLY:linerange=DSL_WIDTH
    \subcaption{Width embedding}
  \end{subfigure} ~
  \begin{subfigure}[b]{0.57\textwidth}
    \lstinputlisting[language=haskell,linerange=27-34]{./examples/Scan.hs}% APPLY:linerange=DSL_DEPTH
    \subcaption{Depth embedding}
  \end{subfigure}
  \caption{Two finally tagless embeddings of circuits.}\label{fig:finally-tagless}
\end{figure}


\paragraph{Basic width and depth embeddings.}

\Cref{fig:finally-tagless} shows two simple shallow embeddings, which represent a circuit
respectively in terms of its width and its depth. The former denotes the number
of inputs/outputs of a circuit, while the latter is the maximal number of
$\oplus$ operators between any input and output.
Both definitions follow the same setup: a new Haskell datatype
(\lstinline[language=haskell]{Width}/\lstinline[language=haskell]{Depth}) wraps the primitive result value and provides an
instance of the \lstinline[language=haskell]{Circuit} type class that interprets the 5 DSL primitives
accordingly.
The following code creates a so-called Brent-Kung parallel prefix circuit~\cite{brent1980chip}:
\lstinputlisting[language=haskell,linerange=39-42]{./examples/Scan.hs}% APPLY:linerange=DSL_E1
Here \lstinline[language=haskell]{e1} evaluates to \lstinline[language=haskell]$W {width = 4}$. If we want to know the
depth of the circuit, we have to change type signature to \lstinline[language=haskell]{Depth}.

%- - - - - - - - - - - - - - - - - - - - - - - - - - - - - - - - - - - - - - - - 
\paragraph{Interpreting multiple ways.}

Fortunately, with the help of polymorphism we can define a type
of circuits that support multiple interpretations at once.
\lstinputlisting[language=haskell,linerange=47-47]{./examples/Scan.hs}% APPLY:linerange=DSL_FORALL
This way we can provide a single Brent-Kung parallel prefix circuit definition that can be reused
for different interpretations.
\lstinputlisting[language=haskell,linerange=51-54]{./examples/Scan.hs}% APPLY:linerange=DSL_BRENT
A type annotation then selects the desired interpretation.
For instance, \lstinline[language=haskell]{brentKung :: Width} yields the width and
\lstinline[language=haskell]{brentKung :: Depth} the depth.

%- - - - - - - - - - - - - - - - - - - - - - - - - - - - - - - - - - - - - - - - 
\paragraph{Composition of embeddings.}

What is not ideal in the above code is that the same \lstinline[language=haskell]{brentKung}
circuit is processed twice, if we want to execute both interpretations. We can do 
better by processing the circuit only once, computing both interpretations simultaneously.
The finally tagless encoding achieves this with a boilerplate instance
for tuples of interpretations.
\lstinputlisting[language=haskell,linerange=59-64]{./examples/Scan.hs}% APPLY:linerange=DSL_TUPLE
Now we can get both embeddings simultaneously as follows:
\lstinputlisting[language=haskell,linerange=68-69]{./examples/Scan.hs}% APPLY:linerange=DSL_E12
This evaluates to \lstinline[language=haskell]$(W {width = 4}, D {depth = 2})$.

%- - - - - - - - - - - - - - - - - - - - - - - - - - - - - - - - - - - - - - - - 
\paragraph{Composition of dependent interpretations.}

The composition above is easy because the two embeddings are
orthogonal. In contrast, the composition of dependent interpretations is
rather cumbersome in the standard finally tagless setup. An example of the
latter is the interpretation of circuits as their well-sizedness, which
captures whether circuits are well-formed. This interpretation depends on the
interpretation of circuits as their width.\footnote{Dependent recursion schemes
are also known as \emph{zygomorphism}~\cite{fokkinga1989tupling} after the ancient Greek word \emph{\textzeta\textupsilon\textgamma\textomikron\textnu}
for yoke. We have labeled the \lstinline{Width} field with \lstinline{ox} because it is pulling the yoke.}
\lstinputlisting[language=haskell,linerange=74-81]{./examples/Scan.hs}% APPLY:linerange=DSL_WS
The \lstinline[language=haskell]{WellSized} datatype represents the well-sizedness of a circuit with
a Boolean, and also keeps track of the circuit's width. The 5 primitives
compute the well-sizedness in terms of both the width and well-sizedness of the subcomponents.
What makes the code cumbersome is that it has to explicitly delegate to the \lstinline[language=haskell]{Width}
interpretation to collect this additional information.

With the help of a substantially more complicated setup that features a dozen
Haskell language extensions, and advanced programming techniques, we can make
the explicit delegation implicit (see the appendix). Nevertheless,
that approach still requires \emph{a lot of boilerplate} that needs to be repeated for
each DSL, as well as explicit projections that need to be written in each
interpretation. Another alternative Haskell encoding that also enables
multiple dependent interpretations is proposed by Zhang and Oliveira~\cite{zhang19shallow},
but it does not eliminate the explicit delegation and still requires
substantial amounts of boilerplate.
A final remark is that adding new primitives (e.g.,
a ``right stretch'' \lstinline{rstretch}
combinator~\cite{hinze2004algebra}) can also be easily 
achieved~\cite{emgm}.

 
%- - - - - - - - - - - - - - - - - - - - - - - - - - - - - - - - - - - - - - - - 
\subsection{The \sedel Encoding}

\sedel is a source language that elaborates to \fnamee, adding
a few convenient source level constructs.
The \sedel setup of the circuit DSL is similar to the finally tagless
approach. Instead of a \lstinline[language=haskell]{Circuit c} type class, there is a \lstinline{Circuit[C]}
type that gathers the 5 circuit primitives in a record. Like in Haskell, the type
parameter \lstinline{C} expresses that the interpretation of circuits
is a parameter.
\lstinputlisting[linerange=42-44]{./examples/scan.sl}% APPLY:linerange=SEDEL_DEF
As a side note if a new constructor (e.g., \lstinline{rstretch}) is
needed, then this is done by means of
intersection types (\lstinline{&} creates an intersection type) in \sedel:
\lstinputlisting[linerange=49-49]{./examples/scan.sl}% APPLY:linerange=SEDEL_DEF2

%- - - - - - - - - - - - - - - - - - - - - - - - - - - - - - - - - - - - - - - - 
% \paragraph{Basic width and depth embeddings.}

\begin{figure}[t]
\lstinputlisting[linerange=59-65]{./examples/scan.sl}% APPLY:linerange=SEDEL_WIDTH
\hrule
\lstinputlisting[linerange=74-80]{./examples/scan.sl}% APPLY:linerange=SEDEL_DEPTH
\caption{Two \sedel embeddings of circuits.}
\label{fig:sedel}
\end{figure}

\Cref{fig:sedel} shows the two basic shallow embeddings for width and
depth. In both cases, a named \sedel definition
replaces the corresponding unnamed
Haskell type class instance in providing the implementations of the 5 language
primitives for a particular interpretation.


The use of the \sedel embeddings is different from that of their Haskell
counterparts. Where Haskell implicitly selects the appropriate type class
instance based on the available type information, in \sedel the programmer
explicitly selects the implementation following the style used by
object algebras.
The following code does this by
% creating an object \lstinline{l1} out of the \lstinline{language1}
% trait and then
building a circuit with \lstinline{l1} (short for \lstinline{language1}).
\lstinputlisting[linerange=85-88]{./examples/scan.sl}% APPLY:linerange=SEDEL_E1
Here \lstinline{e1} evaluates to \lstinline${width = 4}$. If we want to know the
depth of the circuit, we have to replicate the code with \lstinline{language2}.

%- - - - - - - - - - - - - - - - - - - - - - - - - - - - - - - - - - - - - - - - 
\paragraph{Dynamically reusable circuits.}

Just like in Haskell, we can use polymorphism to define a type
of circuits that can be interpreted with different languages.
\lstinputlisting[linerange=93-93]{./examples/scan.sl}% APPLY:linerange=SEDEL_FORALL
In contrast to the Haskell solution, this implementation explicitly accepts
the implementation.
\lstinputlisting[linerange=99-104]{./examples/scan.sl}% APPLY:linerange=SEDEL_BRENT

%- - - - - - - - - - - - - - - - - - - - - - - - - - - - - - - - - - - - - - - - 
\paragraph{Automatic composition of languages.}

Of course, like in Haskell we can also compute both results simultaneously.
However, unlike in Haskell, the composition of the two interpretation requires
no boilerplate whatsoever---in particular, there is no \sedel counterpart of the
\lstinline[language=haskell]{Circuit (c1, c2)} instance. Instead, we can just compose the two interpretations
with the term-level merge operator (\lstinline{,,}) and specify the desired type \lstinline{Circuit[Width & Depth]}.
\lstinputlisting[linerange=109-110]{./examples/scan.sl}% APPLY:linerange=SEDEL_E3
Here the use of the merge operator creates a term with the intersection type
\lstinline{Circuit[Width] & Circuit[Depth]}. Implicitly, the \sedel type system
takes care of the details, turning this intersection type into
\lstinline{Circuit[Width & Depth]}. This is possible because intersection (\lstinline{&}) distributes over function and record types (a distinctive feature of BCD-style subtyping).

%- - - - - - - - - - - - - - - - - - - - - - - - - - - - - - - - - - - - - - - - 
\paragraph{Composition of dependent interpretations.}

In \sedel the composition scales nicely to dependent interpretations.
For instance, the well-sizedness interpretation can be expressed without
explicit projections.
\lstinputlisting[linerange=119-127]{./examples/scan.sl}% APPLY:linerange=SEDEL_WS
% It may be instructive to show the type of \lstinline{language4}:
% \begin{lstlisting}
% { identity : Int -> WellSized
% , fan      : Int -> WellSized
% , above    : WellSized & Width -> WellSized & Width -> WellSized
% , beside   : WellSized -> WellSized -> WellSized
% , stretch  : List[Int] -> WellSized & Width -> WellSized
% }
% \end{lstlisting}
Here the \lstinline{WellSized & Width} type in the \lstinline{above} and \lstinline{stretch} cases
expresses that both the well-sizedness and width of subcircuits must be given,
and that the width implementation is left as a dependency---when \lstinline{language4} is used,
then the width implementation must be provided.
Again, the distributive properties of \lstinline{&} in the type system take care of
merging the two interpretations.
\lstinputlisting[linerange=142-143]{./examples/scan.sl}% APPLY:linerange=SEDEL_E4

%- - - - - - - - - - - - - - - - - - - - - - - - - - - - - - - - - - - - - - - - 
\paragraph{Disjoint polymorphism and dynamic merges.}

While it may seem from the above examples that definitions have to be merged
statically, \sedel in fact supports dynamic merges. For instance, we can
encapsulate the merge operator in the \lstinline{combine} function while
abstracting over the two components \lstinline{x} and \lstinline{y} that are merged
as well as over their types \lstinline{A} and \lstinline{B}.
\lstinputlisting[linerange=132-132]{./examples/scan.sl}% APPLY:linerange=SEDEL_COMBINE
This way the components \lstinline{x} and \lstinline{y} are only known at runtime and
thus the merge can only happen at that time.
The types \lstinline{A} and \lstinline{B} cannot be chosen entirely freely. For
instance, if both components would contribute an implementation for the same
method, which implementation is provided by the combination would be ambiguous.
To avoid this problem the two types \lstinline{A} and \lstinline{B} have to be
\emph{disjoint}. This is expressed in the disjointness constraint \lstinline{* A}
on the quantifier of the type variable \lstinline{B}. If a quantifier mentions
no disjointness constraint, like that of \lstinline{A}, it defaults to the
trivial \lstinline{* Top} constraint which implies no restriction.
% With \lstinline{combine},
% we can rewrite \lstinline{l3} as follows (note that \lstinline{Width} and \lstinline{Depth} are disjoint):
% \lstinputlisting[linerange=137-137]{./examples/scan.sl}% APPLY:linerange=SEDEL_L3


% We can extend our circuit DSL with additional features.
% Suppose we 
% \begin{Verbatim}[fontsize=\small]
%   addBelow[C,S,R * {below: C -> C -> C, above : C -> C -> C}](lang: Trait[S,{above : C -> C -> C} & R])
%     = trait inherits lang { below(c1 : C, c2 : C) = super.above(c2,c2) }
% \end{Verbatim}


% Local Variables:
% org-ref-default-bibliography: "../paper.bib"
% TeX-master: "../paper"
% End:


\input{sections/examples.tex}

\input{sections/explicitcore.tex}

% \input{sections/recursion.tex}

\input{sections/datatypes.tex}

\section{Related Work}
\label{sec:related}

% \bruno{I think (part of) this text can be discussed in here instead:


There are multiple flavours of inheritance. To avoid confusion, since the same
terminology is often used in the literature to mean different things, we use the
following 3 terms when comparing related work with ours.

\begin{itemize}
\item{{\bf Static inheritance:}} Static inheritance refers to what the typical
  model of inheritance in class-based languages. The inheritance model is said
  to be static because when using class extension, the extended classes are
  statically known at compile-time.
\item{{\bf Mutable Inheritance:}} Prototype-based languages allow another model
  of inheritance, which we call \emph{mutable inheritance}. In this inheritance
  model, self-references are mutable and changeable at any point.
\item{{\bf Dynamic Inheritance:}} Dynamic inheritance is a less well-known model
  which stands in between static and mutable inheritance. Unlike the static
  inheritance model, with dynamic inheritance objects can inherit from other
  objects which are not statically known. However, unlike mutable inheritance,
  the self-reference is not mutable and cannot be arbitrarily changed at
  run-time.
\end{itemize}

Figure~\ref{fig:comparision} shows the comparison between \name and various
similar languages that follow \citeauthor{cook1989inheritance}'s ``Inheritance is not
Subtyping'' (i.e. the flexible model), as we will explain below.

\begin{figure}[t]
  \centering
  \begin{small}
  \begin{tabular}{|l||c|c|c|c|}
    \hline
    & \bf{Statically typed} & \bf{Polymorphism} & \bf{Meta-theory} & \bf{Inheritance}  \\
    \hline
    \name & \cmark & \cmark & \cmark & Dynamic \\
    \hline
    \textsc{Self} & \xmark & \xmark & \xmark & Mutable \\
    \hline
    Cecil & \cmark & \cmark & \xmark & Static \\
    \hline
    Cook's Modula-3 & \cmark & \xmark & \xmark & Static \\
    \hline
    IFJ & \cmark & \xmark & \cmark & Dynamic \\
    \hline
    \textsc{Darwin} & \cmark & \xmark & \xmark & Dynamic \\
    \hline
  \end{tabular}
  \end{small}
  \caption{Comparison between \name and various similar languages that
  adopt the \emph{flexible model}.}
  \label{fig:comparision}
\end{figure}



% \paragraph{Dynamically-typed Languages with Delegation Mechanism}

% \begin{itemize}
% \item Clojure Protocols
%   % http://www.ibm.com/developerworks/library/j-clojure-protocols/
% \item Ruby mixin
% \item JS mixin
% \end{itemize}

% They are all dynamically typed.


\paragraph{Delegation-based languages}

\citet{lieberman1986using} is the first to promote the use of prototypes and
delegation as the mechanism to code sharing between objects. Since then many
researchers have studied the mechanisms of
delegation~\cite{wegner1987dimensions,malenfant1995semantic,goldberg1989smalltalk}.
\textsc{Self}~\cite{ungar1988self} is a dynamically typed, prototype-based
language with a simple and uniform object model. \textsc{Self}'s inheritance
model is typical of what we call mutable inheritance, because an object's parent
slots may be assigned new values at run-time. Mutable inheritance is rather
unstructured, and oftentimes access to any clashing methods will generate a
``messageAmbiguous'' error at run-time. Although \name's dynamic inheritance is
not as powerful as mutable inheritance, its static type system can guarantee
that no such errors occur at run-time.

There is not much work on statically-typed, delegation-based languages.
\citet{kniesel1999type} provides a good overview of problems when combining
delegation with a static type discipline. Cecil~\cite{chambers1992object,
  chambers1993cecil} is a prototype-based language, where delegation is the
mechanism for method call and code reuse. Cecil supports a polymorphic static
type system, although no meta-theory of any kind is given. Its type system is
able to detect statically when a message might be ambiguously defined as a
result of multiple inheritance or multiple dispatching. However, one major
omission of Cecil, which is also one of the interesting features of \name, is
dynamic inheritance. There are other
works~\cite{fisher1995delegation,anderson2003can} on delegation in a
statically-typed setting, but none of them provide means (such as the merge
construct, disjointness constraints, etc.) that are needed for extensible
designs.

\citet{cook1989inheritance} were the first to propose a typed model of
inheritance where subtyping and inheritance are two separate concepts. In
particular, they introduce the notion of \textit{type inheritance} and show that
inherited objects have inherited types, not subtypes. An interesting aspect of
their calculus is the \textbf{with} construct, used to join two records. This is
somewhat similar to our merge construct. However two major differences are worth
pointing out: 1) the \textbf{with} construct operates only on records; and 2) it
is a biased operator, favoring values from its right argument. This biased
operator is good for modelling mixins, but not traits. The
\textbf{with} construct seems to be unable to merge two arbitrary (and possible
polymorphic) values, since this seems to require something like
\emph{row polymorphism}~\cite{wand1987complete,wand1989type}, which is not available in their language.
The \textit{onion} construct in the Big Bang
language~\cite{palmer2015building,menon2012big} has a similar bias problem -- it is a
left-associative operator which gives rightmost precedence to one
implementation when conflicts exist.

\paragraph{Mixin-based inheritance}

Mixins have become very popular in many OO languages
~\cite{flatt1998classes,bono1999core, ancona2003jam}. \citeauthor{bracha1990mixin}'s
seminal paper~\citep{bracha1990mixin} extends Modula-3 with mixins. Mixins are subclasses parameterized
over a superclass, and used to produce a variety of classes with the same
functionality and behaviour. Mixin-based inheritance requires that mixins be
composed linearly, and as such, conflicts are resolved implicitly (mixins
appearing later overwrite all the identically named features of earlier mixins).
In comparison, the trait model in \name requires conflicts be resolved
explicitly. We want to emphasize that this conflict detection is essential in
expressing composition operators for Object Algebras, without running
into ambiguities.


\paragraph{Trait-based inheritance}

The seminar paper by \citet{scharli2003traits} introduced the ideas behind
traits, where they also documented an implementation of the trait
mechanism in a dynamically typed version of Smalltalk. Since then many
formalizations of traits have been
proposed~\cite{scharli2003traitsformal,ducasse2006traits,bettini2010prototypical}.
For example \citet{fisher2004typed} presented a statically-typed calculus that
models traits. Conflict detection is the hallmark of trait-based
inheritance, compared with mixin-based inheritance. One important difference
with \name is that those systems support \textit{classes} in addition to traits,
and consider the interaction between them, whereas \name is 
delegation based and the mechanism for code reuse is purely traits
(i.e., there are no classes in \name). The
deviation from traditional class-based models is not only because of its
simplicity, but also because we need a very \textit{dynamic} form of
inheritance, as has been elaborated throughout the paper.

Compared to the traditional trait mode, traits in \name have the following
differences: 1) traditional traits cannot be instantiated but only composed with
a class, whereas traits in \name can be instantiated directly; 2) traditional
traits cannot take constructor parameters whereas ours can; 3) the trait system
in \name lacks a proper notation of inheritance relationship. For example in the
traditional trait model, if the same method (i.e., from the same trait) is
obtained more than once via different paths, there is no conflict. This is not
the case in \name; and 4) traits in \name support dynamic
inheritance. 
%In the
%traditional trait model, when it comes to inheritance, the traits being
%inherited must be statically known.




% \citet{flatt1998classes} proposed MIXEDJAVA, an extension to a subset of
% sequential Java called CLASSICJAVA with mixins. In their model, mixins
% completely subsume the role of classes (classes are mixins that do not inherit
% any services). One interesting aspect in their system is that two identically
% named methods are allowed to coexist, and are resolved at run-time with run-time
% context information provided by the current \textit{view} of an object. In
% comparison, conflicts in \name are detected statically, and resolved by the
% programmers. Like \name, their model also enforces the distinction between
% implementation inheritance and subtyping.

% \citet{bono1999core} develop an imperative class-based calculus that provides a
% formal model for both single and mixin inheritance. Objects are represented by
% records and produced by instantiating classes. In their calculus, the class
% construct is extensible but not subtypable, while objects are subtypable but not
% extensible. Like \name, their system has a clean separation between subtyping
% and inheritance. Also, their type system does not have polymorphism.

% \citet{ancona2003jam} extends the Java language to support mixins, called Jam.
% Since Jam is an upward-compatible extension of Java 1.0, it is inheritantly a
% covariant mode. Unlike MIXEDJAVA, mixins can be only instantiated on classes,
% and there is no notion of mixin composition.


\begin{comment}

\begin{itemize}


\item ``Object-Oriented Multi-Methods in Cecil''

\item ``Dimensions of Object-Based Language Design''

\item ``On the Semantic Diversity of Delegation-Based Programming Languages''

\item ``Self: The power of simplicity''

\item ``Type-safe delegation for run-time component adaptation''

\item ``A delegation-based object calculus with subtyping''

\item ``Can Addresses be Types? (a case study: objects with delegation)''

\item ``Inheritance is not subtyping''


Mixins

\item ``mixin-based inheritance''

\item ``Classes and mixins''

\item ``A core calculus of classes and mixins''

\item ``A core calculus of higher-order mixins and classes''

\item ``Jam—Designing a Java Extension with Mixins''



\end{itemize}

Do they have polymorphic type systems? Do they support mutable self reference?

\end{comment}


\paragraph{Class-based languages with more advanced forms of inheritance}

Incomplete Featherweight Java (IFJ), proposed by \citet{bettini2008type}, is a
conservative extension of Featherweight Java with incomplete objects. Besides
standard classes, programmers can also define incomplete classes, whose
instances are incomplete objects. Incomplete objects can be composed (by object
composition) with complete objects, yielding new complete objects at run-time,
while ensuring statically that the composition is type-safe. Incomplete objects
are quite flexible, and support dynamic inheritance. However, object composition
in IFJ is quite restrictive, compared to \name, in that it can only compose an
incomplete object with a complete object. In that regard, and also because IFJ's
type system is not polymorphic, IFJ is unable to encode composition operators of
Object Algebras. \citet{kniesel1999type} showed that type-safe integration of
delegation with subtyping into a class-based model is possible, resulting in the
\textsc{Darwin} model. In \textsc{Darwin}, the type of the parent object must be
a declared class and this limits the flexibility of dynamic composition.
\citeauthor{ostermann2002dynamically}'s delegation
layers~\citep{ostermann2002dynamically} use delegation for doing dynamic
composition in a system with virtual classes. This is in contrast with most
other approaches that use class-based composition, but closer to the dynamic
composition that we use in \name.

There are many other class-based OO languages that are equipped with more
advanced forms of
inheritance~\cite{meyer1987eiffel,buchi2000generic,ostermann2001object}. Most of
them are heavyweight and are specific to classes. \name is object-centered, more
lightweight, and is dedicated to express extensible designs in a simpler way.


% Eiffel~\cite{meyer1987eiffel} is a class-based language that is based on the
% identification of classes with types and of inheritance with subtyping. Eiffel
% supports multiple inheritance, with the restriction that name collisions are
% considered programming errors, and ambiguities must be resolved explicitly by
% the programmer (by means of renaming). In this regard, \name is quite like
% Eiffel. However, the type system in \name is more lenient in that two
% identically named methods with different signatures can coexist without any
% problems.

% \citet{kniesel1999type} is the first to show that type-safe integration of
% delegation with subtyping into a class-based model is possible, resulting in the
% DARWIN model. In the DARWIN model, the type of the parent object must be a
% declared class and this limits the flexibility of dynamic composition, whereas
% in \name, the merge operator can merge/compose any objects. Another difference
% with \name lies in the conflict resolution, where DARWIN relies on method
% overriding with the assumption that the author of the overriding method is aware
% of the effect.

% Generic wrappers~\cite{buchi2000generic} supports aggregating objects at
% run-time. In their model, once a ``wrappee'' is assigned to a ``wrapper'', the
% wrappee is fixed. GBETA~\cite{ernst2000gbeta} has some dynamic features that are
% related to delegation. Like Generic wrappers, parents in GBETA are fixed at
% run-time.

% \citet{ostermann2001object} proposed compound references (CR) as a abstraction
% for object references, which provides explicit linguistic support for combining
% different composition properties on-demand. The model is statically typed, and
% decouples subtype declaration from implementation reuse.


% \citet{ostermann2002dynamically} proposed delegation layers as an approach to
% decompose a collaboration into layers and compose these layers dynamically at
% run-time. This combines and generalizes delegation and virtual classes concepts.

% \citet{ostermann2008nominal} compared the nominal and structural subtyping
% mechanisms. They argue nominal subtyping gives more safety guarantee, whereas
% structural subtyping is more flexible from a component-based perspective. The
% type system of \name chooses structural subtyping.

\paragraph{Intersection types, polymorphism and the merge construct}

There is a large body of work on intersection types. Here we only talk about
work that have direct influences on ours. \citet{dunfield2014elaborating} shows
significant expressiveness of type systems with intersection types and a merge
construct. However his calculus lacks coherence. The limitation was addressed
by~\citet{oliveira2016disjoint}, where they introduced the notion of
disjointness to ensure coherence. The combination of intersection types, a merge
construct and parametric polymorphism, while achieving coherence was first
studied in the \bname calculus~\cite{alpuimdisjoint}, where they proposed the
notion of disjoint polymorphism. \bname serves as the theoretical foundation of
\name.


\begin{comment}

\begin{itemize}

\item Eiffel

\item ``Delegation by object composition'' (IFJ) and ``Type safe dynamic object
  delegation in class-based languages''

\item ``Dynamically composable collaborations with delegation layers''

\item ``Generic wrappers''

\item ``Object-Oriented Composition Untangled''

\item ``gbeta - a language with virtual attributes, Block Structure, and Propagating, Dynamic Inheritance''

\item ``Nominal and Structural Subtyping in Component-Based Programming''

\item ``Engineering a programming language: The type and class system of Sather ''

\item ``Big Bang Designing a Statically-Typed Scripting Language''

\item ``Building a Typed Scripting Language''



\end{itemize}

\end{comment}



\section{Conclusions and Future Work}
\label{sec:conclusion}

We have proposed \name, a type-safe and coherent calculus with disjoint
intersection types, and support for nested composition/subtyping. \name
improves upon earlier work with a more
flexible notion of disjoint intersection types, which leads to
a clean and elegant formulation of the type system. Due to the added
flexibility we have had to employ a more powerful proof method based on logical
relations to rigorously prove coherence.
We also show how \name supports essential features of family
polymorphism, such as nested composition. We believe \name provides insights into family polymorphism, and
has potential for practical applications for extensible software designs.

A natural direction for future work is to enrich \name with parametric
polymorphism. There is abundant literature on logical relations for parametric
polymorphism~\citep{reynolds1983types} and we foresee no fundamental
difficulties in extending our proof method.\footnote{
Our prototype
  implementation already supports polymorphism, but we
  are still in the process of extending our Coq development with polymorphism. } The resulting calculus will be
more expressive than \fname. An interesting application that we intend to investigate
is native support for \textit{object algebras}~\citep{oliveira2012extensibility}
(or the finally tagless approach~\citep{CARETTE_2009}). For example, we can
define the object algebra interfaces for the Expression Problem example in
\cref{sec:overview} as follows:
\lstinputlisting[linerange=75-76]{../../impl/examples/overview.sl}% APPLY:linerange=LANG_EXT_INTER
By instantiating \lstinline{E} with \lstinline{IPrint}, i.e.,
\lstinline{ExpAlg[IPrint]}, we get the interface of the \lstinline{Lang} family.
In that sense, object algebra interfaces can be viewed as family interfaces.
Moreover, combing algebras implementing \lstinline{ExpAlg[IPrint]} and
\lstinline{ExpAlg[IEval]} to form \lstinline{ExpAlg[IPrint & IEval]} is trivial
with nested composition. Polymorphism also improves code reuse across expressions in the
base and extended languages. For example, the following creates two expressions,
one in the base language, the other in the extended language:
\lstinputlisting[linerange=81-82]{../../impl/examples/overview.sl}% APPLY:linerange=LANG_EXT
Notice how we can  reuse \lstinline{e1} of the base language in the definition
of \lstinline{e2}.



% \jeremy{creating expressions using base and extended expressions, and show more reuse}

% \jeremy{future work} \jeremy{mention in passing this rule is unsound with
%   effects, see ``Intersection types and computational effects''}

% Local Variables:
% mode: latex
% TeX-master: "../paper"
% End:

%% -- References --

% \acks
% Thanks to Blah. This work is supported by Blah.

\newpage

\bibliographystyle{plain}
% \nocite{*}
\bibliography{main}

%% -- Appendix --
\clearpage
\appendix

\section{Some Proofs about the Declarative System}


\lemmaerase*
\begin{proof}
By straightforward induction on the typing derivation.
\end{proof}

\lemmacoherence*
\begin{proof}
According to Lemma~\ref{lemma:erase}, after erasure of types and casts,
$\mathcal{C}\{[[pe1]]\}$ and $\mathcal{C}\{[[pe2]]\}$ are equivalent. So if
$\mathcal{C}\{ [[pe1]] \} \reduce [[n]]$, it is impossible for $\mathcal{C}\{
[[pe2]] \}$ to reduce to a different integer according to the dynamic semantics.
\end{proof}

\proptop*
\begin{proof} \leavevmode
  \begin{itemize}
  \item From first to second: By induction on the derivation of consistent
    subtyping. We have extra case \rref{CS-Top} now, where $B = \top$.
    We can choose $C = A$, and
    $D$ by replacing the unknown types in $C$ by $\nat$. Namely, $D$ is a static
    type, so by \rref{S-Top} we are done.
  \item From second to first: By induction on the derivation of second
    subtyping. We have extra case \rref{S-Top} now, where
    $B = \top$, so $A \tconssub B$ holds by \rref{CS-Top}.
  \end{itemize}
\end{proof}

\propagttop*
\begin{proof} \leavevmode
  \begin{itemize}
  \item From left to right: By induction on the derivation of consistent
    subtyping. We have case \rref{CS-Top} now.
    It follows that for
    every static type $A_1 \in \gamma(A)$, we can derive $A_1 \tsub \top$ by
    \rref{S-Top}.
    We have $B_1 = B = \top$ and we are done.
  \item From right to left: By induction on the derivation of subtyping and
    inversion on the concretization. We have extra case \rref{S-Top} now, where
    $B$ is $\top$. So
    consistent subtyping directly holds.
  \end{itemize}
\end{proof}

\propparalpha*
\begin{proof}
Follows directly from the definition of Translation Pre-order.
\end{proof}

\begin{definition}[Measurements of Translation]
  There are three measurements of a translation $[[pe]]$,
  \begin{enumerate}
  \item $[[ ||pe||e]]$, the size of the expression 
  \item $[[ ||pe||s ]]$, the number of distinct static type parameters in $[[pe]]$
  \item $[[ ||pe||g ]]$, the number of distinct gradual type parameters in $[[pe]]$
  \end{enumerate}
  We use $[[ ||pe|| ]]$ to denote the lexicographical order of the triple
  $([[ ||pe||e ]], -[[ ||pe||s ]], -[[ ||pe||g ]])$.
\end{definition}

\begin{definition}[Size of types]

  \begin{align*}
    [[ || int ||  ]] &= 1 \\
    [[ || a ||  ]] &= 1 \\
    [[ || A -> B  ||  ]] &= [[ || A || ]] + [[ || B || ]] + 1 \\
    [[ || \/a . A ||  ]] &= [[ || A || ]] + 1 \\
    [[ || unknown ||  ]] &= 1 \\
    [[ || static ||  ]] &= 1 \\
    [[ || gradual ||  ]] &= 1
  \end{align*}

\end{definition}

\begin{definition}[Size of expressions]

  \begin{align*}
    [[ || x ||e  ]] &= 1 \\
    [[ || n ||e  ]] &= 1 \\
    [[ || \x : A . pe ||e  ]] &= [[ || A || ]] + [[ || pe ||e ]] + 1 \\
    [[ || /\ a. pe ||e  ]] &= [[ || pe ||e ]] + 1 \\
    [[ || pe1 pe2 ||e  ]] &= [[ || pe1 ||e ]] + [[  || pe2 ||e ]] + 1 \\
    [[ || < A `-> B> pe ||e  ]] &= [[ || pe ||e ]] + [[  || A || ]] + [[  || B || ]] + 1 \\
  \end{align*}

\end{definition}


\begin{lemma} \label{lemma:size_e}
  If $[[dd |- e : A ~~> pe]]$ then $[[ || pe ||e    ]] \geq [[ || e ||e   ]]  $.
\end{lemma}
\begin{proof}
  Immediate by inspecting each typing rule.
\end{proof}

\begin{corollary} \label{lemma:decrease_stop}
  If $[[dd |- e : A ~~> pe]]$ then $[[ || pe ||   ]] > ([[ || e ||e ]], -[[ || e ||e ]], -[[ || e ||e ]] )  $.
\end{corollary}
\begin{proof}
  By \cref{lemma:size_e} and note that $ [[ || pe ||e   ]] > [[  || pe ||s  ]] $ and $ [[ || pe ||e   ]] > [[  || pe ||g  ]] $
\end{proof}

\begin{lemma} \label{lemma:type_decrease}
  $[[ || A || ]] \leq [[ || S(A) || ]]  $.
\end{lemma}
\begin{proof}
  By induction on the structure of $[[A]]$. The interesting cases are $[[ A ]] = [[static]]$ and
  $[[ A ]] = [[gradual]]$. When $[[ A ]] = [[static]]$, $[[ S(A) ]] = [[t]]$
  for some monotype $[[t]]$ and it is immediate that $[[ || static ||  ]]  \leq [[ || t || ]] $
  (note that $[[ || static ||  ]] < [[ || gradual ||  ]] $ by definition).
\end{proof}

\begin{lemma}[Substitution Decreases Measurement]
  \label{lemma:subst_dec_measure}
  If $[[pe1]] \leq [[pe2]]$, then $ {[[ ||pe1|| ]]} \leq [[ ||pe2|| ]]$; unless
  $[[pe2]] \leq [[pe1]]$ also holds, otherwise we have $[[ ||pe1|| ]] < [[ ||pe2|| ]]$.
\end{lemma}
\begin{proof}
  Since $[[ pe1  ]] \leq [[  pe2  ]]$, we know $[[ pe2  ]] = [[ S(pe1)  ]]$ for some $[[S]]$. By induction on
  the structure of $[[pe1]]$.

  \begin{itemize}
  \item Case $[[pe1]] = [[  \x : A . pe ]]$. We have
    $[[ pe2  ]] = [[  \x : S(A) . S(pe)  ]]$. By \cref{lemma:type_decrease} we have $[[ || A || ]] \leq [[ || S(A) || ]]$.
    By i.h., we have $[[ || pe ||  ]] \leq [[ || S(pe) ||  ]]$. Therefore $[[ || \x : A . pe ||    ]] \leq [[ || \x : S(A) . S(pe) ||  ]]$.
  \item Case $[[pe1]] = [[ < A `-> B > pe  ]]$. We have
    $[[pe2]] = [[ < S(A) `-> S(B) > S(pe)  ]]$.  By \cref{lemma:type_decrease} we have $[[ || A || ]] \leq [[ || S(A) || ]]$
    and $[[ || B || ]] \leq [[ || S(B) || ]]$. By i.h., we have $[[ || pe ||  ]] \leq [[ || S(pe) ||  ]]$.
    Therefore $[[  || < A `-> B > pe ||  ]] \leq [[ || < S(A) `-> S(B) > S(pe)  ||   ]]$.

  \item The rest of cases are immediate.
  \end{itemize}
\end{proof}


\lemmareptyping*
\begin{proof}
We already know that at least one translation $[[pe]] = [[pe1]]$ exists
for every typing derivation. If $[[pe1]]$ is a representative translation then we
are done. Otherwise there exists another translation $[[pe2]]$ such that
$[[pe2]] \leq [[pe1]]$ and $ [[pe1]] \not \leq [[pe2]]$. By
\cref{lemma:subst_dec_measure}, we have $[[||pe2||]] < [[ ||pe1|| ]]$. We continue
with $[[pe]] = [[pe2]]$, and get a strictly decreasing sequence $[[ || pe1 ||  ]], [[ || pe2 || ]], \dots$.
By \cref{lemma:decrease_stop}, we know this sequence cannot be infinite long. Suppose it ends at $[[ || pej || ]]$,
by the construction of the sequence, we know that $[[pej]]$ is a representative translation of $[[e]]$.
\end{proof}


\newpage


\section{The Extended Algorithmic System}


\subsection{Syntax}


\begin{center}
\begin{tabular}{lrcl} \toprule
  Expressions & $[[ae]]$ & \syndef & $[[x]] \mid [[n]] \mid [[ \x : aA . ae ]]  \mid  [[\x . ae]] \mid [[ae1 ae2]] \mid [[ae : aA]] \mid [[ let x = ae1 in ae2  ]] $ \\
  Types & $[[aA]], [[aB]]$ & \syndef & $ [[int]] \mid [[a]] \mid [[evar]] \mid [[aA -> aB]] \mid [[\/ a. aA]] \mid [[unknown]] \mid [[static]] \mid [[gradual]] $ \\
  Monotypes & $[[at]], [[as]]$ & \syndef & $ [[int]] \mid [[a]] \mid [[evar]] \mid [[at -> as]] \mid [[static]] \mid [[gradual]]$ \\
  Existential variables & $[[evar]]$ & \syndef & $[[sa]]  \mid [[ga]]  $   \\
  Castable Types & $[[agc]]$ & \syndef & $ [[int]] \mid [[a]] \mid [[evar]] \mid [[agc1 -> agc2]] \mid [[\/ a. agc]] \mid [[unknown]] \mid [[gradual]] $ \\
  Castable Monotypes & $[[atc]]$ & \syndef & $ [[int]] \mid [[a]] \mid [[evar]] \mid [[atc1 -> atc2]] \mid [[gradual]]$ \\
  Algorithmic Contexts & $[[GG]], [[DD]], [[TT]]$ & \syndef & $[[empty]] \mid [[GG , x : aA]] \mid [[GG , a]] \mid [[GG , evar]]  \mid [[GG, sa = at]] \mid [[GG, ga = atc]] \mid [[ GG, mevar ]] $ \\
  Complete Contexts & $[[OO]]$ & \syndef & $[[empty]] \mid [[OO , x : aA]] \mid [[OO , a]] \mid [[OO, sa = at]] \mid [[OO, ga = atc]] \mid [[OO, mevar]] $ \\
  \bottomrule
\end{tabular}

\end{center}


\subsection{Type System}


\drules[as]{$ [[GG |- aA <~ aB -| DD ]] $}{Algorithmic Consistent Subtyping}{tvar, evar, int, arrow, forallR, forallLL, spar, gpar, unknownLL, unknownRR, instL, instR}

\drules[instl]{$ [[ GG |- evar <~~ aA -| DD   ]] $}{Instantiation I}{solveS, solveG, solveUS, solveUG, reachSGOne, reachSGTwo, reachOther, arr, forallR}

\drules[instr]{$ [[ GG |- aA <~~ evar -| DD   ]] $}{Instantiation II}{solveS, solveG, solveUS, solveUG, reachSGOne, reachSGTwo, reachOther, arr, forallLL}

\drules[inf]{$ [[ GG |- ae => aA -| DD ]] $}{Inference}{var, int, lamannTwo, lamTwo, anno, app, letTwo}

\drules[chk]{$ [[ GG |- ae <= aA -| DD ]] $}{Checking}{lam, gen, sub}

\drules[am]{$ [[ GG |- aA |> aA1 -> aA2 -| DD ]] $}{Algorithmic Matching}{forallL, arr, unknown, var}

\newpage

\section{Decidability} \label{app:decidable}

The decidability proofs mostly follow that of DK system. Whenever possible, we only show
the new cases; otherwise we provide full detailed proofs.

\subsection{Decidability of Instantiation}


\begin{lemma}[Left Unsolvedness Preservation]
  Let $[[GG]] = [[  GG0, evar, GG1 ]]$. If  $[[ GG |- evar <~~ aA -| DD    ]]$ or $[[  GG |- aA <~~ evar -| DD  ]]$, and $[[evarb]] \in \textsc{unsolved}([[GG0]])$,
  then  $[[DD]] = [[(DD0, evarb, DD1)]]$ or $[[DD]] = [[(DD0, evarb', evarb = evarb', DD1)]]$ where $[[evarb']]$ is a fresh unsolved existential.
\end{lemma}
\begin{proof}
  By induction on the given derivation. We show the new cases.

  \begin{itemize}
  \item Case \[     \ottaltinferrule{instl-solveUS}{}{  }{ [[  GG0, sa, GG1 |- sa <~~ unknown -| GG0 , ga, sa = ga, GG1 ]] }  \]
    First notice that $[[evarb]]$ cannot be $[[ga]]$. Then to the left of $[[sa]]$, the contexts $[[DD]]$ and $[[GG]]$ are the same $[[GG0]]$.
  \item Case \[  \drule{instl-solveUG}   \]
    Immediate, since to the left of $[[ga]]$, the contexts $[[DD]]$ and $[[GG]]$ are the same.
  \item Case \[  \drule{instl-reachSGOne}    \]
    First notice that $[[evarb]]$ cannot be $[[ga]]$. Then to the left of $[[sa]]$, the contexts $[[DD]]$ and $[[GG]]$ are the same.
  \item Case \[  \drule{instl-reachSGTwo}    \]
    If $[[evarb]] \neq [[sb]] $, immediate, since to the left of $[[ga]]$ ($[[evarb]]$ cannot be $[[gb]]$), the contexts $[[DD]]$ and $[[GG]]$ are the same.
    Otherwise, $[[sb]]$'s solution (i.e., $[[gb]]$) is a fresh unsolved existential that lies just before $[[sb]]$.
  \item Case \rref*{instr-solveUS} is similar to case \rref*{instl-solveUS}.
  \item Case \rref*{instr-solveUG} is similar to case \rref*{instl-solveUG}.
  \item Case \rref*{instr-reachSG1} is similar to case \rref*{instl-reachSG1}.
  \item Case \rref*{instr-reachSG2} is similar to case \rref*{instl-reachSG2}.
  \end{itemize}

\end{proof}


\begin{lemma}[Left Free Variable Preservation]
  Let $[[GG]] = [[  GG0, evar, GG1  ]]$. If $[[ GG |- evar <~~ aA -| DD    ]]$ or $[[  GG |- aA <~~ evar -| DD  ]]$, and $[[ GG |- aB ]]$
  and $[[ evar ]] \notin \textsc{fv}([[ [GG]aB  ]])$ and $[[evarb]] \in \textsc{unsolved}([[GG0]])  $
  and $[[evarb]]  \notin \textsc{fv}([[ [GG]aB  ]])  $,
  then $[[evarb]] \notin \textsc{fv}([[  [DD]aB ]])$.
\end{lemma}
\begin{proof}
  By induction on the given derivation. We show the new cases.
  \begin{itemize}
  \item Case \[  \drule{instl-solveUS}    \]
    Since $[[DD]]$ differs from $[[GG]]$ only in solving $[[sa]]$ to $[[ga]]$, and $[[ga]]$ is fresh,
    applying $[[DD]]$ to a type will not introduce $[[evarb]]$.  We have $[[evarb]]  \notin \textsc{fv}([[ [GG]aB  ]])  $, so
    $[[evarb]]  \notin \textsc{fv}([[ [DD]aB  ]])  $.

  \item Case \[  \drule{instl-solveUG}    \]
    Immediate, since $[[DD]]$ and $[[GG]]$ are the same.

  \item Case \[  \drule{instl-reachSGOne}    \]

    Since $[[DD]]$ differs from $[[GG]]$ only in solving $[[sa]]$ and $[[gb]]$ to $[[ga]]$, and $[[ga]]$ is fresh,
    applying $[[DD]]$ to a type will not introduce $[[evarb]]$.  We have $[[evarb]]  \notin \textsc{fv}([[ [GG]aB  ]])  $, so
    $[[evarb]]  \notin \textsc{fv}([[ [DD]aB  ]])  $.

  \item Case \[  \drule{instl-reachSGTwo}    \]

    Since $[[DD]]$ differs from $[[GG]]$ only in solving $[[sb]]$ and $[[ga]]$ to $[[gb]]$, and $[[gb]]$ is fresh,
    applying $[[DD]]$ to a type will not introduce $[[evarb]]$.  We have $[[evarb]]  \notin \textsc{fv}([[ [GG]aB  ]])  $, so
    $[[evarb]]  \notin \textsc{fv}([[ [DD]aB  ]])  $.

  \item Case \rref*{instr-solveUS} is similar to case \rref*{instl-solveUS}.
  \item Case \rref*{instr-solveUG} is similar to case \rref*{instl-solveUG}.
  \item Case \rref*{instr-reachSG1} is similar to case \rref*{instl-reachSG1}.
  \item Case \rref*{instr-reachSG2} is similar to case \rref*{instl-reachSG2}.


  \end{itemize}

\end{proof}


\begin{lemma}[Instantiation Size Preservation] \label{lemma:dec:instan}
  If $[[GG]] = [[ GG0, evar, GG1 ]]$ and $[[  GG |- evar  <~~ aA -| DD ]]$
  or $[[ GG |- aA <~~ evar -| DD  ]]$, and $[[GG |- aB]]$ and $[[evar notin fv( [GG]aB )]]$, then $ | [[   [GG]aB   ]] | = | [[   [DD]aB   ]] | $, where $| [[   aC  ]] | $ is the plain size of $[[aC]]$.
\end{lemma}
\begin{proof}
  By induction on the given derivation. We show the new cases.
  \begin{itemize}
  \item Case \[ \drule{instl-solveUS} \]
    Since $[[DD]]$ differs $[[GG]]$ only in solving $[[sa]]$, and we know $[[ sa notin fv([GG]aB)  ]]  $, we have $[[  [DD]aB  ]] = [[  [GG]aB ]]$, so
    $ | [[   [GG]aB   ]] | = | [[   [DD]aB   ]] | $.

  \item Case \[  \drule{instl-solveUG}   \]
    Immediate, since $[[DD]]$ and $[[GG]]$ are the same.

  \item Case \[   \drule{instl-reachSGOne}   \]
    Since $[[DD]]$ differs $[[GG]]$ only in solving $[[sa]]$ and $[[gb]]$, and we know $[[ sa notin fv([GG]aB)  ]]  $,
    even if $[[gb]]$ occurs in $[[  [GG]aB]]$, its solution is again an existential variable, so the size does not change,
    so $ | [[   [GG]aB   ]] | = | [[   [DD]aB   ]] | $.

  \item Case \[   \drule{instl-reachSGTwo}   \]
    Since $[[DD]]$ differs $[[GG]]$ only in solving $[[ga]]$ and $[[sb]]$, and we know $[[ ga notin fv([GG]aB)  ]]  $,
    even if $[[sb]]$ occurs in $[[  [GG]aB]]$, its solution is again an existential variable, so the size does not change,
    so $ | [[   [GG]aB   ]] | = | [[   [DD]aB   ]] | $.

  \item Case \rref*{instr-solveUS} is similar to case \rref*{instl-solveUS}.
  \item Case \rref*{instr-solveUG} is similar to case \rref*{instl-solveUG}.
  \item Case \rref*{instr-reachSG1} is similar to case \rref*{instl-reachSG1}.
  \item Case \rref*{instr-reachSG2} is similar to case \rref*{instl-reachSG2}.

  \end{itemize}
\end{proof}


\decInstan*
\begin{proof}
  By induction on the derivation of $[[  GG |- aA   ]]$. We show the new cases.
  \begin{itemize}
  \item Case \[  \drule{ad-unknown}   \]
    By \rref{instl-solveUS} or \rref{instl-solveUG}.

  \item Case \[  \drule{ad-static}   \]
    By \rref{instl-solveS}.

  \item Case \[  \drule{ad-gradual}   \]

    By \rref{instl-solveS} or \rref{instl-solveG}.


  \item Case \[    \ottaltinferrule{ad-evar}{}{  }{ [[  GG0, sa, GG1 |- ga  ]] }  \]
    If $[[ga]] \in [[GG0]]  $, then we have a derivation by \rref{instl-reachOther}. If $[[ga]] \in [[GG1]]$, then
    we have a derivation by \rref{instl-reachSG1}.


  \item Case \[    \ottaltinferrule{ad-evar}{}{  }{ [[  GG0, ga, GG1 |- sa  ]] }  \]
    If $[[sa]] \in [[GG0]]  $, then we have a derivation by \rref{instl-reachSG2}. If $[[sa]] \in [[GG1]]$, then
    we have a derivation by \rref{instl-reachOther}.

  \end{itemize}


\end{proof}



\subsection{Decidability of Algorithmic Consistent Subtyping}


\begin{lemma}[Monotypes Solve Variables] \label{lemma:mono_solve_var}
  If $[[ GG |- evar <~~ at -| DD ]]$ or $[[ GG |- at <~~ evar -| DD  ]]$, then if $[[  [GG]at   ]] = [[at]]$ and
  $[[evar]] \notin \textsc{fv}([[ [GG] at  ]])$, then $| \textsc{unsolved}([[GG]])   | = |  \textsc{unsolved}([[DD]])      | + 1$.
\end{lemma}
\begin{proof}
  By induction on the given derivation. Since our syntax of monotypes differ
  from DK only in having static and gradual parameters, we show only two affected cases.
  \begin{itemize}
  \item Case \[  \drule{instl-solveS}  \]
    It is immediate that $|  \textsc{unsolved}([[GG, sa, GG']])    | = |  \textsc{unsolved}([[GG, sa = at, GG']])    | + 1$.
  \item Case \[  \drule{instl-solveG}  \]
    It is immediate that $|  \textsc{unsolved}([[GG, ga, GG']])    | = |  \textsc{unsolved}([[GG, ga = atc, GG']])    | + 1$.
  \end{itemize}
\end{proof}


\begin{lemma}[Monotype Monotonicity] \label{lemma:mono_mono}
  If $[[GG |- at1 <~ at2 -| DD]]$  then $| \textsc{unsolved}([[DD]]) | \leq |  \textsc{unsolved}([[GG]])    |$.
\end{lemma}
\begin{proof}
  By induction on the derivation. We show the new cases.
  \begin{itemize}
  \item Case \rref*{as-spar,as-gpar}: In these rules, $[[DD]] = [[GG]]$, so $| \textsc{unsolved}([[DD]]) | = |  \textsc{unsolved}([[GG]])    |$.
  \end{itemize}

\end{proof}


\begin{lemma}[Substitution Decreases Size] \label{lemma:subst_decrease}
  If $[[GG |- aA]]$, then $|[[  GG |-   [GG]aA    ]]| \leq  |  [[GG |-aA]]      |   $.
\end{lemma}
\begin{proof}
  By induction on $| [[ GG |- aA  ]] |$. We show the new cases.
  \begin{itemize}
  \item $[[aA]] = [[unknown]]$, or $[[aA]] = [[static]]$, or $[[aA]] = [[gradual]]$ then $[[  [GG] aA ]] = [[aA]]$. Therefore
    $|[[  GG |-   [GG]aA    ]]| =  |  [[GG |-aA]]      |   $.
  \end{itemize}
\end{proof}


\begin{lemma}[Monotype Context Invariance] \label{lemma:mono_inva}
  If $[[GG |- at <~ at' -| DD]]$ where $[[  [GG]at   ]] = [[at]]$ and $[[  [GG]at'   ]] = [[at']]$ and
  $|   \textsc{unsolved}([[GG]])      | = |  \textsc{unsolved}([[DD]])    |$, then $[[DD]] = [[GG]]$.
\end{lemma}
\begin{proof}
  By induction on the derivation. We show the new cases.
  \begin{itemize}
  \item Cases \rref*{as-spar,as-gpar}: In these rules, the output context is the
    same as the input context, so the result is immediate.
  \item Case \[  \drule{as-instL}   \]
    By \cref{lemma:mono_solve_var}, $|\textsc{unsolved}([[ DD ]])| < | \textsc{unsolved}([[ GG[evar]  ]])   |$, which is contrary to what is given,
    so this case is impossible.

  \item Case \rref*{as-instR} is similar to \rref*{as-instL}.
  \end{itemize}

\end{proof}

\decsubtype*
\begin{proof}
  Let the judgment $[[ GG |- aA <~ aB -| DD   ]]$ be measured lexicographically by
 \begin{enumerate}[(M1)]
\item the number of $\forall$-quantifiers in $[[aA]]$ and $[[aB]]$;
\item the number of unknown types in $[[aA]]$ and $[[aB]]$;
\item $| \textsc{unsolved}([[GG]]) |$: the number of unsolved existential
  variables in $[[GG]]$;
\item $| [[GG |- aA]] | + | [[GG |- aB]]   |$.
\end{enumerate}

We focus on the interesting (and new) cases.

\begin{asparaitem}
\item Cases \rref*{as-spar,as-gpar,as-unknownLL,as-unknownRR} have no premises.
\item Case \[  \drule{as-arrow}   \]
  We discuss each premise separately:
  \begin{description}
    \item[First premise:]
      If $[[aA2]]$ or $[[aB2]]$ has a quantifier, then the first premise is smaller by (M1). Otherwise, if $[[aA2]]$ or $[[aB2]]$
      has a unknown type, then first premise is smaller by (M2). Otherwise, the first premise shares the same input context as the conclusion, so it
      has the same (M3), but the types $[[aB1]]$ and $[[aA1]]$ are subterms of the conclusion's types, so the first premise is smaller by (M4).
    \item[Second premise:] If $[[aB1]]$ or $[[aA1]]$ has a quantifier, then the second premise is smaller by (M1) because
      applying contexts will not introduce quantifiers. Otherwise,
      if $[[aB1]]$ or $[[aA1]]$ has a unknown type, then the second premise is smaller by (M2) because
      applying contexts will not introduce unknown types. Otherwise, at this point, we know
      $[[aB1]]$ and $[[aA1]]$ are monotypes, so by \cref{lemma:mono_mono} on the first premise,
      we have $| \textsc{unsolved}([[TT]])  | \leq | \textsc{unsolved}([[GG]])   |$.
      \begin{itemize}
      \item If $| \textsc{unsolved}([[TT]])  | < | \textsc{unsolved}([[GG]])   |$, then the second premise is smaller by (M3).
      \item If $| \textsc{unsolved}([[TT]])  | = | \textsc{unsolved}([[GG]])   |$, then we have the same (M3). By \cref{lemma:mono_inva}
        on the first premise, we know  $[[TT]] = [[GG]]$, so $| [[TT |- [TT]aA2]] | = | [[GG |- [GG]aA2]] |$.
        By \cref{lemma:subst_decrease} we know $| [[GG |- [GG]aA2]] | \leq  | [[ GG |- aA2  ]] |  $. Therefore we have
        \[
          | [[TT |- [TT]aA2]] | \leq  | [[ GG |- aA2  ]] |
        \]
        Same for $[[aB2]]$:
        \[
          | [[TT |- [TT]aB2]] | \leq  | [[ GG |- aB2  ]] |
        \]
        Therefore,
        \[
          | [[TT |- [TT]aA2]] | + | [[TT |- [TT]aB2]] | \leq | [[ GG |- aA2  ]] | +  | [[ GG |- aB2  ]] | < | [[ GG |- aA1 -> aA2  ]] | + | [[ GG |- aB1 -> aB2  ]] |
        \]
        and the second premise is smaller by (M4).
      \end{itemize}
  \end{description}
\end{asparaitem}

\end{proof}

\subsection{Decidability of Algorithmic Typing}

\decmatch*
\begin{proof}
  \Rref{am-arr,am-unknown,am-var} do not have premises. For \rref{am-forall},
  the size of $[[aA]]$ is decreasing in the premise.
\end{proof}


\dectyping*
\begin{proof}

  We consider the following measure:
  \begin{center}
    \begin{tabular}{lllll}
      \multirow{2}{*}{$ \Big \langle$} & \multirow{2}{*}{$[[ae]],$} & $[[=>]]$ & \multirow{2}{*}{$| [[GG |- aA]] |$} & \multirow{2}{*}{$\Big \rangle$} \\
                                       &                    & $[[<=]],$ &  &
    \end{tabular}
  \end{center}
  and show every inference/checking premise is smaller than the conclusion.
  \begin{itemize}
  \item \Rref{inf-var,inf-int} do not have premises.
  \item \Rref{inf-anno,inf-lamann,inf-lam,inf-let,chk-lam} all have strictly
    smaller $[[ae]]$ in the premises.
  \item \Rref{inf-app}: The first and third premises have strictly smaller
    $[[ae]]$. The second (matching) judgment is decidable by
    \cref{lemma:decmatch}.
  \item \Rref{chk-gen}: Both the premise and conclusion type the same term, and
    both are the checking judgments. However $| [[GG, a |- aA]] | < | [[GG |-  \/a. aA]] |$, so
    the premise is smaller.
  \item \Rref{chk-sub}: The first premise uses inference mode, so it is smaller.
    The second premise is decidable by \cref{thm:decsubtype}.
  \end{itemize}

\end{proof}

\newpage



\section{Properties of Consistent Subtyping}


\begin{clemma}[Consistent Subtyping is Reflexive] \label{lemma:sub_refl}%
  If $[[ dd |- A  ]]$ then $[[ dd |- A <~ A  ]]$.
\end{clemma}


\begin{clemma}[Monotype Equality]   \label{lemma:mono_equal}
  If  $[[ dd |- t <~ s  ]]$  then $[[t]] = [[s]]$.
\end{clemma}

\begin{lemma}[Invertibility]  \label{lemma:forall_invert}
  If $ [[dd |- A <~ \/b . B]] $ then $[[ dd, b |- A <~ B ]]$.
\end{lemma}
\begin{proof}
  By induction on the given derivation.
  \begin{itemize}
  \item \Rref{cs-arrow,cs-tvar,cs-int,cs-unknownRR,cs-spar,cs-gpar} are impossible since
    the supertype is not a forall type.
  \item Case \[  \drule{cs-forallR}   \] The premise is exactly what we need.
  \item Case \[  \drule{cs-forallL}  \] where $B = [[  \/b . B0 ]]$. By i.h.,
    we have $[[ dd, b |- A [a ~> t] <~ B0 ]]$. By \rref{cs-forallL} we have
    $[[  dd , b |- \/a . A <~ B0 ]]$.
  \item Case \[ \drule{cs-unknownLL}   \] where $[[gc]] = [[\/b . gc0 ]]$. By \rref{cs-unknownLL}
    we have $[[ dd , b |- unknown <~ gc0 ]]$.
  \end{itemize}
\end{proof}


\newpage


\section{Properties of Context Extension}

\subsection{Syntactic Properties}


Since the definition of the context extension judgment ($[[GG --> DD]]$,
\cref{fig:ctxt-extension}) is exactly the same as that of the DK system, we
refer the reader to their technical report~\citep{dunfield2013complete} for the proofs of the following
syntactic properties of context extension.



\begin{lemma}[Reverse Declaration Order Preservation]   \label{lemma:reverse_preserve}
  If $\Gamma \exto \Delta$ and $a$ and $b$ are both declared in $\Gamma$ and $a$
  is declared to the left of $b$ in $\Delta$, then $a$ is declared to the left
  of $b$ in $\Gamma$.
\end{lemma}



\begin{lemma}[Reflexivity]   \label{lemma:reflexivity}
  If $[[GG]]$ is well-formed then $[[GG --> GG]]$.
\end{lemma}


\begin{lemma}[Transitivity]   \label{lemma:transitivity}
  If $[[GG --> DD]] $ and $[[DD --> TT]]$ then $[[ GG --> TT  ]]$.
\end{lemma}

\begin{definition}[Softness]
  A context $[[TT]]$ is soft iff it consists only of $[[evar]]$ and $[[evar = at]]$ declarations.
\end{definition}


\begin{lemma}[Substitution Extension Invariance]  \label{lemma:subst_ext_invar}
  If $[[TT |- aA]]$ and $[[TT --> GG]]$ then $[[ [GG]aA ]] = [[ [GG] ([TT]aA)]]  $ and $ [[ [GG]aA ]] = [[ [TT] ([GG]aA)]]   $.
\end{lemma}


\begin{lemma}[Extension Order] \label{lemma:extension_order}%
  We have the following:
  \begin{enumerate}
  \item If $[[ GL , a, GR --> DD]]$ then $[[DD]]  = [[(DL, a, DR)]]  $ where
    $[[GL --> DL]]$. Moreover, if $[[GR]]$ is soft then $[[DR]]$ is soft.
  \item If $[[ GL , mevar, GR --> DD]]$ then $[[DD]]  = [[(DL, mevar, DR)]]  $ where
    $[[GL --> DL]]$. Moreover, if $[[GR]]$ is soft then $[[DR]]$ is soft.
  \item If $[[GL, evar, GR --> DD]]$ then $[[DD]]  = [[(DL, TT, DR)]]  $ where $[[GL --> DL]]$ and $[[TT]]$ is either $[[evar]]$ or $[[evar]] = [[at]]$ for some $[[at]]$.
  \item If $[[GL, evar = at, GR --> DD]]$ then $[[DD]]  = [[(DL, evar = at', DR)]]  $ where $[[GL --> DL]]$ and and $[[ [DL]at  ]] = [[ [DL] at' ]]$.
  \item If $[[GL , x : aA , GR --> DD]]$ then $[[DD]]  = [[(DL, x : aA', DR)]]  $ where $[[GL --> DL]]$ and $[[ [DL]aA  ]] = [[ [DL] aA' ]]$. Moreover, $[[GR]]$ is soft if and only if $[[DR]]$ is soft.
  \end{enumerate}
\end{lemma}


\begin{lemma}[Solution Admissibility for Extension] \label{lemma:solution_ext}
  If $[[ GL |- at  ]]$ then $[[  GL, evar, GR --> GL, evar = at, GR ]]$.
\end{lemma}


\begin{lemma}[Unsolved Variable Addition for Extension]   \label{lemma:unsolved_ext}
  We have that $ [[GL, GR --> GL, evar, GR]] $
\end{lemma}


\begin{lemma}[Parallel Admissibility]   \label{lemma:paralell_admit}
  If $\Gamma_L \exto \Delta_L$ and $\Gamma_L, \Gamma_R \exto \Delta_L, \Delta_R$ then:
  \begin{enumerate}
  \item $\Gamma_L, \genA, \Gamma_R \exto \Delta_L, \genA, \Delta_R$
  \item If $\Delta_L \vdash \tau'$ then $\Gamma_L, \genA, \Gamma_R \exto \Delta_L, \genA = \tau', \Delta_R$.
  \item If $\Gamma_L \vdash \tau$ and $\Delta_L \vdash \tau'$ and $\ctxsubst{\Delta_L}{\tau} = \ctxsubst{\Delta_L}{\tau'}$, then $\Gamma_L, \genA = \tau, \Gamma_R \exto \Delta_L, \genA = \tau', \Delta_R$.
  \end{enumerate}
\end{lemma}

\begin{lemma}[Parallel Extension Solution] \label{lemma:paralell_ext_solu}
  If $\Gamma_L, \genA, \Gamma_R \exto \Delta_L, \genA = \tau', \Delta_R$ and
  $\Gamma_L \vdash \tau$ and $\ctxsubst{\Delta_L}{\tau} =
  \ctxsubst{\Delta_L}{\tau'}$, then $\Gamma_L, \genA = \tau, \Gamma_R \exto
  \Delta_L, \genA = \tau', \Delta_R$.
\end{lemma}




\begin{lemma}[Drop Variable for Extension]   \label{lemma:drop_ext}
  If $[[ GG, evar --> DD  ]]$ then $[[ GG --> DD]]$.
\end{lemma}


\begin{lemma}[Finishing Types]
  If $\Omega \vdash A$ and $\Omega \exto \Omega'$ then $\ctxsubst{\Omega}{A} = \ctxsubst{\Omega'}{A}$.
  \label{lemma:finish_types}
\end{lemma}


\begin{lemma}[Finishing completions]
  If $\Omega \exto \Omega'$ then $\ctxsubst{\Omega}{\Omega} = \ctxsubst{\Omega'}{\Omega'}$.
  \label{lemma:finish_complete}
\end{lemma}



\begin{lemma}[Confluence of Completeness]   \label{lemma:confluence}
  If $[[  DD1 --> OO]]$ and $[[ DD2 --> OO]]$ then
  $[[ [OO]DD1 ]]  =  [[ [OO] DD2  ]]$.
\end{lemma}


\begin{lemma}[Variable Preservation]
  If $(x : A) \in \Delta$ or $(x : A) \in \Omega$ and $\Delta \exto \Omega$ then $(x : \ctxsubst{\Omega}{A}) \in \ctxsubst{\Omega}{\Delta}$.
  \label{lemma:variable_preservation}
\end{lemma}


\begin{lemma}[Softness Goes Away]
  If $\Delta, \Theta \exto \Omega, \Omega_Z$ where $\Delta \exto \Omega$ and $\Theta$ is soft, then $\ctxsubst{\Omega, \Omega_Z}{(\Delta, \Theta)} = \ctxsubst{\Omega}{\Delta}$.
  \label{lemma:subst_go_away}
\end{lemma}

\begin{lemma}[Stability of Complete Contexts]
  If $\Gamma \exto \Omega$ then $\ctxsubst{\Omega}{\Gamma} = \ctxsubst{\Omega}{\Omega}$.
  \label{lemma:stable_complete_ctxt}
\end{lemma}



\subsection{Instantiation Extends}

\begin{lemma}[Instantiation Extension]   \label{lemma:inst_extension}
 If $[[  GG |- evar <~~ aA  -| DD   ]]$ or $[[ GG |- aA <~~ evar -| DD  ]]$ then $[[ GG --> DD ]]$.
\end{lemma}
\begin{proof}
  By induction on the given instantiation derivation.
  \begin{itemize}
  \item \Rref{instl-solveS,instl-solveG,instl-reachOther,instr-solveS,instr-solveG,instr-reachOther}
    are immediate from \cref{lemma:solution_ext}.

  \item Case \[ \drule{instl-solveUS} \]
    By \cref{lemma:unsolved_ext} we have $[[ GG[sa] --> GG[ga, sa]  ]]$. By \cref{lemma:solution_ext} we have
    $[[ GG[ga, sa] --> GG[ga, sa = ga]  ]]$. By \cref{lemma:transitivity} we have $[[ GG[sa] --> GG[ga, sa=ga]   ]]$.

  \item Case \[ \drule{instl-solveUG} \] Immediate by \cref{lemma:reflexivity}.


  \item Case \[  \drule{instl-reachSGOne}  \]
    By \cref{lemma:unsolved_ext} we have $[[  GG[sa][gb] --> GG[ga,sa][gb]  ]]$. By applying \cref{lemma:solution_ext} twice,
    we have $[[ GG[ga,sa][gb] --> GG[ga,sa=ga][gb=ga]    ]]$. By \cref{lemma:transitivity} we have $[[ GG[sa][gb] --> GG[ga,sa=ga][gb=ga]  ]]$.

  \item Case \[  \drule{instl-reachSGTwo}   \] Same as the case for \rref{instl-reachSG1}.

  \item Case \[ \drule{instl-arr}\]
    By applying \cref{lemma:unsolved_ext} twice, we have $[[  GG[evar] --> GG[evar2, evar1, evar]  ]]$.
    By \cref{lemma:solution_ext} we have $[[ GG[evar2, evar1, evar] --> GG[evar2, evar1, evar = evar1 -> evar2]  ]]$.
    By i.h., we have $[[ GG[evar2, evar1, evar = evar1 -> evar2] --> TT  ]]$ and $[[ TT --> DD  ]]$. By \cref{lemma:transitivity} we
    have $[[  GG[evar] --> DD  ]]$.

  \item Case \[ \drule{instl-forallR}  \]
    By i.h., we have $[[ GG[evar], b --> DD , b , TT  ]]$. By \cref{lemma:extension_order} (1), we have $[[GG[evar] --> DD]]$.

  \item Case \[ \drule{instr-solveUS}  \] Same as the case for \rref{instl-solveUS}.

  \item Case \[  \drule{instr-solveUG}\] Same as the case for \rref{instl-solveUG}.

  \item Case \[  \drule{instr-reachSGOne}  \]  Same as the case for \rref{instl-reachSG1}.

  \item Case \[  \drule{instr-reachSGTwo}  \]  Same as the case for \rref{instl-reachSG1}.

  \item Case \[ \drule{instr-arr} \] Same as the case for \rref{instl-arr}.

  \item Case \[ \drule{instr-forallLL}   \]
    By i.h., we have $[[ GG[evar] , mevarb, sb --> DD, mevarb, TT ]]$. By \cref{lemma:extension_order}(2) we have $[[ GG[evar] --> DD  ]]$.

  \end{itemize}
\end{proof}

\subsection{Consistent Subtyping Extends}

\begin{lemma} \label{lemma:contamination_extension}
  If $[[GG |- aA ]]$ then $[[GG --> [aA] GG]]$.
\end{lemma}
\begin{proof}
  By induction on the structure of $[[GG]]$. The only interesting case is when $[[GG]] = [[GG' , sa]] $. By \cref{def:contamination},
  we have $[[ [aA] (GG' , sa) ]] = [[ [aA] GG' , ga, sa = ga]] $. By i.h., we have $[[ GG' --> [aA] GG'  ]]$.
  By definition of context extension we have $[[ GG' , sa --> [aA] GG' , sa  ]]$. By \cref{lemma:unsolved_ext} we have
  $[[ [aA] GG' , sa --> [aA] GG' , ga, sa ]]$. By \cref{lemma:solution_ext} we have $[[ [aA] GG' , ga, sa --> [aA] GG' , ga, sa = ga  ]]$.
  By \cref{lemma:transitivity} we have $[[ GG' , sa --> [aA] GG' , ga, sa = ga ]]$.
\end{proof}

\begin{lemma}[Consistent Subtyping Extension]   \label{lemma:sub_extension}
  If $ [[GG |- aA <~ aB -| DD]]  $ then $ [[GG --> DD]]$.
\end{lemma}
\begin{proof}
  By induction on the derivation of consistent subtyping.

  \begin{itemize}
  \item \Rref{as-tvar,as-evar,as-int,as-spar,as-gpar} are immediate from \cref{lemma:reflexivity}.

  \item Case \[ \drule{as-arrow} \]
    By i.h., we have $[[GG --> TT]]$ and $[[TT --> DD]]$. By \cref{lemma:transitivity}, we have $[[GG --> DD]]$.

  \item Case \[ \drule{as-forallR} \]
    By i.h., we have $[[GG , a --> DD , a , TT]]$. By \cref{lemma:extension_order} (1), we have $[[GG --> DD]]$.


  \item Case \[ \drule{as-forallLL} \]
    By i.h., we have $[[GG, mevar, sa --> DD, mevar, TT]]$. By \cref{lemma:extension_order} (2), we have $[[GG --> DD]]$.


  \item Case \[ \drule{as-unknownLL}  \] Immediate by \cref{lemma:contamination_extension}.

  \item Case \[ \drule{as-unknownRR}  \] Immediate by \cref{lemma:contamination_extension}.


  \item \Rref{as-instL,as-instR} are immediate.

  \end{itemize}

\end{proof}


\newpage

\section{Soundness of Consistent Subtyping}


\begin{definition}[Filling]   \label{def:filling}
  The filling of a context $[[ | GG | ]]$ solves all unsolved variables:
  \begin{align*}
    [[| empty |]] &= [[ empty ]] \\
    [[|GG , x : aA|]] &= [[|GG| , x : aA]] \\
    [[|GG, a|]] &= [[|GG| , a]] \\
    [[|GG, evar = at|]] &= [[|GG| , evar = at]] \\
    [[|GG, evar|]] &= [[|GG| , evar = int]] \\
    [[|GG, mevar|]] &= [[|GG| , mevar]]
  \end{align*}
\end{definition}


\begin{lemma}[Substitution Stability]
  For any well-formed complete context $(\Omega, \Omega_Z)$, if $\Omega \vdash A$
  then $\ctxsubst{\Omega}{A} = \ctxsubst{\Omega,\Omega_Z}{A}$.
  \label{lemma:subst_stable}
\end{lemma}
\begin{proof}
  By induction on $\Omega_Z$. If $\Omega_Z = [[empty]]$, the result is
  immediate. Otherwise use the i.h. and the fact that $\Omega \vdash A$ implies
  $\textsc{fv}(A) \cap \textsc{dom}(\Omega_Z) = \emptyset$.
\end{proof}



\begin{lemma}[Filling Completes]   \label{lemma:filling_completes}
  If $[[GG --> OO]]$ and $[[(GG, TT)]]$ is well-formed, then $[[GG , TT --> OO, | TT |]]$.
\end{lemma}
\begin{proof}
  By induction on $[[TT]]$, following \cref{def:filling} and applying the rules for context extension.
\end{proof}

\instsoundness*
\begin{proof}
  By induction on the given instantiation derivation.
  \begin{itemize}
  \item Case \[ \drule{instl-solveS} \] Immediate from \cref{lemma:sub_refl}.

  \item Case \[ \drule{instl-solveG} \] Immediate from \cref{lemma:sub_refl}.

  \item Case \[ \drule{instl-solveUS} \]
    We know $[[ [OO] sa ]] = [[tc]] $ for some castable monotype $[[tc]]$, and $[[tc]] \in [[gc]]$.
    By \rref{cs-unknownRR}, we have $[[  [OO] (GG[sa]) |- tc <~ unknown  ]]$


  \item Case \[  \drule{instl-solveUG} \] Similar to the case for \rref{instl-solveUS}.

  \item Case \[ \drule{instl-reachSGOne}  \]
    We know $[[ [OO] sa ]] = [[ [OO]ga]] =  [[tc]] $ and $[[ [OO] gb  ]] = [[ [OO]ga]] = [[tc]]$ for some castable monotype $[[tc]]$.
    By \cref{lemma:sub_refl} we have $[[  [OO](GG[sa][gb] ) |- tc <~ tc   ]]$.

  \item Case \[  \drule{instl-reachSGTwo}   \] Similar to the case for \rref{instl-reachSG1}.

  \item Case \[ \drule{instl-reachOther}  \]
    Let $[[DD]] = [[GG[evar][evarb]  ]]$, we have $[[ [OO]evar ]] = [[t]] $ and $[[  [OO] evarb  ]]  = [[  [OO] evar ]] = [[t]] $
    for some monotype $[[t]]$. By \cref{lemma:sub_refl} we have $[[ [OO]DD |- t <~ t  ]]$.

  \item Case \[ \drule{instl-arr}   \]
    Let $[[GG1]] = [[ GG[evar2, evar1, evar = evar1 -> evar2]  ]]    $:
    \begin{longtable}[l]{ll|l}
      & $[[TT |- evar2 <~~ [TT]aA2 -| DD]]$ & Premise \\
      & $[[ TT --> DD ]]$ & By \cref{lemma:inst_extension} \\
      & $[[ DD --> OO ]]$ & Given \\
      & $[[TT --> OO]]$ & By \cref{lemma:transitivity} \\
      & $[[ GG1 |- aA1 <~~ evar1 -| TT]]$ & Given \\
      & $[[  [OO]DD |- [OO]aA1 <~ [OO]evar1 ]]$ & By i.h. and \Cref{lemma:confluence} \\
      & $[[TT |- evar2 <~~ [TT]aA2 -| DD]]$& Premise \\
      & $[[ [OO]DD |- [OO]evar2 <~ [OO] ([TT]aA2)  ]]$ & By i.h. \\
      & $[[  TT --> DD ]]$ & Above \\
      & $[[ [OO]DD |- [OO]evar2 <~ [OO] aA2  ]]$ & By \cref{lemma:subst_ext_invar} \\
      & $[[  [OO] DD |- [OO] evar1 -> [OO] evar2 <~ [OO]aA1 -> [OO]aA2  ]]$ & By \rref{cs-arrow} \\
      & $[[  [OO] DD |- [OO] evar <~ [OO] (aA1 -> aA2)  ]] $ & By def. of substitution
    \end{longtable}

  \item Case \[ \drule{instl-forallR}  \]
    \begin{longtable}[l]{ll|l}
      & $[[DD , b, TT -->  OO , b , |TT|]]  $ & By \cref{lemma:filling_completes} \\
      & $[[GG[evar] , b |- evar <~~ aB -| DD , b , TT]]$ & Given \\
      & $[[ [OO , b , |TT|](DD , b , TT)  |- [OO , b , |TT|] evar <~ [OO , b , |TT|] aB    ]]$ & By i.h. \\
      & $[[ [OO , b , |TT|](DD , b , TT)  |- [OO , b ] evar <~ [OO , b ] aB    ]]$ & Free variables in $[[evar]]$ and $[[aB]]$ are declared in $[[(OO, b)]]$\\
      & $[[ [OO , b ](DD , b )  |- [OO , b ] evar <~ [OO , b ] aB    ]]$ & By context partitioning and thinning \\
      & $[[ [OO]DD , b   |- [OO] evar <~ [OO] aB    ]]$ & By context substitution \\
      & $[[ [OO]DD    |- [OO] evar <~ \/b . [OO] aB    ]]$ & By \rref{cs-forallR} \\
       & $[[ [OO]DD    |- [OO] evar <~ [OO] (\/b . aB)    ]]$ & By def. of substitution
    \end{longtable}

  \item Case \[ \drule{instr-forallLL}  \]

    \begin{longtable}[l]{ll}
      & $[[DD , mevarb, TT -->  OO , mevarb , |TT|]]  $ \qquad By \cref{lemma:filling_completes} \\
      & $[[GG[evar] , mevarb, sb |- aB[b ~> sb] <~~ evar -| DD, mevarb, TT]]$ \qquad Premise \\
      & $[[ [OO , mevarb , |TT|] (DD , mevarb, TT) |- [OO , mevarb , |TT|] (aB[b ~> sb]) <~ [OO , mevarb , |TT|]evar ]]$ \qquad By i.h. \\
      & $[[ [OO] DD |- ([OO]aB) [b ~> [OO , mevarb , |TT|] sb] <~ [OO]evar ]]$ \qquad By distributivity of substitution \\
      & $[[  [OO] DD |- [OO, mevarb, |TT|]sb  ]]$ \qquad Follows from def. of context application \\
      & $[[  [OO] DD |- \/ b . [OO] aB  <~ [OO] evar   ]]$ \qquad By \rref{cs-forallL} and $[[ [OO, mevarb, |TT| ] sb ]]$ is a monotype  \\
       & $[[  [OO] DD |- [OO]  (\/ b . aB)  <~ [OO] evar   ]]$ \qquad By def. of substitution
    \end{longtable}

  \item The rest of the cases are similar to the above cases.

  \end{itemize}
\end{proof}


\subsoundness*
\begin{proof}
  By induction on the derivation of consistent subtyping.

  \begin{itemize}
  \item Case \[ \drule{as-tvar}  \]

    \begin{longtable}[l]{ll|l}
      & $ [[a in GG[a] ]]  $ & Given \\
      & $ [[ a in [OO](GG[a])  ]]   $ & Follows from def. of context application \\
      & $ [[ [OO](GG[a]) |- a <~ a    ]]  $ & By \rref{cs-tvar} \\
      & $ [[ [OO](GG[a]) |- [OO]a <~ [OO]a    ]]     $ & By def. of substitution
    \end{longtable}


  \item Case \[  \drule{as-evar}   \]

    \begin{longtable}[l]{ll|l}
      & $ [[ [OO]evar  ]]  $ defined & Follows from def. of context application \\
      & $  [[ [OO]DD |- [OO]evar  ]]  $ & Follows from $ [[DD]] = [[ [GG]evar  ]]  $ \\
      & $  [[ [OO]DD |- [OO]evar <~ [OO]evar    ]]  $ & By \cref{lemma:sub_refl}
    \end{longtable}

  \item Case \[  \drule{as-int}  \] Immediate.


  \item Case \[  \drule{as-arrow}    \]

    \begin{longtable}[l]{ll|l}
      & $  [[ GG |- aB1 <~ aA1 -| TT   ]]  $ & Premise \\
      & $   [[ DD --> OO   ]]  $ & Given \\
      & $   [[ TT --> OO       ]]$ & By \cref{lemma:transitivity} \\
      & $[[  [OO] TT |- [OO]aB1 <~ [OO]aA1      ]]$ & By i.h. \\
      & $[[  [OO] DD |- [OO]aB1 <~ [OO]aA1      ]]$ & By \cref{lemma:confluence} \\
      & $ [[ TT |- [TT] aA2 <~ [TT] aB2 -| DD]]   $ & Premise \\
      & $ [[ [OO]DD |- [OO]([TT] aA2) <~ [OO]([TT] aB2) ]]   $ & By i.h. \\
      & $ [[ [OO]([TT] aA2)  ]] = [[ [OO]aA2  ]]  $ & By \cref{lemma:subst_ext_invar} \\
      & $ [[ [OO]([TT] aB2)  ]] = [[ [OO]aB2  ]]  $ & By \cref{lemma:subst_ext_invar} \\
      & $ [[ [OO]DD |- [OO]aA2 <~ [OO]aB2 ]]    $ & By above equalities \\
      & $  [[ [OO]DD |-[OO]aA1 -> [OO]aA2 <~[OO]aB1 -> [OO]aB2 ]]     $ & By \rref{cs-arrow} \\
      & $  [[ [OO]DD |-[OO](aA1 -> aA2)  <~[OO](aB1 -> aB2) ]]       $ & By def. of substitution
    \end{longtable}


  \item Case \[ \drule{as-forallR}  \]

      \begin{longtable}[l]{ll|l}
      & $  [[GG, a --> DD, a , TT]]   $ & By \cref{lemma:sub_extension} \\
      & $[[TT]]$ is soft & By \cref{lemma:extension_order} (1) where $[[GR]] = [[empty]]$ \\
      & $[[ DD --> OO ]]$ & Given \\
      & $\underbrace{[[DD , a, TT]]}_{[[DD']]} [[-->]] \underbrace{  [[OO , a, |TT| ]]   }_{[[OO']]}$ & By \cref{lemma:filling_completes} \\
      & $  [[ GG, a |- aA <~ aB -| DD, a, TT ]]   $ & Given \\
      & $ [[ [OO']DD' |- [OO']aA <~ [OO']aB  ]]  $ & By i.h. \\
      & $ [[ [OO']aA  ]] = [[ [OO , a] aA  ]]   $ & By \cref{lemma:subst_stable} \\
      & $ [[ [OO']aB  ]] = [[ [OO , a] aB  ]]   $ & By \cref{lemma:subst_stable} \\
      & $ [[ [OO']DD'   ]]  = [[ [OO, a](DD, a)   ]] $ & By \cref{lemma:subst_go_away} \\
      & $[[ [OO, a](DD, a) |- [OO, a]aA <~ [OO, a]aB  ]]$ & By above equalities \\
      & $ [[ [OO]DD, a |- [OO]aA <~ [OO]aB  ]]    $ & By def. of substitution \\
      & $ [[ [OO]DD |- [OO]aA <~ \/a. [OO]aB  ]]  $ & By \rref{cs-forallR} \\
      & $ [[ [OO]DD |- [OO]aA <~ [OO] (\/a. aB)  ]]   $ & By def. of substitution
      \end{longtable}


    \item Case \[  \drule{as-forallLL}  \]

      \begin{longtable}[l]{ll|l}
        & Let $[[OO']] = [[OO, msa, |TT|]]$ \\
        & $[[  DD --> OO  ]]$ & Given \\
        & $[[DD, msa, TT --> OO']]$  & By \cref{lemma:filling_completes} \\
        & $[[ GG, msa, sa |- aA [a ~> sa] <~ aB -| DD, msa, TT]]$ & Premise \\
        & $[[ [OO'](DD, msa, TT) |- [OO'](aA [a ~> sa]) <~ [OO']aB     ]]$ & By i.h. \\
        & $ [[ [OO']aB  ]] = [[ [OO , a] aB  ]]   $ & By \cref{lemma:subst_stable} \\
        & $ [[ [OO](DD, msa, TT) |-  [OO'] aA [a ~> [OO']sa] <~ [OO]aB     ]]  $ & By distributivity of substitution \\
        & $[[ [OO](DD, msa, TT) |- [OO']sa    ]]$ & Follows from def. of context application \\
        & $ [[ [OO](DD, msa, TT) |- \/a. [OO'] aA <~ [OO]aB     ]]  $ & By \rref{cs-forallL} \\
        & $ [[ [OO]DD |- \/a. [OO] aA <~ [OO]aB     ]]  $ & By context partitioning \\
        & $[[ [OO]DD |- [OO] (\/a. aA) <~ [OO]aB     ]] $ & By def. of substitution
      \end{longtable}

    \item Case \[  \drule{as-spar}  \] Immediate from \rref{cs-spar}.

    \item Case \[  \drule{as-gpar}  \] Immediate from \rref{cs-gpar}.


    \item Case \[ \drule{as-unknownLL}   \] Immediate from \rref{cs-unknownLL}.

    \item Case \[ \drule{as-unknownRR}   \] Immediate from \rref{cs-unknownRR}.

    \item Case \[  \drule{as-instL}     \]
      \begin{longtable}[l]{ll|l}
        & $[[ GG[evar] |- evar <~~ aA -| DD]]$ & Premise \\
        & $[[  [OO]DD |- [OO]evar <~ [OO]aA   ]]$ & By \cref{thm:inst_soundness}
      \end{longtable}

    \item Case \[  \drule{as-instR}     \] Similar to the case for \rref{as-instL}.

  \end{itemize}
\end{proof}


\newpage

\section{Soundness of Typing}

\textbf{Note:} We use $\byhave$ to improve readability when the conclusion has several parts.


\begin{lemma}[Matching Extension]   \label{lemma:match_extension}
  If $[[  GG |- aA |> aA1 -> aA2 -| DD]]$ then $ [[GG --> DD]]  $.
\end{lemma}
\begin{proof}
  By induction on the given derivation.
  \begin{itemize}
  \item Case \[ \drule{am-forallL} \]
    By i.h., we have $[[ GG, sa --> DD  ]]$. By \cref{lemma:drop_ext}, we have $[[GG --> DD]]$.

  \item Case \[  \drule{am-arr}  \] Immediate by \cref{lemma:reflexivity}.

  \item Case \[ \drule{am-unknown}   \] Immediate by \cref{lemma:reflexivity}.

  \item Case \[ \drule{am-var} \]
    By applying \cref{lemma:unsolved_ext} twice, we have
    $ [[ GG[evar] --> GG[evar1, evar2, evar]  ]]   $. By
    \cref{lemma:solution_ext}, we have $[[ GG[evar1, evar2, evar] --> GG[evar1, evar2, evar = evar1 -> evar2]  ]]$. By
    \cref{lemma:transitivity}, we have $[[ GG[evar] --> GG[evar1, evar2, evar = evar1 -> evar2]  ]]$.
  \end{itemize}

\end{proof}


\begin{lemma}[Typing Extension]   \label{lemma:typing_extension}
  If $ [[GG |- ae => aA -| DD]] $ or $ [[ GG |- ae <= aA -| DD  ]]  $ then $[[ GG --> DD]]$.
\end{lemma}
\begin{proof}
  By induction on the given derivation.

  \begin{itemize}
  \item Case \[ \drule{inf-var}  \] Immediate by \cref{lemma:reflexivity}.

  \item Case \[  \drule{inf-int}  \] Immediate by \cref{lemma:reflexivity}.

  \item Case \[ \drule{inf-lamTwo}   \]
    By i.h., we have $[[ GG, sa, sb, x : sa --> DD, x : sa, TT  ]]$. By \cref{lemma:extension_order}, we have
    $[[ GG, sa, sb --> DD  ]]$. By definition, we have $[[  GG --> GG, sa, sb  ]]$. By \cref{lemma:transitivity}
    we have $[[ GG --> DD ]]$.

  \item Case \[  \drule{inf-lamannTwo}  \]
    By i.h., we have $[[ GG, sb, x : aA --> DD, x : aA, TT  ]]   $. By \cref{lemma:extension_order},
    we have $[[GG --> DD]]$.

  \item Case \[ \drule{inf-app}   \]
    By i.h., we have $[[  GG --> TT1  ]]$, $[[ TT2 --> DD]]$. By \cref{lemma:match_extension},
    we have $[[ TT1 --> TT2  ]]$. By \cref{lemma:transitivity} , we have
    $[[GG --> DD]]$.


  \item Case \[ \drule{inf-anno}   \]
    By i.h., we have $[[ GG --> DD ]]$.

  \item Case \[ \drule{chk-gen}   \]
    By i.h., we have $[[  GG, a --> DD, a, TT  ]]$. By
    \cref{lemma:extension_order} we have $[[ GG --> DD]]$.

  \item Case \[ \drule{chk-lam} \]
    By i.h., we have $[[GG , x : aA --> DD, x : aA, TT]]$.
    By \cref{lemma:extension_order} we have $[[GG --> DD]]$.

  \item Case \[ \drule{chk-sub} \]
    By i.h., we have $[[GG --> TT]]$. By
    \cref{lemma:sub_extension} we have $[[TT --> DD]]$. By
    \cref{lemma:transitivity} we have $[[GG --> DD]]$.
  \end{itemize}

\end{proof}

\begin{theorem}[Matching Soundness]  \label{thm:match_soundness}%
  If $[[GG |- aA |> aA1 -> aA2 -| DD]]$ where $[[ [GG]aA = aA  ]]$ and $[[ DD --> OO ]]$
  then $[[  [OO]DD |- [OO]aA |> [OO]aA1 -> [OO]aA2 ]]$.
\end{theorem}
\begin{proof}
  By induction on the given derivation.

  \begin{itemize}
  \item Case \[ \drule{am-forallL}  \]
    \begin{longtable}[l]{ll|l}
      & $[[ GG , sa |- aA[a ~> sa] |> aA1 -> aA2 -| DD]]$ & Premise \\
      & $[[DD --> OO]]$ & Given \\
      & $[[ [OO]DD |- [OO](aA[a ~> sa]) |> [OO]aA1 -> [OO]aA2   ]]$ & By i.h. \\
      & $[[ [OO]DD |- [OO]aA [a ~> [OO]sa] |> [OO]aA1 -> [OO]aA2   ]]$ & By distributivity of substitution \\
      & $[[ [OO]DD |- [OO]sa   ]]$ & Follows from def. of context application \\
      & $[[ [OO]DD |- \/a. [OO]aA |> [OO]aA1 -> [OO]aA2   ]]$ & By \rref{m-forall} \\
      & $[[ [OO]DD |- [OO] (\/a. aA) |> [OO]aA1 -> [OO]aA2   ]]$ & By def. of substitution
    \end{longtable}


  \item Case \[  \drule{am-arr}  \] Immediate from \rref{m-arr}.

  \item Case \[  \drule{am-unknown}  \] Immediate from \rref{m-unknown}.

  \item Case \[  \drule{am-var}  \]
      \begin{longtable}[l]{ll|l}
        &$[[DD --> OO]]$& Given \\
        & $[[ [OO]evar  ]] = [[ [OO]evar1 -> [OO]evar2   ]]$ & By def. of context application \\
        & $[[  [OO]DD |- [OO]evar1 -> [OO]evar2 |> [OO]evar1 -> [OO]evar2  ]]$ & By \rref{m-arr}
      \end{longtable}

  \end{itemize}

\end{proof}


\typingsoundness*
\begin{proof}
  By induction on the given derivation.

  \begin{itemize}
  \item Case \[ \drule{inf-var}  \]
    \begin{longtable}[l]{ll|l}
      &$[[(x : aA) in GG]]$ & Premise \\
      &$[[(x : aA) in  DD]]$ & $[[DD]] = [[OO]]$ \\
      &$[[DD --> OO]]$ & Given \\
      &$[[ (x : [OO]aA) in [OO]GG  ]]$ & By \Cref{lemma:variable_preservation} \\
      $\byhave$&$[[ [OO]GG |- x : [OO]aA   ]]$ & By \rref{var} \\
      $\byhave$& $\erase{[[x]]} = \erase{[[x]]}$ & By def. of erasure
    \end{longtable}

  \item Case \[ \drule{inf-int}  \]
    \begin{longtable}[l]{ll|l}
      $\byhave$&$[[ [OO]GG |- n : int  ]]$ & By \rref{int} \\
      $\byhave$& $\erase{[[n]]} = \erase{[[n]]}$ & By def. of erasure
    \end{longtable}

  \item Case \[ \drule{inf-lamannTwo}  \]

    \begin{longtable}[l]{ll|l}
      & $[[  GG , sa, x : aA --> DD, x : aA , TT  ]]$ & By \Cref{lemma:typing_extension} \\
      & $[[TT]]$ is soft & By \Cref{lemma:extension_order} where $[[GR]] = [[empty]]$ \\
      & $[[ DD --> OO]]$ & Given \\
      & $\underbrace{[[DD , x : aA , TT]]}_{[[DD']]} [[-->]] \underbrace{[[OO, x : aA, |TT|]] }_{[[OO']]}$ & By \Cref{lemma:filling_completes} \\
      & $[[ GG, x: aA |- ae => aB -| DD, x : aA, TT ]]$ & Premise \\
      & $[[  [OO'] DD' |- e' : [OO']sb   ]]$ & By i.h. \\
      & $\erase{[[ae]]} = \erase{e'}$ & above \\
      & $[[ [OO']sb   ]] = [[ [OO, x : aA]sb  ]] = [[ [OO]sb  ]]$ & By \Cref{lemma:subst_stable} and def. of substitution \\
      & $[[ [OO']DD' ]]$ = $[[ [OO]DD , x : [OO]aA   ]]$ & By \Cref{lemma:subst_go_away} and def. of context substitution \\
      & $[[  [OO] DD, x : [OO]aA |- e' : [OO]sb   ]] $ & By above equalities \\
      & $[[  [OO] DD |- \ x : [OO]aA . e' : [OO]aA -> [OO]sb   ]] $ & By \rref{lamann} \\
      & $[[ [OO]aA  ]] = [[A]]$ & Type annotations cannot contain evars \\
      & $[[  [OO] DD |- \ x : A . e' :[OO]aA -> [OO]sb   ]]$ & By above equality \\
      $\byhave$& $ [[  [OO] DD |- \ x : A . e' :[OO](aA -> sb)   ]] $ & By def. of substitution \\
      $\byhave$& $\erase{[[\x : A . e']]} = \erlam{x}{\erase{e'}} = \erlam{x}{\erase{e}} = \erase{\blam{x}{A}{e}} $ & By def. of erasure
    \end{longtable}



  \item Case \[  \drule{inf-lam}  \]

    \begin{longtable}[l]{ll|l}
      &$[[ GG, sa, sb, x : sa --> DD, x : sa, TT   ]]$& By \cref{lemma:typing_extension} \\
      &$[[ GG , sa, sb --> DD  ]]$& By \cref{lemma:extension_order} \\
      &$[[TT]]$ is soft & Above \\ \\
      &$[[ DD --> OO  ]]$  & Given \\
      &$[[  DD , x : sa --> OO , x : [OO]sa  ]]$  & By def \\
      &$\underbrace{[[DD, x : sa, TT]]}_{[[DD']]} [[-->]] \underbrace{ [[OO, x : [OO]sa, |TT|]] }_{[[OO']]}$  & By \cref{lemma:filling_completes} \\ \\

      & $ [[  GG, sa, sb, x : sa |- ae <= sb -| DD, x : sa, TT ]] $ & Premise \\
      & $[[ [OO']DD' |- e' : [OO']sb   ]]$ & By i.h. \\
      & $\erase{[[ae]]} = \erase{[[e']]}$ & Above \\
      & $[[ [OO']sb  ]] = [[ [OO]sb  ]]$ & By def. of context substitution \\
      & $[[  [OO']DD'  ]] = [[ [OO]DD, x : [OO]sa  ]]  $ & By def. of context substitution \\
      & $[[ [OO]DD, x : [OO]sa |- e' : [OO]sb   ]]$ & By above equalities \\
      & $[[ [OO]sa ]]$ is a monotype & $\Omega$ is predicative \\
      & $[[ [OO]DD |-\x .  e' : [OO]sa -> [OO]sb   ]] $ & By \rref{lam} \\
      $\byhave$& $[[ [OO]DD |-\x .  e' : [OO](sa -> sb)  ]]$ & By def. of substitution \\
      $\byhave$& $\erase{\erlam{x}{e}} = \erlam{x}{\erase{e}} = \erlam{x}{\erase{e'}} = \erase{\erlam{x}{e'}}$ & By def. of erasure
    \end{longtable}



  \item Case \[  \drule{inf-app}  \]

    \begin{longtable}[l]{ll|l}
      & $[[ DD --> OO  ]]$ & Given \\
      & $[[TT1 --> OO]]$ & By \cref{lemma:match_extension,lemma:typing_extension,lemma:transitivity} \\
      & $[[ GG |- ae1 => aA -| TT1]]$ & Premise \\
      & $[[ [OO]TT1 |- e1' : [OO]aA    ]]$ & By i.h. \\
      & $\erase{[[e1']]} = \erase{[[ae1]]}$ & above \\
      & $[[ [OO]TT1  ]] = [[ [OO]DD  ]]$ & By \Cref{lemma:confluence} \\
      & $[[ [OO]DD |- e1' : [OO]aA    ]]$ & By above equality \\
      & $[[ TT2 |- ae2 <= [TT2]aA1 -| DD ]]$ & Premise \\
      & $[[ [OO] DD |- e2' : [OO]aA1   ]]$ & By i.h. \\
      & $\erase{[[e2']]} = \erase{[[ae2]]}$ & Above \\
      & $[[ TT1 |- [TT1] aA |> aA1 -> aA2 -| TT2]]$ & Premise \\
      & $[[ [OO]TT2 |- [OO]([TT1] aA) |> [OO]aA1 -> [OO]aA2     ]]$ & By \Cref{thm:match_soundness} \\
      & $[[ [OO]TT2  ]] =  [[ [OO]DD  ]]$ & By \Cref{lemma:confluence} \\
      & $[[ [OO]([TT1]aA)  ]] =  [[  [OO]aA  ]]$ & By \Cref{lemma:subst_ext_invar} \\
      & $[[ [OO]DD |- [OO]aA |> [OO]aA1 -> [OO]aA2     ]]$ & By above equalities \\
      & $[[ [OO]DD |- [OO]aA1 <~ [OO]aA1    ]]$ & By \cref{lemma:sub_refl} \\
      $\byhave$& $[[  [OO]DD |- e1' e2' : [OO]aA2   ]]$ & By \rref{app} \\
      $\byhave$& $\erase{e_1' ~ e_2'} = \erase{e_1'} ~ \erase{e_2'} = \erase{e_1} ~ \erase{e_2} = \erase{e_1 ~ e_2}$ & By def. of erasure
    \end{longtable}


  \item Case \[  \drule{inf-anno}  \]
    \begin{longtable}[l]{ll|l}
      & $[[  GG |- ae <= aA -| DD ]]$ & Premise \\
      $\byhave$ & $ [[  [OO]DD |- e' : [OO]aA  ]]  $ & By i.h., \\
      & $\erase{[[e]]} = \erase{[[e']]}$ & Above \\
      $\byhave$ & $\erase{e : A} = \erase{e} = \erase{e'}$ & By above equality and the def. of erasure
    \end{longtable}



  \item Case \[ \drule{chk-gen}  \]

    \begin{longtable}[l]{ll|l}
      & $[[ DD --> OO ]]$ & Given \\
      & $[[ DD , a --> OO , a]]$ & By def \\
      & $[[ GG , a --> DD , a, TT  ]]$ & By \cref{lemma:typing_extension} \\
      & $[[TT]]$ is soft & By \cref{lemma:extension_order} \\
      & $\underbrace{[[DD, a, TT]]}_{[[DD']]} [[-->]] \underbrace{[[OO, a, |TT|]]}_{[[OO']]}$ & By \cref{lemma:filling_completes} \\
      & $[[ GG, a |- ae <= aA -| DD , a , TT]]$  & Premise \\
      & $[[ [OO']DD' |- e' : [OO']aA  ]]$  & By i.h., \\
      $\byhave$& $\erase{[[ae]]} = \erase{[[e']]}$ & Above \\
      & $[[  [OO']aA  ]] = [[ [OO]aA ]]$ & By \cref{lemma:subst_stable} \\
      & $[[ [OO']DD'  ]]$ = $[[ [OO]DD, a  ]]$ & By \Cref{lemma:subst_go_away} and def. of context substitution \\
      & $[[ [OO]DD, a |- e' : [OO]aA  ]]$  & By above equalities \\
      & $[[  [OO]DD |- e' : \/a. [OO]aA  ]]$  & By \rref{gen} \\
      $\byhave$& $[[ [OO]DD |- e' : [OO] (\/a. aA)  ]]$  & By def. of substitution

    \end{longtable}


  \item Case \[ \drule{chk-lam}  \]

    \begin{longtable}[l]{ll|l}
      & $[[  DD --> OO  ]]$ & Given \\
      & $[[ DD, x : aA --> OO, x : [OO]aA   ]]$ & By def \\
      & $[[ GG , x : aA --> DD, x : aA, TT   ]]$ & By \cref{lemma:typing_extension} \\
      & $[[TT]]$ is soft & By \cref{lemma:extension_order} \\
      & $\underbrace{[[DD, x : aA, TT]]}_{[[DD']]} [[-->]] \underbrace{[[OO, x : [OO]aA, |TT| ]]}_{[[OO']]}$ & By \cref{lemma:filling_completes} \\
      & $[[ GG, x : aA |- ae <= aB -| DD, x : aA, TT]]$ & Premise \\
      & $[[ [OO']DD' |- e' : [OO']aB ]]$ & By i.h., \\
      & $\erase{[[ae]]} = \erase{[[e']]}$ & Above \\
      & $[[ [OO']aB  ]] = [[ [OO]aB  ]]$ & By \cref{lemma:subst_stable} \\
      & $[[ [OO']DD'  ]] = [[ [OO]DD, x : [OO]aA  ]] $ & By \Cref{lemma:subst_go_away} and def. of context substitution \\
      & $[[ [OO]DD, x : [OO]aA |- e' : [OO]aB ]]$ & By above equalities \\
      & $[[ [OO]DD |- \x : [OO]aA . e' : [OO]aA -> [OO]aB ]] $ & By \rref{lamann} \\
      $\byhave$ & $[[ [OO]DD |- \x : [OO]aA . e' : [OO](aA -> aB) ]]$ & By def. of substitution \\
      $\byhave$ & $\erase{\erlam x e} = \erlam x {\erase{e}} = \erlam x {\erase{e'}} = \erase{[[\x : [OO]aA . e']]}$ & By the def. of erasure
    \end{longtable}


  \item Case \[ \drule{chk-sub}  \]

    \begin{longtable}[l]{ll|l}
      & $[[ TT |- [TT]aA <~ [TT]aB -| DD]]$ & Premise \\
      & $[[TT --> DD]]$ & By \cref{lemma:sub_extension} \\
      & $[[DD -->OO ]]$ & Given \\
      & $[[TT --> OO]]$ & By \cref{lemma:transitivity} \\
      & $[[ GG |- ae => aA -| TT]]$ & Premise \\
      & $[[ [OO]TT |- e' : [OO]aA   ]]$ & By i.h., \\
      & $\erase{[[ae]]} = \erase{[[e']]}$ & Above \\
      & $[[ [OO]TT   ]] = [[ [OO]DD  ]]$ & By \cref{lemma:confluence} \\
      & $[[ [OO]DD |- e' : [OO]aA   ]] $ & By above equality \\
      & $[[ [OO]DD |- [OO]([TT]aA) <~ [OO]([TT]aB) ]] $ & By \Cref{thm:sub_soundness} \\
      & $[[ [OO]([TT]aA)  ]] = [[ [OO]aA ]]$ & By \cref{lemma:subst_ext_invar} \\
      & $[[ [OO]([TT]aB)  ]] = [[ [OO]aB ]]$ & By \cref{lemma:subst_ext_invar} \\
      & $[[ [OO]DD |- [OO]aA <~ [OO]aB ]]$ & By above equalities \\
      $\byhave$& $\ctxsubst{\Omega}{\Delta} \vdash (e' : [[ [OO]aB   ]]) : [[ [OO]aB   ]]$ & By def. annotation \\
      $\byhave$& $\erase{(e' : [[ [OO]aB  ]])} = \erase{e'} = \erase{e}$ & By def. erasure
    \end{longtable}



  \end{itemize}


\end{proof}

\newpage

\section{Completeness of Consistent Subtyping}

\instcomplete*
\begin{proof}
  By mutual induction on the given derivation.
  \begin{enumerate}
  \item We have $ [[  [OO]GG |- [OO]evar <~ [OO]aA  ]]   $. We case analyze the shape of $[[aA]]$.
    \begin{itemize}
    \item Case $[[aA]] = [[unknown]], [[evar = sa]]$:
      \begin{longtable}[l]{ll|l}
        & $[[  [OO]GG |- [OO]sa <~ [OO]unknown  ]]$ & Given \\
        & $[[ [OO]unknown  ]] = [[unknown]]$ \\
        & $[[  [OO]GG |- [OO]sa <~ unknown  ]]$ & By above equality \\
        & $[[sa]] \notin \textsc{unsolved}([[GG]])$ & Given \\
        & $[[ GG  ]] = [[ GL, sa, GR   ]]$ & Above \\
        & $[[ GL, sa, GR --> OO   ]]$ & Given \\
        & $[[ OO ]] = [[ OL, sa = atc, OR  ]]   $ & By \cref{lemma:extension_order} and $[[OO]]$ is complete and $[[ [OO]sa ]] \in [[agc]]$ \\
        & $[[ GL --> OL   ]]$ & Above \\
        & Let $[[DD]] = [[ GL, ga, sa = ga, GR ]]$ \\
        & and $[[OO']] = [[OL, ga = atc, sa = atc, OR]]$ \\
        $\byhave$& $[[ GG |- sa <~~ unknown -| DD    ]]$ & By \rref{instl-solveUS} \\
        $\byhave$& $\Delta \exto \Omega'$ & By \cref{lemma:paralell_admit,lemma:paralell_ext_solu} \\
        $\byhave$& $\Omega \exto \Omega'$ & By \cref{lemma:unsolved_ext,lemma:solution_ext}
      \end{longtable}
    \item Case $A = [[unknown]], [[evar = ga]]$:
      \begin{longtable}[l]{ll|l}
        & $[[  [OO]GG |- [OO]ga <~ [OO]unknown  ]]$ & Given \\
        & $[[ [OO]unknown  ]] = [[unknown]]$ \\
        & $[[  [OO]GG |- [OO]ga <~ unknown  ]]$ & By above equality \\
        & $[[ga]] \notin \textsc{unsolved}([[GG]])$ & Given \\
        & $[[ GG  ]] = [[ GG0[ga]   ]]$ & Above \\
        & Let $[[DD]] = [[ GG0[ga] ]]$ and $[[OO']] = [[OO]]$\\
        $\byhave$& $[[ GG |- ga <~~ unknown -| DD ]]$ & By \rref{instl-solveUG} \\
        $\byhave$& $[[ DD --> OO' ]]$ & Given \\
        $\byhave$& $[[ OO --> OO']]$ & By \cref{lemma:reflexivity}
      \end{longtable}

    \item Case $[[aA]] = [[gb]], [[evar]] = [[sa]]$:
      \begin{longtable}[l]{ll|l}
        & $[[ [OO]GG |- [OO]sa <~ [OO]gb  ]]$ & Given \\
        & $[[ [OO]GG |- t <~ tc   ]]$ & Let $[[ [OO]sa]] = [[t]]$ and $[[ [OO]gb  ]] = [[tc]]$ and $[[OO]]$ is predicative \\
        & $[[t]] = [[tc]]$ & By \cref{lemma:mono_equal} \\ \\
        & $[[ [GG]gb ]] = [[gb]]$ & Given \\
        & $[[gb]] \in \textsc{unsolved}{([[GG]])}$ & Above
      \end{longtable}
      Now consider whether $[[sa]]$ is declared to the left of $[[gb]]$.
      \begin{itemize}
      \item Case $[[GG]] = [[  GG0, sa, GG1, gb, GG2 ]]$
        \begin{longtable}[l]{ll|l}
          & Let $[[DD]] = [[ GG0, ga, sa = ga, GG1, gb = ga, GG2     ]]$ & \\
          $\byhave$& $[[  GG |- sa <~~ gb -| DD ]]$ & By \rref{instl-reachSG1} \\
          & $ [[GG --> OO]]   $ & Given \\
          & $[[OO]] = [[ OO0, sa = atc, OO1, gb = atc, OO2     ]]$ & By \cref{lemma:extension_order}   \\
          & Let $[[OO']] = [[ OO0, ga = atc, sa = atc, OO1, gb = atc, OO2     ]]$  \\
          $\byhave$& $[[OO --> OO']]$ & By \cref{lemma:unsolved_ext,lemma:solution_ext} \\
          $\byhave$& $[[DD --> OO' ]]$ & By \cref{lemma:paralell_admit,lemma:paralell_ext_solu}
        \end{longtable}
      \item Case $[[GG]] = [[  GG0, gb, GG1, sa, GG2 ]]$
        \begin{longtable}[l]{ll|l}
          & Let $[[DD]] = [[  GG0, gb, GG1, sa = gb, GG2  ]]$ & \\
          $\byhave$& $ [[GG |- sa <~~ gb -| DD]] $ & By \rref{instl-reachOther} \\
          $\byhave$& $[[DD --> OO]]$ & By \Cref{lemma:paralell_ext_solu} \\
          $\byhave$& $[[OO --> OO]]$ & By \Cref{lemma:reflexivity}
        \end{longtable}
      \end{itemize}
    \item Case $[[aA]] = [[sb]]$ is similar to the above case.
    \item Case $A = a$:
      \begin{longtable}[l]{ll|l}
        &$[[  [OO]GG |- [OO]evar <~ [OO]a  ]]$& Given \\
        &$[[  [OO]GG |- [OO]evar <~ a  ]]$ & From $[[  [OO]a ]] = [[a]]$ \\
        & $[[ [OO]evar ]] = [[a]]$ & By inversion of \rref{cs-tvar} \\
        & $[[a]]$ is declared to the left of $[[evar]]$ in $[[OO]]$ & $[[OO]]$ is well-formed \\
        & $[[GG --> OO]]$ & Given \\
        & $[[a]]$ is declared to the left of $[[evar]]$ in $[[GG]]$ & By \Cref{lemma:reverse_preserve} \\
        & Let $[[GG]] = [[ GG0[a][evar]   ]]$ \\
        & Let $[[DD]] = [[ GG0[a][evar = a] ]]$ \\
        $\byhave$& $[[ GG |- evar <~~ a -| DD ]]$ & By \rref{instl-solveS} or \rref{instl-solveG} \\
        $\byhave$& $\Delta \exto \Omega$ & By \Cref{lemma:paralell_ext_solu} \\
        $\byhave$& $\Omega \exto \Omega$ & By \Cref{lemma:reflexivity}
      \end{longtable}
    \item Case $[[aA]] = [[aA1 -> aA2]]$:
      \begin{longtable}[l]{ll|l}
        & $[[  [OO]GG |- [OO]evar <~ [OO]aA1 -> [OO]aA2  ]]$ & Given \\
        & $[[ [OO]evar  ]] = [[ t1 -> t2 ]]$ & $\Omega$ is predicative \\
        & $[[  [OO]GG |- [OO]aA1 <~ t1  ]]$ & By inversion of \rref{cs-arrow} \\
        & $[[  [OO]GG |- t2 <~ [OO]aA2  ]]$ & Above \\
        & $[[GG]] = [[ GG0[evar] ]]$ & From $[[evar]] \in \textsc{unsolved}{([[GG]])}$ \\
        & $[[GG0[evar] ]] [[-->]] \underbrace{[[  GG0[evar2, evar1, evar = evar1 -> evar2 ]   ]]}_{[[GG1]]}$ \\
        & $[[ GG --> OO ]]$ & Given \\
        & $[[OO]] = [[ OO0[evar = at0] ]]$ & From $\genA \in \textsc{unsolved}{(\Gamma)}$ \\
        & $[[ OO0[evar = at0] ]] [[-->]] \underbrace{[[ OO0[evar2 = at2, evar1 = at1, evar = evar1 -> evar2]  ]]}_{[[OO1]]}$ \\ \\
        & $[[ [OO]GG   ]] = [[ [OO1]GG1 ]]$ & By \Cref{lemma:finish_complete} \\
        & $[[ [OO]aA1  ]] = [[ [OO1]aA1  ]]$ & By \Cref{lemma:finish_types} \\
        & $[[t1]] = [[ [OO1]evar1 ]]$ & From def. of $[[OO1]]$ \\
        & $[[  [OO1]GG1 |- [OO1]aA1 <~ [OO1]evar1  ]]$ & By above equalities \\
        & $[[ GG1 |- aA1 <~~ evar1 -| DD2   ]]$ & By i.h. \\
        & $[[ DD2 --> OO2]]$ and $[[ OO1 --> OO2 ]]$ & Above \\ \\
        & $[[ [OO]GG   ]] = [[ [OO2]GG2 ]]$ & By \Cref{lemma:finish_complete} \\
        & $[[ [OO]aA2  ]] = [[ [OO2]aA2  ]] = [[ [OO2]([DD2]aA2)  ]]$ & By \Cref{lemma:finish_types} \\
        & $[[t2]] = [[ [OO2]evar2 ]]$ & By  $[[OO1 --> OO2]]$ \\
        & $[[ [OO2]DD2 |- [OO2]evar2 <~ [OO2]([DD2]aA2)   ]]$ & By above equalities \\
        & $[[  DD2 |-  evar2 <~~ [DD2]aA2 -| DD ]]$ & By i.h. \\
        & $[[ OO2 --> OO' ]]$ & Above \\
        $\byhave$& $[[ DD --> OO'   ]]$ & Above \\
        $\byhave$& $[[ GG0[evar] |- evar <~~ aA1 -> aA2 -| DD  ]]$ & By \rref{instl-arr} \\
        $\byhave$& $[[OO --> OO' ]]$ & By \Cref{lemma:transitivity}
      \end{longtable}
    \item Case $[[A]] = [[int]]$:
      \begin{longtable}[l]{ll|l}
        & $[[  [OO]GG |- [OO]evar <~ [OO]int   ]]$ & Given \\
        & $[[ [OO]int  ]] = [[int]]$ \\
        & $[[  [OO]GG |- [OO]evar <~ int   ]]$ & By above equality \\
        & $[[ [OO]evar ]] = [[int]]$ & $\Omega$ is predicative \\
        & $[[evar]] \in \textsc{unsolved}{(\Gamma)}$ & Given \\
        & $[[GG]] = [[GG0[evar] ]]$ & Above \\
        & Let $[[DD]] = [[ GG0[evar = int]  ]]$ and $[[OO']] = [[OO]]$\\
        & $[[ GG0[evar] |- evar <~~ int -| DD  ]]$ & By \rref{instl-solveS} or \rref{instl-solveG} \\
        & $[[GG --> OO]]$ & Given \\
        & $[[  GG0[evar = int] --> OO  ]]$ & By \Cref{lemma:paralell_ext_solu}
      \end{longtable}
    \item Case $[[aA]] = [[  \/b . aB  ]]$:
      \begin{longtable}[l]{ll|l}
        & $[[ [OO]GG |- [OO]evar <~ \/b. [OO]aB  ]]$ & Given \\
        & $[[ [OO]evar ]]$ cannot be a quantifier & $[[OO]]$ is predicative \\
        & $[[ [OO]GG, b |- [OO]evar <~ [OO]aB  ]] $ & By inversion of \rref{cs-forallR} \\ \\
        & $[[ [OO]GG, b   ]] = [[   [OO, b](GG, b)  ]]$ & By def. of context substitution \\
        & $[[ [OO]evar  ]] = [[  [OO,b]evar ]]$ & By def. of substitution \\
        & $[[ [OO]aB  ]] = [[  [OO, b]aB  ]]$ & By def. of substitution \\
        & $ [[ [OO, b](GG, b) |- [OO, b]evar <~ [OO, b]aB  ]]  $ & By above equalities \\
        & $ [[GG, b |- evar <~~ aB -| DD0 ]]  $ & By i.h. \\
        & $[[ DD0 --> OO' ]]$ & Above \\
        & $[[  OO, b --> OO'  ]]$ & Above \\
        $\byhave$& $[[  OO --> OO'  ]]$ & By \cref{lemma:drop_ext}\\  \\
        & $[[ GG, b --> DD0 ]]$ & By \Cref{lemma:inst_extension} \\
        & $[[DD0]] = [[DD, b, DD']]$ & By \Cref{lemma:extension_order} \\
        & $[[  GG --> DD ]]$ & Above \\
        $\byhave$& $[[  DD --> OO'  ]]$ \\
        $\byhave$& $ [[ GG |- evar <~~ \/b. aB -| DD  ]]  $ & By \rref{instl-forallR}
      \end{longtable}
    \end{itemize}
  \item Now we have $[[ [OO]GG |- [OO]aA <~ [OO]evar ]]$. These cases are mostly symmetric. The one exception is when $[[ aA  ]] = [[  \/b. aB ]]$.
    \begin{itemize}
    \item Case $[[aA]] = [[\/b . aB]]$:
      \begin{longtable}[l]{ll}
        & $[[ [OO]GG |- \/b . [OO]aB <~ [OO]evar   ]]$ \qquad Given \\
        & $[[ [OO]evar  ]]$ cannot be a quantifier \qquad $[[OO]]$ is predicative \\
        & $[[  [OO]GG |- t  ]]$ \qquad By inversion of \rref{cs-forallL} \\
        & $[[ [OO]GG |- ([OO]aB) [b ~> t] <~ [OO]evar    ]]$ \qquad Above \\ \\
        & $[[ [OO]GG ]] = [[  [OO, msb, sb = at] (GG, msb, sb)  ]]$ \qquad By def. of context application \\
        & $[[ ([OO]B) [b ~> t] ]] = [[  [OO, msb, sb = at] (aB [b ~> sb])  ]]$ \qquad by def. of substitution \\
        & $[[  [OO]evar  ]] = [[ [OO, msb, sb = at]evar  ]]$ \qquad By def. of substitution \\
        & $[[   [OO, msb, sb = at] (GG, msb, sb)   |-   [OO, msb, sb = at] (aB [b ~> sb])   <~ [OO, sb = at ]evar    ]]$ \qquad By above equalities \\
        & $[[ GG, msb, sb |- aB [b ~> sb] <~~ evar -| DD   ]]$ \qquad By i.h. \\
        &$[[GG, msb, sb --> DD]]$ \qquad By \cref{lemma:inst_extension} \\
        & $[[DD]] = [[DL, msb, DR]]$ \qquad By \cref{lemma:extension_order} \\
        & $[[ GG --> DL ]]$ \qquad Above \\
        & $[[ OO, msb, sb = at --> OO']]$ \qquad Above \\
        & $[[OO']] = [[OL, msb, OR]]$ \qquad By \cref{lemma:extension_order} \\
        $\byhave$& $[[ OO --> OL ]]$ \qquad Above \\
        $\byhave$& $[[ DL --> OL ]]$ \qquad \cref{lemma:transitivity} \\
        $\byhave$ & $[[ GG |- \/b . aB <~~ evar -| DL  ]]$ \qquad By \rref{instr-forallLL}
      \end{longtable}
    \end{itemize}
  \end{enumerate}
\end{proof}


\begin{landscape}
\begin{table}
  \centering
  \begin{footnotesize}
\begin{tabular}{cccccccccc} \toprule
&                 & \multicolumn{8}{c}{ $[[ [GG]aB  ]]$  }  \\
&                 & $\forall b. B'$  & $\int$      & $a$         & $\genB$     & $\unknown$          & $B_1 \to B_2$ & $[[static]]$ & $[[gradual]]$ \\ \cmidrule{3-10}
   \multirow{6}{*}{$\ctxsubst{\Gamma}{A}$} & $\forall a. A'$ & \hl{1 (B poly)}  & \hl{2.Poly} & \hl{2.Poly} & \hl{2.Poly} & \hl{1 (B unknown)}         &  \hl{2.Poly} & \hl{2.Poly}  & \hl{2.Poly}  \\ \cmidrule{3-10}
& $\int$          & \hl{1 (B poly)}  &  \hl{2.Ints}           &  \textit{Impossible} & \hl{2.BEx.Int}            & \hl{1 (B unknown)}  &  \textit{Impossible} &  \textit{Impossible} &  \textit{Impossible} \\ \cmidrule{3-10}
& $a$             &  \hl{1 (B poly)} & \textit{Impossible} &  \hl{2.UVars}           &  \hl{2.BEx.UVar}           &  \hl{1 (B unknown)} &  \textit{Impossible}  &  \textit{Impossible} &  \textit{Impossible} \\ \cmidrule{3-10}
& \multirow{2}{*}{$\genA$}         & \multirow{2}{*}{\hl{1 (B poly)}}   & \multirow{2}{*}{\hl{2.AEx.Int}} & \multirow{2}{*}{\hl{2.AEx.UVar}} & \hl{2.AEx.SameEx}  & \multirow{2}{*}{\hl{1 (B unknown)}}  & \multirow{2}{*}{\hl{2.AEx.Arrow}} & \multirow{2}{*}{\hl{2.AEx.S}} & \multirow{2}{*}{\hl{2.AEx.G}}  \\
& & & & & \hl{2.AEx.OtherEx} &   &   \\ \cmidrule{3-10}
& $\unknown$      & \hl{1 (B poly)}  & \hl{2.Unknown} &  \hl{2.Unknown}  &  \hl{2.Unknown} & \hl{1 (B unknown)}  & \hl{2.Unknown} & \textit{Impossible} & \hl{2.Unknown}  \\ \cmidrule{3-10}
  & $A_1 \to A_2$     & \hl{1 (B poly)}  & \textit{Impossible} &  \textit{Impossible}  & \hl{2.BEx.Arrow}  & \hl{1 (B unknown)}  & \hl{2.Arrows} &  \textit{Impossible} &  \textit{Impossible} \\ \cmidrule{3-10}
  & $[[static]]$     & \hl{1 (B poly)}  & \textit{Impossible} &  \textit{Impossible}  & \hl{2.BEx.S} & \textit{Impossible} & \textit{Impossible} & \hl{2.S} &  \textit{Impossible}  \\ \cmidrule{3-10}
  & $[[gradual]]$     & \hl{1 (B poly)}  & \textit{Impossible} &  \textit{Impossible}  & \hl{2.BEx.G} & \hl{1 (B unknown)} & \textit{Impossible} & \textit{Impossible} & \hl{2.G} \\ \bottomrule
\end{tabular}
  \end{footnotesize}
\caption{List of cases} \label{table:complete}
\end{table}
\end{landscape}

\subcomplete*
\begin{proof}

  By induction on the given declarative derivation. We list all the possible cases in \cref{table:complete}.

We first split on$\ctxsubst{\Gamma}{B}$.
  \begin{itemize}
  \item Case \hl{1 (B poly)}: $[[ [GG]aB  ]]$ is polymorphic: $[[  [GG]aB  ]] = [[ \/b. aB'   ]]$:
    \begin{longtable}[l]{ll|l}
      &$[[aB]] = [[ \/b . aB0]]   $& $[[GG]]$ is predicative \\
      & $[[aB']] = [[ [GG]aB0  ]]$ & $[[GG]]$ is predicative \\
      & $[[ [OO]aB   ]]  =  [[ \/b. [OO]aB0  ]]    $ & By def. of substitution \\
      & $[[  [OO]GG |- [OO]aA <~ [OO]aB   ]]$ & Premise \\
      & $[[  [OO]GG |- [OO]aA <~ \/b . [OO]aB0   ]]$ & By above equality \\
      & $[[  [OO]GG, b |- [OO]aA <~ [OO]aB0   ]]$ & By \Cref{lemma:forall_invert} \\
      & $[[ [OO]GG, b  ]] = [[ [OO, b] (GG, b)     ]]$ & By def. of substitution \\
      & $[[  [OO]aA  ]] = [[ [OO, b]aA  ]]$ & By def. of substitution \\
      & $[[  [OO]aB  ]] = [[ [OO, b]aB  ]]$ & By def. of substitution \\
      & $[[  [OO,b](GG, b) |- [OO,b]aA <~ [OO,b]aB0   ]]$ & By above equalities \\
      & $[[  GG, b |- [GG, b]aA <~ [GG, b]aB0 -| DD'  ]]$ & By i.h. \\
      & $[[  DD' --> OO0'  ]]$ & Above \\
      & $[[ OO, b --> OO0'   ]]$ & Above \\
      & $[[  GG, b |- [GG]aA <~ [GG]aB0 -| DD'  ]]$ & By def. of substitution \\ \\
      & $[[ GG, b --> DD'   ]]$ & By \Cref{lemma:inst_extension} \\
      & $[[ DD'  ]] = [[ DD, b, TT  ]]$ & By \Cref{lemma:extension_order} \\
      & $[[ GG --> DD   ]]$ & Above \\
      & $[[ DD, b, TT --> OO0'    ]]$ & By $[[DD' --> OO0']]$ and above equality \\
      & $[[ OO0'  ]] = [[ OO', b, OR  ]]$ & By \Cref{lemma:extension_order} \\
      $\byhave$& $[[ DD --> OO'  ]]$ & Above \\
      & $[[ OO, b --> OO', b , OR  ]]$ & By above equality \\
      $\byhave$& $[[ OO --> OO'  ]]$ & By \Cref{lemma:extension_order} \\ \\
      & $[[  GG, b |- [GG]aA <~ [GG]aB0 -| DD, b, TT  ]]$ & By above equality \\
      & $[[  GG |- [GG]aA <~ \/b. [GG]aB0 -| DD  ]]$ & By \rref{as-forallR} \\
      $\byhave$& $[[  GG |- [GG]aA <~ \/b. aB' -| DD  ]]$ & By above equality
    \end{longtable}

  \item Case \hl{1 (B unknown)}: $\ctxsubst{\Gamma}{B} = \unknown$:
        \begin{longtable}[l]{ll|l}
          & $[[ [OO]aB]] = [[unknown]]$ \\
          & $[[ [OO]GG |- [OO]aA <~ unknown  ]]$ & Given \\
          & $[[ [OO]aA ]] \in [[gc]] $ & Above \\
          &$[[ GG --> OO  ]]$& Given \\
          &$[[ [GG]aA  ]] \in [[agc]]  $ & Above \\
          & $[[  GG |- [GG]aA <~ unknown -| [ [GG]aA ] GG ]]$ & By \rref{as-unknownRR} \\
          & There exists $[[OO']]$ such that $[[ [ [GG]aA ] GG --> OO'    ]]$  and $[[  OO --> OO'  ]]$
        \end{longtable}

 \item Case 2.*: $[[ [GG]aB   ]]$ is not polymorphic. We split on the form of $[[  [OO]aA  ]]$.
    \begin{itemize}
    \item Case \hl{2.Poly}: $[[ [OO]aA  ]]$ is polymorphic: $[[ [GG]aA  ]] = [[ \/a . aA'  ]]$:
      \begin{longtable}[l]{ll|l}
        & $[[aA]] = [[\/a. aA0]]$& $[[GG]]$ is predicative \\
        & $[[aA']] = [[ [GG]aA0  ]]$ & $[[GG]]$ is predicative \\
        & $[[ [OO]aA  ]] = [[ \/a . [OO]aA0 ]]$ & By def. of substitution \\
        & $[[ [OO]GG |- [OO]aA <~ [OO]aB   ]]$ & Premise \\
        & $[[ [OO]GG |- \/a. [OO]aA0 <~ [OO]aB   ]]$ & By above equality \\
        & $[[ [OO]GG |- ([OO]aA0) [a ~> t] <~ [OO]aB   ]]$ & By inversion on \rref{cs-forallL} \\
        & $[[ [OO]GG |- t  ]]$ & Above \\ \\
        & $[[ [OO]GG  ]] = [[  [OO, evar = at](GG, evar)   ]]$ & By def. of substitution \\
        & $[[ ([OO]aA0) [a ~> t]  ]] = [[  [OO, evar = at](aA0 [ a ~> evar ])  ]]$ & By def. of substitution \\
        & $[[ [OO]aB ]] = [[ [OO, evar = at]aB   ]]$ & By def. of substitution \\
        & $[[ [OO, evar = at](GG, evar) |- [OO, evar = at](aA0 [ a ~> evar ]) <~ [OO, evar = at]aB   ]] $ & By above equalities \\
        & $[[  GG, evar |- [GG, evar] (aA0 [ a ~> evar ]) <~ [GG, evar]aB -| DD   ]]$ & By i.h. \\
        $\byhave$& $[[ DD --> OO'   ]]$ & Above \\
        & $[[  OO, evar = at --> OO'   ]]$ & Above \\
        $\byhave$& $[[ OO --> OO'  ]]$ & By \cref{lemma:drop_ext} \\
        & $[[ [GG, evar] (aA0 [ a ~> evar ])  ]] = [[  ([GG]aA0) [a ~> evar]   ]]$ & By def. of substitution \\
        & $[[ [GG, evar]aB   ]] = [[ [GG]aB   ]]$ & By def. of substitution \\
        & $[[  GG, evar |- ([GG]aA0) [a ~> evar] <~ [GG]aB -| DD   ]]$ & By above equality \\
        & $[[  GG |- \/a. ([GG]aA0) <~ [GG]aB -| DD   ]]$ & By \rref{as-forallLL} \\
        $\byhave$& $[[  GG |- \/a. aA' <~ [GG]aB -| DD   ]]$ & By above equality \\
      \end{longtable}

    \item Case \hl{2.Unknown}: $[[ [GG]aA  ]] = [[unknown]]$:
        \begin{longtable}[l]{ll|l}
          & $[[ [OO] aA]] = [[unknown]]$ & Obviously, what else? \\
          & $[[ [OO]GG |- unknown <~ [OO]aB  ]]$ & Given \\
          & $[[ [OO]aB ]] \in [[gc]] $ & Above \\
          &$[[ GG --> OO  ]]$& Given \\
          &$[[ [GG]aB  ]] \in [[agc]]  $ & Above \\
          & $[[  GG |- unknown <~ [GG]aB -| [ [GG]aB ] GG ]]$ & By \rref{as-unknownLL} \\
          & There exists $[[OO']]$ such that $[[ [ [GG]aB ] GG --> OO'    ]]$ and $[[  OO --> OO'  ]]$
        \end{longtable}

    \item Case \hl{2.AEx.*}: $[[ [GG]aA ]]$ is an existential variable: $[[ [GG]aA  ]] = [[evar]]$. We split on the form of $[[ [GG]aB  ]]$.
      \begin{itemize}
      \item Case \hl{2.AEx.SameEx}. $[[ [GG]aB ]]$ is the same existential variable $[[ [GG]aB  ]] = [[evar]]$:
        \begin{longtable}[l]{ll|l}
          &$[[  GG |- evar <~ evar -| GG   ]]$& By \rref{as-evar} \\
          $\byhave$& $[[  GG |- [GG]aA <~ [GG]aB -| GG   ]]$ & By above equality \\
          $\byhave$& $[[ DD --> OO   ]]$ & $[[DD]] = [[GG]]$ \\
          $\byhave$& $[[ OO --> OO'  ]]$ & By \Cref{lemma:reflexivity} and $[[OO']] = [[OO]]$
        \end{longtable}
      \item Case \hl{2.AEx.OtherEx}. $[[ [GG]aB  ]]$ is a different existential variable $[[ [GG]aB ]] = [[evarb]]$ where $[[evarb]] \neq [[evar]]$:
        \begin{longtable}[l]{ll|l}
          &$[[ [OO]aA  ]] = [[ [OO]([GG]aA)   ]] = [[ [OO]evar  ]]$& By \Cref{lemma:subst_ext_invar} \\
          &$[[ [OO]aB  ]] = [[ [OO]([GG]aB)   ]] = [[ [OO]evarb  ]]$& By \Cref{lemma:subst_ext_invar} \\
          & $ [[ [OO]GG |- [OO]aA <~ [OO]aB  ]]   $ & Given \\
          & $[[ [OO]GG |- [OO]evar <~ [OO]evarb  ]] $ & By above equalities \\
          & $[[ GG |- evar <~~ evarb -| DD   ]]$ & By \Cref{thm:inst_complete} \\
          $\byhave$& $[[ DD --> OO'  ]]$ & Above \\
          $\byhave$& $[[ OO --> OO'   ]]$ & Above \\
          & $[[ GG |- evar <~ evarb -| DD     ]]$ & By \rref{as-instL} \\
          $\byhave$& $[[ GG |- [GG]aA <~ [GG]aB -| DD     ]]$ & By above equalities
        \end{longtable}
      \item Case \hl{2.AEx.Int}. We have $[[ [GG]aB  ]] = [[int]]$:
        \begin{longtable}[l]{ll|l}
          &$[[ GG --> OO  ]]$& Given \\
          & $ [[ [OO] aB ]] = [[int]] = [[ [OO]int  ]]$ & By def. of substitution \\
          & $[[ [OO]aA  ]] = [[ [OO]([GG]aA)   ]] = [[ [OO]evar  ]]$ & By \Cref{lemma:subst_ext_invar} \\
          & $  [[ [OO]GG |- [OO]aA <~ [OO]aB    ]]  $ & Given \\
          & $[[  [OO]GG |- [OO]evar <~ [OO]int    ]]$ & By above equalities \\
          & $[[  GG |- evar <~~ int -| DD    ]]$ & By \Cref{thm:inst_complete} \\
          $\byhave$& $[[ DD --> OO'  ]]$ & Above \\
          $\byhave$& $[[ OO --> OO'  ]]$ & Above \\
          & $[[  GG |- evar <~ int -| DD    ]]$ & By \rref{as-instL} \\
          $\byhave$& $[[ GG |- [GG]aA <~ [GG]aB -| DD     ]]$ & By above equalities
        \end{longtable}
      \item Case \hl{2.AEx.UVar}. We have $[[ [GG]aB  ]] = [[b]]$. Similar to Case \hl{2.AEx.Int}.
      \item Case \hl{2.AEx.Arrow}. $[[ [GG]aB  ]] = [[ aB1 -> aB2   ]]$. We prove
        $[[evar]] \notin \textsc{fv}([[ [GG]aB  ]])$. Suppose for a
        contradiction, that $[[evar]] \in \textsc{fv}([[ [GG]aB ]])$, then
        $[[evar]]$ must be a subterm of $[[ [GG]aB  ]]$, so is
        $[[  [OO]evar  ]]$ a subterm of
        $[[ [OO]([GG]aB)     ]]$. The latter is equal to
        $[[ [OO]aB   ]]$, so $[[  [OO]evar  ]]$ is a subterm of
        $[[ [OO]aB   ]]$. Since $[[ [GG]aB  ]] = [[ aB1 -> aB2   ]]$, then
        $[[ [OO]aB   ]]$ must have the form $[[ aB1' -> aB2'  ]]$. Therefore
        $[[  [OO]evar   ]]$ must occur in either $[[aB1']]$ or $[[aB2']]$. But we
        have $[[ [OO]GG |- [OO]evar <~ [OO]aB       ]]$. That is , $[[  [OO]evar   ]]$
        cannot be a subterm of $[[ [OO]aB   ]]$. This is a contradiction.
        \begin{longtable}[l]{ll|l}
          &$[[evar]] \notin \textsc{fv}([[ [GG]aB  ]])$ & Proved above \\
          & $[[  GG --> OO  ]]$ & Given \\
          & $[[ [OO]aB   ]] = [[ [OO]([GG]aB)   ]]$ & By \Cref{lemma:subst_ext_invar} \\
          & $[[  [OO]GG |- [OO]evar <~ [OO]aB          ]]$ & Given \\
          & $[[  [OO]GG |- [OO]evar <~ [OO]([GG]aB)         ]]$ & By above equality \\
          & $[[ GG |- evar <~~ [GG]aB -| DD  ]]$ & By \Cref{thm:inst_complete} \\
          $\byhave$& $[[  DD --> OO' ]]$ & Above \\
          $\byhave$& $[[ OO --> OO' ]]$ & Above \\
          &$[[  GG |- evar <~ [GG]aB -| DD  ]]$ & By \rref{cs-instL} \\
          $\byhave$& $[[ GG |- [GG]aA <~ [GG]aB -| DD   ]]$ & By above equalities
        \end{longtable}
      \item Case \hl{2.AEx.S} and \hl{2.AEx.S}. Similar to Case \hl{2.AEx.Int}.
      \end{itemize}

    \item Case \hl{2.BEx.*}. $[[ [GG]aA  ]]$ is not polymorphic and
      $[[ [GG]aB   ]]$ is an existential variable: $[[ [GG]aB  ]] = [[evarb]]$. We split on the form of $[[ [GG]aA  ]]$.
      \begin{itemize}
      \item Case \hl{2.BEx.Int}. Similar to Case \hl{2.AEx.Unit}.
      \item Case \hl{2.BEx.UVar}. Similar to Case \hl{2.AEx.UVar}.
      \item Case \hl{2.BEx.Arrow}. Similar to Case \hl{2.AEx.Arrow}.
      \item Case \hl{2.BEx.S}. Similar to Case \hl{2.AEx.S}.
      \item Case \hl{2.BEx.G}. Similar to Case \hl{2.AEx.G}.
      \end{itemize}
      We use the second part of \Cref{thm:inst_complete} and apply \rref{as-instR}.

    \item Case \hl{2.Ints}. $[[ [GG]aA  ]] = [[ [GG]aB  ]] = [[int]] $:
      \begin{longtable}[l]{ll|l}
        $\byhave$&$[[  GG |- int <~ int -| GG ]]$& By \rref{as-int} \\
                 & $[[ GG --> OO   ]]$ & Given \\
        $\byhave$& $[[ DD --> OO'   ]]$ & $[[DD]] = [[GG]]$ \\
        $\byhave$& $[[ OO --> OO'  ]]$ & By \Cref{lemma:reflexivity} and $[[OO']] = [[OO]]$
      \end{longtable}

    \item Case \hl{2.UVars}. $[[ [GG]aA   ]] = [[ [GG]aB  ]] = [[a]]$:
      \begin{longtable}[l]{ll|l}
      $\byhave$&$[[ GG |-  a <~ a -| GG ]]$& By \rref{as-tvar} \\
                 & $[[ GG --> OO   ]]$ & Given \\
        $\byhave$& $[[ DD --> OO'   ]]$ & $[[DD]] = [[GG]]$ \\
        $\byhave$& $[[ OO --> OO'  ]]$ & By \Cref{lemma:reflexivity} and $[[OO']] = [[OO]]$
      \end{longtable}

    \item Case \hl{2.Arrows}. Let $[[ [GG]aA  ]] = [[ aA1 -> aA2  ]]$ and $[[ [GG]aB  ]] = [[ aB1 -> aB2   ]]$:
      \begin{longtable}[l]{ll|l}
        & $[[  GG --> OO  ]]$ & Given \\
        &$[[ [OO]aA  ]] = [[ [OO]([GG]aA)   ]] = [[ [OO]aA1 -> [OO]aA2  ]]$ & By \Cref{lemma:subst_ext_invar} \\
        &$[[ [OO]aB ]] = [[ [OO]([GG]aB)   ]] = [[ [OO]aB1 -> [OO]aB2  ]]$ & By \Cref{lemma:subst_ext_invar} \\
        & $[[ [OO]GG |- [OO]aA <~ [OO]aB   ]]$ & Given \\
        & $[[ [OO]GG |- [OO]aB1 <~ [OO]aA1   ]]$ & Premise \\
        & $[[ GG |- [GG]aB1 <~ [GG]aA1 -| TT  ]]$ & By i.h. \\
        & $[[ TT --> OO0  ]]$ & Above \\
        & $[[ OO --> OO0  ]]$ & Above \\
        & $[[ GG --> OO0  ]]$ & By \Cref{lemma:transitivity} \\ \\
        & $[[ [OO]GG   ]] = [[  [OO0]TT  ]]$ & By \Cref{lemma:confluence} \\
        & $[[ [OO]aA2   ]] = [[ [OO0]([GG]aA2)   ]]$ & By \Cref{lemma:subst_ext_invar} \\
        & $[[ [OO]aB2   ]] = [[ [OO0]([GG]aB2)   ]]$ & By \Cref{lemma:subst_ext_invar} \\
        & $[[ [OO]GG |- [OO]aA2 <~ [OO]aB2    ]]$ & Premise \\
        & $[[ [OO0]TT |- [OO0]([GG]aA2) <~ [OO0]([GG]aB2)    ]] $ & By above equalities \\
        & $[[ TT |- [TT]([GG]aA2) <~ [TT]([GG]aB2) -| DD   ]]$ & By i.h. \\
        $\byhave$& $[[ DD --> OO'   ]]$ & Above \\
        & $[[ OO0 --> OO'    ]]$ & Above \\ \\
        $\byhave$& $[[  GG |- [GG](aA1 -> aA2) <~ [GG](aB1 -> aB2) -| DD    ]]$ & By \rref{as-arrow} \\
        $\byhave$& $[[ OO --> OO'  ]]$ & By \Cref{lemma:transitivity}
      \end{longtable}

    \item Case \hl{2.S}: $[[ [GG]aA  ]] = [[ [GG]aB ]] = [[static]]$.
      \begin{longtable}[l]{ll|l}
        $\byhave$&$[[  GG |- static <~ static -| GG ]]$& By \rref{as-spar} \\
                 & $[[ GG --> OO   ]]$ & Given \\
        $\byhave$& $[[ DD --> OO'   ]]$ & $[[DD]] = [[GG]]$ \\
        $\byhave$& $[[ OO --> OO'  ]]$ & By \Cref{lemma:reflexivity} and $[[OO']] = [[OO]]$
      \end{longtable}
    \item Case \hl{2.G}. Similar to Case \hl{2.S}.

    \end{itemize}
  \end{itemize}
\end{proof}


\newpage

\section{Completeness of Typing}


\matchcomplete*
\begin{proof}
 By induction on the given derivation. We split on $[[ [GG]aA ]]   $.

 \begin{itemize}
 \item $[[  [GG]aA  ]] = [[ \/a. aA'   ]]$:
    \begin{longtable}[l]{ll|l}
      &$[[aA]] = [[ \/a . aA0]]   $& $[[GG]]$ is predicative \\
      & $[[aA']] = [[ [GG]aA0  ]]$ & $[[GG]]$ is predicative \\
      & $[[ [OO]aA   ]]  =  [[ \/a. [OO]aA0  ]]    $ & By def. of substitution \\
      & $[[  [OO]GG |- [OO]aA |> A1 -> A2  ]]$ & Given \\
      & $[[  [OO]GG |- \/a. [OO]aA0 |> A1 -> A2  ]]$ & By above equality \\
      & $[[  [OO]GG |- ([OO]aA0) [a ~> t]  |> A1 -> A2  ]]$ & By inversion \\
      & $[[  [OO]GG |- t   ]]$ & Above \\
      & $[[ GG --> OO   ]]$ & Given \\
      & $[[ GG, sa --> OO, sa   ]]$ & By def. of context extension \\ \\

      & $[[ [OO]GG  ]] = [[ [OO, sa = at](GG, sa)    ]]$ & By def. of context application \\
      & $[[ ([OO]aA0) [a ~> t]   ]] =  [[  [OO, sa = at]( aA0 [a ~> sa] )   ]]     $ & By def. of substitution \\
      & $[[  [OO, sa = at](GG, sa) |- [OO, sa = at]( aA0 [a ~> sa] )  |> A1 -> A2  ]]$ & By above equalities \\
      & $[[  GG, sa |- [GG, sa](aA0 [a ~> sa]) |> aA1' -> aA2' -| DD   ]]$ & By i.h. \\
      $\byhave$& $[[ DD --> OO'   ]]$ and $[[  OO, sa = at --> OO'    ]]$ & Above \\
      $\byhave$ & $[[A1]] = [[ [OO']aA1'  ]]$ and $[[A2 ]] = [[  [OO']aA2'  ]]$ & Above \\
      & $[[   [GG, sa](aA0 [a ~> sa])   ]] =  [[  ([GG]aA0) [a ~> sa]   ]]   $ & By def. of substitution \\
      & $[[   GG, sa |-   ([GG]aA0) [a ~> sa]   |> aA1' -> aA2' -| DD   ]]$ & By above equality \\
      & $[[ GG |- \/a . [GG]aA0 |> aA1' -> aA2' -| DD   ]]$ & By \rref{am-forallL} \\
      & $[[ [GG]aA ]] = [[ \/a. aA'  ]] = [[ \/a. [GG]aA0   ]]    $ & By above equalities \\
      $\byhave$ & $[[ GG |- [GG]aA |> aA1' -> aA2' -| DD  ]]$ & Above
    \end{longtable}



  \item   $[[  [GG]aA  ]] = [[ aA1' -> aA2'   ]]$:

    \begin{longtable}[l]{ll|l}
      & $[[ [OO]aA  ]] = [[ [OO]([GG]aA)  ]] = [[  [OO]aA1' -> [OO]aA2'  ]]$ & By \Cref{lemma:subst_ext_invar}   \\
      & $[[ [OO]GG |-  [OO]aA1' -> [OO]aA2' |> A1 -> A2  ]]$ & Given \\
      & Let $[[ DD  ]] = [[GG]]$ and $[[ OO' ]] = [[OO]]$ \\
      $\byhave$& $[[ [OO]aA1'   ]] = [[A1]]$ and $[[ [OO]aA2' ]] = [[A2]]$ \\
      $\byhave$ & $[[  GG |- aA1' -> aA2' |> aA1' -> aA2' -| GG  ]]$ & By \rref{am-arr} \\
      $\byhave$ & $[[  DD --> OO' ]]$ & Given $[[  GG --> OO  ]]$ \\
      $\byhave$ & $[[  OO --> OO' ]]$ & By \cref{lemma:reflexivity}

    \end{longtable}

  \item   $[[  [GG]aA  ]] = [[ unknown   ]]$:

    \begin{longtable}[l]{ll|l}
      & $[[ [OO]aA  ]] = [[ [OO]([GG]aA)  ]] = [[  unknown  ]]$ & By \Cref{lemma:subst_ext_invar}   \\
      & $[[ [OO]GG |-  unknown |> A1 -> A2  ]]$ & Given \\
      & Let $[[ DD  ]] = [[GG]]$ and $[[ OO' ]] = [[OO]]$ \\
      $\byhave$& $[[ A1   ]] = [[unknown]]$ and $[[ A2 ]] = [[unknown]]$ \\
      $\byhave$ & $[[  GG |- unknown |> unknown -> unknown -| GG  ]]$ & By \rref{am-unknown} \\
      $\byhave$ & $[[  DD --> OO' ]]$ & Given $[[  GG --> OO  ]]$ \\
      $\byhave$ & $[[  OO --> OO' ]]$ & By \cref{lemma:reflexivity}

    \end{longtable}

  \item   $[[  [GG]aA  ]] = [[ evar  ]]$:

    \begin{longtable}[l]{ll|l}
      & $[[  GG  ]] = [[ GG0[evar] ]]$ & Since $[[evar]] \in \textsc{unsolved}([[GG]]) $ \\
      & $[[ [OO]aA  ]] = [[ [OO]([GG]aA)  ]] = [[  [OO]evar  ]]$ & By \Cref{lemma:subst_ext_invar}   \\
      & $[[ [OO]GG |- [OO]evar |> A1 -> A2  ]]$ & Given \\
      & $[[ [OO]evar ]] = [[ t1 -> t2  ]]$ and $[[ A1 ]] = [[t1]]$ and $[[ A2 ]] = [[t2]]$ & $[[OO]]$ is predicate \\
      & $[[ OO ]] = [[  OO0[ evar = at' ]  ]]$ and $[[  [OO]at'   ]] = [[  t1 -> t2 ]]$ & Above \\
      & Let $[[DD]] = [[ GG0[evar1, evar2, evar = evar1 -> evar2]  ]]$ \\
      & Let $[[ OO'  ]] = [[ OO0[evar1 = at1, evar2 = at2, evar = evar1 -> evar2] ]] $ \\
      $\byhave$& $[[ DD --> OO'  ]]$ & By \Cref{lemma:paralell_admit} twice \\
      $\byhave$& $[[ OO --> OO'    ]]$ & By \Cref{lemma:paralell_ext_solu} and \Cref{lemma:paralell_admit} \\
      $\byhave$& $[[ GG0[evar] |- evar |> evar1 -> evar2 -| DD  ]]$ & By \rref{am-var} \\
      $\byhave$& $[[A1]] = [[t1]] = [[ [OO']evar1  ]]$ and $[[A2]] = [[t2]] = [[ [OO']evar2  ]]$ & Above
    \end{longtable}

 \end{itemize}


\end{proof}

\typingcomplete*
\begin{proof}
  By induction on the given derivation.
  \begin{itemize}
  \item Case \[     \ottaltinferrule{var}{}{ [[  (x : A) in [OO]GG ]] }{ [[ [OO]GG |- x : A]]  }  \]
    \begin{longtable}[l]{ll|l}
      & $ [[   (x : A) in [OO]GG  ]]  $  & Premise \\
      & $[[ GG --> OO  ]]$ & Given \\
      & $  [[ (x : aA') in GG  ]]  $ where $[[ [OO]aA'  ]] = [[  [OO]aA ]]$ & From def. of context application \\
      & Let $[[DD]] = [[GG]]$ and $[[OO']] = [[OO]]$. \\
      $\byhave$& $[[ GG --> OO ]]$ & Given \\
      $\byhave$& $[[ OO --> OO ]]$ & By \Cref{lemma:reflexivity} \\
      $\byhave$& $[[  GG |- x => aA' -| GG ]]$ & By \rref{inf-var} \\
      $\byhave$& $[[ [OO]aA' ]] = [[ [OO]aA ]] = [[A]]$ & A is well-formed in $[[  [OO]GG  ]]$ \\
      $\byhave$& $\erase{x} = \erase{x}$ & By def. of erasure
    \end{longtable}

  \item Case \[     \ottaltinferrule{int}{}{ }{ [[ [OO]GG |- n : int ]] }  \]

    \begin{longtable}[l]{ll|l}
      &Let $[[aA']] = [[int]]$ and $[[DD]] = [[GG]]$ and $[[OO']] = [[OO]]$. & \\
      $\byhave$& $[[ GG --> OO ]]$ & Given \\
      $\byhave$& $[[ OO --> OO ]]$ & By \Cref{lemma:reflexivity} \\
      $\byhave$& $[[ GG |- n => int -| GG ]]$ & By \rref{inf-int} \\
      $\byhave$& $[[ [OO]int ]] = [[int]]$ \\
      $\byhave$& $\erase{n} = \erase{n}$ & By def. of erasure
    \end{longtable}


  \item Case \[     \ottaltinferrule{lamann}{}{ [[ [OO]GG, x : A |- e : B  ]] }{ [[  [OO]GG |- \x : A . e : A -> B  ]] }  \]

    \begin{longtable}[l]{ll|l}
      & Let $[[OO0]] = [[ OO, x : aA]]$. \\
      & $[[ [OO0](GG, x : aA) ]] = [[ [OO]GG, x : aA  ]]$ & From def. of context application \\
      & $[[ [OO0](GG, x : aA) |- e : B   ]]$ & By above equality and premise \\
      & $[[ GG, x : aA |- ae' => aB0 -| DD0   ]]$ & By i.h. \\
      & $[[ DD0 --> OO' ]]$ & Above \\
      & $[[ OO0 --> OO' ]]$ & Above \\
      & $[[B]] = [[ [OO']aB0 ]]$ & Above \\
      & $\erase{e} = \erase{e'}$ & Above \\
      & $[[ GG, x : aA --> DD0  ]]$ & By \cref{lemma:typing_extension} \\
      & $[[ DD0 ]] = [[  DD1, x : aA' , DD2 ]]$ & By \cref{lemma:extension_order} \\
      & $[[ [DD1]aA  ]] = [[  [DD1] aA'  ]]$ & Above \\
      & $[[ GG --> DD1  ]] $ & Above \\
      & $[[ aA  ]] = [[  [DD1] aA'  ]]$ & $[[aA]]$ has no evar \\
      & $[[ GG, x : aA |- ae' => aB0 -| DD1, x : aA, DD2 ]]$ & By above equalities \\
      & $[[ GG, x : aA |- ae' <= aB0 -| DD1, x : aA, DD2 ]]$ & By \rref{chk-sub} \\ \\
      & $[[ DD1, x : aA', DD2 --> OO' ]]$ & By above equalities \\
      & $[[ OO' ]] = [[ OO1, x : aA'' , OO2 ]]$ & By \cref{lemma:extension_order} \\
      & $[[ [OO1]aA'  ]] = [[  [OO1] aA''  ]]$ & Above \\
      $\byhave$ & $[[ DD1 --> OO1  ]] $ & Above \\
      & $[[  OO, x : aA --> OO1, x : aA'', OO2 ]]$ & By above equalities \\
      $\byhave$& $[[  OO --> OO1  ]]$ & By \cref{lemma:extension_order} \\ \\

      & $[[  GG |- \x . ae' <= aA -> aB0 -| DD1  ]]$ & \rref{chk-lam} \\
      $\byhave$& $[[  GG |- (\x . ae') : aA -> aB0 => aA -> aB0 -| DD1  ]]$ & \rref{inf-anno} \\
      $\byhave$& $[[  [OO1](aA -> aB0)  ]] = [[ A -> [OO']aB0  ]] = [[ A -> B ]]$ & From above equality \\
      $\byhave$& $\erase{[[ \x : A . e  ]]} = \erlam{x}{\erase{e}} = \erlam{x}{\erase{e'}} = \erase{[[(\x . ae') : aA -> aB0]]}$ & By def. of erasure
    \end{longtable}

  \item Case \[     \ottaltinferrule{app}{}{ [[ [OO]GG |- e1 : A ]] \\ [[ [OO]GG |- A |> A1 -> A2]] \\ [[  [OO]GG |- e2 : A3  ]] \\ [[  [OO]GG |- A3 <~ A1  ]]  }{ [[  [OO]GG |- e1 e2 : A2  ]]  }  \]

    \begin{longtable}[l]{ll|l}
      & $[[  [OO]GG |- e1 : A  ]]$ & Premise \\
      & $[[ GG --> OO  ]]$ & Given \\
      & $[[ GG |- ae1' => aA' -| TT1  ]]$ & By i.h. \\
      & $[[TT1 --> OO0' ]]$ & Above \\
      & $[[ OO --> OO0'  ]]$ & Above \\
      & $[[A]] = [[ [OO0']aA'  ]]$ & Above \\
      & $\erase{[[e1]]} = \erase{[[ae1']]}$ & Above \\ \\

      & $[[ [OO]GG |- A |> A1 -> A2  ]]$ & Premise \\
      & $[[  [OO]GG  ]] = [[  [OO]OO  ]]$ & By \Cref{lemma:stable_complete_ctxt} \\
      & $ = [[ [OO0']OO0' ]]$ & By \Cref{lemma:finish_complete} \\
      & $ = [[ [OO0']GG   ]]$ & By \Cref{lemma:stable_complete_ctxt} \\
      & $ = [[ [OO0']TT1 ]]$ & By \Cref{lemma:confluence} \\
      & $[[ [OO0']TT1 |- [OO0']aA' |> A1 -> A2    ]]$ & By above equalities \\
      & $[[ TT1 |- [TT1]aA' |> aA1' -> aA2' -| TT2    ]]$ & By \Cref{thm:match_complete} \\
      & $[[ TT2 --> OO' ]]$ & Above \\
      & $[[ OO0' --> OO' ]]$ & Above \\
      & $[[A1]] = [[ [OO']aA1'  ]]$ & Above \\
      & $[[A2]] = [[ [OO']aA2'  ]]$ & Above \\ \\

      & $[[  [OO]GG |- e2 : A3   ]]$ & Premise \\
      & $ [[ [OO]GG  ]]   = [[  [OO]OO  ]]$ & By \Cref{lemma:stable_complete_ctxt} \\
      & $ = [[ [OO']OO'  ]]$ & By \Cref{lemma:finish_complete} \\
      & $ = [[ [OO']GG   ]]$ & By \Cref{lemma:stable_complete_ctxt} \\
      & $ = [[ [OO']TT2   ]]$ & By \Cref{lemma:confluence} \\
      & $[[ [OO']TT2 |- e2 : A3   ]]$ & By above equality \\
      & $[[  TT2 |- ae2' => aA3' -| TT3   ]]$ & By i.h. \\
      & $[[  TT3 --> OO1'   ]]$ & Above \\
      & $[[  OO' --> OO1'  ]]$ & Above \\
      & $[[A3]] = [[  [OO1']aA3'  ]]$ & Above \\
      & $\erase{[[e2]]} = \erase{[[ae2']]}$ & Above \\ \\

      & $[[ [OO]GG |- A3 <~ A1   ]]$ & Premise \\
      & $ [[ [OO]GG  ]]   = [[  [OO]OO  ]]$ & By \Cref{lemma:stable_complete_ctxt} \\
      & $ = [[ [OO1']OO1'  ]]$ & By \Cref{lemma:finish_complete} \\
      & $ = [[ [OO1']GG   ]]$ & By \Cref{lemma:stable_complete_ctxt} \\
      & $ = [[ [OO1']TT3   ]]$ & By \Cref{lemma:confluence} \\
      & $[[A3]] = [[  [OO1']aA3'  ]]$ & Above \\
      & $[[A1]] = [[ [OO']aA1'  ]] = [[ [OO1']aA1'  ]] $ & By \Cref{lemma:finish_types} \\
      & $[[ [OO1']TT3 |- [OO1']aA3' <~ [OO1']aA1'     ]]$ & By above equalities \\
      & $[[  TT3 |- [TT3]aA3' <~ [TT3]aA1' -| DD   ]]$ & By \Cref{thm:sub_completeness} \\
      & $[[ [TT3]aA1' ]] = [[ [TT3]([TT2]aA1)  ]]$ & By \cref{lemma:subst_ext_invar}\\
      & $[[  TT2 |- ae2' <= [TT2]aA1' -| DD  ]]$ & By \rref{chk-sub} \\
      $\byhave$& $[[  DD --> OO2'   ]]$ & Above \\
      & $[[ OO1' --> OO2'   ]]$ & Above \\
      $\byhave$& $[[  GG |- ae1' ae2' => aA2' -| DD  ]]$ & By \rref{inf-app} \\
      $\byhave$& $[[A2]] = [[ [OO']aA2' ]] = [[ [OO2']aA2'  ]]$ & \Cref{lemma:finish_types} \\
      $\byhave$& $[[  GG --> OO2' ]]$ & By \Cref{lemma:transitivity} \\
      $\byhave$& $\erase{[[e1 e2]]} = \erase{e_1} ~ \erase{e_2} = \erase{e_1'} ~ \erase{e_2'} = \erase{[[ ae1' ae2'  ]]}$ & By def. of erasure
    \end{longtable}


  \item Case \[     \ottaltinferrule{lam}{}{ [[ [OO]GG, x : t |- e : B  ]]  }{ [[  [OO]GG |- \x . e : t -> B  ]]   }  \]

    \begin{longtable}[l]{ll|l}
      & $[[ [OO]GG , x : t |- e : B   ]]$ & Given \\
      & $[[ [OO]GG, x : t  ]] = [[ [OO, x : at](GG , x : at)    ]]$ & By def. of context substitution \\
      & $[[ [OO, x : at](GG, x : at) |- e : B   ]]$ & By above equality \\
      & $[[ GG , x : at |- ae' => aB' -| DD'   ]]$ & By i.h., \\
      & $[[DD' --> OO']]$ & Above \\
      & $[[  OO, x : at --> OO'   ]]$ & Above \\
      & $[[B]] = [[ [OO']aB'  ]]$ & Above \\
      & $\erase{e} = \erase{e'}$ & Above \\
      & $[[ GG, x : at --> DD'   ]]$ & By \cref{lemma:typing_extension} \\
      & $[[DD']] = [[ DD, x : at, TT   ]]$ & By \cref{lemma:extension_order} \\
      & $[[ GG , x : at |- ae' => aB' -| DD, x : at, TT    ]]$ & By above equality \\
      $\byhave$& $[[  GG |- \x : at . ae' => at -> aB' -| DD   ]]$ & By \rref{inf-lamann} \\
      $\byhave$& $[[ DD --> OO'  ]]$ & By context extension \\
      $\byhave$& $[[ OO --> OO'   ]]$ & By context extension \\
      $\byhave$& $[[ t -> B  ]] = [[ at -> [OO']aB'   ]] = [[ [OO'](at -> aB')   ]]  $ & By def. of substitution \\
      $\byhave$& $\erase{[[\x . e]]} = \erlam{x}{\erase{e}} = \erlam{x}{\erase{e'}} = \erase{ [[\x : at . ae']] }$ & By def. of erasure
    \end{longtable}



  \item Case \[     \ottaltinferrule{gen}{}{ [[ [OO]GG, a |- e : A  ]]  }{ [[  [OO]GG |- e : \/a . A  ]]   }  \]

    \begin{longtable}[l]{ll|l}
      & $[[ [OO]GG, a |- e : A  ]]$ & Given \\
      & $[[ [OO]GG, a  ]] = [[  [OO, a](GG, a)   ]]$ & By def. of context substitution \\
      & $[[ [OO, a](GG, a) |- e : A    ]]$ & By above equality \\
      & $[[  GG, a |- ae' => aA' -| DD'  ]]$ & By i.h., \\
      & $[[ DD' --> OO'    ]]$ & Above \\
      & $[[ OO, a --> OO'   ]]$ & Above \\
      & $[[A]] = [[  [OO']aA'  ]]$ & Above \\
      $\byhave$& $\erase{e} = \erase{e'}$ & Above \\
      & $[[  GG, a --> DD'   ]]$ & By \cref{lemma:typing_extension} \\
      & $[[DD']] = [[DD, a, TT]]$ & By \cref{lemma:extension_order} \\
      $\byhave$& $[[  DD --> OO'   ]]$ & By context extension \\
      $\byhave$& $[[  OO --> OO'   ]]$ & By context extension \\
      & $[[  GG, a |- ae' => aA' -| DD, a , TT  ]]$ & By above equality \\
      & $ [[ DD, a, TT |- [DD, a, TT]aA' <~ [DD, a, TT]aA' -| DD, a, TT   ]]   $ & By reflexivity of consistent subtyping \\
      & $[[  GG, a |- ae' <= aA' -| DD, a, TT   ]]$ & By \rref{chk-sub} \\
      & $[[  GG |- ae' <= \/ a. aA' -| DD    ]]$ & By \rref{chk-gen} \\
      $\byhave$& $  [[  GG |- ae' : \/ a. aA' => \/ a. aA' -| DD    ]]  $ & By \rref{inf-anno} \\
      $\byhave$& $[[\/a. A]] = \forall a . \ctxsubst{\Omega'}{A'} = \ctxsubst{\Omega'}{([[\/a. aA']])}$ & By def. of substitution \\
    \end{longtable}



  \end{itemize}



\end{proof}



%%% Local Variables:
%%% mode: latex
%%% TeX-master: "../paper"
%%% org-ref-default-bibliography: ../paper.bib
%%% End:


%% -- The end --

\end{document}

%%% Local Variables:
%%% mode: latex
%%% TeX-master: t
%%% End:
