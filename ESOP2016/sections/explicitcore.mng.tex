
\section{Dependent Types with Casts and General Recursion}\label{sec:ecore}

In this section, we present the \name calculus. This calculus is very
close to the calculus of constructions, except for three key differences:
1) the absence of the $\Box$ constant (due to use of the
``type-in-type'' axiom); 2) the existence of two \cast operators; 3)
general recursion on both term level and type level.
Unlike \cc the proof of 
decidability of type checking for \name does not require the strong normalization of the
calculus. Thus, the addition of general recursion does not break decidable
type checking. In the rest of this section, we demonstrate the syntax,
operational semantics, typing rules and metatheory of \name.
Full proofs of metatheory can be found in the full version of this paper\footnote{\url{https://github.com/bixuanzju/full-version}}.

\subsection{Syntax}\label{sec:ecore:syn}

Figure \ref{fig:ecore:syntax} shows the syntax of \name, including
expressions, contexts and values. \name uses a unified 
representation for different syntactic levels by following the
\emph{pure type system} (PTS) representation of \cc~\cite{handbook}. Therefore, there
is no syntactic distinction between terms, types or kinds. This design
brings economy for type checking, since one set of rules can cover
all syntactic levels. By convention, we use metavariables $[[t]]$ and
$[[T]]$ for an expression on the type-level position and $e$ for one
on the term level.

\paragraph{Type of Types}
In \cc, there are two distinct sorts $[[star]]$ and
$[[square]]$ representing the type of \emph{types} and \emph{sorts}
respectively, and an axiom $[[star]]:[[square]]$ specifying the
relation between the two sorts~\cite{handbook}. In \name, we further merge types and
kinds together by including only a single sort $[[star]]$ and an
impredicative axiom $[[star]]:[[star]]$.

\paragraph{Explicit Type Conversion}

We introduce two new primitives $[[castup]]$ and $[[castdown]]$
(pronounced as ``cast up'' and ``cast down'') to replace the implicit
conversion rule of \cc with \emph{one-step} explicit type
conversions. The type-conversions perform two directions of conversion:
$[[castdown]]$ is for the \emph{beta reduction} of types, and
$[[castup]]$ is for the \emph{beta expansion}. The $[[castup]]$
construct takes a
type parameter $[[t]]$ as the result type of one-step beta expansion
for disambiguation (see also Section \ref{subsec:cast}). The $[[castdown]]$ construct
does not need a type parameter,  because the result type of one-step beta reduction
is uniquely determined, as we shall see in Section \ref{sec:ecore:meta}.

\paragraph{General Recursion}
General recursion allows a large number of programs that can be expressed in programming languages such 
as Haskell to be expressed in \name as well.
We add one primitive $[[mu]]$ to represent general recursion.
It has a uniform representation on both term level and type level: the
same construct works both as a term-level fixpoint and a recursive type. The recursive
expression $[[mu x:t.e]]$ is \emph{polymorphic}, in the sense that $[[t]]$ 
is not restricted to $[[star]]$ but can be any type, 
such as a function type $[[int -> int]]$ or a kind $[[star -> star]]$.

\begin{figure}[t]
\centering
\footnotesize
\begin{spacing}{0.9}
    \gram{\otte\ottinterrule
        \ottG\ottinterrule
        \ottv}
    \[
    \begin{array}{l}
    \text{Syntactic Sugar} \\
    \ottcoresugar \end{array}\]
\end{spacing}
    \caption{Syntax of \name}
    \label{fig:ecore:syntax}
\end{figure}

\paragraph{Syntactic Sugar}
Figure \ref{fig:ecore:syntax} also shows the syntactic sugar used in \name.
By convention, we use $[[t1 -> t2]]$ to represent 
$[[Pi x:t1.t2]]$ if $[[x]]$ does not occur free in $[[t2]]$. 
We also interchangeably use the dependent function type $[[(x:t1) -> t2]]$
to denote $[[Pi x:t1.t2]]$.
We use $[[let x:t=e2 in e1]]$ to locally bind a variable $[[x]]$ to 
an expression $[[e2]]$ in $[[e1]]$, and its variant $[[letrec]]$ for
recursive functions.
% Because $[[e2]]$ can be a type,
% we cannot simply desugar the let-binding into an application:
% \[
% [[let x:t=e2 in e1]] [[:=]] [[(\x:t.e1)e2]]
% \]
% We can find a counter-example like
% \[
% [[let x : star= int in (\ y : x . y) three]]
% \]

For the brevity of translation rules in Section \ref{sec:surface},
we use $[[foldn]]$ and $[[unfoldn]]$ to denote
$n$ consecutive cast operators. $[[foldn]]$ is simplified to only take
one type parameter, the last type $[[t1]]$ of the $n$ cast operations.
The original $[[foldn]]$ includes intermediate results $[[t2]], \dots, [[tn]]$
of type conversion:
\[
    [[foldn]] [ [[t1]], \dots, [[tn]] ] [[e]]  [[:=]] [[castup]] [
    [[t1]] ] ([[castup]] [ [[t2]] ] (\dots ( [[castup]] [ [[tn]] ]
  [[e]] ) \dots ))
\]
Due to the decidability of one-step reduction (shown in Section 
\ref{sec:ecore:meta}), the intermediate types can be uniquely determined, 
thus can be left out from the $[[foldn]]$ operator.

\subsection{Operational Semantics}\label{sec:ecore:opsem}

Figure \ref{fig:ecore:opsem} shows the \emph{call-by-name} operational
semantics, defined by one-step reduction. Three base cases include
\ruleref{S\_Beta} for beta reduction, \ruleref{S\_CastDownUp} for
cast canceling and \ruleref{S\_Mu} for recursion unrolling.
Two inductive cases, \ruleref{S\_App} and
\ruleref{S\_CastDown}, define reduction in the head position of an
application, and in the $[[castdown]]$ inner expression respectively.
The reduction rules are \emph{weak} because they are not allowed to do
the reduction inside a $\lambda$-term or $[[castup]]$-term --- both of
them are defined as values (see Figure \ref{fig:ecore:syntax}).

To evaluate the value of a term-level expression, we apply the
one-step reduction multiple times. The number of evaluation steps is
not restricted. So we can define the \emph{multi-step reduction}:

\begin{definition}[Multi-step reduction]
    The relation $[[->>]]$ is the transitive and reflexive closure of
    the one-step reduction $[[-->]]$.
\end{definition}

\begin{figure}[t]
\small
\begin{spacing}{0.7}
    \ottdefnstep{}
    \ottusedrule{\ottdruleSXXMu{}}
\end{spacing}
    \caption{Operational semantics of \name}
    \label{fig:ecore:opsem}
\end{figure}

\subsection{Typing}\label{sec:ecore:type}

Figure \ref{fig:ecore:typing} gives the \emph{syntax-directed} typing
rules of \name, including rules of context well-formedness $[[|- G]]$
and expression typing $[[G |- e : t]]$. Note that there is only a
single set of rules for expression typing, because there is no
distinction of different syntactic levels.

Most typing rules are quite standard. We write $[[|- G]]$ if a context
$[[G]]$ is well-formed. We use $[[G |- t : star]]$ to check if $[[t]]$ is a
well-formed type. Rule \ruleref{T\_Ax} is the ``type-in-type''
axiom. Rule \ruleref{T\_Var} checks the type of variable $[[x]]$ from
the valid context. Rules \ruleref{T\_App} and \ruleref{T\_Lam} check
the validity of application and abstraction respectively. Rule \ruleref{T\_Pi}
checks the type well-formedness of the dependent function. Rule
\ruleref{T\_Mu} checks the validity of a
recursive term. It ensures that the 
recursion $[[mu x:t.e]]$ should have the same type $[[t]]$ as the
binder $[[x]]$ and also the inner expression $[[e]]$.

\paragraph{The Cast Rules}
We focus on rules \ruleref{T\_CastUp} and \ruleref{T\_CastDown} that
define the semantics of \cast operators and replace the conversion
rule of \cc. The relation between the original
and converted type is defined by one-step reduction (see Figure
\ref{fig:ecore:opsem}). 

For example, given a judgement
$[[G |- e : t2]]$ and relation $[[t1 --> t2]] [[-->]] [[t3]]$,
$[[castup [t1] e]]$ expands the type of $[[e]]$ from $[[t2]]$ to
$[[t1]]$, while $[[castdown e]]$ reduces the type of $[[e]]$ from
$[[t2]]$ to $[[t3]]$. We can formally give the typing derivations of 
the examples in Section \ref{subsec:cast}:
{
\small
\[
\inferrule{[[G |- e : (\x : star.x) int]] \\ [[G |- int : star]] \\ [[(\x :
star.x) int --> int]]}{[[G |- (castdown e):int]]}
\]
\[
\inferrule{[[G |- three : int]] \\ [[G |- (\x : star.x) int : star]] \\ [[(\x :
star.x) int --> int]]}{[[G |- (castup[(\x : star.x) int] three):(\x : star.x)
int]]}
\]
}



Importantly, in \name term-level and type-level computation are treated 
differently. Term-level computation is dealt in the usual way, by 
using multi-step reduction until a value is finally obtained. 
Type-level computation, on the other hand, is controlled by the program:
each step of the computation is induced by a cast. If a type-level 
program requires $n$ steps of computation to reach normal form, 
then it will require $n$ casts to compute a type-level value.

\paragraph{Pros and Cons of Type in Type}
The ``type-in-type'' axiom is well-known to give rise 
to logical inconsistency~\cite{systemfw}. However, since our goal is to 
investigate core languages for languages that are logically
inconsistent anyway (due to general recursion), we do not view 
``type-in-type''  as a problematic rule.
On the other hand the rule \ruleref{T\_Ax} brings additional
expressiveness and benefits:
for example \emph{kind polymorphism} is supported in \name.
A term that takes a kind parameter like $[[\
    x:square.x->x]]$ cannot be typed in \cc, since $[[square]]$ is
the highest sort that does not have a type.
In contrast, \name does not have such limitation. Because of
the $[[star]]:[[star]]$ axiom, the term $[[\x:star.x->x]]$ has a legal 
type $[[Pi x:star.star]]$ in \name. It can be applied to 
a kind such as $[[star -> star]]$
to obtain $[[(star->star) -> (star -> star)]]$.

\paragraph{Syntactic Equality}
Finally, the definition of type equality in \name differs from
\cc. Without \cc's conversion rule, the type of a term cannot be
converted freely against beta equality, unless using \cast
operators. Thus, types of expressions are equal only if they are
syntactically equal (up to alpha renaming).

\begin{figure}[t]
\footnotesize
\centering
\begin{spacing}{1.0}
\renewcommand{\ottdrule}[4][]{{\inferrule{#2 }{#3}\,\ottdrulename{#4}}}
\renewenvironment{ottdefnblock}[3][]{\raggedright \framebox{\mbox{#2}} \quad #3 \\[0pt]}{}
\renewcommand{\ottusedrule}[1]{$#1\quad$}
    \ottdefnctx{}\ottinterrule
    \ottdefnexpr{}
    \ottusedrule{\ottdruleTXXMu{}}
\end{spacing}
    \caption{Typing rules of \name}
    \label{fig:ecore:typing}
\end{figure}

\subsection{The Two Faces of Recursion}
\label{sec:ecore:recur}
\paragraph{Term-level Recursion}

In \name, the $[[mu]]$-operator works as a \emph{fixpoint} on the term
level. By rule \ruleref{S\_Mu}, evaluating a term $[[mu x:t.e]]$ will
substitute all $[[x]]$'s in $e$ with the whole $[[mu]]$-term itself,
resulting in the unrolling $[[e [x |-> mu x:t.e] ]]$. The
$[[mu]]$-term is equivalent to a recursive function that should be
allowed to unroll without restriction. Therefore, the definition of
values is not changed in \name and a $[[mu]]$-term is not treated as a
value. 

Recall the factorial function example in Section
\ref{sec:overview:recursion}.
By rule \ruleref{T\_Mu}, the type of $\mathit{fact}$ is $[[int ->
    int]]$. Thus we can apply $\mathit{fact}$ to an integer.
In this example, we assume
evaluating the $\kw{if}$-$\kw{then}$-$\kw{else}$ construct and 
arithmetic expressions follows the one-step
reduction. Together with standard reduction rules \ruleref{S\_Mu} and
\ruleref{S\_App}, we can evaluate the term $\mathit{fact}~3$ as
follows:
\begin{spacing}{0.7}
\small
\begin{align*}
    &\mathit{fact}~3 \\ [[-->]]~& ([[\x : int . @@]]~\kw{if}~
  ~x==0~\kw{then}~1~\kw{else}~x \times \mathit{fact}~(x - 1))~3
  \\ [[-->]]~ & \kw{if}~3==0~\kw{then}~1~\kw{else}~3 \times
  \mathit{fact}~(3 - 1) \\ [[-->]]~& 3 \times \mathit{fact}~(3-1)
  \\ [[-->]]~& \dots [[-->]]~ 6. 
\end{align*}
\end{spacing}

Note that we never check if a $\mu$-term can terminate or not, which
is an undecidable problem for general recursive terms. The
factorial function example above can stop, while there exist some
terms that will loop forever. However, term-level non-termination is
only a runtime concern and does not block the type checker. In Section
\ref{sec:ecore:meta} we show type checking \name is still decidable
in the presence of general recursion.

\paragraph{Type-level Recursion}

On the type level, $[[mu x:t.e]]$ works as a \emph{iso-recursive}
type~\cite{eqi:iso}, a kind of recursive type that is not equal but only isomorphic
to its unrolling. Normally, we need to add two more primitives
$[[fold]]$ and $[[unfold]]$ for the iso-recursive type to map back
and forth between the original and unrolled form. Assume there exist
expressions $[[e1]]$ and $[[e2]]$ such that
\[\begin{array}{lll}
  &[[e1]] &: [[mu x:t.T]]\\
  &[[e2]] &: [[T [x |-> mu x:t.T] ]]
\end{array}\]
We have the following typing results
\[\begin{array}{lll}
  &[[unfold e1]] &: [[T [x |-> mu x:t.T] ]]\\
  &[[fold [mu x:t.T] e2]] &: [[mu x:t.T]]
\end{array}\]
which are derived from the standard rules for recursive types~\cite{tapl}
{
\small
\[ \inferrule{[[G |- e1 :(mu x:t.T)]] \\ [[G |- T[x |-> mu x:t.T] : star]]}
   {[[G |- unfold e1 : (T[x |-> mu x:t.T]) ]]} \]
\[ \inferrule{[[G |- e2 : T[x |-> mu x:t.T] ]] \\ [[G |- (mu x:t.T) : star]] }
   {[[G |- fold [mu x:t.T] e2 : (mu x:t.T) ]]} \]}
Thus, we have the following relation between types of $[[e1]]$ and $[[e2]]$
witnessed by $[[fold]]$ and $[[unfold]]$:
\begin{align*}
  [[mu x:t.T]] \xrightleftharpoons[{[[fold]]~[ [[mu x:t.T]] ]}]
  {[[unfold]]} [[T[x |-> mu x:t.T] ]]
\end{align*}

However, in \name we do not need to introduce $[[fold]]$ and
$[[unfold]]$ operators, because with the rule \ruleref{S\_Mu},
$[[castup]]$ and $[[castdown]]$ \emph{generalize} $[[fold]]$ and $[[unfold]]$.
Consider the same expressions $[[e1]]$ and $[[e2]]$ above. The type of
$[[e2]]$ is the unrolling of $[[e1]]$'s type, which follows the
one-step reduction relation by rule \ruleref{S\_Mu}:
\[ [[mu x:t.T --> T [x |-> mu x:t.T] ]] \]
By applying rules \ruleref{T\_CastUp} and \ruleref{T\_CastDown}, we
can obtain the following typing results:
\[\begin{array}{lll}
  &[[castdown e1]] &: [[T [x |-> mu x:t.T] ]]\\
  &[[castup [mu x:t.T] e2]] &: [[mu x:t.T]]
\end{array}\]
Thus, $[[castup]]$ and $[[castdown]]$ witness the isomorphism between
the original recursive type and its unrolling, behaving in the
same way
as $[[fold]]$ and $[[unfold]]$ in iso-recursive types:
\begin{align*}
  [[mu x:t.T]] \xrightleftharpoons[{[[castup]]~[ [[mu x:t.T]]
  ]}]{[[castdown]]} [[T[x |-> mu x:t.T] ]]
\end{align*}

\paragraph{Casts and Recursive Types}
Figure \ref{fig:ecore:syntax} shows that $[[castup]]$ is a \emph{value} that cannot be further reduced 
during the evaluation. It follows the convention of $[[fold]]$ in iso-recursive types~\cite{Vanderwaart:2003ik}.
But too many $[[castup]]$ constructs left for code generation will
increase the size of the program and cause runtime overhead.
Actually, $[[castup]]$ constructs can be safely erased after type checking:
they are \emph{computationally irrelevant} and do not actually 
transform a term other than changing its type.
%However, noticing that $[[castup [t] e]]$ or $[[castdown e]]$ have
%different types from $[[e]]$, types of terms are not preserved if directly removing cast constructs.
%This breaks the subject reduction theorem, and consequently type-safety.

%Therefore, we stick to the semantics of iso-recursive types for \cast
%operators which has type preservation. This way explicit casts and
%recursion can genuinely be seen as a generalization of recursive
%types (see Section \ref{sec:ecore:type}). Furthermore, we believe that all \cast operators can be
%eliminated through type erasure, when generating code,
%to address the potential performance issue of code generation.

An important remark is that casts are necessary, not only for controlling the unrolling of recursive types,
but also for type conversion of other constructs. 
This is because the ``type-in-type'' axiom~\cite{typeintype} makes it possible
to encode fixpoints even without a fixpoint primitive, i.e., the $\mu$-operator.
Thus if no casts would be performed on terms without recursive types,
it would still be possible to build a term with a non-terminating type and
make type-checking non-terminating.

\subsection{Metatheory}\label{sec:ecore:meta}
We now discuss the metatheory of \name. We focus on two properties:
the decidability of type checking and the type safety of the
language. First, we show that type checking \name is decidable
without requiring strong normalization. Second, the language is type-safe, 
proven by subject reduction and progress theorems.

\paragraph{Decidability of Type Checking}
For the decidability, we need to show there exists a type checking
algorithm, which never loops forever and returns a unique type for a
well-formed expression $[[e]]$. This is done by induction on the
length of $[[e]]$ and ranging over typing rules. Most expression
typing rules, including the rule \ruleref{T\_Mu} for recursion, 
which have only typing judgements in premises, are
already decidable by the induction hypothesis. Thus, it is
straightforward to follow the syntax-directed judgement to derive a
unique type checking result.

The critical case is for rules \ruleref{T\_CastUp} and
\ruleref{T\_CastDown}.  Both rules contain a premise that needs to
judge if two types $[[t1]]$ and $[[t2]]$ follow the one-step
reduction, i.e., if $[[t1 --> t2]]$ holds. We need to show that
$[[t2]]$ is \emph{unique} with respect to the one-step reduction, or
equivalently, reducing $[[t1]]$ by one step will get only a sole
result $[[t2]]$. Otherwise, assume $[[e]]:[[t1]]$ and there exists
$[[t2']]$ such that $[[t1 --> t2]]$ and $[[t1 --> t2']]$. Then the
type of $[[castdown e]]$ can be either $[[t2]]$ or $[[t2']]$ by rule
\ruleref{T\_CastDown}, which would not be decidable. The decidability
of one-step reduction is
given by the following lemma:

\begin{lemma}[Decidability of One-step Reduction]\label{lem:ecore:unique}
	The one-step reduction $[[-->]]$ is called decidable if 
given $[[e]]$ there is a unique $[[e']]$ such that $[[e --> e']]$ or
there is no such $[[e']]$.
\end{lemma}

\begin{proof}
	By induction on the structure of $[[e]]$.
\end{proof}

Note that the presence of recursion does not affect this lemma: 
given a recursive term
$[[mu x:t.e]]$, by rule \ruleref{S\_Mu}, there always exists a unique
term $[[e']]=[[e[x|->mu x:t.e] ]]$ such that $[[mu x:t.e -->
    e']]$.
With this result, we show a decidable algorithm to check whether
the one-step relation $[[t1 --> t2]]$ holds. An intuitive algorithm is to
reduce the type $[[t1]]$ by one step to obtain $[[t1']]$ (which is
unique by Lemma \ref{lem:ecore:unique}), and compare if $[[t1']]$ and
$[[t2]]$ are syntactically equal. Thus, checking if $[[t1 --> t2]]$ is
decidable and rules \ruleref{T\_CastUp} and \ruleref{T\_CastDown} are
therefore decidable. We can conclude the decidability of type
checking:

\begin{theorem}[Decidability of Type Checking \name]\label{lem:ecore:decide}
	There is an algorithm which given $[[G]], [[e]]$ computes the unique
$[[t]]$ such that $[[G |- e:t]]$ or reports there is no such $[[t]]$.
\end{theorem}

\begin{proof}
	By induction on the structure of $[[e]]$.
\end{proof}

We emphasize that when proving the decidability of type checking, we
do not rely on strong normalization. Intuitively,
explicit type conversion rules use one-step reduction, which already
has a decidable checking algorithm according to Lemma
\ref{lem:ecore:unique}. We do not need to further require the
normalization of terms. This is different from the proof for \cc which
requires the language to be strongly normalizing
\cite{pts:normalize}. In \cc the conversion rule needs to examine
the beta equivalence of terms, which is decidable only if every term
has a normal form.

\paragraph{Type Safety}
The proof of the type safety of \name is fairly standard by subject
reduction and progress theorems. The subject reduction proof relies on
the substitution lemma. We give the proof sketch of the related lemma and
theorems as follows:

\begin{lemma}[Substitution]\label{lem:ecore:subst}
	If $[[G1, x:T, G2 |- e1:t]]$ and $[[G1 |- e2:T]]$, then $[[G1, G2 [x |-> e2]
|- e1[x |-> e2]  : t[x |-> e2] ]]$.
\end{lemma}

\begin{proof}
    (\emph{Sketch}) By induction on the derivation of $[[G1, x:T, G2 |- e1:t]]$. We only treat cases \ruleref{T\_Mu}, \ruleref{T\_CastUp} and \ruleref{T\_CastDown} since other cases can be easily followed by the proof for PTS in \cite{handbook}.
\end{proof}

\begin{theorem}[Subject Reduction of \name]\label{lem:ecore:reduct}
If $[[G |- e:T]]$ and $[[e]] [[->>]] e'$ then $[[G |- e':T]]$.
\end{theorem}

\begin{proof}
    (\emph{Sketch}) We prove the case for one-step reduction, i.e., $[[e -->
e']]$. The theorem follows by induction on the number of one-step reductions
of $[[e]] [[->>]] [[e']]$.
    The proof is by induction with respect to the definition of one-step
reduction $[[-->]]$.
\end{proof}

\begin{theorem}[Progress of \name]\label{lem:ecore:prog}
If $[[empty |- e:T]]$ then either $[[e]]$ is a value $v$ or there exists $[[e]]'$
such that $[[e --> e']]$.
\end{theorem}

\begin{proof}
    By induction on the derivation of $[[empty |- e:T]]$.
\end{proof}
