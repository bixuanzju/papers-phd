\section{Formalization of the Surface Language}
\label{sec:surface}

% \begin{itemize}
% \item Expand the core language with datatypes and pattern matching by encoding.
% \item Give translation rules.
% \item Encode GADTs and maybe other Haskell extensions? GADTs seems challenging, so perhaps some other examples would be datatypes like $Fix f$, and $Monad$ as a record. Could formalize records in Haskell style.
% \end{itemize}

We formally present the surface language \sufcc,
built on top of \name with features that are convenient for functional
programming. All these features
can be elaborated into the core language without extending the
built-in language constructs of \name. In what follows, we first give
the syntax of the surface language, followed by the typing rules, then
we show the formal translation rules that translates a surface
language expression to an expression in \ecore. Finally we prove the
type safety of the translation. Note that for clarity reason, the
formalization here does not account for \emph{bidirectional type checking}
and \emph{surface-level casts} which are implemented in the prototype.
Nevertheless, the difference is subtle and does not affect the key
discussion on the encoding of datatypes of this section.


\begin{figure}[t]
\begin{small}
\centering
\begin{spacing}{0.9}
\gram{\ottpgm\ottinterrule
\ottdecl\ottinterrule
\ottu\ottinterrule
\ottp\ottinterrule
\ottE\ottinterrule
\ottGs}
\end{spacing}
\end{small}
\caption{Syntax of \sufcc}
\label{fig:surface:syntax}
\end{figure}

\begin{figure}[t]
\[\small
    \begin{array}{l}
    \text{Syntactic Sugar for \sufcc} \\
    \ottsurfsugar \end{array}\]
\[\small
    \begin{array}{l}
    \text{Additional Syntactic Sugar for \ecore} \\
    \ottcoresugar \end{array}\]
\caption{Syntactic sugar for \sufcc and \ecore}
\label{fig:surface:sugar}
\end{figure}

\subsection{Syntax}

The full syntax of \sufcc is defined in
Figure~\ref{fig:surface:syntax}. \sufcc has a new
syntax category: a program, consisting of a list of datatype
declarations, followed by an expression. For the purpose of
presentation, we adopt the following syntactic convention:
\[
\overline{\tau}^n \rightarrow \tau_r \equiv \tau_1 \rightarrow \dots \rightarrow \tau_n \rightarrow \tau_r
\]

\paragraph{Algebraic Datatypes}
An \emph{algebraic datatype} $D$ is introduced as a top-level
$\mathbf{data}$ declaration with its \emph{data constructors}. The type
of a data constructor $K$ takes the form:
\[
[[K : (<<u:k>>n) -> (<<x : A>>) -> D@<<u>>n]]
\]
The first $n$ quantified type variables $[[<<u>>]]$ appear in the same
order in the result type $[[D@<<u>>]]$.  Note that the use of the
dependent product in the data constructor arguments (i.e.,
$[[(<<x:A>>)]]$) makes it possible to let the types of some data
constructor arguments depend on other data constructor arguments.
%whereas in Haskell, this is not possible, because the function type
%($\rightarrow$) can be seen as an independent product type. 
The $\mathbf{case}$ expression is conventional, used to break up values
built with data constructors.  The patterns of a case expression are
flat (i.e., no nested patterns), and bind variables.

\paragraph{Syntactic Sugar}
Figure~\ref{fig:surface:sugar} shows the syntactic sugar for \sufcc.
We introduce a Haskell-like \emph{record} syntax, which are desugared
to datatypes with accompanying selector functions.
We define the let-binding $[[let x:A=E2 in E1]]$ and its recursive
variant $[[letrec]]$ as syntactic sugar.
Note that we cannot simply desugar the let-binding into an application like
\[
[[let x:A=E2 in E1]] [[:=]] [[(\x:A.E1)E2]]
\]
Since $[[E2]]$ can be a type, the application will be ill-typed if
$[[x]]$ occurs in the type annotation part of $[[E1]]$:
\[
[[let x : star= int in (\ y : x . y) three]] [[:=]] [[(\x:star.(\y:x.y)
3) Int]]
\]
The function $[[\x:star.(\y:x.y) 3]]$ is ill-typed since we cannot
typecheck $[[\y:x.y]]$ without knowing the exact definition of $[[x]]$.

For the brevity of translation rules in Section \ref{sec:suf:typ},
we also define several syntactic sugar of \ecore.
We interchangeably use the dependent function type $[[(x:t1) -> t2]]$
to denote $[[Pi x:t1.t2]]$. We use $[[foldn]]$ and $[[unfoldn]]$ to denote
$n$ consecutive cast operators. $[[foldn]]$ is simplified to only take
one type parameter, i.e., the last type $[[t1]]$ of the $n$ cast operations.
% The original $[[foldn]]$ includes intermediate results $[[t2]], \dots, [[tn]]$
% of type conversion:
% \[
%     [[foldn]] [ [[t1]], \dots, [[tn]] ] [[e]]  [[:=]] [[castup]] [
%     [[t1]] ] ([[castup]] [ [[t2]] ] (\dots ( [[castup]] [ [[tn]] ]
%   [[e]] ) \dots ))
% \]
Due to the determinacy of one-step reduction (see Section 
\ref{sec:ecore:meta}), the intermediate types can be uniquely determined, 
thus can be left out from the $[[foldn]]$ operator.

\paragraph{Differences from the Implementation}
To simplify the formalization and focus on the key technical
difficulties, we omit two features of \sufcc from the prototype
implementation: 1) bidirectional type checking; and 2) surface-level
casts. We consider such changes have subtle impact on formalization,
because 1) it is easy to show the fully annotated version formalized
here is equivalent to the bidirectional version; 2) it is
straightforward to introduce casts to surface level by directly
mapping constructs and typing rules from the core language.

Furthermore, casts are mostly intended to be generated by the
compiler, but not by the programmers. \sufcc generates \emph{weak}
casts when encoding datatypes and pattern matching. The encoding uses
casts and type-level computation steps in a fundamental way: we need
to use casts to simulate \fold/\unfold, and we also need small
type-level computational steps to encode parametrized datatypes (see
Section \ref{sec:fun}). An unfortunate consequence is that, for now it
makes the surface language less expressive (e.g., no explicit
type-level computation) than the core language. One avenue for future
work is to explore the addition of good, surface level, mechanisms for
doing type-level computation built on top of casts.
%However, as we will point in
%Section~\ref{sec:discuss}, we plan to address this issue in the future.
%we will add new language constructs in
%\sufcc that help programmers perform type-level computation in a way
%like type classes or type families do for Haskell programmers.

\begin{comment}
\paragraph{Syntactic Sugar}
For the sake of programming, \sufcc employs some syntactic sugar shown
in Figure \ref{fig:surface:syntax}. A non-dependent function type can
be written as $[[A1 -> A2]]$. A dependent function type
$[[Pi x : A1 . A2]]$ is abbreviated as $[[(x : A1) -> A2]]$ for easier
reading. A standard $\mathbf{letrec}$ construct is used to build
recursive functions. We also introduce a Haskell-like record syntax,
which are desugared to datatypes with accompanying selector functions.
\end{comment}

\begin{figure}[htpb]
\begin{small}
\centering
\renewcommand{\ottinterrule}{\\[1.0mm]}
\begin{spacing}{1.2}
\renewcommand{\ottdrule}[4][]{{\inferrule{#2 }{#3}\,{\scriptsize \ottdrulename{#4}}}}
\renewenvironment{ottdefnblock}[3][]{\raggedright \framebox{\mbox{#2}} \quad #3 \\[0pt]}{}
\renewcommand{\ottusedrule}[1]{$#1\quad$}
\ottdefnctxtrans{}\ottinterrule
\ottdefnpgmtrans{}\ottinterrule
\ottdefndecltrans{}
\[\hlmath{\ottdecltrans}\]\ottinterrule % defined in otthelper.mng.tex
\ottdefnpattrans{}\ottinterrule
% \setlength{\lineskip}{8pt}
\ottdefnexprtrans{}
% \end{spacing}
\end{spacing}
\end{small}
\caption{Type-directed translation rules of \sufcc}
\label{fig:source:translate}
\end{figure}

\subsection{Typing Rules}
Figure~\ref{fig:source:translate} defines the type system of the
surface language. The gray parts can be ignored for the
moment. Several new typing judgments are introduced in the type
system. The use of different subscripts in the judgments is to be
distinct from the ones used in \ecore. Most rules are standard for
systems based on \coc, including the rules for the well-formedness of
contexts (\ruleref{TRenv\_Empty}, \ruleref{TRenv\_Var}), inferring the
types of variables (\ruleref{TR\_Var}), and dependent application
(\ruleref{TR\_App}).

% Two judgments $[[Gs |- pgm : A]]$ and
% $[[Gs |- decl : Gs']]$ type check a whole program, and datatype
% declarations, respectively.

Rule \ruleref{TRpgm\_Pgm} type checks a whole program. It first
type-checks the declarations, which in return gives a new typing
environment. Combined with the original environment, it then continues
to type check the final expression and return the result type. Rule
\ruleref{TRpgm\_Data} is used to type check datatype declarations. It
first ensures the well-formedness of the type of the type constructor
(of sort $\star$). Then it ensures that the types of data constructors
are valid.  Note that since our system adopts the $\star : \star$
axiom, this means we can express kind polymorphism in datatypes.

Rules \ruleref{TR\_Case} and \ruleref{TRpat\_Alt} handle the type
checking of case expressions. The conclusion of \ruleref{TS\_Case}
binds the scrutinee $[[E1]]$ and alternatives
$\overline{p \Rightarrow E_2}$ to the right types. The first premise
of \ruleref{TS\_Case} binds the actual type constructor arguments to
$[[<<U>>]]$. The second premise checks whether the types of the
alternatives, instantiated to the actual type constructor arguments
$[[<<U>>]]$ (via \ruleref{TRpat\_Alt}), are equal. Finally the third
premise checks the well-formedness of the result type.

%As can be seen, currently \sufcc does not support refinement on the
%final result of each data constructor, as in GADTs. However, our
%encoding method does support some form of GADTs, as is discussed in
%Section~\ref{Discussion}.

\subsection{Translation Overview}\label{sec:suf:typ}

We use a type-directed translation. The basic translation rules have the form:

\[
[[Gs  |- E : A ~> e]]
\]

The rule states that \ecore expression $[[e]]$ is the translation of
the surface expression $[[E]]$ of type $[[A]]$. We emphasize that such
translation only involves \emph{weak} cast operators, thus the target
system is \ecore. The extra power of full casts is not needed.
The gray parts in
Figure~\ref{fig:source:translate} define the translation
rules. Among others, rules \ruleref{TRdecl\_Data}, \ruleref{TRpat\_Alt} and
\ruleref{TR\_Case} are the essence of the translation. Rule
\ruleref{TR\_Case} translates case expressions into applications by
first translating the scrutinee $E_1$ to $e_1$, casting it down to the
right type. It is then applied to the result type of the body
expression and a list of \ecore expressions of the alternatives. Rule
\ruleref{TRpat\_Alt} translates each alternative into a lambda
abstraction, where each bound variable in the pattern corresponds to
bound variables in the lambda abstraction in the same order. The body
in the alternative is recursively translated and treated as the body
of the lambda abstraction. % Note that due to the rigidness of the
% translation, pattern matching must be exhaustive, and the order of
% patterns matters (the same order as in the datatype declaration).

As for the translation of datatypes, we choose to work with the Scott
encodings of datatypes~\citeapp{encoding:scott}. Scott encodings encode
case analysis, making it convenient to translate pattern
matching. Rule \ruleref{TRDecl\_Data} translates datatypes into \ecore
expressions. First of all, it results in an incomplete expression (as
can be seen by the incomplete $\mathsf{let}$ expressions). The result
expression is supposed to be prepended to the translation of the last
expression to form a complete \ecore expression, as specified by Rule
\ruleref{TRpgm\_Pgm}. Furthermore, each type constructor is translated
to a recursive type, of which the body is a type-level lambda
abstraction. What is interesting is that each recursive occurrence of
the datatype in the data constructor parameters is replaced with the
variable $[[X]]$. Note that for the moment, the result type variable
$[[bb]]$ is restricted to have kind $\star$.%  This could pose
% difficulties when translating GADTs, which is an interesting topic for
% future work. 
Each data constructor is translated to a lambda
abstraction. Notice how we use $[[castup]]$ in the lambda body to get
the right type.

The rest of the translation rules hold few surprises.

\subsection{Type Safety of the Translation}

Now that we have elaborated the translation, we show the type safety
theorem of the translation and its proof.

\begin{theorem}[Type Safety of Expression Translation]
Given a surface language expression $[[E]]$ and context $[[Gs]]$, 
if $[[Gs |- E:A ~> e]]$, $[[Gs |- A:star ~> t]]$ and $[[|- Gs ~> G]]$, then
$[[G |- e:t]]$.
\end{theorem}

\begin{proof}
    By induction on the derivation of $[[Gs |- E : A ~> e]]$. Suppose there is
a core language context $[[G]]$ such that $[[|- Gs ~> G]]$.
    \begin{description}
        \renewcommand{\hlmath}[1]{#1}
        \item[Case $\ottdruleTRXXAx{}$:] $\quad$ \\ Trivial. $[[e]] = [[t]] = [[star]]$ and
$[[G |- star:star]]$ holds by rule \ruleref{T\_Ax}.
        \item[Case $\ottdruleTRXXVar{}$:] $\quad$ \\ Trivial. By rule \ruleref{T\_Var}, we
have $[[|- Gs ~> G]]$, then $[[x]]:[[t]] [[elt]] [[G]]$ where $[[Gs |-
A:star~>t]]$.
        \item[Case $\ottdruleTRXXApp{}$:] $\quad$ \\ Suppose
            \[\begin{array}{l}
            [[Gs |- E1 E2 : A1[x |-> E2] ~> e1 e2]] \\
            [[Gs |- A1[x |-> E2] : star ~> t1 [x |-> e2] ]].
            \end{array} \]
            By induction
            hypothesis, we have 
            $
            [[G |- e1 : (Pi x:t2.t1)]],
            [[G |- e2:t2]],
            $
            where
            \[\begin{array}{l}
             [[Gs |- E1 : (Pi x:A2.A1) ~> e1]] \\
              [[Gs |- (Pi x:A2.A1) : star ~> (Pi x:t2.t1)]] \\
              [[Gs |- E2 : A2 ~> e2]] \\
              [[Gs |- A2 : star ~> t2]].
            \end{array}\] Thus by rule \ruleref{T\_App}, we can conclude $[[G |- e1 e2 : t1 [x |-> e2] ]]$.
        \item[Case $\ottdruleTRXXLam{}$:] $\quad$ \\ Suppose
            \[\begin{array}{l}
            [[Gs |- (\x:A1.E):(Pi x:A1.A2) ~> \x:t1.e]] \\ 
            [[Gs |- Pi x:A1.A2 : star ~> Pi x:t1.t2]].
            \end{array} \]
            By the induction hypothesis, we have 
            $
            [[G, x : t1 |- e:t2]],
            [[G |- Pi x:t1.t2 : star]]
            $
            where 
            \[
            \begin{array}{ll}
            [[Gs, x : A1 |- E : A2 ~> e]] & \\
            [[Gs |- A1 : star ~> t1]] & [[Gs |- A2 : star ~> t2]] \\
            [[Gs |- (Pi x:A1.A2) : s ~> Pi x:t1.t2]] &
            \end{array}
            \]
            Thus by rule \ruleref{T\_Lam}, we can conclude $[[G |- (\x:t1.e):(Pi x:t1.t2)]]$.
        \item[Case $\ottdruleTRXXPi{}$:] $\quad$ \\ Suppose 
                \[ [[Gs |- (Pi x:A1.A2):s ~> Pi x:t1.t2]]. \] 
            By the induction hypothesis, we have 
            $
                [[G |- t1 : star]], [[G, x : t1 |- t2 : star]]
            $
            where
            $
                [[Gs |- A1 : s ~> t1]], [[Gs, x: A1 |- A2 : s ~> t2]]
            $
            Thus by rule \ruleref{T\_Pi} we can conclude $[[G |- (Pi x:t1.t2) : star]]$.
        \item[Case $\ottdruleTRXXMu{}$:] $\quad$ \\ Suppose 
                \[\begin{array}{l}
                    [[Gs |- (mu x:A . E):A ~> mu x:t.e]] \\
                    [[Gs |- A : star ~> t]]. 
                \end{array}\]
            By the induction hypothesis, we have 
                \[ [[G, x : t |- e : t]],\text{ where }[[Gs, x:A |- E:A ~> e]]. \] 
            Thus by rule \ruleref{T\_Mu}, we can conclude $[[G |- (mu x:t.e) : t]]$.
        \item[Case $\ottdruleTRXXCase{}$:] $\quad$ \\ Suppose 
            \[\begin{array}{l}
                [[Gs |- case E1 of << p => E2>> : B ~> (unfoldnp e1) T <<e2>>]] \\
                [[Gs |- B : star ~> T]].
            \end{array}\]
            By the induction hypothesis, we have 
            \[\begin{array}{ll}
                [[Gs |- E1 : D@<<U>>n ~> e1]] &
                [[Gs |- D@<<U>>n : star ~> t1]] \\
                [[G |- e1 : t1]] &
                [[<< Gs |- p => E2 : D@<<U>>n -> B ~> e2 >>]]            
            \end{array}\]
            By rule \ruleref{TRpat\_Alt}, we have
            \begin{align*}
                [[p]] &[[==]] [[K <<x:A[<< u |-> U >>]>>]] \\
                [[<<e2>>]] &[[==]] [[<<\ <<x:t'>> .e>>]]
            \end{align*}
            where
            \[\begin{array}{ll}
                [[<<Gs |- E2 : B ~> e>>]] &
                [[<<G |- e : T>>]] \\
                [[<<Gs |- U : star ~> uu'>>]] &
                [[<<Gs |- A[<< u |-> U >>]:star ~> t[<<uu |-> uu'>>]>>]] \\
                [[t']] [[==]] [[ t[<<uu |-> uu'>>] ]]
            \end{array}\]
            By rule \ruleref{TRdecl\_Data}, we have $[[D]]  [[ == ]] \ottdeclD$. Thus,
            \[ [[t1]] [[==]] [[D]] [[<<uu'>>]]^n,\text{ where }[[<<G |- uu' : ro>>]].\] 
            Note that by operational semantics, the following reduction sequence follows for $[[t1]]$:
            \begin{align*}
                [[D]] [[<<uu'>>]]^n~
                &[[-->]]~ [[(\ <<u:ro>>n . (bb:star) -> << ((<<x : t[D |-> X][X |-> D]>>) -> bb) >> -> bb) ]][[<<uu'>>]]^n\\
                &[[-->>]]~ [[(bb:star) -> << (<<x:t'>>) -> bb >> -> bb]]
            \end{align*}
            Then by
            rule \ruleref{T\_CastDown} and the definition of $n$-step cast operator, the
            type of $[[unfoldnp e1]]$ is \[ [[(bb:star) -> << (<<x:t'>>) -> bb >> -> bb]].\] Note
            that by rule \ruleref{T\_Lam}, $[[G |- e2 : (<<x:t'>>) -> T]]$. Therefore, by rule
            \ruleref{T\_App}, we can conclude $[[G |- (unfoldnp e1) T <<e2>> : T]]$.
    \end{description}
\end{proof}

%%% Local Variables:
%%% mode: latex
%%% TeX-master: "../main"
%%% End:
