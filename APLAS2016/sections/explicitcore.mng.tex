
\section{Dependent Types with Iso-Types}\label{sec:ecore}

In this section, we present the \ecore calculus, which uses a
(call-by-name) weak-head reduction strategy in casts. This calculus is very
close to the calculus of constructions, except for three key differences:
1) the absence of the $\Box$ constant (due to use of the
``type-in-type'' axiom); 2) the existence of two \cast operators; 3)
general recursion on both term and type level.
Unlike \cc the proof of 
decidability of type checking for \ecore does not require the strong normalization of the
calculus. Thus, the addition of general recursion does not break decidable
type checking. In the rest of this section, we demonstrate the syntax,
operational semantics, typing rules and metatheory of \ecore.
Full proofs of the meta-theory can be found in the extended version of
this paper~\cite{full}.
%We have completely formalized the proofs of metatheory in Coq. Full
%specification can be found in the appendix.

\subsection{Syntax}\label{sec:ecore:syn}

Figure \ref{fig:ecore:syntax} shows the syntax of \ecore, including
expressions, contexts and values. \ecore uses a unified 
representation for different syntactic levels by following the
\emph{pure type system} (PTS) representation of \cc~\cite{handbook}. There
is no syntactic distinction between terms, types or kinds.
We further merge types and
kinds together by including only a single sort $[[star]]$ instead of two distinct sorts $[[star]]$ and
$[[square]]$ of \cc.
This design
brings economy for type checking, since one set of rules can cover
all syntactic levels. We use metavariables $[[t]]$ and
$[[T]]$ for an expression on the type-level position and $e$ for one
on the term level. We use $[[t1 -> t2]]$ as a syntactic sugar for
$[[Pi x:t1.t2]]$ if $[[x]]$ does not occur free in $[[t2]]$.

% \paragraph{Type of Types}
% In \cc, there are two distinct sorts $[[star]]$ and
% $[[square]]$ representing the type of \emph{types} and \emph{sorts}
% respectively, and an axiom $[[star]]:[[square]]$ specifying the
% relation between the two sorts~\cite{handbook}. In \ecore, we further merge types and
% kinds together by including only a single sort $[[star]]$ and an
% impredicative axiom $[[star]]:[[star]]$.

\paragraph{Explicit Type Conversion}

We introduce two new primitives $[[castup]]$ and $[[castdown]]$
(pronounced as ``cast up'' and ``cast down'') to replace the implicit
conversion rule of \cc with \emph{one-step} explicit type
conversions. The type-conversions perform two directions of conversion:
$[[castdown]]$ is for the \emph{one-step reduction} of types, and
$[[castup]]$ is for the \emph{one-step expansion}. The $[[castup]]$
construct takes a
type parameter $[[t]]$ as the result type of one-step expansion
for disambiguation (see also Section \ref{subsec:cast}). The $[[castdown]]$ construct
does not need a type parameter,  because the result type of one-step reduction
is uniquely determined, as we shall see in Section \ref{sec:ecore:meta}.

We use syntactic sugar $[[foldn]]$ and $[[unfoldn]]$ to denote $n$
consecutive cast operators (see Figure
\ref{fig:ecore:syntax}). Alternatively, we can introduce them as
built-in operators but treat one-step casts as syntactic sugar
instead. Making $n$-step casts built-in can reduce the number of
individual cast constructs, but makes cast operators less fundamental
in the discussion of meta-theory. Thus, in the paper, we treat
$n$-step casts as syntactic sugar but make them built-in in the
implementation for better performance. Note that $[[foldn]]$ is
simplified to take just one type parameter, i.e., the last type
$[[t1]]$ of the $n$ cast operations. Due to the determinacy of
one-step reduction (see Lemma \ref{lem:ecore:unique}), the
intermediate types can be uniquely determined, thus can be left out
from the $[[foldn]]$ operator.

\paragraph{General Recursion}
% General recursion allows a large number of programs that can be expressed in programming languages such 
% as Haskell to be expressed in \ecore as well.
We add one primitive $[[mu]]$ to represent general recursion.
It has a uniform representation on both term level and type level: the
same construct works both as a term-level fixpoint and a recursive type. The recursive
expression $[[mu x:t.e]]$ is \emph{polymorphic}, in the sense that $[[t]]$ 
is not restricted to $[[star]]$ but can be any type, 
such as a function type $[[int -> int]]$ or a kind $[[star -> star]]$.

\begin{figure}[t]
\centering
\begin{small}
\begin{tabular}{lrcl}
Expressions & $e,[[t]],[[T]]$ & \; \syndef \; & $x \mid [[star]] \mid [[e1
                                          e2]] \mid [[\x : t . e]] \mid [[Pi x : t1. t2]]$ \\
&& \synor & $[[mu x : t . e]] \mid [[castup [t] e]] \mid [[castdown e]]$ \\
Contexts &
$[[G]]$ & \syndef & $[[empty]] \mid [[G]],x:[[t]]$ \\
Values &
$v$ & \syndef & $[[star]] \mid [[\x :t.e]] \mid
                [[Pi x:t1.t2]] \mid [[castup [t] v]]$ \\
\multicolumn{1}{l}{Syntactic Sugar \;} \\
% \multicolumn{3}{l}{$[[t1 -> t2]] \triangleq [[Pi x : t1.t2]]$, where $x \not \in \mathsf{FV}(\tau_2)$} \\
\end{tabular}
\begin{tabular}{lcl}
$[[t1 -> t2]]$ & \syneq & $[[Pi x : t1.t2]]$, where $x \not \in \mathsf{FV}(\tau_2)$ \\
$[[foldn [t1] e]]$ & \syneq & $[[castup]] [ [[t1]] ] ([[castup]] [ %
        [[t2]] ] (\dots ( [[castup [tn] e]] ) \dots ))$\\
                    && , where $\tau_1 [[-->]] \tau_2 [[-->]] \dots [[-->]] \tau_n$ \\
  $[[unfoldn e]]$ & \syneq & $\underbrace{[[castdown]] ([[castdown]] %
        (\dots ( [[castdown]]}_n [[e]]) \dots ))$ \\
\end{tabular}
\end{small}
\caption{Syntax of \ecore}
\label{fig:ecore:syntax}
\end{figure}

\begin{comment}
\paragraph{Syntactic Sugar}
Figure \ref{fig:ecore:syntax} also shows the syntactic sugar used in \ecore.
By convention, we use $[[t1 -> t2]]$ to represent 
$[[Pi x:t1.t2]]$ if $[[x]]$ does not occur free in $[[t2]]$. 
% We also interchangeably use the dependent function type $[[(x:t1) -> t2]]$
% to denote $[[Pi x:t1.t2]]$.
% We use $[[let x:t=e2 in e1]]$ to locally bind a variable $[[x]]$ to 
% an expression $[[e2]]$ in $[[e1]]$, and its variant $[[letrec]]$ for
% recursive functions.
% Because $[[e2]]$ can be a type,
% we cannot simply desugar the let-binding into an application:
% \[
% [[let x:t=e2 in e1]] [[:=]] [[(\x:t.e1)e2]]
% \]
% We can find a counter-example like
% \[
% [[let x : star= int in (\ y : x . y) three]]
% \]

% For the brevity of translation rules in Section \ref{sec:surface},
We use $[[foldn]]$ and $[[unfoldn]]$ to denote
$n$ consecutive cast operators. $[[foldn]]$ is simplified to only take
one type parameter, i.e., the last type $[[t1]]$ of the $n$ cast operations.
% The original $[[foldn]]$ includes intermediate results $[[t2]], \dots, [[tn]]$
% of type conversion:
% \[
%     [[foldn]] [ [[t1]], \dots, [[tn]] ] [[e]]  [[:=]] [[castup]] [
%     [[t1]] ] ([[castup]] [ [[t2]] ] (\dots ( [[castup]] [ [[tn]] ]
%   [[e]] ) \dots ))
% \]
Due to the decidability of one-step reduction (see Section 
\ref{sec:ecore:meta}), the intermediate types can be uniquely determined, 
thus can be left out from the $[[foldn]]$ operator.
\end{comment}

\subsection{Operational Semantics}\label{sec:ecore:opsem}

Figure \ref{fig:ecore:opsem} shows the small-step, \emph{call-by-name} operational
semantics. Three base cases include
\ruleref{S\_Beta} for beta reduction, \ruleref{S\_Mu}
for recursion unrolling and \ruleref{S\_CastElim} for cast canceling.
Three inductive cases, \ruleref{S\_App},
\ruleref{S\_CastDown} and \ruleref{S\_CastUp}, define reduction at the head position of an
application, and the inner expression of $[[castdown]]$ and
$[[castup]]$ terms, respectively.
Note that \ruleref{S\_CastElim} and \ruleref{S\_CastDown} do not
overlap because in the former rule, the inner term of $[[castdown]]$
is a value (see Figure \ref{fig:ecore:syntax}), i.e., $[[castup [t] v]]$,
but not a value in the latter rule.

The reduction rules are called \emph{weak-head} because only the
head term of an application can be reduced, as indicated by the rule
\ruleref{S\_App}. Reduction is also not
allowed inside the $\lambda$-term and $\Pi$-term which are both defined as values.
Weak-head reduction rules are used for both type conversion and term
evaluation. Thus, we refer to cast operators in \ecore as \emph{weak}
casts. To evaluate the value of a term-level expression, we apply the
one-step (weak-head) reduction multiple times, i.e., multi-step reduction, the
transitive and reflexive closure of the one-step reduction.

% So we can define the \emph{multi-step reduction}:

% \begin{definition}[Multi-step reduction]
%     The relation $[[->>]]$ is the transitive and reflexive closure of
%     the one-step reduction $[[-->]]$.
% \end{definition}

\begin{figure}[t]
\begin{small}
\centering
\renewcommand{\ottdrule}[4][]{{\inferrule{#2 }{#3}\,{\scriptsize\ottdrulename{#4}}}}
\renewenvironment{ottdefnblock}[3][]{\raggedright \framebox{\mbox{#2}} \quad #3 \\[0pt]}{}
\renewcommand{\ottusedrule}[1]{$#1\quad$}
\begin{spacing}{0.8}
\ottdefnstep{}
\end{spacing}
\end{small}
\caption{Operational semantics of \ecore}
    \label{fig:ecore:opsem}
\end{figure}

\subsection{Typing}\label{sec:ecore:type}

Figure \ref{fig:ecore:typing} gives the \emph{syntax-directed} typing
rules of \ecore, including rules of context well-formedness $[[|- G]]$
and expression typing $[[G |- e : t]]$. Note that there is only a
single set of rules for expression typing, because there is no
distinction of different syntactic levels.

Most typing rules are quite standard. We write $[[|- G]]$ if a context
$[[G]]$ is well-formed. We use $[[G |- t : star]]$ to check if $[[t]]$ is a
well-formed type. Rule \ruleref{T\_Ax} is the ``type-in-type''
axiom. Rule \ruleref{T\_Var} checks the type of variable $[[x]]$ from
the valid context. Rules \ruleref{T\_App} and \ruleref{T\_Lam} check
the validity of application and abstraction respectively. Rule \ruleref{T\_Pi}
checks the type well-formedness of the dependent function. Rule
\ruleref{T\_Mu} checks the validity of a
recursive term. It ensures that the 
recursion $[[mu x:t.e]]$ should have the same type $[[t]]$ as the
binder $[[x]]$ and also the inner expression $[[e]]$.

\paragraph{The Cast Rules}
We focus on the rules \ruleref{T\_CastUp} and \ruleref{T\_CastDown} that
define the semantics of \cast operators and replace the conversion
rule of \cc. The relation between the original
and converted type is defined by one-step weak-head reduction (see Figure
\ref{fig:ecore:opsem}). 
For example, given a judgment
$[[G |- e : t2]]$ and relation $[[t1 --> t2]] [[-->]] [[t3]]$,
$[[castup [t1] e]]$ expands the type of $[[e]]$ from $[[t2]]$ to
$[[t1]]$, while $[[castdown e]]$ reduces the type of $[[e]]$ from
$[[t2]]$ to $[[t3]]$. We can formally give the typing derivations of 
the examples in Section \ref{subsec:cast}:
\[
\inferrule{[[G |- e : (\x : star.x) int]]  \\\\ [[(\x :
star.x) int --> int]]}{[[G |- (castdown e):int]]}
\quad
\inferrule{[[G |- three : int]] \\ [[G |- (\x : star.x) int : star]] \\\\ [[(\x :
star.x) int --> int]]}{[[G |- (castup[(\x : star.x) int] three):(\x : star.x)
int]]}
\]
Importantly, in \ecore term-level and type-level computation are treated 
differently. Term-level computation is dealt in the usual way, by 
using multi-step reduction until a value is finally obtained. 
Type-level computation, on the other hand, is controlled by the program:
each step of the computation is induced by a cast. If a type-level 
program requires $n$ steps of computation to reach the normal form, 
then it will require $n$ casts to compute a type-level value.

\paragraph{Pros and Cons of Type in Type}
The ``type-in-type'' axiom is well-known to give rise 
to logical inconsistency~\cite{systemfw}. However, since our goal is to 
investigate core languages for languages that are logically
inconsistent anyway (due to general recursion), we do not view 
``type-in-type''  as a problematic rule.
On the other hand the rule \ruleref{T\_Ax} brings additional
expressiveness and benefits:
for example \emph{kind polymorphism}~\cite{fc:pro} is supported in \ecore.
\begin{comment}
A term that takes a kind parameter like $[[\
    x:square.x->x]]$ cannot be typed in \cc, since $[[square]]$ is
the highest sort that does not have a type.
In contrast, \ecore does not have such limitation. Because of
the $[[star]]:[[star]]$ axiom, the term $[[\x:star.x->x]]$ has a legal 
type $[[Pi x:star.star]]$ in \ecore. It can be applied to 
a kind such as $[[star -> star]]$
to obtain $[[(star->star) -> (star -> star)]]$.
\end{comment}

\paragraph{Syntactic Equality}
Finally, the definition of type equality in \ecore differs from
\cc. Without \cc's conversion rule, the type of a term cannot be
converted freely against beta equality, unless using \cast
operators. Thus, types of expressions are equal only if they are
syntactically equal (up to alpha renaming). 
% Types that are equal up to
% beta equality are only \emph{isomorphic} but not equal, thus called
% \emph{iso-types}. 

\begin{figure}[t]
\footnotesize
\begin{spacing}{0.8}
\renewcommand{\ottdrule}[4][]{{\inferrule{#2 }{#3}\,{\scriptsize \ottdrulename{#4}}}}
\coretyping
\end{spacing}
    \caption{Typing rules of \ecore}
    \label{fig:ecore:typing}
\end{figure}

\subsection{The Two Faces of Recursion}\label{sec:ecore:twoface}
\label{sec:ecore:recur}
% We further elaborate how general recursion works as recursive
% functions at term level, and models iso-recursive types, a special
% case of iso-types, at type level.
\vspace{5pt}
\paragraph{Term-level Recursion}

In \ecore, the $[[mu]]$-operator works as a \emph{fixpoint} on the term
level. By rule \ruleref{S\_Mu}, evaluating a term $[[mu x:t.e]]$ will
substitute all $[[x]]$'s in $e$ with the whole $[[mu]]$-term itself,
resulting in the unrolling $[[e [x |-> mu x:t.e] ]]$. The
$[[mu]]$-term is equivalent to a recursive function that should be
allowed to unroll without restriction. 
% Therefore, the definition of
% values is not changed in \ecore and a $[[mu]]$-term is not treated as a
% value. 
Recall the factorial function example in Section
\ref{sec:overview:recursion}.
By rule \ruleref{T\_Mu}, the type of $\mathit{fact}$ is $[[int ->
    int]]$. Thus we can apply $\mathit{fact}$ to an integer.
Note that by rule \ruleref{S\_Mu}, $\mathit{fact}$ will be
unrolled to a $\lambda$-term.
Assuming the evaluation of $\kw{if}$-$\kw{then}$-$\kw{else}$ construct
and arithmetic expressions follows the one-step reduction, we can
evaluate the term $\mathit{fact}~3$ as follows:
\[
\begin{array}{llr}
    &\mathit{fact}~3 \\ [[-->]]~& ([[\x : int . @@]]\kw{if}
  ~x==0~\kw{then}~1~\kw{else}~x \times \mathit{fact}~(x - 1))~3 &
\text{\texttt{-- by \ruleref{S\_App}}}
  \\ [[-->]]~ & \kw{if}~3==0~\kw{then}~1~\kw{else}~3 \times
  \mathit{fact}~(3 - 1) & \text{\texttt{-- by \ruleref{S\_Beta}}} \\
[[-->]] & \dots [[-->]]~ 6
\end{array}\]

Note that we never check if a $\mu$-term can terminate or not, which
is an undecidable problem for general recursive terms. The
factorial function example above can stop, while there exist some
terms that will loop forever. However, term-level non-termination is
only a runtime concern and does not block the type checker. In Section
\ref{sec:ecore:meta} we show type checking \ecore is still decidable
in the presence of general recursion.

\paragraph{Type-level Recursion}

On the type level, $[[mu x:t.e]]$ works as a \emph{iso-recursive}
type~\cite{eqi:iso}, a kind of recursive type that is not equal but only \emph{isomorphic}
to its unrolling. Normally, we need to add two more primitives
$[[fold]]$ and $[[unfold]]$ for the iso-recursive type to map back
and forth between the original and unrolled form. Assuming there exist
expressions $[[e1]]$ and $[[e2]]$ such that
  $[[e1]] : [[mu x:t.T]]$ and
  $[[e2]] : [[T [x |-> mu x:t.T] ]]$,
we have the following typing results:
\[\begin{array}{lll}
  &[[unfold e1]] &: [[T [x |-> mu x:t.T] ]]\\
  &[[fold [mu x:t.T] e2]] &: [[mu x:t.T]]
\end{array}\]
by applying standard typing rules of iso-recursive types~\cite{tapl}:
\[\inferrule{[[G |- e1 :mu x:t.T]]}
   {[[G |- unfold e1 : T[x |-> mu x:t.T] ]]}  \quad
\inferrule{[[G |- mu x:t.T : star]] \\\\ [[G |- e2 : T[x |-> mu x:t.T] ]] }
{[[G |- fold [mu x:t.T] e2 : mu x:t.T ]]} \]
\begin{comment}
Thus, we have the following relation between types of $[[e1]]$ and $[[e2]]$
witnessed by $[[fold]]$ and $[[unfold]]$:
\begin{align*}
  [[mu x:t.T]] \xrightleftharpoons[{[[fold]]~[ [[mu x:t.T]] ]}]
  {[[unfold]]} [[T[x |-> mu x:t.T] ]]
\end{align*}
\end{comment}

However, in \ecore we do not need to introduce $[[fold]]$ and
$[[unfold]]$ operators, because with the rule \ruleref{S\_Mu},
$[[castup]]$ and $[[castdown]]$ \emph{generalize} $[[fold]]$ and $[[unfold]]$.
Consider the same expressions $[[e1]]$ and $[[e2]]$ above. The type of
$[[e2]]$ is the unrolling of $[[e1]]$'s type, which follows the
one-step reduction relation by rule \ruleref{S\_Mu}:
$ [[mu x:t.T --> T [x |-> mu x:t.T] ]]$.
By applying rules \ruleref{T\_CastUp} and \ruleref{T\_CastDown}, we
can obtain the following typing results:
\[\begin{array}{lll}
  &[[castdown e1]] &: [[T [x |-> mu x:t.T] ]]\\
  &[[castup [mu x:t.T] e2]] &: [[mu x:t.T]]
\end{array}\]
Thus, $[[castup]]$ and $[[castdown]]$ witness the isomorphism between
the original recursive type and its unrolling, behaving in the
same way
as $[[fold]]$ and $[[unfold]]$ in iso-recursive types.
\begin{comment}
\begin{align*}
  [[mu x:t.T]] \xrightleftharpoons[{[[castup]]~[ [[mu x:t.T]]
  ]}]{[[castdown]]} [[T[x |-> mu x:t.T] ]]
\end{align*}
\end{comment}

\begin{comment}
\paragraph{Casts and Recursive Types}
Figure \ref{fig:ecore:syntax} shows that $[[castup]]$ is a \emph{value} that cannot be further reduced 
during the evaluation. It follows the convention of $[[fold]]$ in iso-recursive types~\cite{Vanderwaart:2003ik}.
But too many $[[castup]]$ constructs left for code generation will
increase the size of the program and cause runtime overhead.
Actually, $[[castup]]$ constructs can be safely erased after type checking:
they are \emph{computationally irrelevant} and do not actually 
transform a term other than changing its type.
\end{comment}
% However, noticing that $[[castup [t] e]]$ or $[[castdown e]]$ have
%different types from $[[e]]$, types of terms are not preserved if directly removing cast constructs.
%This breaks the subject reduction theorem, and consequently type-safety.

%Therefore, we stick to the semantics of iso-recursive types for \cast
%operators which has type preservation. This way explicit casts and
%recursion can genuinely be seen as a generalization of recursive
%types (see Section \ref{sec:ecore:type}). Furthermore, we believe that all \cast operators can be
%eliminated through type erasure, when generating code,
%to address the potential performance issue of code generation.

An important remark is that casts are necessary, not only for controlling the unrolling of recursive types,
but also for type conversion of other constructs, which is essential
for encoding parametrized algebraic datatypes (see Section \ref{sec:fun}).
Also, the ``type-in-type'' axiom~\cite{typeintype} makes it possible
to encode fixpoints even without a fixpoint primitive, i.e., the $\mu$-operator.
Thus if no casts would be performed on terms without recursive types,
it would still be possible to build a term with a non-terminating type and
make type-checking non-terminating.

\subsection{Metatheory}\label{sec:ecore:meta}
We now discuss the metatheory of \ecore. We focus on two properties:
the decidability of type checking and the type safety of the
language. First, we show that type checking \ecore is decidable
without requiring strong normalization. Second, the language is type-safe, 
proven by subject reduction and progress theorems.

\paragraph{Decidability of Type Checking}
The proof for decidability of type checking is by induction on the
structure of $[[e]]$. The non-trivial case
is for $\mathsf{cast}$-terms with typing rules \ruleref{T\_CastUp} and
\ruleref{T\_CastDown}. Both rules contain a premise that needs to judge if
two types $[[t1]]$ and $[[t2]]$ follow the one-step reduction, i.e.,
if $[[t1 --> t2]]$ holds. We show that $[[t2]]$ is
\emph{unique} with respect to the one-step reduction, or equivalently,
reducing $[[t1]]$ by one step will get only a sole result
$[[t2]]$. Such property is given by the following lemma:

\begin{lemma}[Determinacy of One-step Weak-head
  Reduction]\label{lem:ecore:unique}
% \verb|[reduct_determ]|
	If $[[e --> e1]]$ and $[[e --> e2]]$, then $[[e1 == e2]]$.
\end{lemma}

We use the notation $[[==]]$ to denote the \emph{alpha}
equivalence of $[[e1]]$ and $[[e2]]$.
Note that the presence of recursion does not affect this lemma: 
given a recursive term
$[[mu x:t.e]]$, by rule \ruleref{S\_Mu}, there always exists a unique
term $[[e'==e[x|->mu x:t.e] ]]$ such that $[[mu x:t.e -->
    e']]$.
With this result, we show it is decidable to check whether
the one-step relation $[[t1 --> t2]]$ holds. We first
reduce $[[t1]]$ by one step to obtain $[[t1']]$ (which is
unique by Lemma \ref{lem:ecore:unique}), and compare if $[[t1']]$ and
$[[t2]]$ are syntactically equal. Thus, we can further show type
checking $\mathsf{cast}$-terms is decidable.

For other forms of terms, the typing rules only contain typing
judgments in the premises. Thus, type checking is decidable by the
induction hypothesis and the following lemma which ensures the typing
result is unique:
\begin{lemma}[Uniqueness of Typing for \ecore]
% \verb|[typing_unique]|
If $[[G |- e : t1]]$ and $[[G |- e : t2]]$, then $[[t1 == t2]]$.
\end{lemma}
Thus, we can conclude the decidability of type
checking:
\begin{theorem}[Decidability of Type Checking for \ecore]\label{lem:ecore:decide}
% \verb|[typing_decidable]|
Given a well-formed context $[[G]]$ and a term $[[e]]$, it is decidable
to determine if there exists $[[t]]$ such that $[[G |- e : t]]$.
\end{theorem}

We emphasize that when proving the decidability of type checking, we
do not rely on strong normalization. Intuitively,
explicit type conversion rules use one-step weak-head reduction, which already
has a decidable checking algorithm according to Lemma
\ref{lem:ecore:unique}. We do not need to further require the
normalization of terms. This is different from the proof for \cc which
requires the language to be strongly normalizing
\cite{pts:normalize}. In \cc the conversion rule needs to examine
the beta equivalence of terms, which is decidable only if every term
has a normal form.

\paragraph{Type Safety}
The proof of the type safety of \ecore is by showing subject reduction
and progress theorems:

\begin{theorem}[Subject Reduction of \ecore]\label{lem:ecore:reduct}
% \verb|[subject_reduction_result]|
  If $[[G |- e:T]]$ and $[[e]] [[-->]] e'$ then $[[G |- e':T]]$.
\end{theorem}

\begin{theorem}[Progress of \ecore]\label{lem:ecore:prog}
% \verb|[progress_result]|
  If $[[empty |- e:T]]$ then either $[[e]]$ is a value $v$ or there
  exists $[[e]]'$ such that $[[e --> e']]$.
\end{theorem}

The proof of subject reduction is straightforward by induction on the
derivation of $[[e]] [[-->]] e'$. Some cases need supporting lemmas:
\ruleref{S\_CastElim} requires Lemma
\ref{lem:ecore:unique}; \ruleref{S\_Beta} and \ruleref{S\_Mu} require
the following substitution lemma:

\begin{lemma}[Substitution of \ecore]\label{lem:ecore:subst}
% \verb|[typing_substitution]|
  If $[[G1, x:T, G2 |- e1:t]]$ and $[[G1 |- e2:T]]$, then
  $[[G1, G2 [x |-> e2] |- e1[x |-> e2] : t[x |-> e2] ]]$.
\end{lemma}

The proof of progress is also standard by induction on
$[[empty |- e : T]]$. Notice that $[[castup [t] v]]$ is a value, while
$[[castdown e_1]]$ is not: by rule \ruleref{S\_CastDown}, $e_1$ will
be constantly reduced until it becomes a value that could only be in
the form $[[castup [t] v]]$ by typing rule \ruleref{T\_CastDown}. Then
rule \ruleref{S\_CastElim} can be further applied and the evaluation
does not get stuck. Another notable remark is that
when proving the case for application $[[e1 e2]]$, if $[[e1]]$ is a
value, it could only be a $\lambda$-term but not a
$[[castup]]$-term. Otherwise, suppose $[[e1]]$ has the form
$[[castup [Pi x:t1.t2] e1']]$. By inversion, we have
$[[empty |- e1' : t1']]$ and $[[Pi x:t1.t2 --> t1']]$. But such
$[[t1']]$ does not exist because $[[Pi x:t1.t2]]$ is a value which is
not reducible.

