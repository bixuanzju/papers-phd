\section{Introduction}

\begin{comment}
Modern statically typed functional languages (such as ML, Haskell,
Scala or OCaml) have increasingly expressive type systems. Often these
large source languages are translated into a much smaller typed core
language. The choice of the core language is essential to ensure that
all the features of the source language can be encoded. For a simple
polymorphic functional language it is possible to pick a
variant of System $F$~\cite{systemfw,Reynolds:1974} as a core
language. However, the desire for more expressive type system features
puts pressure on the core languages, often requiring them to be
extended to support new features.  For example, if the source language
supports \emph{higher-kinded types} or \emph{type-level functions}
then System $F$ is not expressive enough and can no longer be used as
the core language. Instead another core language that does provide
support for higher-kinded types, such as
System~$F_{\omega}$~\cite{systemfw}, needs to be used. Of course the
drive to add more and more advanced type-level features means that
eventually the core language needs to be extended again. Indeed modern
functional languages like Haskell use specially crafted core
languages, such as System $F_{C}$~\cite{fc}, that provide support for all
modern features of Haskell. Although \emph{extensions} of System
$F_{C}$~\cite{fc:pro,Eisenberg:2014} satisfy the current needs of
modern Haskell, it is very likely to be extended again in the
future~\cite{fc:kind}. Moreover System $F_{C}$ has grown to be a relatively
large and complex language, with multiple syntactic levels, and dozens
of language constructs.
\end{comment}

\begin{comment}
However System~$F_{\omega}$ is
significantly more complex than System F and thus harder to
maintain. If later a new feature, such as \emph{kind polymorphism}, is
desired the core language may need to be changed again to account for
the new feature, introducing at the same time new sources of
complexity. Indeed the core language for modern versions of 
functional languages are quite complex, having multiple syntactic 
sorts (such as terms, types and kinds), as well as dozens of 
language constructs~\cite{}\bruno{$F_{C}$}. 
\end{comment}

We live exciting times in the design of functional programming
languages. In recent years, dependent types have inspired novel
designs for new programming languages, such as Agda~\cite{agda} or
Idris~\cite{idris}, as well as numerous programming language research~\cite{cayenne,dep:pisigma,sjoberg:msfp12,guru,fc:kind,zombie:popl14,zombie:popl15}.
Dependently typed languages bring additional expressiveness to type
systems, and they can also support different forms of assurances, such 
as strong normalization and logical consistency, not
typically present in traditional programming languages.
\begin{comment}
 For example,
many dependently typed languages are strongly normalizing and can
ensure termination of programs. Moreover they are typically also
logically consistent, ensuring that programs can be interpreted as
proofs. This is in contrast with traditional functional languages like
Haskell or ML, which have no such additional assurances.
\end{comment}
Nevertheless, traditional designs for functional languages still have
some benefits. While strong normalization and logical consistency are
certainly nice properties to have, and can be valuable to have in many
domains, they can also impose restrictions on how programs are
written. For example, the termination checking algorithms typically employed by
dependently typed languages such as Agda or Idris can only automatically ensure termination
of programs that follow certain patterns.
% Thus, for many programs, it
%may not be possible for the termination checker to automatically establish
%termination and the program is rejected; or, alternatively, 
%a manual termination proof has to be provided. Moreover, some programs are
%non-terminating by nature (for example, an operating system).
%While there is research on how to write such programs in strongly
%normalizing languages (see for example \cite{}), those programs are
%not yet written as directly as in traditional functional languages. 
In contrast Haskell or ML programmers can write their programs much more
freely, since they do not need to worry about retaining strong
normalization and logical consistency. Thus there is still plenty of space
for both types of designs.

From an implementation and foundational point-of-view, dependently
typed languages and traditional functional languages also have
important differences. Languages like Haskell or ML have a strong
separation between terms and types (and also kinds). This separation
often leads to duplication of constructs. For example, when the type
language provides some sort of type level computation, constructs such
as type application (mimicking value level application) may be needed.
In contrast many dependently typed languages unify types and
terms. There are benefits in unifying types and terms.  In
addition to the extra expressiveness afforded, for example, by
dependent types, only one syntactic level is needed. Thus
duplication can be avoided. Having less language constructs
simplifies the language, making
it easier to study (from the meta-theoretical point of view) and maintain
(from the implementation point of view). 

In principle having unified syntax would be beneficial even for more
traditional designs of functional languages, which have no strong
normalization or logical consistency. Not surprisingly, researchers
have in the past considered such an option for implementing functional
languages~\cite{cayenne, typeintype, pts:henk}, by using 
some variant of  \emph{pure type systems} (PTS)~\cite{handbook}
(normally extended with general recursion). Thus, with a simple and tiny 
calculus, they showed that powerful and quite expressive functional 
languages could be built with unified syntax.

However having unified syntax for types and terms brings
challenges. One pressing problem is that integrating (unrestricted)
general recursion in dependently typed calculi with unified syntax,
while retaining logical consistency, strong normalization and
\emph{decidable type-checking} is difficult. Indeed, many early
designs using unified syntax and unrestricted general
recursion~\cite{cayenne, typeintype} lose all three properties. 
For pragmatic reasons languages like Agda or Idris also allow 
turning off the termination checker, which allows for added 
expressiveness, but loses the three properties as well.
More recently, various
researchers~\cite{zombie:popl14,zombie:popl15,Swamy2011} have been
investigating how to combine those properties, general recursion and
dependent types. However, this is usually done by having the type
system carefully control the total and partial parts of computation,
adding significant complexity to those calculi when compared to
systems based on pure type systems.

\begin{comment}
 In
particular, dependently typed calculi typically support type-level
computation, but if general recursion is allowed during such
computation non-terminating programs will make the type-checking
procedure non-terminating as well. 
\end{comment}

Nevertheless, if we are interested in traditional languages, only the
loss of decidable type-checking is problematic. Unlike strong
normalization and logical consistency, decidable type-checking is
normally one property that is expected from a traditional programming
language design.  

This paper proposes \name: a simple call-by-name
variant of the calculus of constructions. The key challenge solved in
this work is how to define a calculus comparable in simplicity to the
calculus of constructions, while featuring both general recursion and
decidable type checking. The main idea, is to recover
decidable type-checking by making each type-level computation step
explicit. In essence, each type-level reduction or expansion is
controlled by a \emph{type-safe} cast. Since single computation steps
are trivially terminating, decidability of type checking is possible
even in the presence of non-terminating programs at the type level.
At the same time term-level programs using general recursion work as
in any conventional functional languages, and can be
non-terminating.

Our design generalizes \emph{iso-recursive
  types}~\cite{tapl}, which are our main source of inspiration. 
In \name, not only folding/unfolding of
recursion at the type level is explicitly controlled by term level
constructs, but also any other type level computation (including beta
reduction/expansion). There is an analogy to language designs with
\emph{equi-recursive} types and \emph{iso-recursive} types, which are
normally the two options for adding recursive types to languages. With
equi-recursive types, type-level recursion is implicitly
folded/unfolded, which makes establishing decidability of
type-checking much more difficult. In iso-recursive designs, the idea is to 
trade some convenience by a simple way to ensure decidability. 
Similarly, we view the design of traditional dependently typed
calculi, such as the calculus of constructions as analogous 
to systems with equi-recursive types. In the calculus of constructions 
it is the \emph{conversion rule} that allows type-level computation to 
by implicitly triggered. However, the proof of decidability
of type checking for the \emph{calculus of constructions}~\cite{coc}
(and other normalizing PTS) is non-trivial, as it depends on strong normalization
~\cite{pts:normalize}. Moreover decidability is lost when adding
general recursion. In contrast, the cast operators in \name have to be used 
to \emph{explicitly} trigger each step of type-level computation, but 
it is easy to ensure decidable type-checking, even with general recursion. 

\begin{comment}
Usually type systems based on PTS have a conversion rule
to support type-level computation.  In such type systems ensuring the
\emph{decidability} of type checking requires type-level computation
to terminate. When the syntax of types and terms is the same, the
decidability of type checking is usually dependent on the strong
normalization of the calculus. An example is the proof of decidability
of type checking for the \emph{calculus of constructions}~\cite{coc}
(and other normalizing PTS), which depends on strong normalization
~\cite{pts:normalize}. Modern dependently
typed languages such as Idris~\cite{idris} and Agda~\cite{agda}, which are also
built on a unified syntax for types and terms, require strong
normalization as well: all recursive programs must pass a termination
checker.  An unfortunate consequence of coupling
decidability of type checking and strong normalization is that adding
(unrestricted) general recursion to such calculi is difficult. Indeed
past work on using a simple PTS-like calculi to model functional languages
with unrestricted general recursion, had to give up on \emph{decidability of
type-checking}~\cite{cayenne, typeintype}.
%There
%is a clear tension between decidability of type checking and allowing
%general recursion in calculi with unified syntax.



This paper proposes \name: a simple yet expressive call-by-name
variant of the calculus of constructions, which has a fraction of the
language constructs of existing core languages. The key challenge
solved in this work is how to define a calculus comparable in
simplicity to the calculus of constructions, while featuring both
general recursion and decidable type checking. The main idea, 
inspired by the traditional treatment of \emph{iso-recursive
  types}~\cite{tapl}, is to recover decidable type-checking by making each
type-level computation step explicit, i.e., each beta reduction or
expansion at the type level is controlled by a \emph{type-safe}
cast. Since single computation steps are trivially terminating, decidability
of type checking is possible even in the presence of non-terminating
programs at the type level.  At the same time term-level programs
using general recursion work as in any conventional functional
languages, and can even be non-terminating.
\end{comment}

\begin{comment}
For example, if a type-level program requires two beta reductions to
reach normal form, then two casts are needed in the program. If a
non-terminating program is used at the type level, it is not possible
to cause non-termination in the type checker, because that would
require a program with an infinite number of casts. Therefore, since
single beta-steps are trivially terminating, decidability of type
checking is possible even in the presence of non-terminating programs
at the type level.  At the same time term-level programs using general
recursion work as in any conventional functional languages, and can
even be non-terminating.
\end{comment}

We study two variants of the calculus that differ on the reduction
strategy employed by the cast operators, and give different trade-offs
in terms of simplicity and expressiveness.  The first variant \ecore uses
weak-head reduction in the cast operators.  This allows for a very
simple calculus, but loses some expressiveness in terms of type level
computation. Nevertheless in this variant it is still possible to
encode useful language constructs such as algebraic datatypes.  
The second variant \namef uses \emph{parallel
  reduction} for casts and is more expressive.
% \linus{I removed the
%   text about Zombie. I think only the approach for proving metatheory is
%   inspired, but perhaps not our design?}
It allows equating terms such as $\mathit{Vec}~(1+1)$ and $\mathit{Vec}~2$
as equal.  The price to pay for this more expressive design is some
additional complexity. For both designs type soundness and
decidability of type-checking are proved.


%We also briefly illustrate a prototype implementation of a simple
%functional language build on top of \name. 

\begin{comment}
To show the expressiveness and applicability of \name and the cast
operators, we develop a simple polymorphic functional language with
algebraic datatypes and some interesting type-level features. While
casts in \name do sacrifice convinience and some expressiveness to
gain the ability of doing arbitrary general recursion at the term
level, they are still powerful enough to express common features
available in traditional functional languages. These features include:
\emph{records}, \emph{higher-kinded types}, \emph{nested
  datatypes}~\cite{nesteddt}, \emph{kind polymorphism}~\cite{fc:pro},
and some \emph{simple forms of dependent types}. Casts are used to in
the implementation of records and algebraic datatypes. The additional
power of generalized casts is needed for encoding \emph{parametrized}
algebraic datatypes and records, since parametrization relies on
\emph{type abstraction}. We do not attempt to encode some non-trivial
type-level features available in modern languages like Haskell, and
also dependently typed languages. However we do present a discussion,
in Section~\ref{}, on how \name could be extended to support more expressiveness.
\end{comment}

It is worth emphasizing that \name does sacrifice some convenience
when performing type-level computations in order to gain the ability
of doing arbitrary general recursion at the term level. The goal of
this work is to show the benefits of unified syntax in terms of
economy of concepts for programming language design, and not use
unified syntax to express computationally intensive type-level
programs.  Investigating how to express computationally intensive
type-level programs (as in dependently typed programming) in \name is left for future work.


\begin{comment}
Our motivation to develop \name is to use it as a simpler alternative
to existing core languages for functional programming. We focus on traditional
functional languages like ML or Haskell extended with many interesting
type-level features, but perhaps not the \emph{full power} of
dependent types.  The paper shows how many of programming language
features of Haskell, including some of the latest extensions, can be
encoded in \name via a surface language. The surface
language supports \emph{algebraic datatypes}, \emph{higher-kinded
  types}, \emph{nested datatypes}~\cite{nesteddt}, \emph{kind
  polymorphism}~\cite{fc:pro} and \emph{datatype
  promotion}~\cite{fc:pro}.  This result is interesting because \name
is a minimal calculus with only 8 language constructs and a single
syntactic sort. In contrast the latest versions of System $F_{C}$
(Haskell's core language) have multiple syntactic sorts and dozens of
language constructs.
%Even if support for equality and
%coercions, which constitutes a significant part of System $F_{C}$,
%would be removed the resulting language would still be significantly
%larger and more complex than \name.

A non-goal of the current work (although a worthy avenue for future
work) is to use \name as a core language for modern dependently typed
languages like Agda or Idris. In contrast to \name, those languages
use a more powerful notion of equality. In particular \name
currently lacks full-reduction and it is unable to exploit injectivity 
properties when comparing two types for equality. Moreover,
\name (and also System $F_{C}$) lack \emph{logical consistency}:
that is ensuring the soundness of proofs written as programs.
This is in contrast to dependently typed languages, where logical
consistency is typically ensured.
Various researchers~\cite{zombie:popl14,zombie:thesis,Swamy2011} have been investigating how to combine logical
consistency, general recursion and dependent types. However, this is
usually done by having the type system carefully control the total and
partial parts of computation, making those calculi significantly more
complex than \name or the calculus of constructions. In
\name, logical consistency is traded by the simplicity of the system.
\end{comment}

\begin{comment}
In particular
the treatment of type-level computation in \name shares similar ideas
with Haskell. Although Haskell's surface language provides a rich set
of mechanisms to do type-level computation~\cite{}, the core language
lacks fundamental mechanisms todo type-level computation. Type
equality in System $F_{C}$ is, like in \name, purely syntactic (modulo
alpha-conversion).
\end{comment}

\begin{comment}
 and there is no type-level
abstraction. In other words in Haskell, mechanisms such as type
classes and type families

Although it may seem that forcing each step of computation 
at the type-level to be explicit will prevent convinient use of 
type-level computation.

Point about the treatment of type-level computation in Haskell. Haskell's
core language has type applications, but no type-level lambda. Equality 
is syntactic modulo alpha-conversion. This design choice was rooted in the 
desire to support Hindley-Milner type-inference... 
\end{comment}

In summary, the contributions of this work are:

\begin{itemize}
\item {\bf The \name calculus:} A simple calculus for functional programming, that collapses terms, types and
  kinds into the same hierarchy and supports general recursion. \name
  is type-safe and the type system is decidable. Full proofs are provided in the extended version of this paper~\cite{full}.

\item {\bf Iso-types:} \name 
 generalizes iso-recursive types by making all type-level computation
 steps explicit via \emph{casts operators}. In \name the combination of
 casts and recursion subsumes iso-recursive types.

%\item {\bf An example surface language}, built on top of \name,
%  that supports datatypes, pattern matching and some advanced
%  type-level features. The type safety of the type-directed
%  translation to \name is proved.\bruno{should we have some paragraphs
%  about the surface language?}

\item {\bf A prototype implementation:} The prototype of \name is
  available online\footnote{\fullurl}.
\end{itemize}

\begin{comment}
\begin{enumerate}[a)]
\item Motivations:

\begin{itemize}

\item Because of the reluctance to introduce dependent
  types\footnote{This might be changed in the near future. See
    \url{https://ghc.haskell.org/trac/ghc/wiki/DependentHaskell/Phase1}.},
  the current intermediate language of Haskell, namely System $F_C$
  \cite{fc}, separates expressions as terms, types and kinds, which
  brings complexity to the implementation as well as further
  extensions \cite{fc:pro,fc:kind}.

\item Popular full-spectrum dependently typed languages, like Agda,
  Coq, Idris, have to ensure the termination of functions for the
  decidability of proofs. No general recursion and the limitation of
  enforcing termination checking make such languages impractical for
  general-purpose programming.

\item We would like to introduce a simple and compiler-friendly
  dependently typed core language with only one hierarchy, which
  supports general recursion at the same time.

\end{itemize}

\item Contribution:

\begin{itemize}

\item A core language based on Calculus of Constructions (CoC) that
  collapses terms, types and kinds into the same hierarchy.

\item General recursion by introducing recursive types for both terms
  and types by the same $\mu$ primitive.

\item Decidable type checking and managed type-level computation by
  replacing implicit conversion rule of CoC with generalized
  \textsf{fold}/\textsf{unfold} semantics.

\item First-class equality by coercion, which is used for encoding
  GADTs or newtypes without runtime overhead.

\item Surface language that supports datatypes, pattern matching and
  other language extensions for Haskell, and can be encoded into the
  core language.

\end{itemize}


\end{enumerate}
\end{comment}
