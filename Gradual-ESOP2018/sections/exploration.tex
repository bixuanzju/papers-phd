\section{Revisiting Consistent Subtyping}
\label{sec:exploration}

In this section we explore the design space of consistent subtyping.
% Now
% metavariables $A,B$ in \cref{fig:original-typing} also range over the unknown type
% $\unknown$.
We start with the definitions of consistency and subtyping for
polymorphic types, and compare with some relevant work.
% (in particular the compatibility relation by \citet{ahmed2011blame} and type
% consistency by \citet{yuu2017poly}).
We then discuss the design decisions
involved towards our new definition of consistent subtyping, and justify the new
definition by demonstrating its equivalence with that of \citet{siek2007gradual}
and the AGT approach~\citep{garcia2016abstracting} on simple types.

% \begin{figure}[t]
%   \centering
%   \begin{small}
% \begin{tabular}{lrcl} \toprule
%   % Expressions & $e$ & \syndef & $x \mid n \mid
%   %                        \blam x A e \mid \erlam x e \mid e~e $ \\
% %%                         \mid \erlam x e \equiv \blam x \unknown e $ \\

%   Types & $A, B$ & \syndef & $ \nat \mid a \mid A \to B \mid \forall a. A \mid \unknown$ \\
%   Monotypes & $\tau, \sigma$ & \syndef & $ \nat \mid a \mid \tau \to \sigma$ \\

%   Contexts & $\dctx$ & \syndef & $\ctxinit \mid \dctx,x: A \mid \dctx, a$ \\
%   % Syntactic Sugar & $e : A$ & $\equiv$ & $(\blam x A x) ~ e$ \\
%   \bottomrule
% \end{tabular}
%   \end{small}
% \caption{Polymorphic types added by $\unknown$.}
% \label{fig:decl-types}
% \end{figure}

% \paragraph{Types}
The syntax of types is given at the top of
\cref{fig:decl:subtyping}.
% Meta-variable $e$ ranges over expressions.
% Expressions are either variables $x$, integers $n$, annotated lambda
% abstractions $\blam x A e$, un-annotated lambda abstractions $\erlam x e$ or
% applications $e_1 ~ e_2$.
We write $A$, $B$ for types. Types are either the integer type $\nat$, type
variables $a$, functions types $A \to B$, universal quantification $\forall a.
A$, or the unknown type $\unknown$. Though we only have one base type $\nat$, we
also use $\bool$ for the purpose of illustration. Note that monotypes $\tau$ contain all
types other than the universal quantifier and the unknown type $\unknown$. % Note that
% $\unknown$ is only added to the category of polymorphic types $A, B$.
We will discuss this restriction when we present the subtyping rules.
Contexts $\dctx$ are \textit{ordered} lists of type variable declarations and term variables.



\subsection{Consistency and Subtyping}
\label{subsec:consistency-subtyping}

We start by giving the definitions of consistency and subtyping for polymorphic
types, and comparing our definitions with the compatibility relation by
\citet{ahmed2011blame} and type consistency by \citet{yuu2017poly}.

\begin{figure}[t]
  \centering
  \begin{small}
\begin{tabular}{lrcl} \toprule
  % Expressions & $e$ & \syndef & $x \mid n \mid
  %                        \blam x A e \mid \erlam x e \mid e~e $ \\
%%                         \mid \erlam x e \equiv \blam x \unknown e $ \\

  Types & $A, B$ & \syndef & $ \nat \mid a \mid A \to B \mid \forall a. A \mid \unknown$ \\
  Monotypes & $\tau, \sigma$ & \syndef & $ \nat \mid a \mid \tau \to \sigma$ \\

  Contexts & $\dctx$ & \syndef & $\ctxinit \mid \dctx,x: A \mid \dctx, a$ \\
  % Syntactic Sugar & $e : A$ & $\equiv$ & $(\blam x A x) ~ e$ \\
  \bottomrule
\end{tabular}
  \begin{mathpar}
    \framebox{$A \sim B$} \\
    \CD \and \CA \and \CB \and \CC \and \CE
  \end{mathpar}
  % \begin{mathpar}
  %   \framebox{$\dctx \bywf A $} \\
  %   \DeclVarWF \and \DeclIntWF \and \DeclUnknownWF \\ \DeclFunWF \and \DeclForallWF
  % \end{mathpar}
  \begin{mathpar}
    \framebox{$\tpresub A \tsub B$} \\
    \HSForallR \quad \HSForallL  \quad \HSTVar \and
    \HSInt \and \HSFun  \and \HSUnknown
  \end{mathpar}
  \end{small}
  \caption{Syntax of types, consistency, and subtyping in the declarative system.}
  \label{fig:decl:subtyping}
\end{figure}

\paragraph{Consistency.}
The key observation here is that consistency is mostly a structural relation,
except that the unknown type $\unknown$ can be regarded as any type. Following
this observation, we naturally extend the definition from
\cref{fig:objects} with polymorphic types, as shown at the middle of
\cref{fig:decl:subtyping}. In particular a polymorphic type $\forall a. A$
is consistent with another polymorphic type $\forall a. B$ if $A$ is consistent
with $B$.

\paragraph{Subtyping.}

We express the fact that one type is a polymorphic generalization of another by
means of the subtyping judgment $\Psi \vdash A \tsub B$. Compared with the
subtyping rules of \citet{odersky1996putting} in
\cref{fig:original-typing}, the only addition is the neutral subtyping of
$\unknown$. Notice
that, in the rule
\rul{S-ForallL}, the universal quantifier is only allowed to be instantiated
with a \emph{monotype}.
The judgment $\dctx \bywf \tau$ checks all the type variables in $\tau$ are
bound in the context $\dctx$. For space reasons, we omit the definition.
According to the syntax in \cref{fig:decl:subtyping},
monotypes do not include the unknown type $\unknown$. This is because if we were
to allow the unknown type to be used for instantiation, we could have  $  \forall a . a \to a \tsub \unknown \to \unknown $
by instantiating $a$ with $\unknown$. Since $\unknown \to \unknown$ is
consistent with any functions $A \to B$, for instance, $\nat \to \bool$, this
means that we could provide an expression of type $\forall a. a \to a$ to a
function where the input type is supposed to be $\nat \to \bool$. However, as we
might expect, $\forall a. a \to a$ is definitely not compatible with $\nat \to
\bool$. This does not hold in any polymorphic type systems without gradual
typing. So the gradual type system should not accept it either. (This is the
so-called \textit{conservative extension} property that will be made precise in
\cref{sec:criteria}.)

Importantly there is a subtle but crucial distinction between a type variable
and the unknown type, although they all represent a kind of ``arbitrary'' type.
The unknown type stands for the absence of type information: it could be
\textit{any type} at \textit{any instance}. Therefore, the unknown type is
consistent with any type, and additional type-checks have to be performed at
runtime. On the other hand, a type variable indicates \textit{parametricity}.
% and is subject to global constraints
In other words, a
type variable can only be instantiated to a single type. For example, in the
type $\forall a. a \to a$, the two occurrences of $a$ represent an arbitrary but
single type (e.g., $\nat \to \nat$, $\bool \to \bool$), while $\unknown \to
\unknown$ could be an arbitrary function (e.g., $\nat \to \bool$) at runtime.

\paragraph{Comparison with Other Relations.}

In other polymorphic gradual calculi, consistency and subtyping are often mixed
up to some extent. In \pbc~\citep{ahmed2011blame}, the compatibility relation for polymorphic types
is defined as follows:
\begin{mathpar}
  \CompAllR \and \CompAllL
\end{mathpar}
Notice that, in rule \rul{Comp-AllL}, the universal quantifier is \textit{always}
instantiated to $\unknown$. However, this way, \pbc allows $\forall a. a \to a
\pbccons \nat \to \bool$, which as we discussed before might not be what we
expect. Indeed \pbc relies on sophisticated runtime checks to rule out such
instances of the compatibility relation a posteriori.

\citet{yuu2017poly} introduced the so-called
\textit{quasi-polymorphic} types for types that may be used where a
$\forall$-type is expected, which is important for their purpose of
conservativity over System F. Their type consistency relation, involving polymorphism, is
defined as follows\footnote{This is a simplified version.}:
\begin{mathpar}
  \inferrule{A \sim B }{\forall a. A \sim \forall a. B}
  \and
  \inferrule{A \sim B \\ B \neq \forall a. B' \\ \unknown \in \mathsf{Types}(B)}
  {\forall a. A \sim B}
\end{mathpar}
Compared with our consistency definition in \cref{fig:decl:subtyping},
their first rule is the same as ours. The second rule says that a non
$\forall$-type can be consistent with a $\forall$-type only if it contains
$\unknown$. In this way, their type system is able to reject $\forall a. a \to a
\sim \nat \to \bool$. However, in order to keep conservativity, they also reject
$\forall a. a \to a \sim \nat \to \nat$, which is perfectly sensible in their
setting (i.e., explicit polymorphism). However with implicit polymorphism, we
would expect $\forall a. a \to a$ to be related with $\nat \to \nat$, since $a$ can be instantiated to
$\nat$.

Nonetheless, when it comes to interactions between dynamically typed and
polymorphically typed terms, both relations allow $\forall a. a \to
\nat$ to be related with $\unknown \to \nat$ for example, which in our view, is some sort of
(implicit) polymorphic subtyping combined with type consistency, and
that should be derivable by the more primitive notions in the type system
(instead of inventing new relations). One of our design principles is that
subtyping and consistency is \textit{orthogonal}, and can be naturally
superimposed, echoing the same opinion of \citet{siek2007gradual}.

\subsection{Towards Consistent Subtyping}
\label{subsec:towards-conssub}

With the definitions of consistency and subtyping, the question now is how to
compose these two relations so that two types can be compared in a way that takes
these two relations into account.

Unfortunately, the original definition of \citeauthor{siek2007gradual}
(\cref{def:old-decl-conssub}) does not work well with our definitions of
consistency and subtyping for polymorphic types. Consider two types: $(\forall
a. a \to \nat) \to \nat$, and $(\unknown \to \nat) \to \nat$. The first type can only reach the
second type in one way (first by applying consistency, then subtyping), but not the
other way, as shown in \cref{fig:example:a}. We use $\bot$ to mean that we
cannot find such a type. Similarly, there are situations where the first type
can only reach the second type by the other way (first applying
subtyping, and then
consistency), as shown in \cref{fig:example:b}.

\begin{figure}[t]
  \begin{subfigure}[b]{.4\linewidth}
    \centering
      \begin{tikzpicture}
        \matrix (m) [matrix of math nodes,row sep=2em,column sep=4em,minimum width=2em]
        {
          \bot & (\unknown \to \nat) \to \nat \\
          (\forall a. a \to \nat) \to \nat & (\forall a. \unknown \to \nat) \to \nat \\};

        \path[-stealth]
        (m-2-1) edge node [left] {$\tsub$} (m-1-1)
        (m-2-2) edge node [left] {$\tsub$} (m-1-2);

        \draw
        (m-1-1) edge node [above] {$\sim$} (m-1-2)
        (m-2-1) edge node [below] {$\sim$} (m-2-2);
      \end{tikzpicture}
      \caption{}
      \label{fig:example:a}
  \end{subfigure}
  \begin{subfigure}[b]{.4\linewidth}
    \centering
    \begin{tikzpicture}
      \matrix (m) [matrix of math nodes,row sep=2em,column sep=4em,minimum width=2em]
      {
        \nat \to \nat & \nat \to \unknown \\
        \forall a. a & \bot \\};

      \path[-stealth]
      (m-2-1) edge node [left] {$\tsub$} (m-1-1)
      (m-2-2) edge node [left] {$\tsub$} (m-1-2);

      \draw
      (m-1-1) edge node [above] {$\sim$} (m-1-2)
      (m-2-1) edge node [below] {$\sim$} (m-2-2);
    \end{tikzpicture}
    \caption{}
    \label{fig:example:b}
  \end{subfigure}
  \begin{subfigure}[b]{.8\linewidth}
    \centering
    \begin{tikzpicture}
      \matrix (m) [matrix of math nodes,row sep=2em,column sep=1em,minimum width=2em]
      {
        \bot &
        (((\unknown \to \nat)\to \nat) \to \bool) \to (\nat \to \unknown)  \\
        (((\forall a. a \to \nat) \to \nat) \to \bool) \to (\forall a. a) &
        \bot \\};

      \path[-stealth]
      (m-2-1) edge node [left] {$\tsub$} (m-1-1)
      (m-2-2) edge node [left] {$\tsub$} (m-1-2);

      \draw
      (m-1-1) edge node [above] {$\sim$} (m-1-2)
      (m-2-1) edge node [below] {$\sim$} (m-2-2);
    \end{tikzpicture}
    \caption{}
    \label{fig:example:c}
  \end{subfigure}
  \caption{Examples that break the original definition of consistent subtyping.}
  \label{fig:example}
\end{figure}

What is worse, if those two examples are composed in a way that those types all
appear co-variantly, then the resulting types cannot reach each other
in either
way. For example, \cref{fig:example:c} shows such two types by putting a
$\bool$ type in the middle, and neither definition of consistent subtyping
works. % But these two types ought to be related somehow!

\paragraph{Observations on Consistent Subtyping Based on Information Propagation.}

In order to develop the correct definition of consistent subtyping for
polymorphic types, we need to understand how consistent subtyping works.
We first review two important properties of subtyping: (1) subtyping induces the
subsumption rule: if $A \tsub B$, then an expression of type $A$ can be used
where $B$ is expected; (2) subtyping is transitive: if $A \tsub B$, and $B \tsub
C$, then $A \tsub C$. Though consistent subtyping takes the unknown type into
consideration, the subsumption rule should also apply: if $A \tconssub B$, then
an expression of type $A$ can also be used where $B$ is expected, given that
there might be some information lost by consistency. A crucial difference from
subtyping is that consistent subtyping is \textit{not} transitive because
information can only be lost once (otherwise, any two types are a consistent
subtype of each other). Now consider a situation where we have both $A \tsub B$,
and $B \tconssub C$, this means that $A$ can be used where $B$ is expected, and
$B$ can be used where $C$ is expected, with possibly some loss of information. In
other words, we should expect that $A$ can be used where $C$ is expected, since
there is at most one-time loss of information.

\begin{observation}
  If $A \tsub B$, and $B \tconssub C$, then $A \tconssub C$.
\end{observation}

This is reflected in \cref{fig:obser:a}. A symmetrical
observation is given in \cref{fig:obser:b}:


\begin{observation}
  If $C \tconssub B$, and $B \tsub A$, then $C \tconssub A$.
\end{observation}

\begin{figure}[t]
  \centering
  \begin{subfigure}[b]{.4\linewidth}
    \centering
    \begin{tikzpicture}
      \matrix (m) [matrix of math nodes,row sep=2.5em,column sep=4em,minimum width=2em]
      {
        T_1 & C \\
        B   & T_2 \\
        A & \\};

      \path[-stealth]
      (m-3-1) edge node [left] {$\tsub$} (m-2-1)
      (m-2-2) edge node [left] {$\tsub$} (m-1-2)
      (m-2-1) edge node [left] {$\tsub$} (m-1-1);

      \draw
      (m-2-1) edge node [below] {$\sim$} (m-2-2)
      (m-1-1) edge node [above] {$\sim$} (m-1-2);

      \draw [dashed, ->]
      (m-2-1) edge node [above] {$\tconssub$} (m-1-2);

      \path [dashed, ->, out=0, in=0, looseness=2]
      (m-3-1) edge node [right] {$\tconssub$} (m-1-2);
    \end{tikzpicture}
    \caption{}
    \label{fig:obser:a}
  \end{subfigure}
  \centering
  \begin{subfigure}[b]{.4\linewidth}
    \centering
    \begin{tikzpicture}
      \matrix (m) [matrix of math nodes,row sep=2.5em,column sep=4em,minimum width=2em]
      {
        & A \\
        T_1 & B \\
        C   & T_2 \\};

      \path[-stealth]
      (m-3-1) edge node [left] {$\tsub$} (m-2-1)
      (m-3-2) edge node [left] {$\tsub$} (m-2-2)
      (m-2-2) edge node [left] {$\tsub$} (m-1-2);

      \draw
      (m-2-1) edge node [above] {$\sim$} (m-2-2)
      (m-3-1) edge node [below] {$\sim$} (m-3-2);

      \draw [dashed, ->]
      (m-3-1) edge node [above] {$\tconssub$} (m-2-2);

      \path [dashed, ->, out=135, in=180, looseness=2]
      (m-3-1) edge node [left] {$\tconssub$} (m-1-2);
    \end{tikzpicture}
    \caption{}
    \label{fig:obser:b}
  \end{subfigure}
  \caption{Observations of consistent subtyping}
  \label{fig:obser}
\end{figure}


From the above observations, we see what the problem is with the original
definition. In \cref{fig:obser:a}, if $B$ can reach $C$ by $T_1$, then by
subtyping transitivity, $A$ can reach $C$ by $T_1$. However, if $B$ can only reach $C$ by
$T_2$, then $A$ cannot reach $C$ through the original definition. A similar
problem is shown in \cref{fig:obser:b}.
% : if $C$ can only reach $B$ by $T_1$, then $C$ cannot reach $A$ through the original definition.

However, it turns out that those two problems can be fixed using the same strategy:
instead of taking one-step subtyping and one-step consistency, our definition of
consistent subtyping allows types to take \textit{one-step subtyping, one-step
consistency, and one more step subtyping}. Specifically, $A \tsub B \sim T_2
\tsub C$ (in \cref{fig:obser:a})
and $C \tsub T_1 \sim B \tsub A$ (in \cref{fig:obser:b}) have the same relation chain: subtyping,
consistency, and subtyping.

\paragraph{Definition of Consistent subtyping.} From the above discussion, we are
ready to modify \cref{def:old-decl-conssub}, and adapt it to our notation:

\begin{definition}[Consistent Subtyping]
  \label{def:decl-conssub}
  \[
    \inferrule{
       \tpresub A \tsub C
       \\ C \sim D
       \\ \tpresub D \tsub B
    }{
      \tpresub A \tconssub B
    }
  \]
  % $\tpresub A \tconssub B$, if and only if $\tpresub A \tsub C$, $C \sim D$, and
  % $\tpresub D \tsub B$ for
  % some $C, D$.
\end{definition}

\noindent With \cref{def:decl-conssub}, \Cref{fig:example:c:fix}
illustrates the correct relation chain for the broken example shown in
\cref{fig:example:c}.
At first sight, \cref{def:decl-conssub}
seems worse than the original: we need to guess \textit{two} types! It turns out
that \cref{def:decl-conssub} is a generalization of
\cref{def:old-decl-conssub}, and they are equivalent in the system of
\citet{siek2007gradual}. However, more generally, \cref{def:decl-conssub}
% We argue that this is a \textit{general} definition of
% consistent subtyping, and
is compatible with polymorphic types.

\begin{figure}[t]
  \centering
  \begin{subfigure}[b]{.4\linewidth}
  \begin{tikzpicture}
    \matrix (m) [matrix of math nodes,row sep=2.5em,column sep=6em,minimum width=2em]
    {
      A_2 &
      A_3
      \\
      A_1
      &
      A_4  \\
      };

    \path[-stealth]
    (m-2-1) edge node [left] {$\tsub$} (m-1-1)
    (m-1-2) edge node [left] {$\tsub$} (m-2-2);
    \path[dashed, ->, out=315, in=225, looseness=0.3]
    (m-2-1) edge node [above] {$\tconssub$} (m-2-2);

    \draw
    (m-1-1) edge node [above] {$\sim$} (m-1-2);
  \end{tikzpicture}
  \end{subfigure}
  \begin{subfigure}[b]{.4\linewidth}
  \begin{align*}
  A_1 &=(((\forall a. a \to \nat) \to \nat) \to \bool) \to (\forall a. a) \\
  A_2 &= ((\forall a. a \to \nat) \to \nat) \to \bool) \to (\nat \to \nat) \\
  A_3 &= ((\forall a. \unknown \to \nat) \to \nat) \to \bool) \to (\nat \to \unknown) \\
  A_4 &= (((\unknown \to \nat) \to \nat) \to \bool) \to (\nat \to \unknown)
  \end{align*}
  \end{subfigure}
  \caption{Example that is fixed by the new definition of consistent subtyping.}
  \label{fig:example:c:fix}
\end{figure}

\begin{proposition}[Generalization of Consistent Subtyping]\leavevmode
  \label{prop:subsumes}
\begin{itemize}
  \item \cref{def:decl-conssub} subsumes
    \cref{def:old-decl-conssub}.
    % In \cref{def:decl-conssub},
    % by choosing $D=B$, we have $A\tsub C$ and $C \sim B$; by choosing $C=A$, we have
    % $A \sim D$, and $D \tsub B$.
  \item \cref{def:old-decl-conssub} is equivalent to
    \cref{def:decl-conssub} in the system of \citeauthor{siek2007gradual}.
    % If $A \tsub C$, $C \sim D$, and $D \tsub
    % B$, by \cref{def:old-decl-conssub},
    % $A \sim C'$, $C' \tsub D$ for some $C'$. By subtyping
    % transitivity, $C' \tsub B$. So $A \tconssub B$ by $A \sim C'$ and $C'
    % \tsub B$.
  \end{itemize}
\end{proposition}


\subsection{Abstracting Gradual Typing}
\label{subsec:agt}

\citet{garcia2016abstracting} presented a new foundation for gradual typing that
they call the \textit{Abstracting Gradual Typing} (AGT) approach. In the AGT
approach, gradual types are interpreted as sets of static types, where static
types refer to types containing no unknown types. In this interpretation,
predicates and functions on static types can then be lifted to apply to gradual
types. Central to their approach is the so-called \textit{concretization}
function. For simple types, a concretization $\gamma$ from gradual types to a
set of static types\footnote{For simplification, we directly regard type
  constructor $\to$ as a set-level operator.} is defined as follows:

\begin{definition}[Concretization]
  \label{def:concret}
  \begin{mathpar}
    \gamma(\nat) = \{\nat\} \and \gamma(A \to B) = \gamma(A) \to \gamma(B) \and
    \gamma(\unknown) = \{\text{All static types}\}
  \end{mathpar}
\end{definition}

Based on the concretization function, subtyping between static types can be
lifted to gradual types, resulting in the consistent subtyping relation:
\begin{definition}[Consistent Subtyping in AGT]
  \label{def:agt-conssub}
  $A \agtconssub B$ if and only if $A_1 \tsub B_1$ for some $A_1 \in \gamma(A)$, $B_1 \in \gamma(B)$.
\end{definition}

Later they proved that this definition of consistent subtyping coincides with
that of \citet{siek2007gradual} (\cref{def:old-decl-conssub}). By
\cref{prop:subsumes}, we can directly conclude that our definition coincides with AGT:

\begin{proposition}[Equivalence to AGT on Simple Types]
  \label{lemma:coincide-agt}
  $A \tconssub B$ iff $A \agtconssub B$.
\end{proposition}

However, AGT does not show how to deal with polymorphism (e.g. the
interpretation of type variables) yet.
% As noted by \citet{garcia2016abstracting} in the conclusion, extending
% AGT to deal with polymorphism remains as an open question.
Still, as noted by \citet{garcia2016abstracting},
it is a promising line of future work for
AGT, and the question remains whether our definition would coincide with it.

Another note related to AGT is that the definition is later adopted by
\citet{castagna2017gradual}, where the static types $A_1, B_1$ in
Definition~\ref{def:agt-conssub} can be algorithmically computed by also
accounting for top and bottom types.

\subsection{Directed Consistency}

\textit{Directed consistency}~\citep{Jafery:2017:SUR:3093333.3009865} is defined
in terms of precision and static subtyping:
\[
  \inferrule{
       A' \lessp A
    \\ A \tsub B
    \\ B' \lessp B
  }{
    A' \tconssub B'
  }
\]
The judgment $A \lessp B$ is read ``$A$ is less precise than
$B$''. In their
setting, precision is defined for type constructors and subtyping for static
types. If we interpret this definition from AGT's point of view, finding a more
precise static type\footnote{The definition of precision of types is given in appendix. % It is a
  % partial order relation with $\unknown$ the least element. It is defined
  % structurally over type construction.
}
has the same effect as concretization. Namely, $A' \lessp A
$ implies $A \in \gamma(A')$ and $B' \lessp B$ implies $B \in \gamma(B')$.
Therefore we consider this definition as AGT-style. From this perspective, this
definition naturally coincides with \cref{def:decl-conssub}.

The value of their definition is that consistent subtyping is derived
compositionally from \textit{static subtyping} and \textit{precision}. These are
two more atomic relations. At first sight, their definition looks very similar
to \cref{def:decl-conssub} (replacing $\lessp$ by $<:$ and $<:$ by $\sim$). Then
a question arises as to \textit{which one is more fundamental}. To answer this,
we need to discuss the relation between consistency and precision.

\paragraph{Relating Consistency and Precision.}

Precision is a partial order (anti-symmetric and transitive), while
consistency is symmetric but not transitive. % One observation
% is that precision cares more about order (over precision), while this order can be mixed under
% the structure in consistency.
Nonetheless, precision and consistency are related by the following proposition:

\begin{proposition}[Consistency and Precision]\leavevmode
  \label{lemma:consistency-precision}
  \begin{itemize}
  \item If $A \sim B$,
    then there exists (static) $C$,
    such that $A \lessp C$,
    and $B \lessp C$.
  \item If for some (static) $C$,
    we have $A \lessp C$,
    and $B \lessp C$,
    then we have $A \sim B$.
  \end{itemize}
\end{proposition}

It may seem that precision is a more atomic relation, since consistency can be
derived from precision. However, recall that consistency is in fact an
equivalence relation lifted from static types to gradual types.
% , while precision
% is a subset relation over concretization of gradual
% types~\citep{garcia2016abstracting}.
Therefore defining consistency independently is straightforward, and it is
theoretically viable to validate the definition of consistency directly. On the
other hand, precision is usually connected with the gradual criteria
\citep{siek2015refined}, and finding a correct partial order that adheres to the
criteria is not always an easy task. For example, \citet{yuu2017poly} argued
that term precision for System $F_G$ is actually nontrivial, leaving the gradual
guarantee of the semantics as a conjecture. Thus precision can be difficult to
extend to more sophisticated type systems, e.g. dependent types.

Still, it is interesting that those two definitions illustrate the
correspondence of different foundations (on simple types): one is defined
directly on gradual types, and the other stems from AGT, which is based on
static subtyping.

\subsection{Consistent Subtyping Without Existentials}

\cref{def:decl-conssub} serves as a fine specification of how consistent
subtyping should behave in general. But it is inherently non-deterministic
because of the two intermediate types $C$ and $D$. As with
\cref{def:old-decl-conssub}, we need a combined relation to directly compare two
types. A natural attempt is to try to extend the restriction operator for
polymorphic types. Unfortunately, as we show below, this does not work. However
it is possible to devise an equivalent inductive definition instead.

\paragraph{Attempt to Extend the Restriction Operator.}
Suppose that we try to extend the restriction operator to account for polymorphic
types. The original restriction operator is structural, meaning that it works
for types of similar structures. But for polymorphic types, two input types
could have different structures due to universal quantifiers, e.g, $\forall a. a
\to \nat$ and $(\nat \to \unknown) \to \nat$. If we try to mask the first type
using the second, it seems hard to maintain the information that $a$ should be
instantiated to a function while ensuring that the return type is masked. There
seems to be no satisfactory way to extend the restriction operator in order to
support this kind of non-structural masking.

\paragraph{Interpretation of the Restriction Operator and Consistent Subtyping.}
If the restriction operator cannot be extended naturally, it is useful to
take a step back and revisit what the restriction operator actually does. For
consistent subtyping, two input types could have unknown types in different
positions, but we only care about the known parts. What the restriction
operator does is (1) erase the type information in one type if the corresponding
position in the other type is the unknown type; and (2) compare the resulting types
using the normal subtyping relation. The example below shows the
masking-off procedure for the types $\nat \to \unknown \to \bool$ and $\nat \to
\nat \to \unknown$. Since the known parts have the relation that $\nat \to
\unknown \to \unknown \tsub \nat \to \unknown \to \unknown$, we conclude that
$\nat \to \unknown \to \bool \tconssub \nat \to \nat \to \unknown$.
\begin{center}
  \begin{tikzpicture}
    \tikzstyle{column 5}=[anchor=base west, nodes={font=\tiny}]
    \matrix (m) [matrix of math nodes,row sep=0.5em,column sep=0em,minimum width=2em]
    {
      \nat \to & \unknown & \to & \bool & \mid \nat \to \nat \to \unknown &
      = \nat \to \unknown \to \unknown
      \\
       \nat \to & \nat & \to & \unknown & \mid \nat \to \unknown \to \bool &
      = \nat \to \unknown \to \unknown \\};

    \path[-stealth, ->, out=0, in=0]
    (m-1-6) edge node [right] {$\tsub$} (m-2-6);

    \draw
    (m-1-2.north west) rectangle (m-2-2.south east)
    (m-1-4.north west) rectangle (m-2-4.south east);
  \end{tikzpicture}
\end{center}
Here differences of the types in boxes are erased because of the restriction
operator. Now if we compare the types in boxes directly instead of through the
lens of the restriction operator, we can observe that the \textit{consistent
  subtyping relation always holds between the unknown type and an arbitrary
  type.} We can interpret this observation directly from
\cref{def:decl-conssub}: the unknown type is neutral to subtyping ($\unknown
\tsub \unknown$), the unknown type is consistent with any type ($\unknown \sim
A$), and subtyping is reflexive ($A \tsub A$). Therefore, \textit{the unknown
  type is a consistent subtype of any type ($\unknown \tconssub A$), and vice
  versa ($A \tconssub \unknown$).} Note that this interpretation provides a
general recipe on how to lift a (static) subtyping relation to a (gradual)
consistent subtyping relation, as discussed below.

\paragraph{Defining Consistent Subtyping Directly.}

From the above discussion, we can define the consistent subtyping relation
directly, \textit{without} resorting to subtyping or consistency at all. The key
idea is that we replace $\tsub$ with $\tconssub$ in
\cref{fig:decl:subtyping}, get rid of rule \rul{S-Unknown} and add two
extra rules concerning $\unknown$, resulting in the rules of consistent
subtyping in \cref{fig:decl:conssub}. Of particular interest are the rules
\rul{CS-UnknownL} and \rul{CS-UnknownR}, both of which correspond to what we
just said: the unknown type is a consistent subtype of any type, and vice versa.
\begin{figure}[t]
  \begin{small}
  \begin{mathpar}
    \framebox{$\tpresub A \tconssub B$} \\
    \CSForallR \and \CSForallL \and \CSFun \and
    \CSTVar \and \CSInt \and \CSUnknownL \and \CSUnknownR
  \end{mathpar}
  \end{small}
  \caption{Consistent Subtyping for implicit polymorphism.}
  \label{fig:decl:conssub}
\end{figure}
From now on, we use the symbol $\tconssub$ to refer to the consistent subtyping
relation in \cref{fig:decl:conssub}. What is more, we can prove that those two
are equivalent\footnote{Theorems with $\mathcal{T}$ are those
  proved in Coq. The same applies to $\mathcal{L}$emmas.}:

\begin{ctheorem}   \label{lemma:properties-conssub}
  $\tpreconssub A \tconssub B  \Leftrightarrow \tpresub A \tsub C$, $C \sim D$, $\tpresub D \tsub B$ for some $C, D$.
\end{ctheorem}

% \noindent Not surprisingly, consistent subtyping is reflexive:

% \begin{clemma}[Consistent Subtyping is Reflexive] \label{lemma:sub_refl}%
%   If $\Psi \vdash A$ then $\Psi \vdash A \tconssub A$.
% \end{clemma}





%%% Local Variables:
%%% mode: latex
%%% TeX-master: "../paper"
%%% org-ref-default-bibliography: ../paper.bib
%%% End:
