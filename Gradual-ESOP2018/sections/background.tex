\section{Background and Motivation}
\label{sec:background}

In this section we review a simple gradually typed language with
objects~\citep{siek2007gradual}, to introduce the concept of consistency
subtyping. We also briefly talk about the Odersky-L{\"a}ufer type system for
higher-rank types~\citep{odersky1996putting}, which serves as the original
language on which our gradually typed calculus with implicit 
higher-rank polymorphism is based.


\subsection{Gradual Subtyping}

\begin{figure}[t]
  \begin{small}
  \begin{mathpar}
    \framebox{$A \tsub B$}\\
    \ObSInt \and \ObSBool \and \ObSFloat \and
    \ObSIntFloat \\ \ObFun \and
    \ObSRecord \and \ObSUnknown
  \end{mathpar}
  \begin{mathpar}
    \framebox{$A \sim B$}\\
    \ObCRefl \and \ObCUnknownR \and
    \ObCUnknownL \and \ObCC \and \ObCRecord
  \end{mathpar}

  \end{small}

  \caption{Subtyping and type consistency in \obb}
  \label{fig:objects}
\end{figure}

\citet{siek2007gradual} developed a gradual typed system for object-oriented
languages that they call \obb. Central to gradual typing is the concept of
\textit{consistency} (written $\sim$) between gradual types, which are types
that may involve the unknown type $\unknown$. The intuition is that consistency
relaxes the structure of a type system to tolerate unknown positions in a
gradual type. They also defined the subtyping relation in a way that static type
safety is preserved. Their key insight is that the unknown type $\unknown$ is
neutral to subtyping, with only $\unknown \tsub \unknown$. Both relations are
found in \cref{fig:objects}.

A primary contribution of their work is to show that consistency and subtyping
are orthogonal. To compose subtyping and consistency,
\citeauthor{siek2007gradual} defined \textit{consistent subtyping} (written
$\tconssub$) in two equivalent ways:
\begin{definition}[Consistent Subtyping \`a la \citet{siek2007gradual}]\leavevmode
\label{def:old-decl-conssub}
\begin{itemize}
\item $A \tconssub B$ if and only if $A \sim C$ and $C \tsub B$ for some $C$.
\item $A \tconssub B$ if and only if $A \tsub C$ and $C \sim B$ for some $C$.
\end{itemize}
\end{definition}
Both definitions are non-deterministic because of the intermediate type $C$. To
remove non-determinism, they proposed a so-called \textit{restriction
  operator}, written $\mask A B$ that masks off the parts of a type $A$
that are unknown in a type $B$.
% For example, $\mask {\nat \to \nat} {\bool \to
%   \unknown} = \nat \to \unknown$, and $\mask {\bool \to \unknown} {\nat \to
%   \nat} = {\bool \to \unknown}$.
% The definition of the restriction operator is given below:
\begin{small}
\begin{align*}
   \mask A B = & \kw{case}~A, B~\kw{of}  \mid (-, \unknown) \Rightarrow \unknown\\
               & \mid A_1 \to A_2, B_1 \to B_2  =  \mask {A_1} {B_1} \to \mask {A_2} {B_2} \\
               & \mid [l_1: A_1,...,l_n:A_n], [l_1:B_1,...,l_m:B_m]~\kw{if}~n \leq m \Rightarrow
                [l_1 : \mask {A_1} {B_1}, ..., l_n : \mask {A_n} {B_n}]\\
               & \mid [l_1: A_1,...,l_n:A_n], [l_1:B_1,...,l_m:B_m]~\kw{if}~n > m \Rightarrow\\
               & \qquad [l_1 : \mask {A_1} {B_1}, ..., l_m : \mask {A_m} {B_m},...,l_n:A_n ]\\
               & \mid \kw{otherwise} \Rightarrow A
\end{align*}
\end{small}
With the restriction operator, consistent subtyping is simply defined
as $A \tconssub B \equiv \mask A B \tsub \mask B A$. Then they proved that this
definition is equivalent to \cref{def:old-decl-conssub}.


\subsection{The Odersky-L{\"a}ufer Type System}

\begin{figure}[t]
  \begin{small}
    \centering
    \begin{tabular}{lrcl} \toprule
      Expressions & $e$ & \syndef & $x \mid n \mid
                                    \blam x A e
                                    \mid \erlam x e
                                    \mid e~e$ \\

      Types & $A, B$ & \syndef & $ \nat \mid a \mid A \to B \mid \forall a. A$ \\
      Monotypes & $\tau, \sigma$ & \syndef & $ \nat \mid a \mid \tau \to \sigma$ \\

      Contexts & $\dctx$ & \syndef & $\ctxinit \mid \dctx,x: A \mid \dctx, a$ \\  \bottomrule
    \end{tabular}
  \begin{mathpar}
    \framebox{$\dctx \byhinf e : A$}\\
    \NVar \and \NNat \and \NLamAnnA \and
    \NApp \and \NSub \and \NLam \and
    \NGen
  \end{mathpar}
  \begin{mathpar}
    \framebox{$\dctx \bysub A \tsub B$}\\
    \NTVar \and \NSInt \and \NForallL \and
    \NForallR \and \NFun
  \end{mathpar}

  \end{small}
  \caption{Syntax and static semantics of the Odersky-L{\"a}ufer type system.}
  \label{fig:original-typing}
\end{figure}


The calculus we are combining gradual typing with is the well-established
predicative type
system for higher-rank types proposed by \citet{odersky1996putting}. One
difference is that, for simplicity, we do not account for a let expression, as
there is already existing work about gradual type systems with let expressions
and let generalization (for example, see \citet{garcia2015principal}). Similar
techniques can be applied to our calculus to enable let generalization.

The syntax of the type system, along with the typing and subtyping judgments is
given in \cref{fig:original-typing}.
An implicit assumption throughout the paper is that
variables in contexts are distinct.
We save the explanations for the
static semantics to \cref{sec:type-system}, where we present our
gradually typed version of the calculus.

% \subsection{Motivation}

% In this section we discuss why we are interested in integrating gradual
% typing and implicit polymorphism.

% \paragraph{Why bring gradual types to implicit polymorphism} The Glasgow
% Haskell Compiler (GHC) is a state-of-art compiler that is able to deal with many
% advanced features, including predicative implicit polymorphism. However,
% consider the program \lstinline{(\f. (f 1, f 'a')) id} that cannot type check in
% current GHC, where \lstinline{id} stands for the identity function. The problem
% with this expression is that type inference fails to infer a polymorphic type
% for $f$, given there is no explicit annotation for it. However, from dynamic
% point of view, this program would run smoothly. Also, requesting programmers to
% add annotations for each polymorphic types could become annoying when the
% program scales up and the annotation is long and complicated.

% Gradual types provides a simple solution for it. Rewriting the above expression
% to \lstinline{(\f: *. (f 1, f 'a')) id} produces the value \lstinline{(1, 'a')}.
% Namely, gradual typing provides an alternative that defers typing errors (if
% exists) into dynamic time. Also, without losing type safety, it allows
% programmer to be sloppy about annotations to be agile for program development,
% and later refine the typing annotations to regain the power of static typing.

% \paragraph{Why bring implicit polymorphism to gradual typing} There are several
% existing work about integrating explicit polymorphism into gradual type systems,
% with or without explicit casts~\citep{ahmed2011blame, yuu2017poly}, yet no work
% investigates to add the expressive power of implicit polymorphism into a
% source language.
% % In implicit
% % polymorphism, type applications can be reconstructed by the type checker, for
% % example, instead of \lstinline{id Int 3}, one can directly write \lstinline{id 3}.
% Implicit polymorphism is a hallmark of functional programming, and
% modern functional languages (such as Haskell) employ sophisticated
% type-inference algorithms that, aided by type annotations, can deal with
% higher-rank polymorphism. Therefore as a step towards gradualizing such type
% systems, this paper develops both declarative and algorithmic versions for a
% type system with implicit higher-rank polymorphism.

\subsection{Motivation: Gradually Typed Higher-Rank Polymorphism}

% \bruno{Moved from the introduction:
% }
% \bruno{You can use the text that was cut from the introduction to
%   motivate the combination between gradual typing and higher-ranked
% polymorphism here. Please expand and justify why this would be helpful
% for programmers.}

Our work combines implicit
(higher-rank) polymorphism with gradual typing. As is well
known, a gradually typed language supports both fully static and fully dynamic
checking of program properties, as well as the continuum between these two
extremes. It also offers programmers fine-grained control over the
static-to-dynamic spectrum, i.e., a program can be evolved by introducing more
or less precise types as needed~\citep{garcia2016abstracting}. 

Haskell is a language that supports implicit
higher-rank polymorphism, but no gradual typing. Therefore
some programs that are safe at run-time may be rejected due to the
conservativity of the type system.
For example,
consider the following Haskell program adapted from \citet{jones2007practical}:
\begin{lstlisting}[language=haskell]
foo :: ([Int], [Char])
foo = let f x = (x [1, 2] , x ['a', 'b']) in f reverse
\end{lstlisting}
This program is rejected by Haskell's type checker because Haskell implements
the Damas-Milner rule that a lambda-bound argument (such as \lstinline[language=haskell]{x}) can
only have a monotype, i.e., the type checker can only assign
\lstinline[language=haskell]{x} the type \lstinline[language=haskell]{[Int] -> [Int]}, or
\lstinline[language=haskell]{[Char] -> [Char]}, but not $\forall a. [a]
\to [a]$. Finding such manual polymorphic annotations can be non-trivial.
Instead of rejecting the program outright, due to missing type annotations,
gradual typing provides a simple alternative by giving
\lstinline[language=haskell]$x$ the unknown type (denoted
$\unknown$). With such typing
the same program type-checks and produces
\lstinline[language=haskell]$([2, 1], ['b', 'a'])$. By running the
program, programmers
can gain some additional insight about
the run-time behaviour. Then, with such insight, they can also give
\lstinline[language=haskell]$x$ a more precise type ($\forall a. [a] \to
[a]$) a posteriori so
that the program continues to type-check via implicit polymorphism and also grants
more static safety. In this paper, we envision such a language that
combines the benefits of both implicit higher-rank polymorphism and gradual
typing.


%%% Local Variables:
%%% mode: latex
%%% TeX-master: "../paper"
%%% org-ref-default-bibliography: ../paper.bib
%%% End: