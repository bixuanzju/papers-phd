
\section{Coherence}
\label{sec:cohe}

This section constructs logical relations to
establish the coherence of \name. It turns out that finding a
suitable definition of coherence for \name is already challenging in its own
right. In what follows we reproduce the steps of finding a definition for coherence
that is both intuitive and applicable. Then we present the
construction of logical (equivalence) relations tailored to this 
definition, and the connection between logical equivalence and coherence.



\subsection{In Search of Coherence}

In both \oname and \fname the definition of coherence is based on
$\alpha$-equivalence. More specifically, their coherence property states that
any two target terms that a source expression elaborates into must be exactly the same (up to
$\alpha$-equivalence). Unfortunately this syntactic notion of coherence is
very fragile with respect to extensions.
For example, it is not obvious how to retain this notion of coherence when adding more subtyping
rules such as those in \cref{fig:subtype_decl}.

If we permit ourselves to consider only the syntactic aspects of expressions,
then very few expressions can be considered equal. The syntactic view is also in conflict
with the intuition that the significance of an expression lies in its
contribution to the \textit{outcome} of a computation~\citep{Harper_2016}.
Drawing inspiration from a wide range of literature on contextual
equivalence~\citep{morris1969lambda}, we want a context-based notion of
coherence. It is helpful to consider several examples before presenting the
formal definition of our new semantically founded notion of coherence.

\begin{example} \label{eg:1}
The same \name expression $3$ can be typed $[[nat]]$ in at least two ways: by
\rref{T-sub,S-refl}; or by \rref{T-sub,S-trans,S-refl}, resulting in two \tname
terms $\app{[[id]]}{3}$ and $\app{([[id o id]])}{3}$, respectively. It is apparent
that these two \tname terms are ``equal'' in the sense that both reduce to the
same numeral $3$.
\end{example}

\paragraph{Expression Contexts and Contextual Equivalence.} To formalize the intuition,
we introduce the notion of \textit{expression contexts}. An expression context $[[cc]]$
is a term with a single hole $[[__]]$ (possibly under some binders) in it. The
syntax of expression contexts can be found in \cref{fig:contexts}. The typing
judgement for expression contexts has the form $[[cc : (G |- T) ~~> (G' |- T')]]$ where $([[G |- T]])$ indicates
the type of the hole. This judgement essentially says that plugging a well-typed
term $[[G |- e : T]]$ into the context $[[cc]]$ gives another well-typed term
$[[G' |- cc{e} : T']]$. We define a \textit{complete program} to mean any closed
term of type $[[nat]]$. The following two definitions capture the notion of
\textit{contextual equivalence}.

\begin{figure}[t]
  \centering
\begin{tabular}{llll}\toprule
  \tname contexts & $[[cc]]$ & $\Coloneqq$ & $[[__]] \mid [[\x . cc]] \mid [[cc e2]] \mid [[e1 cc]] \mid [[< cc, e2 >]] \mid [[< e1, cc >]] \mid [[c cc]] \mid [[ { l = cc }]] \mid [[ cc. l ]]$ \\
  \name contexts & $[[C]]$ & $\Coloneqq$ & $[[__]] \mid [[\x . C]] \mid [[C ee2]] \mid [[ee1 C]] \mid [[ee1 ,, C]] \mid [[C ,, ee2]] \mid [[C : A]] \mid [[ { l = C } ]] \mid [[C.l]]$ \\ \bottomrule
\end{tabular}
  \caption{Expression contexts of \name and \tname}
  \label{fig:contexts}
\end{figure}

\begin{definition}[Kleene Equality]
  Two complete programs, $[[e]]$ and $[[e']]$, are Kleene equal, written
  $\kleq{[[e]]}{[[e']]}$, iff there exists $i$ such that $[[e -->> i]]$ and $[[e' -->> i]]$.
\end{definition}

\begin{definition}[Contextual Equivalence] \label{def:cxtx}
  Let $[[G |- e1 : T]]$ and $[[G |- e2 : T]]$.
  \[
    [[G |- e1 ~= e2 : T]] \defeq \forall [[cc : (G |- T) ~~> (empty |- nat)]]  \Longrightarrow
    \kleq{[[cc{e1}]]}{[[cc{e2}]]}
  \]
\end{definition}

\noindent Regarding Example~\ref{eg:1}, it seems adequate to say that $\app{[[id]]}{3}$ and
$\app{([[id o id]])}{3}$ are contextually equivalent. Does this imply that coherence
can be based on \cref{def:cxtx}? Unfortunately it cannot, as demonstrated by the
following example.


\begin{example} \label{eg:2} It may be counter-intuitive that two \tname terms
  $[[\x . p1 x]]$ and $[[\x . p2 x]]$ should also be considered equal. To see
  why, first note that they are both translations of the same \name expression:
  $[[(\x . x) : nat & nat -> nat]]$. What can we do with this lambda
  abstraction? We can apply it to $1 : [[nat & nat]]$ for example. In that case,
  we get two translations $\app{[[(\x . p1 x)]]}{\pair{1}{1}}$ and $\app{[[(\x .
    p2 x)]]}{\pair{1}{1}}$, which both reduce to the same numeral $1$. 
  % TOM: I do not think this sentence contributes to the story:
  % We can
  % keep trying with different arguments, until we are certain that all the
  % translations of $[[(\x . x) : nat & nat -> nat]]$ really contribute the same
  % to the outcome of a computation. 
  However, $[[\x . p1 x]]$ and $[[\x . p2 x]]$
  are definitely not equal according to \cref{def:cxtx}, as one can find a
  context $\app{[[__]]}{\pair{1}{2}}$ where the two terms reduce to two
  different numerals. The problem is that not every well-typed \tname term
  can be obtained from a well-typed \name expression through the
  elaboration semantics. For
  example, $\app{[[__]]}{\pair{1}{2}}$ should not be considered because the
  (non-disjoint) source expression $\mer{1}{2}$ is rejected by the type system
  of the source calculus \name and thus never gets elaborated into $\pair{1}{2}$.
\end{example}

\paragraph{\name Contexts and Refined Contextual Equivalence.}
\Cref{eg:2} hints at a shift from \tname contexts to \name contexts $[[C]]$,
whose syntax is shown in \cref{fig:contexts}. Due to the bidirectional
nature of the type system, the typing judgement of $[[C]]$ features 4
different forms:
\begin{mathpar}
  [[C : (GG => A) ~> (GG' => A') ~~> cc]] \and
  [[C : (GG <= A) ~> (GG' => A') ~~> cc]] \and
  [[C : (GG => A) ~> (GG' <= A') ~~> cc]] \and
  [[C : (GG <= A) ~> (GG' <= A') ~~> cc]]
\end{mathpar}
Take $[[C : (GG => A) ~> (GG' => A') ~~> cc]]$ for example, it reads given a
well-typed \name expression $[[GG |- ee => A]]$, we have $[[GG' |- C{ee} => A']]$. The judgement also generates a \tname context $[[cc]]$ such that $[[cc : (|GG| |- |A|) ~~> (|GG'| |- |A'|)]]$ by
construction. The typing rules appear in the appendix. Now we are ready to
refine \cref{def:cxtx}'s contextual equivalence to take into
consideration both \name and \tname contexts.

\begin{definition}[Refined Contextual Equivalence]  \label{def:cxtx2}
  Let $([[<=>1]], [[<=>2]]) \, \in \{ [[=>]] , [[<=]] \} \times \{ [[=>]] , [[<=]] \} $, $[[GG |- ee1 <=>1 A ~~> e1]]$ and
  $[[GG |- ee2 <=>1 A ~~> e2]]$, two \name expressions are contextually
  equivalent, written $[[GG |- ee1 ~= ee2 : A]]$, iff
  \[
    \forall [[C : (GG <=>1 A) ~> (empty <=>2 nat) ~~> cc]] \Longrightarrow \kleq{[[cc{e1}]]}{[[cc{e2}]]}
  \]
\end{definition}

\noindent Regarding \cref{eg:2}, a possible \name context is $\app{[[__]]}{(1 :
  [[nat & nat]])}$. According to \cref{def:cxtx2}, a corresponding \tname
context is $\app{[[__]]}{\pair{1}{1}}$. Now both $[[\x . p1 x]]$ and $[[\x . p2
x]]$ produce the same numeral $1$ in this context. Of course we should consider
all possible contexts to be certain they are truly equal. From now on, we
use the symbol $\backsimeq_{ctx}$ to refer to contextual equivalence in
\cref{def:cxtx2}. With \cref{def:cxtx2} we can formally state that \name is coherent
in the following theorem:

\begin{restatable}[Coherence]{mtheorem}{coherence} \label{thm:coherence}
  For every $[[GG]]$, $[[ee]]$ and $[[A]]$ we have
  $[[GG |- ee ~= ee : A]]$.
\end{restatable}

\noindent The rest of the section is devoted to proving that \cref{thm:coherence} holds.

\subsection{Logical Relations}

Intuitive as \cref{def:cxtx2} may seem, it is generally very hard to prove
contextual equivalence directly, since it involves quantification over
\textit{all} possible contexts. Worse still, two kinds of contexts are involved
in \cref{thm:coherence}, which makes reasoning even more tedious. The key to
simplifying the reasoning is to exploit types by using logical
relations~\citep{tait, statman1985logical, plotkin1973lambda}.


\paragraph{In Search of a Logical Relation.}
It is worth pausing to ponder what kind of relation we are looking for. % From
% \cref{eg:2}, it is clear that pairs have a special status in \tname. Indeed they
% ought to be, since pairs originate from merges or subtyping. Also disjointness
% on intersection types should correspond to some sort of constraints over pairs.
The high-level intuition behind the relation is to capture the
notion of ``coherent'' values. These values are unambiguous in every context. A
moment of thought leads us to the following important observations:

\begin{observation}[Disjoint values are unambiguous] \label{ob:1} The relation should contain values originating
  from disjoint intersection types. Those values are essentially translated from
  merges, and since \rref{T-merge} ensures disjointness, they are unambiguous.
  For example, $\pair{1}{\recordCon{l}{1}}$ corresponds to the type $[[nat & {l
    : nat}]]$. It is always clear which one to choose ($1$ or
  $\recordCon{l}{1}$) no matter how this pair is used in certain contexts.
\end{observation}

\begin{observation}[Duplication is unambiguous] \label{ob:2} The relation should
  contain values originating from non-disjoint intersection types. This may
  sound baffling since the whole point of disjointness is to rule out
  (ambiguous) expressions such as $\mer{1}{2}$. However, $\mer{1}{2}$ never gets
  elaborated, and the only values corresponding to $[[nat & nat]]$ are those
  pairs such as $\pair{1}{1}$, $\pair{2}{2}$, etc. Those values are essentially
  generated from \rref{T-sub} and are also unambiguous.
\end{observation}

To formalize values being ``coherent'' based on the above observations,
\Cref{fig:logical} defines two (binary) logical relations for \tname, one for
values ($\valR{[[T1]]}{[[T2]]}$) and one for terms ($\eeR{[[T1]]}{[[T2]]}$). We
require that any two values $([[v1]], [[v2]]) \in \valR{[[T1]]}{[[T2]]}$ are
closed and well-typed. For succinctness, we write $\valRR{\tau}$ to mean
$\valR{\tau}{\tau}$, and similarly for $\eeRR{\tau}$.

\begin{figure}[t]
  \centering
\begin{align*}
  [[(v1 , v2) in V ( nat ; nat )]] &\defeq \exists i, v_1 = v_2 = i \\
  [[(v1, v2) in V (T1 -> T2 ; T1' -> T2')]] &\defeq \forall [[(v, v') in V (T1 ; T1')]], [[(v1 v, v2 v') in E (T2 ; T2')]] \\
  [[( {l = v1} , {l = v2}) in V ( {l : T1} ; { l : T2 } )]] &\defeq [[(v1, v2) in V (T1; T2)]] \\
  [[(<v1, v2> , v3) in V (T1 * T2 ; T3)]] &\defeq [[(v1, v3) in V (T1; T3)]] \land [[(v2, v3) in V (T2; T3)]] \\
  [[(v3, <v1, v2> ) in V (T3; T1 * T2)]] &\defeq [[(v3, v1) in V (T3; T1)]] \land [[(v3, v2) in V (T3; T2)]] \\
  [[(v1 , v2) in V (T1; T2)]]  &\defeq \mathsf{true} \quad \text{otherwise}
\end{align*}
\[
  [[(e1, e2) in E (T1; T2)]] \defeq \exists [[v1]] [[v2]], [[e1 -->> v1]] \land [[e2 -->> v2]] \land [[(v1, v2) in V (T1; T2)]]
\]
  \caption{Logical relations for \tname}
  \label{fig:logical}
\end{figure}


\begin{remark}
  The logical relations are heterogeneous, parameterized by two types, one for
  each argument. This is intended to relate values of different types.
\end{remark}


\begin{remark}
  The logical relations resemble those given by \citet{biernacki2015logical}, as
  both are heterogeneous. However, two important differences are worth pointing
  out. Firstly, our value relation for product types ($\valR{[[T1 * T2]]}{[[T3]]}$ and $\valR{[[T3]]}{[[T1 * T2]]}$) is unusual. As we explain
  shortly, this is related to disjointness of intersection types. Secondly,
  their value relation disallows relating functions with natural numbers, while
  ours does not (because their types are \textit{disjoint}).
\end{remark}


First let us consider $\valR{[[T1]]}{[[T2]]}$. The first three cases are
standard: Two natural numbers are related iff they are the same numeral. Two
functions are related iff they map related arguments to related results. Two
singleton records are related iff they have the same label and their fields are
related. These cases reflect Observation~\ref{ob:2} in that the same type
corresponds to the same value.

In the next two cases one of the parameterized types is a product type. In those
cases, the relation distributes over the product constructor $[[*]]$. This may
look strange at first, since the traditional way of relating pairs is by
relating their components pairwise. That is, $[[<v1,v2>]]$ and $[[<v1', v2'>]]$
are related iff (1) $[[v1]]$ and $[[v1']]$ are related and (2) $[[v2]]$ and
$[[v2']]$ are related. According to our definition, we also require that (3)
$[[v1]]$ and $[[v2']]$ are related and (4) $[[v2]]$ and $[[v1']]$ are related.
The design of these two cases is influenced by the disjointness of intersection
types. Below are two rules dealing with intersection types:
\begin{mathpar}
 \drule{D-andL} \and \drule{D-andR}
\end{mathpar}
Notice the structural similarity between these two rules and the two cases. Now
it is clear that the cases for products manifest disjointness of intersection
types, reflecting Observation~\ref{ob:1}. Together with the last case, we can
show that disjointness and the value relation are connected by the following
lemma:

\begin{lemma}[Disjoint values are in a value relation]\label{lemma:disjoint}
  If $[[A1 ** A2]]$ and $v_1 : |A_1|$ and $v_2 : |A_2|$, then $[[(v1, v2) in V (|A1| ; |A2|)]]$.
\end{lemma}
\begin{proof}
  By induction on the derivation of disjointness. \qed
\end{proof}

Next we consider $\eeR{[[T1]]}{[[T2]]}$, which is standard. Informally it expresses
that two closed terms $[[e1]]$ and $[[e2]]$ are related iff
they evaluate to two values $[[v1]]$ and $[[v2]]$ that are related.



\paragraph{Logical Equivalence.}
The logical relations can be lifted to open terms in the usual way. First we give the
semantic interpretation of typing contexts:
\begin{definition}[Interpretation of Typing Contexts]
  $(\gamma_1, \gamma_2) \in \ggR{[[G1]]}{[[G2]]}$ is inductively defined as follows:
  \begin{mathpar}
    \ottaltinferrule{}{}{ }{(\emptyset, \emptyset) \in  \ggR{[[empty]]}{[[empty]]}} \and
    \ottaltinferrule{}{}{ [[(v1, v2) in V (T1; T2)]]  \\ (\gamma_1,\gamma_2) \in \ggR{[[G1]]}{[[G2]]} \\ \mathsf{fresh}\, x }{(\gamma_1[x \mapsto v_1], \gamma_2[x \mapsto v_2]) \in \ggR{[[G1, x : T1]]}{[[G2 , x : T2]]}}
  \end{mathpar}
\end{definition}
Two open terms are related when every pair of related closing substitutions
makes them related:
\begin{definition}[Logical equivalence]
  Let $[[G1 |- e1 : T1]]$ and $[[G2 |- e2 : T2]]$
  \[
    [[G1 ; G2 |- e1 == e2 : T1 ; T2]] \defeq \forall (\gamma_1, \gamma_2) \in
    \ggR{[[G1]]}{[[G2]]}, (\app{\gamma_1}{[[e1]]}, \app{\gamma_2}{[[e2]]}) \in
    \eeR{[[T1]]}{[[T2]]}
  \]
\end{definition}
For succinctness, we write $[[G |- e1 == e2 : T]]$ to mean $[[G ; G |- e1 == e2 : T ; T]]$.


\subsection{Establishing Coherence}

With all the machinery in place, we now show several auxiliary
lemmas that are crucial in the subsequent proofs. % The proof strategy roughly
% follows that of \citet{biernacki2015logical}.

First we show compatibility lemmas, which state that logical equivalence
is preserved by every language construct. Most are standard and thus are
omitted. We show only one compatibility lemma that is specific to our relations:

\begin{lemma}[Coercion Compatibility]   \label{lemma:co-compa}
  Let $[[c |- T1 tri T2]]$
  % The logical relation is preserved by coercion application.
  \begin{itemize}
  \item If $[[G1 ; G2 |- e1 == e2 : T1 ; T0]]$ then $[[G1 ; G2 |- c e1 == e2 : T2 ; T0]]$.
  \item If $[[G1 ; G2 |- e1 == e2 : T0 ; T1]]$ then $[[G1 ; G2 |- e1 == c e2 : T0 ; T2]]$.
  \end{itemize}
\end{lemma}
\begin{proof}
  By induction on the typing derivation of the coercion $[[c]]$. \qed
\end{proof}

Next we show that logical equivalence is preserved by \name contexts:

\begin{lemma}[Congruence] \label{lemma:cong}
  Let $ ([[<=>1]], [[<=>2]]) \, \in \{ [[=>]] , [[<=]] \} \times \{ [[=>]] , [[<=]] \}$, $[[GG |- ee1 <=>1 A ~~> e1]]$,
  $[[GG |- ee2 <=>1 A ~~> e2]]$ and $[[|GG| |- e1 == e2 : |A|]]$
  \[
    \forall [[C : (GG <=>1 A) ~> (GG' <=>2 A') ~~> cc]] \Longrightarrow [[|GG'| |- cc{e1} == cc{e2} : |A'|]]
  \]
\end{lemma}
\begin{proof}
  By induction on the typing derivation of the context $[[C]]$, and applying
  the compatibility lemmas where appropriate. \qed
\end{proof}


We now are in a position to establish coherence. We
require \cref{lemma:uniq}, which is always true in \oname, \fname and \name.

\begin{mtheorem}[Uniqueness of Type-Inference] \label{lemma:uniq}
  If $[[GG |- ee => A1]]$ and $[[GG |- ee => A2]]$, then $[[A1]] \equiv [[A2]]$.
\end{mtheorem}

Finally we show that \name is coherent with respect to logical equivalence:

\begin{lemma}[Coherence wrt Logical Equivalence]  \label{thm:co-log} \leavevmode
  Let $ [[<=>]] \, \in \{ [[=>]] , [[<=]] \}$, $[[GG |- ee <=> A ~~> e]]$ and
  $[[GG |- ee <=> A ~~> e']]$, we have $[[|GG| |- e == e' : |A| ]]$.
\end{lemma}
\begin{proof}
  The proof follows by induction on the first derivation. The most interesting case is \rref{T-sub}:
  \begin{mathpar}
    \drule{T-sub}
  \end{mathpar}
  We need \cref{lemma:uniq} to show that $[[A]]$ is unique so that we can
  apply the induction hypothesis. Then we apply \cref{lemma:co-compa} to say
  that $[[c]]$ preserves the relation between terms. For the other cases
  we use the appropriate compatibility lemmas. \qed
\end{proof}

\begin{remark}
  Usually, with logical relations, we need to prove the \textit{fundamental
    property}, which states that if $[[G |- e : T]]$ then $[[G |- e == e : T]]$.
  Clearly this does not hold for every well-typed \tname term. However, as we
  have emphasized, we do not need to consider every \tname term. Instead, \Cref{thm:co-log}
  acts as a kind of fundamental property for
  well-typed \name expressions.
\end{remark}

We restate the central theorem below:
\coherence*
\begin{proof}
  By \cref{thm:co-log,lemma:cong} and the definition of logical equivalence.
  \qed
\end{proof}

\subsection{Some Interesting Corollaries}

To showcase the strength of the new proof method, we can derive some
interesting corollaries. For the most part, they are direct consequences of
logical equivalence which carry over to contextual equivalence.


\Cref{lemma:neutral} says that merging a term of some type with something else
does not affect its semantics. \Cref{lemma:commu,lemma:assoc} express that
merges are commutative and associative, respectively. \Cref{lemma:coercion_same}
states that coercions from the same types are ``coherent''.

\begin{corollary}[Merge is Neutral] \label{lemma:neutral}
  For every $[[GG]]$, $[[ee1]]$, $[[ee2]]$ and $[[A]]$ we have
  \[
  [[GG |- ee1 ~= ee1 ,, ee2 : A]]
  \]
\end{corollary}

\begin{corollary}[Merge is Commutative] \label{lemma:commu}
  For every $[[GG]]$, $[[ee1]]$, $[[ee2]]$ and $[[A]]$ we have
  \[
  [[GG |- ee1 ,, ee2 ~= ee2 ,, ee1 : A]]
  \]
\end{corollary}


\begin{corollary}[Merge is Associative] \label{lemma:assoc}
  For every $[[GG]]$, $[[ee1]]$, $[[ee2]]$, $[[ee3]]$ and $[[A]]$ we have
  \[
  [[GG |- (ee1 ,, ee2) ,, ee3 ~= ee1 ,, (ee2 ,, ee3) : A]]
  \]
\end{corollary}



\begin{corollary}[Coercions Preserve Semantics]
  \label{lemma:coercion_same}
  If $[[A <: B ~~> c1]]$ and $[[A <: B ~~> c2]]$, then $[[\x . c1 x]]$ and $[[\x
  . c2 x]]$ can be used interchangeably.
\end{corollary}


% Local Variables:
% org-ref-default-bibliography: ../paper.bib
% End:
