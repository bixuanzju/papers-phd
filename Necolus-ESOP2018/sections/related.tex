
\section{Related Work}
\label{sec:related}

\paragraph{Coherence.}
In calculi that feature coercive subtyping, a semantics that interprets the
subtyping judgement by introducing explicit coercions is typically defined on
typing derivations rather than on typing judgements. A natural
question that arises
for such systems is whether the semantics is \textit{coherent}, i.e.,
distinct typing derivations of the same typing judgement possess the same
meaning. Since \citet{Reynolds_1991} proved the coherence of a calculus with
intersection types, based on the denotational semantics for intersection types,
many researchers have studied the problem of coherence in a variety of typed
calculi. Below we summarize two commonly-found approaches in the literature.

%\paragraph{Normalization-based Approach}
The first approach is based on normalization. \citet{Breazu_Tannen_1991} proved
the coherence of the language Fun~\citep{cardelli1985understanding} extended
with recursive types, by translating Fun into System F and showing that any two
typing derivations of the same typing judgement are normalizable to a unique
normal derivation. % where the correctness of the normalization steps is justified
% by an equational theory in the target calculus, i.e., System F.
\citet{Curien_1992} presented a translation of System F$_\leq$ into a calculus
with explicit coercions and showed that any derivations of the same judgement are
translated to terms that are normalizable to a unique normal form. Following the
same approach, \citet{SCHWINGHAMMER_2008} proved the coherence of coercion
translation from the computational lambda calculus of \citet{Moggi_1991} with
subtyping.


%\paragraph{Logical-relations-based Approach}

Central to the first approach is to find a normal form for a representation of
the derivation and show that normal forms are unique for a given typing
judgement. However, this approach cannot be directly applied to Curry-style
calculi, i.e, where the lambda abstractions are not type annotated. Also this
line of reasoning cannot be used when the source language has general recursion.
Instead, \citet{biernacki2015logical} considered the coherence problem based on
contextual equivalence, where they presented a construction of logical relations
for establishing coherence. They showed that this approach is applicable in a
variety of calculi, including delimited continuations and control-effect
subtyping.


\paragraph{Intersection Types and the Merge Operator.}
Forsythe has intersection types and a merge operator, and was proven to be
coherent~\citep{Reynolds_1991}. However to ensure coherence, various
restrictions were added to limit the use of merges. For example, in Forsythe
merges cannot contain more than one function. \citet{Castagna_1992} proposed a
coherent calculus with a special merge operator that works on functions only.
More recently, \citet{dunfield2014elaborating} shows significant expressiveness
of type systems with intersection types and a merge construct. However his
calculus lacks coherence. The limitation was addressed by
\citet{oliveira2016disjoint}, where they introduced disjointness to ensure
coherence. The combination of intersection types, a merge construct and
parametric polymorphism, while achieving coherence was first studied in the
\fname calculus~\citep{alpuimdisjoint}, where they proposed the notion of
disjoint polymorphism. Compared to prior work, \name simplifies type systems
with disjoint intersection types by removing several restrictions. 
Furthermore, \name adopts a much more powerful subtyping relation based 
on BCD subtyping, which in turn requires the use of a more powerful
logical relations based method for proving coherence.


% \bruno{This is an example of a typical mistake when writting related
%   work: you describe existing work, but do not say how it
%   relates/differs from the work presented in the paper.}


\paragraph{BCD Type System and Decidability.}
The BCD type system was first introduced by \citet{Barendregt_1983}. It is
derived from a filter lambda model in order to characterize exactly the strongly
normalizing terms. The BCD type system features a powerful subtyping relation,
which serves as a base for our subtyping relation. The decidability of BCD
subtyping has been shown in several works~\citep{pierce1989decision,
  Kurata_1995, Rehof_2011, Statman_2015}. \citet{laurent2012intersection} has
formalized the relation in Coq in order to eliminate transitivity cuts from it,
but his formalization does not deliver an algorithm. Based on
\citet{Statman_2015}, \citet{bessaiextracting} show a formally verified
subtyping algorithm in Coq. Our Coq formalization follows a different idea from
\citet{pierce1989decision}, which is shown to be easily extensible to coercions
and records. In the course of our mechanization we identified
several mistakes in Pierce's proofs, as well as some important missing lemmas.

\paragraph{Family Polymorphism.}
There has been much work on family polymorphism since \citeauthor{Ernst_2001}'s
original proposal~\citep{Ernst_2001}. Family polymorphism provides an elegant
solution to the Expression Problem. Although a simple Scala solution does exist without
requiring family polymorphism (e.g., see \citet{wang2016expression}), Scala
does not support nested composition: programmers need to manually compose
all the classes from multiple extensions. Broadly speaking, systems that support
family polymorphism can be divided into two categories: those that support
\textit{object families} and those that support \textit{class families}. In
object families, classes are nested in objects, whereas in class families,
classes are nested in other classes. The former choice is considered more
expressive~\citep{ErnstVirtual}, but requires a complex type system usually
involving dependent types.

%\paragraph{Object Families}
Virtual classes~\citep{Madsen_1989} are one means for achieving family
polymorphism. Virtual classes are attributes of an object, and virtual classes
from different instances are not compatible. The original design of virtual
classes in BETA is not statically safe.
To ensure type safety, virtual classes were formalized using path-dependent
types in the calculus \textit{vc}~\citep{ErnstVirtual}. Due to the introduction
of dependent types, their system is rather complex. Subtyping in \textit{vc} is
more restrictive, compared to \name, in that there is no subtyping relationship
between classes in the base family and classes in the derived family, nor is there between
the base family and the derived family. As for conflicts, \textit{vc} follows the
mixin-style by allowing the rightmost class to take precedence. This is in
contrast to \name where conflicts are detected statically and resolved
explicitly. Tribe~\citep{pubsdoc:tribe-virtual-calculus} is another language
that provides a variant of virtual classes, with support for both class and
object families. % This
% flexibility allows a further bound class to be a subtype of the class it
% overrides.


%\paragraph{Class Families}
Jx~\citep{Nystrom_2004} supports a notion of \textit{nested inheritance}, which
is a limited form of family polymorphism. Nested inheritance works much like
class families with support for nesting of arbitrary depth. Unlike virtual
classes, subclass and subtype relationships are preserved by inheritance: the
overriding class is also a subtype of the class it overrides.
%Nested inheritance
%does not support generic types. As we discussed in \cref{sec:diss}, \name can be
%easily extended to incorporate parametric polymorphism.
J\&~\citep{Nystrom:2006:JNI:1167473.1167476} is a language that supports
\textit{nested intersection}, building on top of Jx. Similar to \name,
intersection types play an important role in J\&, which are used to compose
packages/classes. Unlike \name, J\& does not have a merge-like operator, and
requires special treatment (e.g., prefix types) to deal with conflicts.
% Interestingly, conflicts are distinguished in J\&. If a name was introduced in a
% common ancestor of the intersected namespaces, it is not considered a conflict.
% This is somewhat similar to \name where a value such as $1 :
% \inter{\mathsf{Nat}}{\mathsf{Nat}}$ is allowed. Otherwise, the name is assumed
% to refer to distinct members that coincidentally have the same name, then
% ambiguity occurs, and resolution is required by exploiting prefix type notation.
\citet{SAITO_2007} identified a minimal, lightweight set of language features
to enable family polymorphism, though inheritance relations between nested
classes are not preserved by extension. \citet{Corradi_2012} present a language
design that integrates modular composition and nesting of Java-like classes. It
features a set of composition operators that allow to manipulate nested classes
at any depth level. Compared with those systems, which usually focus on getting
a relatively complex Java-like language with family polymorphism,
\name focuses on a minimal calculus that supports
nested composition. \name still lacks important practical features, such as
recursive types and mutable state. Supporting these features is not the
focus of this paper, but we expect to investigate those features in the future.


% \bruno{Probably comparison, in this last paragraph, can be improved a
%   bit. Also perhaps the discussion does not need to be so detailed. }


%%% Local Variables:
%%% org-ref-default-bibliography: ../paper.bib
%%% End:
