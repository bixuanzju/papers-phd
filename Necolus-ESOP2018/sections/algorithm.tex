
\section{Algorithmic Subtyping}
\label{sec:alg}

This section presents an algorithm that implements the subtyping relation in
\cref{fig:subtype_decl}. While BCD subtyping is well-known, the
presence of a transitivity axiom in the rules means that the system is
not algorithmic. This raises an obvious question: how to obtain an
algorithm for this subtyping relation? \citet{Laurent12note} has shown that simply dropping
the transitivity rule from the BCD system is not possible without losing expressivity. Hence, this avenue for
obtaining an algorithm is not available. 
%Moreover, even if transitivity elimination
%would be possible, the remaining rules are still highly overlapping, and pose
%difficulties for an implementation.  
Instead, we adapt \citeauthor{pierce1989decision}'s decision
procedure~\citep{pierce1989decision} for a subtyping system (closely
related to BCD) to obtain a sound and complete algorithm for our
BCD extension. Our algorithm extends \citeauthor{pierce1989decision}'s decision
procedure with subtyping of singleton records and
coercion generation. We prove in Coq that the algorithm is sound and complete with
respect to the declarative version. At the same time we
find some errors and missing lemmas in \citeauthor{pierce1989decision}'s original manual proofs.

%The algorithm is implemented in our
%prototype implementation. \jeremy{should i say more about implementation?}

%See \cref{sec:alg} for the details. 
%\bruno{The meaning of the paragraph is somewhat obscure to me. After
%  discussing with Tom, it seems that what may be meant here is that we
%cannot do cut elimination, which is a common process that you can try
%for certain systems with subtyping. However Pierce managed to find
%another way to get a sound/complete algorithmic system. Maybe 
%the text can be improved.}


\subsection{The Algorithm Subtyping}

\begin{figure}[t]
  \centering
  \drules[A]{$[[fs |- A <: B ~~> c]]$}{Algorithmic subtyping}{and, arr, rcd, top, arrNat, rcdNat, nat, andNOne, andNTwo}
  \caption{Algorithmic subtyping of \name}
  \label{fig:algorithm}
\end{figure}


\Cref{fig:algorithm} shows the algorithmic subtyping judgement $[[fs |- A <: B
~~> c]]$. This judgement is the algorithmic counterpart of the declarative
judgement $[[A <: fs -> B ~~> c]]$, where the symbol $[[fs]]$ stands for a
queue of types and labels.

\begin{definition} $[[fs -> A]]$ is inductively defined as follows:
  \begin{mathpar}
    [[ [] -> A]] = [[A]] \and
    [[ (fs , B) -> A]] = [[fs -> (B -> A)]] \and
    [[ (fs , {l}) -> A]] = [[fs -> {l : A}]]
  \end{mathpar}
\end{definition}

The basic idea of $[[fs |- A <: B ~~> c]]$ is to first perform a structural
analysis of $[[B]]$, which descends into both sides of $[[&]]$'s (\rref{A-and}),
into the right side of $[[->]]$'s (\rref{A-arr}), and into the fields of
records (\rref{A-rcd}) until it reaches one of the two base cases, $[[nat]]$ or
$[[Top]]$. If the base case is $[[Top]]$, then the subtyping holds trivially (\rref{A-top}).
If the base case is $[[nat]]$, the algorithm performs a structural
analysis of $[[A]]$, in which $[[fs]]$ plays an important role.
The left sides of $[[->]]$'s are pushed onto
$[[fs]]$ as they are encountered in $[[B]]$ and popped off again later, left to right, as
$[[->]]$'s are encountered in $[[A]]$ (\rref{A-arrNat}). Similarly, the labels are pushed
onto $[[fs]]$ as they are encountered in $[[B]]$ and popped off again later, left to right,
as records are encountered in $[[A]]$ (\rref{A-rcdNat}). The remaining rules
are not surprising as they are quite similar to their declarative
counterparts.

Now consider the coercions. The algorithmic subtyping uses the same set of
coercions as in the declarative subtyping. However, because the algorithm
subtyping has a different structure, the rules generate slightly more complicated coercions. For
space reasons, the two meta-functions $\llbracket \cdot \rrbracket_{\top} $ and
$\llbracket \cdot \rrbracket_{\&}$, used in \rref{A-top,A-and} respectively, are
given in the appendix.

% The two meta-functions $\llbracket \cdot \rrbracket_{\top}
% $ and $\llbracket \cdot \rrbracket_{\&}$, used in \rref{A-top,A-and}
% respectively, are defined in \cref{fig:coercion}.

% \begin{figure}[t]
%     \centering
%     \begin{subfigure}[b]{0.4\textwidth}
%       \begin{align*}
%         [[ < [] >1 ]] &=  [[top]] \\
%         [[ < { l } , fs >1 ]] &= [[ {l : < fs >1} o < l >  ]] \\
%         [[ < A , fs >1 ]] &= [[(top -> < fs >1) o (topArr o top)]]
%       \end{align*}
%     \end{subfigure}
%     \begin{subfigure}[b]{0.4\textwidth}
%       \begin{align*}
%         [[ < [] >2 ]] &=  [[id]] \\
%         [[ < { l } , fs >2 ]] &= [[ {l : < fs >2} o distRcd l  ]] \\
%         [[ < A , fs >2 ]] &= [[(id -> < fs >2) o distArr]]
%       \end{align*}
%     \end{subfigure}
%     \caption{Meta-functions of coercions}\label{fig:coercion}
% \end{figure}

\subsection{Correctness of the Algorithm}

To establish the correctness of the algorithm, we must show that the algorithm
is both sound and complete with respect to the declarative specification. While
soundness follows quite easily, completeness is much harder. The proof of
completeness essentially follows that of \citet{pierce1989decision}
%%\footnote{
%%While transferring \citeauthor{pierce1989decision}'s manual proofs to Coq,
%%we discovered several errors, which will be reported along the way.}
in that we
first show that the algorithmic subtyping is both reflexive and
transitive. 


\paragraph{Soundness of the Algorithm.}

% First we show two lemmas that connect the declarative subtyping with the
% meta-functions in \cref{fig:coercion}.

% \begin{lemma} \label{lemma:top}
%   $[[ Top <: fs -> Top ~~> < fs >1]]$
% \end{lemma}
% \begin{proof}
%   By induction on the length of $[[fs]]$. \qed
% \end{proof}

% \begin{lemma} \label{lemma:and}
%   $[[(fs -> A) & (fs -> B) <: fs -> (A & B) ~~> < fs >2]]$
% \end{lemma}
% \begin{proof}
%   By induction on the length of $[[fs]]$. \qed
% \end{proof}

The proof of soundness is straightforward.
\begin{mtheorem}[Soundness] \label{thm:soundness}
  If $[[ fs |- A <: B ~~> c]]$ then $[[A]] <: [[fs]] \rightarrow [[B]] [[~~>]] [[c]]$.
\end{mtheorem}
\begin{proof}
  By induction on the derivation of the algorithmic subtyping. \qed
\end{proof}


\paragraph{Completeness of the Algorithm.}


\newcommand{\U}[1]{\mathcal{U}(#1)}

Completeness, however, is much harder. The reason is that, due to the use of
$[[fs]]$, reflexivity and transitivity are not entirely obvious. We need to
strengthen the induction hypothesis by introducing the notion of a set,
$\U{[[A]]}$, of ``reflexive supertypes'' of $[[A]]$, as defined below:
\begin{mathpar}
  \U{[[Top]]} \defeq \{ [[Top]]  \} \and
  \U{[[nat]]} \defeq \{ [[nat]]  \} \and
  \U{[[{l : A}]]} \defeq \{ [[{l : B}]] \mid [[B]] \in \U{[[A]]}  \} \and
  \U{[[A & B]]} \defeq \U{[[A]]} \cup \U{[[B]]} \cup \{ [[A & B]] \} \and
  \U{[[A -> B]]} \defeq \{ [[A -> CC]] \mid [[CC]] \in \U{[[B]]} \}
\end{mathpar}

We show two lemmas about $\U{[[A]]}$ that are crucial in the subsequent proofs.

\begin{lemma} \label{lemma:set_refl}
  $[[A]] \in \U{[[A]]}$
\end{lemma}
\begin{proof}
  By induction on the structure of $[[A]]$.
\end{proof}

\begin{lemma} \label{lemma:set_trans}
  If $[[A]] \in \U{[[B]]}$ and $[[B]] \in \U{[[CC]]}$ then $[[A]] \in \U{[[CC]]}$.
\end{lemma}
\begin{proof}
  By induction on the structure of $[[B]]$.
\end{proof}

\begin{remark}
  \Cref{lemma:set_trans} is not found in \citet{pierce1989decision}, which is
  crucial in \cref{lemma:refl0}, from which reflexivity (\cref{lemma:refl})
  follows immediately.
\end{remark}

% Next we show the following lemma from which reflexivity (\cref{lemma:refl})

\begin{lemma} \label{lemma:refl0}
  If $[[fs -> B]] \in \U{[[A]]}$ then $\exists [[c]]$ s.t. $[[fs |- A <: B ~~> c]]$.
\end{lemma}
\begin{proof}
  By induction on $\mathsf{size}([[A]]) + \mathsf{size}([[B]]) + \mathsf{size}([[fs]])$.
\end{proof}
% \begin{remark}
%   \citeauthor{pierce1989decision}'s proof is wrong in one case~\citep[pp.~10, Case~ii]{pierce1989decision} because we need \cref{lemma:set_trans} to be able
%   to apply the inductive hypothesis.
% \end{remark}

Now it immediately follows that the algorithmic subtyping is reflexive.

\begin{lemma}[Reflexivity] \label{lemma:refl}
  For every $[[A]]$ then $\exists [[c]]$ s.t. $[[ [] |- A <: A ~~> c]]$.
\end{lemma}
\begin{proof}
  Immediate from \cref{lemma:set_refl,lemma:refl0}.
\end{proof}

We omit the details of the proof of transitivity.
%The proof of transitivity is, to quote \citeauthor{pierce1989decision}, typically
%``the hardest single piece'' of metatheory. We omit the details here for lack of space and
%refer the interested reader to our Coq development.

\begin{lemma}[Transitivity] \label{lemma:trans}
  If $[[ [] |- A1 <: A2 ~~> c1]]$ and $[[ [] |- A2 <: A3 ~~> c2]]$ then $\exists
  [[c]]$ s.t. $[[ [] |- A1 <: A3 ~~> c]]$.
\end{lemma}

With reflexivity and transitivity in position, we show the main theorem.

\begin{mtheorem}[Completeness] \label{thm:complete}
  If $[[A <: B ~~> c]]$ then $\exists [[c']]$ s.t. $[[ [] |- A <: B ~~> c']]$.
\end{mtheorem}
\begin{proof}
  By induction on the derivation of the declarative subtyping and applying \cref{lemma:refl,lemma:trans} where appropriate.
\end{proof}
\begin{remark}
  \citeauthor{pierce1989decision}'s proof is wrong~\citep[pp.~20, Case~F]{pierce1989decision} in the case
  \[
  \drule{S-arr}
  \]
  where he directly concludes from the inductive hypothesis $[[ [] |- B1 <:
  A1]]$ and $[[ [] |- A2 <: B2]]$ that $[[ [] |- A1 -> A2 <: B1 -> B2]]$ holds,
  which is clearly wrong since no algorithmic rules apply here. It turns out we
  need a few technical lemmas to support the argument, which are omitted for 
  reasons of space.
\end{remark}

\begin{remark}
  It is worth pointing out that the two coercions $[[c]]$ and $[[c']]$ in
  \cref{thm:complete} are contextually equivalent, which follows from
  \cref{thm:soundness} and \cref{lemma:coercion_same}.
\end{remark}

% Local Variables:
% org-ref-default-bibliography: ../paper.bib
% End:
