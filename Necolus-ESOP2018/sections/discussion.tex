
\section{Discussion}
\label{sec:diss}

\subsection{An Alternative Target Language}

Our target language is essentially STLC with explicit coercions. The syntactic
category of coercions plays an important role in the establishment of coherence,
especially for proving \cref{lemma:co-compa}. However in \oname, coercive
subtyping directly produces plain $\lambda$-terms. Another issue is that the
elaboration semantics of \fname eliminates records while in \name records are
preserved in the target language. So an interesting question is: can our results
be extended to a target language without coercions and records? The answer is
positive: our results can be easily transferred to such target language. As
discussed by \citet{biernacki2015logical}, the idea is to replace all coercions
by the corresponding $\lambda$-terms and eliminate all records, then show that
the resulting terms preserve the semantics.


% For ease of reference, we name such target language
% $\lambda_\rightarrow$. Below we give a sketch~\citep{biernacki2015logical} of
% the transformation. Let $\parallel e \parallel$ be a term with all the coercions
% replaced by the corresponding $\lambda$-terms, and all the records eliminated.
% We need to show that for any contextually equivalent terms $e_1$ and $e_2$ in
% \tname, terms $\parallel e_1 \parallel$ and $\parallel e_2 \parallel$ are also
% contextually equivalent in $\lambda_\rightarrow$. This can be accomplished by
% the following three facts: \jeremy{should I give precise definition of
%   $\parallel \cdot \parallel$?}

% \begin{enumerate}
% \item Every well-typed $\lambda_\rightarrow$ terms is also well-typed in \tname.
% \item Term $e$ terminates iff $\parallel e \parallel$ terminates.
% \item If context $C$ does not contain coercions and records, then $C\{\parallel
%   e \parallel\} \equiv \parallel C\{ e \} \parallel$.
% \end{enumerate}



\subsection{Parametric Polymorphism and Object Algebras}
\bruno{Maybe we can move this text to future work and drop the rest of
this section.}
Enriching \name with parametric polymorphism is the most natural direction of
our future work. There is abundant literature on logical relations for
parametric polymorphism~\citep{reynolds1983types} and we foresee no fundamental
difficulties in extending our proof method with parametric
polymorphism\footnote{At the time of writing, we are in the process of extending
  our Coq development.}. The resulting calculus will be more expressive than
\fname. An interesting application we intend to investigate is type-safe,
coherent and boilerplate-free composition of \textit{object
  algebras}~\citep{oliveira2012extensibility, oliveira2013feature}. For example,
equipped with parametric polymorphism, we can define the so-called Object
Algebra interface that captures the common interface of the Expression Problem in
\cref{sec:overview}:
\lstinputlisting[linerange=4-4]{../../impl/examples/visitor.sl}% APPLY:linerange=ALGEBRA_DEF
A concrete Object Algebra will implement such an interface by instantiating the
type parameter with a type that constructs objects for a particular purpose. For
example, instantiating \lstinline{E} with \lstinline{IPrint} will give us the
interface of the \lstinline{Lang} family. In that sense, Object Algebra
interfaces can be viewed as family interfaces.


% \paragraph{Recursive Types}
% Adding (iso-)recursive types is also an interesting line of future
% work. We hope that iso-recursive types will enable us to model
% more complex forms of family polymorphism. In particular, once
% recursive types are available it is possible to model methods whose
% arguments have the self type~\cite{thistype}. The key challange in
% adding recusive types is defining subtyping and disjointness, and
% proving coherence. \bruno{Do we have some temptative rules for
%   recursive types?} To adapt our coherence proof we could use step-indexed
% %logical relations~\citep{}.


%It would allow us to emulate Haskell-like type-classes through intersection
%types, and give insight in the foundation of the type-class
%resolution mechanism. \jeremy{is that right?}
%The main difficulty is to figure out how disjointness works for recursive
%types\jeremy{say more?}. To adapt our coherence proof we could use step-indexed
%logical relations~\citep{}. 


% \begin{itemize}

% \item The declarative system is inherently incoherent. Consider the following example yielding different values:
%   \[
%     \inferrule*[right=Sub]{
%       \inferrule*[right=Merge]{
%         (\mer{3}{\code{true}}) : \inter{\code{Int}}{\code{Bool}} \\
%         (\mer{4}{'c'}) : \code{Char}
%       }{\mer{(\mer{3}{\code{true}})}{(\mer{4}{'c'})} : \inter{(\inter{\code{Int}}{\code{Bool}})}{\code{Char}}} \\
%       \inter{(\inter{\code{Int}}{\code{Bool}})}{\code{Char}} <: \code{Int}
%     }{\mer{(\mer{3}{\code{true}})}{(\mer{4}{'c'})} : \code{Int} \rightsquigarrow 3}
%   \]
%   \[
%     \inferrule*[right=Sub]{
%       \inferrule*[right=Merge]{
%         (\mer{3}{\code{true}}) : \code{Bool} \\
%         (\mer{4}{'c'}) : \inter{\code{Int}}{\code{Char}}
%       }
%       { \mer{(\mer{3}{\code{true}})}{(\mer{4}{'c'})} : \inter{\code{Bool}}{(\inter{\code{Int}}{\code{Char}})}} \\
%       \inter{\code{Bool}}{(\inter{\code{Int}}{\code{Char}})} <: \code{Int}
%     }{\mer{(\mer{3}{\code{true}})}{(\mer{4}{'c'})} : \code{Int} \rightsquigarrow 4}
%   \]

%   The bi-directional system ensures that sub-terms in a merge can only be inferred.


% \item mention the current proof strategy cannot deal with the situation where records
%   in the source are erased during elaboration.


% \end{itemize}


% Local Variables:
% org-ref-default-bibliography: ../paper.bib
% End:
