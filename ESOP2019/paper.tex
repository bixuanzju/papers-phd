% This is samplepaper.tex, a sample chapter demonstrating the
% LLNCS macro package for Springer Computer Science proceedings;
% Version 2.20 of 2017/10/04
%
\documentclass[runningheads]{llncs}
\usepackage{llncsdoc}
%
\usepackage{graphicx}
% Used for displaying a sample figure. If possible, figure files should
% be included in EPS format.
%
% If you use the hyperref package, please uncomment the following line
% to display URLs in blue roman font according to Springer's eBook style:
% \renewcommand\UrlFont{\color{blue}\rmfamily}

%%%%%%%%%%%%%%%%%%%%%%%%%%%%%%%%%%%%%%%%%%%%%%%%%%%%%%%%%%%%%%%%%%%%%%%%
% Load up my personal packages
%%%%%%%%%%%%%%%%%%%%%%%%%%%%%%%%%%%%%%%%%%%%%%%%%%%%%%%%%%%%%%%%%%%%%%%%

\usepackage[numbers,sort]{natbib}
% Basics
\usepackage{fixltx2e}
\usepackage{url}
\usepackage{fancyvrb}
\usepackage{mdwlist}  % Miscellaneous list-related commands
\usepackage{xspace}   % Smart spacing
\usepackage{supertabular}

% https://www.nesono.com/?q=book/export/html/347
% Package for inserting TODO statements in nice colorful boxes - so that you
% won’t forget to fix/remove them. To add a todo statement, use something like
% \todo{Find better wording here}.
\usepackage{todonotes}

%% Math
\usepackage{bm}       % Bold symbols in maths mode

% http://tex.stackexchange.com/questions/114151/how-do-i-reference-in-appendix-a-theorem-given-in-the-body
\usepackage{thmtools, thm-restate}

%% Theoretical computer science
\usepackage{stmaryrd}
\usepackage{mathtools}  % For "::=" ( \Coloneqq )

%% Font
% \usepackage[euler-digits,euler-hat-accent]{eulervm}


%% Some recommended packages.
\usepackage{booktabs}   %% For formal tables:
                        %% http://ctan.org/pkg/booktabs
\usepackage{subcaption} %% For complex figures with subfigures/subcaptions
                        %% http://ctan.org/pkg/subcaption


\usepackage{ottalt}

\usepackage{comment}

% Hyper links
\usepackage{url}
\usepackage{
  nameref,%\nameref
  hyperref,%\autoref
}
\usepackage[capitalise]{cleveref}
% \hypersetup{
%    colorlinks,
%    citecolor=black,
%    filecolor=black,
%    linkcolor=blue,
%    urlcolor=black
% }


% Code highlighting
\usepackage{listings}

\lstset{%
  backgroundcolor=\color{white},
  basicstyle=\small\ttfamily,
  keywordstyle=\sffamily\bfseries,
  captionpos=none,
  columns=flexible,
  lineskip=-1pt,
  keepspaces=true,
  showspaces=false,               % show spaces adding particular underscores
  showstringspaces=false,         % underline spaces within strings
  showtabs=false,                 % show tabs within strings adding particular underscores
  breaklines=true,                % sets automatic line breaking
  breakatwhitespace=true,         % sets if automatic breaks should only happen at whitespace
  escapeinside={(*}{*)},
  literate={->}{{$\rightarrow$}}1 {Top}{{$\top$}}1 {=>}{{$\Rightarrow$}}1 {/\\}{{$\Lambda$}}1,
  tabsize=2,
  commentstyle=\color{purple}\ttfamily,
  stringstyle=\color{red}\ttfamily,
  sensitive=false
}

\lstdefinelanguage{sedel}{
  keywords={Int, String, this, trait, inherits, super, type, Trait, override, self, new, if, then, else, let, in},
  identifierstyle=\color{black},
  morecomment=[l]{--},
  morecomment=[l]{//},
  morestring=[b]",
  xleftmargin  = 3mm,
  morestring=[b]'
}

\lstdefinelanguage{gbeta}{%
  language     = java,
  morekeywords = {virtual,refine},
  xleftmargin  = 3mm
}

\lstset{language=sedel}

\theoremstyle{remark}
\newtheorem{observation}{Observation}

% General
\newcommand{\code}[1]{\texttt {#1}}
\newcommand{\highlight}[1]{\colorbox{yellow}{#1}}

% Logic
\newcommand{\turns}{\vdash}

% Math
\newcommand{\im}[1]{\lvert #1 \rvert}

% PL
\newcommand{\subst}[2]{\lbrack #1 / #2 \rbrack}
\newcommand{\concatOp}{+\kern-1.3ex+\kern0.8ex}  % http://tex.stackexchange.com/a/4195/73122

% Constructors
\newcommand{\for}[2]{\forall #1. \, #2}
\newcommand{\lam}[2]{\lambda #1. \, #2}
\newcommand{\app}[2]{#1 \; #2}
\newcommand{\blam}[2]{\Lambda #1. #2}
\newcommand{\tapp}[2]{#1 \; #2}

\newcommand{\pair}[2]{\langle #1, #2 \rangle}
\newcommand{\inter}[2]{#1 \,\&\, #2}
\newcommand{\mer}[2]{#1 \, ,, \, #2}
\newcommand{\proj}[2]{{\code{proj}}_{#1} #2}
\newcommand{\ctx}[2]{#1\left\{#2\right\}}
\newcommand{\bra}[1]{\llbracket #1 \rrbracket}


\newcommand{\recordType}[2]{\{ #1 : #2 \}}
\newcommand{\recordCon}[2]{\{ #1 = #2 \}}

\newcommand{\ifThenElse}[3]{\code{if} \; #1 \; \code{then} \; #2 \; \code{else} \; #3}

\newcommand{\defeq}{\triangleq}

\newcommand{\logeq}[2]{#1 \backsimeq_{log} #2}
\newcommand{\kleq}[2]{#1 \backsimeq #2}
\newcommand{\ctxeq}[3]{#1 \backsimeq_{ctx} #2 : #3}

\newcommand{\stepn}{\longmapsto^*}
\newcommand{\step}{\longmapsto}



%%%%%%%%%%%%%%%%%%%%%%%%%%%%%%%%%%%%%%%%%%%%%%%%%%%%%%%%%%%%%%%%%%%%%%%%
% Hyperlinks
%%%%%%%%%%%%%%%%%%%%%%%%%%%%%%%%%%%%%%%%%%%%%%%%%%%%%%%%%%%%%%%%%%%%%%%%

\usepackage[capitalise]{cleveref}


%%%%%%%%%%%%%%%%%%%%%%%%%%%%%%%%%%%%%%%%%%%%%%%%%%%%%%%%%%%%%%%%%%%%%%%%
% Ott includes
%%%%%%%%%%%%%%%%%%%%%%%%%%%%%%%%%%%%%%%%%%%%%%%%%%%%%%%%%%%%%%%%%%%%%%%%

\usepackage{ottalt}
\inputott{ott-rules}
\renewcommand\ottaltinferrule[4]{
  \inferrule*[narrower=0.6,lab=#1,#2]
    {#3}
    {#4}
}


%%%%%%%%%%%%%%%%%%%%%%%%%%%%%%%%%%%%%%%%%%%%%%%%%%%%%%%%%%%%%%%%%%%%%%%%
% Revision tool
%%%%%%%%%%%%%%%%%%%%%%%%%%%%%%%%%%%%%%%%%%%%%%%%%%%%%%%%%%%%%%%%%%%%%%%%
% \newcommand\mynote[3]{\textcolor{#2}{#1: #3}}
% \newcommand\bruno[1]{\mynote{Bruno}{red}{#1}}
% \newcommand\tom[1]{\mynote{Tom}{blue}{#1}}
% \newcommand\ningning[1]{\mynote{Ningning}{orange}{#1}}
% \newcommand\jeremy[1]{\mynote{Jeremy}{purple}{#1}}



\begin{document}
%
\title{Distributive Disjoint Polymorphism for Compositional Programming}
%
\titlerunning{Distributive Disjoint Polymorphism}
% If the paper title is too long for the running head, you can set
% an abbreviated paper title here
%
\author{Xuan Bi\inst{1} \and
Ningning Xie\inst{1} \and
Bruno C. d. S. Oliveira\inst{1} \and
Tom Schrijvers\inst{2}}
%
\authorrunning{X.\,Bi et al.}
% First names are abbreviated in the running head.
% If there are more than two authors, 'et al.' is used.
%
\institute{The University of Hong Kong, Hong Kong, China \\
\email{\{xbi,nnxie,bruno\}@cs.hku.hk} \and
KU Leuven, Leuven, Belgium \\
\email{tom.schrijvers@cs.kuleuven.be}}
%
\maketitle              % typeset the header of the contribution
%
\begin{abstract}

\begin{comment}
Compositionality is a desirable property in programming
designs. Broadly defined, compositionality is the principle that a
system should be built by composing smaller subsystems.
Programming techniques such as \emph{shallow embeddings} of
Domain Specific Languages (DSLs),  \emph{finally tagless} or \emph{object algebras}
are built on the principle of compositionality.
However, programming languages often only support well simple
compositional designs, but language support for more sophisticated
compositional designs is lacking.

In this paper we present a calculus and polymorphic type system with \emph{(disjoint)
intersection types}, called \fnamee,
that supports a broader notion of compositional designs, and enables the
development of highly modular and reusable programs. \fnamee is a
generalization and extension of Alpuim et al. \fname calculus,
which proposed the idea of \emph{disjoint polymorphism}.
The main novelty of \fnamee is a novel subtyping algorithm with
distributivity laws on types. Distributivity plays a fundamental role
in improving compositional designs, by enabling the automatic
composition of multiple operations/interpretations. The main technical
challenge is the proof of coherence for \fnamee as impredicativity makes it
hard to develop a well-founded logical relation for coherence.
However, by restricting the system to predicative instantiations only
we are able to develop a suitable logical relation and show
coherence.
We illustrate the use of the calculus to model highly modular
interpretations of DSLs, that can be smoothly composed with
the merge operator of \fnamee. Furthermore, we provide a detailed
comparison between \emph{distributive disjoint polymorphism} and
\emph{row types}.
\end{comment}

Popular programming techniques such as \emph{shallow embeddings} of Domain
Specific Languages (DSLs), \emph{finally tagless} or \emph{object
  algebras} are built on the principle of \emph{compositionality}.  However,
existing programming languages only support simple compositional designs well,
and have limited support for more sophisticated ones.

This paper presents the \fnamee calculus, which supports
highly modular and compositional designs that
improve on existing techniques. These improvements are due to the
combination of three features: \emph{disjoint intersection
  types} with a \emph{merge operator}; \emph{parametric (disjoint)
  polymorphism}; and \emph{BCD-style distributive subtyping}.  The
main technical challenge is  \fnamee's proof of coherence. A
naive adaptation of ideas used in System F's \emph{parametricity} to
\emph{canonicity} (the logical relation used by \fnamee to prove coherence) results in an ill-founded logical relation. To solve the
problem our canonicity relation employs a different technique based on
immediate substitutions and a restriction to predicative
instantiations. Besides coherence, we show
several other important meta-theoretical results, such as type-safety, 
sound and complete algorithmic subtyping, and
decidability of the type system. Remarkably, unlike 
\fsub's \emph{bounded polymorphism}, disjoint polymorphism
in \fnamee supports decidable type-checking.


\begin{comment}
\fnamee  is a
generalization and extension of Alpuim et al. \fname calculus,
which proposed the idea of \emph{disjoint polymorphism}.
The main novelty of \fnamee is a novel polymorphic subtyping algorithm with
distributivity laws on types. Distributivity plays a fundamental role
in improving compositional designs, by enabling the automatic
composition of multiple operations/interpretations.
The main technical
challenge is the proof of coherence for \fnamee as impredicativity makes it
hard to develop a well-founded logical relation for coherence.
However, by restricting the system to predicative instantiations only
we are able to develop a suitable logical relation and show
coherence.
We illustrate the use of the calculus to model highly modular
interpretations of DSLs, that can be smoothly composed with
the merge operator of \fnamee. Furthermore, we provide a detailed
comparison between \emph{distributive disjoint polymorphism} and \emph{row types}.
\end{comment}

% \keywords{First keyword  \and Second keyword \and Another keyword.}
\end{abstract}
%
%
%

\section{Introduction}
\label{sec:introduction}

Compositionality is a desirable property in programming
designs. Broadly defined, it is the principle that a
system should be built by composing smaller subsystems. For instance,
in the area of programming languages, compositionality is
a key aspect of \emph{denotational semantics}~\cite{scott1971toward, scott1970outline}, where
the denotation of a program is constructed from the denotations of its parts.
% For example, the semantics for a language of simple arithmetic expressions
% is defined as:
% 
% \[\begin{array}{lcl}
% \llbracket n \rrbracket_{E} & = & n \\
% \llbracket e_1 + e_2 \rrbracket_{E} & = & \llbracket e_1 \rrbracket_E + \llbracket  e_2 \rrbracket_E \\
% \end{array}\]
% 
% \bruno{Replace E by fancier symbol?}
% Here there are two forms of expressions: numeric literals and
% additions. The semantics of a numeric literal is just the numeric
% value denoted by that literal. The semantics of addition is the
% addition of the values denoted by the two subexpressions.
Compositional definitions have many benefits.
One is ease of reasoning: since compositional
definitions are recursively defined over smaller elements they
can typically be reasoned about using induction. Another benefit
is that compositional definitions are easy to extend,
without modifying previous definitions.
% For example, if we also wanted to support multiplication,
% we could simply define an extra case:
% 
% \[\begin{array}{lcl}
% \llbracket e_1 * e_2 \rrbracket_E & = & \llbracket e_1 \rrbracket_E * \llbracket  e_2 \rrbracket_E \\
% \end{array}\]

Programming techniques that support compositional
definitions include:
\emph{shallow embeddings} of
Domain Specific Languages (DSLs)~\cite{DBLP:conf/icfp/GibbonsW14}, \emph{finally
  tagless}~\cite{CARETTE_2009}, \emph{polymorphic embeddings}~\cite{hofer_polymorphic_2008} or
\emph{object algebras}~\cite{oliveira2012extensibility}. These techniques allow us to create
compositional definitions, which are easy to extend without
modifications. Moreover, when modeling semantics, both finally tagless and object algebras
support \emph{multiple interpretations} (or denotations) of
syntax, thus offering a solution to the well-known \emph{Expression Problem}~\cite{wadler1998expression}.
Because of these benefits these techniques have become
popular both in the functional and object-oriented
programming communities.

However, programming languages often only support simple compositional designs
well, while support for more sophisticated compositional designs is lacking.
For instance, once we have multiple interpretations of syntax, we may wish to
compose them. Particularly useful is a \emph{merge} combinator,
which composes two interpretations~\cite{oliveira2012extensibility,
oliveira2013feature, rendel14attributes} to form a new interpretation that,
when executed, returns the results of both interpretations. 

% For example, consider another pretty printing interpretation (or
% semantics) $\llbracket \cdot \rrbracket_P$ for arithmetic expressions, which
% returns the string that denotes the concrete syntax of the
% expression. Using merge we can compose the two interpretations to
% obtain a new interpretation that executes both printing and evaluation:
% \jeremy{Explain what is $E\,\&\,P$?}
% 
% \[\begin{array}{lcl}
% \llbracket \cdot \rrbracket_E \otimes \llbracket \cdot \rrbracket_P & = & \llbracket \cdot \rrbracket_{E\,\&\,P} \\
% \end{array}\]

The merge combinator can be manually defined in existing programming languages,
and be used in combination with techniques such as finally tagless or object
algebras. Moreover variants of the merge combinator are useful to
model more complex combinations
of interpretations. A good example are so-called \emph{dependent} interpretations,
where an interpretation does not depend \emph{only} on itself, but also on 
a different interpretation. These definitions with dependencies are quite
common in practice, and, although they are not orthogonal to the interpretation they
depend on, we would like to model them (and also mutually dependent interpretations)
in a modular and compositional style.

% For example consider the following two
% interpretations ($\llbracket \cdot \rrbracket_{\mathsf{Odd}}$ and
% $\llbracket \cdot \rrbracket_{\mathsf{Even}}$) over Peano-style natural numbers:

% \[\begin{array}{lclclcl}
% \llbracket 0 \rrbracket_{\mathsf{Even}}  & = & \mathsf{True} & ~~~~~~~~~~~~~~~~~~~~ & \llbracket 0 \rrbracket_{\mathsf{Odd}} & = & \mathsf{False} \\
% \llbracket S~e \rrbracket_{\mathsf{Even}} & = & \llbracket e \rrbracket_{\mathsf{Odd}} & ~~ & \llbracket S~e \rrbracket_{\mathsf{Odd}} & = & \llbracket e \rrbracket_{\mathsf{Even}}\\
% \end{array}\]

% \emph{Are these interpretations compositional or not?} Under
% a strict definition of compositionality they are not because
% the interpretation of the parts does not depend \emph{only} on the
% interpretation being defined. Instead both interpretations also depend
% on the other interpretation of the parts. In general,
% definitions with dependencies are quite common in practice.
% In this paper we consider these
% interpretations compositional, and we
% would like to model such dependent (or even mutually dependent)
% interpretations in a modular and compositional style.

Defining the merge combinator in existing
programming languages is verbose and cumbersome, requiring code for every
new kind of syntax. Yet, that code is essentially mechanical and ought to be
automated. 
While using advanced meta-programming techniques enables automating
the merge combinator to a large extent in existing programming
languages~\cite{oliveira2013feature, rendel14attributes}, those techniques have
several problems: error messages can be problematic, type-unsafe reflection
is needed in some approaches~\cite{oliveira2013feature} and
advanced type-level features are required in others~\cite{rendel14attributes}.
An alternative to the merge combinator that supports modular multiple
interpretations and works in OO languages with
support for some form of multiple inheritance and covariant
type-refinement of fields has also been recently
proposed~\cite{zhang19shallow}. 
While this approach is relatively simple, it still
requires a lot of manual boilerplate code for composition of interpretations.

This paper presents a calculus and polymorphic type system with
\emph{(disjoint) intersection types}~\cite{oliveira2016disjoint},
called \fnamee. \fnamee
supports our broader notion of compositional designs, and enables
the development of highly modular and reusable programs. \fnamee
has a built-in merge operator and a powerful subtyping relation that
are used to automate the composition of multiple (possibly dependent)
interpretations. In \fnamee subtyping is coercive and enables the
automatic generation of coercions in a \emph{type-directed} fashion. 
This process is similar to that of other type-directed code generation mechanisms
such as 
\emph{type classes}~\cite{Wadler89typeclasses}, which eliminate 
boilerplate code associated to the \emph{dictionary translation}~\cite{Wadler89typeclasses}.

\fnamee continues a line of
research on disjoint intersection types.
 Previous work on
\emph{disjoint polymorphism} (the \fname calculus)~\cite{alpuimdisjoint} studied the
combination of parametric polymorphism and disjoint intersection
types, but its subtyping relation does not support
BCD-style distributivity rules~\cite{Barendregt_1983} and the type system
also prevents unrestricted intersections~\cite{dunfield2014elaborating}. More recently the \name
calculus (or \namee)~\cite{bi_et_al:LIPIcs:2018:9227} introduced a system with \emph{disjoint
  intersection types} and BCD-style distributivity rules, but did not
account for parametric polymorphism. \fnamee is unique in that it
combines all three features in a single calculus:
\emph{disjoint intersection types} and a \emph{merge operator};
\emph{parametric (disjoint) polymorphism}; and a BCD-style subtyping
relation with \emph{distributivity rules}. The three features together
allow us to improve upon the finally tagless and object
algebra approaches and support advanced compositional designs.
Moreover previous work on disjoint intersection types has shown 
various other applications that are also possible in \fnamee, including: \emph{first-class
  traits} and \emph{dynamic inheritance}~\cite{bi_et_al:LIPIcs:2018:9214}, \emph{extensible records} and \emph{dynamic
  mixins}~\cite{alpuimdisjoint}, and \emph{nested composition} and \emph{family polymorphism}~\cite{bi_et_al:LIPIcs:2018:9227}. 


Unfortunately the combination of the three features has non-trivial
complications. The main technical challenge (like for most other
calculi with disjoint intersection types) is the proof of coherence
for \fnamee. Because of the presence of BCD-style distributivity
rules, our coherence proof is based on the recent approach employed in
\namee~\cite{bi_et_al:LIPIcs:2018:9227}, which uses a
\emph{heterogeneous} logical relation called \emph{canonicity}. To account for polymorphism,
which \namee's canonicity does not support, we originally wanted
to incorporate the relevant parts of System~F's logical relation~\cite{reynolds1983types}.
However, due to a mismatch between the two relations, this did not work. The
parametricity relation has been carefully set up with a delayed type
substitution to avoid ill-foundedness due to its impredicative polymorphism.
Unfortunately, canonicity is a heterogeneous relation and needs to account for
cases that cannot be expressed with the delayed substitution setup of the
homogeneous parametricity relation. Therefore, to handle those heterogeneous
cases, we resorted to immediate substitutions and 
% restricted \fnamee to
\emph{predicative instantiations}.
%other
%measures to avoid the ill-foundedness of impredicative instantiation.
%We have settled on restricting \fnamee to \emph{predicative polymorphism} to
%keep the coherence proof manageable. 
We do not believe that predicativity is a severe restriction in practice, since many source
languages (e.g., those based on the Hindley-Milner type system like Haskell and
OCaml) are themselves predicative and do not require the full generality of an
impredicative core language. Should impredicative instantiation be required,
we expect that step-indexing~\cite{ahmed2006step} can be used to recover well-foundedness, though
at the cost of a much more complicated coherence proof.

The formalization and metatheory of \fnamee are a significant advance over that of
\fname. Besides the support for distributive subtyping, \fnamee removes 
several restrictions imposed by the syntactic coherence
proof in \fname. In particular \fnamee supports unrestricted
intersections, which are forbidden in \fname. Unrestricted
intersections enable, for example, encoding certain forms of 
bounded quantification~\cite{pierce1991programming}.
Moreover the new proof method is more robust
with respect to language extensions. For instance, \fnamee supports the bottom
type without significant complications in the proofs, while it was a challenging
open problem in \fname.
A final interesting aspect is that \fnamee's type-checking is decidable. In the
design space of languages with polymorphism and subtyping, similar mechanisms
have been known to lead to undecidability. Pierce's seminal paper
``\emph{Bounded quantification is undecidable}''~\cite{pierce1994bounded} shows
that the contravariant subtyping rule for bounded quantification in
\fsub leads to undecidability of subtyping.  In \fnamee the
contravariant rule for disjoint quantification retains decidability. 
Since with unrestricted intersections \fnamee can express several
use cases of bounded quantification, \fnamee could be an interesting and
decidable alternative to \fsub.

\begin{comment}
Besides coherence, we show
several other important meta-theoretical results, such as type-safety, 
sound and complete algorithmic subtyping, and
decidability of the type system. Remarkably, unlike 
\fsub's \emph{bounded polymorphism}, disjoint polymorphism
in \fnamee supports decidable type-checking.
\end{comment}

In summary the contributions of this paper are:
\begin{itemize}

\item {\bf The \fnamee calculus,} which is the first calculus to combine 
disjoint intersection types, BCD-style distributive subtyping and 
disjoint polymorphism. We show several meta-theoretical results, such as \emph{type-safety}, \emph{sound and complete algorithmic subtyping},
\emph{coherence} and \emph{decidability} of the type system.
\fnamee includes the \emph{bottom type}, which was considered to be a
significant challenge in previous work on disjoint polymorphism~\cite{alpuimdisjoint}.

\item {\bf An extension of the canonicity relation with polymorphism,}
  which enables the proof of coherence of \fnamee. We show that the ideas of
  System F's \emph{parametricity} cannot be ported to
  \fnamee. To overcome the problem we use a technique based on
  immediate substitutions and a predicativity restriction.

% \item {\bf Disjoint intersection types in the presence of bottom:}
%   Our calculus includes the bottom type, which was considered to be a
% significant challenge in previous work on disjoint polymorphism~\cite{alpuimdisjoint}.

\item {\bf Improved compositional designs:} We show that \fnamee's combination of features
enables improved
compositional programming designs and supports automated composition
of interpretations in programming techniques like object algebras and
finally tagless.

\item {\bf Implementation and proofs:} All of the metatheory
  of this paper, except some manual proofs of decidability, has been
  mechanically formalized in Coq. Furthermore, \fnamee is
  implemented and all code presented in the paper is available. The
  implementation, Coq proofs and extended version with appendices can be found in
  \url{https://github.com/bixuanzju/ESOP2019-artifact}.

\end{itemize}

% \bruno{
% Still need to figure out how to integrate row types in the intro story
% Furthermore, we provide a detailed
% comparison between \emph{distributive disjoint polymorphism} and
% \emph{row types}.
% }

% Compositionality is a desirable property in programming
% designs. Broadly defined, compositionality is the principle that a
% system should be built by composing smaller subsystems.
% In the area of programming languages compositionality is
% a key aspect of \emph{denotational semantics}~\cite{scott1971toward, scott1970outline}, where
% the denotation of a program is constructed from denotations of its parts.
% For example, the semantics for a language of simple arithmetic expressions
% is defined as:
% 
% \[\begin{array}{lcl}
% \llbracket n \rrbracket_{E} & = & n \\
% \llbracket e_1 + e_2 \rrbracket_{E} & = & \llbracket e_1 \rrbracket_E + \llbracket  e_2 \rrbracket_E \\
% \end{array}\]
% 
% \bruno{Replace E by fancier symbol?}
% Here there are two forms of expressions: numeric literals and
% additions. The semantics of a numeric literal is just the numeric
% value denoted by that literal. The semantics of addition is the
% addition of the values denoted by the two subexpressions.
% Compositional definitions have many benefits.
% One is ease of reasoning: since compositional
% definitions are recursively defined over smaller elements they
% can typically be reasoned about using induction. Another benefit
% of compositional definitions is that they are easy to extend,
% without modifying previous definitions.
% For example, if we also wanted to support multiplication,
% we could simply define an extra case:
% 
% \[\begin{array}{lcl}
% \llbracket e_1 * e_2 \rrbracket_E & = & \llbracket e_1 \rrbracket_E * \llbracket  e_2 \rrbracket_E \\
% \end{array}\]
% 
% Programming techniques that support compositional
% definitions include:
% \emph{shallow embeddings} of
% Domain Specific Languages (DSLs)~\cite{DBLP:conf/icfp/GibbonsW14}, \emph{finally
%   tagless}~\cite{CARETTE_2009}, \emph{polymorphic embeddings}~\cite{} or
% \emph{object algebras}~\cite{oliveira2012extensibility}. All those techniques allow us to easily create
% compositional definitions, which are easy to extend without
% modifications. Moreover both finally tagless and object algebras
% support \emph{multiple interpretations} (or denotations) of
% the syntax, thus offering a solution to the infamous \emph{Expression Problem}~\cite{wadler1998expression}.
% Because of these benefits they have become
% popular both in the functional and object-oriented
% programming communities.
% 
% However, programming languages often only support simple
% compositional designs well, while language support for more sophisticated
% compositional designs is lacking. Once we have multiple
% interpretations of syntax, then we may wish to compose those
% interpretations. In particular, when multiple interpretations exist, a useful operation
% is a \emph{merge} combinator ($\otimes$) that composes two
% interpretations~\cite{oliveira2012extensibility, oliveira2013feature, rendel14attributes}, forming a
% new interpretation that, when executed, returns the results of both
% interpretations. For example, consider another pretty printing interpretation (or
% semantics) $\llbracket \cdot \rrbracket_P$ for arithmetic expressions, which
% returns the string that denotes the concrete syntax of the
% expression. Using merge we can compose the two interpretations to
% obtain a new interpretation that executes both printing and evaluation:
% \jeremy{Explain what is $E\,\&\,P$?}
% 
% \[\begin{array}{lcl}
% \llbracket \cdot \rrbracket_E \otimes \llbracket \cdot \rrbracket_P & = & \llbracket \cdot \rrbracket_{E\,\&\,P} \\
% \end{array}\]
% 
% Such merge combinator can be manually defined in existing programming 
% The merge combinator can be manually defined in existing programming
% languages, and be used in combination with techniques such as finally
% tagless or object algebras. Furthermore variants of the
% merge combinator can help express more complex combinations of multiple
% interpretations. For example consider the following two
% interpretations ($\llbracket \cdot \rrbracket_{\mathsf{Odd}}$ and
% $\llbracket \cdot \rrbracket_{\mathsf{Even}}$) over Peano-style natural numbers:
% 
% \[\begin{array}{lclclcl}
% \llbracket 0 \rrbracket_{\mathsf{Even}}  & = & \mathsf{True} & ~~~~~~~~~~~~~~~~~~~~ & \llbracket 0 \rrbracket_{\mathsf{Odd}} & = & \mathsf{False} \\
% \llbracket S~e \rrbracket_{\mathsf{Even}} & = & \llbracket e \rrbracket_{\mathsf{Odd}} & ~~ & \llbracket S~e \rrbracket_{\mathsf{Odd}} & = & \llbracket e \rrbracket_{\mathsf{Even}}\\
% \end{array}\]
% 
% \emph{Are these interpretations compositional or not?} Under
% a strict definition of compositionality they are not because
% the interpretation of the parts does not depend \emph{only} on the
% interpretation being defined. Instead both interpretations also depend
% on the other interpretation of the parts. In general,
% definitions with dependencies are quite common in practice.
% In this paper we consider these
% interpretations compositional, and we
% would like to model such dependent (or even mutually dependent)
% interpretations in a modular and compositional style.
% 
% However defining the merge combinator in existing programming
% languages is verbose and cumbersome, and requires code for every new
% kind of syntax. Yet, that code is essentially mechanical and
% ought to be automated. While using advanced meta-programming
% techniques enables automating the merge combinator to a large extent
% in existing programming languages~\cite{oliveira2013feature, rendel14attributes}, those techniques have
% several problems. For example, error messages can be problematic, some
% techniques rely on type-unsafe reflection, while other techniques
% require highly advanced type-level features.
% 
% This paper presents a calculus and polymorphic type system with
% \emph{(disjoint) intersection types}~\cite{oliveira2016disjoint}, called \fnamee, that
% supports our broader notion of compositional designs, and enables
% the development of highly modular and reusable programs. \fnamee
% has a built-in merge operator and a powerful subtyping relation that
% are used to automate the composition of multiple interpretations
% (including dependent interpretations). \fnamee continues a line of
% research on disjoint intersection types. Previous work on
% \emph{disjoint polymorphism} (the \fname calculus) studied the
% combination between parametric polymorphism and disjoint intersection
% types, but the subtyping relation did not support
% BCD-style distributivity rules~\cite{Barendregt_1983}. More recently the \name
% calculus (or \namee) studied a system with \emph{disjoint
%   intersection types} and BCD-style distributivity rules, but did not
% account for parametric polymorphism. \fnamee is unique in that it
% allows the combination of three useful features in a single calculus:
% \emph{disjoint intersection types} and a \emph{merge operator};
% \emph{parametric (disjoint) polymorphism}; and a BCD-style subtyping
% relation with \emph{distributivity rules}. All three features are
% necessary to use improved versions of finally tagless or object
% algebras that support improved compositional designs.
% 
% Unfortunatelly the combination of the three features has non-trivial
% complications. The main technical challenge (as often is the case for
% calculi with disjoint intersection types) is the proof of coherence
% for \fnamee. Because of the presence BCD-style distributivity
% rules, the proof of coherence is based on the approach using a
% \emph{heterogeneous} logical relation employed in
% \namee~\cite{bi_et_al:LIPIcs:2018:9227}. However the logical relation in
% \namee, which we call here \emph{canonicity}, does not
% account for polymorphism. To account for polymorphism we originally
% expected to simply borrow ideas from \emph{parametricity}~\cite{reynolds1983types} in
% System F~\cite{reynolds1974towards} and adapt them to fit with the canonicity relation.
% However, this did not work. The problem is partly due to the fact that
% canonicity (unlike parametricity) is an heterogenous relation and
% needs to account for heterogeneous cases that are not considered in an
% homogeneous relation such as parametricity. Those heterogeneous cases, combined
% with \emph{impredicative polymorphism}, resulted in an ill-founded logical
% relation. Fortunatelly it turns out that
% restricting the calculus to \emph{predicative polymorphism} and using
% an approach based on substitutions is
% sufficient to recover a well-founded canonicity relation.
% Therefore we
% adopted this approach in \fnamee.
% We do not view
% the predicativity restriction as being very severe in practice, since many
% practical languages have such restriction as well. For example languages based
% on Hindley-Milner style type systems (such as Haskell, OCaml or ML)
% \ningning{it's hard to say this is true. When we say Hindley-Milner type system,
%   or Haskell, we are referring to the source language. However, the core
%   language for, for example Haskell, which is System FC, is impredicative.
%   \fnamee is more close to a core language (which usually has explicit type
%   abstractions/applications). In this sense it's unfair to compare it with other
%   source languages.} all use predicative polymorphism. Furthermore with the
% predicativity restriction, the canonicity relation and corresponding proofs
% remain relatively simple and do not require emplying more complex approaches
% such as \emph{step-indexed logical relations}. \ningning{we should emphasize
%   that predicativity is not a restriction, rather it's choice we made in order
%   to prove coherence in Coq. Step-indexed logical relation might work for
%   impredicativity; it's just we don't know.}
% 
% In summary the contributions of this paper are:
% 
% \begin{itemize}
% 
% \item {\bf The \fnamee calculus,} which integrates disjoint intersection types,
%     distributivity and disjoint polymorphism. \fnamee
%     is the first calculus puts all three features together. The
%     combination is non-trivial, expecially with respect to the
%     coherence proof.
% 
% \bruno{improve text}
% \item {\bf The canonicity logical relation,} which enables the proof
%     of coherence of \fnamee. We show that the ideas of
%   System F's \emph{parametricity} cannot be ported to
%   \fnamee. To overcome the problem we develop a canonicity
%   relation that enables a proof of coherence.
% 
% \item {\bf Disjoint intersection types in the presence of bottom:}
%   Our calculus includes a bottom type, which was considered to be a
% significant challenge in previous work.
% 
% \item {\bf Improved compositional designs:} We show how \fnamee has all the
% features that enable improved
% compositional programming designs and support automated composition
% of interpretations in programming techniques like object algebras and
% finally tagless.
% 
% \item {\bf Implementation and proofs:} All proofs
% (including type-safety, coherence and decidability of the type system)
% are proved in the Coq theorem prover. Furthermore \sedel \ningning{where comes the name \sedel?} and
% \fnamee are implemented and all code presented in the paper is
% available. The implementation, proofs and examples can be found in:
% 
% \url{MISSING}
% 
% \end{itemize}
% 
% \bruno{
% Still need to figure out how to integrate row types in the intro story
% Furthermore, we provide a detailed
% comparison between \emph{distributive disjoint polymorphism} and
% \emph{row types}.
% }

% Local Variables:
% org-ref-default-bibliography: "../paper.bib"
% End:


\section{Overview}
\label{sec:overview}

This section aims at introducing first-class classes and traits, their possible
uses and applications, as well as the typing challenges that arise
from their use.
We start by describing a hypothetical JavaScript library for text editing
widgets, inspired and adapted from Racket's GUI
toolkit~\cite{DBLP:conf/oopsla/TakikawaSDTF12}. The example is illustrative of
typical uses of dynamic inheritance/composition, and also the typing challenges
in the presence of first-class classes/traits. Without diving into
technical details, we then give the corresponding typed version in
\name, and informally presents its salient features.

\subsection{First-Class Classes in JavaScript}

A class construct was officially added to JavaScript in the ECMAScript
2015 Language Specification~\cite{EcmaScript:15}. One purpose of
adding classes to JavaScript was to support a construct that is more
familiar to programmers who come from mainstream class-based languages,
such as Java or C++. However classes in JavaScript are
\emph{first-class} and support functionality not easily mimicked in
statically-typed class-based languages.

\subparagraph{Conventional Classes.}
Before diving into the more advanced features of JavaScript classes, we first
review the more conventional class declarations supported in JavaScript as well
as many other languages. Even for conventional classes there are some
interesting points to note about JavaScript that will be important when we move
into a typed setting. An example of a JavaScript class declaration is:
\begin{lstlisting}[language=JavaScript]
class Editor {
  onKey(key) { return "Pressing " + key; }
  doCut()    { return this.onKey("C-x") + " for cutting text"; }
  showHelp() { return "Version: " + this.version() + " Basic usage..."; }
};
\end{lstlisting}
This form of class definition is standard and very similar to declarations in
class-based languages (for example Java). The \lstinline{Editor} class
defines three methods: \lstinline{onKey} for handling key events,
\lstinline{doCut} for cutting text and \lstinline{showHelp} for displaying help
message. For the purpose of demonstration, we elide the actual implementation,
and replace it with plain messages.

We wish to bring the readers' attention to two points in the above class.
Firstly, note that the \lstinline{doCut} method is defined in terms of the
\lstinline{onKey} method via the keyword
\lstinline[language=JavaScript]{this}. In other words the call to
\lstinline{onKey} is enabled by the \emph{self} reference and is
\emph{dynamically dispatched} (i.e., the particular implementation of
\lstinline{onKey} will only be determined when the class or subclass
is instantiated). % Typically an
% OO programmer seeing this definition would expect the \lstinline{doCut} method
% to call the \lstinline{onKey} method of a subclass of \lstinline{Editor}, even though
% the subclass does not exist when the superclass \lstinline{Editor} is being
% defined.
Secondly, notice that there is no definition of
the \lstinline{version} method in the class body, but such method is used inside the
\lstinline{showHelp} method. In a untyped language, such as JavaScript, using
undefined methods is error prone -- accidentally instantiating \lstinline{Editor}
and then calling \lstinline{showHelp} will cause a runtime error!
Statically-typed languages usually provide some means to protect us from this
situation. For example, in Java, we would need an \textit{abstract} \lstinline{version}
method, which effectively makes \lstinline{Editor} an abstract class and
prevents it from being instantiated. As we will see, \name's treatment of
abstract methods is quite different from mainstream languages. In fact, \name
has a unified (typing) mechanism for dealing with both dynamic dispatch and abstract
methods. We will describe \name's mechanism for dealing with both features and
justify our design in \cref{sec:traits}.

% A couple of things worth pointing out in the above code snippet: (1) the class
% \lstinline{Editor} has no definition of the method
% \lstinline{version}, but such method
% is used in the body of the method \lstinline{showHelp}. In a strongly-typed OO
% language, such as Java, we would need to define an abstract method for
% \lstinline{version}. (2) The \lstinline{Editor} class requires
% \emph{dynamic dispatching}.
%  In the body of the method \lstinline{doCut} we invoke
% the method \lstinline{onKey} defined in the same class through the keyword
% \lstinline[language=JavaScript]{this}. This has the implication that when a
% subclass of \lstinline{Editor} overrides the method \lstinline{onKey}, a call to
% \lstinline{doCut} should invoke \lstinline{onKey} defined in the subclass
% instead of the original one.\bruno{punchline?}
%As we will see later, the type system of \name correctly handles it.

\subparagraph{First-Class Classes and Class Expressions.}
Another way to define a class in JavaScript is via a \emph{class expression}. This is where the class
model in JavaScript is very different from the traditional class model found in
many mainstream OO languages, such as Java, where classes are second-class
(static) entities. JavaScript embraces a dynamic class model that treats classes
as \emph{first-class} expressions: a function can take classes as arguments,
or return them as a result. First-class classes enable programmers to
abstract over patterns in the class hierarchy and to experiment with new forms of OOP
such as mixins and traits. In particular, mixins become programmer-defined
constructs. We illustrate this by presenting a simple mixin that adds
spell checking to an editor:
\begin{lstlisting}[language=JavaScript]
const spellMixin = Base => {
  return class extends Base {
    check()    { return super.onKey("C-c") + " for spell checking"; }
    onKey(key) { return "Process " + key + " on spell editor"; }
  }
};
\end{lstlisting}
In JavaScript, a mixin is simply a function with a superclass as input and a
subclass extending that superclass as an output. Concretely, \lstinline{spellMixin}
adds a method \lstinline{check} for spell checking. It also provides
a method \lstinline{onKey}.
The function \lstinline{spellMixin} shows the typical use of what we call \emph{dynamic inheritance}.
Note that \lstinline{Base}, which is supposed to be a superclass being inherited, is \emph{parameterized}.
Therefore \lstinline{spellMixin} can be applied to any base class at
\emph{runtime}. This is impossible to do, in a type-safe way, in
conventional statically-typed class-based languages like Java or
C++.\footnote{With C++ templates, it is possible to
  implement a so-called mixin pattern~\cite{DBLP:conf/gcse/SmaragdakisB00}, which enables extending
a parameterized class. However C++ templates defer type-checking until
instantiation, and such pattern still does not allow selection of the
base class at runtime (only at up to class instantiation time).}

It is noteworthy that not all applications of \lstinline{spellMixin} to base
classes are successful. Notice the use of the \lstinline{super} keyword in the
\lstinline{check} method. If the base class does not implement the
\lstinline{onKey} method, then mixin application fails with a runtime error. In
a typed setting, a type system must express this requirement (i.e., the presence of
the \lstinline{onKey} method) on the (statically unknown) base class that is
being inherited.


% The class expression inside the function body has no
% definition of the method \lstinline{version}, but which is used in the body of
% the method \lstinline{showHelp}. In a statically-typed OO language, such as Java,
% we would need an \emph{abstract method} for
% \lstinline{version}.


We invite the readers to pause for a while and think about what the type of
\lstinline{spellMixin} would look like. Clearly our type system should be
flexible enough to express this kind of dynamic pattern of composition in order
to accommodate mixins (or traits), but also not too lenient to allow any
composition.


\subparagraph{Mixin Composition and Conflicts.}
The real power of mixins is that \lstinline{spellMixin}'s functionality is not
tied to a particular class hierarchy and is composable with other features. For
example, we can define another mixin that adds simple modal editing -- as in Vim
-- to an arbitrary editor:
\begin{lstlisting}[language=JavaScript]
const modalMixin = Base => {
  return class extends Base {
    constructor() {
      super();
      this.mode = "command";
    }
    toggleMode() { return "toggle succeeded"; }
    onKey(key)   { return "Process " + key + " on modal editor"; }
  };
};
\end{lstlisting}
\lstinline{modalMixin} adds a \lstinline{mode} field that controls which
keybindings are active, initially set to the command mode, and a method
\lstinline{toggleMode} that is used to switch between modes. It also provides a method \lstinline{onKey}.

Now we can compose \lstinline{spellMixin} with \lstinline{modalMixin} to produce
a combination of functionality, mimicking some form of multiple inheritance:
\begin{lstlisting}[language=JavaScript]
class IDEEditor extends modalMixin(spellMixin(Editor)) {
  version() { return 0.2; }
}
\end{lstlisting}
The class \lstinline{IDEEditor} extends the base class \lstinline{Editor} with
modal editing and spell checking capabilities. It also defines the missing
\lstinline{version} method.

At first glance, \lstinline{IDEEditor} looks quite fine, but it has a subtle
issue. Recall that two mixins \lstinline{modalMixin} and \lstinline{spellMixin}
both provide a method \lstinline{onKey}, and the \lstinline{Editor} class also
defines an \lstinline{onKey} method of its own. Now we have a name clash. A
question arises as to which one gets picked inside the \lstinline{IDEEditor}
class. A typical mixin model resolves this issue by looking at the order of mixin applications. Mixins appearing later in the order
overrides \emph{all} the identically named methods of earlier mixins. So in our
case, \lstinline{onKey} in \lstinline{modalMixin} gets picked. If we
change the order of application to \lstinline{spellMixin(modalMixin(Editor))},
then \lstinline{onKey} in \lstinline{spellMixin} is inherited.

\subparagraph{Problem of Mixin Composition.}
From the above discussion, we can see that mixin are composed linearly: all the
mixins used by a class must be applied one at a time. However, when we wish to
resolve conflicts by selecting features from different mixins, we may not be
able to find a suitable order. For example, when we compose the two mixins to
make the class \lstinline{IDEEditor}, we can choose which of them comes first,
but in either order, \lstinline{IDEEditor} cannot access to the \lstinline{onKey}
method in the \lstinline{Editor} class.

\subparagraph{Trait Model.}
Because of the total ordering and the limited means for resolving conflicts imposed by the mixin model,
researchers have proposed a simple compositional model called
traits~\cite{scharli2003traits, Ducasse_2006}. Traits are lightweight entities and serve as
the primitive units of code reuse. Among others, the key difference from
mixins is that the order of trait composition is irrelevant, and conflicting
methods must be resolved \emph{explicitly}. This gives programmers
fine-grained control, when conflicts arise, of selecting desired features from
different components. Thus we believe traits are a better model for multiple
inheritance in statically-typed OO languages, and in \name we realize this
vision by giving traits a first-class status in the language,
achieving more expressive power compared with traditional (second-class) traits.


\subparagraph{Summary of Typing Challenges.}
From our previous discussion, we can identify the following typing challenges
for a type system to accommodate the programming patterns (first-class classes/mixins)
we have just seen in a typed setting:
\begin{itemize}
\item How to account for, in a typed way, abstract methods and dynamic dispatch.
\item What are the types of first-class classes or mixins.
\item How to type dynamic inheritance.
\item How to express constraints on method presence and absence (the use of
  \lstinline{super} clearly demands that).
% \item How to ensure that composition of mixins is going to be valid, i.e., how
%   to reflect linearity in a type system.
\item In the presence of first-class traits, how to detect conflicts statically,
  even when the traits involved are not statically known.
\end{itemize}
\name elegantly solves the above challenges in a unified way, as
we will see next.


% From a pragmatic point of view, this implicit conflict resolution
% sometimes give programmers more surprises than convenience. What if the compiler can alarm us when a
% potential conflict may occur. Because of the dynamic nature of JavaScript, we
% would not know before actually running the code that there is a conflict. We
% miss the guarantee that a static type system can provide: such conflict can be
% detected at compile-time.

% Given the flexibility of first-class classes in dynamically-typed languages, we
% -- being advocates of statically-typed languages -- were wondering how to
% incorporate this same expressive power into statically-typed
% languages. As it
% turns out, designing a sound type system that fully supports first-class classes
% is notoriously hard; there are only a few, quite sophisticated, languages that
% manage this~\cite{DBLP:conf/oopsla/TakikawaSDTF12, DBLP:conf/ecoop/LeeASP15}. We
% pushed it further: \name has support for typed first-class
% traits.\bruno{Better to say there's no work on typed first-class
%   traits, and little work on first-class classes/mixins, despite
%  many dynamic languages prominently supporting such features.}

\subsection{A Glance at Typed First-Class Traits in \name}

We now rewrite the above library in \name, but this time with types. The resulting code has the same functionality as the dynamic version, but is
statically typed. All code snippets in this and later sections are runnable in
our prototype implementation. Before proceeding, we ask the readers to bear in mind that in this section we are not using traits
in the most canonical way, i.e., we use traits as if they are classes (but with
built-in conflict detection). This is because we are trying to stay as close as possible
to the structure of the JavaScript version for ease of comparison. In
\cref{sec:traits} we will remedy this to make better use of traits.

\subparagraph{Simple Traits.}
Below is a simple trait \lstinline{editor}, which corresponds to the JavaScript
class \lstinline{Editor}. The \lstinline{editor} trait defines the same set of
methods: \lstinline{on_key}, \lstinline{do_cut} and \lstinline{show_help}:
\lstinputlisting[linerange=14-18]{../../examples/overview2.sl}% APPLY:linerange=OVERVIEW_EDITOR
The first thing to notice is that \name uses a syntax (similar to Scala's
self type annotations~\cite{odersky2004overview}) where we can give a type annotation to the
\lstinline{self} reference. In the type of \lstinline{self} we use
\lstinline{&} construct to create intersection types. \lstinline{Editor} and \lstinline{Version} are two record types:
\lstinputlisting[linerange=7-8]{../../examples/overview2.sl}% APPLY:linerange=OVERVIEW_EDITOR_TYPES
For the sake of conciseness, \name uses \lstinline{type} aliases to abbreviate types.

\subparagraph{Self-Types Encode Abstract Methods.}
Recall that in the JavaScript class \lstinline{Editor}, the \lstinline{version}
method is undefined, but is used inside \lstinline{showHelp}. How can we express
this in the typed setting, if not with an abstract method? In \name, self-types
play the role of trait requirements. As the first approximation, we
can justify the use of \lstinline{self.version} by noticing that (part of) the
type of \lstinline{self} (i.e., \lstinline{Version}) contains the declaration of
\lstinline{version}. An interesting aspect of \name's trait model is that there
is no need for abstract methods. Instead, abstract methods can be simulated as
requirements of a trait. Later, when the trait is composed with other
traits, \emph{all} requirements on the self-types must be
satisfied and one of the traits in the composition must provide an
implementation of the method \lstinline{version}.
%to this point in \cref{sec:traits}.

As in the JavaScript version, the \lstinline{on_key} method is invoked on
\lstinline{self} in the body of \lstinline{do_cut}. This is allowed as (part of)
the type of \lstinline{self} (i.e., \lstinline{Editor}) contains the signature
of \lstinline{on_key}. Comparing \lstinline{editor} to the JavaScript class
\lstinline{Editor}, almost everything stays the same, except that we now have
the typed version. As a side note, since \name is currently a pure functional OO
language, there is no difference between fields and methods, so we can omit
empty arguments and parameter parentheses.

\subparagraph{First-Class Traits and Trait Expressions.}

\name treats traits as first-class expressions, putting them in the same
syntactic category as objects, functions, and other primitive forms. To
illustrate this, we give the \name version of \lstinline{spellMixin}:
\lstinputlisting[linerange=22-29]{../../examples/overview2.sl}% APPLY:linerange=OVERVIEW_HELP
This looks daunting at first, but \lstinline{spell_mixin} has almost the same structure as
its JavaScript cousin \lstinline{spellMixin}, albeit with
some type annotations. In \name, we use capital letters (\lstinline{A}, \lstinline{B}, $\dots$) to denote type variables, and trait
expressions \lstinline$trait [self : ...] inherits ... => {...}$ to create
first-class traits. Trait expressions have trait
types of the form \lstinline{Trait[T1, T2]} where \lstinline{T1} and \lstinline{T2} denote trait requirements and functionality respectively.
We will explain trait types in \cref{sec:traits}. Despite the structural similarities, there are several significant
features that are unique to \name (e.g., the disjointness operator \lstinline{*}).
We discuss these in the following.



\subparagraph{Disjoint Polymorphism and Conflict Detection.}

\name uses a type system based on \emph{disjoint intersection types}~\cite{oliveira2016disjoint} and
\emph{disjoint polymorphism}~\cite{alpuimdisjoint}. Disjoint intersections
empower \name to detect conflicts statically when trying to compose two
traits with identically named features. For example composing two traits
\lstinline{a} and \lstinline{b} that both provide \lstinline{foo} gives a
type error (the overloaded \lstinline{&} operator denotes trait composition):
\begin{lstlisting}
trait a => { foo = 1 };
trait b => { foo = 2 };
trait c inherits a & b => {}; -- type error!
\end{lstlisting}
Disjoint polymorphism, as a more advanced mechanism, allows detecting conflicts
even in the presence of polymorphism -- for example when a trait is parameterized and its
full set of methods is not statically known. As can be seen,
\lstinline{spell_mixin} is actually a polymorphic function. Unlike ordinary
parametric polymorphism, in \name, a type variable can also have a disjointness
constraint. For instance, \lstinline{A * Spelling & OnKey}
means that \lstinline{A} can be instantiated to any type as long as it \emph{does not}
contain \lstinline{check} and \lstinline{on_key}. To mimic mixins, the
argument \lstinline{base}, which is supposed to be some trait, serves as the
``base'' trait that is being inherited. Notice that the type variable
\lstinline{A} appears in the type of \lstinline{base}, which essentially states
that \lstinline{base} is a trait that contains at least those methods specified
by \lstinline{Editor}, and possibly more (which we do not know statically).
% In summary, \lstinline{Trait[Editor & Version, Editor & A]} (the assigned type
% of \lstinline{base}) specifies that both method \emph{presence} and \emph{absence}.
Also note that leaving out the \lstinline{override} keyword will result in a
type error. The type system is forcing us to be very specific as to what is the
intention of the \lstinline{on_key} method because it sees the same method is
also declared in \lstinline{base}, and blindly inheriting \lstinline{base}
will definitely cause a method conflict. As a final note, the use of \lstinline{super}
inside \lstinline{check} is allowed because the ``super'' trait \lstinline{base}
implements \lstinline{on_key}, as can be seen from its type.


\subparagraph{Dynamic Inheritance.}

Disjoint polymorphism enables us to correctly type dynamic inheritance:
\lstinline{spell_mixin} is able to take any trait that conforms with its
assigned type, equips it with the \lstinline{check} method and overrides its
old \lstinline{on_key} method. As a side note, the use of disjoint polymorphism
is essential to correctly model the mixin semantics. From the type we know
\lstinline{base} has some features specified by \lstinline{Editor}, plus
something more denoted by \lstinline{A}. By inheriting \lstinline{base}, we are
guaranteed that the result trait will have everything that is already contained
in \lstinline{base}, plus more features. This is in some sense similar to row
polymorphism~\cite{wand1994type} in that the result trait is prohibited from
forgetting methods from the argument trait. As we will discuss in
\cref{sec:related}, disjoint polymorphism is more expressive than row
polymorphism.


\subparagraph{Typing Mixin Composition.}
Next we give the typed version of \lstinline{modalMixin} as follows:
\lstinputlisting[linerange=34-41]{../../examples/overview2.sl}% APPLY:linerange=OVERVIEW_MODAL
Now the definition of \lstinline{modal_mixin} should be self-explanatory.
Finally we can apply both ``mixins'' one by one to \lstinline{editor} to create
a concrete editor:
\lstinputlisting[linerange=46-49]{../../examples/overview2.sl}% APPLY:linerange=OVERVIEW_LINE
As with the JavaScript version, we need to fill in the missing
\lstinline{version} method. It is easy to verify that the \lstinline{on_key} method
in \lstinline{modal_mixin} is inherited. Compared with the untyped version,
here this behaviour is reasonable because in each mixin we specifically tags the
\lstinline{on_key} method to be an overriding method. Let us take a close look
at the mixin applications. Since \name is currently explicitly typed, we need to
provide concrete types when using \lstinline{modal_mixin} and \lstinline{spell_mixin}.
In the inner application (\lstinline{spell_mixin Top editor}), we use the top
type \lstinline{Top} to instantiate \lstinline{A} because the \lstinline{editor} trait
provides exactly those method specified by \lstinline{Editor} and nothing more
(hence \lstinline{Top}). In the outer application, we use \lstinline{Spelling}
to instantiate \lstinline{A}. This is where implicit conflict resolution of
mixins happens. We know the result of the inner application actually forms a
trait that provides both \lstinline{check} and \lstinline{on_key}, but the
disjointness constraint of \lstinline{A} requires the absence of \lstinline{on_key},
thus we cannot instantiate \lstinline{A} to \lstinline{Spelling & OnKey} for example
when applying \lstinline{modal_mixin}. Therefore the outer application effectively excludes
\lstinline{on_key} from \lstinline{spell_mixin}.
In summary, the order of mixin applications is reflected by the order
of function applications, and conflict resolution code is implicitly embedded.
Of course changing the mixin application order to \lstinline{spell_mixin ModalEdit (modal_mixin Top editor)} gives the expected behaviour.


Admittedly the typed version is unnecessarily complicated as we were
mimicking mixins by functions over traits. The final editor
\lstinline{ide_editor} suffers from the same problem as the class
\lstinline{IDEEditor}, since there is no obvious way to access the
\lstinline{on_key} method in the \lstinline{editor} trait.\footnote{In fact, as
  we will see in \cref{sec:traits}, we can still access \lstinline{on_key} in
  \lstinline{editor} by the forwarding operator.} \cref{sec:traits}
makes better use of traits to simplify the editor code.



% Note that the use of \lstinline{override} is valid because the type system knows the inherit clause contains \lstinline{on_key}.
% As a bonus, since \name guarantees that there are no potential conflicts in a program,
% we can reason that the version number in \lstinline{modal_editor} is
% \lstinline{0.1}.

%%% Local Variables:
%%% mode: latex
%%% TeX-master: "../paper"
%%% org-ref-default-bibliography: ../paper.bib
%%% End:


\section{Declarative System}


\begin{center}
\begin{tabular}{lrcl} \toprule
  Types & $[[A]], [[B]]$ & \syndef & $[[int]] \mid [[a]] \mid [[A -> B]] \mid [[\/ a. A]] \mid [[unknown]] \mid [[static]] \mid [[gradual]] $ \\
  Monotypes & $[[t]], [[s]]$ & \syndef & $ [[int]] \mid [[a]] \mid [[t -> s]] \mid [[static]] \mid [[gradual]]$ \\
  Castable Types & $[[gc]]$ & \syndef & $ [[int]] \mid [[a]] \mid [[gc1 -> gc2]] \mid [[\/ a. gc]] \mid [[unknown]] \mid [[gradual]] $ \\
  Castable Monotypes & $[[tc]]$ & \syndef & $ [[int]] \mid [[a]] \mid [[tc1 -> tc2]] \mid [[gradual]]$ \\

  Contexts & $[[dd]]$ & \syndef & $[[empty]] \mid [[dd, x: A]] \mid [[dd, a]] $ \\
  Colored Types & $[[A]], [[B]]$ & \syndef & $ [[r@(int)]] \mid [[b@(int)]] \mid [[r@(a)]] \mid [[b@(a)]] \mid [[A -> B]] \mid [[r@ \/ a . A]] \mid [[b@ \/ a. A]] \mid [[b@(unknown)]] \mid [[r@(static)]] \mid [[r@(gradual)]] \mid [[b@(gradual)]]$\\
  Blue Castable Types & $[[b@(gc)]]$ & \syndef & $ [[b@(int)]] \mid [[b@(a)]] \mid [[b@(gc1) -> b@(gc2)]] \mid [[b@ \/ a. b@(gc)]] \mid [[b@(unknown)]] \mid [[b@(gradual)]] $ \\
  Blue Monotypes & $[[b@(t)]]$ & \syndef & $ [[b@(int)]] \mid [[b@(a)]] \mid [[b@(t -> s)]] \mid [[b@(gradual)]]$ \\
  Red Monotypes & $[[r@(t)]]$ & \syndef & $ [[r@(int)]] \mid [[r@(a)]] \mid [[ r@(t)  -> r@(s)]] \mid [[ r@(t) -> b@(s) ]] \mid [[ b@(t) ->  r@(s) ]] \mid [[r@(static)]] \mid [[r@(gradual)]]$ \\
  \bottomrule
\end{tabular}
\end{center}


\renewcommand\ottaltinferrule[4]{
  \inferrule*[narrower=0.7]
    {#3}
    {#4}
}

\drules[dconsist]{$ [[ A ~ B ]] $}{Type Consistent}{refl, unknownR, unknownL, arrow, forall}

\renewcommand\ottaltinferrule[4]{
  \inferrule*[narrower=0.7,right=\scriptsize{#1}]
    {#3}
    {#4}
}

\drules[s]{$ [[dd |- A <: B ]] $}{Subtyping}{forallR, forallLr, forallLb, tvarr, tvarb, intr, intb, arrow,
  unknown, spar, gparr, gparb}


% \begin{definition}[Specification of Consistent Subtyping]
%   \begin{mathpar}
%   \drule{cs-spec}
%   \end{mathpar}
% \end{definition}

\drules[cs]{$ [[dd |- A <~ B ]] $}{Consistent Subtyping}{forallR, forallL, arrow, tvar, int, unknownL, unknownR, spar, gpar}

\drules[]{$ [[dd |- e : A ~~> pe]] $}{Typing}{var, int, gen, lamann, lam, app}

\drules[m]{$ [[dd |- A |> A1 -> A2]] $}{Matching}{forall, arr, unknown}


\section{Target: PBC}

\begin{center}
\begin{tabular}{lrcl} \toprule
  Terms & $[[pe]]$ & \syndef & $[[x]] \mid [[n]] \mid [[\x : A. pe]] \mid [[/\a. pe]] \mid [[pe1 pe2]] \mid [[<A `-> B> pe]] $
  \\ \bottomrule
\end{tabular}
\end{center}


\clearpage
\section{Metatheory}

% \renewcommand{\hlmath}{}

\begin{definition}[Substitution]
  \begin{enumerate}
    \item Gradual type parameter substitution $\gsubst :: [[gradual]] \to [[tc]]$
    \item Static type parameter substitution $\ssubst :: [[static]] \to [[t]]$
    \item Type parameter Substitution $\psubst = \gsubst \cup \ssubst$
  \end{enumerate}
\end{definition}

\ningning{Note substitution ranges are monotypes.}

\begin{definition}[Translation Pre-order]
  Suppose $[[dd |- e : A ~~> pe1]]$ and $[[dd |- e : A ~~> pe2]]$,
  we define $[[pe1]] \leq [[pe2]]$ to mean $[[pe2]] = [[S(pe1)]]$ for
  some $[[S]]$.
\end{definition}


\begin{proposition}
  If $[[ pe1 ]] \leq [[pe2]]$ and $[[ pe2 ]] \leq [[pe1]]$, then $[[pe1]]$ and $[[pe2]]$
  are equal up to $\alpha$-renaming of type parameters.
\end{proposition}

 
\begin{definition}[Representative Translation]
  $[[pe]]$ is a representative translation of a typing derivation $[[dd |- e : A
  ~~> pe]]$ if and only if for any other translation $[[dd |- e : A ~~> pe']]$ such that $[[pe']]
  \leq [[pe]]$, we have $[[pe]] \leq [[pe']]$. From now on we use $[[rpe]]$ to
  denote a representative translation.
\end{definition}

\begin{definition}[Measurements of Translation]
  There are three measurements of a translation $[[pe]]$,
  \begin{enumerate}
  \item $[[ ||pe||e]]$, the size of the expression 
  \item $[[ ||pe||s ]]$, the number of distinct static type parameters in $[[pe]]$
  \item $[[ ||pe||g ]]$, the number of distinct gradual type parameters in $[[pe]]$
  \end{enumerate}
  We use $[[ ||pe|| ]]$ to denote the lexicographical order of the triple
  $([[ ||pe||e ]], -[[ ||pe||s ]], -[[ ||pe||g ]])$.
\end{definition}

\begin{definition}[Size of types]

  \begin{align*}
    [[ || int ||  ]] &= 1 \\
    [[ || a ||  ]] &= 1 \\
    [[ || A -> B  ||  ]] &= [[ || A || ]] + [[ || B || ]] + 1 \\
    [[ || \/a . A ||  ]] &= [[ || A || ]] + 1 \\
    [[ || unknown ||  ]] &= 1 \\
    [[ || static ||  ]] &= 1 \\
    [[ || gradual ||  ]] &= 1
  \end{align*}

\end{definition}


\begin{definition}[Size of expressions]

  \begin{align*}
    [[ || x ||e  ]] &= 1 \\
    [[ || n ||e  ]] &= 1 \\
    [[ || \x : A . pe ||e  ]] &= [[ || A || ]] + [[ || pe ||e ]] + 1 \\
    [[ || /\ a. pe ||e  ]] &= [[ || pe ||e ]] + 1 \\
    [[ || pe1 pe2 ||e  ]] &= [[ || pe1 ||e ]] + [[  || pe2 ||e ]] + 1 \\
    [[ || < A `-> B> pe ||e  ]] &= [[ || pe ||e ]] + [[  || A || ]] + [[  || B || ]] + 1 \\
  \end{align*}

\end{definition}


\begin{lemma} \label{lemma:size_e}
  If $[[dd |- e : A ~~> pe]]$ then $[[ || pe ||e    ]] \geq [[ || e ||e   ]]  $.
\end{lemma}
\begin{proof}
  Immediate by inspecting each typing rule.
\end{proof}

\begin{corollary} \label{lemma:decrease_stop}
  If $[[dd |- e : A ~~> pe]]$ then $[[ || pe ||   ]] > ([[ || e ||e ]], -[[ || e ||e ]], -[[ || e ||e ]] )  $.
\end{corollary}
\begin{proof}
  By \cref{lemma:size_e} and note that $ [[ || pe ||e   ]] > [[  || pe ||s  ]] $ and $ [[ || pe ||e   ]] > [[  || pe ||g  ]] $
\end{proof}


\begin{lemma} \label{lemma:type_decrease}
  $[[ || A || ]] \leq [[ || S(A) || ]]  $.
\end{lemma}
\begin{proof}
  By induction on the structure of $[[A]]$. The interesting cases are $[[ A ]] = [[static]]$ and
  $[[ A ]] = [[gradual]]$. When $[[ A ]] = [[static]]$, $[[ S(A) ]] = [[t]]$
  for some monotype $[[t]]$ and it is immediate that $[[ || static ||  ]]  \leq [[ || t || ]] $
  (note that $[[ || static ||  ]] < [[ || gradual ||  ]] $ by definition).
\end{proof}


\begin{lemma}[Substitution Decreases Measurement]
  \label{lemma:subst_dec_measure}
  If $[[pe1]] \leq [[pe2]]$, then $ {[[ ||pe1|| ]]} \leq [[ ||pe2|| ]]$; unless
  $[[pe2]] \leq [[pe1]]$ also holds, otherwise we have $[[ ||pe1|| ]] < [[ ||pe2|| ]]$.
\end{lemma}
\begin{proof}
  Since $[[ pe1  ]] \leq [[  pe2  ]]$, we know $[[ pe2  ]] = [[ S(pe1)  ]]$ for some $[[S]]$. By induction on
  the structure of $[[pe1]]$.

  \begin{itemize}
  \item Case $[[pe1]] = [[  \x : A . pe ]]$. We have
    $[[ pe2  ]] = [[  \x : S(A) . S(pe)  ]]$. By \cref{lemma:type_decrease} we have $[[ || A || ]] \leq [[ || S(A) || ]]$.
    By i.h., we have $[[ || pe ||  ]] \leq [[ || S(pe) ||  ]]$. Therefore $[[ || \x : A . pe ||    ]] \leq [[ || \x : S(A) . S(pe) ||  ]]$.
  \item Case $[[pe1]] = [[ < A `-> B > pe  ]]$. We have
    $[[pe2]] = [[ < S(A) `-> S(B) > S(pe)  ]]$.  By \cref{lemma:type_decrease} we have $[[ || A || ]] \leq [[ || S(A) || ]]$
    and $[[ || B || ]] \leq [[ || S(B) || ]]$. By i.h., we have $[[ || pe ||  ]] \leq [[ || S(pe) ||  ]]$.
    Therefore $[[  || < A `-> B > pe ||  ]] \leq [[ || < S(A) `-> S(B) > S(pe)  ||   ]]$.

  \item The rest of cases are immediate.
  \end{itemize}

\end{proof}


\begin{lemma}[Representative Translation for Typing]
  For any typing derivation that $[[dd |- e : A]]$, there exists at least one representative
  translation $r$ such that $[[dd |- e : A ~~> rpe]]$.
\end{lemma}
\begin{proof}
We already know that at least one translation $[[pe]] = [[pe1]]$ exists
for every typing derivation. If $[[pe1]]$ is a representative translation then we
are done. Otherwise there exists another translation $[[pe2]]$ such that
$[[pe2]] \leq [[pe1]]$ and $ [[pe1]] \not \leq [[pe2]]$. By
\cref{lemma:subst_dec_measure}, we have $[[||pe2||]] < [[ ||pe1|| ]]$. We continue
with $[[pe]] = [[pe2]]$, and get a strictly decreasing sequence $[[ || pe1 ||  ]], [[ || pe2 || ]], \dots$.
By \cref{lemma:decrease_stop}, we know this sequence cannot be infinite long. Suppose it ends at $[[ || pen || ]]$,
by the construction of the sequence, we know that $[[pen]]$ is a representative translation of $[[e]]$.
\end{proof}


\begin{conjecture}[Property of Representative Translation] \label{lemma:repr}
  If $[[empty |- e : A ~~> pe]]$, $\erasetp s \Downarrow v$, then we
  have $[[empty |- e : A ~~> rpe]]$, and $\erasetp r \Downarrow v'$.
\end{conjecture}

\ningning{shall we focus on values of type integer?}

\begin{definition}[Erasure of Type Parameters]
  \begin{center}
\begin{tabular}{p{5cm}l}
  $\erasetp \nat = \nat $ &
  $\erasetp a = a $ \\
  $\erasetp {A \to B} = \erasetp A \to \erasetp B $ &
  $\erasetp {\forall a. A} = \forall a. \erasetp A$ \\
  $\erasetp {\unknown} = \unknown  $&
  $\erasetp {\static} = \nat  $\\
  $\erasetp {\gradual} = \unknown  $\\
\end{tabular}

  \end{center}
\end{definition}


\begin{corollary}[Coherence up to cast errors]
  Suppose $[[ empty |- e : int ~~> pe1 ]]$ and $[[ empty |- e : int ~~> pe2 ]]$, if $| [[ pe1 ]] | \Downarrow [[n]]$
  then either $ | [[  pe2  ]] | \Downarrow n$ or $ | [[  pe2  ]] | \Downarrow \blamev$.
\end{corollary}
\jeremy{maybe Conjecture~\ref{lemma:repr} is enough to prove it? }


\begin{conjecture}[Dynamic Gradual Guarantee]
  Suppose $e' \lessp e$,
  \begin{enumerate}
  \item If $[[empty |- e : A ~~> rpe]]$, $\erasetp {r} \Downarrow v$,
    then for some $B$ and $r'$, we have $[[ empty |- e' : B ~~> rpe']]$,
    and $B \lessp A$,
    and $\erasetp {r'} \Downarrow v'$,
    and $v' \lessp v$.
  \item If $[[empty |- e' : B ~~> rpe']]$, $\erasetp {r'} \Downarrow v'$,
    then for some $A$ and $[[rpe]]$, we have $ [[empty |- e : A ~~> rpe]]$,
    and $B \lessp A$. Moreover,
    $\erasetp r \Downarrow v$ and $v' \lessp v$,
    or $\erasetp r \Downarrow \blamev$.
  \end{enumerate}
\end{conjecture}



\section{Efficient (Almost) Typed Encodings of ADTs}


\begin{itemize}
\item Scott encodings of simple first-order ADTs (e.g. naturals)
\item Parigot encodings improves Scott encodings with recursive schemes, but
  occupies exponential space, whereas Church encoding only occupies linear
  space.
\item An alternative encoding which retains constant-time destructors but also
  occupies linear space.
\item Parametric ADTs also possible?
\item Typing rules
\end{itemize}

\begin{example}[Scott Encoding of Naturals]
\begin{align*}
  [[nat]] &\triangleq [[  \/a. a -> (unknown -> a) -> a ]] \\
  \mathsf{zero} &\triangleq [[ \x . \f . x  ]] \\
  \mathsf{succ} &\triangleq [[ \y : nat . \x . \f . f y ]]
\end{align*}
\end{example}
Scott encodings give constant-time destructors (e.g., predecessor), but one has to
get recursion somewhere. Since our calculus admits untyped lambda calculus, we
could use a fixed point combinator.

\begin{example}[Parigot Encoding of Naturals]
\begin{align*}
  [[nat]] &\triangleq [[  \/a. a -> (unknown -> a -> a) -> a ]] \\
  \mathsf{zero} &\triangleq [[ \x . \f . x  ]] \\
  \mathsf{succ} &\triangleq [[ \y : nat . \x . \f . f y (y x f) ]]
\end{align*}
\end{example}
Parigot encodings give primitive recursion, apart form constant-time
destructors, but at the cost of exponential space complexity (notice in
$\mathsf{succ}$ there are two occurances of $[[y]]$).

Both Scott and Parigot encodings are typable in System F with positive recursive
types, which is strong normalizing.

\begin{example}[Alternative Encoding of Naturals]
\begin{align*}
  [[nat]] &\triangleq [[  \/a. a -> (unknown -> (unknown -> a) -> a) -> a ]] \\
  \mathsf{zero} &\triangleq [[ \x . \f . x  ]] \\
  \mathsf{succ} &\triangleq [[ \y : nat . \x . \f .  f y (\g . g x f) ]]
\end{align*}
\end{example}
This encoding enjoys constant-time destructors, linear space complexity, and
primitive recursion.
The static version is $[[ mu b . \/ a . a -> (b -> (b -> a) -> a) -> a ]]$,
which can only be expressed in System F with
general recursive types (notice the second $[[b]]$ appears in a negative position).





\section{Algorithmic System}

\begin{center}
\begin{tabular}{lrcl} \toprule
  Expressions & $[[ae]]$ & \syndef & $[[x]] \mid [[n]] \mid [[\x : aA . ae]] \mid [[\x . ae]] \mid [[ae1 ae2]] \mid [[ae : aA]] $ \\
  Existential variables & $[[evar]]$ & \syndef & $[[sa]]  \mid [[ga]]  $   \\
  Types & $[[aA]], [[aB]]$ & \syndef & $ [[int]] \mid [[a]] \mid [[evar]] \mid [[aA -> aB]] \mid [[\/ a. aA]] \mid [[unknown]] \mid [[static]] \mid [[gradual]] $ \\
  Static Types & $[[aT]]$ & \syndef & $ [[int]] \mid [[a]] \mid [[evar]] \mid [[aT1 -> aT2]] \mid [[\/ a. aT]] \mid [[static]] \mid [[gradual]] $ \\
  Monotypes & $[[at]], [[as]]$ & \syndef & $ [[int]] \mid [[a]] \mid [[evar]] \mid [[at -> as]] \mid [[static]] \mid [[gradual]]$ \\
  Castable Monotypes & $[[atc]]$ & \syndef & $ [[int]] \mid [[a]] \mid [[evar]] \mid [[atc1 -> atc2]] \mid [[gradual]]$ \\
  Castable Types & $[[agc]]$ & \syndef & $ [[int]] \mid [[a]] \mid [[evar]] \mid [[agc1 -> agc2]] \mid [[\/ a. agc]] \mid [[unknown]] \mid [[gradual]] $ \\
  Static Castable Types & $[[asc]]$ & \syndef & $ [[int]] \mid [[a]] \mid [[evar]] \mid [[asc1 -> asc2]] \mid [[\/ a. asc]] \mid [[gradual]] $ \\
  Contexts & $[[GG]], [[DD]], [[TT]]$ & \syndef & $[[empty]] \mid [[GG , x : aA]] \mid [[GG , a]] \mid [[GG , evar]] \mid [[GG, evar = at]] $ \\
  Complete Contexts & $[[OO]]$ & \syndef & $[[empty]] \mid [[OO , x : aA]] \mid [[OO , a]] \mid [[OO, evar = at]]$ \\ \bottomrule
\end{tabular}
\end{center}



\begin{definition}[Existential variable contamination] \label{def:contamination}
  \begin{align*}
    [[ [aA] empty    ]] &= [[empty]] \\
    [[ [aA] (GG, x : aA)  ]] &= [[ [aA] GG , x : aA     ]] \\
    [[ [aA] (GG, a)  ]] &= [[ [aA] GG , a     ]] \\
    [[ [aA] (GG, sa)  ]] &= [[ [aA] GG , ga , sa = ga  ]]  \quad \text{if $[[sa in fv(aA)]]$ }    \\
    [[ [aA] (GG, ga)  ]] &= [[ [aA] GG , ga     ]] \\
    [[ [aA] (GG, evar = at)  ]] &= [[ [aA] GG , evar = at     ]] \\
  \end{align*}
\end{definition}



\drules[ad]{$ [[GG |- aA ]] $}{Well-formedness of types}{int, unknown, static, gradual, tvar, evar, solvedEvar, arrow, forall}

\drules[wf]{$ [[ |- GG ]] $}{Well-formedness of algorithmic contexts}{empty, var, tvar, evar, solvedEvar}

\drules[as]{$ [[GG |- aA <~ aB -| DD ]] $}{Algorithmic Consistent Subtyping}{tvar, evar, int, arrow, forallR, forallL, spar, gpar, unknownL, unknownR, instL, instR}

\drules[instl]{$ [[ GG |- evar <~~ aA -| DD   ]] $}{Instantiation I}{solveS, solveG, solveUS, solveUG, reachSGOne, reachSGTwo, reachOtherwise, arr, forallR}

\drules[instr]{$ [[ GG |- aA <~~ evar -| DD   ]] $}{Instantiation II}{solveS, solveG, solveUS, solveUG, reachSGOne, reachSGTwo, reachOtherwise, arr, forallL}

\drules[inf]{$ [[ GG |- ae => aA -| DD ]] $}{Inference}{var, int, lamann, lam, anno, app}

\drules[chk]{$ [[ GG |- ae <= aA -| DD ]] $}{Checking}{lam, gen, sub}

\drules[am]{$ [[ GG |- aA |> aA1 -> aA2 -| DD ]] $}{Algorithmic Matching}{forall, arr, unknown, var}

\drules[ext]{$ [[ GG --> DD  ]] $}{Context extension}{id, var, tvar, evar, solvedEvar, solveS, solveG, add, addSolveS, addSolveG}



\clearpage


\section{Metatheory}

\begin{restatable}[Instantiation Soundness]{mtheorem}{instsoundness} \label{thm:inst_soundness}%
  Given $[[ DD --> OO ]]$ and $[[ [GG]aA = aA ]]$ and  $[[evar notin fv(aA)]]$:

  \begin{enumerate}
  \item If $[[GG |- evar <~~  aA -| DD ]]$ then $[[  [OO]DD |- [OO]evar <~ [OO]aA  ]] $.
  \item If $[[GG |- aA <~~ evar -| DD ]]$ then $[[  [OO]DD |- [OO]aA <~ [OO]evar  ]] $.
  \end{enumerate}
\end{restatable}


\begin{restatable}[Soundness of Algorithmic Consistent Subtyping]{mtheorem}{subsoundness} \label{thm:sub_soundness}%
  If $[[  GG |- aA <~ aB -| DD ]]$ where $[[ [GG]aA = aA  ]]$ and $[[  [GG] aB = aB  ]]$ and $[[  DD --> OO ]]$ then
  $[[  [OO]DD |- [OO]aA <~ [OO]aB   ]]$.
\end{restatable}



\begin{restatable}[Soundness of Algorithmic Typing]{mtheorem}{typingsoundness} \label{thm:type_sound}%
  Given $[[DD --> OO]]$:
  \begin{enumerate}
  \item If $[[  GG |- ae => aA -| DD  ]]$ then $\exists [[e']]$ such that $ [[  [OO]DD |- e' : [OO] aA  ]]   $ and $\erase{[[ae]]} = \erase{[[e']]}$.
  \item If $[[  GG |- ae <= aA -| DD  ]]$ then $\exists [[e']]$ such that $ [[  [OO]DD |- e' : [OO] aA  ]]   $ and $\erase{[[ae]]} = \erase{[[e']]}$.
  \end{enumerate}
\end{restatable}

\begin{restatable}[Instantiation Completeness]{mtheorem}{instcomplete} \label{thm:inst_complete}
  Given $[[GG --> OO]]$ and $[[aA = [GG]aA]]$ and $[[evar]] \notin \textsc{unsolved}([[GG]]) $ and $[[  evar notin fv(aA)  ]]$:
  \begin{enumerate}
  \item If $[[ [OO]GG |- [OO] evar <~ [OO]aA   ]]$ then there are $[[DD]], [[OO']]$ such that $[[OO --> OO']]$
    and $[[DD --> OO']]$ and $[[GG |- evar <~~ aA -| DD]]$.
  \item If $[[ [OO]GG |- [OO]aA  <~ [OO] evar  ]]$ then there are $[[DD]], [[OO']]$ such that $[[OO --> OO']]$
    and $[[DD --> OO']]$ and $[[GG |- aA <~~ evar -| DD]]$.
  \end{enumerate}

\end{restatable}


\begin{restatable}[Generalized Completeness of Consistent Subtyping]{mtheorem}{subcomplete} \label{thm:sub_completeness}
  If $[[ GG --> OO  ]]$ and $[[ GG |- aA  ]]$ and $[[ GG |- aB  ]]$ and $[[  [OO]GG |- [OO]aA <~ [OO]aB  ]]$ then
  there exist $[[DD]]$ and $[[OO']]$ such that $[[DD --> OO']]$ and $[[OO --> OO']]$ and $[[  GG |- [GG]aA <~ [GG]aB -| DD ]]$.
\end{restatable}


\begin{restatable}[Matching Completeness]{mtheorem}{matchcomplete} \label{thm:match_complete}%
  Given $[[ GG --> OO  ]]$ and $[[ GG |- aA  ]]$, if
  $[[ [OO]GG |- [OO]aA |> A1 -> A2  ]]$
  then there exist $[[DD]]$, $[[OO']]$, $[[aA1']]$ and $[[aA2']]$ such that $[[ GG |- [GG]aA |> aA1' -> aA2' -| DD   ]]$
  and $[[ DD --> OO'  ]]$ and $[[ OO --> OO'  ]]$ and $[[A1]] = [[ [OO']aA1'  ]]$ and $[[A2]] = [[ [OO']aA2'  ]]$.
\end{restatable}



\begin{restatable}[Completeness of Algorithmic Typing]{mtheorem}{typingcomplete}
  Given $[[GG --> OO]]$ and $[[GG |- aA]]$, if $[[ [OO]GG |- e : A ]]$ then there exist $[[DD]]$, $[[OO']]$, $[[aA']]$ and $[[ae']]$
  such that $[[DD --> OO']]$ and $[[OO --> OO']]$ and $[[  GG |- ae' => aA' -| DD  ]]$ and $[[A]] = [[ [OO']aA'  ]]$ and $\erase{[[e]]} = \erase{[[ae']]}$.
\end{restatable}



%%%%%%%%%%%%%%%%%%%%%%%%%%%%%%%%%%%%%%%%%%%%%%%%%%%%%%%%%%%%%%%%%%%%%%%%
\section{Establishing Coherence for \fnamee}
\label{sec:coherence:poly}
%%%%%%%%%%%%%%%%%%%%%%%%%%%%%%%%%%%%%%%%%%%%%%%%%%%%%%%%%%%%%%%%%%%%%%%%

In this section, we establish the coherence property for \fnamee. The proof
strategy mostly follows that of \namee, but the construction of the
heterogeneous logical relation is significantly more complicated. Firstly in
\cref{sec:para:intuition} we discuss why adding BCD subtyping to disjoint
polymorphism introduces significant complications. In
\cref{sec:failed:lr}, we discuss why a natural extension of
System F's logical relation to deal with disjoint polymorphism fails. The technical
difficulty is \emph{well-foundedness}, stemming from the interaction between
impredicativity and disjointness. Finally in \cref{sec:succeed:lr}, we present
our (predicative) logical relation that is specially crafted to prove coherence
for \fnamee.
% and allude to a potential solution to lift the predicativity restriction.

\subsection{The Challenge}
\label{sec:para:intuition}

Before we tackle the coherence of \fnamee, let us first consider how \fname
(and its predecessor \oname) enforces coherence. Its essentially syntactic
approach is to make sure that there is at most one subtyping derivation for any
two types. As an immediate consequence, the produced coercions are uniquely determined and thus
the calculus is clearly coherent. Key to this approach is the invariant that
the type system only produces \emph{disjoint} intersection types. As we
mentioned in \cref{sec:typesystem}, this invariant complicates the calculus
and its metatheory, and leads to a weaker substitution lemma.
% To see this, consider the judgment $[[ X ** nat |- X & nat ]]$. 
% Clearly $[[X]]$ cannot be instantiated to an arbitrary type. For
% instance, substituting $[[X]]$ with $[[nat]]$ would lead to an ill-formed
% intersection type $[[nat & nat]]$ in \fname. 
% Therefore in the
% substitution lemma, the range of substituted types is narrowed down to those
% that respect the disjointness constraints.
% The motivation of maintaining this invariant was to enable
% Generally speaking, in \fname all meta-theoretic properties are weakened to
% account for disjointness pre-conditions. All of these contribute
Moreover, the syntactic coherence approach is incompatible with BCD subtyping,
which leads to multiple subtyping derivations with different coercions and
requires a more general substitution lemma. For example, consider the
coercions produced by $[[ \X ** nat . X & X <: \X ** nat & nat . X ]]$ (neither
type is ``well-formed'' in the sense of \fname). Two possible ones are
$[[ \f . \X . pp1 (f X) ]]$ and $[[ \f . \X . pp2 (f X) ]]$. It is not at all
obvious that they should be equivalent in an appropriate sense.
To accommodate BCD into \oname, Bi et al.~\cite{bi_et_al:LIPIcs:2018:9227}
have created the \namee calculus and
developed a semantically-founded proof method based on logical relations.
Because \namee does not feature polymorphism, the problem at hand is to
incorporate support for polymorphism in this semantic approach to coherence,
which turns out to be more challenging than is apparent.

% preclude the possibility of adding BCD
% subtyping, which requires a general substitution lemma. This implies that the
% avenue taken by Alpuim et al.~\cite{alpuimdisjoint} to prove coherence does not
% work for \fnamee anymore. In particular, subtyping does not necessarily produces unique
% coercions. For example, consider the possible coercions generated by $[[ \X ** nat . X & X <: \X ** nat & nat . X ]]$ (neither of which is ``well-formed''
% in the sense of \fname). Two possible coercions are $[[ \f . \X . pp1 (f X) ]]$
% and $[[ \f . \X . pp2 (f X) ]]$. It is not at all obvious that these two
% coercions are equivalent in an appropriate sense. Moreover, the addition of BCD subtyping
% aggravates the matter even more---the subtyping relation can produce additional
% syntactically different coercions that are harder to argue to be equivalent.
% Inspired by Bi et al.~\cite{bi_et_al:LIPIcs:2018:9227}, a new semantically-founded
% proof method is called for. Logical relations \`a la System F might shed some
% light, as we will discuss next.

\begin{figure}[t]
  \centering
  \begin{tabular}{rll}
    $[[(v1 , v2) in V ( nat ; nat ) ]]$  & $\defeq$ & $\exists [[i]].\, [[v1]] = [[v2]] = [[i]]$ \\
    $[[(v1, v2)  in V(T1 -> T2; T1' -> T2') ]]$ &$\defeq$ & $\forall [[(v, v') in V (T1; T1')   ]].\, [[  (v1 v , v2 v') in E (T2 ; T2') ]]$ \\
    $[[( < v1 , v2 > , v3  )  in V ( T1 * T2 ;  T3  )  ]]$  &$\defeq$& $[[ (v1, v3)  in V (T1 ; T3)  ]] \land [[ (v2, v3)  in V (T2 ; T3)  ]]$ \\
    $[[( v3 , < v1 , v2 >  )  in V ( T3 ; T1 * T2  )  ]]$  &$\defeq$& $[[ (v3, v1)  in V (T3 ; T1)  ]] \land [[ (v3, v2)  in V (T3 ; T2)  ]]$
  \end{tabular}
  \caption{Selected cases from \namee's canonicity relation}
  \label{fig:logical:necolus}
\end{figure}

\subsection{Impredicativity and Disjointness at Odds}
\label{sec:failed:lr}

\Cref{fig:logical:necolus} shows selected cases of \emph{canonicity},
which is \namee's (heterogeneous) logical relation used
in the coherence proof. The definition captures that two values
$[[v1]]$ and $[[v2]]$ of types $[[ T1 ]]$ and $[[T2]]$ are in $\valR{[[T1]]}{[[T2]]}$ iff
either the types are disjoint or the types are equal and the values are
semantically equivalent. Because both alternatives entail coherence, 
canonicity is key to \namee's coherence proof.

\paragraph{Well-foundedness issues.}
For \fnamee, we need to extend canonicity with additional cases to
account for universally quantified types. For reasons that will become clear in
\cref{sec:succeed:lr}, the type indices become source types (rather than target types as in \cref{fig:logical:necolus}).
A naive formulation of one case rule is:
{\small
\begin{align*}
    &[[(v1, v2)  in V(\X ** A1 . B1; \X ** A2 . B2) ]] \defeq  \\
    &\qquad \forall [[C1 ** A1]], [[C2 ** A2]].\ [[( v1 | C1 | , v2 | C2 | ) in E ( B1 [X ~> C1]; B2 [X ~> C2]) ]]
\end{align*}
}%
This case is problematic because it destroys the well-foundedness of \namee's
logical relation, which is based on structural induction on the type indices.
Indeed, the type $[[ B1 [X ~> C1] ]]$ may well be larger than $[[ \X ** A1 . B1 ]]$.


% \begin{verbatim}
% Further outline
% - show System F-style case with deferred substitions
% - introduce variable case
% - show well-foundedness problem with variable case (also present in System F)
% - show System F solution for the problem by adding a relation parameter R
% - introduce problem with heterogeneous case
% \end{verbatim}

However, System F's well-known parametricity logical
relation~\cite{reynolds1983types} provides us with a means to avoid this
problem.  Rather than performing the type substitution immediately as in the
above rule, we can defer it to a later point by adding it to an extra parameter
$[[pq]]$ of the relation, which accumulates the deferred substitutions. This yields a modified rule where the type indices in the recursive occurrences are indeed smaller:
{\small
\begin{align*}
  &[[(v1, v2)  in V(\X ** A1 . B1; \X ** A2 . B2) with pq ]]  \defeq  \\
  &\qquad \forall [[C1 ** A1]], [[C2 ** A2]]. ([[v1 | C1 | ]] ,  [[v2 | C2 |]]) \in \eeR{[[B1]]}{{[[B2]]}}_{[[pq]] [ [[X]] \mapsto ([[C1]], [[C2]])]}
\end{align*}
}%
Of course, the deferred substitution has to be performed eventually, to be precise when the type indices are type variables.
\[
    [[(v1, v2)  in V(X ; X) with pq ]] \defeq [[ (v1, v2) in V(pq1 (X); pq2 (X)) with emp  ]]
\]
Unfortunately, this way we have not only moved the type substitution to the type variable case, but also the ill-foundedness problem. Indeed, this problem is also
present in System F. The standard solution is to not fix the relation $[[Rel]]$ by which values
at type $[[X]]$ are related to $\valR{[[pq1 (X)]]}{[[pq2 (X)]]}$, but instead to make it a parameter that is tracked by $[[pq]]$.
This yields the following two rules for disjoint quantification and type variables:
{\small
\begin{align*}
  [[(v1, v2)  in V(\X ** A1 . B1; \X ** A2 . B2) with pq ]] &\defeq \forall [[C1 ** A1]], [[C2 ** A2]], [[Rel]] \subseteq [[C1]] \times [[C2]]. \\
                                                            & ([[v1 | C1 | ]] ,  [[v2 | C2 |]]) \in \eeR{[[B1]]}{{[[B2]]}}_{[[pq]] [ [[X]] \mapsto ([[C1]], [[C2]], [[Rel]])]} \\
    [[(v1, v2)  in V(X; X) with pq ]] & \defeq ([[v1]], [[v2]]) \in [[pq]]_{[[Rel]]}([[X]])
\end{align*}
}%
Now we have finally recovered the well-foundedness of the relation. It is again
structurally inductive on the size of the type indexes.


\paragraph{Heterogeneous issues.}

We have not yet accounted for one major difference between the parametricity relation, from which we have borrowed ideas, and the canonicity relation, to which we have been adding. The former is homogeneous (i.e., the types of the two values is the same) and therefore has one type index, while the latter is heterogeneous (i.e., the two values may have different types) and therefore has two type indices. Thus we must also consider cases like
$\valR{[[X]]}{[[nat]]}$. A definition that seems to handle this case
appropriately is:
{\small
  \begin{align} \label{eq:var}
    [[(v1, v2)  in V(X; nat) with pq ]] \defeq [[ (v1, v2) in V(pq1 (X); nat) with emp  ]]
  \end{align}
}%
Here is an example to motivate this case.
Let  $  [[ee]] = [[\ X ** Top . (\x . x) : X & nat -> X & nat]] $.
We expect that $[[ee nat 1 -->> <1 , 1> ]]$, which
%%\footnote{The reader is advised to try it out in our prototype interpreter.}
boils down to showing $  (1 , 1)   \in \valR{[[X]]}{[[nat]]}_{[ [[X]] \mapsto ([[nat]], [[nat]], [[Rel]])   ]}  $.
According to \cref{eq:var}, this is indeed the case. However, we run into ill-foundedness issue again, because
$[[pq1 (X)]]$ could be larger than $[[X]]$. Alas, this time the parametricity relation has no solution for us.


\subsection{The Canonicity Relation for \fnamee}
\label{sec:succeed:lr}

% \bruno{Perhaps we are still showing too many auxiliary lemmas here? We
% could cut on some of these if we are looking for space.}
In light of the fact that substitution in the logical relation seems unavoidable
in our setting, and that impredicativity is at odds with substitution, we turn
to \emph{predicativity}: we change \rref{T-tapp} to its predicative version:
{\small
\[
  \drule{T-tappMono}
\]
}%
where metavariable $[[t]]$ ranges over monotypes (types minus disjoint quantification).
We do not believe that predicativity is a severe restriction in practice, since many source
languages (e.g., those based on the Hindley-Milner type system~\cite{milner1978theory, hindley1969principal} like Haskell and
OCaml) are themselves predicative and do not require the full generality of an
impredicative core language.

% The restriction to
% predicative polymorphism, though reducing expressiveness in theory, does not seem to cost much
% in practice. Languages based on the Hindley–Milner type
% system~\cite{milner1978theory, hindley1969principal}, such as Haskell and ML,
% have such restriction. We also plan to study a variant of \fnamee with implicit
% polymorphism in the future, where a predicativity restriction is
% likely to be required anyway.

\begin{figure}[t]
  \centering
  \begin{tabular}{rll}
    $[[(v1 , v2) in V ( nat ; nat ) ]]$  & $\defeq$ & $\exists [[i]].\, [[v1]] = [[v2]] = [[i]]$ \\
    $[[(v1, v2) in V ( {l : A}  ; {l : B} ) ]]$ & $\defeq$ & $[[ (v1, v2) in V ( A ; B ) ]]$\\
    $[[(v1 , v2) in V ( A1 -> B1 ; A2 -> B2 ) ]]$  & $\defeq$ & $\forall [[(v2' , v1') in V ( A2 ; A1 ) ]].\, [[ (v1 v1' , v2 v2') in E ( B1 ; B2 ) ]]$ \\
    $[[( < v1 , v2 > , v3  )  in V ( A & B ;  C  ) ]]$  & $\defeq$ & $[[ (v1, v3)  in V (A ; C) ]] \land [[ (v2, v3)  in V (B ; C) ]]$  \\
    $[[( v3 , < v1 , v2 >  )  in V ( C; A & B  ) ]]$  & $\defeq$ & $[[ (v3, v1)  in V (C ; A) ]] \land [[ (v3, v2)  in V (C ; B) ]]$  \\
    $[[(v1, v2)  in V ( \ X ** A1 . B1; \ X ** A2 . B2 ) ]]$  &$\defeq$ & $\forall [[empty |- t ** A1 & A2 ]].\ [[  (v1 |t| , v2 |t|) in E ( B1 [X ~> t] ;  B2 [ X ~> t]) ]]$ \\
  % $[[(v1, v2) in V ( A  ; B ) ]]$ & $\defeq$ & $[[A top]] \, \lor \, [[B top]]    $ \\
    $[[(v1 , v2) in V (A; B)]] $  &$\defeq$ & $\mathsf{true} \quad \text{otherwise} $ \\
    $[[(e1, e2) in E (A; B)]]$ & $\defeq$ & $\exists [[v1]], [[v2]].\, [[e1 -->> v1]] \land [[e2 -->> v2]] \ \land [[(v1, v2) in V (A; B)]]$ \\ \\
  \end{tabular}

  \begin{tabular}{rrll}
    $[[p in  DD]]$ & $\defeq$ &  $\ottaltinferrule{}{}{  }{ [[empp in empty]] }$ &     $\ottaltinferrule{}{}{ [[p in DD]] \\ [[empty |- t ** p(B)]] \\  }{ [[p [ X -> t ] in DD , X ** B]]  }$ \\ \\
    $[[  (g1, g2)  in GG with p ]]$ & $\defeq$ &  $\ottaltinferrule{}{}{  }{ [[(empg, empg) in empty with p ]]  }$ & $\ottaltinferrule{}{}{ [[(g1, g2) in GG with p ]] \\ [[(v1, v2) in V (p(A) ; p(A)) ]] }{ [[(g1 [ x -> v1 ] , g2 [ x -> v2 ]  )  in GG , x : A with p ]] }$
  \end{tabular}
  \caption{The canonicity relation for \fnamee}
  \label{fig:logical:fi}
\end{figure}

Luckily, substitution with monotypes does not prevent well-foundedness.
\Cref{fig:logical:fi} defines the \emph{canonicity} relation for
\fnamee. The canonicity relation is a family of binary relations over \tnamee
values that are \emph{heterogeneous}, i.e., indexed by two \fnamee types. Two
points are worth mentioning. (1) An apparent difference from \namee's logical
relation is that our relation is now indexed by \emph{source types}. The reason is that
the type translation function (\cref{def:type:translate:fi}) discards disjointness
constraints, which are crucial in our setting, whereas \namee's
type translation does not have information loss. (2) Heterogeneity
allows relating values of different types, and in particular values whose types are
disjoint. The rationale behind the canonicity relation is to combine equality
checking from traditional (homogeneous) logical relations with disjointness
checking. It consists of two relations: the value relation $\valR{[[A]]}{[[B]]}$
relates \emph{closed} values; and the expression relation
$\eeR{[[A]]}{[[B]]}$---defined in terms of the value relation---relates closed
expressions.

% \paragraph{Value relation.}

The relation $\valR{[[A]]}{[[B]]}$ is defined by induction on the structures of $[[A]]$ and
$[[B]]$. For integers, it requires the two values to be literally the same. For
two records to behave the same, their fields must behave the same. For two
functions to behave the same, they are required to produce outputs related at
$[[B1]]$ and $[[B2]]$ when given related inputs at $[[A1]]$ and $[[A2]]$. For
the next two cases regarding intersection types, the relation distributes
over intersection constructor $[[&]]$. Of particular interest is the case for
disjoint quantification. Notice that it \emph{does not} quantify over arbitrary
relations, but directly substitutes $[[X]]$ with monotype $[[t]]$ in $[[B1]]$ and
$[[B2]]$. This means that our canonicity relation \emph{does not} entail
parametricity. % , and as such, the free theorem in \cref{sec:failed:lr}
% cannot be proved using the canonicity relation.
However, it suffices for our
purposes to prove coherence. Another noticeable thing is that we keep the
invariant that $[[A]]$ and $[[B]]$ are closed types throughout the relation, so
we no longer need to consider type variables. This simplifies things a lot. % The
% other cases are quite standard.
Note that when one type is $[[Bot]]$, two
values are vacuously related because there simply are no values of type $[[Bot]]$.
% We refer to Bi et al.~\cite{bi_et_al:LIPIcs:2018:9227} for more explanations of
% the canonicity relation.
We need to show that the relation is indeed well-founded:

\begin{restatable}[Well-foundedness]{lemma}{wellfounded}\label{lemma:well-founded}
  The canonicity relation of \fnamee is well-founded.
\end{restatable}
\proof
  Let $| \cdot |_{\forall}$ and $| \cdot |_s$ be the number of
  $\forall$-quantifies and the size of types, respectively. We consider the measure $\langle
  | \cdot |_{\forall} , | \cdot |_s \rangle$,
  where $\langle \dots \rangle$ denotes lexicographic order. For the case of
  disjoint quantification, the number of $\forall$-quantifiers decreases because monotype $[[t]]$ does not contain $\forall$-quantifiers.
  For the other cases, the measure of $| \cdot |_{\forall}$ does not increase, and
  the measure of $| \cdot |_s$ strictly decreases.
\qed

% \begin{lemma}[Symmetry]
%   If $[[ (v1, v2) in V ( A ; B ) ]]$ then $[[ (v2, v1) in V ( B ; A ) ]]$.
% \end{lemma}
% \begin{proof}
%   The proof proceeds by first induction on $ | [[A]] |_{\forall} $, then
%   simultaneous induction on the structures of $[[A]]$ and $[[B]]$.
% \end{proof}

% We give the logical interpretations of type and term contexts ($[[p]]$ is a mapping
% from type variables to monotypes, $[[g]]$ is a mapping from variables to values).

% The canonicity relation is so constructed to contain values of disjoint types:
% We need to first show an auxiliary lemma regarding top-like types:

% \begin{lemma}
%   If $[[  empty ; empty |-  v1 : |A|  ]]$,
%   $[[  empty ; empty |-  v2 : |B|  ]]$ and
%   $[[ A top  ]]$,
%   then $[[   (v1, v2) in V ( A ; B  )  ]]$.
% \end{lemma}
% \begin{proof}
%   By simultaneous induction on $[[t1]]$ and $[[t2]]$.
% \end{proof}

% \begin{lemma}[Disjoint values are related]
%   If $[[DD |- A ** B]]$, $[[ p in DD  ]]$, $[[  empty ; empty |-  v1 : |p (A)|  ]]$ and $[[  empty ; empty |-  v2 : |p (B)|  ]]$
%   then $[[   (v1, v2) in V ( p(A) ; p(B)  )    ]]$.
% \end{lemma}


\subsection{Establishing Coherence}

\paragraph{Logical equivalence.}

The canonicity relation can be lifted to open expressions in the standard way,
i.e., by considering all possible interpretations of free type and term variables.
The logical interpretations of type and term contexts are found in the bottom
half of \cref{fig:logical:fi}.
\begin{definition}[Logical equivalence $\backsimeq_{log}$]
  {\small
  \begin{align*}
    &[[DD ; GG |- e1 == e2 : A ; B]]   \defeq  [[|DD| ; |GG| |- e1 : |A|]] \land [[ |DD | ; |GG| |- e2 : | B | ]] \ \land \\
    &\qquad (\forall [[p]], [[g1]], [[g2]]. \ [[p in DD]] \land [[(g1, g2) in GG with p ]] \Longrightarrow [[(g1 (p1 (e1)), g2 (p2 (e2)))  in E (p(A) ; p(B)) ]])
  \end{align*}
  }%
\end{definition}
For conciseness, we write $[[DD ; GG |- e1 == e2 : A]]$ to mean $[[DD ; GG |- e1 == e2 : A ; A]]$.

\paragraph{Contextual equivalence.}

\begin{figure}[t]
  \centering
\begin{tabular}{llll}\toprule
  \tnamee contexts & $[[cc]]$ & $\Coloneqq$ &  $[[__]] \mid [[\ x . cc]] \mid [[\ X . cc]]  \mid [[ cc T  ]] \mid [[cc e]] \mid [[e cc]] \mid [[< cc , e>]] \mid [[<e , cc>]] \mid [[c cc]] $ \\
  \fnamee contexts & $[[CC]]$ & $\Coloneqq$ &  $[[__]] \mid [[\ x . CC]] \mid [[\ X ** A. CC]] \mid [[ CC A  ]] \mid [[CC ee]] \mid [[ee CC]] \mid [[ CC ,, ee  ]] \mid [[ ee ,, CC  ]] \mid [[ { l = CC}  ]]  \mid [[ CC . l]] \mid [[ CC : A ]] $ \\ \bottomrule
\end{tabular}
  \caption{Expression contexts}
  \label{fig:contexts:fi}
\end{figure}

Following \namee, the notion of coherence is based on \emph{contextual
  equivalence}. The intuition is that two programs are equivalent if we
\emph{cannot} tell them apart in any context. More formally, we introduce
\emph{expression contexts}, whose syntax is shown in \cref{fig:contexts:fi}. Due
to the bidirectional nature of the type system, the typing judgment of $[[C]]$
features 4 different forms (see \cref{appendix:fnamee}),
e.g., $[[CC : (DD; GG => A) ~> (DD'; GG' => A') ~~> cc]]$ reads if $[[DD ; GG |- ee => A]]$
then $[[DD' ; GG' |- CC { ee } => A']]$. The judgment also generates a well-typed \tnamee context $[[cc]]$. The
following two definitions capture the notion of contextual equivalence:

\begin{definition}[Kleene Equality $\backsimeq$]
  Two complete programs, $[[e]]$ and $[[e']]$, are Kleene equal, written
  $\kleq{[[e]]}{[[e']]}$, iff there exists $[[ii]]$ such that $[[e -->> ii]]$ and
  $[[e' -->> ii]]$.
\end{definition}

\begin{definition}[Contextual Equivalence $\backsimeq_{ctx}$] \label{def:cxtx2}
  {\small
  \begin{align*}
    &[[DD ; GG |- ee1 ~= ee2 : A]]  \defeq \forall [[e1]], [[e2]].\  [[DD ; GG |- ee1 => A ~~> e1]] \land [[DD ; GG |- ee2 => A ~~> e2]] \ \land   \\
    &\qquad (\forall [[C]], [[cc]].\ [[CC : (DD; GG => A) ~> (empty ; empty => nat) ~~> cc]] \Longrightarrow \kleq{[[cc{e1}]]}{[[cc{e2}]]})
  \end{align*}
  }%
\end{definition}

% \noindent In other words, for all possible experiments $[[ cc ]]$, the outcome of an
% experiment on $[[e1]]$ is the same as the outcome on $[[e2]]$
% (i.e., $\kleq{[[cc{e1}]]}{[[cc{e2}]]}$).

% \begin{proof}
%   By induction on the derivation of disjointness. The most interesting case is the variable rule:
%   \[
%     \drule{D-tvarL}
%   \]
%   By the definition of $[[p]]$, we know $[[p(X)]]$ is a monotype. If $[[B]]$ is
%   a polytype, then it follows easily from the definition of logical relation. If
%   $[[B]]$ is also a monotype, we know $[[p(X)]]$ and $[[p(A)]]$ are disjoint by
%   definition. Then by \cref{lemma:covariance:disjoint} and $[[A <: B]]$,
%   we have $[[p(X)]]$ and $[[p(B)]]$ are also disjoint. Finally we apply
%   \cref{lemma:disjoint:mono}.
% \end{proof}

% \paragraph{Compatibility.}

% Firstly we need the compatibility lemmas. Most of them are standard and are thus
% omitted. We show only two compatibility lemmas that are specific to our setting:

% \begin{lemma}[Coercion compatibility] \label{lemma:co-compa} % APPLYCOQ=COERCION_COMPAT
%   Suppose that $[[A1 <: A2 ~~> c]]$,
%   \begin{itemize}
%   \item If $[[DD ; GG |- e1 == e2 : A1 ; A0]]$ then $[[DD ; GG |- c e1 == e2 : A2 ; A0]]$.
%   \item If $[[DD ; GG |- e1 == e2 : A0 ; A1]]$ then $[[DD ; GG |- e1 == c e2 : A0 ; A2]]$.
%   \end{itemize}
% \end{lemma}
% % \begin{proof}
% %   By induction on the subtyping derivation.
% % \end{proof}

% \begin{lemma}[Merge compatibility] % APPLYCOQ=MERGE_COMPAT
%   If $[[ DD ;   GG |- e1 == e1' : A ]]$, $[[  DD ; GG |- e2 == e2' : B ]]$ and $[[ DD |- A ** B ]]$,
%   then $[[ DD ;  GG |- < e1, e2 > == <e1', e2'> : A & B ]]$.
% \end{lemma}
% \begin{proof}
%   By the definition of logical relation and \cref{lemma:disjoint}.
% \end{proof}


% \paragraph{Fundamental property.}

% The ``Fundamental Property'' states that any well-typed expression is related to
% itself by the logical relation. In our elaboration setting, we rephrase it so
% that any two \tnamee terms elaborated from the \emph{same} \fnamee expression
% are related To prove it, we require \cref{thm:uniq}.

% \begin{theorem} \label{thm:uniq}
%   If $[[DD ; GG |- ee => A1]]$ and $[[DD ; GG |- ee => A2]]$, then $[[A1]] \equiv_\alpha [[A2]]$.
% \end{theorem}

% \begin{theorem}[Fundamental property] We have that:
%   \begin{itemize}
%   \item If $[[DD; GG |- ee => A ~~> e]]$ and $[[DD; GG |- ee => A ~~> e']]$, then $[[DD; GG |- e == e' : A ]]$.
%   \item If $[[DD ; GG |- ee <= A ~~> e]]$ and $[[DD ; GG |- ee <= A ~~> e']]$, then $[[DD; GG |- e == e' : A ]]$.
%   \end{itemize}
% \end{theorem}


% We show that logical equivalence is preserved by \fnamee contexts:

% \begin{theorem}[Congruence]
%  If $[[CC : (DD ; GG dirflag A) ~> (DD' ; GG' dirflag' A') ~~> cc]]$, $[[DD ; GG |- ee1 dirflag A ~~> e1]]$, $[[DD ; GG |- ee2 dirflag A ~~> e2]]$
%  and $[[DD ; GG |- e1 == e2 : A ]]$, then $[[DD' ; GG' |- cc{e1} == cc{e2} : A']]$.
% \end{theorem}

\paragraph{Coherence.}

For space reasons, we directly show the coherence statement of \fnamee.
We need several technical lemmas such as compatibility lemmas, fundamental property, etc.
The interested reader can refer to our Coq formalization.

\begin{theorem}[Coherence] \label{thm:coherence:fi}
  We have that
  \begin{itemize}
  \item If $[[DD ; GG |- ee => A ]]$ then $[[DD ; GG |- ee ~= ee : A]]$.
  \item If $[[DD ; GG |- ee <= A ]]$ then $[[DD ; GG |- ee ~= ee : A]]$.
  \end{itemize}
\end{theorem}
\noindent That is, coherence is a special case of \cref{def:cxtx2} where
$[[ee1]]$ and $[[ee2]]$ are the same. At first glance, this
appears underwhelming: of course $[[ee]]$ behaves the same as itself! The tricky
part is that, if we expand it according to \cref{def:cxtx2}, it is not $[[ee]]$
itself but all its translations $[[e1]]$ and $[[e2]]$ that behave the same!




% Local Variables:
% org-ref-default-bibliography: "../paper.bib"
% End:

% \renewcommand{\rulehl}[2][gray!40]{%
  \colorbox{#1}{$\displaystyle#2$}}


\section{Taming Row Polymorphism}

In this section we show how to systematically translate
\rname~\cite{Harper:1991:RCB:99583.99603}---a polymorphic record calculus---into
\fnamee. The translation itself is interesting in two regards: first, it shows
that disjoint polymorphism can simulate row polymorphism;\ningning{won't this be
too strong an argument? there are many row polymorphism systems and we are
only talking about one of them.} second, it also
reveals a significant difference of expressiveness between disjoint polymorphism
and row polymorphism---in particular, we point out that row polymorphism alone
is impossible to encode nested composition, which is crucial for applications of
extensible designs. We first review the syntax and semantics of
\rname. We then discuss a seemingly correct translation that failed to
faithfully convey the essence of row polymorphism. By a careful comparison of
the two calculi, we present a type-directed translation,
and prove that the translation is type safe, i.e., well-typed \rname terms map
to well-typed \fnamee terms.
\bruno{besides nested composition, the other advantage of disjoint
  polymorphism is subtyping. Row polymorphism usually does not support
subtyping, so you cannnot write a function f : \{x : Int\} $\to$ Int and
type-check the following application: f \{x=2,y=3\}. With row
polymorphism you must generalize the type of f, and make the interface
of the function more complicated.}
\jeremy{I don't think this is entirely true for row polymorphism in general. Some row systems do have subtyping (see Harper's paper).
  As argued by Harper, it was their design choice to not have subsumption (or subtyping) in the first place,
  in order to have a simpler system. Also with row type inference, you don't actually need to write complicated types for this example.}

% In the process, we identified one broken lemma of \rname due to the design of type equivalence, which is remedied in our presentation.


\subsection{Syntax of \rname}

\begin{figure}[t]
  \centering
\begin{tabular}{llll@{\hskip 0.6cm}llll} \toprule
  Types & $[[rt]]$ & $\Coloneqq$ & $[[base]] \mid [[rt1 -> rt2]] \mid [[\/ a # R .  rt]] \mid [[ r  ]]$ & Constraint lists & $[[R]]$&  $\Coloneqq$ &$[[ <>  ]] \mid [[ r , R ]] $ \\ 
  Records & $[[r]]$ & $\Coloneqq$ & $[[a]] \mid [[Empty]] \mid [[ {l : rt}  ]]  \mid [[  r1 || r2 ]] $  & Term contexts & $[[Gtx]]$ &  $\Coloneqq$ &  $[[ <> ]] \mid [[Gtx , x : rt ]] $ \\
  Terms & $[[re]]$ & $\Coloneqq$ & $[[x]] \mid [[\x : rt . re]] \mid [[re1 re2]] \mid [[rempty]] $ & Type contexts & $[[Ttx]]$ & $\Coloneqq$ & $[[ <> ]] \mid [[Ttx , a # R ]] $ \\
        &          & $\mid $ & $[[{ l = re }]] \mid [[re1 || re2]] \mid [[ re \ l  ]]  \mid [[ re . l  ]] $ \\
        &          & $ \mid$ & $ [[ /\ a # R . re  ]] \mid [[  re [ r ]  ]]$ \\
    \bottomrule
\end{tabular}
  \caption{Syntax of \rname}
  \label{fig:syntax:record}
\end{figure}

We start by briefly reviewing the syntax of \rname, shown in \cref{fig:syntax:record}.

\paragraph{Types.}

Metavariable $[[rt]]$ ranges over types, which include integer types
$[[base]]$, function types $[[rt1 -> rt2]]$, constrained quantification $[[ \/ a # R . rt ]]$
and record types $[[r]]$. The record types are built from record type
variables $[[a]]$, empty record $[[Empty]]$, single-field record types $[[ { l : rt}]]$
and record merges $[[ r1 || r2 ]]$.\footnote{The original \rname also include record
  type restrictions $[[r \ l]]$, which, as they later proved, can be systematically
  erased using type equivalence, thus we omit type-level restrictions but keep term-level restrictions.}
A constraint list $[[R]]$ is a list of record types, used to constrain instantiations of record type variables.
% , and plays
% an important role in the calculus, as we will explain shortly.

\paragraph{Terms.}

Metavariable $[[re]]$ ranges over terms, which include term
variables $[[x]]$, lambda abstractions $[[ \x : rt . re ]]$, function applications $[[re1 re2]]$, empty records $[[rempty]]$,
single-filed records $[[{l = re}]]$, record merges $[[re1 || re2]]$, record restrictions $[[ re \ l ]]$, record projections $[[ re . l  ]]$,
type abstractions $[[  /\ a # R . re ]]$ and type applications $[[ re [ r ]   ]]$.
As a side note, from the syntax of type applications $[[re [ r ] ]]$, it already can be seen that \rname only supports
quantification over \emph{record types}.
% ---though a separate form of quantifier that quantifies over \emph{all types}.
% can be added, Harper and Pierce decided to have only one form of quantifier for the sake of simplicity.

% \paragraph{An example.}

% Before proceeding to the formal semantics, let us first see some examples that
% can be written in \rname, which may be of help in understanding the overall
% system better.
\paragraph{An example.}

When it comes to extension, every record calculus must decide what to do with
duplicate labels. According to Leijen~\cite{leijen2005extensible}, record calculi can
be divided into those that support \emph{free} extension, and those that support
\emph{strict} extension. The former allows duplicate labels to coexist, whereas
the latter does not. In that sense, \rname belongs to the strict camp. What sets
\rname apart from other strict record calculi is its ability to merge records
with statically unknown fields, and a mechanism to ensure the resulting record
is conflict-free (i.e., no duplicate labels). For example, the following
function merges two records:
\[
  \mathsf{mergeRcd} = [[  /\ a1 # Empty . /\ a2 # a1  . \ x1 : a1 . \ x2 : a2 . x1 || x2  ]]
\]
which takes two type variables: the first one has no constraint
($[[Empty]]$) at all and the second one has only one constraint ($[[ a1 ]]$). It
may come as no surprise that $\mathsf{mergeRcd}$ can take any record type as its
first argument, but the second type must be \emph{compatible} with the first. In
other words, the second record cannot have any labels that already exist in the
first. These constraints are enough to ensure that the resulting record $[[x1 ||
x2]]$ has no duplicate labels. If later we want to say that the first record
$[[x1]]$ has \emph{at least} a field $[[l1]]$ of type $[[nat]]$, we can refine
the constraint list of $[[a1]]$ and the type of $[[x1]]$ accordingly:
\[
  [[  /\ a1 # {l1 : nat} . /\ a2 # a1  . \ x1 : a1 || {l1 : nat} . \ x2 : a2 . x1 || x2  ]]
\]
The above examples suggest an important point: the form of constraint used in
\rname can only express \emph{negative} information about record type variables.
Nonetheless, with the help of the merge operator, positive information can be
encoded as merges of record type variables, e.g., the assigned type of $[[x1]]$
illustrates that the missing field $[[ {l1 : nat} ]]$ is merged back into
$[[a1]]$.

The acute reader may have noticed some correspondences between \rname and
\fnamee: for instance, $[[ /\ a # R . re ]]$ vs. $[[ \ X ** A . ee ]]$,
and $[[x1 || x2]]$ vs.  $[[ x1 ,, x2 ]]$. Indeed, the very
function can be written in \fnamee almost verbatim:
\[
  \mathsf{mergeAny} = [[\ a1 ** Top . \ a2 ** a1 . \x1 : a1 . \x2 : a2 . x1 ,, x2 ]]
\]
However, as the name suggests, $\mathsf{mergeAny}$ works for \emph{any} two types,
not just record types.

\subsection{Typing Rules of \rname}
\label{sec:typing_rname}

% \jeremy{We may only want to show selected rules. }

The type system of \rname consists of several conventional judgments. The
complete set of rules appear in \cref{appendix:rname}.
\Cref{fig:rname_well_formed} presents the well-formedness rules for record
types. % Most cases are quite standard.
A merge $[[r1 || r2]]$ is well-formed in
$[[Ttx]]$ if $[[r1]]$ and $[[r2]]$ are well-formed, and moreover,
$[[r1]]$ and $[[r2]]$ are compatible in $[[Ttx]]$ (\rref{wfr-Merge})---the most
important judgment in \rname, as we will explain next.

\begin{figure}[t]
  \centering
% \drules[wftc]{$[[  Ttx ok ]]$}{Well-formed type contexts}{Empty, Tvar}

% \drules[wfc]{$[[  Ttx |- Gtx ok ]]$}{Well-formed term contexts}{Empty, Var}

% \drules[wfrt]{$[[  Ttx |- rt type ]]$}{Well-formed types}{Prim, Arrow, All, Rec}

\drules[wfr]{$[[  Ttx |- r record ]]$}{Well-formed record types}{Var, Merge}

% \drules[wfcl]{$[[  Ttx |- R ok ]]$}{Well-formed constraint lists}{Nil, Cons}
  \caption{Selected rules for well-formedness of record types}
  \label{fig:rname_well_formed}
\end{figure}



\paragraph{Compatibility.}

The compatibility relation in \cref{fig:compatible} plays a central role in \rname. It is the underlying
mechanism of deciding when merging two records is ``sensible''. Informally,
$[[Ttx |- r1 # r2]]$ holds if $[[r1]]$ and $[[r2]]$ are mergeable, that is,
$[[r1]]$ lacks every field contained in $[[r2]]$ and vice versa.
Compatibility is
symmetric (\rref{cmp-Symm}) and respects type equivalence (\rref{cmp-Eq}).
\Rref{cmp-Base} says that if a record is compatible with a single-field record
$[[{l : t}]]$, it is also compatible with every record $[[{l : t'}]]$. A type variable is compatible
with the records in its constraint list (\rref{cmp-Tvar}). Two single-field
records are compatible if they have different labels (\rref{cmp-BaseBase}). The
rest are self-explanatory.

% and we refer the reader to their paper for further explanations.

% The judgment of constraint list satisfaction $[[Ttx |- r # R]]$
% ensures that $[[r]]$ must be compatible with every record in the constraint list $[[R]]$.
% With the compatibility rules, let us go back to the definition of $\mathsf{mergeRcd}$
% and see why it can type check, i.e.,  why $[[a1]]$ and $[[a2]]$ are compatible---because
% $[[a1]]$ is in the constraint list of $[[a2]]$, and by \rref{cmp-Tvar}, they are compatible.


\begin{figure}[t]
  \centering
\drules[cmp]{$[[  Ttx |- r1 # r2 ]]$}{Compatibility}{Eq, Symm, Base, Tvar, MergeE,Empty,MergeI,BaseBase}
% \drules[cmpList]{$[[  Ttx |- r # R ]]$}{Constraint list satisfaction}{Nil, Cons}
\caption{Compatibility}
\label{fig:compatible}

\end{figure}

\begin{figure}[t]
  \centering
\drules[teq]{$[[  rt1 ~ rt2 ]]$}{Type equivalence}{MergeUnit,MergeAssoc,MergeComm,CongAll}
\drules[ceq]{$[[  R1 ~ R2 ]]$}{Constraint list equivalence}{Swap,Empty,Merge,Dupl,Base}
\caption{Selected type equivalence rules}
\label{fig:type_equivalence}
\end{figure}

\paragraph{Type equivalence.}

Unlike \fnamee, \rname does not have subtyping. Instead, \rname uses type
equivalence to convert terms of one type to another. A selection of the rules
defining equivalence of types and constraint lists appears in
\cref{fig:type_equivalence}. The relation $[[rt1 ~ rt2]]$ is an
equivalence relation, and is a congruence with respect to the type constructors.
Finally merge is associative (\rref{teq-MergeAssoc}), commutative
(\rref{teq-MergeComm}), and has $[[Empty]]$ as its unit (\rref{teq-MergeUnit}).
As a consequence, records are identified up to permutations. Since the
quantifier in \rname is constrained, the equivalence of constrained
quantification (\rref{teq-CongAll}) relies on the equivalence of constraint
lists $[[R1 ~ R2]]$. Again, it is an equivalence relation, and respects
type equivalence. Constraint lists are essentially finite sets, so order and
multiplicity of constraints are irrelevant (\rref{ceq-Swap,ceq-Dupl}). Merges of
constraints can be ``flattened'' (\rref{ceq-Merge}), and occurences of
$[[Empty]]$ may be eliminated (\rref{ceq-Empty}). The last rule \rref*{ceq-Base}
is quite interesting: it says that the types of single-field records are
ignored. The reason is that as far as compatibility is concerned, only labels
matter, thus changing the types of records will not affect their compatibility
relation. We will have more to say about this in \cref{sec:trouble}, in
particular, this is the rule that forbids a simple translation.

% \begin{remark}
% \jeremy{If we have space trouble, we can delete this}
%   The original rules of type equivalence~\cite{Harper:1991:RCB:99583.99603} do
%   not have contexts (i.e.,  judgment of the form $[[rt1 ~ rt2]]$). However this is incorrect, as it invalidates one of the key
%   lemmas (Lemma 2.3.1.7) in their type system, which says that
%     if $[[Ttx |- r1 # r2]]$, then $[[Ttx |- r1 record]]$ and $[[Ttx |- r2 record]]$.
%   Consider two types $[[  {l1 : nat}  ]]$ and $[[ {l2 : \/ a # {l : nat} || {l : bool} . nat  }   ]]$.
%   According to the original rules, they are compatible because
%   \begin{inparaenum}[(1)]
%   \item $[[  {l1 : nat} ]]  $ is compatible with $ [[ {l2 : \/ a # {l : nat} , {l : bool} . nat }  ]]$ by \rref{cmp-BaseBase};
%   \item $ [[ {l2 : \/ a # {l : nat} , {l : bool} . nat }  ~ {l2 : \/ a # {l : nat} || {l : bool} . nat } ]]$.
%   \end{inparaenum}
%   Then it follows that $[[ {l2 : \/ a # {l : nat} || {l : bool} . nat } ]]$ is well-formed.
%   However, this record type is not well-formed in any context because $[[{l : nat} || {l : bool}]]$
%   is not well-formed in any context. To fix this, we add context throughout type equivalence.
%   % The culprit is \rref{ceq-Merge}---the well-formedness of $[[ r1 , (r2, R) ]]$
%   % does not necessarily entail the well-formedness of $[[ (r1 || r2) ,R]]$, as
%   % the latter also requires the compatibility of $[[r1]]$ and $[[r2]]$.
%   % That is why we need to explicitly add contexts to type equivalence
%   % so that $ [[ {l2 : \/ a # {l : nat} , {l : bool} . nat } ]] $ and $[[ {l2 : \/ a # {l : nat} || {l : bool} . nat } ]]$
%   % are not considered equivalent in the first place.
% \end{remark}


\paragraph{Typing rules.}

A selection of typing rules are shown in \cref{fig:typing_rname}. At
first reading, the gray parts can be ignored, which will be covered in
\cref{sec:row_trans}. % Most of the typing rules are quite standard.
% Typing is
% invariant under type equivalence (\rref{wtt-Eq}).
Two terms can be merged if their types are compatible (\rref{wtt-Merge}). Type
application $[[ re [ r ] ]]$ is well-typed if the type argument $[[r]]$
satisfies the constraints $[[R]]$ (\rref{wtt-AllE}).


\begin{remark}
  A few simplifications have been placed compared to the original \rname,
  notably the typing of record selection (\rref{wtt-Select}) and restriction
  (\rref{wtt-Restr}). In the original formulation, both typing rules need a
  partial function $ r \_ l $ which means the type associated with label $[[l]]$
  in $[[r]]$. Instead of using partial functions, here we explicitly expose the
  expected label in a record. It can be shown that if label $[[l]]$ is present
  in record type $[[r]]$, then the fields in $[[r]]$ can be rearranged so that
  $[[l]]$ comes first by type equivalence. This formulation was also adopted by
  Leijen~\cite{leijen2005extensible}.
\end{remark}


\begin{figure}[t]
  \centering
\drules[wtt]{$[[  Ttx ; Gtx |- re : rt ~~> ee  ]]$}{Type-directed translation}{Base,Restr,Select,Empty,Merge,AllE,AllI}
\caption{Selected typing rules with translations}
\label{fig:typing_rname}
\end{figure}



\renewcommand{\rulehl}[1]{#1}

\subsection{A Failed Attempt}
\label{sec:trouble}

In this section, we sketch out an intuitive translation scheme.
On the syntactic level, it is pretty straightforward to see a one-to-one
correspondence between \rname terms and \fnamee expressions. % For example,
% constrained type abstractions $[[/\ a # R . re ]]$ correspond to \fnamee type
% abstractions $[[ \ a ** A . ee]]$; record merges can be simulated by the more
% general merge operator of \fnamee; record restriction can be modeled as annotate terms, and so on.
On the semantic level, all well-formedness judgments of \rname have corresponding well-formedness judgments
of \fnamee, given a ``suitable'' translation function $[[< rt >]]$ from \rname types to \fnamee types
Compatibility relation corresponds to disjointness relation. What might not be
so obvious is that type equivalence can be expressed via subtyping. More
specifically, $[[ rt1 ~ rt2 ]]$ induces mutual subtyping relations
$[[ < rt1 > <: < rt2 > ]]$ and $[[ < rt2 > <: < rt1 > ]]$.
Informally, type-safety of translation is something along the lines of
``if a term has type $[[rt]]$, then its translation has type $[[< rt > ]]$''.
With all these in mind, let us consider two examples:

\begin{example} \label{eg:1} %
  Consider term $[[ /\ a # {l : nat} . \x : a . x ]]$. It could be
  assigned type (among others) $[[ \/ a # {l : nat} . a -> a ]]$, and its ``obvious'' translation
  $[[  \ X ** {l : nat} . \ x : X . x  ]]$ has type $[[ \ X ** { l : nat} . X -> X   ]]$, which corresponds very well to
  $[[ \/ a # {l : nat} . a -> a  ]]$. The same term could also be assigned type $[[  \/ a # {l : bool} . a -> a   ]]$, since
  $[[  \/ a # {l : bool} . a -> a   ]]$ is equivalent to $[[  \/ a # {l : nat} . a -> a   ]]$ by \rref{teq-CongAll,ceq-Base}. However,
  as far as \fnamee is concerned, these two types  have no relationship at all---$[[  \ X ** {l : nat} . \ x : X . x  ]]$
  cannot have type $[[  \ X ** {l : bool} . X -> X   ]]$, and indeed it should not, as these two types have completely different meanings!
\end{example}

\begin{remark}
  Interestingly, the algorithmic system of \rname can only infer
  type $[[ \/ a # {l : nat} . a -> a ]]$ for the aforementioned term.
  To relate to the declarative system (in particular, to prove completeness of the algorithm),
  they show that the type inferred by the algorithm is equivalent
  (in the sense of type equivalence) to the assignable type in the declarative system.
  Proving type-safety of translation is, in a sense, like proving completeness. So
  maybe we should change the type-safety statement to
  ``if a term has type $[[rt]]$, then there exists type $[[rt']]$ such that $[[ rt ~ rt' ]]$ and the
  translation has type $[[ < rt' > ]]$''. As we shall see, this is still incorrect.
\end{remark}

\begin{example} \label{eg:2} %
  Consider term $[[re]] = [[  /\ a # {l : bool} . \ x : a . \ y : {l : nat} . x || y  ]]$.
  It has type $[[ \/ a # {l : bool} . a -> {l : nat} -> a || {l : nat}    ]]$, and
  its ``obvious'' translation is $[[ee]] = [[ \ X ** {l : bool} . \x  : X . \ y : {l : nat} . x ,, y  ]]$.
  However, expression $[[ee]]$ is ill-typed in \fnamee
  for the reasons of coherence: think about
  the result of evaluating $[[ (ee {l : nat} {l = 1} {l = 2}).l ]]$---it could evaluate to $1$ or $2$!
\end{example}

\begin{remark}
  Let us think about why \rname allows type-checking $[[re]]$. Unlike \fnamee,
  the existence of $[[re]]$ in \rname will not cause incoherence because \rname
  would reject type application $[[re [{l : nat}] ]]$ in the first place---more
  generally, $[[re]]$ can only be applied to records that do not contain label
  $[[l]]$ due to the stringency of the compatibility relation. This example
  underlines a crucial difference between the compatibility relation and the
  disjointness relation. The former can only relate records with different
  labels, whereas the latter is more fine-grained in the sense that it can also
  relate records with the same label (\rref{D-rcdEq}). Note that \rref{D-rcdEq}
  is very important for the applications of extensible designs, as we need to
  combine records with the same label, which is impossible to do in \rname.
\end{remark}


\paragraph{Taming \rname.}

It seems to imply that \rname and \fnamee are incompatible in that there are
some \rname programs that are not typable in \fnamee, and vice versa. A careful
comparison between the two calculi reveals that \rref{cmp-Base,ceq-Base} are
``to blame''. For \rname in general, these two rules are reasonable, as ``the
relevant properties of a record, for the purposes of consistency checking, are
its atomic components''~\cite{Harper:1991:RCB:99583.99603}. As far as
compatibility is concerned, a constraint list is just a list of labels and type
variables, whereas in \fnamee, disjointness constraints also care about record
types. This subtle discrepancy tells that we should have a different treatment
for those records that appear in a constraint list from those that appear
elsewhere: we translate a single-field record $[[ {l : rt}
]]$ in a constraint list to $[[ { l : Bot} ]]$. For \cref{eg:1}, both $[[ \/ a #
{l : nat} . a -> a ]]$ and $[[ \/ a # {l : bool} . a -> a ]]$ translate to $[[ \
X ** { l : Bot} . X -> X ]]$. For \cref{eg:2}, $[[re]]$ is translated to
$[[ee']] = [[ \ X ** {l : Bot} . \x : X . \ y : {l : nat} . x ,, y ]]$, which
type checks in \fnamee. Moreover, $[[ee' {l : nat} ]]$ gets rejected because
$[[Bot]]$ is not disjoint with $[[nat]]$.



\subsection{Type-Directed Translation}
\label{sec:row_trans}

\begin{figure}[t]
  \centering
\begin{tabular}{rrlllrlll} \toprule
  $[[< rt >]] \defeq \,$ & $[[ <base> ]]$ & $=$ &  $[[nat]]$ & $,$ & $[[< rt1 -> rt2 >]]$ & $=$ & $[[<rt1> -> <rt2>]]$ & $,$  \\
                       &$[[< \/ a # R . rt>]]$ & $=$ &  $[[\ X ** <R>. \ Xb ** <R>. <rt>]]$ & $,$ & & & & \\
                       &$[[ <a> ]]$ & $=$ &  $[[X]]$ & $,$ & $[[< Empty > ]]$ & $=$ & $[[Top]] $ & $,$ \\
                       &$[[ <{l:rt}> ]]$ & $=$ &  $[[ {l:<rt>} ]]$ & $,$ & $[[<r1 ||  r2> ]]$ & $=$ & $[[<r1> & <r2>]] $ \\
  $[[ <r>_b  ]] \defeq\,$ & $[[ <a>_b ]]$ & $=$ &  $[[Xb]]$ & $,$ & $[[< Empty >_b ]]$ & $=$ & $[[Top]] $ & $,$  \\
                       &$[[ <{l:rt}>_b ]]$ & $=$ &  $[[ {l:Bot} ]]$ & $,$ & $[[<r1 ||  r2>_b ]]$ & $=$ & $[[<r1>_b & <r2>_b]] $ \\
  $[[< R >]] \defeq \,$  &$[[ < <> >  ]]$ & $=$ &  $[[ Top  ]]$ & $,$ & $[[  <r , R>    ]]$ & $=$ & $[[<r>_b & <R>]] $ \\
  $[[< Ttx >]] \defeq \,$ & $[[ < <> > ]]$ & $=$ &  $[[empty]]$ & $,$ & $[[ <Ttx, a # R>  ]]$ & $=$ & $[[ <Ttx>, X ** <R>, Xb ** <R>]] $  \\
  $[[ < Gtx> ]] \defeq \,$ & $[[ < <> > ]]$ & $=$ &  $[[empty]]$ & $,$ & $[[ <Gtx, x : rt>  ]]$ & $=$ & $[[  <Gtx>, x : <rt>   ]] $ \\   \bottomrule
\end{tabular}
\caption{Translation functions}
\label{fig:trans_func}
\end{figure}




Now we can give a formal account of the translation. But there is still a twist.
Having two ways of translating records does not work out of the box. To see
this, consider $[[ \/ a # b . b ]]$, and note that a reasonably defined translation function
should commute with substitution, i.e., $[[ < [r / a] rt > ]] = [[ <rt> [X ~> <r>] ]] $.
We have LHS:
$$[[ < [ {l : nat} / b  ] (\/ a # b . b ) >  ]] =  [[  < \/ a # {l : nat} . { l : nat} > ]] = [[  \ X ** { l : Bot} . { l : nat}   ]]  $$
which is not the same as RHS:
$$[[ <\/ a # b . b>  [Y ~> < {l : nat} > ]       ]] = [[ ( \ X ** Y . Y) [Y ~> < {l : nat} > ]   ]] = [[   \ X ** {l : nat} . {l : nat}    ]]  $$
The tricky part is that we should also distinguish those record type variables
that appear in a constraint list from those that appear elsewhere. To do so, we
map record type variable $[[a]]$ to a pair of type variables $[[ X ]]$ and
$[[Xb]]$, where $[[Xb]]$ is supposed to be substituted by records with bottom
types. More specifically, we define the translation functions as in
\cref{fig:trans_func}. There are two ways of translating records: $[[<r>]]$ for
regular translation and $[[ < r >_b ]]$ for bottom translation; the latter is
used by $[[< R >]]$ for translating constraint lists. The most interesting one
is translating quantifiers: each quantifier $[[\/ a # R . rt]]$ in \rname is
split into two quantifiers $[[ \ X ** <R>. \ Xb ** <R>. <rt> ]]$ in \fnamee.
Correspondingly, each record type variable $[[a]]$ is translated to either
$[[X]]$ or $[[Xb]]$, depending on whether it appears in a constraint list or
not. The relative order of $[[X]]$ and $[[Xb]]$ is not so much relevant, as long
as we respect the order when translating type applications. Now let us go back
to the gray parts in \cref{fig:typing_rname}. In the type application $[[ re [ r
] ]]$ (\rref{wtt-AllE}), we first translate $[[e]]$ to $[[ee]]$. The translation
$[[ee]]$ is then applied to two types $[[ <r> ]]$ and $[[ <r >_b ]]$, because as
we mentioned earlier, $[[ee]]$ has two quantifiers resulting from the
translation. It is of great importance that the relative order of $[[<r>]]$ and
$[[< r >_b]]$ should match the order of $[[ X ]]$ and $[[Xb]]$ in translating
quantifiers. There is a ``protocol'' that we must keep during translation: if
$[[X]]$ is substituted by $[[ <r> ]]$, then $[[ Xb ]]$ is substitute by $[[ < r
>_b ]]$. In other words, we can safely assume $[[ Xb <: X ]]$ because $[[ < r>_b <: <r> ]]$ always holds.
Similarly, in \rref{wtt-AllI} we translate constrained type abstractions to disjointness type abstractions
with two quantifiers, matching the translation of constrained quantification.
The other rules are mostly straightforward translations. Finally we show that our translation function does commute with
substitution, but in a slightly involved form:

\begin{restatable}{lemma}{substrt} \label{lemma:subst_rt}
  $[[ <[r / a] rt> ]]$ = $ [[ <rt> [X ~> <r>] [Xb ~> <r>_b] ]] $.
\end{restatable}

% With the modified \rname, we are now ready to explain the gray parts in \cref{fig:typing_rname}. First we
% show how to translate \rname types to \fnamee types in
% \cref{fig:type_trans_rname}. Most of them are straightforward. Record merges are
% translated into intersection types, so are the constraint lists. Next we look at the
% translations of terms. Most of the them are quite intuitive. In \rref{wtt-eq},
% we put annotation $[[ | rt' | ]]$ around the translation of $[[re]]$. Record
% restrictions are translated to annotated terms (\rref{wtt-Restr}) since we
% already know the type without label $[[l]]$. Record merges are translated to
% general merges (\rref{wtt-Merge}). The translation of record selections (\rref{wtt-Select}) is  a bit
% complicated. Note that we cannot simply translate to $[[ ee . l ]]$ because our
% typing rule for record selections (\rref{T-proj}) only applies when $[[ee]]$ is a
% single-field record. Instead, we need to first transform $[[ee]]$ to a
% single-field record by annotation, and then project.

% \begin{remark}
%   The acute reader may have noticed that in \rref{wtt-AllE}, the translation type
%   $| [[r]] |$ could be a quantifier, but our rule of type applications
%   (\rref{T-tapp}) only applies to monotypes. The reason is that, for the
%   purposes of translation, we lift the monotype restrictions, which does not
%   compromise type-safety of \fnamee.
% \end{remark}

\paragraph{Type-safety of translation.}

With everything in place, we prove that our translation in
\cref{fig:typing_rname} is type-safe. The main idea is to map each judgment in
\rname to a corresponding judgment in \fnamee: well-formedness to
well-formedness, compatibility to disjointness, type-equivalence to subtyping.
The reader can refer to \cref{appendix:proofs} for detailed proofs. We
show a key lemma that bridges the ``gap'' (i.e., \rref{cmp-Base}) between row and disjoint polymorphism.

\begin{restatable}{lemma}{cmprcd} \label{lemma:cmp-rcd}
  If $[[ Ttx |- r # {l:rt} ]]$ then $[[ < Ttx > |-  < r > ** {l:A}    ]] $ and $[[ < Ttx > |-  < r >_b ** {l:A}    ]]$
  for all $A$.
\end{restatable}

Finally here is the central type-safety theorem:

\begin{restatable}{theorem}{typesafe}
  If $[[ Ttx ; Gtx |- re : rt ~~> ee ]]$ then $[[ < Ttx > ; < Gtx > |-  ee => < rt >  ]]$.
\end{restatable}
% \begin{proof}
%   By induction on the typing derivation.
% \end{proof}




% Local Variables:
% TeX-master: "../paper"
% org-ref-default-bibliography: "../paper.bib"
% End:


\section{Related Work}
\label{sec:related}

Along the way we discussed some of the most relevant work to motivate,
compare and
promote our gradual typing design. In what follows, we briefly discuss related
work on gradual typing and polymorphism.


\paragraph{Gradual Typing}

The seminal paper by \citet{siek2006gradual} is the first to propose gradual
typing, which enables programmers to mix static and dynamic typing in a program
by providing a mechanism to control which parts of a program are statically
checked. The original proposal extends the simply typed lambda calculus by
introducing the unknown type $\unknown$ and replacing type equality with type
consistency. Casts are introduced to mediate between statically and dynamically
typed code. Later \citet{siek2007gradual} incorporated gradual typing into a
simple object oriented language, and showed that subtyping and consistency are
orthogonal -- an insight that partly inspired our work. We show that subtyping
and consistency are orthogonal in a much richer type system with higher-rank
polymorphism. \citet{siek2009exploring} explores the design space of different
dynamic semantics for simply typed lambda calculus with casts and unknown types.
In the light of the ever-growing popularity of gradual typing, and its somewhat
murky theoretical foundations, \citet{siek2015refined} felt the urge to have a
complete formal characterization of what it means to be gradually typed. They
proposed a set of criteria that provides important guidelines for designers of
gradually typed languages. \citet{cimini2016gradualizer} introduced the
\emph{Gradualizer}, a general methodology for generating gradual type systems
from static type systems. Later they also develop an algorithm to generate
dynamic semantics~\cite{CiminiPOPL}. \citet{garcia2016abstracting} introduced
the AGT approach based on abstract interpretation. As we discussed, none of
these approaches instructed us how to define consistent subtyping for
polymorphic types.

There is some work on integrating gradual typing with rich type disciplines.
\citet{Ba_ados_Schwerter_2014} establish a framework to combine gradual typing and
effects, with which a static effect system can be transformed to a dynamic
effect system or any intermediate blend. \citet{Jafery:2017:SUR:3093333.3009865}
present a type system with \emph{gradual sums}, which combines refinement and
imprecision. We have discussed the interesting definition of \emph{directed
  consistency} in Section~\ref{sec:exploration}. \citet{castagna2017gradual} develop a gradual type system with
intersection and union types, with consistent subtyping defined by following
the idea of \citet{garcia2016abstracting}.
TypeScript~\citep{typescript} has a distinguished dynamic type, written {\color{blue} any}, whose fundamental feature is that any type can be
implicitly converted to and from {\color{blue} any}.
% They prove that the conversion
% definition (called \emph{assignment compatibility}) coincides with the
% restriction operator from \citet{siek2007gradual}.
Our treatment of the unknown type in \cref{fig:decl:conssub} is similar to their
treatment of {\color{blue} any}. However, their type system does not have
polymorphic types. Also, Unlike our consistent subtyping which inserts runtime
casts, in TypeScript, type information is erased after compilation so there are
no runtime casts, which makes runtime type errors possible.
% dynamic checks does not contribute to type safety.


\paragraph{Gradual Type Systems with Explicit Polymorphism}

\citet{Morris:1973:TS:512927.512938} dynamically enforces
parametric polymorphism and uses \emph{sealing} functions as the
dynamic type mechanism. More recent works on integrating gradual typing with
parametric polymorphism include the dynamic type of \citet{abadi1995dynamic} and
the \emph{Sage} language of \citet{gronski2006sage}. None of these has carefully
studied the interaction between statically and dynamically typed code.
\citet{ahmed2011blame} proposed \pbc that extends the blame
calculus~\cite{Wadler_2009} to incorporate polymorphism. The key novelty of
their work is to use dynamic sealing to enforce parametricity. As such, they end
up with a sophisticated dynamic semantics. Later, \citet{amal2017blame} prove
that with more restrictions, \pbc satisfies parametricity. Compared to their
work, our type system can catch more errors earlier since, as we argued, 
their notion of \emph{compatibility} is too permissive. For example, the
following is rejected (more precisely, the corresponding source program never
gets elaborated) by our type system:
\[
  (\blam x \unknown x + 1) : \forall a. a \to a \rightsquigarrow \cast {\unknown \to \nat}
  {\forall a. a \to a} (\blam x \unknown x + 1)
\]
while the type system of \pbc would accept the translation, though at runtime,
the program would result in a cast error as it violates parametricity.
% This does not imply, in any regard that \pbc is not well-designed; there are
% circumstances where runtime checks are \emph{needed} to ensure
% parametricity.
We emphasize that it is the combination of our powerful type system together
with the powerful dynamic semantics of \pbc that makes it possible to have
implicit higher-rank polymorphism in a gradually typed setting.
% without compromising parametricity.
\citet{devriese2017parametricity} proved that
embedding of System F terms into \pbc is not fully abstract. \citet{yuu2017poly}
also studied integrating gradual typing with parametric polymorphism. They
proposed System F$_G$, a gradually typed extension of System F, and System
F$_C$, a new polymorphic blame calculus. As has been discussed extensively,
their definition of type consistency does not apply to our setting (implicit
polymorphism). All of these approaches mix consistency with subtyping to some
extent, which we argue should be orthogonal. On a side note, it seems that our
calculus can also be safely translated to System F$_C$. However we do not
understand all the tradeoffs involved in the choice between \pbc and System
F$_C$ as a target.



\paragraph{Gradual Type Inference}
\citet{siek2008gradual} studied unification-based type inference for gradual
typing, where they show why three straightforward approaches fail to meet their
design goals. One of their main observations is
that simply ignoring dynamic types during unification does not work. Therefore,
their type system assigns unknown types to type variables and infers gradual
types, which results in a complicated type system and inference algorithm. In
our algorithm presented in \cref{sec:advanced-extension}, comparisons between
existential variables and unknown types are emphasized by the distinction
between static existential variables and gradual existential variables. By
syntactically refining unsolved gradual existential variables with unknown types, we gain a
similar effect as assigning unknown types, while keeping the algorithm relatively
simple.
\citet{garcia2015principal} presented a new approach where gradual type
inference only produces static types, which is adopted in our type system. They
also deal with let-polymorphism (rank 1 types). They proposed the distinction
between static and gradual type parameters, which inspired our extension to
restore the dynamic gradual guarantee. Although those existing works all involve
gradual types and inference, none of these works deal with higher-rank
implicit polymorphism.


\paragraph{Higher-rank Implicit Polymorphism}

\citet{odersky1996putting} introduced a type system for higher-rank implicit
polymorphic types. Based on that, \citet{jones2007practical} developed an
approach for type checking higher-rank predicative polymorphism.
\citet{dunfield2013complete} proposed a bidirectional account of higher-rank
polymorphism, and an algorithm for implementing the declarative system, which
serves as the main inspiration for our algorithmic system. The key difference,
however, is the integration of gradual typing.
% \citet{vytiniotis2012defer}
% defers static type errors to runtime, which is fundamentally different from
% gradual typing, where programmers can control over static or runtime checks by
% precision of the annotations.
As our work, those works are in a
\emph{predicative} setting, since complete type inference for higher-rank
types in an impredicative setting is undecidable. Still, there are many type
systems trying to infer some impredicative types, such as
\texttt{$ML^F$}~\citep{le2014mlf,remy2008graphic,le2009recasting}, the HML
system~\citep{leijen2009flexible}, the FPH system~\citep{vytiniotis2008fph} and
so on. Those type systems usually end up with non-standard System F types, and
sophisticated forms of type inference.

%%% Local Variables:
%%% mode: latex
%%% TeX-master: "../paper"
%%% org-ref-default-bibliography: "../paper.bib"
%%% End:


\section{Conclusion and Future Work}
\label{sec:conclusion}

We have proposed \fnamee, a type-safe and coherent calculus with disjoint
intersection types, BCD subtyping and parametric polymorphism. \fnamee improves
the state-of-art of compositional designs, and enables the development of highly
modular and reusable programs. One interesting and useful further extension
would be implicit polymorphism. For that we want to combine
Dunfield and Krishnaswami's approach~\cite{dunfield2013complete} with our bidirectional type system.
We would also like to study the parametricity of \fnamee. As we have seen in
\cref{sec:failed:lr}, it is not at all obvious how to extend the standard
logical relation of System F to account for disjointness, and avoid potential
circularity due to impredicativity. A promising solution is to use step-indexed
logical relations~\cite{ahmed2006step}. 
% TOM: This sentence is broken. Do we even need it?
% We have yet investigated further on that direction.


\section*{Acknowledgments}

We thank the anonymous reviewers and Yaoda Zhou for their helpful comments.
This work has been sponsored by the Hong Kong Research Grant
Council projects number 17210617 and 17258816, and by the Research Foundation -
Flanders.



%%% Local Variables:
%%% mode: latex
%%% TeX-master: "../paper"
%%% org-ref-default-bibliography: "../paper.bib"
%%% End:


\bibliographystyle{splncs04}
\bibliography{paper}


\newpage
\appendix

\section{Full Typing Rules of \fnamee}
\label{appendix:fnamee}

\drules[swfte]{$[[||- DD]]$}{Well-formedness}{empty, var}

\drules[swfe]{$[[DD ||- GG]]$}{Well-formedness}{empty, var}

\drules[swft]{$[[DD |- A]]$}{Well-formedness of type}{top, bot, nat, var, rcd, arrow, all, and}

\drules[S]{$ [[A <: B ~~> c]]  $}{Declarative subtyping}{refl,trans,top,rcd,andl,andr,arr,and,distArr,topArr,distRcd,topRcd,bot,forall,topAll,distAll}

\drules[TL]{$[[ A top  ]]$}{Top-like types}{top,and,arr,rcd,all}

\drules[D]{$[[DD |- A ** B]]$}{Disjointness}{topL, topR, arr, andL, andR, rcdEq, rcdNeq, tvarL, tvarR, forall,ax}

\drules[Dax]{$[[A **a B]]$}{Disjointness Axiom}{intArr, intRcd, intAll, arrAll, arrRcd, allRcd}

\textbf{Note:}   For each form $[[A **a B]]$, we also have a symmetric one $[[B **a A]]$.


\drules[T]{$[[DD; GG |- ee => A ~~> e]]$}{Inference}{top, nat, var, app, merge, anno, tabs, tapp, rcd, proj}

\drules[T]{$[[DD ; GG |- ee <= A ~~> e]]$}{Checking}{abs, sub}

\begin{definition}
  \begin{align*}
    [[ < [] >1 ]] &=  [[top]] \\
    [[ < l , fs >1 ]] &= [[ < fs >1 o id  ]] \\
    [[ < A , fs >1 ]] &= [[(top -> < fs >1) o topArr]] \\
    [[ < X ** A , fs >1 ]] &= [[ \ < fs >1 o topAll ]] \\ \\
    [[ < [] >2 ]] &=  [[id]] \\
    [[ < l , fs >2 ]] &= [[ < fs >2 o id  ]] \\
    [[ < A , fs >2 ]] &= [[(id -> < fs >2) o distArr]] \\
    [[ < X ** A , fs >2 ]] &= [[ \ < fs >2 o distPoly]]
  \end{align*}
\end{definition}

\drules[A]{$[[fs |- A <: B ~~> c]]$}{Algorithmic subtyping}{const, top, bot,and,arr,rcd,forall,arrConst,rcdConst,andConst,allConst}


\drules[CTyp]{$[[CC : (DD ;  GG => A ) ~> (DD' ; GG' => B ) ~~> cc]]$}{Context typing I}{emptyOne, appLOne, appROne, mergeLOne, mergeROne, rcdOne, projOne, annoOne, tabsOne,tappOne}

\drules[CTyp]{$[[CC : ( DD ; GG <= A ) ~> (DD' ; GG' <= B ) ~~> cc]]$}{Context typing II}{emptyTwo, absTwo}

\drules[CTyp]{$[[CC : ( DD ; GG <= A ) ~> (DD' ; GG' => B ) ~~> cc]]$}{Context typing III}{appLTwo, appRTwo, mergeLTwo, mergeRTwo, rcdTwo, projTwo, annoTwo, tabsTwo, tappTwo}

\drules[CTyp]{$[[CC : ( DD ; GG => A ) ~> ( DD' ; GG' <= B ) ~~> cc]]$}{Context typing IV}{absOne}



\section{Full Typing Rules of \tnamee}

\drules[wfe]{$[[ dd |- gg   ]]$}{Well-formedness of value context}{empty, var}

\drules[wft]{$[[ dd |- T   ]]$}{Well-formedness of types}{unit, nat, var, arrow,prod, all}

\drules[ct]{$[[ c |- T1 tri T2  ]]$}{Coercion typing}{refl,trans,top,bot,topArr,arr,pair,distArr,distAll,projl,projr,forall,topAll}

\drules[t]{$[[ dd ; gg |- e : T ]]$}{Static semantics}{unit, nat, var, abs, app, tabs, tapp, pair, capp}

\drules[r]{$[[e --> e']]$}{Single-step reduction}{topArr,topAll,distArr,distAll,id,trans,top,arr,pair,projl,projr,forall,app,tapp,ctxt}

% \section{Well-Foundedness}

% \wellfounded*


\section{Decidability}
\label{appendix:decidable}

\begin{definition}[Size of $[[fs]]$]
  \begin{align*}
    size([[ [] ]]) &=  0 \\
    size([[ fs, l ]]) &= size([[ fs ]]) \\
    size([[ fs, A ]]) &= size([[ fs ]]) + size ([[ A ]]) \\
    size([[ fs, X ** A ]]) &= size([[ fs ]]) + size([[ A ]]) \\
  \end{align*}
\end{definition}

\begin{definition}[Size of types]
  \begin{align*}
    size([[ rho ]]) &= 1 \\
    size([[ A -> B ]]) &= size([[A]]) + size([[B]]) + 1 \\
    size([[ A & B ]]) &= size([[A]]) + size([[B]]) + 1 \\
    size([[ {l:A} ]]) &= size([[A]]) + 1 \\
    size([[ \X ** A. B ]]) &= size([[A]]) + size([[B]]) + 1
  \end{align*}
\end{definition}

% \begin{theorem}[Decidability of Algorithmic Subtyping]
%   \label{lemma:decide-sub}
%   Given $[[fs]]$, $[[A]]$ and $[[B]]$, it is decidable whether there exists
%   $[[c]]$, such that $[[fs |- A <: B ~~> c]]$.
% \end{theorem}
\decidesub*
\proof
Let the judgment $[[fs |- A <: B ~~> c]]$ be measured by $size([[fs]]) +
size([[A]]) + size([[B]])$. For each subtyping rule, we show that every
inductive premise
is smaller than the conclusion.

\begin{itemize}
  \item Rules \rref*{A-const,A-top, A-bot} have no premise.
    \item Case \[ \drule{A-and} \]
      In both premises, they have the same $[[fs]]$ and $[[A]]$, but $[[B1]]$
      and $[[B2]]$ are smaller than $[[B1 & B2]]$.
    \item Case \[\drule{A-arr} \]
      The size of premise is smaller than the conclusion as $size([[B1 -> B2]])
      = size([[B1]]) + size([[B2]]) + 1$.
    \item Case \[ \drule{A-rcd} \]
      In premise, the size is $size([[fs,l]]) + size ([[A]]) + size([[B]]) =
      size([[fs]]) + size([[A]]) + size([[B]])$, which is smaller than
      $size([[fs]]) + size([[A]]) + size([[{l:B}]]) = size([[fs]]) + size([[A]])
      + size([[B]]) + 1$.
    \item Case \[\drule{A-forall} \]
      The size of premise is smaller than the conclusion as $size([[fs]]) +
      size([[A]]) + size([[\X ** B1.B2]])
      = size([[fs]]) + size([[A]]) + size([[B1]]) + size([[B2]]) + 1
      > size([[fs, X ** B1]]) + size([[A]]) + size([[B2]])
      = size([[fs]]) + size([[B1]]) + size([[A]]) + size([[B2]])$.
    \item Case \[\drule{A-arrConst} \]
      In the first premise, the size is smaller than the conclusion because of
      the size of $[[fs]]$ and $[[A2]]$. In the second premise, the size is
      smaller than the conclusion because $size([[A1 -> A2]]) > size([[A2]])$.
    \item Case \[\drule{A-rcdConst} \]
      The size of premise is smaller as $size([[ l, fs ]]) + size([[{l:A}]]) +
      size([[rho]])
      = size([[fs]]) + size([[A]]) + size([[rho]]) + 1
      > size([[fs]]) + size([[A]]) + size([[rho]])$.
    \item Case \[\drule{A-andConst} \]
      The size of premise is smaller as $size([[A1 & A2]]) = size([[A1]]) +
      size([[A2]]) + 1 > size([[Ai]])$.
    \item Case \[\drule{A-allConst} \]
      In the first premise, the size is smaller than the conclusion because of
      the size of $[[fs]]$ and $[[A2]]$. In the second premise, the size is
      smaller than the conclusion because $size([[\Y**A1. A2]]) > size([[A2]])$.
\end{itemize}
\qed

\begin{lemma}[Decidability of Top-like types]
  \label{lemma:decide-top}
  Given a type $[[A]]$, it is decidable whether $[[ A top ]]$.
\end{lemma}
\proof Induction on the judgment $[[A top]]$. For each subtyping rule, we show
that every inductive premise is smaller than the conclusion.
\begin{itemize}
  \item \rref{TL-top} has no premise.
  \item Case \[\drule{TL-and}\]
    $size([[A & B]]) = size([[A]]) + size([[B]]) + 1$, which is greater than
    $size([[A]])$ and $size([[B]])$.
  \item Case \[\drule{TL-arr}\]
    $size([[A -> B]]) = size([[A]]) + size([[B]]) + 1$, which is greater than
    $size([[B]])$.
  \item Case \[\drule{TL-rcd}\]
    $size([[{l:A}]]) = size([[A]]) + 1$, which is greater than
    $size([[A]])$.
  \item Case \[\drule{TL-all}\]
    $size([[\X ** A. B]]) = size([[A]]) + size([[B]]) + 1$, which is greater than
    $size([[B]])$.
\end{itemize}
\qed

\begin{lemma}[Decidability of disjointness axioms checking]
  \label{lemma:decide-disa}
  Given $[[A]]$ and $[[B]]$, it is decidable whether $[[ A **a B ]]$.
\end{lemma}
\proof $[[ A **a B ]]$ is decided by the shape of types, and thus it's
decidable. 
\qed

% \begin{theorem}[Decidability of disjointness checking]
%   \label{lemma:decide-dis}
%   Given $[[DD]]$, $[[A]]$ and $[[B]]$, it is decidable whether $[[ DD |- A ** B ]]$.
% \end{theorem}
\decidedis*
\proof
Let the judgment $[[ DD |- A ** B ]]$ be measured by $ size([[A]]) +
size([[B]])$. For each subtyping rule, we show that every inductive premise
is smaller than the conclusion.
\begin{itemize}
\item Case \[\drule{D-topL}\]
  By \cref{lemma:decide-top}, we know $[[A top]]$ is decidable.
\item Case \[\drule{D-topR}\]
  By \cref{lemma:decide-top}, we know $[[B top]]$ is decidable.
\item Case \[\drule{D-arr}\]
  $size([[A1 -> A2]]) + size ([[B1 -> B2]]) > size([[A2]]) + size([[B2]])$.
\item Case \[\drule{D-andL}\]
  $size([[A1 & A2]]) + size ([[B]])$ is greater than $size([[A1]]) +
  size([[B]])$ and $size([[A2]]) + size([[B]])$.
\item Case \[\drule{D-andR}\]
  $size([[B1 & B2]]) + size ([[A]])$ is greater than $size([[B1]]) +
  size([[A]])$ and $size([[B2]]) + size([[A]])$.
\item Case \[\drule{D-rcdEq}\]
  $size([[{l:A}]]) + size ([[{l:B}]])$ is greater than $size([[A]]) +
  size([[B]])$.
\item Case \[\drule{D-rcdNeq}\]
  It's decidable whether $[[l1]]$ is equal to $[[l2]]$.
\item Case \[\drule{D-tvarL}\]
  By \cref{lemma:decide-sub}, it's decidable whether $[[A<:B]]$.
\item Case \[\drule{D-tvarR}\]
  By \cref{lemma:decide-sub}, it's decidable whether $[[A<:B]]$.
\item Case \[\drule{D-forall}\]
  $size([[\X**A1.B1]]) + size ([[\X**A2.B2]])$ is greater than $size([[B1]]) +
  size([[B2]])$.
\item Case \[\drule{D-ax}\]
  By \cref{lemma:decide-disa} it's decidable whether $[[A **a B]]$.
\end{itemize}
\qed

% \begin{theorem}[Decidability of typing]
%   \label{lemma:decide-typing}
%   Given $[[DD]]$, $[[GG]]$, $[[ee]]$ and $[[A]]$, it is decidable whether $[[DD ; GG  |- ee dirflag A]]$.
% \end{theorem}
% \decidetyp*
% \proof
% The typing judgment $[[DD ; GG  |- ee dirflag A]]$ is syntax-directed.
% And by \cref{lemma:decide-sub} and \cref{lemma:decide-dis}, we know that typing
% is decidable.
% \qed

\section{Circuit Embeddings}
\label{appendix:circuit}

\lstinputlisting[language=haskell,linerange=2-140]{./examples/Scan2.hs}% APPLY:linerange=DSL_FULL



\end{document}
