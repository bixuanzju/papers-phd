\renewcommand{\rulehl}[2][gray!40]{%
  \colorbox{#1}{$\displaystyle#2$}}


\section{Taming Row Polymorphism}

In this section we show how to systematically translate
\rname~\cite{Harper:1991:RCB:99583.99603}---a polymorphic record calculus---into
\fnamee. The translation itself is interesting in two regards: first, it shows
that disjoint polymorphism can simulate row polymorphism;\ningning{won't this be
too strong an argument? there are many row polymorphism systems and we are
only talking about one of them.} second, it also
reveals a significant difference of expressiveness between disjoint polymorphism
and row polymorphism---in particular, we point out that row polymorphism alone
is impossible to encode nested composition, which is crucial for applications of
extensible designs. We first review the syntax and semantics of
\rname. We then discuss a seemingly correct translation that failed to
faithfully convey the essence of row polymorphism. By a careful comparison of
the two calculi, we present a type-directed translation,
and prove that the translation is type safe, i.e., well-typed \rname terms map
to well-typed \fnamee terms.
\bruno{besides nested composition, the other advantage of disjoint
  polymorphism is subtyping. Row polymorphism usually does not support
subtyping, so you cannnot write a function f : \{x : Int\} $\to$ Int and
type-check the following application: f \{x=2,y=3\}. With row
polymorphism you must generalize the type of f, and make the interface
of the function more complicated.}
\jeremy{I don't think this is entirely true for row polymorphism in general. Some row systems do have subtyping (see Harper's paper).
  As argued by Harper, it was their design choice to not have subsumption (or subtyping) in the first place,
  in order to have a simpler system. Also with row type inference, you don't actually need to write complicated types for this example.}

% In the process, we identified one broken lemma of \rname due to the design of type equivalence, which is remedied in our presentation.


\subsection{Syntax of \rname}

\begin{figure}[t]
  \centering
\begin{tabular}{llll@{\hskip 0.6cm}llll} \toprule
  Types & $[[rt]]$ & $\Coloneqq$ & $[[base]] \mid [[rt1 -> rt2]] \mid [[\/ a # R .  rt]] \mid [[ r  ]]$ & Constraint lists & $[[R]]$&  $\Coloneqq$ &$[[ <>  ]] \mid [[ r , R ]] $ \\ 
  Records & $[[r]]$ & $\Coloneqq$ & $[[a]] \mid [[Empty]] \mid [[ {l : rt}  ]]  \mid [[  r1 || r2 ]] $  & Term contexts & $[[Gtx]]$ &  $\Coloneqq$ &  $[[ <> ]] \mid [[Gtx , x : rt ]] $ \\
  Terms & $[[re]]$ & $\Coloneqq$ & $[[x]] \mid [[\x : rt . re]] \mid [[re1 re2]] \mid [[rempty]] $ & Type contexts & $[[Ttx]]$ & $\Coloneqq$ & $[[ <> ]] \mid [[Ttx , a # R ]] $ \\
        &          & $\mid $ & $[[{ l = re }]] \mid [[re1 || re2]] \mid [[ re \ l  ]]  \mid [[ re . l  ]] $ \\
        &          & $ \mid$ & $ [[ /\ a # R . re  ]] \mid [[  re [ r ]  ]]$ \\
    \bottomrule
\end{tabular}
  \caption{Syntax of \rname}
  \label{fig:syntax:record}
\end{figure}

We start by briefly reviewing the syntax of \rname, shown in \cref{fig:syntax:record}.

\paragraph{Types.}

Metavariable $[[rt]]$ ranges over types, which include integer types
$[[base]]$, function types $[[rt1 -> rt2]]$, constrained quantification $[[ \/ a # R . rt ]]$
and record types $[[r]]$. The record types are built from record type
variables $[[a]]$, empty record $[[Empty]]$, single-field record types $[[ { l : rt}]]$
and record merges $[[ r1 || r2 ]]$.\footnote{The original \rname also include record
  type restrictions $[[r \ l]]$, which, as they later proved, can be systematically
  erased using type equivalence, thus we omit type-level restrictions but keep term-level restrictions.}
A constraint list $[[R]]$ is a list of record types, used to constrain instantiations of record type variables.
% , and plays
% an important role in the calculus, as we will explain shortly.

\paragraph{Terms.}

Metavariable $[[re]]$ ranges over terms, which include term
variables $[[x]]$, lambda abstractions $[[ \x : rt . re ]]$, function applications $[[re1 re2]]$, empty records $[[rempty]]$,
single-filed records $[[{l = re}]]$, record merges $[[re1 || re2]]$, record restrictions $[[ re \ l ]]$, record projections $[[ re . l  ]]$,
type abstractions $[[  /\ a # R . re ]]$ and type applications $[[ re [ r ]   ]]$.
As a side note, from the syntax of type applications $[[re [ r ] ]]$, it already can be seen that \rname only supports
quantification over \emph{record types}.
% ---though a separate form of quantifier that quantifies over \emph{all types}.
% can be added, Harper and Pierce decided to have only one form of quantifier for the sake of simplicity.

% \paragraph{An example.}

% Before proceeding to the formal semantics, let us first see some examples that
% can be written in \rname, which may be of help in understanding the overall
% system better.
\paragraph{An example.}

When it comes to extension, every record calculus must decide what to do with
duplicate labels. According to Leijen~\cite{leijen2005extensible}, record calculi can
be divided into those that support \emph{free} extension, and those that support
\emph{strict} extension. The former allows duplicate labels to coexist, whereas
the latter does not. In that sense, \rname belongs to the strict camp. What sets
\rname apart from other strict record calculi is its ability to merge records
with statically unknown fields, and a mechanism to ensure the resulting record
is conflict-free (i.e., no duplicate labels). For example, the following
function merges two records:
\[
  \mathsf{mergeRcd} = [[  /\ a1 # Empty . /\ a2 # a1  . \ x1 : a1 . \ x2 : a2 . x1 || x2  ]]
\]
which takes two type variables: the first one has no constraint
($[[Empty]]$) at all and the second one has only one constraint ($[[ a1 ]]$). It
may come as no surprise that $\mathsf{mergeRcd}$ can take any record type as its
first argument, but the second type must be \emph{compatible} with the first. In
other words, the second record cannot have any labels that already exist in the
first. These constraints are enough to ensure that the resulting record $[[x1 ||
x2]]$ has no duplicate labels. If later we want to say that the first record
$[[x1]]$ has \emph{at least} a field $[[l1]]$ of type $[[nat]]$, we can refine
the constraint list of $[[a1]]$ and the type of $[[x1]]$ accordingly:
\[
  [[  /\ a1 # {l1 : nat} . /\ a2 # a1  . \ x1 : a1 || {l1 : nat} . \ x2 : a2 . x1 || x2  ]]
\]
The above examples suggest an important point: the form of constraint used in
\rname can only express \emph{negative} information about record type variables.
Nonetheless, with the help of the merge operator, positive information can be
encoded as merges of record type variables, e.g., the assigned type of $[[x1]]$
illustrates that the missing field $[[ {l1 : nat} ]]$ is merged back into
$[[a1]]$.

The acute reader may have noticed some correspondences between \rname and
\fnamee: for instance, $[[ /\ a # R . re ]]$ vs. $[[ \ X ** A . ee ]]$,
and $[[x1 || x2]]$ vs.  $[[ x1 ,, x2 ]]$. Indeed, the very
function can be written in \fnamee almost verbatim:
\[
  \mathsf{mergeAny} = [[\ a1 ** Top . \ a2 ** a1 . \x1 : a1 . \x2 : a2 . x1 ,, x2 ]]
\]
However, as the name suggests, $\mathsf{mergeAny}$ works for \emph{any} two types,
not just record types.

\subsection{Typing Rules of \rname}
\label{sec:typing_rname}

% \jeremy{We may only want to show selected rules. }

The type system of \rname consists of several conventional judgments. The
complete set of rules appear in \cref{appendix:rname}.
\Cref{fig:rname_well_formed} presents the well-formedness rules for record
types. % Most cases are quite standard.
A merge $[[r1 || r2]]$ is well-formed in
$[[Ttx]]$ if $[[r1]]$ and $[[r2]]$ are well-formed, and moreover,
$[[r1]]$ and $[[r2]]$ are compatible in $[[Ttx]]$ (\rref{wfr-Merge})---the most
important judgment in \rname, as we will explain next.

\begin{figure}[t]
  \centering
% \drules[wftc]{$[[  Ttx ok ]]$}{Well-formed type contexts}{Empty, Tvar}

% \drules[wfc]{$[[  Ttx |- Gtx ok ]]$}{Well-formed term contexts}{Empty, Var}

% \drules[wfrt]{$[[  Ttx |- rt type ]]$}{Well-formed types}{Prim, Arrow, All, Rec}

\drules[wfr]{$[[  Ttx |- r record ]]$}{Well-formed record types}{Var, Merge}

% \drules[wfcl]{$[[  Ttx |- R ok ]]$}{Well-formed constraint lists}{Nil, Cons}
  \caption{Selected rules for well-formedness of record types}
  \label{fig:rname_well_formed}
\end{figure}



\paragraph{Compatibility.}

The compatibility relation in \cref{fig:compatible} plays a central role in \rname. It is the underlying
mechanism of deciding when merging two records is ``sensible''. Informally,
$[[Ttx |- r1 # r2]]$ holds if $[[r1]]$ and $[[r2]]$ are mergeable, that is,
$[[r1]]$ lacks every field contained in $[[r2]]$ and vice versa.
Compatibility is
symmetric (\rref{cmp-Symm}) and respects type equivalence (\rref{cmp-Eq}).
\Rref{cmp-Base} says that if a record is compatible with a single-field record
$[[{l : t}]]$, it is also compatible with every record $[[{l : t'}]]$. A type variable is compatible
with the records in its constraint list (\rref{cmp-Tvar}). Two single-field
records are compatible if they have different labels (\rref{cmp-BaseBase}). The
rest are self-explanatory.

% and we refer the reader to their paper for further explanations.

% The judgment of constraint list satisfaction $[[Ttx |- r # R]]$
% ensures that $[[r]]$ must be compatible with every record in the constraint list $[[R]]$.
% With the compatibility rules, let us go back to the definition of $\mathsf{mergeRcd}$
% and see why it can type check, i.e.,  why $[[a1]]$ and $[[a2]]$ are compatible---because
% $[[a1]]$ is in the constraint list of $[[a2]]$, and by \rref{cmp-Tvar}, they are compatible.


\begin{figure}[t]
  \centering
\drules[cmp]{$[[  Ttx |- r1 # r2 ]]$}{Compatibility}{Eq, Symm, Base, Tvar, MergeE,Empty,MergeI,BaseBase}
% \drules[cmpList]{$[[  Ttx |- r # R ]]$}{Constraint list satisfaction}{Nil, Cons}
\caption{Compatibility}
\label{fig:compatible}

\end{figure}

\begin{figure}[t]
  \centering
\drules[teq]{$[[  rt1 ~ rt2 ]]$}{Type equivalence}{MergeUnit,MergeAssoc,MergeComm,CongAll}
\drules[ceq]{$[[  R1 ~ R2 ]]$}{Constraint list equivalence}{Swap,Empty,Merge,Dupl,Base}
\caption{Selected type equivalence rules}
\label{fig:type_equivalence}
\end{figure}

\paragraph{Type equivalence.}

Unlike \fnamee, \rname does not have subtyping. Instead, \rname uses type
equivalence to convert terms of one type to another. A selection of the rules
defining equivalence of types and constraint lists appears in
\cref{fig:type_equivalence}. The relation $[[rt1 ~ rt2]]$ is an
equivalence relation, and is a congruence with respect to the type constructors.
Finally merge is associative (\rref{teq-MergeAssoc}), commutative
(\rref{teq-MergeComm}), and has $[[Empty]]$ as its unit (\rref{teq-MergeUnit}).
As a consequence, records are identified up to permutations. Since the
quantifier in \rname is constrained, the equivalence of constrained
quantification (\rref{teq-CongAll}) relies on the equivalence of constraint
lists $[[R1 ~ R2]]$. Again, it is an equivalence relation, and respects
type equivalence. Constraint lists are essentially finite sets, so order and
multiplicity of constraints are irrelevant (\rref{ceq-Swap,ceq-Dupl}). Merges of
constraints can be ``flattened'' (\rref{ceq-Merge}), and occurences of
$[[Empty]]$ may be eliminated (\rref{ceq-Empty}). The last rule \rref*{ceq-Base}
is quite interesting: it says that the types of single-field records are
ignored. The reason is that as far as compatibility is concerned, only labels
matter, thus changing the types of records will not affect their compatibility
relation. We will have more to say about this in \cref{sec:trouble}, in
particular, this is the rule that forbids a simple translation.

% \begin{remark}
% \jeremy{If we have space trouble, we can delete this}
%   The original rules of type equivalence~\cite{Harper:1991:RCB:99583.99603} do
%   not have contexts (i.e.,  judgment of the form $[[rt1 ~ rt2]]$). However this is incorrect, as it invalidates one of the key
%   lemmas (Lemma 2.3.1.7) in their type system, which says that
%     if $[[Ttx |- r1 # r2]]$, then $[[Ttx |- r1 record]]$ and $[[Ttx |- r2 record]]$.
%   Consider two types $[[  {l1 : nat}  ]]$ and $[[ {l2 : \/ a # {l : nat} || {l : bool} . nat  }   ]]$.
%   According to the original rules, they are compatible because
%   \begin{inparaenum}[(1)]
%   \item $[[  {l1 : nat} ]]  $ is compatible with $ [[ {l2 : \/ a # {l : nat} , {l : bool} . nat }  ]]$ by \rref{cmp-BaseBase};
%   \item $ [[ {l2 : \/ a # {l : nat} , {l : bool} . nat }  ~ {l2 : \/ a # {l : nat} || {l : bool} . nat } ]]$.
%   \end{inparaenum}
%   Then it follows that $[[ {l2 : \/ a # {l : nat} || {l : bool} . nat } ]]$ is well-formed.
%   However, this record type is not well-formed in any context because $[[{l : nat} || {l : bool}]]$
%   is not well-formed in any context. To fix this, we add context throughout type equivalence.
%   % The culprit is \rref{ceq-Merge}---the well-formedness of $[[ r1 , (r2, R) ]]$
%   % does not necessarily entail the well-formedness of $[[ (r1 || r2) ,R]]$, as
%   % the latter also requires the compatibility of $[[r1]]$ and $[[r2]]$.
%   % That is why we need to explicitly add contexts to type equivalence
%   % so that $ [[ {l2 : \/ a # {l : nat} , {l : bool} . nat } ]] $ and $[[ {l2 : \/ a # {l : nat} || {l : bool} . nat } ]]$
%   % are not considered equivalent in the first place.
% \end{remark}


\paragraph{Typing rules.}

A selection of typing rules are shown in \cref{fig:typing_rname}. At
first reading, the gray parts can be ignored, which will be covered in
\cref{sec:row_trans}. % Most of the typing rules are quite standard.
% Typing is
% invariant under type equivalence (\rref{wtt-Eq}).
Two terms can be merged if their types are compatible (\rref{wtt-Merge}). Type
application $[[ re [ r ] ]]$ is well-typed if the type argument $[[r]]$
satisfies the constraints $[[R]]$ (\rref{wtt-AllE}).


\begin{remark}
  A few simplifications have been placed compared to the original \rname,
  notably the typing of record selection (\rref{wtt-Select}) and restriction
  (\rref{wtt-Restr}). In the original formulation, both typing rules need a
  partial function $ r \_ l $ which means the type associated with label $[[l]]$
  in $[[r]]$. Instead of using partial functions, here we explicitly expose the
  expected label in a record. It can be shown that if label $[[l]]$ is present
  in record type $[[r]]$, then the fields in $[[r]]$ can be rearranged so that
  $[[l]]$ comes first by type equivalence. This formulation was also adopted by
  Leijen~\cite{leijen2005extensible}.
\end{remark}


\begin{figure}[t]
  \centering
\drules[wtt]{$[[  Ttx ; Gtx |- re : rt ~~> ee  ]]$}{Type-directed translation}{Base,Restr,Select,Empty,Merge,AllE,AllI}
\caption{Selected typing rules with translations}
\label{fig:typing_rname}
\end{figure}



\renewcommand{\rulehl}[1]{#1}

\subsection{A Failed Attempt}
\label{sec:trouble}

In this section, we sketch out an intuitive translation scheme.
On the syntactic level, it is pretty straightforward to see a one-to-one
correspondence between \rname terms and \fnamee expressions. % For example,
% constrained type abstractions $[[/\ a # R . re ]]$ correspond to \fnamee type
% abstractions $[[ \ a ** A . ee]]$; record merges can be simulated by the more
% general merge operator of \fnamee; record restriction can be modeled as annotate terms, and so on.
On the semantic level, all well-formedness judgments of \rname have corresponding well-formedness judgments
of \fnamee, given a ``suitable'' translation function $[[< rt >]]$ from \rname types to \fnamee types
Compatibility relation corresponds to disjointness relation. What might not be
so obvious is that type equivalence can be expressed via subtyping. More
specifically, $[[ rt1 ~ rt2 ]]$ induces mutual subtyping relations
$[[ < rt1 > <: < rt2 > ]]$ and $[[ < rt2 > <: < rt1 > ]]$.
Informally, type-safety of translation is something along the lines of
``if a term has type $[[rt]]$, then its translation has type $[[< rt > ]]$''.
With all these in mind, let us consider two examples:

\begin{example} \label{eg:1} %
  Consider term $[[ /\ a # {l : nat} . \x : a . x ]]$. It could be
  assigned type (among others) $[[ \/ a # {l : nat} . a -> a ]]$, and its ``obvious'' translation
  $[[  \ X ** {l : nat} . \ x : X . x  ]]$ has type $[[ \ X ** { l : nat} . X -> X   ]]$, which corresponds very well to
  $[[ \/ a # {l : nat} . a -> a  ]]$. The same term could also be assigned type $[[  \/ a # {l : bool} . a -> a   ]]$, since
  $[[  \/ a # {l : bool} . a -> a   ]]$ is equivalent to $[[  \/ a # {l : nat} . a -> a   ]]$ by \rref{teq-CongAll,ceq-Base}. However,
  as far as \fnamee is concerned, these two types  have no relationship at all---$[[  \ X ** {l : nat} . \ x : X . x  ]]$
  cannot have type $[[  \ X ** {l : bool} . X -> X   ]]$, and indeed it should not, as these two types have completely different meanings!
\end{example}

\begin{remark}
  Interestingly, the algorithmic system of \rname can only infer
  type $[[ \/ a # {l : nat} . a -> a ]]$ for the aforementioned term.
  To relate to the declarative system (in particular, to prove completeness of the algorithm),
  they show that the type inferred by the algorithm is equivalent
  (in the sense of type equivalence) to the assignable type in the declarative system.
  Proving type-safety of translation is, in a sense, like proving completeness. So
  maybe we should change the type-safety statement to
  ``if a term has type $[[rt]]$, then there exists type $[[rt']]$ such that $[[ rt ~ rt' ]]$ and the
  translation has type $[[ < rt' > ]]$''. As we shall see, this is still incorrect.
\end{remark}

\begin{example} \label{eg:2} %
  Consider term $[[re]] = [[  /\ a # {l : bool} . \ x : a . \ y : {l : nat} . x || y  ]]$.
  It has type $[[ \/ a # {l : bool} . a -> {l : nat} -> a || {l : nat}    ]]$, and
  its ``obvious'' translation is $[[ee]] = [[ \ X ** {l : bool} . \x  : X . \ y : {l : nat} . x ,, y  ]]$.
  However, expression $[[ee]]$ is ill-typed in \fnamee
  for the reasons of coherence: think about
  the result of evaluating $[[ (ee {l : nat} {l = 1} {l = 2}).l ]]$---it could evaluate to $1$ or $2$!
\end{example}

\begin{remark}
  Let us think about why \rname allows type-checking $[[re]]$. Unlike \fnamee,
  the existence of $[[re]]$ in \rname will not cause incoherence because \rname
  would reject type application $[[re [{l : nat}] ]]$ in the first place---more
  generally, $[[re]]$ can only be applied to records that do not contain label
  $[[l]]$ due to the stringency of the compatibility relation. This example
  underlines a crucial difference between the compatibility relation and the
  disjointness relation. The former can only relate records with different
  labels, whereas the latter is more fine-grained in the sense that it can also
  relate records with the same label (\rref{D-rcdEq}). Note that \rref{D-rcdEq}
  is very important for the applications of extensible designs, as we need to
  combine records with the same label, which is impossible to do in \rname.
\end{remark}


\paragraph{Taming \rname.}

It seems to imply that \rname and \fnamee are incompatible in that there are
some \rname programs that are not typable in \fnamee, and vice versa. A careful
comparison between the two calculi reveals that \rref{cmp-Base,ceq-Base} are
``to blame''. For \rname in general, these two rules are reasonable, as ``the
relevant properties of a record, for the purposes of consistency checking, are
its atomic components''~\cite{Harper:1991:RCB:99583.99603}. As far as
compatibility is concerned, a constraint list is just a list of labels and type
variables, whereas in \fnamee, disjointness constraints also care about record
types. This subtle discrepancy tells that we should have a different treatment
for those records that appear in a constraint list from those that appear
elsewhere: we translate a single-field record $[[ {l : rt}
]]$ in a constraint list to $[[ { l : Bot} ]]$. For \cref{eg:1}, both $[[ \/ a #
{l : nat} . a -> a ]]$ and $[[ \/ a # {l : bool} . a -> a ]]$ translate to $[[ \
X ** { l : Bot} . X -> X ]]$. For \cref{eg:2}, $[[re]]$ is translated to
$[[ee']] = [[ \ X ** {l : Bot} . \x : X . \ y : {l : nat} . x ,, y ]]$, which
type checks in \fnamee. Moreover, $[[ee' {l : nat} ]]$ gets rejected because
$[[Bot]]$ is not disjoint with $[[nat]]$.



\subsection{Type-Directed Translation}
\label{sec:row_trans}

\begin{figure}[t]
  \centering
\begin{tabular}{rrlllrlll} \toprule
  $[[< rt >]] \defeq \,$ & $[[ <base> ]]$ & $=$ &  $[[nat]]$ & $,$ & $[[< rt1 -> rt2 >]]$ & $=$ & $[[<rt1> -> <rt2>]]$ & $,$  \\
                       &$[[< \/ a # R . rt>]]$ & $=$ &  $[[\ X ** <R>. \ Xb ** <R>. <rt>]]$ & $,$ & & & & \\
                       &$[[ <a> ]]$ & $=$ &  $[[X]]$ & $,$ & $[[< Empty > ]]$ & $=$ & $[[Top]] $ & $,$ \\
                       &$[[ <{l:rt}> ]]$ & $=$ &  $[[ {l:<rt>} ]]$ & $,$ & $[[<r1 ||  r2> ]]$ & $=$ & $[[<r1> & <r2>]] $ \\
  $[[ <r>_b  ]] \defeq\,$ & $[[ <a>_b ]]$ & $=$ &  $[[Xb]]$ & $,$ & $[[< Empty >_b ]]$ & $=$ & $[[Top]] $ & $,$  \\
                       &$[[ <{l:rt}>_b ]]$ & $=$ &  $[[ {l:Bot} ]]$ & $,$ & $[[<r1 ||  r2>_b ]]$ & $=$ & $[[<r1>_b & <r2>_b]] $ \\
  $[[< R >]] \defeq \,$  &$[[ < <> >  ]]$ & $=$ &  $[[ Top  ]]$ & $,$ & $[[  <r , R>    ]]$ & $=$ & $[[<r>_b & <R>]] $ \\
  $[[< Ttx >]] \defeq \,$ & $[[ < <> > ]]$ & $=$ &  $[[empty]]$ & $,$ & $[[ <Ttx, a # R>  ]]$ & $=$ & $[[ <Ttx>, X ** <R>, Xb ** <R>]] $  \\
  $[[ < Gtx> ]] \defeq \,$ & $[[ < <> > ]]$ & $=$ &  $[[empty]]$ & $,$ & $[[ <Gtx, x : rt>  ]]$ & $=$ & $[[  <Gtx>, x : <rt>   ]] $ \\   \bottomrule
\end{tabular}
\caption{Translation functions}
\label{fig:trans_func}
\end{figure}




Now we can give a formal account of the translation. But there is still a twist.
Having two ways of translating records does not work out of the box. To see
this, consider $[[ \/ a # b . b ]]$, and note that a reasonably defined translation function
should commute with substitution, i.e., $[[ < [r / a] rt > ]] = [[ <rt> [X ~> <r>] ]] $.
We have LHS:
$$[[ < [ {l : nat} / b  ] (\/ a # b . b ) >  ]] =  [[  < \/ a # {l : nat} . { l : nat} > ]] = [[  \ X ** { l : Bot} . { l : nat}   ]]  $$
which is not the same as RHS:
$$[[ <\/ a # b . b>  [Y ~> < {l : nat} > ]       ]] = [[ ( \ X ** Y . Y) [Y ~> < {l : nat} > ]   ]] = [[   \ X ** {l : nat} . {l : nat}    ]]  $$
The tricky part is that we should also distinguish those record type variables
that appear in a constraint list from those that appear elsewhere. To do so, we
map record type variable $[[a]]$ to a pair of type variables $[[ X ]]$ and
$[[Xb]]$, where $[[Xb]]$ is supposed to be substituted by records with bottom
types. More specifically, we define the translation functions as in
\cref{fig:trans_func}. There are two ways of translating records: $[[<r>]]$ for
regular translation and $[[ < r >_b ]]$ for bottom translation; the latter is
used by $[[< R >]]$ for translating constraint lists. The most interesting one
is translating quantifiers: each quantifier $[[\/ a # R . rt]]$ in \rname is
split into two quantifiers $[[ \ X ** <R>. \ Xb ** <R>. <rt> ]]$ in \fnamee.
Correspondingly, each record type variable $[[a]]$ is translated to either
$[[X]]$ or $[[Xb]]$, depending on whether it appears in a constraint list or
not. The relative order of $[[X]]$ and $[[Xb]]$ is not so much relevant, as long
as we respect the order when translating type applications. Now let us go back
to the gray parts in \cref{fig:typing_rname}. In the type application $[[ re [ r
] ]]$ (\rref{wtt-AllE}), we first translate $[[e]]$ to $[[ee]]$. The translation
$[[ee]]$ is then applied to two types $[[ <r> ]]$ and $[[ <r >_b ]]$, because as
we mentioned earlier, $[[ee]]$ has two quantifiers resulting from the
translation. It is of great importance that the relative order of $[[<r>]]$ and
$[[< r >_b]]$ should match the order of $[[ X ]]$ and $[[Xb]]$ in translating
quantifiers. There is a ``protocol'' that we must keep during translation: if
$[[X]]$ is substituted by $[[ <r> ]]$, then $[[ Xb ]]$ is substitute by $[[ < r
>_b ]]$. In other words, we can safely assume $[[ Xb <: X ]]$ because $[[ < r>_b <: <r> ]]$ always holds.
Similarly, in \rref{wtt-AllI} we translate constrained type abstractions to disjointness type abstractions
with two quantifiers, matching the translation of constrained quantification.
The other rules are mostly straightforward translations. Finally we show that our translation function does commute with
substitution, but in a slightly involved form:

\begin{restatable}{lemma}{substrt} \label{lemma:subst_rt}
  $[[ <[r / a] rt> ]]$ = $ [[ <rt> [X ~> <r>] [Xb ~> <r>_b] ]] $.
\end{restatable}

% With the modified \rname, we are now ready to explain the gray parts in \cref{fig:typing_rname}. First we
% show how to translate \rname types to \fnamee types in
% \cref{fig:type_trans_rname}. Most of them are straightforward. Record merges are
% translated into intersection types, so are the constraint lists. Next we look at the
% translations of terms. Most of the them are quite intuitive. In \rref{wtt-eq},
% we put annotation $[[ | rt' | ]]$ around the translation of $[[re]]$. Record
% restrictions are translated to annotated terms (\rref{wtt-Restr}) since we
% already know the type without label $[[l]]$. Record merges are translated to
% general merges (\rref{wtt-Merge}). The translation of record selections (\rref{wtt-Select}) is  a bit
% complicated. Note that we cannot simply translate to $[[ ee . l ]]$ because our
% typing rule for record selections (\rref{T-proj}) only applies when $[[ee]]$ is a
% single-field record. Instead, we need to first transform $[[ee]]$ to a
% single-field record by annotation, and then project.

% \begin{remark}
%   The acute reader may have noticed that in \rref{wtt-AllE}, the translation type
%   $| [[r]] |$ could be a quantifier, but our rule of type applications
%   (\rref{T-tapp}) only applies to monotypes. The reason is that, for the
%   purposes of translation, we lift the monotype restrictions, which does not
%   compromise type-safety of \fnamee.
% \end{remark}

\paragraph{Type-safety of translation.}

With everything in place, we prove that our translation in
\cref{fig:typing_rname} is type-safe. The main idea is to map each judgment in
\rname to a corresponding judgment in \fnamee: well-formedness to
well-formedness, compatibility to disjointness, type-equivalence to subtyping.
The reader can refer to \cref{appendix:proofs} for detailed proofs. We
show a key lemma that bridges the ``gap'' (i.e., \rref{cmp-Base}) between row and disjoint polymorphism.

\begin{restatable}{lemma}{cmprcd} \label{lemma:cmp-rcd}
  If $[[ Ttx |- r # {l:rt} ]]$ then $[[ < Ttx > |-  < r > ** {l:A}    ]] $ and $[[ < Ttx > |-  < r >_b ** {l:A}    ]]$
  for all $A$.
\end{restatable}

Finally here is the central type-safety theorem:

\begin{restatable}{theorem}{typesafe}
  If $[[ Ttx ; Gtx |- re : rt ~~> ee ]]$ then $[[ < Ttx > ; < Gtx > |-  ee => < rt >  ]]$.
\end{restatable}
% \begin{proof}
%   By induction on the typing derivation.
% \end{proof}




% Local Variables:
% TeX-master: "../paper"
% org-ref-default-bibliography: "../paper.bib"
% End:
