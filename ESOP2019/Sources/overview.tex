
% \section{Overview}


% - Shallow embedding in Haskell (2 interpretations);
% 
% - How to compose? possible but lots of boilerplate;
% 
% - Finally tagless solves some problems, but how about dependencies?
% Still some boilerplate needed. 
% 
% - Introduce the solution in our calculus. Show that we can do
% everything finally tagless can + more because we have 
% distributivity in the type system.

%-------------------------------------------------------------------------------
\section{Compositional Programming}
\label{sec:overview}

% \bruno{Do we need something about easily adding new cases? Although
% this is a solved problem, people may wonder about this? Perhaps
% we need some text (1 or 2 sentences) at least. }


To demonstrate the compositional properties of \fnamee we use Gibbons and Wu's shallow embeddings of
parallel prefix circuits~\cite{DBLP:conf/icfp/GibbonsW14}. By means of several different shallow
embeddings, we first illustrate the short-comings of a state-of-the-art
compositional approach, namely a \emph{finally tagless}
encoding~\cite{CARETTE_2009} in Haskell.
Next we show how parametric polymorphism and distributive intersection types provide
a more elegant and compact solution in \sedel~\cite{bi_et_al:LIPIcs:2018:9214}, a source language built on top of
our \fnamee calculus.


%- - - - - - - - - - - - - - - - - - - - - - - - - - - - - - - - - - - - - - - - 
\subsection{A Finally Tagless Encoding in Haskell}

The circuit DSL represents networks that map a number of inputs (known as the width) of some type $A$ onto
the same number of outputs of the same type. The outputs combine (with repetitions) one or more
inputs using a binary associative operator $\oplus : A \times A \to A$.
A particularly interesting class of circuits that can be expressed in the DSL are
\emph{parallel prefix circuits}. These represent computations that take $n > 0$
inputs $x_1, \ldots, x_n$ and produce $n$ outputs $y_1, \ldots, y_n$, where
$y_i = x_1 \oplus x_2 \oplus \ldots \oplus x_i$.

The DSL features 5 language primitives: two basic circuit constructors and
three circuit combinators. These are captured in the Haskell type class \lstinline[language=haskell]{Circuit}:
\lstinputlisting[language=haskell,linerange=5-10]{./examples/Scan.hs}% APPLY:linerange=DSL_DEF
An \lstinline[language=haskell]{identity} circuit with $n$ inputs $x_i$,  has
$n$ outputs $y_i = x_i$. A \lstinline[language=haskell]{fan} circuit has $n$ inputs $x_i$ and $n$
outputs $y_i$, where $y_1 = x_1$ and $y_j = x_1 \oplus x_j$ ($j > 1)$.
The binary \lstinline[language=haskell]{beside} combinator puts two circuits in parallel; the combined circuit
takes the inputs of both circuits to the outputs of both circuits.
The binary \lstinline[language=haskell]{above} combinator connects the outputs of the first circuit to
the inputs of the second; the width of both circuits has to be same.
Finally,
\lstinline[language=haskell]{stretch ws c} interleaves the wires of circuit \lstinline[language=haskell]{c} with
bundles of additional wires that map their input straight on their output.
The \lstinline[language=haskell]{ws} parameter specifies the width of the consecutive bundles;
the $i$th wire of \lstinline[language=haskell]{c} is preceded by a bundle of width $\textit{ws}_i - 1$.

%- - - - - - - - - - - - - - - - - - - - - - - - - - - - - - - - - - - - - - - -

\begin{figure}[t]
  \begin{subfigure}[b]{0.44\textwidth}
    \lstinputlisting[language=haskell,linerange=15-22]{./examples/Scan.hs}% APPLY:linerange=DSL_WIDTH
    \subcaption{Width embedding}
  \end{subfigure} ~
  \begin{subfigure}[b]{0.57\textwidth}
    \lstinputlisting[language=haskell,linerange=27-34]{./examples/Scan.hs}% APPLY:linerange=DSL_DEPTH
    \subcaption{Depth embedding}
  \end{subfigure}
  \caption{Two finally tagless embeddings of circuits.}\label{fig:finally-tagless}
\end{figure}


\paragraph{Basic width and depth embeddings.}

\Cref{fig:finally-tagless} shows two simple shallow embeddings, which represent a circuit
respectively in terms of its width and its depth. The former denotes the number
of inputs/outputs of a circuit, while the latter is the maximal number of
$\oplus$ operators between any input and output.
Both definitions follow the same setup: a new Haskell datatype
(\lstinline[language=haskell]{Width}/\lstinline[language=haskell]{Depth}) wraps the primitive result value and provides an
instance of the \lstinline[language=haskell]{Circuit} type class that interprets the 5 DSL primitives
accordingly.
The following code creates a so-called Brent-Kung parallel prefix circuit~\cite{brent1980chip}:
\lstinputlisting[language=haskell,linerange=39-42]{./examples/Scan.hs}% APPLY:linerange=DSL_E1
Here \lstinline[language=haskell]{e1} evaluates to \lstinline[language=haskell]$W {width = 4}$. If we want to know the
depth of the circuit, we have to change type signature to \lstinline[language=haskell]{Depth}.

%- - - - - - - - - - - - - - - - - - - - - - - - - - - - - - - - - - - - - - - - 
\paragraph{Interpreting multiple ways.}

Fortunately, with the help of polymorphism we can define a type
of circuits that support multiple interpretations at once.
\lstinputlisting[language=haskell,linerange=47-47]{./examples/Scan.hs}% APPLY:linerange=DSL_FORALL
This way we can provide a single Brent-Kung parallel prefix circuit definition that can be reused
for different interpretations.
\lstinputlisting[language=haskell,linerange=51-54]{./examples/Scan.hs}% APPLY:linerange=DSL_BRENT
A type annotation then selects the desired interpretation.
For instance, \lstinline[language=haskell]{brentKung :: Width} yields the width and
\lstinline[language=haskell]{brentKung :: Depth} the depth.

%- - - - - - - - - - - - - - - - - - - - - - - - - - - - - - - - - - - - - - - - 
\paragraph{Composition of embeddings.}

What is not ideal in the above code is that the same \lstinline[language=haskell]{brentKung}
circuit is processed twice, if we want to execute both interpretations. We can do 
better by processing the circuit only once, computing both interpretations simultaneously.
The finally tagless encoding achieves this with a boilerplate instance
for tuples of interpretations.
\lstinputlisting[language=haskell,linerange=59-64]{./examples/Scan.hs}% APPLY:linerange=DSL_TUPLE
Now we can get both embeddings simultaneously as follows:
\lstinputlisting[language=haskell,linerange=68-69]{./examples/Scan.hs}% APPLY:linerange=DSL_E12
This evaluates to \lstinline[language=haskell]$(W {width = 4}, D {depth = 2})$.

%- - - - - - - - - - - - - - - - - - - - - - - - - - - - - - - - - - - - - - - - 
\paragraph{Composition of dependent interpretations.}

The composition above is easy because the two embeddings are
orthogonal. In contrast, the composition of dependent interpretations is
rather cumbersome in the standard finally tagless setup. An example of the
latter is the interpretation of circuits as their well-sizedness, which
captures whether circuits are well-formed. This interpretation depends on the
interpretation of circuits as their width.\footnote{Dependent recursion schemes
are also known as \emph{zygomorphism}~\cite{fokkinga1989tupling} after the ancient Greek word \emph{\textzeta\textupsilon\textgamma\textomikron\textnu}
for yoke. We have labeled the \lstinline{Width} field with \lstinline{ox} because it is pulling the yoke.}
\lstinputlisting[language=haskell,linerange=74-82]{./examples/Scan.hs}% APPLY:linerange=DSL_WS
The \lstinline[language=haskell]{WellSized} datatype represents the well-sizedness of a circuit with
a Boolean, and also keeps track of the circuit's width. The 5 primitives
compute the well-sizedness in terms of both the width and well-sizedness of the subcomponents.
What makes the code cumbersome is that it has to explicitly delegate to the \lstinline[language=haskell]{Width}
interpretation to collect this additional information.

With the help of a substantially more complicated setup that features a dozen
Haskell language extensions, and advanced programming techniques, we can make
the explicit delegation implicit (see \cref{appendix:circuit}). Nevertheless,
that approach still requires a lot of boilerplate that needs to be repeated for
each DSL, as well as explicit projections that need to be written in each
interpretation. A final remark is that adding new primitives (e.g.,
a ``right stretch'' \lstinline{rstretch}
combinator~\cite{hinze2004algebra}) can also be easily 
achieved in the finally tagless approach~\cite{kiselyov2012typed}.

 
%- - - - - - - - - - - - - - - - - - - - - - - - - - - - - - - - - - - - - - - - 
\subsection{The \sedel Encoding}

\sedel is a source language that elaborates to \fnamee, adding
a few convenient source level constructs.
The \sedel setup of the circuit DSL is similar to the finally tagless
approach. Instead of a \lstinline[language=haskell]{Circuit c} type class, there is a \lstinline{Circuit[C]}
type that gathers the 5 circuit primitives in a record. Like in Haskell, the type
parameter \lstinline{C} expresses that the interpretation of circuits
is a parameter.
\lstinputlisting[linerange=42-45]{./examples/scan.sl}% APPLY:linerange=SEDEL_DEF
As a side note if a new constructor (e.g., \lstinline{rstretch}) is
needed, then this is done by means of
intersection types (\lstinline{&} creates an intersection type) in \sedel:
\lstinputlisting[linerange=50-50]{./examples/scan.sl}% APPLY:linerange=SEDEL_DEF2

%- - - - - - - - - - - - - - - - - - - - - - - - - - - - - - - - - - - - - - - - 
\paragraph{Basic width and depth embeddings.}

\begin{figure}[t]
\lstinputlisting[linerange=60-67]{./examples/scan.sl}% APPLY:linerange=SEDEL_WIDTH
\hrule
\lstinputlisting[linerange=76-83]{./examples/scan.sl}% APPLY:linerange=SEDEL_DEPTH
\caption{Two \sedel embeddings of circuits.}
\label{fig:sedel}
\end{figure}

\Cref{fig:sedel} shows the two basic shallow embeddings for width and
depth. In both cases, a named \sedel definition
replaces the corresponding unnamed
Haskell type class instance in providing the implementations of the 5 language
primitives for a particular interpretation.


The use of the \sedel embeddings is different from that of their Haskell
counterparts. Where Haskell implicitly selects the appropriate type class
instance based on the available type information, in \sedel the programmer
explicitly selects the implementation following the style used by
object algebras.
The following code does this by
% creating an object \lstinline{l1} out of the \lstinline{language1}
% trait and then
building a circuit with \lstinline{l1} (short for \lstinline{language1}).
\lstinputlisting[linerange=88-91]{./examples/scan.sl}% APPLY:linerange=SEDEL_E1
Here \lstinline{e1} evaluates to \lstinline${width = 4}$. If we want to know the
depth of the circuit, we have to replicate the code with \lstinline{language2}.

%- - - - - - - - - - - - - - - - - - - - - - - - - - - - - - - - - - - - - - - - 
\paragraph{Dynamically reusable circuits.}

Just like in Haskell, we can use polymorphism to define a type
of circuits that can be interpreted with different languages.
\lstinputlisting[linerange=96-96]{./examples/scan.sl}% APPLY:linerange=SEDEL_FORALL
In contrast to the Haskell solution, this implementation explicitly accepts
the implementation.
\lstinputlisting[linerange=102-109]{./examples/scan.sl}% APPLY:linerange=SEDEL_BRENT

%- - - - - - - - - - - - - - - - - - - - - - - - - - - - - - - - - - - - - - - - 
\paragraph{Automatic composition of languages.}

Of course, like in Haskell we can also compute both results simultaneously.
However, unlike in Haskell, the composition of the two interpretation requires
no boilerplate whatsoever---in particular, there is no \sedel counterpart of the
\lstinline[language=haskell]{Circuit (c1, c2)} instance. Instead, we can just compose the two interpretations
with the term-level merge operator (\lstinline{,,}) and specify the desired type \lstinline{Circuit[Width & Depth]}.
\lstinputlisting[linerange=114-115]{./examples/scan.sl}% APPLY:linerange=SEDEL_E3
Here the use of the merge operator creates a term with the intersection type
\lstinline{Circuit[Width] & Circuit[Depth]}. Implicitly, the \sedel type system
takes care of the details, turning this intersection type into
\lstinline{Circuit[Width & Depth]}. This is possible because intersection (\lstinline{&}) distributes over function and record types.

%- - - - - - - - - - - - - - - - - - - - - - - - - - - - - - - - - - - - - - - - 
\paragraph{Composition of dependent interpretations.}

In \sedel the composition scales nicely to dependent interpretations.
For instance, the well-sizedness interpretation can be expressed without
explicit projections.
\lstinputlisting[linerange=124-133]{./examples/scan.sl}% APPLY:linerange=SEDEL_WS
Here the \lstinline{WellSized & Width} type in the \lstinline{above} and \lstinline{stretch} cases
expresses that both the well-sizedness and width of subcircuits must be given,
and that the width implementation is left as a dependency---when \lstinline{language3} is used,
then the width implementation must be provided.
Again, the distributive properties of \lstinline{&} in the type system take care of
merging the two interpretations.
\lstinputlisting[linerange=148-150]{./examples/scan.sl}% APPLY:linerange=SEDEL_E4

%- - - - - - - - - - - - - - - - - - - - - - - - - - - - - - - - - - - - - - - - 
\paragraph{Disjoint polymorphism and dynamic merges.}

While it may seem from the above examples that definitions have to be merged
statically, \sedel in fact supports dynamic merges. For instance, we can
encapsulate the merge operator in the \lstinline{combine} function while
abstracting over the two components \lstinline{x} and \lstinline{y} that are merged
as well as over their types \lstinline{A} and \lstinline{B}.
\lstinputlisting[linerange=138-138]{./examples/scan.sl}% APPLY:linerange=SEDEL_COMBINE
This way the components \lstinline{x} and \lstinline{y} are only known at runtime and
thus the merge can only happen at that time.
The types \lstinline{A} and \lstinline{B} cannot be chosen entirely freely. For
instance, if both components would contribute an implementation for the same
method, which implementation is provided by the combination would be ambiguous.
To avoid this problem the two types \lstinline{A} and \lstinline{B} have to be
\emph{disjoint}. This is expressed in the disjointness constraint \lstinline{* A}
on the quantifier of the type variable \lstinline{B}. If a quantifier mentions
no disjointness constraint, like that of \lstinline{A}, it defaults to the
trivial \lstinline{* Top} constraint which implies no restriction. With \lstinline{combine},
we can rewrite \lstinline{l3} as follows (note that \lstinline{Width} and \lstinline{Depth} are disjoint):
\lstinputlisting[linerange=143-143]{./examples/scan.sl}% APPLY:linerange=SEDEL_L3


% We can extend our circuit DSL with additional features.
% Suppose we 
% \begin{Verbatim}[fontsize=\small]
%   addBelow[C,S,R * {below: C -> C -> C, above : C -> C -> C}](lang: Trait[S,{above : C -> C -> C} & R])
%     = trait inherits lang { below(c1 : C, c2 : C) = super.above(c2,c2) }
% \end{Verbatim}


% Local Variables:
% org-ref-default-bibliography: "../paper.bib"
% TeX-master: "../paper"
% End:
