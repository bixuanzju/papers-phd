\section{Background}
\label{sec:background}

In this section we review a simple gradually typed language with
objects~\cite{siek2007gradual}, to introduce the concept of consistency
subtyping. We also briefly talk about the Odersky-L{\"a}ufer type system for
higher-rank types~\cite{odersky1996putting}, which serves as the original
language on which our gradually typed calculus with implicit
higher-rank polymorphism is based.


\subsection{Gradual Subtyping}

\begin{figure}[t]
  \begin{small}
  \begin{mathpar}
    \framebox{$A \tsub B$}\\
    \ObSInt \and \ObSBool \and \ObSFloat \and
    \ObSIntFloat \\ \ObFun \and
    \ObSRecord \and \ObSUnknown
  \end{mathpar}

  \begin{mathpar}
    \framebox{$A \sim B$}\\
    \ObCRefl \and \ObCUnknownR \and
    \ObCUnknownL \and \ObCC \and \ObCRecord
  \end{mathpar}

  \end{small}

  \caption{Subtyping and type consistency in \obb}
  \label{fig:objects}
\end{figure}

\citet{siek2007gradual} developed a gradual typed system for object-oriented
languages that they call \obb. Central to gradual typing is the concept of
\textit{consistency} (written $\sim$) between gradual types, which are types
that may involve the unknown type $\unknown$. The intuition is that consistency
relaxes the structure of a type system to tolerate unknown positions in a
gradual type. They also defined the subtyping relation in a way that static type
safety is preserved. Their key insight is that the unknown type $\unknown$ is
neutral to subtyping, with only $\unknown \tsub \unknown$. Both relations are
found in \Cref{fig:objects}.

A primary contribution of their work is to show that type consistency and
subtyping are orthogonal, and can be superimposed. To compose subtyping and
consistency, \citeauthor{siek2007gradual} defined \textit{consistent subtyping}
(written $\tconssub$) in multiple equivalent ways:

\begin{mdef}[Consistent Subtyping \`a la \citet{siek2007gradual}]\leavevmode
\label{def:old-decl-conssub}
\begin{itemize}
\item $A \tconssub B$ if and only if $A \sim C$ and $C \tsub B$ for some $C$.
\item $A \tconssub B$ if and only if $A \tsub C$ and $C \sim B$ for some $C$.
\end{itemize}
\end{mdef}

Both definitions are non-deterministic because of the intermediate type $C$. To
remove non-determinism, they came up with a so-called \textit{restriction
  operator}, written $\mask A B$ that masks off the parts of a type $A$ that are
unknown in a type $B$. The definition of the restriction operator is given
below:
\begin{small}
\begin{align*}
  \mask A B & =  \kw{case}~(A, B)~\kw{of}\\
               & \mid (-, \unknown) \Rightarrow \unknown\\
               & \mid ([l_1: A_1,...,l_n:A_n], [l_1:B_1,...,l_m:B_m]) \quad \kw{where} n \leq m \Rightarrow\\
               & \qquad [l_1 : \mask {A_1} {B_1}, ..., l_n : \mask {A_n} {B_n}]\\
               & \mid ([l_1: A_1,...,l_n:A_n], [l_1:B_1,...,l_m:B_m]) \quad \kw{where} n > m \Rightarrow\\
               & \qquad [l_1 : \mask {A_1} {B_1}, ..., l_m : \mask {A_m} {B_m},...,l_n:A_n ]\\
               & \mid (-, -) \Rightarrow A\\
  \mask {A_1 \to A_2} {B_1 \to B_2} & =  \mask {A_1} {B_1} \to \mask {A_2} {B_2}
\end{align*}
\end{small}
With the restriction operator, consistent subtyping is simply defined
as $A \tconssub B \equiv \mask A B \tsub \mask B A$. And they proved that this
definition is equivalent to \Cref{def:old-decl-conssub}.


\subsection{The Odersky-L{\"a}ufer Type System}

\begin{figure}[t]
  \begin{small}

    \begin{tabular}{lrcl} \toprule
      Expressions & $e$ & \syndef & $x \mid n \mid
                                    \blam x A e \mid e~e$ \\

      Types & $A, B$ & \syndef & $ \nat \mid a \mid A \to B \mid \forall a. A$ \\
      Monotypes & $\tau, \sigma$ & \syndef & $ \nat \mid a \mid \tau \to \sigma$ \\

      Contexts & $\dctx$ & \syndef & $\ctxinit \mid \dctx,x: A \mid \dctx, a$ \\  \bottomrule
    \end{tabular}

  \begin{mathpar}
    \framebox{$\dctx \byhinf e : A$}\\
    \NVar \and \NNat \and \NLamAnnA \and
    \NApp \and \NSub
  \end{mathpar}

  \begin{mathpar}
    \framebox{$\dctx \bysub A \tsub B$}\\
    \NForallL \and \NForallR \and
    \NFun \and \NTVar \and \NSInt
  \end{mathpar}

  \end{small}
  \caption{Syntax and static semantics of the Odersky-L{\"a}ufer type system.}
  \label{fig:original-typing}
\end{figure}


The calculus we are combining gradual typing with is the fully annotated version
of the well-established type system for higher-rank types proposed by
\citet{odersky1996putting}. One difference is that, for simplicity, we do not account 
for a let expression and, consequently, we do not have
let-generalization. However, there is already existing work about gradual type
systems with let expressions and let generalization (for example, see
\citep{garcia2015principal}). Similar techniques can
be applied to our calculus to enable let generalization.

The syntax of the type system, along with the typing and subtyping judgments is
given in \Cref{fig:original-typing}. We save the explanations for the
static semantics to \Cref{sec:type-system}, where we present our
gradually typed version of the calculus.

%%% Local Variables:
%%% mode: latex
%%% TeX-master: "../paper"
%%% org-ref-default-bibliography: "../paper.bib"
%%% End: