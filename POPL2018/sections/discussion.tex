
\section{Discussion}
\label{sec:discussion}

In this section we consider a simple extension to the language to further
demonstrate the applicability of our definition of consistent subtyping. We also
discuss the design decisions involved in the algorithmic system and show how it
helps address the issue arising from the ambiguity of the type-directed
translation.

\subsection{Extension with Top}
\label{subsec:extension-top}

We argued that our definition of consistent subtyping (\Cref{def:decl-conssub})
is a \textit{general} definition in that it is independent of language features.
We have shown its applicability to polymorphic types, for which neither
\citet{siek2007gradual} nor the AGT approach~\citep{garcia2016abstracting} can
be extended naturally. To strengthen our argument, we consider extending the
language with the $\tope$ type and show that our approach naturally embraces the
extension, with all the desired properties preserved. To aid comparison, we also
show how to adapt the AGT approach to support $\tope$ and verify that these two
approaches, though rooted in different foundations, coincide again on
\textit{simple types}. However, \Cref{def:old-decl-conssub} of
\citet{siek2007gradual} fails to support $\tope$.


\paragraph{Extending Definitions}

In order to preserve the orthogonality between subtyping and consistency, we
require $\top$ to be a common supertype of all static types, as shown in rule
\rul{S-Top}. This rule might seem strange at first glance, since even
if we remove
the requirement $A~static$, the rule seems reasonable.
However, the important point is that because of the orthogonality between
subtyping and consistency, subtyping itself should not contain a potential cast
in principle! Therefore, subtyping instances such as $\unknown \tsub \top$ are not allowed.
For consistency, we add the rule that $\top$ is consistent with $\top$, which is
actually included in the original reflexive rule $A \sim A$. For consistent
subtyping, every type is a consistent subtype of $\top$, for example, $\nat \to
\unknown \tconssub \top$.
\begin{mathpar}
  \SubTop \and \CTop \and \CSTop
\end{mathpar}
It is easy to verify that \Cref{def:decl-conssub} is still equivalent to that in
\Cref{fig:decl:conssub} extended with rule \rul{CS-Top}. That is,
\Cref{lemma:properties-conssub} holds:
\begin{mprop}[Extension with $\top$]
  The following are equivalent:
  \begin{itemize}
  \item  $\tpreconssub A \tconssub B$.
  \item  $\tpresub A \tsub C$, $C \sim D$, $\tpresub D \tsub B$, for some $C, D$.
  \end{itemize}
\end{mprop}
% \begin{proof}\leavevmode
%   \begin{itemize}
%   \item From first to second: By induction on the derivation of consistent
%     subtyping. We have extra case \rul{CS-Top} now, where $B = \top$.
%     We can choose $C = A$, and
%     $D$ by replacing the unknown types in $C$ by $\nat$. Namely, $D$ is a static
%     type, so by \rul{S-Top} we are done.
%   \item From second to first: By induction on the derivation of second
%     subtyping. We have extra case \rul{S-Top} now, where
%     $B = \top$, so $A \tconssub B$ holds by \rul{CS-Top}.
%   \end{itemize}
% \end{proof}

\paragraph{Extending AGT}

We now extend the definition of concretization (\Cref{def:concret}) with $\top$
by adding another equation:
\[
  \gamma(\top) = \{\top\}
\]
It is easy to verify that \Cref{lemma:coincide-agt} still holds:
\begin{mprop}[Equivalent to AGT Extended with $\top$ on Simple Types]
  \label{prop:agt-top}
  $A \tconssub B$ if only if $A \agtconssub B$.
\end{mprop}

% \begin{proof}\leavevmode
%   \begin{itemize}
%   \item From left to right: By induction on the derivation of consistent
%     subtyping. We have case \rul{CS-Top} now.
%     It follows that for
%     every static type $A_1 \in \gamma(A)$, we can derive $A_1 \tsub \top$ by
%     \rul{S-Top}.
%     We have $B_1 = B = \top$ and we are done.
%   \item From right to left: By induction on the derivation of subtyping and
%     inversion on the concretization. We have extra case \rul{S-Top} now, where
%     $B$ is $\top$. So
%     consistent subtyping directly holds.
%   \end{itemize}
% \end{proof}

\paragraph{\citeauthor{siek2007gradual}'s definition of consistent subtyping does not work for $\top$}

Similarly to the analysis in \Cref{subsec:towards-conssub}, $\nat \to \unknown
\tconssub \top$ only holds when we first apply consistency, then subtyping, as
shown in the following diagram. However we cannot find a type $A$ such that
$\nat \to \unknown \tsub A$ and $A \sim \top$. Also we have a similar problem in
extending the restriction operator: \textit{non-structural} masking between
$\nat \to \unknown$ and $\top$ cannot be easily achieved.
\begin{center}
  \begin{tikzpicture}
    \matrix (m) [matrix of math nodes,row sep=3em,column sep=4em,minimum width=2em]
    {
      \bot & \top \\
      \nat \to \unknown &
      \nat \to \nat \\};

    \path[-stealth]
    (m-2-1) edge node [left] {$\tsub$} (m-1-1)
    (m-2-2) edge node [left] {$\tsub$} (m-1-2);

    \draw
    (m-1-1) edge node [above] {$\sim$} (m-1-2)
    (m-2-1) edge node [below] {$\sim$} (m-2-2);
  \end{tikzpicture}
\end{center}


\subsection{Better to be Unknown}
\label{subsec:algo:discuss}

In \Cref{sec:type:trans} we have seen an example where a source expression could
produce two different target expressions with different runtime behaviour. As we
explained, this is due to the guessing nature of the declarative system, and
from the typing point of view, no type is particularly better than others.
However in practice, this is not desirable. Let us revisit the same example, now
from the algorithmic point of view (we omit the translation for space reasons,
the interested reader can try to write down the full derivation):
\[
  f: \forall a. a \to a \byinf (\blam x \unknown {f ~ x}) \infto \unknown \to \genA \dashv f : \forall a. a \to a, \genA
\]
Compared with declarative typing, which can produce as many types as
possible ($\unknown \to \nat$, $\unknown \to \bool$, and so on), the algorithm
computes the type $\unknown \to \genA$ with $\genA$ unsolved in the output
context. This is due to rule \rul{ACS-UnknownL}. What can we know from the
output context? The only thing we know is that $\genA$ is not constrained at
all! As we discussed, any monotype for $\genA$ is inappropriate. Instead, we
replace $\genA$ with the unknown type $\unknown$, which helps to avoid unnecessary
down-casts at runtime (any cast to $\unknown$ is safe), resulting in the final
type $\unknown \to \unknown$.


\paragraph{Do soundness and completeness still hold?}

The reader may ask if the declarative system produces types such as $\unknown
\to \nat$ and $\unknown \to \bool$, but the algorithmic system computes the type
$\unknown \to \unknown$, does it imply that the algorithmic system is no longer
sound and complete with respect to the declarative system? The answer is no.
First of all, note that \Cref{thm:type_complete} reads ``$\dots$ there exist
$\Delta$, $\Omega'$ and $A'$ such that $\dots$ $A = \ctxsubst{\Omega'}{A'}$''.
Now if $A'$ (which is produced by the algorithmic system) contains some unsolved
existential variables, it is up to us to pick solutions in $\Omega'$ to match up
with $A$ (which is produced by the declarative system). More concretely, let us
assume that the declarative system produces type $\unknown \to \nat$, and let
$\Omega = f : \forall a. a \to a$ and $\Delta = f : \forall a. a \to a, \genA$,
the completeness theorem asks if we can find a complete context $\Omega'$ that
extends both $\Delta$ and $\Omega$ such that $\ctxsubst{\Omega'}{(\unknown \to
  \genA)} = \unknown \to \nat$. It is obvious that $\Omega' = f : \forall a. a
\to a, \genA = \nat$ is one such complete context, and we have
$\ctxsubst{\Omega'}{(\unknown \to \genA)} = \unknown \to \nat$. So the
algorithmic system is still complete. Similar arguments apply to the soundness
theorem. 
Secondly, an observation (which follows from the soundness and
completeness theorems) is: if an
expression is typeable with many types in the declarative system, the same expression must be
typeable with a single type that contains some unsolved existential variables in the algorithmic
algorithmic system.
In that case, replacing them with $\unknown$ is the best
strategy for the sake of execution.

\paragraph{Existential variables do not indicate parametricity}

Another reading of the above example may suggest that the result type $\unknown
\to \genA$ implies parametricity, with the implication that $\genA$ can be
changed arbitrarily without affecting the runtime behaviour of the
program. A similar phenomenon is discussed by \citet{siek2008gradual}, where they
argue that we cannot simply ``ignore dynamic types during unification''. In
their view, a type signature with a type variable indicates parametricity, but
this type does not. We agree that type variables do indicate parametricity, but
\textit{existential variables} do not! An \textit{unsolved} existential variable
indicates that a value's only constraint is that it may be cast to and from
$\unknown$, thus may introduce runtime casts. A similar observation is also
found in \citet{garcia2015principal}, where they have to distinguish between
\textit{static polymorphism} and \textit{gradual polymorphism}, and in addition
to gradual type parameters, they have to introduce the so-called \textit{static
  type parameters}. We argue that our language design is much simpler, and our
algorithmic system naturally embraces this distinction.

Now the question is can we really do that? The syntax in \Cref{fig:algo-syntax}
specifies that the solution of a existential variable can only be a monotype.
This is true if the existential variable has a solution. Reading of the output
context reveals that $\genA$ does not have any solution at all. What is more,
our target language (i.e., \pbc) has a nice property, the so-called
``Jack-of-All-Trade Principle''~\cite{ahmed2011blame} that says if instantiating
a type parameter to any given type yields an answer then instantiating that type
parameter to $\unknown$ yields the same answer. In light of these, no type is
more suitable than $\unknown$.

We need to note that this does not mean we should always instantiate a type
parameter to $\unknown$, as is the case in \pbc. One of our design principles is
that we should extract as much information as possible from the static aspects of
the type system, until there is nothing more we can know, then leave the job to
the runtime checks.



%%% Local Variables:
%%% mode: latex
%%% TeX-master: "../paper"
%%% org-ref-default-bibliography: "../paper.bib"
%%% End:
