\section{Revisiting Consistent Subtyping}
\label{sec:exploration}

In this section we explore the design space of consistent subtyping. In addition
to the unknown type $\unknown$, we have the same syntax of types as in
\Cref{fig:original-typing}. We start with the definitions of consistency and
subtyping for polymorphic types, and compare with some relevant work (in
particular the compatibility relation by \citet{ahmed2011blame} and type
consistency by \citet{yuu2017poly}). We then discuss the design decisions
involved towards our new definition of consistent subtyping, and justify the new
definition by demonstrating its equivalence with that of \citet{siek2007gradual}
and the AGT approach~\cite{garcia2016abstracting} on simple types.

% \subsection{Language Overview}

% \begin{figure}[t]
%   \centering
%   \begin{small}
% \begin{tabular}{lrcl}
%   Expressions & $e$ & \syndef & $x \mid n \mid
%                          \blam x A e \mid e~e$ \\
% %%                         \mid \erlam x e \equiv \blam x \unknown e $ \\

%   Types & $A, B$ & \syndef & $ \nat \mid a \mid A \to B \mid \forall a. A \mid \unknown$ \\
%   Monotypes & $\tau, \sigma$ & \syndef & $ \nat \mid a \mid \tau \to \sigma$ \\

%   Contexts & $\dctx$ & \syndef & $\ctxinit \mid \dctx,x: A \mid \dctx, a$ \\
%   Syntactic Sugar & $\erlam x e$ & $\equiv$ & $\blam x \unknown e$ \\
%                   & $e : A$ & $\equiv$ & $(\blam x A x) ~ e$
% \end{tabular}
%   \end{small}
% \caption{Syntax of the declarative type system}
% \label{fig:decl-syntax}
% \end{figure}
%%\bruno{Do not use the notation for sugar in the lambda expression: it
%%  is confusing. Explain it in text.}


%  The syntax of our language is given in \Cref{fig:decl-syntax}.
% Compared
% with the Odersky-L{\"a}ufer type system, the only addition is the unknown type
% $\unknown$. We use the meta-variable $e$ to range over expressions. There are
% variables $x$, integers $n$, annotated lambda abstraction $\blam x A e$, and
% application $e_1 ~ e_2$. We write $A$, $B$ for types. They are the integer type
% $\nat$, type variables $a$, functions $A \to B$, universal quantification
% $\forall a. A$, and the unknown type $\unknown$. Monotypes $\tau$ contain all
% types other than the universal quantifier and the unknown type. Contexts $\dctx$
% map term variables to their types, and record all type variables with the
% expected well-formdness condition. Following \citet{siek2006gradual}, if a
% lambda binder is without annotation, it is automatically annotated with
% $\unknown$. As a convenience, the language also provides type ascription $e :
% A$, which is simulated as $(\blam x A x) ~ e$.

\subsection{Consistency and Subtyping}
\label{subsec:consistency-subtyping}

We start by giving the definitions of consistency and subtyping for polymorphic
types, and showing the differences of our definitions from other works, in
particular the compatibility relation by \citet{ahmed2011blame} and type
consistency by \citet{yuu2017poly}.

\begin{figure}[t]
  \begin{small}
  \begin{mathpar}
    \framebox{$A \sim B$} \\
    \CD \and \CA \and \CB \and \CC \and \CE
  \end{mathpar}

  \begin{mathpar}
    \framebox{$\dctx \bywf A $} \\
    \DeclVarWF \and \DeclIntWF \and \DeclUnknownWF \\ \DeclFunWF \and \DeclForallWF
  \end{mathpar}

  \begin{mathpar}
    \framebox{$\tpresub A \tsub B$} \\
    \HSForallR \and \HSForallL \and \HSFun \and
    \HSTVar \and \HSInt \and \HSUnknown
  \end{mathpar}
  \end{small}
  \caption{Consistency, well-formedness of types and subtyping in the declarative system.}
  \label{fig:decl:subtyping}
\end{figure}

\paragraph{Consistency}
The key observation here is that consistency is mostly a structural relation,
except that the unknown type $\unknown$ can be regarded as any type. Following
this observation, we naturally extend the definition from
\Cref{fig:objects} with polymorphic types, as shown at the top of
\Cref{fig:decl:subtyping}. In particular a polymorphic type $\forall a. A$
is consistent with another polymorphic type $\forall a. B$ if $A \sim B$.

\paragraph{Subtyping}

We express the fact that one type is a polymorphic generalization of another by
means of the subtyping judgment $\Psi \vdash A \tsub B$. Compared with the
subtyping rules of \citet{odersky1996putting} in
\Cref{fig:original-typing}, the only addition is the neutral subtyping of
$\unknown$, given at the bottom of \Cref{fig:decl:subtyping}. Notice
that in the rule
\rul{S-ForallL}, the universal quantifier is only allowed to be instantiated
with a \emph{monotype}. According to the syntax in \Cref{fig:original-typing},
monotypes do not contain the unknown type $\unknown$. This is because if we were
to allow the unknown type to be used for instantiation, we could have the
following subtyping relation
\[
  \forall a . a \to a \tsub \unknown \to \unknown
\]
by instantiating $a$ with $\unknown$. Since $\unknown \to \unknown$ is
consistent with any functions $A \to B$, for instance, $\nat \to \bool$, this
means that we could provide an expression of type $\forall a. a \to a$ to a
function where the input type is supposed to be $\nat \to \bool$. However, as we
might expect, $\forall a. a \to a$ is definitely not compatible with $\nat \to
\bool$. This does not hold in any polymorphic type systems without gradual
typing. So the gradual type system should not accept it either. (This is the
so-called \textit{conservative extension} property that will be made precise in
\Cref{sec:criteria}.)

Importantly there is a subtle but crucial distinction between a type variable
and the unknown type, although they all represent a kind of ``arbitrary'' type.
The unknown type stands for the absence of type information: it could be
\textit{any type} at \textit{any instance}. Because of the absence of type
information, the unknown type is consistent with any type, and additional
type checks may have to be performed at runtime. On the other hand, a type
variable denotes some instantiation of a universal quantifier, and is subject to
global constraints. In other words, a type variable can only be instantiated to
a single type. For example, in the type $\forall a. a \to a$, the two
occurrences of $a$ represent an arbitrary but single type (e.g., $\nat \to
\nat$, $\bool \to \bool$), while $\unknown \to \unknown$ could be an arbitrary
function (e.g., $\nat \to \bool$) at runtime.

\paragraph{Comparison with Other Relations}

In other polymorphic gradual calculi, consistency and subtyping are often mixed
up to some extent. In the Polymorphic Blame Calculus
(\pbc)~\citep{ahmed2011blame}, the compatibility relation for polymorphic types
is defined as follows:
\begin{mathpar}
  \CompAllR \and \CompAllL
\end{mathpar}
Notice that, in rule \rul{Comp-AllL}, the universal quantifier is \textit{always}
instantiated to $\unknown$. However, in this way, \pbc allows $\forall a. a \to a
\pbccons \nat \to \bool$, which as we discussed before might not be what we
expect. Indeed \pbc relies on sophisticated runtime checks to rule out such
instances of the compatibility relation \`a posteriori.

\citet{yuu2017poly} introduced the so-called
\textit{quasi-polymorphic} types for types that may be used where a
$\forall$-type is expected, which is important for their purpose of
conservativity over System F. Their type consistency relation, involving polymorphism, is
defined as follows\footnote{This is a simplified version.}:
\begin{mathpar}
  \inferrule{A \sim B }{\forall a. A \sim \forall a. B}
  \and
  \inferrule{A \sim B \\ B \neq \forall a. B' \\ \unknown \in \mathsf{Types}(B)}
  {\forall a. A \sim B}
\end{mathpar}
Compared with our consistency definition in \Cref{fig:decl:subtyping},
their first rule is the same as ours. The second rule says that a non
$\forall$-type can be consistent with a $\forall$-type only if it contains
$\unknown$. In this way, their type system is able to reject $\forall a. a \to a
\sim \nat \to \bool$. However, in order to keep conservativity, they also reject
$\forall a. a \to a \sim \nat \to \nat$, which is perfectly sensible in their
setting (i.e., explicit polymorphism). However with implicit polymorphism, we
would expect $\forall a. a \to a$ to be related with $\nat \to \nat$, which is
exactly the case in our subtyping relation since $a$ can be instantiated to
$\nat$.

Nonetheless, when it comes to interactions between dynamically typed and
polymorphically typed terms, both relations allow for example $(\forall a. a) \to
\nat$ to be related with $\unknown \to \nat$, which in our view, is some sort of
(implicit) polymorphic subtyping, and that should be achievable by the more
primitive notions in the type system (instead of inventing new relations). One
of our design principles is that, subtyping and consistency should be
\textit{orthogonal}, and can be naturally superimposed, echoing the same opinion
by \citet{siek2007gradual}.

\subsection{Towards Consistent Subtyping}
\label{subsec:towards-conssub}

With the definitions of consistency and subtyping, the question now is how to
compose these two relations so that two types can be compared in a way that takes
these two relations into account.

Unfortunately, the original definition of \citet{siek2007gradual}
(\Cref{def:old-decl-conssub}) does not work well with our definitions of
consistency and subtyping for polymorphic types. Consider two types: $(\forall
a. a) \to \nat$, and $\unknown \to \nat$. The first type can only reach the
second type in one way (first by applying consistency, then subtyping), but not the
other way, as shown in \Cref{fig:example:a}. We use $\bot$ to mean that we
cannot find such a type. Similarly, there are situations where the first type
can only reach the second type using the other way (first applying
subtyping, and then
consistency), as shown in \Cref{fig:example:b}.

\begin{figure}
  \begin{subfigure}[b]{.4\linewidth}
    \centering
      \begin{tikzpicture}
        \matrix (m) [matrix of math nodes,row sep=3em,column sep=4em,minimum width=2em]
        {
          \bot & \unknown \to \nat \\
          (\forall a. a) \to \nat & (\forall a. \unknown) \to \nat \\};

        \path[-stealth]
        (m-2-1) edge node [left] {$\tsub$} (m-1-1)
        (m-2-2) edge node [left] {$\tsub$} (m-1-2);

        \draw
        (m-1-1) edge node [above] {$\sim$} (m-1-2)
        (m-2-1) edge node [below] {$\sim$} (m-2-2);
      \end{tikzpicture}
      \caption{}
      \label{fig:example:a}
  \end{subfigure}
  \begin{subfigure}[b]{.4\linewidth}
    \centering
    \begin{tikzpicture}
      \matrix (m) [matrix of math nodes,row sep=3em,column sep=4em,minimum width=2em]
      {
        \nat \to \nat & \nat \to \unknown \\
        \forall a. a & \bot \\};

      \path[-stealth]
      (m-2-1) edge node [left] {$\tsub$} (m-1-1)
      (m-2-2) edge node [left] {$\tsub$} (m-1-2);

      \draw
      (m-1-1) edge node [above] {$\sim$} (m-1-2)
      (m-2-1) edge node [below] {$\sim$} (m-2-2);
    \end{tikzpicture}
    \caption{}
    \label{fig:example:b}
  \end{subfigure}
  \begin{subfigure}[b]{.8\linewidth}
    \centering
    \begin{tikzpicture}
      \matrix (m) [matrix of math nodes,row sep=3em,column sep=4em,minimum width=2em]
      {
        \bot &
        ((\unknown \to \nat) \to \bool) \to (\nat \to \unknown)  \\
        (((\forall a. a) \to \nat) \to \bool) \to (\forall a. a) &
        \bot \\};

      \path[-stealth]
      (m-2-1) edge node [left] {$\tsub$} (m-1-1)
      (m-2-2) edge node [left] {$\tsub$} (m-1-2);

      \draw
      (m-1-1) edge node [above] {$\sim$} (m-1-2)
      (m-2-1) edge node [below] {$\sim$} (m-2-2);
    \end{tikzpicture}
    \caption{}
    \label{fig:example:c}
  \end{subfigure}
  \caption{Examples that break the original definition of consistent subtyping.}
  \label{fig:example}
\end{figure}

What is worse, if those two examples are composed in a way that those types all
appear co-variantly, then the resulting types cannot reach each other by either
way. For example, \Cref{fig:example:c} shows such two types by putting a
$\bool$ type in the middle, and neither definition of consistent subtyping
works. But such types ought to be related somehow!

\paragraph{Observations on consistent subtyping}

In order to develop the correct definition of consistent subtyping for
polymorphic types, we need to understand how consistent subtyping works.
We first review two important properties of subtyping: 1) subtyping induces the
subsumption rule: if $A \tsub B$, then an expression of type $A$ can be used
where $B$ is expected; 2) subtyping is transitive: if $A \tsub B$, and $B \tsub
C$, then $A \tsub C$. Though consistent subtyping takes the unknown type into
consideration, the subsumption rule should also apply: if $A \tconssub B$, then
an expression of type $A$ can also be used where $B$ is expected, given that
there might be some information lost by consistency. A crucial difference from
subtyping is that consistent subtyping is \textit{not} transitive because
information can only be lost once (otherwise, any two types are a consistent
subtype of each other). Now consider a situation where we have both $A \tsub B$,
and $B \tconssub C$, this means that $A$ can be used where $B$ is expected, and
$B$ can be used where $C$ is expected, with possibly some loss of information. In
other words, we should expect that $A$ can be used where $C$ is expected, since
there is at most one-time loss of information.

\begin{observation}
  If $A \tsub B$, and $B \tconssub C$, then $A \tconssub C$.
\end{observation}

This is reflected in \Cref{fig:obser:a}. A similar and symmetrical
observation is given in \Cref{fig:obser:b}:

\begin{observation}
  If $C \tconssub B$, and $B \tsub A$, then $C \tconssub A$.
\end{observation}

\begin{figure}[t]
  \centering
  \begin{subfigure}[b]{.4\linewidth}
    \centering
    \begin{tikzpicture}
      \matrix (m) [matrix of math nodes,row sep=3em,column sep=4em,minimum width=2em]
      {
        T_1 & C \\
        B   & T_2 \\
        A & \\};

      \path[-stealth]
      (m-3-1) edge node [left] {$\tsub$} (m-2-1)
      (m-2-2) edge node [left] {$\tsub$} (m-1-2)
      (m-2-1) edge node [left] {$\tsub$} (m-1-1);

      \draw
      (m-2-1) edge node [above] {$\sim$} (m-2-2)
      (m-1-1) edge node [above] {$\sim$} (m-1-2);

      \draw [dashed, ->]
      (m-2-1) edge node [above] {$\tconssub$} (m-1-2);

      \path [dashed, ->, out=0, in=0, looseness=2]
      (m-3-1) edge node [right] {$\tconssub$} (m-1-2);
    \end{tikzpicture}
    \caption{}
    \label{fig:obser:a}
  \end{subfigure}
  \centering
  \begin{subfigure}[b]{.4\linewidth}
    \centering
    \begin{tikzpicture}
      \matrix (m) [matrix of math nodes,row sep=3em,column sep=4em,minimum width=2em]
      {
        & A \\
        T_1 & B \\
        C   & T_2 \\};

      \path[-stealth]
      (m-3-1) edge node [left] {$\tsub$} (m-2-1)
      (m-3-2) edge node [left] {$\tsub$} (m-2-2)
      (m-2-2) edge node [left] {$\tsub$} (m-1-2);

      \draw
      (m-2-1) edge node [above] {$\sim$} (m-2-2)
      (m-3-1) edge node [below] {$\sim$} (m-3-2);

      \draw [dashed, ->]
      (m-3-1) edge node [above] {$\tconssub$} (m-2-2);

      \path [dashed, ->, out=135, in=180, looseness=2]
      (m-3-1) edge node [left] {$\tconssub$} (m-1-2);
    \end{tikzpicture}
    \caption{}
    \label{fig:obser:b}
  \end{subfigure}
  \caption{Observations of consistent subtyping}
  \label{fig:obser}
\end{figure}


From the above observations, we can see what the problem is with the original
definition. In \Cref{fig:obser:a}, if $B$ can reach $C$ by $T_1$, then
according to the transitivity of subtyping, $A$ can reach $C$ by $T_1$. However,
if $B$ can only reach $C$ by $T_2$, then $A$ cannot reach $C$ through the
original definition. A similar problem is shown in \Cref{fig:obser:b}: if $C$ can
only reach $B$ by $T_1$, then $C$ cannot reach $A$ through the original definition.

However, it turns out that those two problems can be fixed by the same strategy:
instead of taking one-step subtyping and one-step consistency, our definition of
consistent subtyping allows types to take one-step subtyping, one-step
consistency, and one more step subtyping. Specifically, $A \tsub B \sim T_2 \tsub C$
and $C \tsub T_1 \sim B \tsub A$ have the same relation chain: subtyping,
consistency, and subtyping.

\paragraph{Definition of consistent subtyping} From the above discussion, we are
ready to modify \Cref{def:old-decl-conssub}, and adapt it to our notation:

\begin{mdef}[Consistent Subtyping]
  \label{def:decl-conssub}
  $\tpresub A \tconssub B$, if and only if $\tpresub A \tsub C$, $C \sim D$, and
  $\tpresub D \tsub B$ for
  some $C, D$.
\end{mdef}

\noindent With \Cref{def:decl-conssub}, \Cref{fig:example:c:fix}
illustrates the correct relation chain for the broken example shown in
\Cref{fig:example:c}.

At first sight, \Cref{def:decl-conssub}
seems worse than the original: we need to guess \textit{two} types! It turns out
that \Cref{def:decl-conssub} is a generalization of
\Cref{def:old-decl-conssub}, and they are equivalent in the system by
\citet{siek2007gradual}. We argue that this is the \textit{general} definition of
consistent subtyping, independent of language features, and in particular is
compatible with polymorphic types.


\begin{figure}[t]
  \centering
  \begin{tikzpicture}
    \matrix (m) [matrix of math nodes,row sep=3em,column sep=4em,minimum width=2em]
    {
      ((\forall a. a) \to \nat) \to \bool) \to (\nat \to \nat) &
      ((\forall a. \unknown) \to \nat) \to \bool) \to (\nat \to \unknown)\\
      (((\forall a. a) \to \nat) \to \bool) \to (\forall a. a) &
      ((\unknown \to \nat) \to \bool) \to (\nat \to \unknown)  \\
      };

    \path[-stealth]
    (m-2-1) edge node [left] {$\tsub$} (m-1-1)
    (m-1-2) edge node [left] {$\tsub$} (m-2-2)
    (m-2-1) edge node [above] {$\tconssub$} (m-2-2);

    \draw
    (m-1-1) edge node [above] {$\sim$} (m-1-2);
  \end{tikzpicture}
  \caption{Example that is fixed by the new definition of consistent subtyping.}
  \label{fig:example:c:fix}
\end{figure}


\begin{mprop}\leavevmode
  \label{prop:subsumes}
\begin{itemize}
  \item \Cref{def:decl-conssub} subsumes
    \Cref{def:old-decl-conssub}:
    in \Cref{def:decl-conssub},
    by choosing $D=B$, we have $A\tsub C$ and $C \sim B$; by choosing $C=A$, we have
    $A \sim D$, and $D \tsub B$.
  \item \Cref{def:old-decl-conssub} is equivalent to
    \Cref{def:decl-conssub} in the system by~\citet{siek2007gradual}:
    if $A \tsub C$, $C \sim D$, and $D \tsub
    B$, by \Cref{def:old-decl-conssub},
    we have $A \sim C'$, $C' \tsub D$ for some $C'$. By subtyping
    transitivity, we have $C' \tsub B$. So we have $A \tconssub B$ by $A \sim C'$, and $C'
    \tsub B$.
  \end{itemize}
\end{mprop}

\subsection{Consistent Subtyping Without Existentials}

\Cref{def:decl-conssub} serves as a fine specification of how consistent
subtyping should behave in general. But it is inherently non-deterministic
because of the two intermediate types $C$ and $D$. As with
\citet{siek2007gradual}'s definition,
we need a combined relation to directly compare two types. A first,
and natural attempt is to try to extend the restriction operator for
polymorphic types. 
Unfortunately this does not work, but it is possible to devise a
simple and elegant inductive definition instead.

\paragraph{Attempt on extending the restriction operator}
Suppose we try to extend the restriction operator to account for polymorphic
types. The original restriction operator is structural, meaning that it works
for types of similar structures. But for polymorphic types, two input types
could have different structures due to universal quantifiers, e.g, $\forall a. a
\to \nat$ and $(\nat \to \unknown) \to \nat$. If we try to mask the first type
using the second, it seems hard to maintain the information that $a$ should be
instantiated to a function while ensuring that the return type is masked. There
seems to be no satisfactory way to extend the restriction operator in order to
support this kind of non-structural masking.

\paragraph{Interpretation of the restriction operator and consistent subtyping}
If the restriction operator cannot be extended naturally, it is useful to
take a step back and revisit what the restriction operator actually does. For
consistent subtyping, two input types could have unknown types in different
positions, but we only care about the known parts. To do that, the restriction
operator is used to: 1) erase the type information in one type if the corresponding
position in the other type is the unknown type; and 2) compare the resulting types 
using the normal subtyping relation. The example below shows the
masking-off procedure for the types $\nat \to \unknown \to \bool$ and $\nat \to
\nat \to \unknown$. Since the known parts have the relation that $\nat \to
\unknown \to \unknown \tsub \nat \to \unknown \to \unknown$, we conclude that
$\nat \to \unknown \to \bool \tconssub \nat \to \nat \to \unknown$.
\begin{center}
  \begin{tikzpicture}
    \tikzstyle{column 5}=[anchor=base west, nodes={font=\tiny}]
    \matrix (m) [matrix of math nodes,row sep=1em,column sep=0em,minimum width=2em]
    {
      \nat \to & \unknown & \to & \bool & \mid \nat \to \nat \to \unknown &
      = \nat \to \unknown \to \unknown
      \\
       \nat \to & \nat & \to & \unknown & \mid \nat \to \unknown \to \bool &
      = \nat \to \unknown \to \unknown \\};

    \path[-stealth, ->, out=0, in=0]
    (m-1-6) edge node [right] {$\tsub$} (m-2-6);

    \draw
    (m-1-2.north west) rectangle (m-2-2.south east)
    (m-1-4.north west) rectangle (m-2-4.south east);
  \end{tikzpicture}
\end{center}
Here differences of the types in boxes are erased because of the
restriction operator. Now if we compare the types in boxes directly instead of
through the lens of the restriction operator, we can observe that the
\textit{consistent subtyping relation always holds between the unknown type and
  an arbitrary type.} We can interpret this observation directly using
\Cref{def:decl-conssub}: the unknown type is neutral to subtyping
($\unknown \tsub \unknown$), the unknown type is consistent with any type
($\unknown \sim A$), and subtyping is reflexive ($A \tsub A$). Therefore,
\textit{the unknown type is a consistent subtype of any type ($\unknown
  \tconssub A$), and vice versa ($A \tconssub \unknown$).}

\paragraph{Defining consistent subtyping directly}

From the above discussion, we can define the consistent subtyping relation
directly, \textit{without} resorting to subtyping or consistency at all. The key
idea is that we replace $\tsub$ with $\tconssub$ in
\Cref{fig:decl:subtyping}, get rid of rule \rul{S-Unknown} and add two
extra rules concerning $\unknown$, resulting in the rules of consistent
subtyping in \Cref{fig:decl:conssub}. Of particular interest are the rules
\rul{CS-UnknownL} and \rul{CS-UnknownR}, both of which correspond to what we
just said: the unknown type is a consistent subtype of any type, and vice versa.
\begin{figure}[t]
  \begin{small}
  \begin{mathpar}
    \framebox{$\tpresub A \tconssub B$} \\
    \CSForallR \and \CSForallL \and \CSFun \and
    \CSTVar \and \CSInt \and \CSUnknownL \and \CSUnknownR
  \end{mathpar}
  \end{small}
  \caption{Consistent Subtyping for implicit polymorphism.}
  \label{fig:decl:conssub}
\end{figure}
From now on, we use the symbol $\tconssub$ to refer to the consistent subtyping
relation in \Cref{fig:decl:conssub}, instead of the one in
\Cref{def:decl-conssub}. What is more, we can prove that those two are
equivalent\footnote{Note to reviewers: Theorems with $\mathcal{T}$ are those
  proved in Coq. The same applies to $\mathcal{L}$emmas.}:

\begin{ctheorem} The following definitions are equivalent:
  \label{lemma:properties-conssub}
  \begin{itemize}
  \item  $\tpreconssub A \tconssub B$.
  \item  $\tpresub A \tsub C$, $C \sim D$, $\tpresub D \tsub B$, for some $C, D$.
  \end{itemize}
\end{ctheorem}

\noindent Not surprisingly, consistent subtyping is reflexive:

\begin{clemma}[Consistent Subtyping is Reflexive] \label{lemma:sub_refl}%
  If $\Psi \vdash A$ then $\Psi \vdash A \tconssub A$.
\end{clemma}

\subsection{Abstracting Gradual Typing}
\label{subsec:agt}

\citet{garcia2016abstracting} presented a new foundation for gradual typing that
they call the \textit{Abstracting Gradual Typing} (AGT) approach. In the AGT
approach, gradual types are interpreted as sets of static types, where static
types refer to types containing no unknown types. In this interpretation,
predicates and functions on static types can then be lifted to apply to gradual
types. Central to their approach is the so-called \textit{concretization}
function. For simple types, a concretization $\gamma$ from gradual types to a
set of static types\footnote{For simplification, we directly regard type
  constructor $\to$ as a set-level operator.} is defined as follows:
\begin{mdef}[Concretization]
  \label{def:concret}
  \begin{mathpar}
    \gamma(\nat) = \{\nat\} \and \gamma(A \to B) = \gamma(A) \to \gamma(B) \and
    \gamma(\unknown) = \{\text{All static types}\}
  \end{mathpar}
\end{mdef}

Based on the concretization function, subtyping between static types can be
lifted to gradual types, resulting in the consistent subtyping relation:
\begin{mdef}[Consistent Subtyping in AGT]
  \label{def:agt-conssub}
  $A \agtconssub B$ if and only if $A_1 \tsub B_1$ for some $A_1 \in \gamma(A)$, $B_1 \in \gamma(B)$.
\end{mdef}

\noindent Later they proved that this definition of consistent subtyping coincides with
that of \citet{siek2007gradual} (\Cref{def:old-decl-conssub}).

It seems that the AGT approach of consistent subtyping is quite different from
ours: theirs is defined purely in terms of static subtyping; we directly define
consistent subtyping on gradual types (\Cref{fig:decl:conssub}).
Nonetheless, the two approaches coincide on \textit{simple types}:

\begin{mprop}[Equivalence to AGT on Simple Types]
  \label{lemma:coincide-agt}
  $A \tconssub B$ if only if $A \agtconssub B$.
\end{mprop}
% \begin{proof}\leavevmode
%   \begin{itemize}
%   \item From left to right: By induction on the derivation of consistent
%     subtyping. In cases \rul{CS-UnknownL} and \rul{CS-UnknownR}, since the
%     static set of $\unknown$ contains all static types, it follows that for
%     every static type $A_1 \in \gamma(A)$, we can always find $A_1 \in
%     \gamma(\unknown)$, and by the reflexivity of subtyping, we are done. The
%     rest are trivial cases.
%   \item From right to left: By induction on the derivation of subtyping and
%     inversion on the concretization. If $A$ or $B$ is a unknown type, then
%     consistent subtyping directly holds. Other cases are trivial.
%   \end{itemize}
% \end{proof}

% Further more, the coincidence between our definition and AGT on consistent
% subtyping for objects
% can be proved by showing both are coincided with
% \citet{siek2007gradual}.

This proposition is rather trivial:
in \Cref{fig:decl:conssub},
rule \rul{CS-UnknownL} and \rul{CS-UnknownR} correspond directly to the
concretization of $\unknown$, which contains all static types. Nonetheless,
this reveals two points. First, it validates our definition of consistent
subtyping by another interpretation. Second, as noted by
\citet{garcia2016abstracting}, it shows that consistent subtyping can be derived
from two quite different foundations: one is defined directly on gradual types,
the other is defined purely in terms of static subtyping.

However, as noted by \citet{garcia2016abstracting} in the conclusion, extending
AGT to deal with polymorphism still remains as an open question. The difficulty
possibly stems from the fundamental conflicts between the set-theoretic
interpretation of gradual types and the parametric interpretation of
polymorphic types~\cite{reynolds1983types} (in particular, the interpretation of
type variables). This shows one advantage of our approach in that it is
independent of other concepts. Still, it is a promising line of future work for
AGT, and the question remains whether our definition would coincide with it.




%%% Local Variables:
%%% mode: latex
%%% TeX-master: "../paper"
%%% org-ref-default-bibliography: "../paper.bib"
%%% End: