\section{A Type System with Gradually Typed Implicit Polymorphism}
\label{sec:type-system}

In \Cref{sec:exploration} we have introduced the consistent
subtyping relation that naturally extends to polymorphic types. In
this section we continue with the development by giving a declarative
type system for implicit polymorphism that employs the consistent
subtyping relation. The declarative system itself is already quite
interesting as it is equipped with both higher-rank polymorphism and
the unknown type. Moreover, unlike non-gradual type systems with
higher-rank polymorphism, guessed types affect runtime behaviour if
used by the implicit casts, which raises concerns with respect to
coherency. Our response to those concerns is given in \Cref{subsec:algo:discuss},
after we give a simple
algorithm that implements the declarative system
(\Cref{sec:algorithm}) and discuss soundness and completeness.

% Later in \Cref{sec:algorithm} we give a simple
%algorithm that implements the declarative system.

\subsection{Language Overview}

\begin{figure}[t]
  \centering
  \begin{small}
\begin{tabular}{lrcl} \toprule
  Expressions & $e$ & \syndef & $x \mid n \mid
                         \blam x A e \mid e~e$ \\
%%                         \mid \erlam x e \equiv \blam x \unknown e $ \\

  Types & $A, B$ & \syndef & $ \nat \mid a \mid A \to B \mid \forall a. A \mid \unknown$ \\
  Monotypes & $\tau, \sigma$ & \syndef & $ \nat \mid a \mid \tau \to \sigma$ \\

  Contexts & $\dctx$ & \syndef & $\ctxinit \mid \dctx,x: A \mid \dctx, a$ \\
  Syntactic Sugar & $\erlam x e$ & $\equiv$ & $\blam x \unknown e$ \\
              & $e : A$ & $\equiv$ & $(\blam x A x) ~ e$ \\ \bottomrule
\end{tabular}
  \end{small}
\caption{Syntax of the declarative type system}
\label{fig:decl-syntax}
\end{figure}

The complete syntax of the declarative system is given in
\Cref{fig:decl-syntax}. We use the meta-variable $e$ to range over expressions.
Expressions are either variables $x$, integers $n$, annotated lambda
abstractions $\blam x A e$, or applications $e_1 ~ e_2$. We write $A$, $B$ for
types. Types are either the integer type $\nat$, type variables $a$, functions
types $A \to B$, universal quantification $\forall a. A$, or the unknown type
$\unknown$. Though we only have one base type $\nat$, we also use $\bool$ for
the purpose of illustration. Monotypes $\tau$ contain all types other than the
universal quantifier and the unknown type. Contexts $\dctx$ map term variables
to their types, and record all type variables with the expected well-formedness
condition. Following \citet{siek2006gradual}, if a lambda binder does not have
an annotation, it is automatically annotated with $\unknown$. As a convenience,
the language also provides type ascription $e : A$, which is simulated by
$(\blam x A x) ~ e$.

\subsection{Typing in Detail}

\Cref{fig:decl-typing} gives the typing rules for our declarative system
(the reader is advised to ignore the gray-shaded parts for now). Rule \rul{Var}
extracts the type of the variable from the typing context. Rule \rul{Nat} always
infers integer types. Rule \rul{LamAnn} puts $x$ with type annotation $A$ into
the context, and continues type checking the body $e$. Rule \rul{App} first
infers the type of $e_1$, then the matching judgment $\tprematch A \match A_1
\to A_2$ extracts the domain type $A_1$ and the codomain type $A_2$ from type
$A$. The type $A_3$ of the argument $e_2$ is then compared with $A_1$ using the
consistent subtyping judgment.

\renewcommand{\trto}[1]{\hlmath{\rightsquigarrow{#1}}}
\begin{figure}[t]
  \begin{small}
  \begin{mathpar}
    \framebox{$\tpreinf e : A \trto s$} \\
    \DVar \and \DNat \and \DLamAnnA \and \DApp
  \end{mathpar}

  \begin{mathpar}
    \framebox{$\tprematch A \match A_1 \to A_2$} \\
    \MMC \\ \MMA \and \MMB
  \end{mathpar}

  \end{small}
  \caption{Declarative typing}
  \label{fig:decl-typing}
\end{figure}

\paragraph{Matching} It turns out that matching~\cite{siek2015refined} can be
extended to polymorphic types naturally. In \rul{M-Forall}, a monotype $\tau$ is
guessed to instantiate the universal quantifier $a$. This natural extension is
also inspired by the \textit{application judgment} $\tpreinf A \bullet e \infto
C$ by \citet{dunfield2013complete}, which says that if we apply a term of type
$A$ to an argument $e$, we get something of type $C$. If $A$ is a polymorphic
type, the judgment works by guessing instantiations of polymorphic quantifiers
until it reaches an arrow type. Rule \rul{M-Arr} and \rul{M-Unknown} are the
same as \citet{siek2015refined}.


\renewcommand{\trto}[1]{\rightsquigarrow{#1}}
\subsection{Type-directed Translation}
\label{sec:type:trans}

We give the dynamic semantics of our language by translating it to
\pbc~\cite{ahmed2011blame}. Below we show a subset of the terms in \pbc that are
used in the translation:
\[
  \text{Terms}\quad s ::= x \mid n \mid \blam x A s \mid s~s \mid \cast A B s
\]
A cast $\cast A B {s}$ converts the value of term $s$ from type $A$ to type $B$.
A cast from $A$ to $B$ is permitted only if the types are \textit{compatible},
written $A \pbccons B$, as briefly mentioned in
\Cref{subsec:consistency-subtyping}. The syntax of types in \pbc is the
same as ours.

The translation is given in the gray-shaded parts in \Cref{fig:decl-typing}. The
only interesting case here is to insert explicit casts in the application rule.
Note that there is no need to translate matching or consistent subtyping,
instead we insert the source and target types of a cast directly in the
translated expressions, thanks to the following two lemmas:

\begin{clemma}[Compatibility of Matching]
  \label{lemma:comp-match}
  If $\tprematch A \match A_1 \to A_2$, then $A \pbccons A_1 \to A_2$.
\end{clemma}

\begin{clemma}[Compatibility of Consistent Subtyping]
  \label{lemma:comp-conssub}
  If $\tpreconssub A \tconssub B$, then $A \pbccons B$.
\end{clemma}

In order to show the correctness of the translation, we prove that our
translation always produces well-typed expressions in \pbc. By
\Cref{lemma:comp-match,lemma:comp-conssub}, we have the following theorem:

\begin{ctheorem}[Type Safety]
  \label{lemma:type-safety}
  If $\tpreinf e : A \trto s$, then $\dctx \bypinf s : A$.
\end{ctheorem}

\paragraph{Parametricity} An important semantic property of polymorphic types is
\textit{relational parametricity}~\cite{reynolds1983types}. The parametricity
property says that all instances of a parametrically polymorphic function should
behave \textit{uniformly}. In other words, functions cannot inspect into a type
variable, and act differently for different instances of the type variable. A
classic example is a function with the type $\forall a . a \to a$. The
parametricity property guarantees that a value of this type must be either the
identity function (i.e., $\lambda x . x$) or the undefined function (one which
never returns a value). However, with the addition of the unknown type
$\unknown$, careful measures are to be taken to ensure parametricity. This is
exactly the circumstance that \pbc was designed to address. \citet{amal2017blame}
proved that \pbc satisfies relational parametricity. Based on their result, and
by \Cref{lemma:type-safety}, parametricity is preserved in our system.

\paragraph{Guessed types affect runtime behaviour}

However, the translation does not always produce a unique target expression.
This is because when we guess a monotype $\tau$ in rule \rul{M-Forall} and
\rul{CS-ForallL}, we could have different choices, which inevitably leads to
different types. Unlike (non-gradual) polymorphic type systems
\citep{jones2007practical, dunfield2013complete}, the guessed types affect
runtime behaviour of the translated programs, since they could appear inside the
explicit casts. For example, the following shows two possible translations for
the same source expression $\blam x \unknown {f ~ x}$, where $f$ is
instantiated to $\nat \to \nat$ and $\bool \to \bool$, respectively:
\begin{align*}
  f: \forall a. a \to a &\byinf (\blam x \unknown {f ~ x})
                          : \unknown \to \nat \\
                          &\trto (\blam x \unknown (\cast {\forall a. a \to a} {\nat \to \nat} f) ~
                          (\hlmath{\cast \unknown \nat} x))
  \\
  f: \forall a. a \to a &\byinf (\blam x \unknown {f ~ x})
                          : \unknown \to \bool \\
                          &\trto (\blam x \unknown (\cast {\forall a. a \to a} {\bool \to \bool} f) ~
                          (\hlmath{\cast \unknown \bool} x))
\end{align*}
If we apply $\blam x \unknown {f ~ x}$ to $3$ for example, which should be fine
since the function can take any input, the first translation runs smoothly in
\pbc, while the second one will raise a cast error ($\nat$ cannot be cast to
$\bool$). Similarly, if we apply it to $\truee$, then the second succeeds while
the first fails. The culprit lies in the highlighted parts where any
instantiation of $a$ would be put inside the explicit cast. More generally, any
choice introduces an explicit cast to that type in the translation, which causes
a runtime cast error if the function is applied to a value whose type does not
match the guessed type. Note that this does not compromise the type safety of
the translated expressions, since cast errors are part of the type safety
guarantees.

\paragraph{Coherency}

The ambiguity of translation seems to imply that the
declarative is \textit{incoherent}. Coherence is a desired
property for a semantics. A semantic is coherent if any \textit{valid program}
has exactly one meaning~\cite{Reynolds_coherence}. We argue that the declarative
system is still coherent in the sense that if a program produces a value, this
value is unique. In the above example, whatever the translation might be,
applying $\blam x \unknown {f ~ x}$ to $3$ either results in a cast error, or
produces $3$, and not any other values.

This discrepancy is due to the guessing nature of the \textit{declarative}
system. As far as the declarative system is concerned, both $\nat \to \nat$ and
$\bool \to \bool$ are equally acceptable. But this is not the case at runtime.
The acute reader may have found that the \textit{only} appropriate choice is to
instantiate $f$ to $\unknown \to \unknown$. However, as specified by rule
\rul{M-Forall} in \Cref{fig:decl-typing}, we can only instantiate type variables
to monotypes, but $\unknown$ is \textit{not} a monotype! We will get back to
this issue in \Cref{subsec:algo:discuss} after we present the corresponding
algorithmic system in \Cref{sec:algorithm}.


\subsection{Correctness Criteria}
\label{sec:criteria}

\citet{siek2015refined} present a set of properties that a well-designed gradual
typing calculus must have, which they call refined criteria. Among all the
criteria, those related to the static aspects of gradual typing are well
summarized by \citet{cimini2016gradualizer}. Here we review those criteria and
adapt them to our notation. We have proved in Coq that our type system satisfies
all of these criteria.

\begin{clemma}[Correctness Criteria]\leavevmode
  \begin{itemize}
  \item \textbf{Conservative extension:}
    for all static $\dctx$, $e$, and $A$,
    $\dctx \byhinf e : A $ if and only if $\dctx \byinf e : A$.
  \item \textbf{Monotonicity w.r.t. precision:}
    for all $\dctx, e, e', A$,
    if $\dctx \byinf e : A$,
    and $e' \lessp e$,
    then $\dctx \byinf e' : B$,
    and $B \lessp A$ for some B.
  \item \textbf{Type Preservation of cast insertion:}
    for all $\dctx, e, A$,
    if $\dctx \byinf e : A$,
    then $\dctx \byinf e : A \trto s$,
    and $\dctx \bypinf s : A$ for some $s$.
  \item \textbf{Monotonicity of cast insertion:}
    for all $\dctx, e_1, e_2, e_1', e_2', A$,
    if $\dctx \byinf e_1 : A \trto e_1'$,
    and $\dctx \byinf e_2 : A \trto e_2'$,
    and $e_1 \lessp e_2$,
    then $\dctx \ctxsplit \dctx \bylessp e_1' \lesspp e_2'$.
  \end{itemize}
\end{clemma}

\begin{figure}[t]
  \begin{small}
  \begin{mathpar}
    \framebox{$A \lessp B$}{\quad \text{Type precision}} \\
    \LUnknown \and \LNat \and \LArrow \and \LTVar
    \and \LForall
  \end{mathpar}

  \begin{mathpar}
    \framebox{$e_1 \lessp e_2$}{\quad \text{Term precision}} \\
    \LRefl \and \LAbsAnn \and \LApp
  \end{mathpar}

  \begin{mathpar}
    \framebox{$\dctx_1 \ctxsplit \dctx_2 \bylessp e_1 \lesspp e_2$}
    {\quad \text{Term less precision in \pbc}} \\
    \LVar \and \LNatP \and \LAbsAnnP \and
    \LAppP \and \LCast \and \LCastL \and
    \LCastR
  \end{mathpar}
  \end{small}
  \caption{Less Precision}
  \label{fig:lessp}
\end{figure}


The first criterion states that the gradual type system should be a conservative
extension of the original system (i.e., the Odersky-L{\"a}ufer type system in
our case). In other words, a \textit{static} program that is typeable in the
original type system should remain typeable in the gradual type
system. A static program is one that does not contain any type $\unknown$. It also
ensures that ill-typed programs of the original language remain so in the
gradual type system.

The second criterion states that if a typeable expression loses some type
information, it remains typeable. This criterion depends on the definition of
the precision relation, written $A \lessp B$, which is given in the top of
\Cref{fig:lessp}. The relation intuitively captures a notion of types containing
more or less unknown types ($\unknown$). The precision relation over types lifts
to programs, i.e., $e_1 \lessp e_2$ means that $e_1$ and $e_2$ are the same
program except that $e_2$ has more unknown types.

The first two criteria are fundamental to gradual typing. They explain for
example why these two programs $(\blam x \nat {x + 1})$ and $(\blam x \unknown
{x + 1})$ are typeable, as the former is typeable in the Odersky-L{\"a}ufer type
system and the latter is a less-precise version of it.

The last two criteria relate to the compilation to the cast calculus. The
third criterion is essentially the same as \Cref{lemma:type-safety}, given that
a target expression should always exist, which can be easily seen from
\Cref{fig:decl-typing}. The last criterion ensures that the translation
must be monotonic over the precision relation $\lessp$. (The definition of the
precision relation $\lesspp$ for \pbc is found in the bottom of
\Cref{fig:lessp}.)


%%% Local Variables:
%%% mode: latex
%%% TeX-master: "../paper"
%%% org-ref-default-bibliography: "../paper.bib"
%%% End: