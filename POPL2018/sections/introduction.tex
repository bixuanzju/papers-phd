\section{Introduction}
\label{sec:introduction}

Gradual typing~\cite{siek2006gradual} is an increasingly popular topic in both
programming language practice and theory. On the practical side there is growing
number of programming languages adopting gradual typing. Those languages include
Clojure~\cite{Bonnaire_Sergeant_2016}, Python~\cite{Vitousek_2014},
TypeScript~\cite{typescript}, Hack~\cite{verlaguet2013facebook}, and the
addition of Dynamic to C\#~\cite{Bierman_2010}, to cite a few. On the
theoretical side, recent years have seen a large body of research that defines
the foundations of gradual typing~\cite{garcia2016abstracting,
  cimini2016gradualizer, CiminiPOPL}, explores their use for both functional and
object-oriented programming~\cite{siek2006gradual, siek2007gradual}, as well as its applications
to many other areas~\cite{siek2016key, Ba_ados_Schwerter_2014}.

A key concept in gradual type systems is
\emph{consistency}~\cite{siek2006gradual}. Consistency weakens type equality to allow
for the presence of \emph{unknown} types. In some gradual type systems
with subtyping, consistency is combined with subtyping to give rise to
the notion of \emph{consistent subtyping}~\cite{siek2007gradual}. Consistent
subtyping is employed by gradual type systems to validate type
conversions arising from conventional subtyping. One nice feature of consistent
subtyping is that it is derivable from the more primitive
notions of \emph{consistency} (arising from gradual typing) and
\emph{subtyping}. As \citet{siek2007gradual}
put it this is a good feature to have since it shows that
``\emph{gradual typing and subtyping are orthogonal and can be combined in a principled fashion}''.
Thus consistent subtyping is often used as a guideline for
designing gradual type systems with subtyping. 

Unfortunately, as noted by \citet{garcia2016abstracting}, notions of
consistency and/or consistent subtyping ``\emph{become more difficult to adapt
  as type systems get more complex}''. In particular, for the case of type
systems with subtyping, certain kinds of subtyping do not fit well with the
original definition of consistent subtyping by \citet{siek2007gradual}. One
important case where such mismatch happens is in type systems supporting
implicit (higher-rank) polymorphism~\cite{jones2007practical,dunfield2013complete}. 
It is well-known that polymorphic types
\`a la System F induce a subtyping relation that relates polymorphic types to
their instantiations~\cite{odersky1996putting, mitchell1990polymorphic}. However
\citeauthor{siek2007gradual}'s definition is not adequate for this kind of
subtyping. Moreover the current framework for \emph{Abstracting Gradual
Typing} (AGT)~\cite{garcia2016abstracting} also does not account for polymorphism, with the authors
acknowledging that this is one of the interesting avenues for future work. 

Existing work on gradual type systems with polymorphism does not use
consistent subtyping. The Polymorphic Blame Calculus (\pbc)~\cite{ahmed2011blame} is
an \emph{explicitly} polymorphic calculus with explicit casts, which is
often used as a target language for gradual type systems with
polymorphism. In \pbc a notion of \emph{compatibility} is employed
to validate conversions allowed by casts. Interestingly \emph{\pbc
  allows conversions from polymorphic types to their instantiations}.
For example, it is possible to cast a value with type
$\forall a. a \to a$ into $\nat \to \nat$. Thus an
important remark here is that while \pbc is explicitly polymorphic,
casting and conversions are closer to \emph{implicit}
polymorphism. That is, in a conventional explicitly polymorphic calculus
(such as System F), the primary notion is type equality, where
instantiation is not taken into account. Thus the types
$\forall a. a \to a$ and $\nat \to \nat$ are deemed
\emph{incompatible}. However in \emph{implicitly} polymorphic
calculi~\cite{jones2007practical,dunfield2013complete}
$\forall a. a \to a$ and $\nat \to \nat$ are deemed
\emph{compatible}, since the latter type is an instantiation of the
former.  Therefore \pbc is in a sense a hybrid between implicit and
explicit polymorphism, utilizing type equality (\`a la System F) for
validating applications, and \emph{compatibility} for validating
casts. 

An alternative approach to polymorphism has recently been proposed by
\citet{yuu2017poly}. Like \pbc their calculus is explicitly
polymorphic. However, in that work they employ type consistency to validate
cast conversions, and forbid conversions from
$\forall a. a \to a$ to $\nat \to \nat$. This makes their
casts closer to explicit polymorphism, in contrast to \pbc.
Nonetheless, there is still same flavour of implicit polymorphism in
their calculus when it comes to interactions between dynamically typed
and polymorphically typed code. In their calculus type consistency
allows types such as $(\forall a. a) \to \nat$ to be related to $\unknown \to
\nat$, which can be viewed as a form of subtyping.


\begin{comment}
Instead several works use alternative notions such as
\emph{compatibility} of types~\cite{} or adapted versions of type
consistency~\cite{}. 
For instance both \citet{ahmed2011blame} (in the
Polymorphic Blame calculus) and \citet{yuu2017poly} define
notions of \emph{compatibility}. \jeremy{inaccurate} These two notions of compatibility
are different from each other, and they are not derived from
consistency and subtyping.
\ningning{Mention however they are trying to mixing subtyping and consistency to some degree?}
\ningning{one advantage of their systems is that they do no guessing}
Thus one first criticism to compatibility is 
that it lacks the orthogonality of notions (subtyping and consistency)
afforded by consistent subtyping~\cite{siek2007gradual}  \bruno{Ningning, has this criticism
  been pointed out by someone else before? If so, who? Citing them
  would strenghten our point}.
\ningning{I think no except Siek}
Moreover, the 
proposals for compatibility are different, which makes it unclear 
of which one is more appropriate.
\end{comment}

The first goal of this paper is to study the gradually typed subtyping
and consistent subtyping relations for \emph{implicit polymorphism}. 
To accomplish this, we first show how
to reconcile consistent subtyping with polymorphism by generalizing
the original consistent subtyping definition by
\citeauthor{siek2007gradual}. The new definition of consistent
subtyping can deal with polymorphism, and preserves the orthogonality
between consistency and subtyping. To slightly rephrase \citeauthor{siek2007gradual},
the motto of our paper is that:

\begin{quote}\emph{gradual typing and {\bf polymorphism} are orthogonal and can be combined 
in a principled fashion.}\footnote{Note here that we borrow \citeauthor{siek2007gradual}'s motto
  mostly to talk about the static semantics. As \citet{ahmed2011blame}
work shows there are several non-trivial interactions between
polymorphism and casts at the level of the dynamic semantics.}
\end{quote}
%\footnote{Note that here orthogonality refers to the static
%semantics aspects. As \cite{ahmed} shows there are clearly non-trivial
%interactions between gradual typing and polymorphism in the dynamic semantics?}

\noindent With the insights gained from our work, we argue that
\citeauthor{ahmed2011blame}'s notion of compatibility is too permissive (i.e.
too many programs are allowed to type-check), and that
\citeauthor{yuu2017poly}'s notion of type consistency, while fine for explicit
polymorphism, is too conservative for implicit polymorphism.

\tikzset{
  state/.style={
    rectangle,
    rounded corners,
    minimum height=2em,
    inner sep=2pt,
    text centered,
    align=center
  },
}

\begin{figure}[t]
\begin{tikzpicture}[->,>=stealth']
  \footnotesize

  \node[state,
  draw=black, very thick,
  % text width=2.3cm
  ](cons-and-sub)
 {\begin{tabular}{ll}
    \parbox{2.3cm}{\begin{itemize}
      \item Subtyping
      \item Consistency
      \end{itemize}
    }
    & (Fig.~\ref{fig:decl:subtyping})\\
  \end{tabular}
  };

 \node[state,    	% layout (defined above)
  % text width=3cm, 	% max text width
  below of=cons-and-sub, 	% Position is to the right of
  node distance=2cm, 	% distance
  anchor=center] (conssub) 	% posistion relative to the center of the 'box'
 {
   \parbox{2.2cm}{Consistent\\
     Subtyping (Def.~\ref{def:decl-conssub})}
 };

 \node[state,
  right of=conssub,
  anchor=center,
  node distance=3.8cm,
  ] (syntax-conssub)
 {
   \parbox{1.5cm}{\centering
     $\dctx \bysub A \tconssub B$\\
     (Fig.~\ref{fig:decl:conssub})}
 };

 \node[state,
  right of=syntax-conssub,
  anchor=center,
  node distance=3.2cm,
  ] (decl)
 {
   \parbox{1.2cm}{\centering
     $\dctx \byinf e : A$\\
     (Fig.~\ref{fig:decl-typing})}
 };

 \node[state,
  right of=decl,
  anchor=center,
  node distance=3.5cm,
  ] (algo)
 {
   \parbox{1.7cm}{\centering
     $\tctx \byinf e \infto A \toctxr$\\
     (Fig.~\ref{fig:algo:typing})}
 };

 \node[state,
  below of=conssub,
  anchor=center,
  node distance=2.5cm,
  ] (siek)
 {
   \parbox{3cm}{\citeauthor{siek2007gradual}
     (Def.~\ref{def:old-decl-conssub})}
 };

 \node[state,
  below of=syntax-conssub,
  anchor=center,
  node distance=2.5cm,
 ] (agt)
 {
   \parbox{0.6cm}{AGT}
 };

 \node[state,
  above of=decl,
  anchor=center,
  node distance=2cm,
  ] (ol)
 {
   \parbox{2.5cm}{\centering
     $\dctx \byhinf e : A$\\
     (Fig.~\ref{fig:original-typing})}
 };

 \node[state,
  below of=decl,
  anchor=center,
  node distance=2.5cm,
  ] (pbc)
 {
   \parbox{1.2cm}{$\dctx \bypinf s : A$}
 };

 % draw the paths and and print some Text below/above the graph
 \path
 (cons-and-sub)   edge   node[right]{superimposed}                                      (conssub)
 (conssub)        edge
     node[above]{Thm.~\ref{lemma:properties-conssub}}                                 (syntax-conssub)
 (syntax-conssub) edge                                                                  (conssub)
 (decl)           edge   node[above]{based on}                                          (syntax-conssub)
 ([yshift=0.2cm]decl.east) edge
     node[above]{Thm.~\ref{thm:type_complete}}                               ([yshift=0.2cm]algo.west)
 (algo)           edge   node[below]{Thm.~\ref{thm:type_sound}}                 (decl)
 (conssub)        edge
     node[right, text width=1.5cm]{generalizes (Prop.~\ref{prop:subsumes})}              (siek)
 (agt)            edge
     node[left, text width=1.5cm]{verified on simple types (Prop.~\ref{lemma:coincide-agt})}         (syntax-conssub)
 (agt)            edge[bend right]
     node[right, text width=2.5cm]{verified on $\tope$ (Prop.~\ref{prop:agt-top})}   (syntax-conssub)
 (decl)           edge   node[right]{gradualizes}                                        (ol)
 (decl)           edge
     node[right, text width=2cm]{type-directed translation (Thm.~\ref{lemma:type-safety})} (pbc)
     ;
 % (QUERY) 	edge[bend left=20]  node[anchor=south,above]{$SC_n=0$}
 %                                    node[anchor=north,below]{$RN_{16}$} (ACK)
 % (QUERY)     	edge[bend right=20] node[anchor=south,above]{$SC_n\neq 0$} (QUERYREP)
 % (ACK)       	edge                                                     (EPC)
 % (EPC)       	edge[bend left]                                          (QUERYREP)
 % (QUERYREP)  	edge[loop below]    node[anchor=north,below]{$SC_n\neq 0$} (QUERYREP)
 % (QUERYREP)  	edge                node[anchor=left,right]{$SC_n = 0$} (ACK);

\end{tikzpicture}
\caption{Some key results in the paper.}
\label{fig:key-results}
\end{figure}

As a step towards an algorithmic version of subtyping, we present a
syntax-directed version of consistent subtyping that is sound and
complete with respect to our formal definition of consistent
subtyping. The syntax-directed version of consistent subtyping is
remarkably simple and well-behaved, and is able to get rid of the
ad-hoc \textit{restriction} operator~\cite{siek2007gradual}. 
Moreover, to further illustrate the generality of our consistent
subtyping definition, we show that it can also account for top types,
which cannot be dealt with by \citeauthor{siek2007gradual}'s
definition either.

%From the technical point of
%view the main innovation is a novel definition of consistent subtyping
%that generalizes Siek and Taha's, and preserves the orthogonality
%between subtyping and consistency. 

\begin{comment}
The new definition of consistent subtyping provides novel insights
with respect to previous polymorphic gradual type systems, which did
not not employ consistent subtyping. For instance both Ahmed et
al. (in the Polymorphic Blame calculus) (2011) and Igarashi et
al. (2017) define notions of \emph{compatibility} that are used
instead of consistent subtyping. We argue that Ahmed et al.'s notion
of compatibility is too permissive (i.e. too many programs are allowed
to type-check), and that Igarashi et al.'s notion of compatibility is
too conservative (i.e. programs that should type check are rejected).
\end{comment}

\begin{comment}
To further validate our revised notion of consistent subtyping, we
show that it coincides with the notion of consistent subtyping for an
extension of Garcia et al.'s \emph{Abstracting Gradual Typing} (AGT)
(2016) with polymorphism.
\end{comment}

The second goal of this paper is to present a gradually typed calculus for
implicit (higher-rank) polymorphism that uses our new notion of consistent
subtyping. Implicit polymorphism is a hallmark of functional programming, and
modern functional languages (such as Haskell) employ sophisticated
type-inference algorithms that, aided by type annotations, can deal with
higher-rank polymorphism. Therefore as a step towards gradualizing such type
systems, this paper develops both declarative and algorithmic versions for a
type system with implicit higher-rank polymorphism. The algorithmic version
employs techniques developed by \citet{dunfield2013complete} to deal with
instantiation. We prove that the new calculus satisfies all of the refined
criteria for gradual typing proposed by \citet{siek2015refined}.

In summary, the contributions of this paper are:

\begin{itemize}
\item We define a framework for consistent subtyping in gradual type systems
  with:
  \begin{itemize}
  \item a new definition of consistent subtyping that subsumes and generalizes that
    of \citet{siek2007gradual}. This new definition can deal with
    polymorphism and top types.
  %Unlike previous definitions of
  %\textit{compatibility} in \textit{Polymorphic Blame Calculus}
  %\citep{ahmed2011blame}, we show that the orthogonality between
  %subtyping and consistency can be preserved.
  %  \ningning{there are differences between Definiton~\ref{def:decl-conssub},
  %    \Cref{fig:decl:conssub} and \Cref{fig:algo:subtype}. Here I
  %  try to distinguish the first two. I am not sure the correct terminology for
  %  it.}
  \item a syntax-directed version of consistent subtyping that is
    sound and complete with respect to our definition of consistent
    subtyping, but still guesses polymorphic instantiations.
  \end{itemize}
\item Based on consistent subtyping, we present a declarative gradual type
  system with implicit higher-rank polymorphism. We prove that our calculus
  satisfies the static aspects of the refined criteria for gradual
  typing~\cite{siek2015refined}, and is type-safe by a type-directed
  translation to the Polymorphic Blame Calculus~\cite{ahmed2011blame}, and thus
  hereditarily preserves parametricity~\cite{amal2017blame}.
\item We show that gradual typing is compatible with type inference involving
  higher-rank polymorphism, by presenting an algorithm for implementing the
  declarative system based on the approach developed by
  \citet{dunfield2013complete}. The algorithm is also illustrative of a
  situation where the simple approach of \textit{ignoring dynamic types during
    unification}~\cite{siek2008gradual} actually works.
\item All of the metatheory of this paper, except some manual proofs for the
  algorithmic type system, has been mechanically formalized using the Coq proof
  assistant.
\end{itemize}

\Cref{fig:key-results} depicts some key results in the paper. The reader is
advised to keep this figure in mind when reading the rest of the paper.

% The rest of the paper is organized as follows. In \Cref{sec:background}
% we review a simple gradually typed language with objects, to introduce the
% concept of consistent subtyping. We also briefly review our original language:
% the Odersky-L{\"a}ufer type system for higher-rank types. We motivate and
% justify our new definition of consistent subtyping in
% \Cref{sec:exploration}. In \Cref{sec:type-system} we present the
% declarative type system for our gradually typed calculus with implicit
% higher-rank polymorphism. In \Cref{sec:algorithm} we give a simple
% algorithm to implement the declarative system, and show it is sound and complete
% with respect to the declarative system in \Cref{sec:sound:complete}. We
% then show a simple extension and discuss some issues in
% \Cref{sec:discussion}. After discussing related work in
% \Cref{sec:related}, we conclude in \Cref{sec:conclusion}.




%%% Local Variables:
%%% mode: latex
%%% TeX-master: "../paper"
%%% org-ref-default-bibliography: "../paper.bib"
%%% End:
