
\section{Related Work}
\label{sec:related}

Along the way we discussed some of the most relevant work to motivate,
compare and
promote our gradual typing design. In what follows, we briefly discuss related
work on gradual typing.

\paragraph{Gradual Typing}

The seminal paper by \citet{siek2006gradual} is the first to propose gradual
typing, which enables programmers to mix static and dynamic typing in a program
by providing a mechanism to control which parts of a program are statically
checked. The original proposal extends the simply typed lambda calculus by
introducing the unknown type $\unknown$ and replacing type equality with type
consistency. Casts are introduced to mediate between statically and dynamically
typed code. Later \citet{siek2007gradual} incorporated gradual typing into a
simple object oriented language, and showed that subtyping and consistency are
orthogonal -- an insight that partly inspired our work. We show that subtyping
and consistency are orthogonal in a much richer type system with higher-rank
polymorphism. In light of the ever-growing popularity of gradual typing, and its
somewhat mucky theoretical foundations, \citet{siek2015refined} felt the urge to
have a complete formal characterization of what it means to be gradually typed.
They proposed a set of criteria that provides important guidelines for designers
of gradually typed languages. \citet{cimini2016gradualizer} introduced the
\textit{Gradualizer}, a general algorithmic methodology for generating gradual
type systems from static type systems. Later they extend it so that the
Gradualizer can generate dynamic semantics as well~\cite{CiminiPOPL}.
\citet{garcia2016abstracting} introduced the AGT approach based on abstract
interpretation. As we discussed, none of these approaches can instruct us how to
define consistent subtyping for polymorphic types.


\paragraph{Polymorphic Gradual Type Systems}

There is not much work on integrating gradual typing with parametric
polymorphism. Early works include the dynamic type of
\citet{abadi1995dynamic} and
the \textit{Sage} language of \citet{gronski2006sage}. None of them carefully
studied parametricity, nor the interactions between statically and dynamically
typed code. \citet{ahmed2011blame} proposed the Polymorphic Blame Calculus that
extends the blame calculus~\cite{Wadler_2009} to incorporate polymorphism. The
key novelty of their work is to use dynamic sealing to enforce parametricity. As
such, they end up with a sophisticated dynamic semantics. Compared to our
work, our type system can catch more errors earlier. For example, the following
is rejected by the type system:
\[
  (\blam x \unknown x + 1) : \forall a. a \to a \rightsquigarrow \cast {\unknown \to \nat}
  {\forall a. a \to a} (\blam x \unknown x + 1)
\]
while the type system of \pbc would accept the translation, though at runtime,
such program would result in a cast error as this violates the parametricity
property. This does not imply, in any regard that \pbc is not well-designed;
there are circumstances where runtime checks are \textit{needed} to ensure
parametricity. We emphasize that it is the combination of our powerful type
system together with the powerful dynamic semantics of \pbc that makes it
possible to have implicit higher-rank polymorphism in a gradually typed setting,
without compromising parametricity. \citet{yuu2017poly} also studied integrating
gradual typing with parametric polymorphism. They proposed System F$_G$, a
gradually typed extension of System F, and System F$_C$, a new polymorphic blame
calculus. As has been discussed extensively, their definition of type consistency
does not apply to our setting. All of these approaches mix consistency with
subtyping to some extent, which we argue should be orthogonal. On a side note,
it seems our calculus can also be safely translated to System F$_C$. However we
do not understand all the tradeoffs involved in the choice between \pbc and
System F$_C$ as a target.



\paragraph{Gradual Type Inference}

\citet{siek2008gradual} studied unification-based type inference for gradual
typing, where they show why three straightforward approaches fail to meet their
design goals. As mentioned before, our work shows that one of the approaches,
that is, ``ignoring dynamic types during unification'' actually works in our
setting. We also distinguish between the different roles of type variables and
existential variables when it comes to parametricity. This distinction explains
\citeauthor{siek2008gradual}'s observation about assigning the unknown type.
\citet{garcia2015principal} presented a new approach to gradual type inference,
where they proposed the distinction between static and gradual type parameters,
which is similar to our distinction of type variables and existential variables.
Another similarity is that both systems only infer static types, and share the
same view that gradual types are introduced only via program annotations.
Although our system and the existing work all involve gradual types and
inference, they have essential differences that make them complementary. First,
both \citet{siek2008gradual} and \citet{garcia2015principal} do not support
first-class polymorphism: each type is concrete. Second, like
\citet{siek2006gradual}, in our system, missing annotations are syntactic sugar
for $\unknown$ annotations. Both \citet{siek2008gradual} and
\citet{garcia2015principal} begin with an implicitly typed language and extend
it with gradual typing. As future work, we are interested to increase the amount
of type inference done in our system.


\paragraph{Higher-rank Polymorphism}

\citet{odersky1996putting} introduced a type system for higher-rank types. Based
on that, \citet{jones2007practical} developed an approach for type checking
higher-rank predicative polymorphism. \citet{dunfield2013complete} proposed a
bidirectional account of higher-rank polymorphism, and an algorithm for
implementing the declarative system, which serves as a sole inspiration for our
algorithmic system. The key difference, however, is the use of matching
judgment, which we think is a simplification over their application judgment,
since the former is independent of typing.

%%% Local Variables:
%%% mode: latex
%%% TeX-master: "../paper"
%%% org-ref-default-bibliography: "../paper.bib"
%%% End:
