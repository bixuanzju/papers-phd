%% For double-blind review submission
\documentclass[acmsmall,10pt,review,anonymous]{acmart}\settopmatter{printfolios=true}
%% For single-blind review submission
%\documentclass[acmsmall,10pt,review]{acmart}\settopmatter{printfolios=true}
%% For final camera-ready submission
%\documentclass[acmsmall10pt,]{acmart}\settopmatter{}

%% Note: Authors migrating a paper from PACMPL format to traditional
%% SIGPLAN proceedings format should change 'acmsmall' to
%% 'sigplan'.


%% Some recommended packages.
\usepackage{booktabs}   %% For formal tables:
                        %% http://ctan.org/pkg/booktabs
\usepackage{subcaption} %% For complex figures with subfigures/subcaptions
                        %% http://ctan.org/pkg/subcaption

% Hyper links
\usepackage{url}
\usepackage{
  nameref,%\nameref
  hyperref,%\autoref
  % n.b. \Autoref is defined by thmtools
  cleveref,% \cref
  % n.b. cleveref after! hyperref
}
\hypersetup{
   colorlinks,
   citecolor=black,
   filecolor=black,
   linkcolor=black,
   urlcolor=black
}

% AMS packages
\usepackage{amsmath}
\usepackage{amssymb}
\usepackage{amsthm, thmtools, thm-restate}

\declaretheorem{utheorem}

\declaretheorem[name=Proposition]{mprop}
\declaretheorem[name=Definition]{mdef}

\declaretheorem[name=Lemma,
  refname={lemma,lemmas},
  Refname={Lemma,Lemmas}]{mlemma}

\declaretheorem[name=$\mathcal{L}$emma,
  numberwithin=section,
  refname={lemma,lemmas},
  Refname={Lemma,Lemmas}]{clemma}

\declaretheorem[name=Theorem,
  refname={theorem,theorems},
  Refname={Theorem,Theorems}]{mtheorem}

\declaretheorem[name=$\mathcal{T}$heorem,
  numberwithin=section,
  refname={theorem,theorems},
  Refname={Theorem,Theorems}]{ctheorem}


\declaretheorem[name=Observation]{observation}


\usepackage{mathtools}
\usepackage{mdwlist}
\usepackage{pifont}


% Miscellaneous
\usepackage{paralist}
\usepackage{graphicx}
\usepackage{epstopdf}
\usepackage{float}
\usepackage{longtable}
\usepackage{multirow}


% Revision tools
\usepackage{xspace}
\usepackage{comment}
\newcommand\mynote[3]{\textcolor{#2}{#1: #3}}
\newcommand\bruno[1]{\mynote{Bruno}{red}{#1}}
\newcommand\ningning[1]{\mynote{Ningning}{orange}{#1}}
\newcommand\jeremy[1]{\mynote{Jeremy}{blue}{#1}}

\newcommand{\hl}[2][gray!40]{\colorbox{#1}{#2}}
\newcommand{\hlmath}[2][gray!40]{%
  \colorbox{#1}{$\displaystyle#2$}}

% Graphs
\usepackage{tikz}
\usetikzlibrary{matrix}
\usetikzlibrary{arrows,automata}
\usetikzlibrary{positioning}


\input{paper_utility.tex}


% ------------------------------------------------------
% ORIGINAL TYPING
% ------------------------------------------------------

\newcommand*{\OVar}{\inferrule{
            x : A \in \tctx
            }{
            \tctx \byoinf x \infto A
            }\rname{Var}}

\newcommand*{\ONat}{\inferrule{
            }{
            \tctx \byoinf n \infto \nat
            }\rname{Nat}}

\newcommand*{\OLamAnnA}{\inferrule{
            \tctx, x: A \byoinf e \infto B
            }{
            \tctx \byoinf (\blam x A e) \infto A \to B
            }\rname{LamAnn-I}}

\newcommand*{\OLamAnnB}{\inferrule{
            \tctx, x: A \byoinf e \chkby B
            }{
            \tctx \byoinf (\blam x A e) \chkby A \to B
            }\rname{LamAnn-C}}

\newcommand*{\OApp}{\inferrule{
            \tctx \byoinf e_1 \infto A_1 \to A_2
         \\ \tctx \byochk e_2 \chkby A_1
            }{
            \tctx \byoinf e_1 ~ e_2 \infto A_2
            }\rname{App}}

\newcommand*{\OLamB}{\inferrule{
            \tctx, x: A \byochk e \chkby B
            }{
            \tctx \byochk \erlam x e \chkby A \to B
            }\rname{Lam-C}}

\newcommand*{\OSub}{\inferrule{
            \tctx \byoinf e \infto A
            }{
            \tctx \byochk e \chkby A
            }\rname{Sub}}

% ------------------------------------------------------
% TYPING
% ------------------------------------------------------

\newcommand*{\Var}{\inferrule{
            x : A \in \tctx
            }{
            \tctx \byinf x \infto A
            \trto{x}
            }\rname{Var}}

\newcommand*{\Nat}{\inferrule{
            }{
            \tctx \byinf n \infto \nat
            \trto{n}
            }\rname{Nat}}

\newcommand*{\LamAnnA}{\inferrule{
            \tctx, x: A \byinf e \infto B \trto {e'}
            }{
            \tctx \byinf (\blam x A e) \infto A \to B
            \trto{\blam x A {e'}}
            }\rname{LamAnn-I}}

\newcommand*{\LamAnnB}{\inferrule{
            C \match A_1 \to B
         \\ A = A_1 \glb A_2
         \\ \tctx, x: A \byinf e \chkby B \trto {e'}
            }{
            \tctx \bychk (\blam x {A_2} e) \chkby C
            \trto{\cast{A \to B}{C}(\blam x {A} {e'})}
            }\rname{LamAnn-C}}

\newcommand*{\App}{\inferrule{
            \tctx \byinf e_1 \infto A \trto{e_1'}
         \\ A \match A_1 \to A_2
         \\ \tctx \bychk e_2 \chkby A_1 \trto{e_2'}
            }{
            \tctx \byinf e_1 ~ e_2 \infto A_2
            \trto {(\cast{A}{A_1 \to A_2} e_1') ~ e_2'}
            }\rname{App}}

\newcommand*{\LamB}{\inferrule{
            C \match A \to B
         \\ \tctx, x: A \bychk e \chkby B \trto{e_1'}
            }{
            \tctx \bychk \erlam x e \chkby C
            \trto{\cast{A \to B}{\erlam x e}}
            }\rname{Lam-C}}

\newcommand*{\Sub}{\inferrule{
            e \neq (\blam x C e')
         \\ \tctx \byinf e \infto B \trto{e'}
         \\ B \sim A
            }{
            \tctx \bychk e \chkby A
            \trto{\cast B A e'}
            }\rname{Sub}}

% ------------------------------------------------------
% CAST CALCULUS
% ------------------------------------------------------

\newcommand*{\CaVar}{\inferrule{
            x : A \in \tctx
            }{
            \tpreinf x : A
            }\rname{C-Var}}

\newcommand*{\CaNat}{\inferrule{
            }{
            \tctx \byinf n : \nat
            }\rname{Nat}}

\newcommand*{\CaLam}{\inferrule{
            \tctx, x: A \byinf e : B
            }{
            \tpreinf \blam x A e : A \to B
            }\rname{C-Lam}}

\newcommand*{\CaApp}{\inferrule{
            \tpreinf e_1 : A \to B
         \\ \tpreinf e_2 : A
            }{
            \tpreinf e_1 ~ e_2 : B
            }\rname{C-App}}

\newcommand*{\CaCast}{\inferrule{
            \tpreinf e : A
         \\ B \sim A
            }{
            \tpreinf \cast A B e : B
            }\rname{C-Cast}}

\newcommand*{\CaBlame}{\inferrule{
            }{
            \tpreinf \blame A l : A
            }\rname{C-Blame}}

% ------------------------------------------------------
% Matching
% ------------------------------------------------------

\newcommand*{\MA}{\inferrule{}{
            (A_1 \to A_2) \match (A_1 \to A_2)
            }}

\newcommand*{\MB}{\inferrule{}{
            \unknown \match \unknown \to \unknown
            }}

\newcommand*{\MMA}{\inferrule{ }{
            \tprematch (A_1 \to A_2) \match (A_1 \to A_2)
            }\rname{M-Arr}}

\newcommand*{\MMB}{\inferrule{ }{
            \tprematch \unknown \match \unknown \to \unknown
            }\rname{M-Unknown}}

\newcommand*{\MMC}{\inferrule{
            \tprewf \tau
         \\ \tprematch A \subst a \tau \match A_1 \to A_2
            }{
            \tprematch \forall a. A \match A_1 \to A_2
            }\rname{M-Forall}}


% ------------------------------------------------------
% Matching (Algorithmic)
% ------------------------------------------------------

\newcommand*{\AMMA}{\inferrule{ }{
            \Gamma \vdash (A_1 \to A_2) \match (A_1 \to A_2) \toctxo
            }\rname{AM-Arr}}

\newcommand*{\AMMB}{\inferrule{ }{
            \Gamma \vdash \unknown \match \unknown \to \unknown \toctxo
            }\rname{AM-Unknown}}

\newcommand*{\AMMC}{\inferrule{ \tctx, \genA \vdash A \subst a \genA \match A_1 \to A_2 \toctxr
            }{
            \Gamma \vdash \forall a. A \match A_1 \to A_2 \toctxr
            }\rname{AM-Forall}}

\newcommand*{\AMMD}{\inferrule{ }{
            \tctx[\genC] \vdash \genC \match \genA \to \genB \dashv \tctx[\genA, \genB, \genC = \genA \to \genB]
            }\rname{AM-Var}}



% ------------------------------------------------------
% Instantiation
% ------------------------------------------------------

\newcommand*{\InstLSolve}{\inferrule{ \tctx \bywf \tau}
            {\tctx, \genA, \tctx' \vdash \genA \unif \tau \dashv \tctx, \genA = \tau, \tctx'
            }\rname{InstLSolve}}

\newcommand*{\InstLSolveU}{\inferrule{ }
            {\tctx[\genA] \vdash \genA \unif \unknown \dashv \tctx[\genA]
            }\rname{InstLSolveU}}

\newcommand*{\InstLReach}{\inferrule{ }
            {\tctx[\genA][\genB] \vdash \genA \unif \genB \dashv \tctx[\genA][\genB = \genA]
            }\rname{InstLReach}}

\newcommand*{\InstLArr}{\inferrule{ \tctx[\genA_2, \genA_1, \genA = \genA_1 \to \genA_2] \vdash A_1 \unif \genA_1 \toctx \\
             \ctxl \vdash \genA_2 \unif \ctxsubst{\ctxl}{A_2} \toctxr
            }
            {\tctx[\genA] \vdash \genA \unif A_1 \to A_2 \toctxr
            }\rname{InstLArr}}

\newcommand*{\InstLAllR}{\inferrule{ \tctx[\genA], b \vdash \genA \unif B \toctxr, b, \Delta'
            }
            {\tctx[\genA] \vdash \genA \unif \forall b . B \toctxr
            }\rname{InstLAllR}}


\newcommand*{\InstRSolve}{\inferrule{ \tctx \bywf \tau}
            {\tctx, \genA, \tctx' \vdash \tau \unif \genA \dashv \tctx, \genA = \tau, \tctx'
            }\rname{InstRSolve}}

\newcommand*{\InstRSolveU}{\inferrule{ }
            {\tctx[\genA] \vdash \unknown \unif \genA \dashv \tctx[\genA]
            }\rname{InstRSolveU}}

\newcommand*{\InstRReach}{\inferrule{ }
            {\tctx[\genA][\genB] \vdash \genB \unif \genA \dashv \tctx[\genA][\genB = \genA]
            }\rname{InstRReach}}

\newcommand*{\InstRArr}{\inferrule{ \tctx[\genA_2, \genA_1, \genA = \genA_1 \to \genA_2] \vdash \genA_1 \unif A_1 \toctx \\
             \ctxl \vdash \ctxsubst{\ctxl}{A_2}  \unif \genA_2  \toctxr
            }
            {\tctx[\genA] \vdash A_1 \to A_2  \unif \genA \toctxr
            }\rname{InstRArr}}

\newcommand*{\InstRAllL}{\inferrule{ \tctx[\genA], \genB \vdash B \subst b \genB \unif \genA \toctxr
            }
            {\tctx[\genA] \vdash \forall b . B \unif \genA  \toctxr
            }\rname{InstRAllL}}



% ------------------------------------------------------
% Consistency
% ------------------------------------------------------

\newcommand*{\CA}{\inferrule{}{
            \gcastable \sim \unknown
            }}

\newcommand*{\CB}{\inferrule{}{
            \unknown \sim \gcastable
            }}

\newcommand*{\CC}{\inferrule{
            A_1 \sim B_1
         \\ A_2 \sim B_2
            }{
            A_1 \to A_2 \sim B_1 \to B_2
            }}

\newcommand*{\CD}{\inferrule{}{
            A \sim A
            }}

\newcommand*{\CE}{\inferrule{
            A \sim B
            }{
            \forall a. A \sim \forall a. B
            }}

% ------------------------------------------------------
% GREATEST LOWER BOUND
% ------------------------------------------------------

\newcommand*{\GA}{\inferrule{}{
            A \glb A = A
            }}

\newcommand*{\GB}{\inferrule{}{
            A \glb \unknown = \unknown \glb A = A
            }}

\newcommand*{\GC}{\inferrule{}{
            (A_1 \to A_2) \glb (B_1 \to B_2) = (A_1 \glb B_1) \to (A_2 \glb B_2)
            }}

\newcommand*{\GGA}{\inferrule{
            }{
            \tpreglb A \glb A = A
            }}

\newcommand*{\GGB}{\inferrule{
            }{
            \tpreglb A \glb \unknown = A
            }}

\newcommand*{\GGF}{\inferrule{
            A ~ is ~ G
            }{
            \tpreglb \unknown \glb A = A
            }}

\newcommand*{\GGG}{\inferrule{
            A ~ isnot~ G
            }{
            \tpreglb \unknown \glb A = \unknown
            }}

\newcommand*{\GGC}{\inferrule{
            \tpreglb[,a] A \glb B = C
            }{
            \tpreglb A \glb \forall a. B = C
            }}

\newcommand*{\GGD}{\inferrule{
         \\ \tpreglb[, a] A \glb B = C
            }{
            \tpreglb \forall a. A \glb B = \forall a. C
            }}

\newcommand*{\GGE}{\inferrule{
            \tpreglb A_1 \glb B_1 = C_1
         \\ \tpreglb A_2 \glb B_2 = C_2
            }{
            \tpreglb A_1 \to A_2 \glb B_1 \to B_2 = C_1 \to C_2
            }}

% ------------------------------------------------------
% MASK
% ------------------------------------------------------

\newcommand*{\MSUnknownL}{\inferrule{
            }{
            \tctx \bymask \mask \unknown B = \unknown
            }\rname{Mask-UnknownL}}

\newcommand*{\MSUnknownR}{\inferrule{
            }{
            \tctx \bymask \mask A \unknown = \unknown
            }\rname{Mask-UnknownR}}

\newcommand*{\MSForallL}{\inferrule{
            \tctx, a \bymask \mask A B = C
            }{
            \tctx \bymask \mask {\forall a. A} B  = \forall a. C
            }\rname{Mask-ForallL}}

\newcommand*{\MSForallR}{\inferrule{
            \tctx, b \bymask \mask A B = C
            }{
            \tctx \bymask \mask A {\forall b. B}  = C
            }\rname{Mask-ForallR}}

\newcommand*{\MSArrow}{\inferrule{
            \tctx \bymask \mask {A_1} {B_1} = {C_1}
         \\ \tctx \bymask \mask {A_2} {B_2} = {C_2}
            }{
            \tctx \bymask \mask {A_1 \to A_2} {B_1 \to B_2} = C_1 \to C_2
            }\rname{Mask-Arrow}}

\newcommand*{\MSNat}{\inferrule{
            }{
            \tctx \bymask \mask \nat \nat = \nat
            }\rname{Mask-Int}}

% ------------------------------------------------------
% CONSISTENT SUBTYPING
% ------------------------------------------------------

\newcommand*{\CSForallR}{\inferrule{
            \tpresub[,a] A \tconssub B
            }{
            \tpresub A \tconssub \forall a. B
            }\rname{CS-ForallR}}

\newcommand*{\CSForallL}{\inferrule{
            \dctx \bywf \tau
         \\ \tpreconssub A \subst a \tau \tconssub B
            }{
            \tpresub \forall a. A \tconssub B
            }\rname{CS-ForallL}}

\newcommand*{\CSFun}{\inferrule{
            \tpreconssub B_1 \tconssub A_1
         \\ \tpreconssub A_2 \tconssub B_2
            }{
            \tpreconssub A_1 \to A_2 \tconssub B_1 \to B_2
            }\rname{     CS-Fun}}

\newcommand*{\CSTVar}{\inferrule{
            a \in \dctx
            }{
            \tpreconssub a \tconssub a
            }\rname{CS-TVar}}

\newcommand*{\CSInt}{\inferrule{
            }{
            \tpreconssub \nat \tconssub \nat
            }\rname{CS-Int}}

\newcommand*{\CSUnknownL}{\inferrule{
            }{
            \tpreconssub \unknown \tconssub \gcastable
            }\rname{CS-UnknownL}}

\newcommand*{\CSUnknownR}{\inferrule{
            }{
            \tpreconssub \gcastable \tconssub \unknown
            }\rname{CS-UnknownR}}

\newcommand*{\CSSVar}{\inferrule{
            }{
            \tpreconssub \static \tconssub \static
            }\rname{CS-SVar}}

\newcommand*{\CSGVar}{\inferrule{
            }{
            \tpreconssub \gradual \tconssub \gradual
            }\rname{CS-GVar}}

% ------------------------------------------------------
% CONSISTENT SUBTYPING (Algorithmic)
% ------------------------------------------------------

\newcommand*{\ACSForallR}{\inferrule{ \tctx, a \vdash A \tconssub B \toctxr, a, \ctxl
            }{
            \Gamma \vdash A \tconssub \forall a. B \toctxr
            }\rname{ACS-ForallR}}

\newcommand*{\ACSForallL}{\inferrule{ \tctx, \genA \vdash A \subst a \genA \tconssub B \toctxr
            }{
            \Gamma \vdash \forall a. A \tconssub B \toctxr
            }\rname{ACS-ForallL}}

\newcommand*{\ACSFun}{\inferrule{\Gamma \vdash B_1 \tconssub A_1 \toctx \\
             \ctxl \vdash \ctxsubst{\ctxl}{A_2} \tconssub \ctxsubst{\ctxl}{B_2} \toctxr
            }{
            \Gamma \vdash A_1 \to A_2 \tconssub B_1 \to B_2 \toctxr
            }\rname{ACS-Fun}}

\newcommand*{\ACSTVar}{\inferrule{
            }{
            \tctx[a] \vdash a \tconssub a \dashv \tctx[a]
            }\rname{ACS-TVar}}

\newcommand*{\ACSExVar}{\inferrule{
            }{
            \tctx[\genA] \vdash \genA \tconssub \genA \dashv \tctx[\genA]
            }\rname{ACS-ExVar}}


\newcommand*{\ACSInt}{\inferrule{
            }{
            \Gamma \vdash \nat \tconssub \nat \toctxo
            }\rname{ACS-Int}}

\newcommand*{\ACSUnknownL}{\inferrule{
            }{
            \Gamma \vdash \unknown \tconssub \gcastable \toctxo
            }\rname{ACS-UnknownL}}

\newcommand*{\ACSUnknownR}{\inferrule{
            }{
            \Gamma \vdash \gcastable \tconssub \unknown \toctxo
            }\rname{ACS-UnknownR}}

\newcommand*{\AInstantiateL}{\inferrule{ \genA \notin \mathit{fv}(A) \\
             \tctx[\genA] \vdash \genA \unif A \toctxr
            }{
            \tctx[\genA] \vdash \genA \tconssub A \toctxr
            }\rname{ACS-InstL}}

\newcommand*{\AInstantiateR}{\inferrule{ \genA \notin \mathit{fv}(A) \\
             \tctx[\genA] \vdash  A \unif \genA \toctxr
            }{
            \tctx[\genA] \vdash A \tconssub \genA  \toctxr
            }\rname{ACS-InstR}}

\newcommand*{\ACSSVar}{\inferrule{
            }{
            \Gamma \vdash \static \tconssub \static \toctxo
            }\rname{ACS-SVar}}

\newcommand*{\ACSGVar}{\inferrule{
            }{
            \Gamma \vdash \gradual \tconssub \gradual \toctxo
            }\rname{ACS-GVar}}


% ------------------------------------------------------
% LESS PRECISION
% ------------------------------------------------------

\newcommand*{\LUnknown}{\inferrule{
            }{
            \unknown \lessp A
            }\rname{L-Unknown}}

\newcommand*{\LNat}{\inferrule{
            }{
            \nat \lessp \nat
            }\rname{L-Nat}}

\newcommand*{\LArrow}{\inferrule{
            A_1 \lessp B_1
         \\ A_2 \lessp B_2
            }{
            A_1 \to A_2 \lessp B_1 \to B_2
            }\rname{L-Arrow}}

\newcommand*{\LTVar}{\inferrule{
            }{
            a \lessp a
            }\rname{L-TVar}}

\newcommand*{\LForall}{\inferrule{
            A \lessp B
            }{
            \forall a. A \lessp \forall a. B
            }\rname{L-Forall}}

% Term level

\newcommand*{\LRefl}{\inferrule{
            }{
            e \lessp e
            }\rname{L-Refl}}

\newcommand*{\LAbsAnn}{\inferrule{
            A_1 \lessp A_2
         \\ e_1 \lessp e_2
            }{
            \blam x {A_1} {e_1} \lessp \blam x {A_2} {e_2}
            }\rname{L-LamAnn}}

\newcommand*{\LApp}{\inferrule{
            e_1 \lessp e_3
         \\ e_2 \lessp e_4
            }{
            e_1 ~ e_2 \lessp e_3 ~ e_4
            }\rname{L-App}}

% PBC Term level

\newcommand*{\LVar}{\inferrule{
            x : A \in \dctx_1
         \\ x : B \in  \dctx_2
            }{
            \dctx_1 \ctxsplit \dctx_2 \bylessp x \lesspp x
            }\rname{L-Var}}

\newcommand*{\LNatP}{\inferrule{
            }{
            \dctx_1 \ctxsplit \dctx_2 \bylessp n \lesspp n
            }\rname{L-Nat}}

\newcommand*{\LAbsAnnP}{\inferrule{
            A_1 \lessp A_2
         \\ \dctx_1, x: A_1 \ctxsplit \dctx_2, x: A_2 \bylessp e_1 \lesspp e_2
            }{
            \dctx_1 \ctxsplit \dctx_2 \bylessp \blam x {A_1} {e_1} \lesspp \blam x {A_2} {e_2}
            }\rname{L-LamAnn}}

\newcommand*{\LAppP}{\inferrule{
            \dctx_1 \ctxsplit \dctx_2 \bylessp e_1 \lesspp e_3
         \\ \dctx_1 \ctxsplit \dctx_2 \bylessp e_2 \lesspp e_4
            }{
            \dctx_1 \ctxsplit \dctx_2 \bylessp e_1 ~ e_2 \lesspp e_3 ~ e_4
            }\rname{L-App}}

\newcommand*{\LCast}{\inferrule{
            A_1 \lessp B_1
         \\ A_2 \lessp B_2
         \\ \dctx_1 \ctxsplit \dctx_2 \bylessp e_1 \lesspp e_2
            }{
            \dctx_1 \ctxsplit \dctx_2 \bylessp \cast {A_1} {A_2} {e_1} \lesspp \cast{B_1} {B_2} {e_2}
            }\rname{L-Cast}}

\newcommand*{\LCastL}{\inferrule{
            \dctx_1 \ctxsplit \dctx_2 \bylessp e_1 \lesspp e_2
         \\ \dctx_2 \bypinf e_2 : B
         \\ A_1 \lessp B
         \\ A_2 \lessp B
            }{
            \dctx_1 \ctxsplit \dctx_2 \bylessp \cast {A_1} {A_2} {e_1} \lesspp {e_2}
            }\rname{L-CastL}}

\newcommand*{\LCastR}{\inferrule{
            \dctx_1 \ctxsplit \dctx_2 \bylessp e_1 \lesspp e_2
         \\ \dctx_1 \bypinf e_1 : A
         \\ A \lessp B_1
         \\ A \lessp B_2
            }{
            \dctx_1 \ctxsplit \dctx_2 \bylessp e_1 \lesspp \cast {B_1} {B_2} {e_2}
            }\rname{L-CastR}}

% Env level

\newcommand*{\LERefl}{\inferrule{
            }{
            \tctx \lessp \tctx
            }\rname{L-ERefl}}

\newcommand*{\LEPush}{\inferrule{
            \tctx_1 \lessp \tctx_2
         \\ A_1 \lessp A_2
            }{
            \tctx_1, x: A_1 \lessp \tctx_2, x:A_2
            }\rname{L-EPush}}

% ------------------------------------------------------
% ORIGINAL HIGHER-RANKED TYPE
% ------------------------------------------------------

\newcommand*{\HVar}{\inferrule{
            x : A \in \tctx
            }{
            \tctx \byinf x \infto A
            }\rname{Var}}

\newcommand*{\HNat}{\inferrule{
            }{
            \tpreinf n \infto \nat
            }\rname{Nat}}

\newcommand*{\HLamAnnA}{\inferrule{
            \tctx, x: A \byinf e \infto B
            }{
            \tctx \byinf (\blam x A e) \infto A \to B
            }\rname{LamAnn-I}}

\newcommand*{\HLamAnnB}{\inferrule{
            B \tsub A
         \\ \tctx, x: A \bychk e \chkby C
            }{
            \tctx \byinf (\blam x A e) \chkby B \to C
            }\rname{LamAnn-C}}

\newcommand*{\HApp}{\inferrule{
            \tctx \byinf e_1 \infto A
         \\ \tctx \byinf A \bullet e \appto B
            }{
            \tctx \byinf e_1 ~ e_2 \infto B
            }\rname{App}}

\newcommand*{\HAppPoly}{\inferrule{
            \tctx \byinf A \subst a \tau \bullet e \appto B
            }{
            \tctx \byinf \forall a. A \bullet e \appto B
            }\rname{AppPoly}}

\newcommand*{\HAppFun}{\inferrule{
            \tprechk e \chkby A_1
            }{
            \tctx \byinf A_1 \to A_2 \bullet e \appto A_2
            }\rname{AppFun}}

\newcommand*{\HSub}{\inferrule{
            \tctx \byinf e \infto A
         \\ \tpresub A \tsub B
            }{
            \tctx \bychk e \chkby B
            }\rname{Sub}}

\newcommand*{\HAll}{\inferrule{
            \tctx, a \bychk e \chkby A
            }{
            \tctx \bychk e \chkby \forall a. A
            }\rname{Forall}}

\newcommand*{\SForallR}{\inferrule{
            \tctx, a \bysub A \tsub B
            }{
            A \tsub \forall a. B
            }\rname{S-ForallR}}

\newcommand*{\SForallL}{\inferrule{
            \tctx \bywf \tau
         \\ \tctx \bysub A \subst a \tau \tsub B
            }{
            \tpresub \forall a. A \tsub B
            }\rname{S-ForallL}}

\newcommand*{\SFun}{\inferrule{
            \tpresub B_1 \tsub A_1
         \\ \tpresub A_2 \tsub B_2
            }{
            \tpresub A_1 \to A_2 \tsub B_1 \to B_2
            }\rname{S-Fun}}

\newcommand*{\STVar}{\inferrule{
            a \in \tctx
            }{
            \tctx \bysub a \tsub a
            }\rname{S-TVar}}

\newcommand*{\SInt}{\inferrule{
            }{
            \tctx \bysub \nat \tsub \nat
            }\rname{S-Int}}

% ------------------------------------------------------
% GRADUAL HIGHER-RANKED TYPE
% ------------------------------------------------------

\newcommand*{\HRVar}{\inferrule{
            x : A \in \tctx
            }{
            \tctx \byinf x \infto A
            }\rname{Var}}

\newcommand*{\HRNat}{\inferrule{
            }{
            \tpreinf n \infto \nat
            }\rname{Nat}}

\newcommand*{\HRLamAnnA}{\inferrule{
            \tctx, x: A \byinf e \infto B
            }{
            \tctx \byinf \blam x A e \infto A \to B
            }\rname{LamAnn-I}}

\newcommand*{\HRLamAnnB}{\inferrule{
            C \match A_1 \to B
         \\ \tpresub A_1 \tconssub A_2
         \\ A = A_2 \glb A_1
         \\ \tctx, x: A \bychk e \chkby B
            }{
            \tctx \byinf \blam x {A_2} e \chkby C
            }\rname{LamAnn-C}}

\newcommand*{\HRApp}{\inferrule{
            \tctx \byinf e_1 \infto A
         \\ A \match A_1 \to A_2
         \\ \tctx \bychk e_2 \chkby A_1
            }{
            \tctx \byinf e_1 ~ e_2 \infto A_2
            }\rname{App}}

\newcommand*{\HRSub}{\inferrule{
            e \neq (\blam x C e')
         \\ \tctx \byinf e \infto A
         \\ \tpresub A \tconssub B
            }{
            \tctx \bychk e \chkby B
            }\rname{Sub}}

\newcommand*{\HRAll}{\inferrule{
            \tctx, a \bychk e \chkby A
            }{
            \tctx \bychk e \chkby \forall a. A
            }\rname{Forall}}

% -- SUBTYPING

\newcommand*{\HSForallR}{\inferrule{
            \tpresub[,a] A \tsub B
            }{
            \tpresub A \tsub \forall a. B
            }\rname{S-ForallR}}

\newcommand*{\HSForallL}{\inferrule{
            \dctx \bywf \tau
         \\ \tpresub A \subst a \tau \tsub B
            }{
            \tpresub \forall a. A \tsub B
            }\rname{S-ForallL}}

\newcommand*{\HSFun}{\inferrule{
            \tpresub B_1 \tsub A_1
         \\ \tpresub A_2 \tsub B_2
            }{
            \tpresub A_1 \to A_2 \tsub B_1 \to B_2
            }\rname{S-Fun}}

\newcommand*{\HSTVar}{\inferrule{
            a \in \dctx
            }{
            \tpresub a \tsub a
            }\rname{S-TVar}}

\newcommand*{\HSInt}{\inferrule{
            }{
            \tpresub \nat \tsub \nat
            }\rname{S-Int}}

\newcommand*{\HSUnknown}{\inferrule{
            }{
            \tpresub \unknown \tsub \unknown
            }\rname{S-Unknown}}

\newcommand*{\HSSVar}{\inferrule{
            }{
            \tpresub \static \tsub \static
            }\rname{S-SVar}}

\newcommand*{\HSGVar}{\inferrule{
            }{
            \tpresub \gradual \tsub \gradual
            }\rname{S-GVar}}

% ------------------------------------------------------
% GRADUAL HIGHER-RANKED TYPING : DECLARATIVE
% ------------------------------------------------------

\newcommand*{\DVar}{\inferrule{
            x : A \in \dctx
            }{
            \dctx \byinf x : A
            \trto {x}
            }\rname{Var}}

\newcommand*{\DNat}{\inferrule{
            }{
            \tpreinf n : \nat
            \trto {n}
            }\rname{Nat}}

\newcommand*{\DLam}{\inferrule{
            \dctx, x: \tau \byinf e : B
            \trto {s}
            }{
            \dctx \byinf \erlam x e : \tau \to B
            \trto {\blam x \tau s}
            }\rname{Lam}}

\newcommand*{\DLamAnnA}{\inferrule{
            \dctx, x: A \byinf e : B
            \trto {s}
            }{
            \dctx \byinf \blam x A e : A \to B
            \trto {\blam x A s}
            }\rname{LamAnn}}

\newcommand*{\DApp}{\inferrule{
            \dctx \byinf e_1 : A
            \trto {s_1}
         \\ \dctx \byinf A \match A_1 \to A_2
         \\ \dctx \byinf e_2 : A_3
            \trto {s_2}
         \\ \tpreconssub A_3 \tconssub A_1
            }{
            \dctx \byinf e_1 ~ e_2 : A_2
            \trto {(\cast A {A_1 \to A_2} s_1) ~
            (\cast {A_3} {A_1} s_2)
            }
            }\rname{App}}

\newcommand*{\DGen}{\inferrule{
            \dctx, a \byinf e : A \trto {s}
            }{
            \dctx \byinf e : \forall a. A
            \trto {\Lambda a. s}
            }\rname{Gen}}

% ------------------------------------------------------
% GRADUAL HIGHER-RANKED TYPING : Algorithmic
% ------------------------------------------------------

\newcommand*{\AVar}{\inferrule{
            (x : A) \in \tctx
            }{
            \Gamma \vdash x \infto A \toctxo
            }\rname{AVar}}

\newcommand*{\ANat}{\inferrule{
            }{
            \Gamma \vdash n \infto \nat \toctxo
            }\rname{ANat}}

\newcommand*{\ALamU}{\inferrule{
             \tctx, \genA, \genB, x : \genA \bychk e \chkby \genB \toctxr, x : \genA, \ctxl
            }{
            \tctx \byinf \erlam x e \infto \genA \to \genB \toctxr
            }\rname{ALamU}}


\newcommand*{\ALamAnnA}{\inferrule{
            \tctx, x: A \byinf e \infto B \toctxr,  x : A , \ctxl
            }{
            \tctx \byinf \blam x A e \infto A \to B \toctxr
            }\rname{ALamAnnA}}

\newcommand*{\ALam}{\inferrule{
            \tctx, x: A \byinf e \chkby B \toctxr,  x : A , \ctxl
            }{
            \tctx \byinf \erlam x e \chkby A \to B \toctxr
            }\rname{ALam}}

\newcommand*{\AGen}{\inferrule{
            \tctx, a \bychk e \chkby A \toctxr, a, \ctxl
            }{
            \tctx \bychk e \chkby \forall a. A \toctxr
            }\rname{AGen}}

\newcommand*{\AAnno}{\inferrule{
            \tctx \vdash A
            \\
            \tctx \bychk e \chkby A \toctxr
            }{
            \tctx \bychk e : A \infto A \toctxr
            }\rname{AAnno}}


\newcommand*{\AApp}{\inferrule{
            \Gamma \vdash e_1 \infto A \dashv \ctxl_1
         \\ \ctxl_1 \byinf \ctxsubst{\ctxl_1}{A} \match A_1 \to A_2 \dashv \ctxl_2
         \\ \ctxl_2 \byinf e_2 \chkby \ctxsubst{\ctxl_2}{A_1} \dashv \ctxr
            }{
            \Gamma \vdash e_1 ~ e_2 \infto A_2 \toctxr
            }\rname{AApp}}

\newcommand*{\ASub}{\inferrule{
            \tctx \byinf e \infto A \toctx
         \\ \ctxl \bysub \ctxsubst{\ctxl} A \tconssub \ctxsubst{\ctxl} B \toctxr
            }{
            \tctx \bychk e \chkby B \toctxr
            }\rname{ASub}}


% ------------------------------------------------------
% Context extension
% ------------------------------------------------------


\newcommand*{\ExtID}{\inferrule{
            }{
            \ctxinit \exto \ctxinit
            }\rname{ExtID}}

\newcommand*{\ExtVar}{\inferrule{
              \Gamma \exto \Delta \\
              \ctxsubst{\Delta}{A} = \ctxsubst{\Delta}{A'}
            }{
            \Gamma, x : A \exto \Delta, x : A'
            }\rname{ExtVar}}

\newcommand*{\ExtUVar}{\inferrule{
              \Gamma \exto \Delta
            }{
            \Gamma, a \exto \Delta, a
            }\rname{ExtUVar}}

\newcommand*{\ExtEVar}{\inferrule{
              \Gamma \exto \Delta
            }{
            \Gamma, \genA \exto \Delta, \genA
            }\rname{ExtEVar}}

\newcommand*{\ExtSolved}{\inferrule{
              \Gamma \exto \Delta \\
              \ctxsubst{\Delta}{\tau} = \ctxsubst{\Delta}{\tau'}
            }{
            \Gamma, \genA = \tau \exto \Delta, \genA = \tau'
            }\rname{ExtSolved}}

\newcommand*{\ExtSolve}{\inferrule{
              \Gamma \exto \Delta
            }{
            \Gamma, \genA \exto \Delta, \genA = \tau
            }\rname{ExtSolve}}

\newcommand*{\ExtAdd}{\inferrule{
              \Gamma \exto \Delta
            }{
            \Gamma \exto \Delta, \genA
            }\rname{ExtAdd}}

\newcommand*{\ExtAddS}{\inferrule{
              \Gamma \exto \Delta
            }{
            \Gamma \exto \Delta, \genA = \tau
            }\rname{ExtAddSolved}}



% ------------------------------------------------------
% HIGHER-RANKED TYPING : NON-BI
% ------------------------------------------------------

\newcommand*{\NVar}{\inferrule{
            x : A \in \dctx
            }{
            \dctx \byhinf x : A
            }\rname{Var}}

\newcommand*{\NNat}{\inferrule{
            }{
            \dctx \byhinf n : \nat
            }\rname{Nat}}

\newcommand*{\NLam}{\inferrule{
            \dctx, x: \tau \byhinf e : B
            }{
            \dctx \byhinf \erlam x e : \tau \to B
            }\rname{Lam}}

\newcommand*{\NGen}{\inferrule{
            \dctx, a \byhinf e : A
            }{
            \dctx \byhinf e : \forall a. A
            }\rname{Gen}}

\newcommand*{\NLamAnnA}{\inferrule{
            \dctx, x: A \byhinf e : B
            }{
            \dctx \byhinf \blam x A e : A \to B
            }\rname{LamAnn}}

\newcommand*{\NApp}{\inferrule{
            \dctx \byhinf e_1 : A_1 \to A_2
         \\ \dctx \byhinf e_2 : A_1
            }{
            \dctx \byhinf e_1 ~ e_2 : A_2
            }\rname{App}}

\newcommand*{\NSub}{\inferrule{
            \dctx \byhinf e : A_1
         \\ \tpresub A_1 \tsub A_2
            }{
            \dctx \byhinf e : A_2
            }\rname{Sub}}

\newcommand*{\NForallR}{\inferrule{
            \tpresub[,a] A \tsub B
            }{
            \tpresub A \tsub \forall a. B
            }\rname{ForallR}}

\newcommand*{\NForallL}{\inferrule{
            \dctx \bywf \tau
         \\ \tpresub A \subst a \tau \tsub B
            }{
            \tpresub \forall a. A \tsub B
            }\rname{ForallL}}

\newcommand*{\NFun}{\inferrule{
            \tpresub B_1 \tsub A_1
         \\ \tpresub A_2 \tsub B_2
            }{
            \tpresub A_1 \to A_2 \tsub B_1 \to B_2
            }\rname{CS-Fun}}

\newcommand*{\NTVar}{\inferrule{
            a \in \dctx
            }{
            \tpresub a \tsub a
            }\rname{CS-TVar}}

\newcommand*{\NSInt}{\inferrule{
            }{
            \tpresub \nat \tsub \nat
            }\rname{CS-Int}}


% ------------------------------------------------------
% MASK OFF
% ------------------------------------------------------

\newcommand*{\FA}{\inferrule{
            }{
            \mask A \unknown = \unknown
            }\rname{F-StarR}}

\newcommand*{\FB}{\inferrule{
            }{
            \mask \unknown A = \unknown
            }\rname{F-StarL}}

\newcommand*{\FC}{\inferrule{
            }{
            \mask {\forall a. A} B = \mask A B
            }\rname{F-ForallL}}

\newcommand*{\FD}{\inferrule{
            }{
            \mask A {\forall a. B} = \mask A B
            }\rname{F-ForallR}}

\newcommand*{\FE}{\inferrule{
            }{
            \mask {A_1 \to A_2} {B_1 \to B_2} = \mask {A_1} {B_1} \to \mask {A_2} {B_2}
            }\rname{F-Fun}}

% ------------------------------------------------------
% PBC
% ------------------------------------------------------

\newcommand*{\PBCVar}{\inferrule{
            x : A \in \tctx
            }{
            \tctx \bypinf x \infto A
            }\rname{PBC-Var}}

\newcommand*{\PBCNat}{\inferrule{
            }{
            \tctx \bypinf n \infto \nat
            }\rname{PBC-Var}}

\newcommand*{\PBCApp}{\inferrule{
            \tctx \bypinf e_1 \infto A_1 \to A_2
         \\ \tctx \bypinf e_2 \infto A_1
            }{
            \tctx \bypinf e_1 ~ e_2 \infto A_2
            }\rname{PBC-App}}

\newcommand*{\PBCLam}{\inferrule{
            \tctx, x: A \bypinf e \infto B
            }{
            \tctx \bypinf \blam x A e \infto A \to B
            }\rname{PBC-Lam}}

\newcommand*{\PBCBLam}{\inferrule{
            \tctx, X \bypinf e \infto A
            }{
            \tctx \bypinf \Lambda X. e \infto \forall X. A
            }\rname{PBC-BLam}}

\newcommand*{\PBCTApp}{\inferrule{
            \tctx, X \bypinf e \infto  \forall X. A
            }{
            \tctx \bypinf e ~ [B] \infto A \subst X B
            }\rname{PBC-TApp}}

\newcommand*{\PBCCast}{\inferrule{
            \tctx \bypinf e \infto A
         \\ A \pbccons B
            }{
            \tctx \bypinf \cast A B e \infto B
            }\rname{PBC-TApp}}

% ------------------------------------------------------
% PBC Compatibility
% ------------------------------------------------------

\newcommand*{\CompRefl}{\inferrule{
            }{
            A \pbccons A
            }\rname{Comp-Refl}}

\newcommand*{\CompUnknownR}{\inferrule{
            }{
            A \pbccons \unknown
            }\rname{Comp-UnknownR}}

\newcommand*{\CompUnknownL}{\inferrule{
            }{
            \unknown \pbccons A
            }\rname{Comp-UnknownL}}

\newcommand*{\CompArrow}{\inferrule{
            A_1 \pbccons B_1
         \\ A_2 \pbccons B_2
            }{
            A_1 \to A_2 \pbccons B_1 \to B_2
            }\rname{Comp-Arrow}}

\newcommand*{\CompAllR}{\inferrule{
            A \pbccons B
            }{
            A \pbccons \forall X. B
            }\rname{Comp-AllR}}

\newcommand*{\CompAllL}{\inferrule{
            A \subst X \star \pbccons B
            }{
            \forall X. A \pbccons B
            }\rname{Comp-AllL}}

% ------------------------------------------------------
% EXTENSION
% ------------------------------------------------------

\newcommand*{\SubTop}{\inferrule{
            A ~ static
            }{
            \dctx \bysub A \tsub \top
            }\rname{S-Top}}

\newcommand*{\CTop}{\inferrule{}{
            \top \sim \top
            }}

\newcommand*{\CSTop}{\inferrule{
            }{
            \dctx \bysub A \tconssub \top
            }\rname{CS-Top}}


% ------------------------------------------------------
% Well-formedess of type under declarative context
% ------------------------------------------------------

\newcommand*{\DeclVarWF}{\inferrule{
              a \in \dctx
            }{
            \dctx \vdash a
            }\rname{DeclVarWF}}

\newcommand*{\DeclIntWF}{\inferrule{
            }{
            \dctx \vdash \nat
            }\rname{DeclIntWF}}

\newcommand*{\DeclUnknownWF}{\inferrule{
            }{
            \dctx \vdash \unknown
            }\rname{DeclUnknownWF}}


\newcommand*{\DeclFunWF}{\inferrule{
              \dctx \vdash A \\ \dctx \vdash B
            }{
            \dctx \vdash A \to B
            }\rname{DeclFunWF}}

\newcommand*{\DeclForallWF}{\inferrule{
              \dctx, a \vdash A
            }{
            \dctx \vdash \forall a. A
            }\rname{DeclForallWF}}

% ------------------------------------------------------
% Well-formedess of type under algorithmic context
% ------------------------------------------------------

\newcommand*{\VarWF}{\inferrule{
            }{
            \Gamma[a] \vdash a
            }\rname{VarWF}}

\newcommand*{\IntWF}{\inferrule{
            }{
            \Gamma \vdash \nat
            }\rname{IntWF}}

\newcommand*{\UnknownWF}{\hlmath{\inferrule{
            }{
            \Gamma \vdash \unknown
            }\rname{UnknownWF}}}

\newcommand*{\FunWF}{\inferrule{
              \Gamma \vdash A \\ \Gamma \vdash B
            }{
            \Gamma \vdash A \to B
            }\rname{FunWF}}

\newcommand*{\ForallWF}{\inferrule{
              \Gamma, a \vdash A
            }{
            \Gamma \vdash \forall a. A
            }\rname{ForallWF}}

\newcommand*{\EVarWF}{\inferrule{
            }{
            \Gamma[\genA] \vdash \genA
            }\rname{EVarWF}}

\newcommand*{\SolvedEVarWF}{\inferrule{
            }{
            \Gamma[\genA = \tau] \vdash \genA
            }\rname{SolvedEVarWF}}

% ------------------------------------------------------
% OBJECTS: SUBTYPING
% ------------------------------------------------------

\newcommand*{\ObSInt}{\inferrule{}
            {
            \nat \tsub \nat
            }}

\newcommand*{\ObSBool}{\inferrule{}
            {
            \bool \tsub \bool
            }}

\newcommand*{\ObSFloat}{\inferrule{}
            {
            \float \tsub \float
            }}

\newcommand*{\ObSIntFloat}{\inferrule{}
            {
            \nat \tsub \float
            }}

\newcommand*{\ObFun}{\inferrule{B_1 \tsub A_1 \\ A_2 \tsub B_2}
            {
            A_1 \to A_2 \tsub B_1 \to B_2
            }}

\newcommand*{\ObSUnknown}{\inferrule{}
            {
            \unknown \tsub \unknown
            }}

\newcommand*{\ObSRecord}{\inferrule{}
            {
            [l_i : A_i^{i \in 1...n+m}] \tsub
            [l_i : A_i^{i \in 1...n}]
            }}

% ------------------------------------------------------
% OBJECTS: CONSISTENCY
% ------------------------------------------------------

\newcommand*{\ObCRefl}{\inferrule{}
            {
            A \sim A
            }}

\newcommand*{\ObCUnknownR}{\inferrule{}
            {
            A \sim \unknown
            }}

\newcommand*{\ObCUnknownL}{\inferrule{}
            {
            \unknown \sim A
            }}

\newcommand*{\ObCC}{\inferrule{
            A_1 \sim B_1
         \\ A_2 \sim B_2
            }{
            A_1 \to A_2 \sim B_1 \to B_2
            }}


\newcommand*{\ObCRecord}{\inferrule{
            A_i \sim B_i
            }
            {
            [l_i: A_i] \sim [l_i:B_i]
            }}





% \makeatletter\if@ACM@journal\makeatother
% %% Journal information (used by PACMPL format)
% %% Supplied to authors by publisher for camera-ready submission
% \acmJournal{PACMPL}
% \acmVolume{1}
% \acmNumber{1}
% \acmArticle{1}
% \acmYear{2017}
% \acmMonth{1}
% \acmDOI{10.1145/nnnnnnn.nnnnnnn}
% \startPage{1}
% \else\makeatother
% %% Conference information (used by SIGPLAN proceedings format)
% %% Supplied to authors by publisher for camera-ready submission
% \acmConference[PL'17]{ACM SIGPLAN Conference on Programming Languages}{January 01--03, 2017}{New York, NY, USA}
% \acmYear{2017}
% \acmISBN{978-x-xxxx-xxxx-x/YY/MM}
% \acmDOI{10.1145/nnnnnnn.nnnnnnn}
% \startPage{1}
% \fi


%% Copyright information
%% Supplied to authors (based on authors' rights management selection;
%% see authors.acm.org) by publisher for camera-ready submission
\setcopyright{none}             %% For review submission
%\setcopyright{acmcopyright}
%\setcopyright{acmlicensed}
%\setcopyright{rightsretained}
%\copyrightyear{2017}           %% If different from \acmYear


%% Bibliography style
\bibliographystyle{ACM-Reference-Format}
%% Citation style
%% Note: author/year citations are required for papers published as an
%% issue of PACMPL.
\citestyle{acmauthoryear}   %% For author/year citations



\begin{document}

%% Title information
\title{Consistent Subtyping for All}
\ifdefined\submitoption
\subtitle{Long version of paper, including supplementary meterial}         %% [Short Title] is optional;
\fi

                                        %% when present, will be used in
                                        %% header instead of Full Title.
% \titlenote{with title note}             %% \titlenote is optional;
                                        %% can be repeated if necessary;
                                        %% contents suppressed with 'anonymous'
% \subtitle{Subtitle}                     %% \subtitle is optional
% \subtitlenote{with subtitle note}       %% \subtitlenote is optional;
                                        %% can be repeated if necessary;
                                        %% contents suppressed with 'anonymous'


%% Author information
%% Contents and number of authors suppressed with 'anonymous'.
%% Each author should be introduced by \author, followed by
%% \authornote (optional), \orcid (optional), \affiliation, and
%% \email.
%% An author may have multiple affiliations and/or emails; repeat the
%% appropriate command.
%% Many elements are not rendered, but should be provided for metadata
%% extraction tools.

%% Author with single affiliation.
\author{First1 Last1}
\authornote{with author1 note}          %% \authornote is optional;
                                        %% can be repeated if necessary
\orcid{nnnn-nnnn-nnnn-nnnn}             %% \orcid is optional
\affiliation{
  \position{Position1}
  \department{Department1}              %% \department is recommended
  \institution{Institution1}            %% \institution is required
  \streetaddress{Street1 Address1}
  \city{City1}
  \state{State1}
  \postcode{Post-Code1}
  \country{Country1}
}
\email{first1.last1@inst1.edu}          %% \email is recommended

%% Author with two affiliations and emails.
\author{First2 Last2}
\authornote{with author2 note}          %% \authornote is optional;
                                        %% can be repeated if necessary
\orcid{nnnn-nnnn-nnnn-nnnn}             %% \orcid is optional
\affiliation{
  \position{Position2a}
  \department{Department2a}             %% \department is recommended
  \institution{Institution2a}           %% \institution is required
  \streetaddress{Street2a Address2a}
  \city{City2a}
  \state{State2a}
  \postcode{Post-Code2a}
  \country{Country2a}
}
\email{first2.last2@inst2a.com}         %% \email is recommended
\affiliation{
  \position{Position2b}
  \department{Department2b}             %% \department is recommended
  \institution{Institution2b}           %% \institution is required
  \streetaddress{Street3b Address2b}
  \city{City2b}
  \state{State2b}
  \postcode{Post-Code2b}
  \country{Country2b}
}
\email{first2.last2@inst2b.org}         %% \email is recommended


%% Paper note
%% The \thanks command may be used to create a "paper note" ---
%% similar to a title note or an author note, but not explicitly
%% associated with a particular element.  It will appear immediately
%% above the permission/copyright statement.
% \thanks{with paper note}                %% \thanks is optional
                                        %% can be repeated if necesary
                                        %% contents suppressed with 'anonymous'


%% Abstract
%% Note: \begin{abstract}...\end{abstract} environment must come
%% before \maketitle command
\begin{abstract}
Consistent subtyping is employed in some gradual type systems to
validate type conversions. The original
definition, proposed by \citeauthor{siek2007gradual}, serves as a guideline for
designing many gradual type systems with subtyping. Polymorphic types \`a la
System F also induce a subtyping relation that relates polymorphic types
to their instantiations. However \citeauthor{siek2007gradual}'s definition
is not adequate for some kinds of subtyping, including the subtyping
relation arising from implicit polymorphism.

The first goal of this paper is to propose a generalization of consistent
subtyping that is adequate for polymorphic subtyping, and subsumes the original
definition by \citeauthor{siek2007gradual}. The new definition of consistent
subtyping provides novel insights with respect to previous polymorphic gradual
type systems, which did not employ consistent subtyping. For instance both
\citeauthor{ahmed2011blame} (in the Polymorphic Blame Calculus) and
\citeauthor{yuu2017poly} use, respectively, 
notions of \emph{compatibility} and \emph{type consistency} instead
of consistent subtyping to validate casts. We argue that, for implicit
polymorphism, \citeauthor{ahmed2011blame}'s notion of
compatibility is too permissive (i.e. too many programs are allowed to
type-check), and that \citeauthor{yuu2017poly}'s notion of type consistency is too
conservative (i.e. programs that should type check are rejected).
To further validate our
generalized notion of consistent subtyping we also study the addition 
of top types to gradual type systems. Like polymorphism, the notion of 
consistent subtyping proposed by \citeauthor{siek2007gradual} does not work for top
types, but our revised definition can account for top types.
%To further validate our revised notion of
%consistent subtyping, we show that it coincides with a notion of
%consistent subtyping arizing from an extension of 
%Garcia et al.'s \emph{Abstracting Gradual Typing} (AGT) (2016) with
%polymorphism. 
The second goal of this paper is to present a gradually typed calculus for
implicit (higher-rank) polymorphism that uses our new notion of consistent
subtyping. We develop both declarative and algorithmic versions for the type
system. The algorithmic version employs techniques developed by
\citeauthor{dunfield2013complete} to deal with instantiation. We prove that the
new calculus satisfies all of the refined criteria for gradual typing. All of
the metatheory of this paper, except some manual proofs for the algorithmic type
system, has been mechanically formalized using the Coq proof assistant.
\end{abstract}


%% 2012 ACM Computing Classification System (CSS) concepts
%% Generate at 'http://dl.acm.org/ccs/ccs.cfm'.
\begin{CCSXML}
<ccs2012>
<concept>
<concept_id>10011007.10011006.10011008</concept_id>
<concept_desc>Software and its engineering~General programming languages</concept_desc>
<concept_significance>500</concept_significance>
</concept>
<concept>
<concept_id>10003456.10003457.10003521.10003525</concept_id>
<concept_desc>Social and professional topics~History of programming languages</concept_desc>
<concept_significance>300</concept_significance>
</concept>
</ccs2012>
\end{CCSXML}

% \ccsdesc[500]{Software and its engineering~General programming languages}
% \ccsdesc[300]{Social and professional topics~History of programming languages}
%% End of generated code


%% Keywords
%% comma separated list
% \keywords{keyword1, keyword2, keyword3}  %% \keywords is optional


%% \maketitle
%% Note: \maketitle command must come after title commands, author
%% commands, abstract environment, Computing Classification System
%% environment and commands, and keywords command.
\maketitle


%% -- Starting Point --

\section{Introduction}

Modern statically typed functional languages (such as ML, Haskell,
Scala or OCaml) have increasingly expressive type systems. Often these
large source languages are translated into a much smaller typed core
language. The choice of the core language is essential to ensure that
all the features of the source language can be encoded. For a simple
polymorphic functional language it is possible to pick a
variant of System $F$~\cite{systemfw,Reynolds:1974} as a core
language. However, the desire for more expressive type system features
puts pressure on the core languages, often requiring them to be
extended to support new features.  For example, if the source language
supports \emph{higher-kinded types} or \emph{type-level functions}
then System $F$ is not expressive enough and can no longer be used as
the core language. Instead another core language that does provide
support for higher-kinded types, such as
System~$F_{\omega}$~\cite{systemfw}, needs to be used. Of course the
drive to add more and more advanced type-level features means that
eventually the core language needs to be extended again. Indeed modern
functional languages like Haskell use specially crafted core
languages, such as System $F_{C}$~\cite{fc}, that provide support for all
modern features of Haskell. Although \emph{extensions} of System
$F_{C}$~\cite{fc:pro,Eisenberg:2014} satisfy the current needs of
modern Haskell, it is very likely to be extended again in the
future~\cite{fc:kind}. Moreover System $F_{C}$ has grown to be a relatively
large and complex language, with multiple syntactic levels, and dozens
of language constructs.

\begin{comment}
However System~$F_{\omega}$ is
significantly more complex than System F and thus harder to
maintain. If later a new feature, such as \emph{kind polymorphism}, is
desired the core language may need to be changed again to account for
the new feature, introducing at the same time new sources of
complexity. Indeed the core language for modern versions of 
functional languages are quite complex, having multiple syntactic 
sorts (such as terms, types and kinds), as well as dozens of 
language constructs~\cite{}\bruno{$F_{C}$}. 
\end{comment}

The more expressive type (and kind) systems become, the more types become similar
to the terms. Therefore a natural idea is to unify terms and
types. There are obvious benefits in this approach: only one syntactic
level (terms) is needed; and there are much less language constructs,
making the core language easier to reason, implement and maintain. At the same
time the core language becomes more expressive, giving us for free
many useful language features. Moreover, due to the inherent
expressiveness, extensions are less likely to be required.
\emph{Pure type systems} (PTS)~\cite{handbook} build
on such observations and show how a whole family of type systems
(including System $F$ and System $F_{\omega}$) can be implemented
using just a single syntactic form. With the added expressiveness it
is even possible to have type-level programs expressed using the same
syntax as terms, as well as dependently typed programs~\cite{coc}.
Because the idea of using a unified syntax is so appealing several
researchers have in the past considered such an
option for implementing functional languages~\cite{cayenne, typeintype, pts:henk}.

However having the same syntax for types and terms can also be
problematic. Usually type systems based on PTS have a conversion rule
to support type-level computation.  In such type systems ensuring the
\emph{decidability} of type checking requires type-level computation
to terminate. When the syntax of types and terms is the same, the
decidability of type checking is usually dependent on the strong
normalization of the calculus. An example is the proof of decidability
of type checking for the \emph{calculus of constructions}~\cite{coc}
(and other normalizing PTS), which depends on strong normalization
~\cite{pts:normalize}. Modern dependently
typed languages such as Idris~\cite{idris} and Agda~\cite{agda}, which are also
built on a unified syntax for types and terms, require strong
normalization as well: all recursive programs must pass a termination
checker.  An unfortunate consequence of coupling
decidability of type checking and strong normalization is that adding
(unrestricted) general recursion to such calculi is difficult. Indeed
past work on using a simple PTS-like calculi to model functional languages
with unrestricted general recursion, had to give up on decidability of
type-checking~\cite{cayenne, typeintype}.
%There
%is a clear tension between decidability of type checking and allowing
%general recursion in calculi with unified syntax.

This paper proposes \name: a simple yet expressive call-by-name
variant of the calculus of constructions, which has a fraction of the
language constructs of existing core languages. The key challenge
solved in this work is how to define a calculus comparable in
simplicity to the calculus of constructions, while featuring both
general recursion and decidable type checking. The main idea, 
inspired by the traditional treatment of \emph{iso-recursive
  types}~\cite{tapl}, is to recover decidable type-checking by making each
type-level computation step explicit, i.e., each beta reduction or
expansion at the type level is controlled by a \emph{type-safe}
cast. Since single computation steps are trivially terminating, decidability
of type checking is possible even in the presence of non-terminating
programs at the type level.  At the same time term-level programs
using general recursion work as in any conventional functional
languages, and can even be non-terminating.

\begin{comment}
For example, if a type-level program requires two beta reductions to
reach normal form, then two casts are needed in the program. If a
non-terminating program is used at the type level, it is not possible
to cause non-termination in the type checker, because that would
require a program with an infinite number of casts. Therefore, since
single beta-steps are trivially terminating, decidability of type
checking is possible even in the presence of non-terminating programs
at the type level.  At the same time term-level programs using general
recursion work as in any conventional functional languages, and can
even be non-terminating.
\end{comment}

Our motivation to develop \name is to use it as a simpler alternative
to existing core languages for functional programming. We focus on traditional
functional languages like ML or Haskell extended with many interesting
type-level features, but perhaps not the \emph{full power} of
dependent types.  The paper shows how many of programming language
features of Haskell, including some of the latest extensions, can be
encoded in \name via a surface language. The surface
language supports \emph{algebraic datatypes}, \emph{higher-kinded
  types}, \emph{nested datatypes}~\cite{nesteddt}, \emph{kind
  polymorphism}~\cite{fc:pro} and \emph{datatype
  promotion}~\cite{fc:pro}.  This result is interesting because \name
is a minimal calculus with only 8 language constructs and a single
syntactic sort. In contrast the latest versions of System $F_{C}$
(Haskell's core language) have multiple syntactic sorts and dozens of
language constructs.
%Even if support for equality and
%coercions, which constitutes a significant part of System $F_{C}$,
%would be removed the resulting language would still be significantly
%larger and more complex than \name.

It is worth emphasizing that \name does sacrifice having an expressive form
of type equality to gain the ability of doing arbitrary general
recursion at the term level.  Nevertheless, 
the core language (System $F_{C}$) of Haskell also comes with a similarly weak
notion of type equality.  In both System $F_{C}$ and \name, type
equality in \name is purely syntactic (modulo alpha-conversion).

A non-goal of the current work (although a worthy avenue for future
work) is to use \name as a core language for modern dependently typed
languages like Agda or Idris. In contrast to \name, those languages
use a more powerful notion of equality. In particular \name
currently lacks full-reduction and it is unable to exploit injectivity 
properties when comparing two types for equality. Moreover,
\name (and also System $F_{C}$) lack \emph{logical consistency}:
that is ensuring the soundness of proofs written as programs.
This is in contrast to dependently typed languages, where logical
consistency is typically ensured.
Various researchers~\cite{zombie:popl14,zombie:thesis,Swamy2011} have been investigating how to combine logical
consistency, general recursion and dependent types. However, this is
usually done by having the type system carefully control the total and
partial parts of computation, making those calculi significantly more
complex than \name or the calculus of constructions. In
\name, logical consistency is traded by the simplicity of the system.

\begin{comment}
In particular
the treatment of type-level computation in \name shares similar ideas
with Haskell. Although Haskell's surface language provides a rich set
of mechanisms to do type-level computation~\cite{}, the core language
lacks fundamental mechanisms todo type-level computation. Type
equality in System $F_{C}$ is, like in \name, purely syntactic (modulo
alpha-conversion).
\end{comment}

\begin{comment}
 and there is no type-level
abstraction. In other words in Haskell, mechanisms such as type
classes and type families

Although it may seem that forcing each step of computation 
at the type-level to be explicit will prevent convinient use of 
type-level computation.

Point about the treatment of type-level computation in Haskell. Haskell's
core language has type applications, but no type-level lambda. Equality 
is syntactic modulo alpha-conversion. This design choice was rooted in the 
desire to support Hindley-Milner type-inference... 
\end{comment}

In summary, the contributions of this work are:

\begin{itemize}
\item {\bf The \name calculus:} A simple core calculus for functional programming, that collapses terms, types and
  kinds into the same hierarchy and supports general recursion. \name
  is type-safe and the type system is decidable.

\item {\bf One-step casts and a generalization of iso-recursive types:} \name 
 generalizes iso-recursive types by making all type-level computation
 steps explicit via \emph{one-step casts}. In \name the combination of
  one-step casts and recursion subsumes iso-recursive types.

\item {\bf An expressive surface language}, built on top of \name,
  that supports datatypes, pattern matching and various advanced
  language extensions of Haskell. The type safety of the type-directed
  translation to \name is proved.

\item {\bf A prototype implementation:} The implementation of \name is
  available\footnote{\url{https://github.com/bixuanzju/full-version}}.
\end{itemize}

\begin{comment}
\begin{enumerate}[a)]
\item Motivations:

\begin{itemize}

\item Because of the reluctance to introduce dependent
  types\footnote{This might be changed in the near future. See
    \url{https://ghc.haskell.org/trac/ghc/wiki/DependentHaskell/Phase1}.},
  the current intermediate language of Haskell, namely System $F_C$
  \cite{fc}, separates expressions as terms, types and kinds, which
  brings complexity to the implementation as well as further
  extensions \cite{fc:pro,fc:kind}.

\item Popular full-spectrum dependently typed languages, like Agda,
  Coq, Idris, have to ensure the termination of functions for the
  decidability of proofs. No general recursion and the limitation of
  enforcing termination checking make such languages impractical for
  general-purpose programming.

\item We would like to introduce a simple and compiler-friendly
  dependently typed core language with only one hierarchy, which
  supports general recursion at the same time.

\end{itemize}

\item Contribution:

\begin{itemize}

\item A core language based on Calculus of Constructions (CoC) that
  collapses terms, types and kinds into the same hierarchy.

\item General recursion by introducing recursive types for both terms
  and types by the same $\mu$ primitive.

\item Decidable type checking and managed type-level computation by
  replacing implicit conversion rule of CoC with generalized
  \textsf{fold}/\textsf{unfold} semantics.

\item First-class equality by coercion, which is used for encoding
  GADTs or newtypes without runtime overhead.

\item Surface language that supports datatypes, pattern matching and
  other language extensions for Haskell, and can be encoded into the
  core language.

\end{itemize}


\end{enumerate}
\end{comment}

\section{Background}
\label{sec:background}

In this section we review a simple gradually typed language with
objects~\cite{siek2007gradual}, to introduce the concept of consistency
subtyping. We also briefly talk about the Odersky-L{\"a}ufer type system for
higher-rank types~\cite{odersky1996putting}, which serves as the original
language on which our gradually typed calculus with implicit
higher-rank polymorphism is based.


\subsection{Gradual Subtyping}

\begin{figure}[t]
  \begin{small}
  \begin{mathpar}
    \framebox{$A \tsub B$}\\
    \ObSInt \and \ObSBool \and \ObSFloat \and
    \ObSIntFloat \\ \ObFun \and
    \ObSRecord \and \ObSUnknown
  \end{mathpar}

  \begin{mathpar}
    \framebox{$A \sim B$}\\
    \ObCRefl \and \ObCUnknownR \and
    \ObCUnknownL \and \ObCC \and \ObCRecord
  \end{mathpar}

  \end{small}

  \caption{Subtyping and type consistency in \obb}
  \label{fig:objects}
\end{figure}

\citet{siek2007gradual} developed a gradual typed system for object-oriented
languages that they call \obb. Central to gradual typing is the concept of
\textit{consistency} (written $\sim$) between gradual types, which are types
that may involve the unknown type $\unknown$. The intuition is that consistency
relaxes the structure of a type system to tolerate unknown positions in a
gradual type. They also defined the subtyping relation in a way that static type
safety is preserved. Their key insight is that the unknown type $\unknown$ is
neutral to subtyping, with only $\unknown \tsub \unknown$. Both relations are
found in \Cref{fig:objects}.

A primary contribution of their work is to show that type consistency and
subtyping are orthogonal, and can be superimposed. To compose subtyping and
consistency, \citeauthor{siek2007gradual} defined \textit{consistent subtyping}
(written $\tconssub$) in multiple equivalent ways:

\begin{mdef}[Consistent Subtyping \`a la \citet{siek2007gradual}]\leavevmode
\label{def:old-decl-conssub}
\begin{itemize}
\item $A \tconssub B$ if and only if $A \sim C$ and $C \tsub B$ for some $C$.
\item $A \tconssub B$ if and only if $A \tsub C$ and $C \sim B$ for some $C$.
\end{itemize}
\end{mdef}

Both definitions are non-deterministic because of the intermediate type $C$. To
remove non-determinism, they came up with a so-called \textit{restriction
  operator}, written $\mask A B$ that masks off the parts of a type $A$ that are
unknown in a type $B$. The definition of the restriction operator is given
below:
\begin{small}
\begin{align*}
  \mask A B & =  \kw{case}~(A, B)~\kw{of}\\
               & \mid (-, \unknown) \Rightarrow \unknown\\
               & \mid ([l_1: A_1,...,l_n:A_n], [l_1:B_1,...,l_m:B_m]) \quad \kw{where} n \leq m \Rightarrow\\
               & \qquad [l_1 : \mask {A_1} {B_1}, ..., l_n : \mask {A_n} {B_n}]\\
               & \mid ([l_1: A_1,...,l_n:A_n], [l_1:B_1,...,l_m:B_m]) \quad \kw{where} n > m \Rightarrow\\
               & \qquad [l_1 : \mask {A_1} {B_1}, ..., l_m : \mask {A_m} {B_m},...,l_n:A_n ]\\
               & \mid (-, -) \Rightarrow A\\
  \mask {A_1 \to A_2} {B_1 \to B_2} & =  \mask {A_1} {B_1} \to \mask {A_2} {B_2}
\end{align*}
\end{small}
With the restriction operator, consistent subtyping is simply defined
as $A \tconssub B \equiv \mask A B \tsub \mask B A$. And they proved that this
definition is equivalent to \Cref{def:old-decl-conssub}.


\subsection{The Odersky-L{\"a}ufer Type System}

\begin{figure}[t]
  \begin{small}

    \begin{tabular}{lrcl} \toprule
      Expressions & $e$ & \syndef & $x \mid n \mid
                                    \blam x A e \mid e~e$ \\

      Types & $A, B$ & \syndef & $ \nat \mid a \mid A \to B \mid \forall a. A$ \\
      Monotypes & $\tau, \sigma$ & \syndef & $ \nat \mid a \mid \tau \to \sigma$ \\

      Contexts & $\dctx$ & \syndef & $\ctxinit \mid \dctx,x: A \mid \dctx, a$ \\  \bottomrule
    \end{tabular}

  \begin{mathpar}
    \framebox{$\dctx \byhinf e : A$}\\
    \NVar \and \NNat \and \NLamAnnA \and
    \NApp \and \NSub
  \end{mathpar}

  \begin{mathpar}
    \framebox{$\dctx \bysub A \tsub B$}\\
    \NForallL \and \NForallR \and
    \NFun \and \NTVar \and \NSInt
  \end{mathpar}

  \end{small}
  \caption{Syntax and static semantics of the Odersky-L{\"a}ufer type system.}
  \label{fig:original-typing}
\end{figure}


The calculus we are combining gradual typing with is the fully annotated version
of the well-established type system for higher-rank types proposed by
\citet{odersky1996putting}. One difference is that, for simplicity, we do not account 
for a let expression and, consequently, we do not have
let-generalization. However, there is already existing work about gradual type
systems with let expressions and let generalization (for example, see
\citep{garcia2015principal}). Similar techniques can
be applied to our calculus to enable let generalization.

The syntax of the type system, along with the typing and subtyping judgments is
given in \Cref{fig:original-typing}. We save the explanations for the
static semantics to \Cref{sec:type-system}, where we present our
gradually typed version of the calculus.

%%% Local Variables:
%%% mode: latex
%%% TeX-master: "../paper"
%%% org-ref-default-bibliography: "../paper.bib"
%%% End:
\section{Revisiting Consistent Subtyping}
\label{sec:exploration}

In this section we explore the design space of consistent subtyping. In addition
to the unknown type $\unknown$, we have the same syntax of types as in
\Cref{fig:original-typing}. We start with the definitions of consistency and
subtyping for polymorphic types, and compare with some relevant work (in
particular the compatibility relation by \citet{ahmed2011blame} and type
consistency by \citet{yuu2017poly}). We then discuss the design decisions
involved towards our new definition of consistent subtyping, and justify the new
definition by demonstrating its equivalence with that of \citet{siek2007gradual}
and the AGT approach~\cite{garcia2016abstracting} on simple types.

% \subsection{Language Overview}

% \begin{figure}[t]
%   \centering
%   \begin{small}
% \begin{tabular}{lrcl}
%   Expressions & $e$ & \syndef & $x \mid n \mid
%                          \blam x A e \mid e~e$ \\
% %%                         \mid \erlam x e \equiv \blam x \unknown e $ \\

%   Types & $A, B$ & \syndef & $ \nat \mid a \mid A \to B \mid \forall a. A \mid \unknown$ \\
%   Monotypes & $\tau, \sigma$ & \syndef & $ \nat \mid a \mid \tau \to \sigma$ \\

%   Contexts & $\dctx$ & \syndef & $\ctxinit \mid \dctx,x: A \mid \dctx, a$ \\
%   Syntactic Sugar & $\erlam x e$ & $\equiv$ & $\blam x \unknown e$ \\
%                   & $e : A$ & $\equiv$ & $(\blam x A x) ~ e$
% \end{tabular}
%   \end{small}
% \caption{Syntax of the declarative type system}
% \label{fig:decl-syntax}
% \end{figure}
%%\bruno{Do not use the notation for sugar in the lambda expression: it
%%  is confusing. Explain it in text.}


%  The syntax of our language is given in \Cref{fig:decl-syntax}.
% Compared
% with the Odersky-L{\"a}ufer type system, the only addition is the unknown type
% $\unknown$. We use the meta-variable $e$ to range over expressions. There are
% variables $x$, integers $n$, annotated lambda abstraction $\blam x A e$, and
% application $e_1 ~ e_2$. We write $A$, $B$ for types. They are the integer type
% $\nat$, type variables $a$, functions $A \to B$, universal quantification
% $\forall a. A$, and the unknown type $\unknown$. Monotypes $\tau$ contain all
% types other than the universal quantifier and the unknown type. Contexts $\dctx$
% map term variables to their types, and record all type variables with the
% expected well-formdness condition. Following \citet{siek2006gradual}, if a
% lambda binder is without annotation, it is automatically annotated with
% $\unknown$. As a convenience, the language also provides type ascription $e :
% A$, which is simulated as $(\blam x A x) ~ e$.

\subsection{Consistency and Subtyping}
\label{subsec:consistency-subtyping}

We start by giving the definitions of consistency and subtyping for polymorphic
types, and showing the differences of our definitions from other works, in
particular the compatibility relation by \citet{ahmed2011blame} and type
consistency by \citet{yuu2017poly}.

\begin{figure}[t]
  \begin{small}
  \begin{mathpar}
    \framebox{$A \sim B$} \\
    \CD \and \CA \and \CB \and \CC \and \CE
  \end{mathpar}

  \begin{mathpar}
    \framebox{$\dctx \bywf A $} \\
    \DeclVarWF \and \DeclIntWF \and \DeclUnknownWF \\ \DeclFunWF \and \DeclForallWF
  \end{mathpar}

  \begin{mathpar}
    \framebox{$\tpresub A \tsub B$} \\
    \HSForallR \and \HSForallL \and \HSFun \and
    \HSTVar \and \HSInt \and \HSUnknown
  \end{mathpar}
  \end{small}
  \caption{Consistency, well-formedness of types and subtyping in the declarative system.}
  \label{fig:decl:subtyping}
\end{figure}

\paragraph{Consistency}
The key observation here is that consistency is mostly a structural relation,
except that the unknown type $\unknown$ can be regarded as any type. Following
this observation, we naturally extend the definition from
\Cref{fig:objects} with polymorphic types, as shown at the top of
\Cref{fig:decl:subtyping}. In particular a polymorphic type $\forall a. A$
is consistent with another polymorphic type $\forall a. B$ if $A \sim B$.

\paragraph{Subtyping}

We express the fact that one type is a polymorphic generalization of another by
means of the subtyping judgment $\Psi \vdash A \tsub B$. Compared with the
subtyping rules of \citet{odersky1996putting} in
\Cref{fig:original-typing}, the only addition is the neutral subtyping of
$\unknown$, given at the bottom of \Cref{fig:decl:subtyping}. Notice
that in the rule
\rul{S-ForallL}, the universal quantifier is only allowed to be instantiated
with a \emph{monotype}. According to the syntax in \Cref{fig:original-typing},
monotypes do not contain the unknown type $\unknown$. This is because if we were
to allow the unknown type to be used for instantiation, we could have the
following subtyping relation
\[
  \forall a . a \to a \tsub \unknown \to \unknown
\]
by instantiating $a$ with $\unknown$. Since $\unknown \to \unknown$ is
consistent with any functions $A \to B$, for instance, $\nat \to \bool$, this
means that we could provide an expression of type $\forall a. a \to a$ to a
function where the input type is supposed to be $\nat \to \bool$. However, as we
might expect, $\forall a. a \to a$ is definitely not compatible with $\nat \to
\bool$. This does not hold in any polymorphic type systems without gradual
typing. So the gradual type system should not accept it either. (This is the
so-called \textit{conservative extension} property that will be made precise in
\Cref{sec:criteria}.)

Importantly there is a subtle but crucial distinction between a type variable
and the unknown type, although they all represent a kind of ``arbitrary'' type.
The unknown type stands for the absence of type information: it could be
\textit{any type} at \textit{any instance}. Because of the absence of type
information, the unknown type is consistent with any type, and additional
type checks may have to be performed at runtime. On the other hand, a type
variable denotes some instantiation of a universal quantifier, and is subject to
global constraints. In other words, a type variable can only be instantiated to
a single type. For example, in the type $\forall a. a \to a$, the two
occurrences of $a$ represent an arbitrary but single type (e.g., $\nat \to
\nat$, $\bool \to \bool$), while $\unknown \to \unknown$ could be an arbitrary
function (e.g., $\nat \to \bool$) at runtime.

\paragraph{Comparison with Other Relations}

In other polymorphic gradual calculi, consistency and subtyping are often mixed
up to some extent. In the Polymorphic Blame Calculus
(\pbc)~\citep{ahmed2011blame}, the compatibility relation for polymorphic types
is defined as follows:
\begin{mathpar}
  \CompAllR \and \CompAllL
\end{mathpar}
Notice that, in rule \rul{Comp-AllL}, the universal quantifier is \textit{always}
instantiated to $\unknown$. However, in this way, \pbc allows $\forall a. a \to a
\pbccons \nat \to \bool$, which as we discussed before might not be what we
expect. Indeed \pbc relies on sophisticated runtime checks to rule out such
instances of the compatibility relation \`a posteriori.

\citet{yuu2017poly} introduced the so-called
\textit{quasi-polymorphic} types for types that may be used where a
$\forall$-type is expected, which is important for their purpose of
conservativity over System F. Their type consistency relation, involving polymorphism, is
defined as follows\footnote{This is a simplified version.}:
\begin{mathpar}
  \inferrule{A \sim B }{\forall a. A \sim \forall a. B}
  \and
  \inferrule{A \sim B \\ B \neq \forall a. B' \\ \unknown \in \mathsf{Types}(B)}
  {\forall a. A \sim B}
\end{mathpar}
Compared with our consistency definition in \Cref{fig:decl:subtyping},
their first rule is the same as ours. The second rule says that a non
$\forall$-type can be consistent with a $\forall$-type only if it contains
$\unknown$. In this way, their type system is able to reject $\forall a. a \to a
\sim \nat \to \bool$. However, in order to keep conservativity, they also reject
$\forall a. a \to a \sim \nat \to \nat$, which is perfectly sensible in their
setting (i.e., explicit polymorphism). However with implicit polymorphism, we
would expect $\forall a. a \to a$ to be related with $\nat \to \nat$, which is
exactly the case in our subtyping relation since $a$ can be instantiated to
$\nat$.

Nonetheless, when it comes to interactions between dynamically typed and
polymorphically typed terms, both relations allow for example $(\forall a. a) \to
\nat$ to be related with $\unknown \to \nat$, which in our view, is some sort of
(implicit) polymorphic subtyping, and that should be achievable by the more
primitive notions in the type system (instead of inventing new relations). One
of our design principles is that, subtyping and consistency should be
\textit{orthogonal}, and can be naturally superimposed, echoing the same opinion
by \citet{siek2007gradual}.

\subsection{Towards Consistent Subtyping}
\label{subsec:towards-conssub}

With the definitions of consistency and subtyping, the question now is how to
compose these two relations so that two types can be compared in a way that takes
these two relations into account.

Unfortunately, the original definition of \citet{siek2007gradual}
(\Cref{def:old-decl-conssub}) does not work well with our definitions of
consistency and subtyping for polymorphic types. Consider two types: $(\forall
a. a) \to \nat$, and $\unknown \to \nat$. The first type can only reach the
second type in one way (first by applying consistency, then subtyping), but not the
other way, as shown in \Cref{fig:example:a}. We use $\bot$ to mean that we
cannot find such a type. Similarly, there are situations where the first type
can only reach the second type using the other way (first applying
subtyping, and then
consistency), as shown in \Cref{fig:example:b}.

\begin{figure}
  \begin{subfigure}[b]{.4\linewidth}
    \centering
      \begin{tikzpicture}
        \matrix (m) [matrix of math nodes,row sep=3em,column sep=4em,minimum width=2em]
        {
          \bot & \unknown \to \nat \\
          (\forall a. a) \to \nat & (\forall a. \unknown) \to \nat \\};

        \path[-stealth]
        (m-2-1) edge node [left] {$\tsub$} (m-1-1)
        (m-2-2) edge node [left] {$\tsub$} (m-1-2);

        \draw
        (m-1-1) edge node [above] {$\sim$} (m-1-2)
        (m-2-1) edge node [below] {$\sim$} (m-2-2);
      \end{tikzpicture}
      \caption{}
      \label{fig:example:a}
  \end{subfigure}
  \begin{subfigure}[b]{.4\linewidth}
    \centering
    \begin{tikzpicture}
      \matrix (m) [matrix of math nodes,row sep=3em,column sep=4em,minimum width=2em]
      {
        \nat \to \nat & \nat \to \unknown \\
        \forall a. a & \bot \\};

      \path[-stealth]
      (m-2-1) edge node [left] {$\tsub$} (m-1-1)
      (m-2-2) edge node [left] {$\tsub$} (m-1-2);

      \draw
      (m-1-1) edge node [above] {$\sim$} (m-1-2)
      (m-2-1) edge node [below] {$\sim$} (m-2-2);
    \end{tikzpicture}
    \caption{}
    \label{fig:example:b}
  \end{subfigure}
  \begin{subfigure}[b]{.8\linewidth}
    \centering
    \begin{tikzpicture}
      \matrix (m) [matrix of math nodes,row sep=3em,column sep=4em,minimum width=2em]
      {
        \bot &
        ((\unknown \to \nat) \to \bool) \to (\nat \to \unknown)  \\
        (((\forall a. a) \to \nat) \to \bool) \to (\forall a. a) &
        \bot \\};

      \path[-stealth]
      (m-2-1) edge node [left] {$\tsub$} (m-1-1)
      (m-2-2) edge node [left] {$\tsub$} (m-1-2);

      \draw
      (m-1-1) edge node [above] {$\sim$} (m-1-2)
      (m-2-1) edge node [below] {$\sim$} (m-2-2);
    \end{tikzpicture}
    \caption{}
    \label{fig:example:c}
  \end{subfigure}
  \caption{Examples that break the original definition of consistent subtyping.}
  \label{fig:example}
\end{figure}

What is worse, if those two examples are composed in a way that those types all
appear co-variantly, then the resulting types cannot reach each other by either
way. For example, \Cref{fig:example:c} shows such two types by putting a
$\bool$ type in the middle, and neither definition of consistent subtyping
works. But such types ought to be related somehow!

\paragraph{Observations on consistent subtyping}

In order to develop the correct definition of consistent subtyping for
polymorphic types, we need to understand how consistent subtyping works.
We first review two important properties of subtyping: 1) subtyping induces the
subsumption rule: if $A \tsub B$, then an expression of type $A$ can be used
where $B$ is expected; 2) subtyping is transitive: if $A \tsub B$, and $B \tsub
C$, then $A \tsub C$. Though consistent subtyping takes the unknown type into
consideration, the subsumption rule should also apply: if $A \tconssub B$, then
an expression of type $A$ can also be used where $B$ is expected, given that
there might be some information lost by consistency. A crucial difference from
subtyping is that consistent subtyping is \textit{not} transitive because
information can only be lost once (otherwise, any two types are a consistent
subtype of each other). Now consider a situation where we have both $A \tsub B$,
and $B \tconssub C$, this means that $A$ can be used where $B$ is expected, and
$B$ can be used where $C$ is expected, with possibly some loss of information. In
other words, we should expect that $A$ can be used where $C$ is expected, since
there is at most one-time loss of information.

\begin{observation}
  If $A \tsub B$, and $B \tconssub C$, then $A \tconssub C$.
\end{observation}

This is reflected in \Cref{fig:obser:a}. A similar and symmetrical
observation is given in \Cref{fig:obser:b}:

\begin{observation}
  If $C \tconssub B$, and $B \tsub A$, then $C \tconssub A$.
\end{observation}

\begin{figure}[t]
  \centering
  \begin{subfigure}[b]{.4\linewidth}
    \centering
    \begin{tikzpicture}
      \matrix (m) [matrix of math nodes,row sep=3em,column sep=4em,minimum width=2em]
      {
        T_1 & C \\
        B   & T_2 \\
        A & \\};

      \path[-stealth]
      (m-3-1) edge node [left] {$\tsub$} (m-2-1)
      (m-2-2) edge node [left] {$\tsub$} (m-1-2)
      (m-2-1) edge node [left] {$\tsub$} (m-1-1);

      \draw
      (m-2-1) edge node [above] {$\sim$} (m-2-2)
      (m-1-1) edge node [above] {$\sim$} (m-1-2);

      \draw [dashed, ->]
      (m-2-1) edge node [above] {$\tconssub$} (m-1-2);

      \path [dashed, ->, out=0, in=0, looseness=2]
      (m-3-1) edge node [right] {$\tconssub$} (m-1-2);
    \end{tikzpicture}
    \caption{}
    \label{fig:obser:a}
  \end{subfigure}
  \centering
  \begin{subfigure}[b]{.4\linewidth}
    \centering
    \begin{tikzpicture}
      \matrix (m) [matrix of math nodes,row sep=3em,column sep=4em,minimum width=2em]
      {
        & A \\
        T_1 & B \\
        C   & T_2 \\};

      \path[-stealth]
      (m-3-1) edge node [left] {$\tsub$} (m-2-1)
      (m-3-2) edge node [left] {$\tsub$} (m-2-2)
      (m-2-2) edge node [left] {$\tsub$} (m-1-2);

      \draw
      (m-2-1) edge node [above] {$\sim$} (m-2-2)
      (m-3-1) edge node [below] {$\sim$} (m-3-2);

      \draw [dashed, ->]
      (m-3-1) edge node [above] {$\tconssub$} (m-2-2);

      \path [dashed, ->, out=135, in=180, looseness=2]
      (m-3-1) edge node [left] {$\tconssub$} (m-1-2);
    \end{tikzpicture}
    \caption{}
    \label{fig:obser:b}
  \end{subfigure}
  \caption{Observations of consistent subtyping}
  \label{fig:obser}
\end{figure}


From the above observations, we can see what the problem is with the original
definition. In \Cref{fig:obser:a}, if $B$ can reach $C$ by $T_1$, then
according to the transitivity of subtyping, $A$ can reach $C$ by $T_1$. However,
if $B$ can only reach $C$ by $T_2$, then $A$ cannot reach $C$ through the
original definition. A similar problem is shown in \Cref{fig:obser:b}: if $C$ can
only reach $B$ by $T_1$, then $C$ cannot reach $A$ through the original definition.

However, it turns out that those two problems can be fixed by the same strategy:
instead of taking one-step subtyping and one-step consistency, our definition of
consistent subtyping allows types to take one-step subtyping, one-step
consistency, and one more step subtyping. Specifically, $A \tsub B \sim T_2 \tsub C$
and $C \tsub T_1 \sim B \tsub A$ have the same relation chain: subtyping,
consistency, and subtyping.

\paragraph{Definition of consistent subtyping} From the above discussion, we are
ready to modify \Cref{def:old-decl-conssub}, and adapt it to our notation:

\begin{mdef}[Consistent Subtyping]
  \label{def:decl-conssub}
  $\tpresub A \tconssub B$, if and only if $\tpresub A \tsub C$, $C \sim D$, and
  $\tpresub D \tsub B$ for
  some $C, D$.
\end{mdef}

\noindent With \Cref{def:decl-conssub}, \Cref{fig:example:c:fix}
illustrates the correct relation chain for the broken example shown in
\Cref{fig:example:c}.

At first sight, \Cref{def:decl-conssub}
seems worse than the original: we need to guess \textit{two} types! It turns out
that \Cref{def:decl-conssub} is a generalization of
\Cref{def:old-decl-conssub}, and they are equivalent in the system by
\citet{siek2007gradual}. We argue that this is the \textit{general} definition of
consistent subtyping, independent of language features, and in particular is
compatible with polymorphic types.


\begin{figure}[t]
  \centering
  \begin{tikzpicture}
    \matrix (m) [matrix of math nodes,row sep=3em,column sep=4em,minimum width=2em]
    {
      ((\forall a. a) \to \nat) \to \bool) \to (\nat \to \nat) &
      ((\forall a. \unknown) \to \nat) \to \bool) \to (\nat \to \unknown)\\
      (((\forall a. a) \to \nat) \to \bool) \to (\forall a. a) &
      ((\unknown \to \nat) \to \bool) \to (\nat \to \unknown)  \\
      };

    \path[-stealth]
    (m-2-1) edge node [left] {$\tsub$} (m-1-1)
    (m-1-2) edge node [left] {$\tsub$} (m-2-2)
    (m-2-1) edge node [above] {$\tconssub$} (m-2-2);

    \draw
    (m-1-1) edge node [above] {$\sim$} (m-1-2);
  \end{tikzpicture}
  \caption{Example that is fixed by the new definition of consistent subtyping.}
  \label{fig:example:c:fix}
\end{figure}


\begin{mprop}\leavevmode
  \label{prop:subsumes}
\begin{itemize}
  \item \Cref{def:decl-conssub} subsumes
    \Cref{def:old-decl-conssub}:
    in \Cref{def:decl-conssub},
    by choosing $D=B$, we have $A\tsub C$ and $C \sim B$; by choosing $C=A$, we have
    $A \sim D$, and $D \tsub B$.
  \item \Cref{def:old-decl-conssub} is equivalent to
    \Cref{def:decl-conssub} in the system by~\citet{siek2007gradual}:
    if $A \tsub C$, $C \sim D$, and $D \tsub
    B$, by \Cref{def:old-decl-conssub},
    we have $A \sim C'$, $C' \tsub D$ for some $C'$. By subtyping
    transitivity, we have $C' \tsub B$. So we have $A \tconssub B$ by $A \sim C'$, and $C'
    \tsub B$.
  \end{itemize}
\end{mprop}

\subsection{Consistent Subtyping Without Existentials}

\Cref{def:decl-conssub} serves as a fine specification of how consistent
subtyping should behave in general. But it is inherently non-deterministic
because of the two intermediate types $C$ and $D$. As with
\citet{siek2007gradual}'s definition,
we need a combined relation to directly compare two types. A first,
and natural attempt is to try to extend the restriction operator for
polymorphic types. 
Unfortunately this does not work, but it is possible to devise a
simple and elegant inductive definition instead.

\paragraph{Attempt on extending the restriction operator}
Suppose we try to extend the restriction operator to account for polymorphic
types. The original restriction operator is structural, meaning that it works
for types of similar structures. But for polymorphic types, two input types
could have different structures due to universal quantifiers, e.g, $\forall a. a
\to \nat$ and $(\nat \to \unknown) \to \nat$. If we try to mask the first type
using the second, it seems hard to maintain the information that $a$ should be
instantiated to a function while ensuring that the return type is masked. There
seems to be no satisfactory way to extend the restriction operator in order to
support this kind of non-structural masking.

\paragraph{Interpretation of the restriction operator and consistent subtyping}
If the restriction operator cannot be extended naturally, it is useful to
take a step back and revisit what the restriction operator actually does. For
consistent subtyping, two input types could have unknown types in different
positions, but we only care about the known parts. To do that, the restriction
operator is used to: 1) erase the type information in one type if the corresponding
position in the other type is the unknown type; and 2) compare the resulting types 
using the normal subtyping relation. The example below shows the
masking-off procedure for the types $\nat \to \unknown \to \bool$ and $\nat \to
\nat \to \unknown$. Since the known parts have the relation that $\nat \to
\unknown \to \unknown \tsub \nat \to \unknown \to \unknown$, we conclude that
$\nat \to \unknown \to \bool \tconssub \nat \to \nat \to \unknown$.
\begin{center}
  \begin{tikzpicture}
    \tikzstyle{column 5}=[anchor=base west, nodes={font=\tiny}]
    \matrix (m) [matrix of math nodes,row sep=1em,column sep=0em,minimum width=2em]
    {
      \nat \to & \unknown & \to & \bool & \mid \nat \to \nat \to \unknown &
      = \nat \to \unknown \to \unknown
      \\
       \nat \to & \nat & \to & \unknown & \mid \nat \to \unknown \to \bool &
      = \nat \to \unknown \to \unknown \\};

    \path[-stealth, ->, out=0, in=0]
    (m-1-6) edge node [right] {$\tsub$} (m-2-6);

    \draw
    (m-1-2.north west) rectangle (m-2-2.south east)
    (m-1-4.north west) rectangle (m-2-4.south east);
  \end{tikzpicture}
\end{center}
Here differences of the types in boxes are erased because of the
restriction operator. Now if we compare the types in boxes directly instead of
through the lens of the restriction operator, we can observe that the
\textit{consistent subtyping relation always holds between the unknown type and
  an arbitrary type.} We can interpret this observation directly using
\Cref{def:decl-conssub}: the unknown type is neutral to subtyping
($\unknown \tsub \unknown$), the unknown type is consistent with any type
($\unknown \sim A$), and subtyping is reflexive ($A \tsub A$). Therefore,
\textit{the unknown type is a consistent subtype of any type ($\unknown
  \tconssub A$), and vice versa ($A \tconssub \unknown$).}

\paragraph{Defining consistent subtyping directly}

From the above discussion, we can define the consistent subtyping relation
directly, \textit{without} resorting to subtyping or consistency at all. The key
idea is that we replace $\tsub$ with $\tconssub$ in
\Cref{fig:decl:subtyping}, get rid of rule \rul{S-Unknown} and add two
extra rules concerning $\unknown$, resulting in the rules of consistent
subtyping in \Cref{fig:decl:conssub}. Of particular interest are the rules
\rul{CS-UnknownL} and \rul{CS-UnknownR}, both of which correspond to what we
just said: the unknown type is a consistent subtype of any type, and vice versa.
\begin{figure}[t]
  \begin{small}
  \begin{mathpar}
    \framebox{$\tpresub A \tconssub B$} \\
    \CSForallR \and \CSForallL \and \CSFun \and
    \CSTVar \and \CSInt \and \CSUnknownL \and \CSUnknownR
  \end{mathpar}
  \end{small}
  \caption{Consistent Subtyping for implicit polymorphism.}
  \label{fig:decl:conssub}
\end{figure}
From now on, we use the symbol $\tconssub$ to refer to the consistent subtyping
relation in \Cref{fig:decl:conssub}, instead of the one in
\Cref{def:decl-conssub}. What is more, we can prove that those two are
equivalent\footnote{Note to reviewers: Theorems with $\mathcal{T}$ are those
  proved in Coq. The same applies to $\mathcal{L}$emmas.}:

\begin{ctheorem} The following definitions are equivalent:
  \label{lemma:properties-conssub}
  \begin{itemize}
  \item  $\tpreconssub A \tconssub B$.
  \item  $\tpresub A \tsub C$, $C \sim D$, $\tpresub D \tsub B$, for some $C, D$.
  \end{itemize}
\end{ctheorem}

\noindent Not surprisingly, consistent subtyping is reflexive:

\begin{clemma}[Consistent Subtyping is Reflexive] \label{lemma:sub_refl}%
  If $\Psi \vdash A$ then $\Psi \vdash A \tconssub A$.
\end{clemma}

\subsection{Abstracting Gradual Typing}
\label{subsec:agt}

\citet{garcia2016abstracting} presented a new foundation for gradual typing that
they call the \textit{Abstracting Gradual Typing} (AGT) approach. In the AGT
approach, gradual types are interpreted as sets of static types, where static
types refer to types containing no unknown types. In this interpretation,
predicates and functions on static types can then be lifted to apply to gradual
types. Central to their approach is the so-called \textit{concretization}
function. For simple types, a concretization $\gamma$ from gradual types to a
set of static types\footnote{For simplification, we directly regard type
  constructor $\to$ as a set-level operator.} is defined as follows:
\begin{mdef}[Concretization]
  \label{def:concret}
  \begin{mathpar}
    \gamma(\nat) = \{\nat\} \and \gamma(A \to B) = \gamma(A) \to \gamma(B) \and
    \gamma(\unknown) = \{\text{All static types}\}
  \end{mathpar}
\end{mdef}

Based on the concretization function, subtyping between static types can be
lifted to gradual types, resulting in the consistent subtyping relation:
\begin{mdef}[Consistent Subtyping in AGT]
  \label{def:agt-conssub}
  $A \agtconssub B$ if and only if $A_1 \tsub B_1$ for some $A_1 \in \gamma(A)$, $B_1 \in \gamma(B)$.
\end{mdef}

\noindent Later they proved that this definition of consistent subtyping coincides with
that of \citet{siek2007gradual} (\Cref{def:old-decl-conssub}).

It seems that the AGT approach of consistent subtyping is quite different from
ours: theirs is defined purely in terms of static subtyping; we directly define
consistent subtyping on gradual types (\Cref{fig:decl:conssub}).
Nonetheless, the two approaches coincide on \textit{simple types}:

\begin{mprop}[Equivalence to AGT on Simple Types]
  \label{lemma:coincide-agt}
  $A \tconssub B$ if only if $A \agtconssub B$.
\end{mprop}
% \begin{proof}\leavevmode
%   \begin{itemize}
%   \item From left to right: By induction on the derivation of consistent
%     subtyping. In cases \rul{CS-UnknownL} and \rul{CS-UnknownR}, since the
%     static set of $\unknown$ contains all static types, it follows that for
%     every static type $A_1 \in \gamma(A)$, we can always find $A_1 \in
%     \gamma(\unknown)$, and by the reflexivity of subtyping, we are done. The
%     rest are trivial cases.
%   \item From right to left: By induction on the derivation of subtyping and
%     inversion on the concretization. If $A$ or $B$ is a unknown type, then
%     consistent subtyping directly holds. Other cases are trivial.
%   \end{itemize}
% \end{proof}

% Further more, the coincidence between our definition and AGT on consistent
% subtyping for objects
% can be proved by showing both are coincided with
% \citet{siek2007gradual}.

This proposition is rather trivial:
in \Cref{fig:decl:conssub},
rule \rul{CS-UnknownL} and \rul{CS-UnknownR} correspond directly to the
concretization of $\unknown$, which contains all static types. Nonetheless,
this reveals two points. First, it validates our definition of consistent
subtyping by another interpretation. Second, as noted by
\citet{garcia2016abstracting}, it shows that consistent subtyping can be derived
from two quite different foundations: one is defined directly on gradual types,
the other is defined purely in terms of static subtyping.

However, as noted by \citet{garcia2016abstracting} in the conclusion, extending
AGT to deal with polymorphism still remains as an open question. The difficulty
possibly stems from the fundamental conflicts between the set-theoretic
interpretation of gradual types and the parametric interpretation of
polymorphic types~\cite{reynolds1983types} (in particular, the interpretation of
type variables). This shows one advantage of our approach in that it is
independent of other concepts. Still, it is a promising line of future work for
AGT, and the question remains whether our definition would coincide with it.




%%% Local Variables:
%%% mode: latex
%%% TeX-master: "../paper"
%%% org-ref-default-bibliography: "../paper.bib"
%%% End:
\section{A Type System with Gradually Typed Implicit Polymorphism}
\label{sec:type-system}

In \Cref{sec:exploration} we have introduced the consistent
subtyping relation that naturally extends to polymorphic types. In
this section we continue with the development by giving a declarative
type system for implicit polymorphism that employs the consistent
subtyping relation. The declarative system itself is already quite
interesting as it is equipped with both higher-rank polymorphism and
the unknown type. Moreover, unlike non-gradual type systems with
higher-rank polymorphism, guessed types affect runtime behaviour if
used by the implicit casts, which raises concerns with respect to
coherency. Our response to those concerns is given in \Cref{subsec:algo:discuss},
after we give a simple
algorithm that implements the declarative system
(\Cref{sec:algorithm}) and discuss soundness and completeness.

% Later in \Cref{sec:algorithm} we give a simple
%algorithm that implements the declarative system.

\subsection{Language Overview}

\begin{figure}[t]
  \centering
  \begin{small}
\begin{tabular}{lrcl} \toprule
  Expressions & $e$ & \syndef & $x \mid n \mid
                         \blam x A e \mid e~e$ \\
%%                         \mid \erlam x e \equiv \blam x \unknown e $ \\

  Types & $A, B$ & \syndef & $ \nat \mid a \mid A \to B \mid \forall a. A \mid \unknown$ \\
  Monotypes & $\tau, \sigma$ & \syndef & $ \nat \mid a \mid \tau \to \sigma$ \\

  Contexts & $\dctx$ & \syndef & $\ctxinit \mid \dctx,x: A \mid \dctx, a$ \\
  Syntactic Sugar & $\erlam x e$ & $\equiv$ & $\blam x \unknown e$ \\
              & $e : A$ & $\equiv$ & $(\blam x A x) ~ e$ \\ \bottomrule
\end{tabular}
  \end{small}
\caption{Syntax of the declarative type system}
\label{fig:decl-syntax}
\end{figure}

The complete syntax of the declarative system is given in
\Cref{fig:decl-syntax}. We use the meta-variable $e$ to range over expressions.
Expressions are either variables $x$, integers $n$, annotated lambda
abstractions $\blam x A e$, or applications $e_1 ~ e_2$. We write $A$, $B$ for
types. Types are either the integer type $\nat$, type variables $a$, functions
types $A \to B$, universal quantification $\forall a. A$, or the unknown type
$\unknown$. Though we only have one base type $\nat$, we also use $\bool$ for
the purpose of illustration. Monotypes $\tau$ contain all types other than the
universal quantifier and the unknown type. Contexts $\dctx$ map term variables
to their types, and record all type variables with the expected well-formedness
condition. Following \citet{siek2006gradual}, if a lambda binder does not have
an annotation, it is automatically annotated with $\unknown$. As a convenience,
the language also provides type ascription $e : A$, which is simulated by
$(\blam x A x) ~ e$.

\subsection{Typing in Detail}

\Cref{fig:decl-typing} gives the typing rules for our declarative system
(the reader is advised to ignore the gray-shaded parts for now). Rule \rul{Var}
extracts the type of the variable from the typing context. Rule \rul{Nat} always
infers integer types. Rule \rul{LamAnn} puts $x$ with type annotation $A$ into
the context, and continues type checking the body $e$. Rule \rul{App} first
infers the type of $e_1$, then the matching judgment $\tprematch A \match A_1
\to A_2$ extracts the domain type $A_1$ and the codomain type $A_2$ from type
$A$. The type $A_3$ of the argument $e_2$ is then compared with $A_1$ using the
consistent subtyping judgment.

\renewcommand{\trto}[1]{\hlmath{\rightsquigarrow{#1}}}
\begin{figure}[t]
  \begin{small}
  \begin{mathpar}
    \framebox{$\tpreinf e : A \trto s$} \\
    \DVar \and \DNat \and \DLamAnnA \and \DApp
  \end{mathpar}

  \begin{mathpar}
    \framebox{$\tprematch A \match A_1 \to A_2$} \\
    \MMC \\ \MMA \and \MMB
  \end{mathpar}

  \end{small}
  \caption{Declarative typing}
  \label{fig:decl-typing}
\end{figure}

\paragraph{Matching} It turns out that matching~\cite{siek2015refined} can be
extended to polymorphic types naturally. In \rul{M-Forall}, a monotype $\tau$ is
guessed to instantiate the universal quantifier $a$. This natural extension is
also inspired by the \textit{application judgment} $\tpreinf A \bullet e \infto
C$ by \citet{dunfield2013complete}, which says that if we apply a term of type
$A$ to an argument $e$, we get something of type $C$. If $A$ is a polymorphic
type, the judgment works by guessing instantiations of polymorphic quantifiers
until it reaches an arrow type. Rule \rul{M-Arr} and \rul{M-Unknown} are the
same as \citet{siek2015refined}.


\renewcommand{\trto}[1]{\rightsquigarrow{#1}}
\subsection{Type-directed Translation}
\label{sec:type:trans}

We give the dynamic semantics of our language by translating it to
\pbc~\cite{ahmed2011blame}. Below we show a subset of the terms in \pbc that are
used in the translation:
\[
  \text{Terms}\quad s ::= x \mid n \mid \blam x A s \mid s~s \mid \cast A B s
\]
A cast $\cast A B {s}$ converts the value of term $s$ from type $A$ to type $B$.
A cast from $A$ to $B$ is permitted only if the types are \textit{compatible},
written $A \pbccons B$, as briefly mentioned in
\Cref{subsec:consistency-subtyping}. The syntax of types in \pbc is the
same as ours.

The translation is given in the gray-shaded parts in \Cref{fig:decl-typing}. The
only interesting case here is to insert explicit casts in the application rule.
Note that there is no need to translate matching or consistent subtyping,
instead we insert the source and target types of a cast directly in the
translated expressions, thanks to the following two lemmas:

\begin{clemma}[Compatibility of Matching]
  \label{lemma:comp-match}
  If $\tprematch A \match A_1 \to A_2$, then $A \pbccons A_1 \to A_2$.
\end{clemma}

\begin{clemma}[Compatibility of Consistent Subtyping]
  \label{lemma:comp-conssub}
  If $\tpreconssub A \tconssub B$, then $A \pbccons B$.
\end{clemma}

In order to show the correctness of the translation, we prove that our
translation always produces well-typed expressions in \pbc. By
\Cref{lemma:comp-match,lemma:comp-conssub}, we have the following theorem:

\begin{ctheorem}[Type Safety]
  \label{lemma:type-safety}
  If $\tpreinf e : A \trto s$, then $\dctx \bypinf s : A$.
\end{ctheorem}

\paragraph{Parametricity} An important semantic property of polymorphic types is
\textit{relational parametricity}~\cite{reynolds1983types}. The parametricity
property says that all instances of a parametrically polymorphic function should
behave \textit{uniformly}. In other words, functions cannot inspect into a type
variable, and act differently for different instances of the type variable. A
classic example is a function with the type $\forall a . a \to a$. The
parametricity property guarantees that a value of this type must be either the
identity function (i.e., $\lambda x . x$) or the undefined function (one which
never returns a value). However, with the addition of the unknown type
$\unknown$, careful measures are to be taken to ensure parametricity. This is
exactly the circumstance that \pbc was designed to address. \citet{amal2017blame}
proved that \pbc satisfies relational parametricity. Based on their result, and
by \Cref{lemma:type-safety}, parametricity is preserved in our system.

\paragraph{Guessed types affect runtime behaviour}

However, the translation does not always produce a unique target expression.
This is because when we guess a monotype $\tau$ in rule \rul{M-Forall} and
\rul{CS-ForallL}, we could have different choices, which inevitably leads to
different types. Unlike (non-gradual) polymorphic type systems
\citep{jones2007practical, dunfield2013complete}, the guessed types affect
runtime behaviour of the translated programs, since they could appear inside the
explicit casts. For example, the following shows two possible translations for
the same source expression $\blam x \unknown {f ~ x}$, where $f$ is
instantiated to $\nat \to \nat$ and $\bool \to \bool$, respectively:
\begin{align*}
  f: \forall a. a \to a &\byinf (\blam x \unknown {f ~ x})
                          : \unknown \to \nat \\
                          &\trto (\blam x \unknown (\cast {\forall a. a \to a} {\nat \to \nat} f) ~
                          (\hlmath{\cast \unknown \nat} x))
  \\
  f: \forall a. a \to a &\byinf (\blam x \unknown {f ~ x})
                          : \unknown \to \bool \\
                          &\trto (\blam x \unknown (\cast {\forall a. a \to a} {\bool \to \bool} f) ~
                          (\hlmath{\cast \unknown \bool} x))
\end{align*}
If we apply $\blam x \unknown {f ~ x}$ to $3$ for example, which should be fine
since the function can take any input, the first translation runs smoothly in
\pbc, while the second one will raise a cast error ($\nat$ cannot be cast to
$\bool$). Similarly, if we apply it to $\truee$, then the second succeeds while
the first fails. The culprit lies in the highlighted parts where any
instantiation of $a$ would be put inside the explicit cast. More generally, any
choice introduces an explicit cast to that type in the translation, which causes
a runtime cast error if the function is applied to a value whose type does not
match the guessed type. Note that this does not compromise the type safety of
the translated expressions, since cast errors are part of the type safety
guarantees.

\paragraph{Coherency}

The ambiguity of translation seems to imply that the
declarative is \textit{incoherent}. Coherence is a desired
property for a semantics. A semantic is coherent if any \textit{valid program}
has exactly one meaning~\cite{Reynolds_coherence}. We argue that the declarative
system is still coherent in the sense that if a program produces a value, this
value is unique. In the above example, whatever the translation might be,
applying $\blam x \unknown {f ~ x}$ to $3$ either results in a cast error, or
produces $3$, and not any other values.

This discrepancy is due to the guessing nature of the \textit{declarative}
system. As far as the declarative system is concerned, both $\nat \to \nat$ and
$\bool \to \bool$ are equally acceptable. But this is not the case at runtime.
The acute reader may have found that the \textit{only} appropriate choice is to
instantiate $f$ to $\unknown \to \unknown$. However, as specified by rule
\rul{M-Forall} in \Cref{fig:decl-typing}, we can only instantiate type variables
to monotypes, but $\unknown$ is \textit{not} a monotype! We will get back to
this issue in \Cref{subsec:algo:discuss} after we present the corresponding
algorithmic system in \Cref{sec:algorithm}.


\subsection{Correctness Criteria}
\label{sec:criteria}

\citet{siek2015refined} present a set of properties that a well-designed gradual
typing calculus must have, which they call refined criteria. Among all the
criteria, those related to the static aspects of gradual typing are well
summarized by \citet{cimini2016gradualizer}. Here we review those criteria and
adapt them to our notation. We have proved in Coq that our type system satisfies
all of these criteria.

\begin{clemma}[Correctness Criteria]\leavevmode
  \begin{itemize}
  \item \textbf{Conservative extension:}
    for all static $\dctx$, $e$, and $A$,
    $\dctx \byhinf e : A $ if and only if $\dctx \byinf e : A$.
  \item \textbf{Monotonicity w.r.t. precision:}
    for all $\dctx, e, e', A$,
    if $\dctx \byinf e : A$,
    and $e' \lessp e$,
    then $\dctx \byinf e' : B$,
    and $B \lessp A$ for some B.
  \item \textbf{Type Preservation of cast insertion:}
    for all $\dctx, e, A$,
    if $\dctx \byinf e : A$,
    then $\dctx \byinf e : A \trto s$,
    and $\dctx \bypinf s : A$ for some $s$.
  \item \textbf{Monotonicity of cast insertion:}
    for all $\dctx, e_1, e_2, e_1', e_2', A$,
    if $\dctx \byinf e_1 : A \trto e_1'$,
    and $\dctx \byinf e_2 : A \trto e_2'$,
    and $e_1 \lessp e_2$,
    then $\dctx \ctxsplit \dctx \bylessp e_1' \lesspp e_2'$.
  \end{itemize}
\end{clemma}

\begin{figure}[t]
  \begin{small}
  \begin{mathpar}
    \framebox{$A \lessp B$}{\quad \text{Type precision}} \\
    \LUnknown \and \LNat \and \LArrow \and \LTVar
    \and \LForall
  \end{mathpar}

  \begin{mathpar}
    \framebox{$e_1 \lessp e_2$}{\quad \text{Term precision}} \\
    \LRefl \and \LAbsAnn \and \LApp
  \end{mathpar}

  \begin{mathpar}
    \framebox{$\dctx_1 \ctxsplit \dctx_2 \bylessp e_1 \lesspp e_2$}
    {\quad \text{Term less precision in \pbc}} \\
    \LVar \and \LNatP \and \LAbsAnnP \and
    \LAppP \and \LCast \and \LCastL \and
    \LCastR
  \end{mathpar}
  \end{small}
  \caption{Less Precision}
  \label{fig:lessp}
\end{figure}


The first criterion states that the gradual type system should be a conservative
extension of the original system (i.e., the Odersky-L{\"a}ufer type system in
our case). In other words, a \textit{static} program that is typeable in the
original type system should remain typeable in the gradual type
system. A static program is one that does not contain any type $\unknown$. It also
ensures that ill-typed programs of the original language remain so in the
gradual type system.

The second criterion states that if a typeable expression loses some type
information, it remains typeable. This criterion depends on the definition of
the precision relation, written $A \lessp B$, which is given in the top of
\Cref{fig:lessp}. The relation intuitively captures a notion of types containing
more or less unknown types ($\unknown$). The precision relation over types lifts
to programs, i.e., $e_1 \lessp e_2$ means that $e_1$ and $e_2$ are the same
program except that $e_2$ has more unknown types.

The first two criteria are fundamental to gradual typing. They explain for
example why these two programs $(\blam x \nat {x + 1})$ and $(\blam x \unknown
{x + 1})$ are typeable, as the former is typeable in the Odersky-L{\"a}ufer type
system and the latter is a less-precise version of it.

The last two criteria relate to the compilation to the cast calculus. The
third criterion is essentially the same as \Cref{lemma:type-safety}, given that
a target expression should always exist, which can be easily seen from
\Cref{fig:decl-typing}. The last criterion ensures that the translation
must be monotonic over the precision relation $\lessp$. (The definition of the
precision relation $\lesspp$ for \pbc is found in the bottom of
\Cref{fig:lessp}.)


%%% Local Variables:
%%% mode: latex
%%% TeX-master: "../paper"
%%% org-ref-default-bibliography: "../paper.bib"
%%% End:
\section{Algorithmic Type System}
\label{sec:algorithm}

\begin{figure}[t]
  \centering
  \begin{small}
\begin{tabular}{lrcl} \toprule
  Expressions & $e$ & \syndef & $x \mid n \mid
                         \blam x A e \mid \erlam x e \mid e~e \mid e : A $ \\
  Types & $A, B$ & \syndef & $ \nat \mid a \mid \genA \mid A \to B \mid \forall a. A \mid \unknown$ \\
  Monotypes & $\tau, \sigma$ & \syndef & $ \nat \mid a \mid \genA \mid \tau \to \sigma$ \\
  Contexts & $\Gamma, \Delta, \Theta$ & \syndef & $\ctxinit \mid \tctx,x: A \mid \tctx, a \mid \tctx, \genA \mid \tctx, \genA = \tau$ \\
  Complete Contexts & $\Omega$ & \syndef & $\ctxinit \mid \Omega,x: A \mid \Omega, a \mid \Omega, \genA = \tau$ \\ \bottomrule
\end{tabular}
  \end{small}
\caption{Syntax of the algorithmic system}
\label{fig:algo-syntax}
\end{figure}


% The declarative type system in \cref{sec:type-system} serves as a good
% specification for how typing should behave. It remains to see whether this
% specification delivers an algorithm. The main challenge lies in the rules \rul{CS-ForallL} in
% \cref{fig:decl:conssub} and rule \rul{M-Forall} in
% \cref{fig:decl-typing}, which both need to guess a monotype.

% \bruno{why are we not highlightinh the differences in gray anymore?}
In this section we give a bidirectional account of the algorithmic type system
that implements the declarative specification. The algorithm is largely inspired
by the algorithmic bidirectional system of \citet{dunfield2013complete}
(henceforth DK system). However our algorithmic system differs from theirs in
three aspects: 1) the addition of the unknown type $\unknown$; 2) the use of the
matching judgment; and 3) the approach of \textit{gradual inference only
  producing static types}~\citep{garcia2015principal}. We then prove that our
algorithm is both sound and complete with respect to the declarative type
system. Full proofs can be found in the appendix.

\paragraph{Algorithmic Contexts.}

The algorithmic context $\Gamma$ is an
\textit{ordered} list containing declarations of type variables $a$ and term
variables $x : A$. Unlike declarative contexts, algorithmic contexts also
contain declarations of existential type variables $\genA$, which can be either
unsolved (written $\genA$) or solved to some monotype (written $\genA = \tau$).
Complete contexts $\Omega$ are those that contain no unsolved existential type
variables. \Cref{fig:algo-syntax} shows the syntax of the algorithmic system.
Apart from expressions in the declarative system, we have annotated expressions
$e : A$.

% \paragraph{Notational convenience}
% Following \citet{dunfield2013complete}, we use contexts as substitutions on
% types. We write $\ctxsubst{\Gamma}{A}$ to mean $\Gamma$ applied as a
% substitution to type $A$. We also use a hole notation, which is useful when
% manipulating contexts by inserting and replacing declarations in the middle. The
% hole notation is used extensively in proving soundness and completeness. For
% example, $\Gamma[\Theta]$ means $\Gamma$ has the form $\Gamma_L, \Theta,
% \Gamma_R$; if we have $\Gamma[\genA] = (\Gamma_L, \genA, \Gamma_R)$, then
% $\Gamma[\genA = \tau] = (\Gamma_L, \genA = \tau, \Gamma_R)$.

% \paragraph{Input and output contexts}
% The algorithmic system, compared with the declarative system, includes similar
% judgment forms, except that we replace the declarative context $\Psi$ with an
% algorithmic context $\Gamma$ (the \textit{input context}), and add an
% \textit{output context} $\Delta$ after a backward turnstile. For example,
% $\Gamma \vdash A \tconssub B \dashv \Delta$ is the judgment form for the
% algorithmic consistent subtyping, and so on. All rules manipulate input and
% output contexts in a way that is consistent with the notion of \textit{context
%   extension}, which is described in \cref{sec:ctxt:extension}.

% We start with the explanation of the algorithmic consistent subtyping as it
% involves manipulating existential type variables explicitly (and solving them if
% possible).

\subsection{Algorithmic Consistent Subtyping and Instantiation}
\label{sec:algo:subtype}

\begin{figure}[t]
  \centering
  \begin{small}
  %   \begin{mathpar}
  % \framebox{$\Gamma \vdash A$} \\
  % \VarWF \and \IntWF \and \UnknownWF \and \FunWF \and \ForallWF \and \EVarWF
  % \and \SolvedEVarWF
  %   \end{mathpar}

\begin{mathpar}
  \framebox{$\Gamma \vdash A \tconssub B \toctxr$} \\
  \ACSTVar \and \ACSExVar \and \ACSInt \quad \ACSUnknownL \quad \ACSUnknownR \and
  \ACSFun \and \ACSForallR \and \ACSForallL \and \AInstantiateL \quad \AInstantiateR
\end{mathpar}
  \end{small}
  \caption{Algorithmic consistent subtyping}
  \label{fig:algo:subtype}
\end{figure}

\Cref{fig:algo:subtype} shows the algorithmic consistent subtyping rules.
The first five rules do not manipulate contexts. % Rules \rul{ACS-TVar} and
% \rul{ACS-Int} do not involve existential variables, so the output context
% remains unchanged. Rule \rul{ACS-ExVar} says that any unsolved existential
% variable is a consistent subtype of itself. The output is still the same as the
% input context as this gives no clue as to what is the solution of that
% existential variable.
% Rules \rul{ACS-UnknownL} and \rul{ACS-UnknownR} are the verbatim
% correspondences of rule \rul{CS-UnknownL} and \rul{CS-UnknownR}.
Rule \rul{ACS-Fun} is a natural extension of its declarative counterpart. The
output context of the first premise is used by the second premise, and the
output context of the second premise is the output context of the conclusion.
Note that we do not simply check $A_2 \tconssub B_2$, but apply $\Theta$
% (the input context of the second premise)
to both types (e.g., $\ctxsubst{\Theta}{A_2} $). This is
to maintain an important invariant that types
% : whenever we try to derive $\Gamma \vdash A \tconssub B \dashv \Delta$, the types $A$ and $B$
are fully applied
under input context $\Gamma$ (they contain no existential variables already solved in
$\Gamma$). The same invariant applies to every algorithmic judgment.
Rule \rul{ACS-ForallR} looks similar to its declarative counterpart, except that
we need to drop the trailing context $a, \Theta$ from the concluding output
context since they become out of scope.
% again, bears a similarity with the declarative
% version. Note that the output context of its premise allows additional elements
% to appear after the type variable $a$, in a trailing context $\Theta$. Since $a$
% becomes out of scope in the conclusion, we need to drop the trailing context
% $\Theta$ together with $a$ from the concluding output context, resulting in
% $\Delta$.
% The next rule is essential to eliminating the guessing work, thus appears
% significantly different from its declarative version. Instead of guessing a
% monotype $\tau$ out of thin air,
Rule \rul{ACS-ForallL} generates a fresh
existential variable $\genA$, and replaces $a$ with $\genA$ in the body $A$. The
new existential variable $\genA$ is then added to the premise's input context.
% Unlike rule \rul{ACS-ForallR}, the output context $\Delta$ of the premise
% remains unchanged in the conclusion.
% A central idea behind this rule is that we
% defer the decision of choosing a monotype for a type variable, and hope that it
% could be solved later when we have more information at hand.
As a side note, when both types are quantifiers, then either \rul{ACS-ForallR}
or \rul{ACS-ForallR} could be tried. In practice, one can apply
\rul{ACS-ForallR} eagerly.
The last two rules % are specific to the algorithm, thus having no counterparts in
% the declarative version. They
together check consistent subtyping with an
unsolved existential variable on one side and an arbitrary type on the other
side by the help of the instantiation judgment. % Apart from checking that the existential variable does not occur in the
% type $A$, both of the rules do not directly solve the existential variables, but
% leave the real work to the instantiation judgment.

% \subsection{Instantiation}
% \label{sec:algo:instantiate}

\begin{figure}[t]
  \centering
  \begin{small}
\begin{mathpar}
  \framebox{$\tctx \vdash \genA \unif A \toctxr$} \\
  % {\quad \text{Under input context $\Gamma$, instantiate $\genA$ such that
  %     $\genA \tconssub A$, with output context $\Delta$ }} \\
  \InstLSolve \and \InstLReach \and \InstLSolveU   \and \InstLAllR \and \InstLArr
\end{mathpar}

% \begin{mathpar}
%   \framebox{$\tctx \vdash A \unif \genA  \toctxr$} \\
%   % {\quad \text{Under input context $\Gamma$, instantiate $\genA$ such that
%   %     $A \tconssub \genA$, with output context $\Delta$}} \\
%   \InstRSolve \and \InstRReach \and \InstRSolveU  \and \InstRAllL \and \InstRArr
% \end{mathpar}

  \end{small}
  \caption{Algorithmic instantiation}
  \label{fig:algo:instantiate}
\end{figure}

% A central idea of the algorithmic system is to defer the decision of picking a
% monotype to as late as possible.
The judgment $\Gamma \vdash \genA \unif A \dashv \Delta$ defined in
\cref{fig:algo:instantiate} instantiates unsolved existential variables.
Judgment $\genA \unif A$ reads ``instantiate $\genA$ to a consistent subtype of
$A$''. For space reasons, we omit its symmetric judgement $\Gamma \vdash A \unif
\genA \dashv \Delta$.
% Since these two are mutually defined, we
% discuss them together, and omit symmetric rules when convenient.
Rule \rul{InstLSolve} and rule \rul{InstLReach} set $\genA$ to
$\tau$ and $\genB$ in the output context, respectively.
% is the simplest
% one -- when an existential variable meets a monotype. In that case, we simply
% set the solution of $\genA$ to the monotype $\tau$ in the output context. We
% also need to check that the monotype $\tau$ is well-formed under the prefix
% context $\Gamma$.
Rule \rul{InstLSolveU} is similar to \rul{ACS-UnknownR} in that we put no
constraint on $\genA$ when it meets the unknown type $\unknown$. This design
decision reflects the point that type inference only produces static
types~\citep{garcia2015principal}. We will get back to this point in
\cref{subsec:algo:discuss}.
% Rule \rul{InstLReach} deals with the situation where two existential variables
% meet. Note that $\Gamma[\genA][\genB]$ denotes a context where some unsolved existential
% variable $\genA$ is declared before $\genB$. In this situation, the only logical
% thing we can do is to set the solution of one existential variable to the other
% one, depending on which is declared before which. For example, in the output
% context of rule \rul{InstLReach}, we have $\genB = \genA$ because in the input
% context, $\genA$ is declared before $\genB$.
Rule \rul{InstLAllR} is the instantiation version of rule \rul{ACS-ForallR}.
% Since our system is predicative, $\genA$ cannot be instantiated to $\forall b.
% B$, but we can decompose $\forall b. B$ in the same way as in \rul{ACS-ForallR}.
% Rule \rul{InstRAllL} is the instantiation version of rule \rul{ACS-ForallL}.
The last rule \rul{InstLArr} applies when $\genA$ meets a function type. It
follows that the solution must also be a function type.
% looks a bit complicated, but it is actually very
% intuitive: what does the solution of $\genA$ look like when $A$ is a function
% type? The solution must also be a function type!
That is why, in the first premise, we generate two fresh existential variables
$\genA_1$ and $\genA_2$, and insert them just before $\genA$ in the input
context, so that the solution of $\genA$ can mention them. Note that $A_1 \unif
\genA_1$ switches to the other instantiation judgment.


% \paragraph{Example}

% We show a derivation of $\Gamma[\genA] \vdash \forall b. b \to \unknown \unif
% \genA$ to demonstrate the interplay between instantiation, quantifiers and the
% unknown type:
% \[
%   \inferrule*[right=InstRAllL]
%       {
%         \inferrule*[right=InstRArr]
%         {
%           \inferrule*[right=InstLReach]{ }{\Gamma', \genB \vdash \genA_1 \unif \genB \dashv \Gamma' , \genB = \genA_1} \\
%           \inferrule*[right=InstRSolveU]{ }{\Gamma', \genB = \genA_1 \vdash \unknown \unif \genA_2 \dashv \Gamma', \genB = \genA_1}
%         }
%         {
%           \Gamma[\genA], \genB \vdash \genB \to \unknown \unif \genA \dashv \Gamma', \genB = \genA_1
%         }
%       }
%       {
%         \Gamma[\genA] \vdash \forall b. b \to \unknown \unif \genA \dashv \Gamma', \genB = \genA_1
%       }
% \]
% where $\Gamma' = \Gamma[\genA_2, \genA_1, \genA = \genA_1 \to \genA_2]$. Note
% that in the output context, $\genA$ is solved to $\genA_1 \to \genA_2$, and
% $\genA_2$ remains unsolved because the unknown type $\unknown$ puts no
% constraint on it. Essentially this means that the solution of $\genA$ can be any
% function, which is intuitively correct since $\forall b. b \to \unknown$ can be
% interpreted, from the parametricity point of view, as any function.

\subsection{Algorithmic Typing}
\label{sec:algo:typing}

\begin{figure}[t]
  \centering
  \begin{small}
\begin{mathpar}
  \framebox{$\Gamma \vdash e \Rightarrow A \toctxr $} \\
  % {\quad \text{Under input context $\Gamma$, $e$ synthesizes output type $A$,
  %     with output context $\Delta$}} \\
  \AVar \and \ANat \and \ALamU \and \ALamAnnA \and \AAnno \and \AApp
\end{mathpar}
\begin{mathpar}
  \framebox{$\Gamma \vdash e \Leftarrow A \toctxr $} \\
  % {\quad \text{Under input context $\Gamma$, $e$ synthesizes output type $A$,
  %     with output context $\Delta$}} \\
  \ALam \and \AGen \and \ASub
\end{mathpar}
\begin{mathpar}
  \framebox{$\Gamma \vdash A \match A_1 \to A_2 \toctxr$} \\
  % {\quad \text{Under input context $\Gamma$, $A$ synthesizes output type $A_1
  %     \to A_2$, with output context $\Delta$}} \\
  \AMMC \quad \AMMA \and \AMMB \and \AMMD
\end{mathpar}
  \end{small}
  \caption{Algorithmic typing}
  \label{fig:algo:typing}
\end{figure}

We now turn to the algorithmic typing rules in \cref{fig:algo:typing}. The
algorithmic system uses bidirectional type checking to accommodate polymorphism.
Most of them are quite standard.
% All of them are direct analogies of their declarative counterparts. Rules \rul{AVar}
% and \rul{ANat} do not generate any new information, thus the output context is
% the same as the input context. Rule \rul{ALamAnnA} infers the type of a lambda
% abstraction. It does so by pushing $x : A$ into the input context and continues
% to infer the type of the body $B$. The output context in the premise has
% additional declarations in the trailing context $\Theta$, which is discarded in
% the concluding output context.
Perhaps rule \rul{AApp} (which differs significantly from that in the DK system)
deserves attention. It relies on the algorithmic matching judgment $\Gamma
\vdash A \match A_1 \to A_2 \dashv \Delta$.
% The matching judgment
% algorithmically synthesizes a function type from an arbitrary type.
Rule
\rul{AM-ForallL} replaces $a$ with a fresh existential variable $\genA$, thus
eliminating guessing. Rule \rul{AM-Arr} and \rul{AM-Unknown} correspond
directly to the declarative rules.
% self-explanatory. Rule
% \rul{AM-Unknown} says that the unknown type $\unknown$ can be split into a
% function type $\unknown \to \unknown$.
Rule \rul{AM-Var}, which has no
corresponding declarative version, is similar to \rul{InstRArr}/\rul{InstLArr}:
we create $\genA$ and $\genB$ and add $\genC = \genA \to \genB$ to the context.

% Back to \rul{AApp}. This rule first infers the type of $e_1$, producing a output
% context $\Theta_1$. Then it applies $\Theta_1$ to $A$ and goes into the matching
% judgment, which delivers a function type $A_1 \to A_2$ and another output
% context $\Theta_2$. $\Theta_2$ is used as the input context when inferring the
% type of $e_2$. The last premise algorithmically checks if
% $\ctxsubst{\Theta_3}{A_3}$ is a consistent subtype of
% $\ctxsubst{\Theta_3}{A_1}$. $A_2$ and $\Delta$ are the concluding output type
% and the concluding output context, respectively.


% \section{Soundness and Completeness}
% \label{sec:sound:complete}

% To be confident that our algorithmic type system and the declarative type system
% accept exactly the same programs, we need to prove that the algorithmic rules
% are sound and complete with respect to the declarative specifications. Before we
% give the formal statements of the soundness and completeness theorems, we need a
% meta-theoretical device, called \textit{context extension}~\cite{dunfield2013complete}, to help capture a notion of
% information increase from input contexts to output contexts.

% \subsection{Context Extension}
% \label{sec:ctxt:extension}


% A context extension judgment $\Gamma \exto \Delta$ reads ``$\Gamma$ is extended
% by $\Delta$''. Intuitively, this judgment says that $\Delta$ has at least as
% much information as $\Gamma$: some unsolved existential variables in $\Gamma$
% may be solved in $\Delta$. (The full inductive definition can be found in the
% supplementary material. We refer the reader to \citet[][Section
% 4]{dunfield2013complete} for further explanations of context extension.)

\subsection{Completeness and Soundness}

We prove that the algorithmic rules are sound and complete with
respect to the declarative specifications. We need an auxiliary judgment
$\Gamma \exto \Delta$ that captures a notion of information increase from input
contexts $\Gamma$ to output contexts $\Delta$~\citep{dunfield2013complete}.

\paragraph{Soundness.} Roughly speaking, soundness of the algorithmic system says
that given an expression $e$ that type checks in the algorithmic system, there exists
a corresponding expression $e'$ that type checks in the declarative system.
However there is one complication: $e$ does not necessarily have more annotations
than $e'$. For example, by \rul{ALam} we have $\erlam{x}{x} \chkby (\forall a.
a) \rightarrow (\forall a . a)$, but $\erlam{x}{x}$ itself cannot have type
$(\forall a. a) \rightarrow (\forall a . a)$ in the declarative system. To
circumvent that, we add an annotation to the lambda abstraction, resulting in
$\blam{x}{(\forall a . a)}{x}$, which is typeable in the declarative system with
the same type. To relate $\erlam{x}{x}$ and $\blam{x}{(\forall a . a)}{x}$, we
erase all annotations on both expressions. The definition of erasure $\erase{\cdot}$ is
standard and thus omitted.

% \jeremy{mention erasure and why (talk about \rul{ALam} and \rul{ASub})}


% \begin{restatable}[Instantiation Soundness]{mtheorem}{instsoundness} \label{thm:inst_soundness}%
%   Given $\Delta \exto \Omega$ and $\ctxsubst{\Gamma}{A} = A$ and $\genA \notin \mathit{fv}(A)$:
%   \begin{itemize}
%   \item If $\Gamma \vdash \genA \unif A \dashv \Delta$ then $\ctxsubst{\Omega}{\Delta} \vdash \ctxsubst{\Omega}{\genA} \tconssub \ctxsubst{\Omega}{A}$.
%   \item If $\Gamma \vdash A \unif \genA \dashv \Delta$ then $\ctxsubst{\Omega}{\Delta} \vdash \ctxsubst{\Omega}{A} \tconssub \ctxsubst{\Omega}{\genA}$.
%   \end{itemize}
% \end{restatable}

% Notice that the declarative judgment uses $\ctxsubst{\Omega}{\Delta}$, a
% operation that applies a complete context $\Omega$ to the algorithmic context
% $\Delta$, essentially plugging in all known solutions and removing all
% declarations of existential variables (both solved and unsolved), resulting in a
% declarative context.

% With instantiation soundness, next we show that the algorithmic consistent
% subtyping is sound:

% \begin{restatable}[Soundness of Algorithmic Consistent Subtyping]{mtheorem}{subsoudness} \label{thm:sub_soundness}%
%   If $\Gamma \vdash A \tconssub B \toctxr$ where $\ctxsubst{\tctx}{A} = A$ and
%   $\ctxsubst{\tctx}{B} = B$ and $\ctxr \exto \cctx$ then
%   $\ctxsubst{\cctx}{\Delta} \vdash \ctxsubst{\cctx}{A} \tconssub
%   \ctxsubst{\cctx}{B}$.
% \end{restatable}

% At this point, we are ``two thirds of the way'' to proving the ultimate theorem.
% The remaining third concerns with the soundness of matching:

% \begin{restatable}[Matching Soundness]{mtheorem}{matchsoundness}  \label{thm:match_soundness}%
%   If $\Gamma \vdash A \match A_1 \to A_2 \dashv \Delta$ where
%   $\ctxsubst{\Gamma}{A} = A$ and $\Delta \exto \Omega$ then
%   $\ctxsubst{\Omega}{\Delta} \vdash \ctxsubst{\Omega}{A} \match
%   \ctxsubst{\Omega}{A_1} \to \ctxsubst{\Omega}{A_2}$.
% \end{restatable}


% Finally the soundness theorem of algorithmic typing is:

\begin{restatable}[Soundness of Algorithmic Typing]{mtheorem}{typingsoundness} \label{thm:type_sound}
  Given $\ctxr \exto \cctx$,

  \begin{enumerate}
  \item If $\Gamma \vdash e \infto A \toctxr$ then $\exists e'$ such
    that $\ctxsubst{\cctx}{\Delta} \vdash e' : \ctxsubst{\cctx}{A}$ and
    $\erase{e} = \erase{e'}$.
  \item If $\Gamma \vdash e \chkby A \toctxr$ then $\exists e'$ such
    that $\ctxsubst{\cctx}{\Delta} \vdash e' : \ctxsubst{\cctx}{A}$ and
    $\erase{e} = \erase{e'}$.
  \end{enumerate}


\end{restatable}


\paragraph{Completeness.}
Completeness of the algorithmic system is the reverse of soundness: given a
declarative judgment of the form $\ctxsubst{\Omega}{\Gamma} \vdash
\ctxsubst{\Omega} \dots $, we want to get an algorithmic derivation of $\Gamma
\vdash \dots \dashv \Delta$. It turns out that completeness is a bit trickier to
state in that the algorithmic rules generate existential variables on the fly,
so $\Delta$ could contain unsolved existential variables that are not found in
$\Gamma$, nor in $\Omega$. Therefore the completeness proof must produce another
complete context $\Omega'$ that extends both the output context $\Delta$, and
the given complete context $\Omega$. As with soundness, we need erasure to
relate both expressions.

% \jeremy{talk about \rul{Gen}}

% \begin{restatable}[Instantiation Completeness]{mtheorem}{instcomplete}  \label{thm:inst_complete}%
%   Given $\Gamma \exto \Omega$ and $A = \ctxsubst{\Gamma}{A}$ and $\genA \in
%   \mathit{unsolved}(\Gamma)$ and $\genA \notin \mathit{fv}(A)$:
%   \begin{enumerate}
%   \item If $\ctxsubst{\Omega}{\Gamma} \vdash \ctxsubst{\Omega}{\genA} \tconssub
%     \ctxsubst{\Omega}{A}$ then there exist $\Delta$, $\Omega'$ such that $\Omega \exto
%     \Omega'$ and $\Delta \exto \Omega'$ and $\Gamma \vdash \genA \unif A \dashv \Delta$.
%   \item If $\ctxsubst{\Omega}{\Gamma} \vdash \ctxsubst{\Omega}{A} \tconssub
%     \ctxsubst{\Omega}{\genA}$ then there exist $\Delta$, $\Omega'$ such that $\Omega \exto
%     \Omega'$ and $\Delta \exto \Omega'$ and $\Gamma \vdash A \unif \genA \dashv \Delta$.
%   \end{enumerate}
% \end{restatable}


% Next is the completeness of consistent subtyping:

% \begin{restatable}[Generalized Completeness of Subtyping]{mtheorem}{subcomplete}  \label{thm:sub_completeness}%
%   If $\Gamma \exto \Omega$ and $\Gamma \vdash A$ and $\Gamma \vdash B$ and
%   $\ctxsubst{\Omega}{\Gamma} \vdash \ctxsubst{\Omega}{A} \tconssub
%   \ctxsubst{\Omega}{B}$ then there exist $\Delta$, $\Omega'$ such that $\Delta
%   \exto \Omega'$ and $\Omega \exto \Omega'$ and $\Gamma \vdash
%   \ctxsubst{\Gamma}{A} \tconssub \ctxsubst{\Gamma}{B \dashv \Delta}$.
% \end{restatable}


% We prove that the algorithmic matching is complete with respect to the
% declarative matching:

% \begin{restatable}[Matching Completeness]{mtheorem}{matchcomplete} \label{thm:match_complete}%
%   Given $\Gamma \exto \Omega$ and $\Gamma \vdash A$, if
%   $\ctxsubst{\Omega}{\Gamma} \vdash \ctxsubst{\Omega}{A} \match A_1 \to A_2$
%   then there exist $\Delta$, $\Omega'$, $A_1'$ and $A_2'$ such that $\Gamma
%   \vdash \ctxsubst{\Gamma}{A} \match A_1' \to A_2' \dashv \Delta$ and $\Delta \exto \Omega'$ and
%   $\Omega \exto \Omega'$ and $A_1 = \ctxsubst{\Omega'}{A_1'}$ and $A_2 =
%   \ctxsubst{\Omega'}{A_2'}$.
% \end{restatable}


% Finally here is the completeness theorem of the algorithmic typing:

\begin{restatable}[Completeness of Algorithmic Typing]{mtheorem}{typingcomplete}  \label{thm:type_complete}
  Given $\Gamma \exto \Omega$ and $\Gamma \vdash A $, if
  $\ctxsubst{\Omega}{\Gamma} \vdash e : A$ then there exist $\Delta$,
  $\Omega'$, $A'$ and $e'$ such that $\Delta \exto \Omega'$ and $\Omega \exto \Omega'$
  and $\Gamma \vdash e' \infto A' \dashv \Delta$ and $A = \ctxsubst{\Omega'}{A'}$ and $\erase{e} = \erase{e'}$.
\end{restatable}





%%% Local Variables:
%%% mode: latex
%%% TeX-master: "../paper"
%%% org-ref-default-bibliography: "../paper.bib"
%%% End:


\section{Discussion}
\label{sec:discuss}

Discuss the limitations/issues of \name. The differences from the original trait model.

\begin{itemize}
\item Override: Form a new trait by layering additional methods over an existing
  trait. This operation is an asymmetric sum.
\item Exclusion: forms a new trait by removing a method from an existing trait
\end{itemize}


\begin{itemize}
\item Traits have no proper notation of inheritance relationship (no super keyword)
\item Traits have no nice syntax of redefining
\item traits allow annotation of type, then term declarations don't need to
\end{itemize}
\section{Related Work}
\label{sec:related}

% \bruno{I think (part of) this text can be discussed in here instead:


There are multiple flavours of inheritance. To avoid confusion, since the same
terminology is often used in the literature to mean different things, we use the
following 3 terms when comparing related work with ours.

\begin{itemize}
\item{{\bf Static inheritance:}} Static inheritance refers to what the typical
  model of inheritance in class-based languages. The inheritance model is said
  to be static because when using class extension, the extended classes are
  statically known at compile-time.
\item{{\bf Mutable Inheritance:}} Prototype-based languages allow another model
  of inheritance, which we call \emph{mutable inheritance}. In this inheritance
  model, self-references are mutable and changeable at any point.
\item{{\bf Dynamic Inheritance:}} Dynamic inheritance is a less well-known model
  which stands in between static and mutable inheritance. Unlike the static
  inheritance model, with dynamic inheritance objects can inherit from other
  objects which are not statically known. However, unlike mutable inheritance,
  the self-reference is not mutable and cannot be arbitrarily changed at
  run-time.
\end{itemize}

Figure~\ref{fig:comparision} shows the comparison between \name and various
similar languages that follow \citeauthor{cook1989inheritance}'s ``Inheritance is not
Subtyping'' (i.e. the flexible model), as we will explain below.

\begin{figure}[t]
  \centering
  \begin{small}
  \begin{tabular}{|l||c|c|c|c|}
    \hline
    & \bf{Statically typed} & \bf{Polymorphism} & \bf{Meta-theory} & \bf{Inheritance}  \\
    \hline
    \name & \cmark & \cmark & \cmark & Dynamic \\
    \hline
    \textsc{Self} & \xmark & \xmark & \xmark & Mutable \\
    \hline
    Cecil & \cmark & \cmark & \xmark & Static \\
    \hline
    Cook's Modula-3 & \cmark & \xmark & \xmark & Static \\
    \hline
    IFJ & \cmark & \xmark & \cmark & Dynamic \\
    \hline
    \textsc{Darwin} & \cmark & \xmark & \xmark & Dynamic \\
    \hline
  \end{tabular}
  \end{small}
  \caption{Comparison between \name and various similar languages that
  adopt the \emph{flexible model}.}
  \label{fig:comparision}
\end{figure}



% \paragraph{Dynamically-typed Languages with Delegation Mechanism}

% \begin{itemize}
% \item Clojure Protocols
%   % http://www.ibm.com/developerworks/library/j-clojure-protocols/
% \item Ruby mixin
% \item JS mixin
% \end{itemize}

% They are all dynamically typed.


\paragraph{Delegation-based languages}

\citet{lieberman1986using} is the first to promote the use of prototypes and
delegation as the mechanism to code sharing between objects. Since then many
researchers have studied the mechanisms of
delegation~\cite{wegner1987dimensions,malenfant1995semantic,goldberg1989smalltalk}.
\textsc{Self}~\cite{ungar1988self} is a dynamically typed, prototype-based
language with a simple and uniform object model. \textsc{Self}'s inheritance
model is typical of what we call mutable inheritance, because an object's parent
slots may be assigned new values at run-time. Mutable inheritance is rather
unstructured, and oftentimes access to any clashing methods will generate a
``messageAmbiguous'' error at run-time. Although \name's dynamic inheritance is
not as powerful as mutable inheritance, its static type system can guarantee
that no such errors occur at run-time.

There is not much work on statically-typed, delegation-based languages.
\citet{kniesel1999type} provides a good overview of problems when combining
delegation with a static type discipline. Cecil~\cite{chambers1992object,
  chambers1993cecil} is a prototype-based language, where delegation is the
mechanism for method call and code reuse. Cecil supports a polymorphic static
type system, although no meta-theory of any kind is given. Its type system is
able to detect statically when a message might be ambiguously defined as a
result of multiple inheritance or multiple dispatching. However, one major
omission of Cecil, which is also one of the interesting features of \name, is
dynamic inheritance. There are other
works~\cite{fisher1995delegation,anderson2003can} on delegation in a
statically-typed setting, but none of them provide means (such as the merge
construct, disjointness constraints, etc.) that are needed for extensible
designs.

\citet{cook1989inheritance} were the first to propose a typed model of
inheritance where subtyping and inheritance are two separate concepts. In
particular, they introduce the notion of \textit{type inheritance} and show that
inherited objects have inherited types, not subtypes. An interesting aspect of
their calculus is the \textbf{with} construct, used to join two records. This is
somewhat similar to our merge construct. However two major differences are worth
pointing out: 1) the \textbf{with} construct operates only on records; and 2) it
is a biased operator, favoring values from its right argument. This biased
operator is good for modelling mixins, but not traits. The
\textbf{with} construct seems to be unable to merge two arbitrary (and possible
polymorphic) values, since this seems to require something like
\emph{row polymorphism}~\cite{wand1987complete,wand1989type}, which is not available in their language.
The \textit{onion} construct in the Big Bang
language~\cite{palmer2015building,menon2012big} has a similar bias problem -- it is a
left-associative operator which gives rightmost precedence to one
implementation when conflicts exist.

\paragraph{Mixin-based inheritance}

Mixins have become very popular in many OO languages
~\cite{flatt1998classes,bono1999core, ancona2003jam}. \citeauthor{bracha1990mixin}'s
seminal paper~\citep{bracha1990mixin} extends Modula-3 with mixins. Mixins are subclasses parameterized
over a superclass, and used to produce a variety of classes with the same
functionality and behaviour. Mixin-based inheritance requires that mixins be
composed linearly, and as such, conflicts are resolved implicitly (mixins
appearing later overwrite all the identically named features of earlier mixins).
In comparison, the trait model in \name requires conflicts be resolved
explicitly. We want to emphasize that this conflict detection is essential in
expressing composition operators for Object Algebras, without running
into ambiguities.


\paragraph{Trait-based inheritance}

The seminar paper by \citet{scharli2003traits} introduced the ideas behind
traits, where they also documented an implementation of the trait
mechanism in a dynamically typed version of Smalltalk. Since then many
formalizations of traits have been
proposed~\cite{scharli2003traitsformal,ducasse2006traits,bettini2010prototypical}.
For example \citet{fisher2004typed} presented a statically-typed calculus that
models traits. Conflict detection is the hallmark of trait-based
inheritance, compared with mixin-based inheritance. One important difference
with \name is that those systems support \textit{classes} in addition to traits,
and consider the interaction between them, whereas \name is 
delegation based and the mechanism for code reuse is purely traits
(i.e., there are no classes in \name). The
deviation from traditional class-based models is not only because of its
simplicity, but also because we need a very \textit{dynamic} form of
inheritance, as has been elaborated throughout the paper.

Compared to the traditional trait mode, traits in \name have the following
differences: 1) traditional traits cannot be instantiated but only composed with
a class, whereas traits in \name can be instantiated directly; 2) traditional
traits cannot take constructor parameters whereas ours can; 3) the trait system
in \name lacks a proper notation of inheritance relationship. For example in the
traditional trait model, if the same method (i.e., from the same trait) is
obtained more than once via different paths, there is no conflict. This is not
the case in \name; and 4) traits in \name support dynamic
inheritance. 
%In the
%traditional trait model, when it comes to inheritance, the traits being
%inherited must be statically known.




% \citet{flatt1998classes} proposed MIXEDJAVA, an extension to a subset of
% sequential Java called CLASSICJAVA with mixins. In their model, mixins
% completely subsume the role of classes (classes are mixins that do not inherit
% any services). One interesting aspect in their system is that two identically
% named methods are allowed to coexist, and are resolved at run-time with run-time
% context information provided by the current \textit{view} of an object. In
% comparison, conflicts in \name are detected statically, and resolved by the
% programmers. Like \name, their model also enforces the distinction between
% implementation inheritance and subtyping.

% \citet{bono1999core} develop an imperative class-based calculus that provides a
% formal model for both single and mixin inheritance. Objects are represented by
% records and produced by instantiating classes. In their calculus, the class
% construct is extensible but not subtypable, while objects are subtypable but not
% extensible. Like \name, their system has a clean separation between subtyping
% and inheritance. Also, their type system does not have polymorphism.

% \citet{ancona2003jam} extends the Java language to support mixins, called Jam.
% Since Jam is an upward-compatible extension of Java 1.0, it is inheritantly a
% covariant mode. Unlike MIXEDJAVA, mixins can be only instantiated on classes,
% and there is no notion of mixin composition.


\begin{comment}

\begin{itemize}


\item ``Object-Oriented Multi-Methods in Cecil''

\item ``Dimensions of Object-Based Language Design''

\item ``On the Semantic Diversity of Delegation-Based Programming Languages''

\item ``Self: The power of simplicity''

\item ``Type-safe delegation for run-time component adaptation''

\item ``A delegation-based object calculus with subtyping''

\item ``Can Addresses be Types? (a case study: objects with delegation)''

\item ``Inheritance is not subtyping''


Mixins

\item ``mixin-based inheritance''

\item ``Classes and mixins''

\item ``A core calculus of classes and mixins''

\item ``A core calculus of higher-order mixins and classes''

\item ``Jam—Designing a Java Extension with Mixins''



\end{itemize}

Do they have polymorphic type systems? Do they support mutable self reference?

\end{comment}


\paragraph{Class-based languages with more advanced forms of inheritance}

Incomplete Featherweight Java (IFJ), proposed by \citet{bettini2008type}, is a
conservative extension of Featherweight Java with incomplete objects. Besides
standard classes, programmers can also define incomplete classes, whose
instances are incomplete objects. Incomplete objects can be composed (by object
composition) with complete objects, yielding new complete objects at run-time,
while ensuring statically that the composition is type-safe. Incomplete objects
are quite flexible, and support dynamic inheritance. However, object composition
in IFJ is quite restrictive, compared to \name, in that it can only compose an
incomplete object with a complete object. In that regard, and also because IFJ's
type system is not polymorphic, IFJ is unable to encode composition operators of
Object Algebras. \citet{kniesel1999type} showed that type-safe integration of
delegation with subtyping into a class-based model is possible, resulting in the
\textsc{Darwin} model. In \textsc{Darwin}, the type of the parent object must be
a declared class and this limits the flexibility of dynamic composition.
\citeauthor{ostermann2002dynamically}'s delegation
layers~\citep{ostermann2002dynamically} use delegation for doing dynamic
composition in a system with virtual classes. This is in contrast with most
other approaches that use class-based composition, but closer to the dynamic
composition that we use in \name.

There are many other class-based OO languages that are equipped with more
advanced forms of
inheritance~\cite{meyer1987eiffel,buchi2000generic,ostermann2001object}. Most of
them are heavyweight and are specific to classes. \name is object-centered, more
lightweight, and is dedicated to express extensible designs in a simpler way.


% Eiffel~\cite{meyer1987eiffel} is a class-based language that is based on the
% identification of classes with types and of inheritance with subtyping. Eiffel
% supports multiple inheritance, with the restriction that name collisions are
% considered programming errors, and ambiguities must be resolved explicitly by
% the programmer (by means of renaming). In this regard, \name is quite like
% Eiffel. However, the type system in \name is more lenient in that two
% identically named methods with different signatures can coexist without any
% problems.

% \citet{kniesel1999type} is the first to show that type-safe integration of
% delegation with subtyping into a class-based model is possible, resulting in the
% DARWIN model. In the DARWIN model, the type of the parent object must be a
% declared class and this limits the flexibility of dynamic composition, whereas
% in \name, the merge operator can merge/compose any objects. Another difference
% with \name lies in the conflict resolution, where DARWIN relies on method
% overriding with the assumption that the author of the overriding method is aware
% of the effect.

% Generic wrappers~\cite{buchi2000generic} supports aggregating objects at
% run-time. In their model, once a ``wrappee'' is assigned to a ``wrapper'', the
% wrappee is fixed. GBETA~\cite{ernst2000gbeta} has some dynamic features that are
% related to delegation. Like Generic wrappers, parents in GBETA are fixed at
% run-time.

% \citet{ostermann2001object} proposed compound references (CR) as a abstraction
% for object references, which provides explicit linguistic support for combining
% different composition properties on-demand. The model is statically typed, and
% decouples subtype declaration from implementation reuse.


% \citet{ostermann2002dynamically} proposed delegation layers as an approach to
% decompose a collaboration into layers and compose these layers dynamically at
% run-time. This combines and generalizes delegation and virtual classes concepts.

% \citet{ostermann2008nominal} compared the nominal and structural subtyping
% mechanisms. They argue nominal subtyping gives more safety guarantee, whereas
% structural subtyping is more flexible from a component-based perspective. The
% type system of \name chooses structural subtyping.

\paragraph{Intersection types, polymorphism and the merge construct}

There is a large body of work on intersection types. Here we only talk about
work that have direct influences on ours. \citet{dunfield2014elaborating} shows
significant expressiveness of type systems with intersection types and a merge
construct. However his calculus lacks coherence. The limitation was addressed
by~\citet{oliveira2016disjoint}, where they introduced the notion of
disjointness to ensure coherence. The combination of intersection types, a merge
construct and parametric polymorphism, while achieving coherence was first
studied in the \bname calculus~\cite{alpuimdisjoint}, where they proposed the
notion of disjoint polymorphism. \bname serves as the theoretical foundation of
\name.


\begin{comment}

\begin{itemize}

\item Eiffel

\item ``Delegation by object composition'' (IFJ) and ``Type safe dynamic object
  delegation in class-based languages''

\item ``Dynamically composable collaborations with delegation layers''

\item ``Generic wrappers''

\item ``Object-Oriented Composition Untangled''

\item ``gbeta - a language with virtual attributes, Block Structure, and Propagating, Dynamic Inheritance''

\item ``Nominal and Structural Subtyping in Component-Based Programming''

\item ``Engineering a programming language: The type and class system of Sather ''

\item ``Big Bang Designing a Statically-Typed Scripting Language''

\item ``Building a Typed Scripting Language''



\end{itemize}

\end{comment}


\section{Conclusions and Future Work}
\label{sec:conclusion}

We have proposed \name, a type-safe and coherent calculus with disjoint
intersection types, and support for nested composition/subtyping. \name
improves upon earlier work with a more
flexible notion of disjoint intersection types, which leads to
a clean and elegant formulation of the type system. Due to the added
flexibility we have had to employ a more powerful proof method based on logical
relations to rigorously prove coherence.
We also show how \name supports essential features of family
polymorphism, such as nested composition. We believe \name provides insights into family polymorphism, and
has potential for practical applications for extensible software designs.

A natural direction for future work is to enrich \name with parametric
polymorphism. There is abundant literature on logical relations for parametric
polymorphism~\citep{reynolds1983types} and we foresee no fundamental
difficulties in extending our proof method.\footnote{
Our prototype
  implementation already supports polymorphism, but we
  are still in the process of extending our Coq development with polymorphism. } The resulting calculus will be
more expressive than \fname. An interesting application that we intend to investigate
is native support for \textit{object algebras}~\citep{oliveira2012extensibility}
(or the finally tagless approach~\citep{CARETTE_2009}). For example, we can
define the object algebra interfaces for the Expression Problem example in
\cref{sec:overview} as follows:
\lstinputlisting[linerange=75-76]{../../impl/examples/overview.sl}% APPLY:linerange=LANG_EXT_INTER
By instantiating \lstinline{E} with \lstinline{IPrint}, i.e.,
\lstinline{ExpAlg[IPrint]}, we get the interface of the \lstinline{Lang} family.
In that sense, object algebra interfaces can be viewed as family interfaces.
Moreover, combing algebras implementing \lstinline{ExpAlg[IPrint]} and
\lstinline{ExpAlg[IEval]} to form \lstinline{ExpAlg[IPrint & IEval]} is trivial
with nested composition. Polymorphism also improves code reuse across expressions in the
base and extended languages. For example, the following creates two expressions,
one in the base language, the other in the extended language:
\lstinputlisting[linerange=81-82]{../../impl/examples/overview.sl}% APPLY:linerange=LANG_EXT
Notice how we can  reuse \lstinline{e1} of the base language in the definition
of \lstinline{e2}.



% \jeremy{creating expressions using base and extended expressions, and show more reuse}

% \jeremy{future work} \jeremy{mention in passing this rule is unsound with
%   effects, see ``Intersection types and computational effects''}

% Local Variables:
% mode: latex
% TeX-master: "../paper"
% End:




%% Acknowledgments
\begin{acks}                            %% acks environment is optional
                                        %% contents suppressed with 'anonymous'
  %% Commands \grantsponsor{<sponsorID>}{<name>}{<url>} and
  %% \grantnum[<url>]{<sponsorID>}{<number>} should be used to
  %% acknowledge financial support and will be used by metadata
  %% extraction tools.
  This material is based upon work supported by the
  \grantsponsor{GS100000001}{National Science
    Foundation}{http://dx.doi.org/10.13039/100000001} under Grant
  No.~\grantnum{GS100000001}{nnnnnnn} and Grant
  No.~\grantnum{GS100000001}{mmmmmmm}.  Any opinions, findings, and
  conclusions or recommendations expressed in this material are those
  of the author and do not necessarily reflect the views of the
  National Science Foundation.
\end{acks}

%% Bibliography
\bibliography{paper}

\ifdefined\submitoption
\newpage
\appendix

\section{Some Proofs about the Declarative System}


\lemmaerase*
\begin{proof}
By straightforward induction on the typing derivation.
\end{proof}

\lemmacoherence*
\begin{proof}
According to Lemma~\ref{lemma:erase}, after erasure of types and casts,
$\mathcal{C}\{[[pe1]]\}$ and $\mathcal{C}\{[[pe2]]\}$ are equivalent. So if
$\mathcal{C}\{ [[pe1]] \} \reduce [[n]]$, it is impossible for $\mathcal{C}\{
[[pe2]] \}$ to reduce to a different integer according to the dynamic semantics.
\end{proof}

\proptop*
\begin{proof} \leavevmode
  \begin{itemize}
  \item From first to second: By induction on the derivation of consistent
    subtyping. We have extra case \rref{CS-Top} now, where $B = \top$.
    We can choose $C = A$, and
    $D$ by replacing the unknown types in $C$ by $\nat$. Namely, $D$ is a static
    type, so by \rref{S-Top} we are done.
  \item From second to first: By induction on the derivation of second
    subtyping. We have extra case \rref{S-Top} now, where
    $B = \top$, so $A \tconssub B$ holds by \rref{CS-Top}.
  \end{itemize}
\end{proof}

\propagttop*
\begin{proof} \leavevmode
  \begin{itemize}
  \item From left to right: By induction on the derivation of consistent
    subtyping. We have case \rref{CS-Top} now.
    It follows that for
    every static type $A_1 \in \gamma(A)$, we can derive $A_1 \tsub \top$ by
    \rref{S-Top}.
    We have $B_1 = B = \top$ and we are done.
  \item From right to left: By induction on the derivation of subtyping and
    inversion on the concretization. We have extra case \rref{S-Top} now, where
    $B$ is $\top$. So
    consistent subtyping directly holds.
  \end{itemize}
\end{proof}

\propparalpha*
\begin{proof}
Follows directly from the definition of Translation Pre-order.
\end{proof}

\begin{definition}[Measurements of Translation]
  There are three measurements of a translation $[[pe]]$,
  \begin{enumerate}
  \item $[[ ||pe||e]]$, the size of the expression 
  \item $[[ ||pe||s ]]$, the number of distinct static type parameters in $[[pe]]$
  \item $[[ ||pe||g ]]$, the number of distinct gradual type parameters in $[[pe]]$
  \end{enumerate}
  We use $[[ ||pe|| ]]$ to denote the lexicographical order of the triple
  $([[ ||pe||e ]], -[[ ||pe||s ]], -[[ ||pe||g ]])$.
\end{definition}

\begin{definition}[Size of types]

  \begin{align*}
    [[ || int ||  ]] &= 1 \\
    [[ || a ||  ]] &= 1 \\
    [[ || A -> B  ||  ]] &= [[ || A || ]] + [[ || B || ]] + 1 \\
    [[ || \/a . A ||  ]] &= [[ || A || ]] + 1 \\
    [[ || unknown ||  ]] &= 1 \\
    [[ || static ||  ]] &= 1 \\
    [[ || gradual ||  ]] &= 1
  \end{align*}

\end{definition}

\begin{definition}[Size of expressions]

  \begin{align*}
    [[ || x ||e  ]] &= 1 \\
    [[ || n ||e  ]] &= 1 \\
    [[ || \x : A . pe ||e  ]] &= [[ || A || ]] + [[ || pe ||e ]] + 1 \\
    [[ || /\ a. pe ||e  ]] &= [[ || pe ||e ]] + 1 \\
    [[ || pe1 pe2 ||e  ]] &= [[ || pe1 ||e ]] + [[  || pe2 ||e ]] + 1 \\
    [[ || < A `-> B> pe ||e  ]] &= [[ || pe ||e ]] + [[  || A || ]] + [[  || B || ]] + 1 \\
  \end{align*}

\end{definition}


\begin{lemma} \label{lemma:size_e}
  If $[[dd |- e : A ~~> pe]]$ then $[[ || pe ||e    ]] \geq [[ || e ||e   ]]  $.
\end{lemma}
\begin{proof}
  Immediate by inspecting each typing rule.
\end{proof}

\begin{corollary} \label{lemma:decrease_stop}
  If $[[dd |- e : A ~~> pe]]$ then $[[ || pe ||   ]] > ([[ || e ||e ]], -[[ || e ||e ]], -[[ || e ||e ]] )  $.
\end{corollary}
\begin{proof}
  By \cref{lemma:size_e} and note that $ [[ || pe ||e   ]] > [[  || pe ||s  ]] $ and $ [[ || pe ||e   ]] > [[  || pe ||g  ]] $
\end{proof}

\begin{lemma} \label{lemma:type_decrease}
  $[[ || A || ]] \leq [[ || S(A) || ]]  $.
\end{lemma}
\begin{proof}
  By induction on the structure of $[[A]]$. The interesting cases are $[[ A ]] = [[static]]$ and
  $[[ A ]] = [[gradual]]$. When $[[ A ]] = [[static]]$, $[[ S(A) ]] = [[t]]$
  for some monotype $[[t]]$ and it is immediate that $[[ || static ||  ]]  \leq [[ || t || ]] $
  (note that $[[ || static ||  ]] < [[ || gradual ||  ]] $ by definition).
\end{proof}

\begin{lemma}[Substitution Decreases Measurement]
  \label{lemma:subst_dec_measure}
  If $[[pe1]] \leq [[pe2]]$, then $ {[[ ||pe1|| ]]} \leq [[ ||pe2|| ]]$; unless
  $[[pe2]] \leq [[pe1]]$ also holds, otherwise we have $[[ ||pe1|| ]] < [[ ||pe2|| ]]$.
\end{lemma}
\begin{proof}
  Since $[[ pe1  ]] \leq [[  pe2  ]]$, we know $[[ pe2  ]] = [[ S(pe1)  ]]$ for some $[[S]]$. By induction on
  the structure of $[[pe1]]$.

  \begin{itemize}
  \item Case $[[pe1]] = [[  \x : A . pe ]]$. We have
    $[[ pe2  ]] = [[  \x : S(A) . S(pe)  ]]$. By \cref{lemma:type_decrease} we have $[[ || A || ]] \leq [[ || S(A) || ]]$.
    By i.h., we have $[[ || pe ||  ]] \leq [[ || S(pe) ||  ]]$. Therefore $[[ || \x : A . pe ||    ]] \leq [[ || \x : S(A) . S(pe) ||  ]]$.
  \item Case $[[pe1]] = [[ < A `-> B > pe  ]]$. We have
    $[[pe2]] = [[ < S(A) `-> S(B) > S(pe)  ]]$.  By \cref{lemma:type_decrease} we have $[[ || A || ]] \leq [[ || S(A) || ]]$
    and $[[ || B || ]] \leq [[ || S(B) || ]]$. By i.h., we have $[[ || pe ||  ]] \leq [[ || S(pe) ||  ]]$.
    Therefore $[[  || < A `-> B > pe ||  ]] \leq [[ || < S(A) `-> S(B) > S(pe)  ||   ]]$.

  \item The rest of cases are immediate.
  \end{itemize}
\end{proof}


\lemmareptyping*
\begin{proof}
We already know that at least one translation $[[pe]] = [[pe1]]$ exists
for every typing derivation. If $[[pe1]]$ is a representative translation then we
are done. Otherwise there exists another translation $[[pe2]]$ such that
$[[pe2]] \leq [[pe1]]$ and $ [[pe1]] \not \leq [[pe2]]$. By
\cref{lemma:subst_dec_measure}, we have $[[||pe2||]] < [[ ||pe1|| ]]$. We continue
with $[[pe]] = [[pe2]]$, and get a strictly decreasing sequence $[[ || pe1 ||  ]], [[ || pe2 || ]], \dots$.
By \cref{lemma:decrease_stop}, we know this sequence cannot be infinite long. Suppose it ends at $[[ || pej || ]]$,
by the construction of the sequence, we know that $[[pej]]$ is a representative translation of $[[e]]$.
\end{proof}


\newpage


\section{The Extended Algorithmic System}


\subsection{Syntax}


\begin{center}
\begin{tabular}{lrcl} \toprule
  Expressions & $[[ae]]$ & \syndef & $[[x]] \mid [[n]] \mid [[ \x : aA . ae ]]  \mid  [[\x . ae]] \mid [[ae1 ae2]] \mid [[ae : aA]] \mid [[ let x = ae1 in ae2  ]] $ \\
  Types & $[[aA]], [[aB]]$ & \syndef & $ [[int]] \mid [[a]] \mid [[evar]] \mid [[aA -> aB]] \mid [[\/ a. aA]] \mid [[unknown]] \mid [[static]] \mid [[gradual]] $ \\
  Monotypes & $[[at]], [[as]]$ & \syndef & $ [[int]] \mid [[a]] \mid [[evar]] \mid [[at -> as]] \mid [[static]] \mid [[gradual]]$ \\
  Existential variables & $[[evar]]$ & \syndef & $[[sa]]  \mid [[ga]]  $   \\
  Castable Types & $[[agc]]$ & \syndef & $ [[int]] \mid [[a]] \mid [[evar]] \mid [[agc1 -> agc2]] \mid [[\/ a. agc]] \mid [[unknown]] \mid [[gradual]] $ \\
  Castable Monotypes & $[[atc]]$ & \syndef & $ [[int]] \mid [[a]] \mid [[evar]] \mid [[atc1 -> atc2]] \mid [[gradual]]$ \\
  Algorithmic Contexts & $[[GG]], [[DD]], [[TT]]$ & \syndef & $[[empty]] \mid [[GG , x : aA]] \mid [[GG , a]] \mid [[GG , evar]]  \mid [[GG, sa = at]] \mid [[GG, ga = atc]] \mid [[ GG, mevar ]] $ \\
  Complete Contexts & $[[OO]]$ & \syndef & $[[empty]] \mid [[OO , x : aA]] \mid [[OO , a]] \mid [[OO, sa = at]] \mid [[OO, ga = atc]] \mid [[OO, mevar]] $ \\
  \bottomrule
\end{tabular}

\end{center}


\subsection{Type System}


\drules[as]{$ [[GG |- aA <~ aB -| DD ]] $}{Algorithmic Consistent Subtyping}{tvar, evar, int, arrow, forallR, forallLL, spar, gpar, unknownLL, unknownRR, instL, instR}

\drules[instl]{$ [[ GG |- evar <~~ aA -| DD   ]] $}{Instantiation I}{solveS, solveG, solveUS, solveUG, reachSGOne, reachSGTwo, reachOther, arr, forallR}

\drules[instr]{$ [[ GG |- aA <~~ evar -| DD   ]] $}{Instantiation II}{solveS, solveG, solveUS, solveUG, reachSGOne, reachSGTwo, reachOther, arr, forallLL}

\drules[inf]{$ [[ GG |- ae => aA -| DD ]] $}{Inference}{var, int, lamannTwo, lamTwo, anno, app, letTwo}

\drules[chk]{$ [[ GG |- ae <= aA -| DD ]] $}{Checking}{lam, gen, sub}

\drules[am]{$ [[ GG |- aA |> aA1 -> aA2 -| DD ]] $}{Algorithmic Matching}{forallL, arr, unknown, var}

\newpage

\section{Decidability} \label{app:decidable}

The decidability proofs mostly follow that of DK system. Whenever possible, we only show
the new cases; otherwise we provide full detailed proofs.

\subsection{Decidability of Instantiation}


\begin{lemma}[Left Unsolvedness Preservation]
  Let $[[GG]] = [[  GG0, evar, GG1 ]]$. If  $[[ GG |- evar <~~ aA -| DD    ]]$ or $[[  GG |- aA <~~ evar -| DD  ]]$, and $[[evarb]] \in \textsc{unsolved}([[GG0]])$,
  then  $[[DD]] = [[(DD0, evarb, DD1)]]$ or $[[DD]] = [[(DD0, evarb', evarb = evarb', DD1)]]$ where $[[evarb']]$ is a fresh unsolved existential.
\end{lemma}
\begin{proof}
  By induction on the given derivation. We show the new cases.

  \begin{itemize}
  \item Case \[     \ottaltinferrule{instl-solveUS}{}{  }{ [[  GG0, sa, GG1 |- sa <~~ unknown -| GG0 , ga, sa = ga, GG1 ]] }  \]
    First notice that $[[evarb]]$ cannot be $[[ga]]$. Then to the left of $[[sa]]$, the contexts $[[DD]]$ and $[[GG]]$ are the same $[[GG0]]$.
  \item Case \[  \drule{instl-solveUG}   \]
    Immediate, since to the left of $[[ga]]$, the contexts $[[DD]]$ and $[[GG]]$ are the same.
  \item Case \[  \drule{instl-reachSGOne}    \]
    First notice that $[[evarb]]$ cannot be $[[ga]]$. Then to the left of $[[sa]]$, the contexts $[[DD]]$ and $[[GG]]$ are the same.
  \item Case \[  \drule{instl-reachSGTwo}    \]
    If $[[evarb]] \neq [[sb]] $, immediate, since to the left of $[[ga]]$ ($[[evarb]]$ cannot be $[[gb]]$), the contexts $[[DD]]$ and $[[GG]]$ are the same.
    Otherwise, $[[sb]]$'s solution (i.e., $[[gb]]$) is a fresh unsolved existential that lies just before $[[sb]]$.
  \item Case \rref*{instr-solveUS} is similar to case \rref*{instl-solveUS}.
  \item Case \rref*{instr-solveUG} is similar to case \rref*{instl-solveUG}.
  \item Case \rref*{instr-reachSG1} is similar to case \rref*{instl-reachSG1}.
  \item Case \rref*{instr-reachSG2} is similar to case \rref*{instl-reachSG2}.
  \end{itemize}

\end{proof}


\begin{lemma}[Left Free Variable Preservation]
  Let $[[GG]] = [[  GG0, evar, GG1  ]]$. If $[[ GG |- evar <~~ aA -| DD    ]]$ or $[[  GG |- aA <~~ evar -| DD  ]]$, and $[[ GG |- aB ]]$
  and $[[ evar ]] \notin \textsc{fv}([[ [GG]aB  ]])$ and $[[evarb]] \in \textsc{unsolved}([[GG0]])  $
  and $[[evarb]]  \notin \textsc{fv}([[ [GG]aB  ]])  $,
  then $[[evarb]] \notin \textsc{fv}([[  [DD]aB ]])$.
\end{lemma}
\begin{proof}
  By induction on the given derivation. We show the new cases.
  \begin{itemize}
  \item Case \[  \drule{instl-solveUS}    \]
    Since $[[DD]]$ differs from $[[GG]]$ only in solving $[[sa]]$ to $[[ga]]$, and $[[ga]]$ is fresh,
    applying $[[DD]]$ to a type will not introduce $[[evarb]]$.  We have $[[evarb]]  \notin \textsc{fv}([[ [GG]aB  ]])  $, so
    $[[evarb]]  \notin \textsc{fv}([[ [DD]aB  ]])  $.

  \item Case \[  \drule{instl-solveUG}    \]
    Immediate, since $[[DD]]$ and $[[GG]]$ are the same.

  \item Case \[  \drule{instl-reachSGOne}    \]

    Since $[[DD]]$ differs from $[[GG]]$ only in solving $[[sa]]$ and $[[gb]]$ to $[[ga]]$, and $[[ga]]$ is fresh,
    applying $[[DD]]$ to a type will not introduce $[[evarb]]$.  We have $[[evarb]]  \notin \textsc{fv}([[ [GG]aB  ]])  $, so
    $[[evarb]]  \notin \textsc{fv}([[ [DD]aB  ]])  $.

  \item Case \[  \drule{instl-reachSGTwo}    \]

    Since $[[DD]]$ differs from $[[GG]]$ only in solving $[[sb]]$ and $[[ga]]$ to $[[gb]]$, and $[[gb]]$ is fresh,
    applying $[[DD]]$ to a type will not introduce $[[evarb]]$.  We have $[[evarb]]  \notin \textsc{fv}([[ [GG]aB  ]])  $, so
    $[[evarb]]  \notin \textsc{fv}([[ [DD]aB  ]])  $.

  \item Case \rref*{instr-solveUS} is similar to case \rref*{instl-solveUS}.
  \item Case \rref*{instr-solveUG} is similar to case \rref*{instl-solveUG}.
  \item Case \rref*{instr-reachSG1} is similar to case \rref*{instl-reachSG1}.
  \item Case \rref*{instr-reachSG2} is similar to case \rref*{instl-reachSG2}.


  \end{itemize}

\end{proof}


\begin{lemma}[Instantiation Size Preservation] \label{lemma:dec:instan}
  If $[[GG]] = [[ GG0, evar, GG1 ]]$ and $[[  GG |- evar  <~~ aA -| DD ]]$
  or $[[ GG |- aA <~~ evar -| DD  ]]$, and $[[GG |- aB]]$ and $[[evar notin fv( [GG]aB )]]$, then $ | [[   [GG]aB   ]] | = | [[   [DD]aB   ]] | $, where $| [[   aC  ]] | $ is the plain size of $[[aC]]$.
\end{lemma}
\begin{proof}
  By induction on the given derivation. We show the new cases.
  \begin{itemize}
  \item Case \[ \drule{instl-solveUS} \]
    Since $[[DD]]$ differs $[[GG]]$ only in solving $[[sa]]$, and we know $[[ sa notin fv([GG]aB)  ]]  $, we have $[[  [DD]aB  ]] = [[  [GG]aB ]]$, so
    $ | [[   [GG]aB   ]] | = | [[   [DD]aB   ]] | $.

  \item Case \[  \drule{instl-solveUG}   \]
    Immediate, since $[[DD]]$ and $[[GG]]$ are the same.

  \item Case \[   \drule{instl-reachSGOne}   \]
    Since $[[DD]]$ differs $[[GG]]$ only in solving $[[sa]]$ and $[[gb]]$, and we know $[[ sa notin fv([GG]aB)  ]]  $,
    even if $[[gb]]$ occurs in $[[  [GG]aB]]$, its solution is again an existential variable, so the size does not change,
    so $ | [[   [GG]aB   ]] | = | [[   [DD]aB   ]] | $.

  \item Case \[   \drule{instl-reachSGTwo}   \]
    Since $[[DD]]$ differs $[[GG]]$ only in solving $[[ga]]$ and $[[sb]]$, and we know $[[ ga notin fv([GG]aB)  ]]  $,
    even if $[[sb]]$ occurs in $[[  [GG]aB]]$, its solution is again an existential variable, so the size does not change,
    so $ | [[   [GG]aB   ]] | = | [[   [DD]aB   ]] | $.

  \item Case \rref*{instr-solveUS} is similar to case \rref*{instl-solveUS}.
  \item Case \rref*{instr-solveUG} is similar to case \rref*{instl-solveUG}.
  \item Case \rref*{instr-reachSG1} is similar to case \rref*{instl-reachSG1}.
  \item Case \rref*{instr-reachSG2} is similar to case \rref*{instl-reachSG2}.

  \end{itemize}
\end{proof}


\decInstan*
\begin{proof}
  By induction on the derivation of $[[  GG |- aA   ]]$. We show the new cases.
  \begin{itemize}
  \item Case \[  \drule{ad-unknown}   \]
    By \rref{instl-solveUS} or \rref{instl-solveUG}.

  \item Case \[  \drule{ad-static}   \]
    By \rref{instl-solveS}.

  \item Case \[  \drule{ad-gradual}   \]

    By \rref{instl-solveS} or \rref{instl-solveG}.


  \item Case \[    \ottaltinferrule{ad-evar}{}{  }{ [[  GG0, sa, GG1 |- ga  ]] }  \]
    If $[[ga]] \in [[GG0]]  $, then we have a derivation by \rref{instl-reachOther}. If $[[ga]] \in [[GG1]]$, then
    we have a derivation by \rref{instl-reachSG1}.


  \item Case \[    \ottaltinferrule{ad-evar}{}{  }{ [[  GG0, ga, GG1 |- sa  ]] }  \]
    If $[[sa]] \in [[GG0]]  $, then we have a derivation by \rref{instl-reachSG2}. If $[[sa]] \in [[GG1]]$, then
    we have a derivation by \rref{instl-reachOther}.

  \end{itemize}


\end{proof}



\subsection{Decidability of Algorithmic Consistent Subtyping}


\begin{lemma}[Monotypes Solve Variables] \label{lemma:mono_solve_var}
  If $[[ GG |- evar <~~ at -| DD ]]$ or $[[ GG |- at <~~ evar -| DD  ]]$, then if $[[  [GG]at   ]] = [[at]]$ and
  $[[evar]] \notin \textsc{fv}([[ [GG] at  ]])$, then $| \textsc{unsolved}([[GG]])   | = |  \textsc{unsolved}([[DD]])      | + 1$.
\end{lemma}
\begin{proof}
  By induction on the given derivation. Since our syntax of monotypes differ
  from DK only in having static and gradual parameters, we show only two affected cases.
  \begin{itemize}
  \item Case \[  \drule{instl-solveS}  \]
    It is immediate that $|  \textsc{unsolved}([[GG, sa, GG']])    | = |  \textsc{unsolved}([[GG, sa = at, GG']])    | + 1$.
  \item Case \[  \drule{instl-solveG}  \]
    It is immediate that $|  \textsc{unsolved}([[GG, ga, GG']])    | = |  \textsc{unsolved}([[GG, ga = atc, GG']])    | + 1$.
  \end{itemize}
\end{proof}


\begin{lemma}[Monotype Monotonicity] \label{lemma:mono_mono}
  If $[[GG |- at1 <~ at2 -| DD]]$  then $| \textsc{unsolved}([[DD]]) | \leq |  \textsc{unsolved}([[GG]])    |$.
\end{lemma}
\begin{proof}
  By induction on the derivation. We show the new cases.
  \begin{itemize}
  \item Case \rref*{as-spar,as-gpar}: In these rules, $[[DD]] = [[GG]]$, so $| \textsc{unsolved}([[DD]]) | = |  \textsc{unsolved}([[GG]])    |$.
  \end{itemize}

\end{proof}


\begin{lemma}[Substitution Decreases Size] \label{lemma:subst_decrease}
  If $[[GG |- aA]]$, then $|[[  GG |-   [GG]aA    ]]| \leq  |  [[GG |-aA]]      |   $.
\end{lemma}
\begin{proof}
  By induction on $| [[ GG |- aA  ]] |$. We show the new cases.
  \begin{itemize}
  \item $[[aA]] = [[unknown]]$, or $[[aA]] = [[static]]$, or $[[aA]] = [[gradual]]$ then $[[  [GG] aA ]] = [[aA]]$. Therefore
    $|[[  GG |-   [GG]aA    ]]| =  |  [[GG |-aA]]      |   $.
  \end{itemize}
\end{proof}


\begin{lemma}[Monotype Context Invariance] \label{lemma:mono_inva}
  If $[[GG |- at <~ at' -| DD]]$ where $[[  [GG]at   ]] = [[at]]$ and $[[  [GG]at'   ]] = [[at']]$ and
  $|   \textsc{unsolved}([[GG]])      | = |  \textsc{unsolved}([[DD]])    |$, then $[[DD]] = [[GG]]$.
\end{lemma}
\begin{proof}
  By induction on the derivation. We show the new cases.
  \begin{itemize}
  \item Cases \rref*{as-spar,as-gpar}: In these rules, the output context is the
    same as the input context, so the result is immediate.
  \item Case \[  \drule{as-instL}   \]
    By \cref{lemma:mono_solve_var}, $|\textsc{unsolved}([[ DD ]])| < | \textsc{unsolved}([[ GG[evar]  ]])   |$, which is contrary to what is given,
    so this case is impossible.

  \item Case \rref*{as-instR} is similar to \rref*{as-instL}.
  \end{itemize}

\end{proof}

\decsubtype*
\begin{proof}
  Let the judgment $[[ GG |- aA <~ aB -| DD   ]]$ be measured lexicographically by
 \begin{enumerate}[(M1)]
\item the number of $\forall$-quantifiers in $[[aA]]$ and $[[aB]]$;
\item the number of unknown types in $[[aA]]$ and $[[aB]]$;
\item $| \textsc{unsolved}([[GG]]) |$: the number of unsolved existential
  variables in $[[GG]]$;
\item $| [[GG |- aA]] | + | [[GG |- aB]]   |$.
\end{enumerate}

We focus on the interesting (and new) cases.

\begin{asparaitem}
\item Cases \rref*{as-spar,as-gpar,as-unknownLL,as-unknownRR} have no premises.
\item Case \[  \drule{as-arrow}   \]
  We discuss each premise separately:
  \begin{description}
    \item[First premise:]
      If $[[aA2]]$ or $[[aB2]]$ has a quantifier, then the first premise is smaller by (M1). Otherwise, if $[[aA2]]$ or $[[aB2]]$
      has a unknown type, then first premise is smaller by (M2). Otherwise, the first premise shares the same input context as the conclusion, so it
      has the same (M3), but the types $[[aB1]]$ and $[[aA1]]$ are subterms of the conclusion's types, so the first premise is smaller by (M4).
    \item[Second premise:] If $[[aB1]]$ or $[[aA1]]$ has a quantifier, then the second premise is smaller by (M1) because
      applying contexts will not introduce quantifiers. Otherwise,
      if $[[aB1]]$ or $[[aA1]]$ has a unknown type, then the second premise is smaller by (M2) because
      applying contexts will not introduce unknown types. Otherwise, at this point, we know
      $[[aB1]]$ and $[[aA1]]$ are monotypes, so by \cref{lemma:mono_mono} on the first premise,
      we have $| \textsc{unsolved}([[TT]])  | \leq | \textsc{unsolved}([[GG]])   |$.
      \begin{itemize}
      \item If $| \textsc{unsolved}([[TT]])  | < | \textsc{unsolved}([[GG]])   |$, then the second premise is smaller by (M3).
      \item If $| \textsc{unsolved}([[TT]])  | = | \textsc{unsolved}([[GG]])   |$, then we have the same (M3). By \cref{lemma:mono_inva}
        on the first premise, we know  $[[TT]] = [[GG]]$, so $| [[TT |- [TT]aA2]] | = | [[GG |- [GG]aA2]] |$.
        By \cref{lemma:subst_decrease} we know $| [[GG |- [GG]aA2]] | \leq  | [[ GG |- aA2  ]] |  $. Therefore we have
        \[
          | [[TT |- [TT]aA2]] | \leq  | [[ GG |- aA2  ]] |
        \]
        Same for $[[aB2]]$:
        \[
          | [[TT |- [TT]aB2]] | \leq  | [[ GG |- aB2  ]] |
        \]
        Therefore,
        \[
          | [[TT |- [TT]aA2]] | + | [[TT |- [TT]aB2]] | \leq | [[ GG |- aA2  ]] | +  | [[ GG |- aB2  ]] | < | [[ GG |- aA1 -> aA2  ]] | + | [[ GG |- aB1 -> aB2  ]] |
        \]
        and the second premise is smaller by (M4).
      \end{itemize}
  \end{description}
\end{asparaitem}

\end{proof}

\subsection{Decidability of Algorithmic Typing}

\decmatch*
\begin{proof}
  \Rref{am-arr,am-unknown,am-var} do not have premises. For \rref{am-forall},
  the size of $[[aA]]$ is decreasing in the premise.
\end{proof}


\dectyping*
\begin{proof}

  We consider the following measure:
  \begin{center}
    \begin{tabular}{lllll}
      \multirow{2}{*}{$ \Big \langle$} & \multirow{2}{*}{$[[ae]],$} & $[[=>]]$ & \multirow{2}{*}{$| [[GG |- aA]] |$} & \multirow{2}{*}{$\Big \rangle$} \\
                                       &                    & $[[<=]],$ &  &
    \end{tabular}
  \end{center}
  and show every inference/checking premise is smaller than the conclusion.
  \begin{itemize}
  \item \Rref{inf-var,inf-int} do not have premises.
  \item \Rref{inf-anno,inf-lamann,inf-lam,inf-let,chk-lam} all have strictly
    smaller $[[ae]]$ in the premises.
  \item \Rref{inf-app}: The first and third premises have strictly smaller
    $[[ae]]$. The second (matching) judgment is decidable by
    \cref{lemma:decmatch}.
  \item \Rref{chk-gen}: Both the premise and conclusion type the same term, and
    both are the checking judgments. However $| [[GG, a |- aA]] | < | [[GG |-  \/a. aA]] |$, so
    the premise is smaller.
  \item \Rref{chk-sub}: The first premise uses inference mode, so it is smaller.
    The second premise is decidable by \cref{thm:decsubtype}.
  \end{itemize}

\end{proof}

\newpage



\section{Properties of Consistent Subtyping}


\begin{clemma}[Consistent Subtyping is Reflexive] \label{lemma:sub_refl}%
  If $[[ dd |- A  ]]$ then $[[ dd |- A <~ A  ]]$.
\end{clemma}


\begin{clemma}[Monotype Equality]   \label{lemma:mono_equal}
  If  $[[ dd |- t <~ s  ]]$  then $[[t]] = [[s]]$.
\end{clemma}

\begin{lemma}[Invertibility]  \label{lemma:forall_invert}
  If $ [[dd |- A <~ \/b . B]] $ then $[[ dd, b |- A <~ B ]]$.
\end{lemma}
\begin{proof}
  By induction on the given derivation.
  \begin{itemize}
  \item \Rref{cs-arrow,cs-tvar,cs-int,cs-unknownRR,cs-spar,cs-gpar} are impossible since
    the supertype is not a forall type.
  \item Case \[  \drule{cs-forallR}   \] The premise is exactly what we need.
  \item Case \[  \drule{cs-forallL}  \] where $B = [[  \/b . B0 ]]$. By i.h.,
    we have $[[ dd, b |- A [a ~> t] <~ B0 ]]$. By \rref{cs-forallL} we have
    $[[  dd , b |- \/a . A <~ B0 ]]$.
  \item Case \[ \drule{cs-unknownLL}   \] where $[[gc]] = [[\/b . gc0 ]]$. By \rref{cs-unknownLL}
    we have $[[ dd , b |- unknown <~ gc0 ]]$.
  \end{itemize}
\end{proof}


\newpage


\section{Properties of Context Extension}

\subsection{Syntactic Properties}


Since the definition of the context extension judgment ($[[GG --> DD]]$,
\cref{fig:ctxt-extension}) is exactly the same as that of the DK system, we
refer the reader to their technical report~\citep{dunfield2013complete} for the proofs of the following
syntactic properties of context extension.



\begin{lemma}[Reverse Declaration Order Preservation]   \label{lemma:reverse_preserve}
  If $\Gamma \exto \Delta$ and $a$ and $b$ are both declared in $\Gamma$ and $a$
  is declared to the left of $b$ in $\Delta$, then $a$ is declared to the left
  of $b$ in $\Gamma$.
\end{lemma}



\begin{lemma}[Reflexivity]   \label{lemma:reflexivity}
  If $[[GG]]$ is well-formed then $[[GG --> GG]]$.
\end{lemma}


\begin{lemma}[Transitivity]   \label{lemma:transitivity}
  If $[[GG --> DD]] $ and $[[DD --> TT]]$ then $[[ GG --> TT  ]]$.
\end{lemma}

\begin{definition}[Softness]
  A context $[[TT]]$ is soft iff it consists only of $[[evar]]$ and $[[evar = at]]$ declarations.
\end{definition}


\begin{lemma}[Substitution Extension Invariance]  \label{lemma:subst_ext_invar}
  If $[[TT |- aA]]$ and $[[TT --> GG]]$ then $[[ [GG]aA ]] = [[ [GG] ([TT]aA)]]  $ and $ [[ [GG]aA ]] = [[ [TT] ([GG]aA)]]   $.
\end{lemma}


\begin{lemma}[Extension Order] \label{lemma:extension_order}%
  We have the following:
  \begin{enumerate}
  \item If $[[ GL , a, GR --> DD]]$ then $[[DD]]  = [[(DL, a, DR)]]  $ where
    $[[GL --> DL]]$. Moreover, if $[[GR]]$ is soft then $[[DR]]$ is soft.
  \item If $[[ GL , mevar, GR --> DD]]$ then $[[DD]]  = [[(DL, mevar, DR)]]  $ where
    $[[GL --> DL]]$. Moreover, if $[[GR]]$ is soft then $[[DR]]$ is soft.
  \item If $[[GL, evar, GR --> DD]]$ then $[[DD]]  = [[(DL, TT, DR)]]  $ where $[[GL --> DL]]$ and $[[TT]]$ is either $[[evar]]$ or $[[evar]] = [[at]]$ for some $[[at]]$.
  \item If $[[GL, evar = at, GR --> DD]]$ then $[[DD]]  = [[(DL, evar = at', DR)]]  $ where $[[GL --> DL]]$ and and $[[ [DL]at  ]] = [[ [DL] at' ]]$.
  \item If $[[GL , x : aA , GR --> DD]]$ then $[[DD]]  = [[(DL, x : aA', DR)]]  $ where $[[GL --> DL]]$ and $[[ [DL]aA  ]] = [[ [DL] aA' ]]$. Moreover, $[[GR]]$ is soft if and only if $[[DR]]$ is soft.
  \end{enumerate}
\end{lemma}


\begin{lemma}[Solution Admissibility for Extension] \label{lemma:solution_ext}
  If $[[ GL |- at  ]]$ then $[[  GL, evar, GR --> GL, evar = at, GR ]]$.
\end{lemma}


\begin{lemma}[Unsolved Variable Addition for Extension]   \label{lemma:unsolved_ext}
  We have that $ [[GL, GR --> GL, evar, GR]] $
\end{lemma}


\begin{lemma}[Parallel Admissibility]   \label{lemma:paralell_admit}
  If $\Gamma_L \exto \Delta_L$ and $\Gamma_L, \Gamma_R \exto \Delta_L, \Delta_R$ then:
  \begin{enumerate}
  \item $\Gamma_L, \genA, \Gamma_R \exto \Delta_L, \genA, \Delta_R$
  \item If $\Delta_L \vdash \tau'$ then $\Gamma_L, \genA, \Gamma_R \exto \Delta_L, \genA = \tau', \Delta_R$.
  \item If $\Gamma_L \vdash \tau$ and $\Delta_L \vdash \tau'$ and $\ctxsubst{\Delta_L}{\tau} = \ctxsubst{\Delta_L}{\tau'}$, then $\Gamma_L, \genA = \tau, \Gamma_R \exto \Delta_L, \genA = \tau', \Delta_R$.
  \end{enumerate}
\end{lemma}

\begin{lemma}[Parallel Extension Solution] \label{lemma:paralell_ext_solu}
  If $\Gamma_L, \genA, \Gamma_R \exto \Delta_L, \genA = \tau', \Delta_R$ and
  $\Gamma_L \vdash \tau$ and $\ctxsubst{\Delta_L}{\tau} =
  \ctxsubst{\Delta_L}{\tau'}$, then $\Gamma_L, \genA = \tau, \Gamma_R \exto
  \Delta_L, \genA = \tau', \Delta_R$.
\end{lemma}




\begin{lemma}[Drop Variable for Extension]   \label{lemma:drop_ext}
  If $[[ GG, evar --> DD  ]]$ then $[[ GG --> DD]]$.
\end{lemma}


\begin{lemma}[Finishing Types]
  If $\Omega \vdash A$ and $\Omega \exto \Omega'$ then $\ctxsubst{\Omega}{A} = \ctxsubst{\Omega'}{A}$.
  \label{lemma:finish_types}
\end{lemma}


\begin{lemma}[Finishing completions]
  If $\Omega \exto \Omega'$ then $\ctxsubst{\Omega}{\Omega} = \ctxsubst{\Omega'}{\Omega'}$.
  \label{lemma:finish_complete}
\end{lemma}



\begin{lemma}[Confluence of Completeness]   \label{lemma:confluence}
  If $[[  DD1 --> OO]]$ and $[[ DD2 --> OO]]$ then
  $[[ [OO]DD1 ]]  =  [[ [OO] DD2  ]]$.
\end{lemma}


\begin{lemma}[Variable Preservation]
  If $(x : A) \in \Delta$ or $(x : A) \in \Omega$ and $\Delta \exto \Omega$ then $(x : \ctxsubst{\Omega}{A}) \in \ctxsubst{\Omega}{\Delta}$.
  \label{lemma:variable_preservation}
\end{lemma}


\begin{lemma}[Softness Goes Away]
  If $\Delta, \Theta \exto \Omega, \Omega_Z$ where $\Delta \exto \Omega$ and $\Theta$ is soft, then $\ctxsubst{\Omega, \Omega_Z}{(\Delta, \Theta)} = \ctxsubst{\Omega}{\Delta}$.
  \label{lemma:subst_go_away}
\end{lemma}

\begin{lemma}[Stability of Complete Contexts]
  If $\Gamma \exto \Omega$ then $\ctxsubst{\Omega}{\Gamma} = \ctxsubst{\Omega}{\Omega}$.
  \label{lemma:stable_complete_ctxt}
\end{lemma}



\subsection{Instantiation Extends}

\begin{lemma}[Instantiation Extension]   \label{lemma:inst_extension}
 If $[[  GG |- evar <~~ aA  -| DD   ]]$ or $[[ GG |- aA <~~ evar -| DD  ]]$ then $[[ GG --> DD ]]$.
\end{lemma}
\begin{proof}
  By induction on the given instantiation derivation.
  \begin{itemize}
  \item \Rref{instl-solveS,instl-solveG,instl-reachOther,instr-solveS,instr-solveG,instr-reachOther}
    are immediate from \cref{lemma:solution_ext}.

  \item Case \[ \drule{instl-solveUS} \]
    By \cref{lemma:unsolved_ext} we have $[[ GG[sa] --> GG[ga, sa]  ]]$. By \cref{lemma:solution_ext} we have
    $[[ GG[ga, sa] --> GG[ga, sa = ga]  ]]$. By \cref{lemma:transitivity} we have $[[ GG[sa] --> GG[ga, sa=ga]   ]]$.

  \item Case \[ \drule{instl-solveUG} \] Immediate by \cref{lemma:reflexivity}.


  \item Case \[  \drule{instl-reachSGOne}  \]
    By \cref{lemma:unsolved_ext} we have $[[  GG[sa][gb] --> GG[ga,sa][gb]  ]]$. By applying \cref{lemma:solution_ext} twice,
    we have $[[ GG[ga,sa][gb] --> GG[ga,sa=ga][gb=ga]    ]]$. By \cref{lemma:transitivity} we have $[[ GG[sa][gb] --> GG[ga,sa=ga][gb=ga]  ]]$.

  \item Case \[  \drule{instl-reachSGTwo}   \] Same as the case for \rref{instl-reachSG1}.

  \item Case \[ \drule{instl-arr}\]
    By applying \cref{lemma:unsolved_ext} twice, we have $[[  GG[evar] --> GG[evar2, evar1, evar]  ]]$.
    By \cref{lemma:solution_ext} we have $[[ GG[evar2, evar1, evar] --> GG[evar2, evar1, evar = evar1 -> evar2]  ]]$.
    By i.h., we have $[[ GG[evar2, evar1, evar = evar1 -> evar2] --> TT  ]]$ and $[[ TT --> DD  ]]$. By \cref{lemma:transitivity} we
    have $[[  GG[evar] --> DD  ]]$.

  \item Case \[ \drule{instl-forallR}  \]
    By i.h., we have $[[ GG[evar], b --> DD , b , TT  ]]$. By \cref{lemma:extension_order} (1), we have $[[GG[evar] --> DD]]$.

  \item Case \[ \drule{instr-solveUS}  \] Same as the case for \rref{instl-solveUS}.

  \item Case \[  \drule{instr-solveUG}\] Same as the case for \rref{instl-solveUG}.

  \item Case \[  \drule{instr-reachSGOne}  \]  Same as the case for \rref{instl-reachSG1}.

  \item Case \[  \drule{instr-reachSGTwo}  \]  Same as the case for \rref{instl-reachSG1}.

  \item Case \[ \drule{instr-arr} \] Same as the case for \rref{instl-arr}.

  \item Case \[ \drule{instr-forallLL}   \]
    By i.h., we have $[[ GG[evar] , mevarb, sb --> DD, mevarb, TT ]]$. By \cref{lemma:extension_order}(2) we have $[[ GG[evar] --> DD  ]]$.

  \end{itemize}
\end{proof}

\subsection{Consistent Subtyping Extends}

\begin{lemma} \label{lemma:contamination_extension}
  If $[[GG |- aA ]]$ then $[[GG --> [aA] GG]]$.
\end{lemma}
\begin{proof}
  By induction on the structure of $[[GG]]$. The only interesting case is when $[[GG]] = [[GG' , sa]] $. By \cref{def:contamination},
  we have $[[ [aA] (GG' , sa) ]] = [[ [aA] GG' , ga, sa = ga]] $. By i.h., we have $[[ GG' --> [aA] GG'  ]]$.
  By definition of context extension we have $[[ GG' , sa --> [aA] GG' , sa  ]]$. By \cref{lemma:unsolved_ext} we have
  $[[ [aA] GG' , sa --> [aA] GG' , ga, sa ]]$. By \cref{lemma:solution_ext} we have $[[ [aA] GG' , ga, sa --> [aA] GG' , ga, sa = ga  ]]$.
  By \cref{lemma:transitivity} we have $[[ GG' , sa --> [aA] GG' , ga, sa = ga ]]$.
\end{proof}

\begin{lemma}[Consistent Subtyping Extension]   \label{lemma:sub_extension}
  If $ [[GG |- aA <~ aB -| DD]]  $ then $ [[GG --> DD]]$.
\end{lemma}
\begin{proof}
  By induction on the derivation of consistent subtyping.

  \begin{itemize}
  \item \Rref{as-tvar,as-evar,as-int,as-spar,as-gpar} are immediate from \cref{lemma:reflexivity}.

  \item Case \[ \drule{as-arrow} \]
    By i.h., we have $[[GG --> TT]]$ and $[[TT --> DD]]$. By \cref{lemma:transitivity}, we have $[[GG --> DD]]$.

  \item Case \[ \drule{as-forallR} \]
    By i.h., we have $[[GG , a --> DD , a , TT]]$. By \cref{lemma:extension_order} (1), we have $[[GG --> DD]]$.


  \item Case \[ \drule{as-forallLL} \]
    By i.h., we have $[[GG, mevar, sa --> DD, mevar, TT]]$. By \cref{lemma:extension_order} (2), we have $[[GG --> DD]]$.


  \item Case \[ \drule{as-unknownLL}  \] Immediate by \cref{lemma:contamination_extension}.

  \item Case \[ \drule{as-unknownRR}  \] Immediate by \cref{lemma:contamination_extension}.


  \item \Rref{as-instL,as-instR} are immediate.

  \end{itemize}

\end{proof}


\newpage

\section{Soundness of Consistent Subtyping}


\begin{definition}[Filling]   \label{def:filling}
  The filling of a context $[[ | GG | ]]$ solves all unsolved variables:
  \begin{align*}
    [[| empty |]] &= [[ empty ]] \\
    [[|GG , x : aA|]] &= [[|GG| , x : aA]] \\
    [[|GG, a|]] &= [[|GG| , a]] \\
    [[|GG, evar = at|]] &= [[|GG| , evar = at]] \\
    [[|GG, evar|]] &= [[|GG| , evar = int]] \\
    [[|GG, mevar|]] &= [[|GG| , mevar]]
  \end{align*}
\end{definition}


\begin{lemma}[Substitution Stability]
  For any well-formed complete context $(\Omega, \Omega_Z)$, if $\Omega \vdash A$
  then $\ctxsubst{\Omega}{A} = \ctxsubst{\Omega,\Omega_Z}{A}$.
  \label{lemma:subst_stable}
\end{lemma}
\begin{proof}
  By induction on $\Omega_Z$. If $\Omega_Z = [[empty]]$, the result is
  immediate. Otherwise use the i.h. and the fact that $\Omega \vdash A$ implies
  $\textsc{fv}(A) \cap \textsc{dom}(\Omega_Z) = \emptyset$.
\end{proof}



\begin{lemma}[Filling Completes]   \label{lemma:filling_completes}
  If $[[GG --> OO]]$ and $[[(GG, TT)]]$ is well-formed, then $[[GG , TT --> OO, | TT |]]$.
\end{lemma}
\begin{proof}
  By induction on $[[TT]]$, following \cref{def:filling} and applying the rules for context extension.
\end{proof}

\instsoundness*
\begin{proof}
  By induction on the given instantiation derivation.
  \begin{itemize}
  \item Case \[ \drule{instl-solveS} \] Immediate from \cref{lemma:sub_refl}.

  \item Case \[ \drule{instl-solveG} \] Immediate from \cref{lemma:sub_refl}.

  \item Case \[ \drule{instl-solveUS} \]
    We know $[[ [OO] sa ]] = [[tc]] $ for some castable monotype $[[tc]]$, and $[[tc]] \in [[gc]]$.
    By \rref{cs-unknownRR}, we have $[[  [OO] (GG[sa]) |- tc <~ unknown  ]]$


  \item Case \[  \drule{instl-solveUG} \] Similar to the case for \rref{instl-solveUS}.

  \item Case \[ \drule{instl-reachSGOne}  \]
    We know $[[ [OO] sa ]] = [[ [OO]ga]] =  [[tc]] $ and $[[ [OO] gb  ]] = [[ [OO]ga]] = [[tc]]$ for some castable monotype $[[tc]]$.
    By \cref{lemma:sub_refl} we have $[[  [OO](GG[sa][gb] ) |- tc <~ tc   ]]$.

  \item Case \[  \drule{instl-reachSGTwo}   \] Similar to the case for \rref{instl-reachSG1}.

  \item Case \[ \drule{instl-reachOther}  \]
    Let $[[DD]] = [[GG[evar][evarb]  ]]$, we have $[[ [OO]evar ]] = [[t]] $ and $[[  [OO] evarb  ]]  = [[  [OO] evar ]] = [[t]] $
    for some monotype $[[t]]$. By \cref{lemma:sub_refl} we have $[[ [OO]DD |- t <~ t  ]]$.

  \item Case \[ \drule{instl-arr}   \]
    Let $[[GG1]] = [[ GG[evar2, evar1, evar = evar1 -> evar2]  ]]    $:
    \begin{longtable}[l]{ll|l}
      & $[[TT |- evar2 <~~ [TT]aA2 -| DD]]$ & Premise \\
      & $[[ TT --> DD ]]$ & By \cref{lemma:inst_extension} \\
      & $[[ DD --> OO ]]$ & Given \\
      & $[[TT --> OO]]$ & By \cref{lemma:transitivity} \\
      & $[[ GG1 |- aA1 <~~ evar1 -| TT]]$ & Given \\
      & $[[  [OO]DD |- [OO]aA1 <~ [OO]evar1 ]]$ & By i.h. and \Cref{lemma:confluence} \\
      & $[[TT |- evar2 <~~ [TT]aA2 -| DD]]$& Premise \\
      & $[[ [OO]DD |- [OO]evar2 <~ [OO] ([TT]aA2)  ]]$ & By i.h. \\
      & $[[  TT --> DD ]]$ & Above \\
      & $[[ [OO]DD |- [OO]evar2 <~ [OO] aA2  ]]$ & By \cref{lemma:subst_ext_invar} \\
      & $[[  [OO] DD |- [OO] evar1 -> [OO] evar2 <~ [OO]aA1 -> [OO]aA2  ]]$ & By \rref{cs-arrow} \\
      & $[[  [OO] DD |- [OO] evar <~ [OO] (aA1 -> aA2)  ]] $ & By def. of substitution
    \end{longtable}

  \item Case \[ \drule{instl-forallR}  \]
    \begin{longtable}[l]{ll|l}
      & $[[DD , b, TT -->  OO , b , |TT|]]  $ & By \cref{lemma:filling_completes} \\
      & $[[GG[evar] , b |- evar <~~ aB -| DD , b , TT]]$ & Given \\
      & $[[ [OO , b , |TT|](DD , b , TT)  |- [OO , b , |TT|] evar <~ [OO , b , |TT|] aB    ]]$ & By i.h. \\
      & $[[ [OO , b , |TT|](DD , b , TT)  |- [OO , b ] evar <~ [OO , b ] aB    ]]$ & Free variables in $[[evar]]$ and $[[aB]]$ are declared in $[[(OO, b)]]$\\
      & $[[ [OO , b ](DD , b )  |- [OO , b ] evar <~ [OO , b ] aB    ]]$ & By context partitioning and thinning \\
      & $[[ [OO]DD , b   |- [OO] evar <~ [OO] aB    ]]$ & By context substitution \\
      & $[[ [OO]DD    |- [OO] evar <~ \/b . [OO] aB    ]]$ & By \rref{cs-forallR} \\
       & $[[ [OO]DD    |- [OO] evar <~ [OO] (\/b . aB)    ]]$ & By def. of substitution
    \end{longtable}

  \item Case \[ \drule{instr-forallLL}  \]

    \begin{longtable}[l]{ll}
      & $[[DD , mevarb, TT -->  OO , mevarb , |TT|]]  $ \qquad By \cref{lemma:filling_completes} \\
      & $[[GG[evar] , mevarb, sb |- aB[b ~> sb] <~~ evar -| DD, mevarb, TT]]$ \qquad Premise \\
      & $[[ [OO , mevarb , |TT|] (DD , mevarb, TT) |- [OO , mevarb , |TT|] (aB[b ~> sb]) <~ [OO , mevarb , |TT|]evar ]]$ \qquad By i.h. \\
      & $[[ [OO] DD |- ([OO]aB) [b ~> [OO , mevarb , |TT|] sb] <~ [OO]evar ]]$ \qquad By distributivity of substitution \\
      & $[[  [OO] DD |- [OO, mevarb, |TT|]sb  ]]$ \qquad Follows from def. of context application \\
      & $[[  [OO] DD |- \/ b . [OO] aB  <~ [OO] evar   ]]$ \qquad By \rref{cs-forallL} and $[[ [OO, mevarb, |TT| ] sb ]]$ is a monotype  \\
       & $[[  [OO] DD |- [OO]  (\/ b . aB)  <~ [OO] evar   ]]$ \qquad By def. of substitution
    \end{longtable}

  \item The rest of the cases are similar to the above cases.

  \end{itemize}
\end{proof}


\subsoundness*
\begin{proof}
  By induction on the derivation of consistent subtyping.

  \begin{itemize}
  \item Case \[ \drule{as-tvar}  \]

    \begin{longtable}[l]{ll|l}
      & $ [[a in GG[a] ]]  $ & Given \\
      & $ [[ a in [OO](GG[a])  ]]   $ & Follows from def. of context application \\
      & $ [[ [OO](GG[a]) |- a <~ a    ]]  $ & By \rref{cs-tvar} \\
      & $ [[ [OO](GG[a]) |- [OO]a <~ [OO]a    ]]     $ & By def. of substitution
    \end{longtable}


  \item Case \[  \drule{as-evar}   \]

    \begin{longtable}[l]{ll|l}
      & $ [[ [OO]evar  ]]  $ defined & Follows from def. of context application \\
      & $  [[ [OO]DD |- [OO]evar  ]]  $ & Follows from $ [[DD]] = [[ [GG]evar  ]]  $ \\
      & $  [[ [OO]DD |- [OO]evar <~ [OO]evar    ]]  $ & By \cref{lemma:sub_refl}
    \end{longtable}

  \item Case \[  \drule{as-int}  \] Immediate.


  \item Case \[  \drule{as-arrow}    \]

    \begin{longtable}[l]{ll|l}
      & $  [[ GG |- aB1 <~ aA1 -| TT   ]]  $ & Premise \\
      & $   [[ DD --> OO   ]]  $ & Given \\
      & $   [[ TT --> OO       ]]$ & By \cref{lemma:transitivity} \\
      & $[[  [OO] TT |- [OO]aB1 <~ [OO]aA1      ]]$ & By i.h. \\
      & $[[  [OO] DD |- [OO]aB1 <~ [OO]aA1      ]]$ & By \cref{lemma:confluence} \\
      & $ [[ TT |- [TT] aA2 <~ [TT] aB2 -| DD]]   $ & Premise \\
      & $ [[ [OO]DD |- [OO]([TT] aA2) <~ [OO]([TT] aB2) ]]   $ & By i.h. \\
      & $ [[ [OO]([TT] aA2)  ]] = [[ [OO]aA2  ]]  $ & By \cref{lemma:subst_ext_invar} \\
      & $ [[ [OO]([TT] aB2)  ]] = [[ [OO]aB2  ]]  $ & By \cref{lemma:subst_ext_invar} \\
      & $ [[ [OO]DD |- [OO]aA2 <~ [OO]aB2 ]]    $ & By above equalities \\
      & $  [[ [OO]DD |-[OO]aA1 -> [OO]aA2 <~[OO]aB1 -> [OO]aB2 ]]     $ & By \rref{cs-arrow} \\
      & $  [[ [OO]DD |-[OO](aA1 -> aA2)  <~[OO](aB1 -> aB2) ]]       $ & By def. of substitution
    \end{longtable}


  \item Case \[ \drule{as-forallR}  \]

      \begin{longtable}[l]{ll|l}
      & $  [[GG, a --> DD, a , TT]]   $ & By \cref{lemma:sub_extension} \\
      & $[[TT]]$ is soft & By \cref{lemma:extension_order} (1) where $[[GR]] = [[empty]]$ \\
      & $[[ DD --> OO ]]$ & Given \\
      & $\underbrace{[[DD , a, TT]]}_{[[DD']]} [[-->]] \underbrace{  [[OO , a, |TT| ]]   }_{[[OO']]}$ & By \cref{lemma:filling_completes} \\
      & $  [[ GG, a |- aA <~ aB -| DD, a, TT ]]   $ & Given \\
      & $ [[ [OO']DD' |- [OO']aA <~ [OO']aB  ]]  $ & By i.h. \\
      & $ [[ [OO']aA  ]] = [[ [OO , a] aA  ]]   $ & By \cref{lemma:subst_stable} \\
      & $ [[ [OO']aB  ]] = [[ [OO , a] aB  ]]   $ & By \cref{lemma:subst_stable} \\
      & $ [[ [OO']DD'   ]]  = [[ [OO, a](DD, a)   ]] $ & By \cref{lemma:subst_go_away} \\
      & $[[ [OO, a](DD, a) |- [OO, a]aA <~ [OO, a]aB  ]]$ & By above equalities \\
      & $ [[ [OO]DD, a |- [OO]aA <~ [OO]aB  ]]    $ & By def. of substitution \\
      & $ [[ [OO]DD |- [OO]aA <~ \/a. [OO]aB  ]]  $ & By \rref{cs-forallR} \\
      & $ [[ [OO]DD |- [OO]aA <~ [OO] (\/a. aB)  ]]   $ & By def. of substitution
      \end{longtable}


    \item Case \[  \drule{as-forallLL}  \]

      \begin{longtable}[l]{ll|l}
        & Let $[[OO']] = [[OO, msa, |TT|]]$ \\
        & $[[  DD --> OO  ]]$ & Given \\
        & $[[DD, msa, TT --> OO']]$  & By \cref{lemma:filling_completes} \\
        & $[[ GG, msa, sa |- aA [a ~> sa] <~ aB -| DD, msa, TT]]$ & Premise \\
        & $[[ [OO'](DD, msa, TT) |- [OO'](aA [a ~> sa]) <~ [OO']aB     ]]$ & By i.h. \\
        & $ [[ [OO']aB  ]] = [[ [OO , a] aB  ]]   $ & By \cref{lemma:subst_stable} \\
        & $ [[ [OO](DD, msa, TT) |-  [OO'] aA [a ~> [OO']sa] <~ [OO]aB     ]]  $ & By distributivity of substitution \\
        & $[[ [OO](DD, msa, TT) |- [OO']sa    ]]$ & Follows from def. of context application \\
        & $ [[ [OO](DD, msa, TT) |- \/a. [OO'] aA <~ [OO]aB     ]]  $ & By \rref{cs-forallL} \\
        & $ [[ [OO]DD |- \/a. [OO] aA <~ [OO]aB     ]]  $ & By context partitioning \\
        & $[[ [OO]DD |- [OO] (\/a. aA) <~ [OO]aB     ]] $ & By def. of substitution
      \end{longtable}

    \item Case \[  \drule{as-spar}  \] Immediate from \rref{cs-spar}.

    \item Case \[  \drule{as-gpar}  \] Immediate from \rref{cs-gpar}.


    \item Case \[ \drule{as-unknownLL}   \] Immediate from \rref{cs-unknownLL}.

    \item Case \[ \drule{as-unknownRR}   \] Immediate from \rref{cs-unknownRR}.

    \item Case \[  \drule{as-instL}     \]
      \begin{longtable}[l]{ll|l}
        & $[[ GG[evar] |- evar <~~ aA -| DD]]$ & Premise \\
        & $[[  [OO]DD |- [OO]evar <~ [OO]aA   ]]$ & By \cref{thm:inst_soundness}
      \end{longtable}

    \item Case \[  \drule{as-instR}     \] Similar to the case for \rref{as-instL}.

  \end{itemize}
\end{proof}


\newpage

\section{Soundness of Typing}

\textbf{Note:} We use $\byhave$ to improve readability when the conclusion has several parts.


\begin{lemma}[Matching Extension]   \label{lemma:match_extension}
  If $[[  GG |- aA |> aA1 -> aA2 -| DD]]$ then $ [[GG --> DD]]  $.
\end{lemma}
\begin{proof}
  By induction on the given derivation.
  \begin{itemize}
  \item Case \[ \drule{am-forallL} \]
    By i.h., we have $[[ GG, sa --> DD  ]]$. By \cref{lemma:drop_ext}, we have $[[GG --> DD]]$.

  \item Case \[  \drule{am-arr}  \] Immediate by \cref{lemma:reflexivity}.

  \item Case \[ \drule{am-unknown}   \] Immediate by \cref{lemma:reflexivity}.

  \item Case \[ \drule{am-var} \]
    By applying \cref{lemma:unsolved_ext} twice, we have
    $ [[ GG[evar] --> GG[evar1, evar2, evar]  ]]   $. By
    \cref{lemma:solution_ext}, we have $[[ GG[evar1, evar2, evar] --> GG[evar1, evar2, evar = evar1 -> evar2]  ]]$. By
    \cref{lemma:transitivity}, we have $[[ GG[evar] --> GG[evar1, evar2, evar = evar1 -> evar2]  ]]$.
  \end{itemize}

\end{proof}


\begin{lemma}[Typing Extension]   \label{lemma:typing_extension}
  If $ [[GG |- ae => aA -| DD]] $ or $ [[ GG |- ae <= aA -| DD  ]]  $ then $[[ GG --> DD]]$.
\end{lemma}
\begin{proof}
  By induction on the given derivation.

  \begin{itemize}
  \item Case \[ \drule{inf-var}  \] Immediate by \cref{lemma:reflexivity}.

  \item Case \[  \drule{inf-int}  \] Immediate by \cref{lemma:reflexivity}.

  \item Case \[ \drule{inf-lamTwo}   \]
    By i.h., we have $[[ GG, sa, sb, x : sa --> DD, x : sa, TT  ]]$. By \cref{lemma:extension_order}, we have
    $[[ GG, sa, sb --> DD  ]]$. By definition, we have $[[  GG --> GG, sa, sb  ]]$. By \cref{lemma:transitivity}
    we have $[[ GG --> DD ]]$.

  \item Case \[  \drule{inf-lamannTwo}  \]
    By i.h., we have $[[ GG, sb, x : aA --> DD, x : aA, TT  ]]   $. By \cref{lemma:extension_order},
    we have $[[GG --> DD]]$.

  \item Case \[ \drule{inf-app}   \]
    By i.h., we have $[[  GG --> TT1  ]]$, $[[ TT2 --> DD]]$. By \cref{lemma:match_extension},
    we have $[[ TT1 --> TT2  ]]$. By \cref{lemma:transitivity} , we have
    $[[GG --> DD]]$.


  \item Case \[ \drule{inf-anno}   \]
    By i.h., we have $[[ GG --> DD ]]$.

  \item Case \[ \drule{chk-gen}   \]
    By i.h., we have $[[  GG, a --> DD, a, TT  ]]$. By
    \cref{lemma:extension_order} we have $[[ GG --> DD]]$.

  \item Case \[ \drule{chk-lam} \]
    By i.h., we have $[[GG , x : aA --> DD, x : aA, TT]]$.
    By \cref{lemma:extension_order} we have $[[GG --> DD]]$.

  \item Case \[ \drule{chk-sub} \]
    By i.h., we have $[[GG --> TT]]$. By
    \cref{lemma:sub_extension} we have $[[TT --> DD]]$. By
    \cref{lemma:transitivity} we have $[[GG --> DD]]$.
  \end{itemize}

\end{proof}

\begin{theorem}[Matching Soundness]  \label{thm:match_soundness}%
  If $[[GG |- aA |> aA1 -> aA2 -| DD]]$ where $[[ [GG]aA = aA  ]]$ and $[[ DD --> OO ]]$
  then $[[  [OO]DD |- [OO]aA |> [OO]aA1 -> [OO]aA2 ]]$.
\end{theorem}
\begin{proof}
  By induction on the given derivation.

  \begin{itemize}
  \item Case \[ \drule{am-forallL}  \]
    \begin{longtable}[l]{ll|l}
      & $[[ GG , sa |- aA[a ~> sa] |> aA1 -> aA2 -| DD]]$ & Premise \\
      & $[[DD --> OO]]$ & Given \\
      & $[[ [OO]DD |- [OO](aA[a ~> sa]) |> [OO]aA1 -> [OO]aA2   ]]$ & By i.h. \\
      & $[[ [OO]DD |- [OO]aA [a ~> [OO]sa] |> [OO]aA1 -> [OO]aA2   ]]$ & By distributivity of substitution \\
      & $[[ [OO]DD |- [OO]sa   ]]$ & Follows from def. of context application \\
      & $[[ [OO]DD |- \/a. [OO]aA |> [OO]aA1 -> [OO]aA2   ]]$ & By \rref{m-forall} \\
      & $[[ [OO]DD |- [OO] (\/a. aA) |> [OO]aA1 -> [OO]aA2   ]]$ & By def. of substitution
    \end{longtable}


  \item Case \[  \drule{am-arr}  \] Immediate from \rref{m-arr}.

  \item Case \[  \drule{am-unknown}  \] Immediate from \rref{m-unknown}.

  \item Case \[  \drule{am-var}  \]
      \begin{longtable}[l]{ll|l}
        &$[[DD --> OO]]$& Given \\
        & $[[ [OO]evar  ]] = [[ [OO]evar1 -> [OO]evar2   ]]$ & By def. of context application \\
        & $[[  [OO]DD |- [OO]evar1 -> [OO]evar2 |> [OO]evar1 -> [OO]evar2  ]]$ & By \rref{m-arr}
      \end{longtable}

  \end{itemize}

\end{proof}


\typingsoundness*
\begin{proof}
  By induction on the given derivation.

  \begin{itemize}
  \item Case \[ \drule{inf-var}  \]
    \begin{longtable}[l]{ll|l}
      &$[[(x : aA) in GG]]$ & Premise \\
      &$[[(x : aA) in  DD]]$ & $[[DD]] = [[OO]]$ \\
      &$[[DD --> OO]]$ & Given \\
      &$[[ (x : [OO]aA) in [OO]GG  ]]$ & By \Cref{lemma:variable_preservation} \\
      $\byhave$&$[[ [OO]GG |- x : [OO]aA   ]]$ & By \rref{var} \\
      $\byhave$& $\erase{[[x]]} = \erase{[[x]]}$ & By def. of erasure
    \end{longtable}

  \item Case \[ \drule{inf-int}  \]
    \begin{longtable}[l]{ll|l}
      $\byhave$&$[[ [OO]GG |- n : int  ]]$ & By \rref{int} \\
      $\byhave$& $\erase{[[n]]} = \erase{[[n]]}$ & By def. of erasure
    \end{longtable}

  \item Case \[ \drule{inf-lamannTwo}  \]

    \begin{longtable}[l]{ll|l}
      & $[[  GG , sa, x : aA --> DD, x : aA , TT  ]]$ & By \Cref{lemma:typing_extension} \\
      & $[[TT]]$ is soft & By \Cref{lemma:extension_order} where $[[GR]] = [[empty]]$ \\
      & $[[ DD --> OO]]$ & Given \\
      & $\underbrace{[[DD , x : aA , TT]]}_{[[DD']]} [[-->]] \underbrace{[[OO, x : aA, |TT|]] }_{[[OO']]}$ & By \Cref{lemma:filling_completes} \\
      & $[[ GG, x: aA |- ae => aB -| DD, x : aA, TT ]]$ & Premise \\
      & $[[  [OO'] DD' |- e' : [OO']sb   ]]$ & By i.h. \\
      & $\erase{[[ae]]} = \erase{e'}$ & above \\
      & $[[ [OO']sb   ]] = [[ [OO, x : aA]sb  ]] = [[ [OO]sb  ]]$ & By \Cref{lemma:subst_stable} and def. of substitution \\
      & $[[ [OO']DD' ]]$ = $[[ [OO]DD , x : [OO]aA   ]]$ & By \Cref{lemma:subst_go_away} and def. of context substitution \\
      & $[[  [OO] DD, x : [OO]aA |- e' : [OO]sb   ]] $ & By above equalities \\
      & $[[  [OO] DD |- \ x : [OO]aA . e' : [OO]aA -> [OO]sb   ]] $ & By \rref{lamann} \\
      & $[[ [OO]aA  ]] = [[A]]$ & Type annotations cannot contain evars \\
      & $[[  [OO] DD |- \ x : A . e' :[OO]aA -> [OO]sb   ]]$ & By above equality \\
      $\byhave$& $ [[  [OO] DD |- \ x : A . e' :[OO](aA -> sb)   ]] $ & By def. of substitution \\
      $\byhave$& $\erase{[[\x : A . e']]} = \erlam{x}{\erase{e'}} = \erlam{x}{\erase{e}} = \erase{\blam{x}{A}{e}} $ & By def. of erasure
    \end{longtable}



  \item Case \[  \drule{inf-lam}  \]

    \begin{longtable}[l]{ll|l}
      &$[[ GG, sa, sb, x : sa --> DD, x : sa, TT   ]]$& By \cref{lemma:typing_extension} \\
      &$[[ GG , sa, sb --> DD  ]]$& By \cref{lemma:extension_order} \\
      &$[[TT]]$ is soft & Above \\ \\
      &$[[ DD --> OO  ]]$  & Given \\
      &$[[  DD , x : sa --> OO , x : [OO]sa  ]]$  & By def \\
      &$\underbrace{[[DD, x : sa, TT]]}_{[[DD']]} [[-->]] \underbrace{ [[OO, x : [OO]sa, |TT|]] }_{[[OO']]}$  & By \cref{lemma:filling_completes} \\ \\

      & $ [[  GG, sa, sb, x : sa |- ae <= sb -| DD, x : sa, TT ]] $ & Premise \\
      & $[[ [OO']DD' |- e' : [OO']sb   ]]$ & By i.h. \\
      & $\erase{[[ae]]} = \erase{[[e']]}$ & Above \\
      & $[[ [OO']sb  ]] = [[ [OO]sb  ]]$ & By def. of context substitution \\
      & $[[  [OO']DD'  ]] = [[ [OO]DD, x : [OO]sa  ]]  $ & By def. of context substitution \\
      & $[[ [OO]DD, x : [OO]sa |- e' : [OO]sb   ]]$ & By above equalities \\
      & $[[ [OO]sa ]]$ is a monotype & $\Omega$ is predicative \\
      & $[[ [OO]DD |-\x .  e' : [OO]sa -> [OO]sb   ]] $ & By \rref{lam} \\
      $\byhave$& $[[ [OO]DD |-\x .  e' : [OO](sa -> sb)  ]]$ & By def. of substitution \\
      $\byhave$& $\erase{\erlam{x}{e}} = \erlam{x}{\erase{e}} = \erlam{x}{\erase{e'}} = \erase{\erlam{x}{e'}}$ & By def. of erasure
    \end{longtable}



  \item Case \[  \drule{inf-app}  \]

    \begin{longtable}[l]{ll|l}
      & $[[ DD --> OO  ]]$ & Given \\
      & $[[TT1 --> OO]]$ & By \cref{lemma:match_extension,lemma:typing_extension,lemma:transitivity} \\
      & $[[ GG |- ae1 => aA -| TT1]]$ & Premise \\
      & $[[ [OO]TT1 |- e1' : [OO]aA    ]]$ & By i.h. \\
      & $\erase{[[e1']]} = \erase{[[ae1]]}$ & above \\
      & $[[ [OO]TT1  ]] = [[ [OO]DD  ]]$ & By \Cref{lemma:confluence} \\
      & $[[ [OO]DD |- e1' : [OO]aA    ]]$ & By above equality \\
      & $[[ TT2 |- ae2 <= [TT2]aA1 -| DD ]]$ & Premise \\
      & $[[ [OO] DD |- e2' : [OO]aA1   ]]$ & By i.h. \\
      & $\erase{[[e2']]} = \erase{[[ae2]]}$ & Above \\
      & $[[ TT1 |- [TT1] aA |> aA1 -> aA2 -| TT2]]$ & Premise \\
      & $[[ [OO]TT2 |- [OO]([TT1] aA) |> [OO]aA1 -> [OO]aA2     ]]$ & By \Cref{thm:match_soundness} \\
      & $[[ [OO]TT2  ]] =  [[ [OO]DD  ]]$ & By \Cref{lemma:confluence} \\
      & $[[ [OO]([TT1]aA)  ]] =  [[  [OO]aA  ]]$ & By \Cref{lemma:subst_ext_invar} \\
      & $[[ [OO]DD |- [OO]aA |> [OO]aA1 -> [OO]aA2     ]]$ & By above equalities \\
      & $[[ [OO]DD |- [OO]aA1 <~ [OO]aA1    ]]$ & By \cref{lemma:sub_refl} \\
      $\byhave$& $[[  [OO]DD |- e1' e2' : [OO]aA2   ]]$ & By \rref{app} \\
      $\byhave$& $\erase{e_1' ~ e_2'} = \erase{e_1'} ~ \erase{e_2'} = \erase{e_1} ~ \erase{e_2} = \erase{e_1 ~ e_2}$ & By def. of erasure
    \end{longtable}


  \item Case \[  \drule{inf-anno}  \]
    \begin{longtable}[l]{ll|l}
      & $[[  GG |- ae <= aA -| DD ]]$ & Premise \\
      $\byhave$ & $ [[  [OO]DD |- e' : [OO]aA  ]]  $ & By i.h., \\
      & $\erase{[[e]]} = \erase{[[e']]}$ & Above \\
      $\byhave$ & $\erase{e : A} = \erase{e} = \erase{e'}$ & By above equality and the def. of erasure
    \end{longtable}



  \item Case \[ \drule{chk-gen}  \]

    \begin{longtable}[l]{ll|l}
      & $[[ DD --> OO ]]$ & Given \\
      & $[[ DD , a --> OO , a]]$ & By def \\
      & $[[ GG , a --> DD , a, TT  ]]$ & By \cref{lemma:typing_extension} \\
      & $[[TT]]$ is soft & By \cref{lemma:extension_order} \\
      & $\underbrace{[[DD, a, TT]]}_{[[DD']]} [[-->]] \underbrace{[[OO, a, |TT|]]}_{[[OO']]}$ & By \cref{lemma:filling_completes} \\
      & $[[ GG, a |- ae <= aA -| DD , a , TT]]$  & Premise \\
      & $[[ [OO']DD' |- e' : [OO']aA  ]]$  & By i.h., \\
      $\byhave$& $\erase{[[ae]]} = \erase{[[e']]}$ & Above \\
      & $[[  [OO']aA  ]] = [[ [OO]aA ]]$ & By \cref{lemma:subst_stable} \\
      & $[[ [OO']DD'  ]]$ = $[[ [OO]DD, a  ]]$ & By \Cref{lemma:subst_go_away} and def. of context substitution \\
      & $[[ [OO]DD, a |- e' : [OO]aA  ]]$  & By above equalities \\
      & $[[  [OO]DD |- e' : \/a. [OO]aA  ]]$  & By \rref{gen} \\
      $\byhave$& $[[ [OO]DD |- e' : [OO] (\/a. aA)  ]]$  & By def. of substitution

    \end{longtable}


  \item Case \[ \drule{chk-lam}  \]

    \begin{longtable}[l]{ll|l}
      & $[[  DD --> OO  ]]$ & Given \\
      & $[[ DD, x : aA --> OO, x : [OO]aA   ]]$ & By def \\
      & $[[ GG , x : aA --> DD, x : aA, TT   ]]$ & By \cref{lemma:typing_extension} \\
      & $[[TT]]$ is soft & By \cref{lemma:extension_order} \\
      & $\underbrace{[[DD, x : aA, TT]]}_{[[DD']]} [[-->]] \underbrace{[[OO, x : [OO]aA, |TT| ]]}_{[[OO']]}$ & By \cref{lemma:filling_completes} \\
      & $[[ GG, x : aA |- ae <= aB -| DD, x : aA, TT]]$ & Premise \\
      & $[[ [OO']DD' |- e' : [OO']aB ]]$ & By i.h., \\
      & $\erase{[[ae]]} = \erase{[[e']]}$ & Above \\
      & $[[ [OO']aB  ]] = [[ [OO]aB  ]]$ & By \cref{lemma:subst_stable} \\
      & $[[ [OO']DD'  ]] = [[ [OO]DD, x : [OO]aA  ]] $ & By \Cref{lemma:subst_go_away} and def. of context substitution \\
      & $[[ [OO]DD, x : [OO]aA |- e' : [OO]aB ]]$ & By above equalities \\
      & $[[ [OO]DD |- \x : [OO]aA . e' : [OO]aA -> [OO]aB ]] $ & By \rref{lamann} \\
      $\byhave$ & $[[ [OO]DD |- \x : [OO]aA . e' : [OO](aA -> aB) ]]$ & By def. of substitution \\
      $\byhave$ & $\erase{\erlam x e} = \erlam x {\erase{e}} = \erlam x {\erase{e'}} = \erase{[[\x : [OO]aA . e']]}$ & By the def. of erasure
    \end{longtable}


  \item Case \[ \drule{chk-sub}  \]

    \begin{longtable}[l]{ll|l}
      & $[[ TT |- [TT]aA <~ [TT]aB -| DD]]$ & Premise \\
      & $[[TT --> DD]]$ & By \cref{lemma:sub_extension} \\
      & $[[DD -->OO ]]$ & Given \\
      & $[[TT --> OO]]$ & By \cref{lemma:transitivity} \\
      & $[[ GG |- ae => aA -| TT]]$ & Premise \\
      & $[[ [OO]TT |- e' : [OO]aA   ]]$ & By i.h., \\
      & $\erase{[[ae]]} = \erase{[[e']]}$ & Above \\
      & $[[ [OO]TT   ]] = [[ [OO]DD  ]]$ & By \cref{lemma:confluence} \\
      & $[[ [OO]DD |- e' : [OO]aA   ]] $ & By above equality \\
      & $[[ [OO]DD |- [OO]([TT]aA) <~ [OO]([TT]aB) ]] $ & By \Cref{thm:sub_soundness} \\
      & $[[ [OO]([TT]aA)  ]] = [[ [OO]aA ]]$ & By \cref{lemma:subst_ext_invar} \\
      & $[[ [OO]([TT]aB)  ]] = [[ [OO]aB ]]$ & By \cref{lemma:subst_ext_invar} \\
      & $[[ [OO]DD |- [OO]aA <~ [OO]aB ]]$ & By above equalities \\
      $\byhave$& $\ctxsubst{\Omega}{\Delta} \vdash (e' : [[ [OO]aB   ]]) : [[ [OO]aB   ]]$ & By def. annotation \\
      $\byhave$& $\erase{(e' : [[ [OO]aB  ]])} = \erase{e'} = \erase{e}$ & By def. erasure
    \end{longtable}



  \end{itemize}


\end{proof}

\newpage

\section{Completeness of Consistent Subtyping}

\instcomplete*
\begin{proof}
  By mutual induction on the given derivation.
  \begin{enumerate}
  \item We have $ [[  [OO]GG |- [OO]evar <~ [OO]aA  ]]   $. We case analyze the shape of $[[aA]]$.
    \begin{itemize}
    \item Case $[[aA]] = [[unknown]], [[evar = sa]]$:
      \begin{longtable}[l]{ll|l}
        & $[[  [OO]GG |- [OO]sa <~ [OO]unknown  ]]$ & Given \\
        & $[[ [OO]unknown  ]] = [[unknown]]$ \\
        & $[[  [OO]GG |- [OO]sa <~ unknown  ]]$ & By above equality \\
        & $[[sa]] \notin \textsc{unsolved}([[GG]])$ & Given \\
        & $[[ GG  ]] = [[ GL, sa, GR   ]]$ & Above \\
        & $[[ GL, sa, GR --> OO   ]]$ & Given \\
        & $[[ OO ]] = [[ OL, sa = atc, OR  ]]   $ & By \cref{lemma:extension_order} and $[[OO]]$ is complete and $[[ [OO]sa ]] \in [[agc]]$ \\
        & $[[ GL --> OL   ]]$ & Above \\
        & Let $[[DD]] = [[ GL, ga, sa = ga, GR ]]$ \\
        & and $[[OO']] = [[OL, ga = atc, sa = atc, OR]]$ \\
        $\byhave$& $[[ GG |- sa <~~ unknown -| DD    ]]$ & By \rref{instl-solveUS} \\
        $\byhave$& $\Delta \exto \Omega'$ & By \cref{lemma:paralell_admit,lemma:paralell_ext_solu} \\
        $\byhave$& $\Omega \exto \Omega'$ & By \cref{lemma:unsolved_ext,lemma:solution_ext}
      \end{longtable}
    \item Case $A = [[unknown]], [[evar = ga]]$:
      \begin{longtable}[l]{ll|l}
        & $[[  [OO]GG |- [OO]ga <~ [OO]unknown  ]]$ & Given \\
        & $[[ [OO]unknown  ]] = [[unknown]]$ \\
        & $[[  [OO]GG |- [OO]ga <~ unknown  ]]$ & By above equality \\
        & $[[ga]] \notin \textsc{unsolved}([[GG]])$ & Given \\
        & $[[ GG  ]] = [[ GG0[ga]   ]]$ & Above \\
        & Let $[[DD]] = [[ GG0[ga] ]]$ and $[[OO']] = [[OO]]$\\
        $\byhave$& $[[ GG |- ga <~~ unknown -| DD ]]$ & By \rref{instl-solveUG} \\
        $\byhave$& $[[ DD --> OO' ]]$ & Given \\
        $\byhave$& $[[ OO --> OO']]$ & By \cref{lemma:reflexivity}
      \end{longtable}

    \item Case $[[aA]] = [[gb]], [[evar]] = [[sa]]$:
      \begin{longtable}[l]{ll|l}
        & $[[ [OO]GG |- [OO]sa <~ [OO]gb  ]]$ & Given \\
        & $[[ [OO]GG |- t <~ tc   ]]$ & Let $[[ [OO]sa]] = [[t]]$ and $[[ [OO]gb  ]] = [[tc]]$ and $[[OO]]$ is predicative \\
        & $[[t]] = [[tc]]$ & By \cref{lemma:mono_equal} \\ \\
        & $[[ [GG]gb ]] = [[gb]]$ & Given \\
        & $[[gb]] \in \textsc{unsolved}{([[GG]])}$ & Above
      \end{longtable}
      Now consider whether $[[sa]]$ is declared to the left of $[[gb]]$.
      \begin{itemize}
      \item Case $[[GG]] = [[  GG0, sa, GG1, gb, GG2 ]]$
        \begin{longtable}[l]{ll|l}
          & Let $[[DD]] = [[ GG0, ga, sa = ga, GG1, gb = ga, GG2     ]]$ & \\
          $\byhave$& $[[  GG |- sa <~~ gb -| DD ]]$ & By \rref{instl-reachSG1} \\
          & $ [[GG --> OO]]   $ & Given \\
          & $[[OO]] = [[ OO0, sa = atc, OO1, gb = atc, OO2     ]]$ & By \cref{lemma:extension_order}   \\
          & Let $[[OO']] = [[ OO0, ga = atc, sa = atc, OO1, gb = atc, OO2     ]]$  \\
          $\byhave$& $[[OO --> OO']]$ & By \cref{lemma:unsolved_ext,lemma:solution_ext} \\
          $\byhave$& $[[DD --> OO' ]]$ & By \cref{lemma:paralell_admit,lemma:paralell_ext_solu}
        \end{longtable}
      \item Case $[[GG]] = [[  GG0, gb, GG1, sa, GG2 ]]$
        \begin{longtable}[l]{ll|l}
          & Let $[[DD]] = [[  GG0, gb, GG1, sa = gb, GG2  ]]$ & \\
          $\byhave$& $ [[GG |- sa <~~ gb -| DD]] $ & By \rref{instl-reachOther} \\
          $\byhave$& $[[DD --> OO]]$ & By \Cref{lemma:paralell_ext_solu} \\
          $\byhave$& $[[OO --> OO]]$ & By \Cref{lemma:reflexivity}
        \end{longtable}
      \end{itemize}
    \item Case $[[aA]] = [[sb]]$ is similar to the above case.
    \item Case $A = a$:
      \begin{longtable}[l]{ll|l}
        &$[[  [OO]GG |- [OO]evar <~ [OO]a  ]]$& Given \\
        &$[[  [OO]GG |- [OO]evar <~ a  ]]$ & From $[[  [OO]a ]] = [[a]]$ \\
        & $[[ [OO]evar ]] = [[a]]$ & By inversion of \rref{cs-tvar} \\
        & $[[a]]$ is declared to the left of $[[evar]]$ in $[[OO]]$ & $[[OO]]$ is well-formed \\
        & $[[GG --> OO]]$ & Given \\
        & $[[a]]$ is declared to the left of $[[evar]]$ in $[[GG]]$ & By \Cref{lemma:reverse_preserve} \\
        & Let $[[GG]] = [[ GG0[a][evar]   ]]$ \\
        & Let $[[DD]] = [[ GG0[a][evar = a] ]]$ \\
        $\byhave$& $[[ GG |- evar <~~ a -| DD ]]$ & By \rref{instl-solveS} or \rref{instl-solveG} \\
        $\byhave$& $\Delta \exto \Omega$ & By \Cref{lemma:paralell_ext_solu} \\
        $\byhave$& $\Omega \exto \Omega$ & By \Cref{lemma:reflexivity}
      \end{longtable}
    \item Case $[[aA]] = [[aA1 -> aA2]]$:
      \begin{longtable}[l]{ll|l}
        & $[[  [OO]GG |- [OO]evar <~ [OO]aA1 -> [OO]aA2  ]]$ & Given \\
        & $[[ [OO]evar  ]] = [[ t1 -> t2 ]]$ & $\Omega$ is predicative \\
        & $[[  [OO]GG |- [OO]aA1 <~ t1  ]]$ & By inversion of \rref{cs-arrow} \\
        & $[[  [OO]GG |- t2 <~ [OO]aA2  ]]$ & Above \\
        & $[[GG]] = [[ GG0[evar] ]]$ & From $[[evar]] \in \textsc{unsolved}{([[GG]])}$ \\
        & $[[GG0[evar] ]] [[-->]] \underbrace{[[  GG0[evar2, evar1, evar = evar1 -> evar2 ]   ]]}_{[[GG1]]}$ \\
        & $[[ GG --> OO ]]$ & Given \\
        & $[[OO]] = [[ OO0[evar = at0] ]]$ & From $\genA \in \textsc{unsolved}{(\Gamma)}$ \\
        & $[[ OO0[evar = at0] ]] [[-->]] \underbrace{[[ OO0[evar2 = at2, evar1 = at1, evar = evar1 -> evar2]  ]]}_{[[OO1]]}$ \\ \\
        & $[[ [OO]GG   ]] = [[ [OO1]GG1 ]]$ & By \Cref{lemma:finish_complete} \\
        & $[[ [OO]aA1  ]] = [[ [OO1]aA1  ]]$ & By \Cref{lemma:finish_types} \\
        & $[[t1]] = [[ [OO1]evar1 ]]$ & From def. of $[[OO1]]$ \\
        & $[[  [OO1]GG1 |- [OO1]aA1 <~ [OO1]evar1  ]]$ & By above equalities \\
        & $[[ GG1 |- aA1 <~~ evar1 -| DD2   ]]$ & By i.h. \\
        & $[[ DD2 --> OO2]]$ and $[[ OO1 --> OO2 ]]$ & Above \\ \\
        & $[[ [OO]GG   ]] = [[ [OO2]GG2 ]]$ & By \Cref{lemma:finish_complete} \\
        & $[[ [OO]aA2  ]] = [[ [OO2]aA2  ]] = [[ [OO2]([DD2]aA2)  ]]$ & By \Cref{lemma:finish_types} \\
        & $[[t2]] = [[ [OO2]evar2 ]]$ & By  $[[OO1 --> OO2]]$ \\
        & $[[ [OO2]DD2 |- [OO2]evar2 <~ [OO2]([DD2]aA2)   ]]$ & By above equalities \\
        & $[[  DD2 |-  evar2 <~~ [DD2]aA2 -| DD ]]$ & By i.h. \\
        & $[[ OO2 --> OO' ]]$ & Above \\
        $\byhave$& $[[ DD --> OO'   ]]$ & Above \\
        $\byhave$& $[[ GG0[evar] |- evar <~~ aA1 -> aA2 -| DD  ]]$ & By \rref{instl-arr} \\
        $\byhave$& $[[OO --> OO' ]]$ & By \Cref{lemma:transitivity}
      \end{longtable}
    \item Case $[[A]] = [[int]]$:
      \begin{longtable}[l]{ll|l}
        & $[[  [OO]GG |- [OO]evar <~ [OO]int   ]]$ & Given \\
        & $[[ [OO]int  ]] = [[int]]$ \\
        & $[[  [OO]GG |- [OO]evar <~ int   ]]$ & By above equality \\
        & $[[ [OO]evar ]] = [[int]]$ & $\Omega$ is predicative \\
        & $[[evar]] \in \textsc{unsolved}{(\Gamma)}$ & Given \\
        & $[[GG]] = [[GG0[evar] ]]$ & Above \\
        & Let $[[DD]] = [[ GG0[evar = int]  ]]$ and $[[OO']] = [[OO]]$\\
        & $[[ GG0[evar] |- evar <~~ int -| DD  ]]$ & By \rref{instl-solveS} or \rref{instl-solveG} \\
        & $[[GG --> OO]]$ & Given \\
        & $[[  GG0[evar = int] --> OO  ]]$ & By \Cref{lemma:paralell_ext_solu}
      \end{longtable}
    \item Case $[[aA]] = [[  \/b . aB  ]]$:
      \begin{longtable}[l]{ll|l}
        & $[[ [OO]GG |- [OO]evar <~ \/b. [OO]aB  ]]$ & Given \\
        & $[[ [OO]evar ]]$ cannot be a quantifier & $[[OO]]$ is predicative \\
        & $[[ [OO]GG, b |- [OO]evar <~ [OO]aB  ]] $ & By inversion of \rref{cs-forallR} \\ \\
        & $[[ [OO]GG, b   ]] = [[   [OO, b](GG, b)  ]]$ & By def. of context substitution \\
        & $[[ [OO]evar  ]] = [[  [OO,b]evar ]]$ & By def. of substitution \\
        & $[[ [OO]aB  ]] = [[  [OO, b]aB  ]]$ & By def. of substitution \\
        & $ [[ [OO, b](GG, b) |- [OO, b]evar <~ [OO, b]aB  ]]  $ & By above equalities \\
        & $ [[GG, b |- evar <~~ aB -| DD0 ]]  $ & By i.h. \\
        & $[[ DD0 --> OO' ]]$ & Above \\
        & $[[  OO, b --> OO'  ]]$ & Above \\
        $\byhave$& $[[  OO --> OO'  ]]$ & By \cref{lemma:drop_ext}\\  \\
        & $[[ GG, b --> DD0 ]]$ & By \Cref{lemma:inst_extension} \\
        & $[[DD0]] = [[DD, b, DD']]$ & By \Cref{lemma:extension_order} \\
        & $[[  GG --> DD ]]$ & Above \\
        $\byhave$& $[[  DD --> OO'  ]]$ \\
        $\byhave$& $ [[ GG |- evar <~~ \/b. aB -| DD  ]]  $ & By \rref{instl-forallR}
      \end{longtable}
    \end{itemize}
  \item Now we have $[[ [OO]GG |- [OO]aA <~ [OO]evar ]]$. These cases are mostly symmetric. The one exception is when $[[ aA  ]] = [[  \/b. aB ]]$.
    \begin{itemize}
    \item Case $[[aA]] = [[\/b . aB]]$:
      \begin{longtable}[l]{ll}
        & $[[ [OO]GG |- \/b . [OO]aB <~ [OO]evar   ]]$ \qquad Given \\
        & $[[ [OO]evar  ]]$ cannot be a quantifier \qquad $[[OO]]$ is predicative \\
        & $[[  [OO]GG |- t  ]]$ \qquad By inversion of \rref{cs-forallL} \\
        & $[[ [OO]GG |- ([OO]aB) [b ~> t] <~ [OO]evar    ]]$ \qquad Above \\ \\
        & $[[ [OO]GG ]] = [[  [OO, msb, sb = at] (GG, msb, sb)  ]]$ \qquad By def. of context application \\
        & $[[ ([OO]B) [b ~> t] ]] = [[  [OO, msb, sb = at] (aB [b ~> sb])  ]]$ \qquad by def. of substitution \\
        & $[[  [OO]evar  ]] = [[ [OO, msb, sb = at]evar  ]]$ \qquad By def. of substitution \\
        & $[[   [OO, msb, sb = at] (GG, msb, sb)   |-   [OO, msb, sb = at] (aB [b ~> sb])   <~ [OO, sb = at ]evar    ]]$ \qquad By above equalities \\
        & $[[ GG, msb, sb |- aB [b ~> sb] <~~ evar -| DD   ]]$ \qquad By i.h. \\
        &$[[GG, msb, sb --> DD]]$ \qquad By \cref{lemma:inst_extension} \\
        & $[[DD]] = [[DL, msb, DR]]$ \qquad By \cref{lemma:extension_order} \\
        & $[[ GG --> DL ]]$ \qquad Above \\
        & $[[ OO, msb, sb = at --> OO']]$ \qquad Above \\
        & $[[OO']] = [[OL, msb, OR]]$ \qquad By \cref{lemma:extension_order} \\
        $\byhave$& $[[ OO --> OL ]]$ \qquad Above \\
        $\byhave$& $[[ DL --> OL ]]$ \qquad \cref{lemma:transitivity} \\
        $\byhave$ & $[[ GG |- \/b . aB <~~ evar -| DL  ]]$ \qquad By \rref{instr-forallLL}
      \end{longtable}
    \end{itemize}
  \end{enumerate}
\end{proof}


\begin{landscape}
\begin{table}
  \centering
  \begin{footnotesize}
\begin{tabular}{cccccccccc} \toprule
&                 & \multicolumn{8}{c}{ $[[ [GG]aB  ]]$  }  \\
&                 & $\forall b. B'$  & $\int$      & $a$         & $\genB$     & $\unknown$          & $B_1 \to B_2$ & $[[static]]$ & $[[gradual]]$ \\ \cmidrule{3-10}
   \multirow{6}{*}{$\ctxsubst{\Gamma}{A}$} & $\forall a. A'$ & \hl{1 (B poly)}  & \hl{2.Poly} & \hl{2.Poly} & \hl{2.Poly} & \hl{1 (B unknown)}         &  \hl{2.Poly} & \hl{2.Poly}  & \hl{2.Poly}  \\ \cmidrule{3-10}
& $\int$          & \hl{1 (B poly)}  &  \hl{2.Ints}           &  \textit{Impossible} & \hl{2.BEx.Int}            & \hl{1 (B unknown)}  &  \textit{Impossible} &  \textit{Impossible} &  \textit{Impossible} \\ \cmidrule{3-10}
& $a$             &  \hl{1 (B poly)} & \textit{Impossible} &  \hl{2.UVars}           &  \hl{2.BEx.UVar}           &  \hl{1 (B unknown)} &  \textit{Impossible}  &  \textit{Impossible} &  \textit{Impossible} \\ \cmidrule{3-10}
& \multirow{2}{*}{$\genA$}         & \multirow{2}{*}{\hl{1 (B poly)}}   & \multirow{2}{*}{\hl{2.AEx.Int}} & \multirow{2}{*}{\hl{2.AEx.UVar}} & \hl{2.AEx.SameEx}  & \multirow{2}{*}{\hl{1 (B unknown)}}  & \multirow{2}{*}{\hl{2.AEx.Arrow}} & \multirow{2}{*}{\hl{2.AEx.S}} & \multirow{2}{*}{\hl{2.AEx.G}}  \\
& & & & & \hl{2.AEx.OtherEx} &   &   \\ \cmidrule{3-10}
& $\unknown$      & \hl{1 (B poly)}  & \hl{2.Unknown} &  \hl{2.Unknown}  &  \hl{2.Unknown} & \hl{1 (B unknown)}  & \hl{2.Unknown} & \textit{Impossible} & \hl{2.Unknown}  \\ \cmidrule{3-10}
  & $A_1 \to A_2$     & \hl{1 (B poly)}  & \textit{Impossible} &  \textit{Impossible}  & \hl{2.BEx.Arrow}  & \hl{1 (B unknown)}  & \hl{2.Arrows} &  \textit{Impossible} &  \textit{Impossible} \\ \cmidrule{3-10}
  & $[[static]]$     & \hl{1 (B poly)}  & \textit{Impossible} &  \textit{Impossible}  & \hl{2.BEx.S} & \textit{Impossible} & \textit{Impossible} & \hl{2.S} &  \textit{Impossible}  \\ \cmidrule{3-10}
  & $[[gradual]]$     & \hl{1 (B poly)}  & \textit{Impossible} &  \textit{Impossible}  & \hl{2.BEx.G} & \hl{1 (B unknown)} & \textit{Impossible} & \textit{Impossible} & \hl{2.G} \\ \bottomrule
\end{tabular}
  \end{footnotesize}
\caption{List of cases} \label{table:complete}
\end{table}
\end{landscape}

\subcomplete*
\begin{proof}

  By induction on the given declarative derivation. We list all the possible cases in \cref{table:complete}.

We first split on$\ctxsubst{\Gamma}{B}$.
  \begin{itemize}
  \item Case \hl{1 (B poly)}: $[[ [GG]aB  ]]$ is polymorphic: $[[  [GG]aB  ]] = [[ \/b. aB'   ]]$:
    \begin{longtable}[l]{ll|l}
      &$[[aB]] = [[ \/b . aB0]]   $& $[[GG]]$ is predicative \\
      & $[[aB']] = [[ [GG]aB0  ]]$ & $[[GG]]$ is predicative \\
      & $[[ [OO]aB   ]]  =  [[ \/b. [OO]aB0  ]]    $ & By def. of substitution \\
      & $[[  [OO]GG |- [OO]aA <~ [OO]aB   ]]$ & Premise \\
      & $[[  [OO]GG |- [OO]aA <~ \/b . [OO]aB0   ]]$ & By above equality \\
      & $[[  [OO]GG, b |- [OO]aA <~ [OO]aB0   ]]$ & By \Cref{lemma:forall_invert} \\
      & $[[ [OO]GG, b  ]] = [[ [OO, b] (GG, b)     ]]$ & By def. of substitution \\
      & $[[  [OO]aA  ]] = [[ [OO, b]aA  ]]$ & By def. of substitution \\
      & $[[  [OO]aB  ]] = [[ [OO, b]aB  ]]$ & By def. of substitution \\
      & $[[  [OO,b](GG, b) |- [OO,b]aA <~ [OO,b]aB0   ]]$ & By above equalities \\
      & $[[  GG, b |- [GG, b]aA <~ [GG, b]aB0 -| DD'  ]]$ & By i.h. \\
      & $[[  DD' --> OO0'  ]]$ & Above \\
      & $[[ OO, b --> OO0'   ]]$ & Above \\
      & $[[  GG, b |- [GG]aA <~ [GG]aB0 -| DD'  ]]$ & By def. of substitution \\ \\
      & $[[ GG, b --> DD'   ]]$ & By \Cref{lemma:inst_extension} \\
      & $[[ DD'  ]] = [[ DD, b, TT  ]]$ & By \Cref{lemma:extension_order} \\
      & $[[ GG --> DD   ]]$ & Above \\
      & $[[ DD, b, TT --> OO0'    ]]$ & By $[[DD' --> OO0']]$ and above equality \\
      & $[[ OO0'  ]] = [[ OO', b, OR  ]]$ & By \Cref{lemma:extension_order} \\
      $\byhave$& $[[ DD --> OO'  ]]$ & Above \\
      & $[[ OO, b --> OO', b , OR  ]]$ & By above equality \\
      $\byhave$& $[[ OO --> OO'  ]]$ & By \Cref{lemma:extension_order} \\ \\
      & $[[  GG, b |- [GG]aA <~ [GG]aB0 -| DD, b, TT  ]]$ & By above equality \\
      & $[[  GG |- [GG]aA <~ \/b. [GG]aB0 -| DD  ]]$ & By \rref{as-forallR} \\
      $\byhave$& $[[  GG |- [GG]aA <~ \/b. aB' -| DD  ]]$ & By above equality
    \end{longtable}

  \item Case \hl{1 (B unknown)}: $\ctxsubst{\Gamma}{B} = \unknown$:
        \begin{longtable}[l]{ll|l}
          & $[[ [OO]aB]] = [[unknown]]$ \\
          & $[[ [OO]GG |- [OO]aA <~ unknown  ]]$ & Given \\
          & $[[ [OO]aA ]] \in [[gc]] $ & Above \\
          &$[[ GG --> OO  ]]$& Given \\
          &$[[ [GG]aA  ]] \in [[agc]]  $ & Above \\
          & $[[  GG |- [GG]aA <~ unknown -| [ [GG]aA ] GG ]]$ & By \rref{as-unknownRR} \\
          & There exists $[[OO']]$ such that $[[ [ [GG]aA ] GG --> OO'    ]]$  and $[[  OO --> OO'  ]]$
        \end{longtable}

 \item Case 2.*: $[[ [GG]aB   ]]$ is not polymorphic. We split on the form of $[[  [OO]aA  ]]$.
    \begin{itemize}
    \item Case \hl{2.Poly}: $[[ [OO]aA  ]]$ is polymorphic: $[[ [GG]aA  ]] = [[ \/a . aA'  ]]$:
      \begin{longtable}[l]{ll|l}
        & $[[aA]] = [[\/a. aA0]]$& $[[GG]]$ is predicative \\
        & $[[aA']] = [[ [GG]aA0  ]]$ & $[[GG]]$ is predicative \\
        & $[[ [OO]aA  ]] = [[ \/a . [OO]aA0 ]]$ & By def. of substitution \\
        & $[[ [OO]GG |- [OO]aA <~ [OO]aB   ]]$ & Premise \\
        & $[[ [OO]GG |- \/a. [OO]aA0 <~ [OO]aB   ]]$ & By above equality \\
        & $[[ [OO]GG |- ([OO]aA0) [a ~> t] <~ [OO]aB   ]]$ & By inversion on \rref{cs-forallL} \\
        & $[[ [OO]GG |- t  ]]$ & Above \\ \\
        & $[[ [OO]GG  ]] = [[  [OO, evar = at](GG, evar)   ]]$ & By def. of substitution \\
        & $[[ ([OO]aA0) [a ~> t]  ]] = [[  [OO, evar = at](aA0 [ a ~> evar ])  ]]$ & By def. of substitution \\
        & $[[ [OO]aB ]] = [[ [OO, evar = at]aB   ]]$ & By def. of substitution \\
        & $[[ [OO, evar = at](GG, evar) |- [OO, evar = at](aA0 [ a ~> evar ]) <~ [OO, evar = at]aB   ]] $ & By above equalities \\
        & $[[  GG, evar |- [GG, evar] (aA0 [ a ~> evar ]) <~ [GG, evar]aB -| DD   ]]$ & By i.h. \\
        $\byhave$& $[[ DD --> OO'   ]]$ & Above \\
        & $[[  OO, evar = at --> OO'   ]]$ & Above \\
        $\byhave$& $[[ OO --> OO'  ]]$ & By \cref{lemma:drop_ext} \\
        & $[[ [GG, evar] (aA0 [ a ~> evar ])  ]] = [[  ([GG]aA0) [a ~> evar]   ]]$ & By def. of substitution \\
        & $[[ [GG, evar]aB   ]] = [[ [GG]aB   ]]$ & By def. of substitution \\
        & $[[  GG, evar |- ([GG]aA0) [a ~> evar] <~ [GG]aB -| DD   ]]$ & By above equality \\
        & $[[  GG |- \/a. ([GG]aA0) <~ [GG]aB -| DD   ]]$ & By \rref{as-forallLL} \\
        $\byhave$& $[[  GG |- \/a. aA' <~ [GG]aB -| DD   ]]$ & By above equality \\
      \end{longtable}

    \item Case \hl{2.Unknown}: $[[ [GG]aA  ]] = [[unknown]]$:
        \begin{longtable}[l]{ll|l}
          & $[[ [OO] aA]] = [[unknown]]$ & Obviously, what else? \\
          & $[[ [OO]GG |- unknown <~ [OO]aB  ]]$ & Given \\
          & $[[ [OO]aB ]] \in [[gc]] $ & Above \\
          &$[[ GG --> OO  ]]$& Given \\
          &$[[ [GG]aB  ]] \in [[agc]]  $ & Above \\
          & $[[  GG |- unknown <~ [GG]aB -| [ [GG]aB ] GG ]]$ & By \rref{as-unknownLL} \\
          & There exists $[[OO']]$ such that $[[ [ [GG]aB ] GG --> OO'    ]]$ and $[[  OO --> OO'  ]]$
        \end{longtable}

    \item Case \hl{2.AEx.*}: $[[ [GG]aA ]]$ is an existential variable: $[[ [GG]aA  ]] = [[evar]]$. We split on the form of $[[ [GG]aB  ]]$.
      \begin{itemize}
      \item Case \hl{2.AEx.SameEx}. $[[ [GG]aB ]]$ is the same existential variable $[[ [GG]aB  ]] = [[evar]]$:
        \begin{longtable}[l]{ll|l}
          &$[[  GG |- evar <~ evar -| GG   ]]$& By \rref{as-evar} \\
          $\byhave$& $[[  GG |- [GG]aA <~ [GG]aB -| GG   ]]$ & By above equality \\
          $\byhave$& $[[ DD --> OO   ]]$ & $[[DD]] = [[GG]]$ \\
          $\byhave$& $[[ OO --> OO'  ]]$ & By \Cref{lemma:reflexivity} and $[[OO']] = [[OO]]$
        \end{longtable}
      \item Case \hl{2.AEx.OtherEx}. $[[ [GG]aB  ]]$ is a different existential variable $[[ [GG]aB ]] = [[evarb]]$ where $[[evarb]] \neq [[evar]]$:
        \begin{longtable}[l]{ll|l}
          &$[[ [OO]aA  ]] = [[ [OO]([GG]aA)   ]] = [[ [OO]evar  ]]$& By \Cref{lemma:subst_ext_invar} \\
          &$[[ [OO]aB  ]] = [[ [OO]([GG]aB)   ]] = [[ [OO]evarb  ]]$& By \Cref{lemma:subst_ext_invar} \\
          & $ [[ [OO]GG |- [OO]aA <~ [OO]aB  ]]   $ & Given \\
          & $[[ [OO]GG |- [OO]evar <~ [OO]evarb  ]] $ & By above equalities \\
          & $[[ GG |- evar <~~ evarb -| DD   ]]$ & By \Cref{thm:inst_complete} \\
          $\byhave$& $[[ DD --> OO'  ]]$ & Above \\
          $\byhave$& $[[ OO --> OO'   ]]$ & Above \\
          & $[[ GG |- evar <~ evarb -| DD     ]]$ & By \rref{as-instL} \\
          $\byhave$& $[[ GG |- [GG]aA <~ [GG]aB -| DD     ]]$ & By above equalities
        \end{longtable}
      \item Case \hl{2.AEx.Int}. We have $[[ [GG]aB  ]] = [[int]]$:
        \begin{longtable}[l]{ll|l}
          &$[[ GG --> OO  ]]$& Given \\
          & $ [[ [OO] aB ]] = [[int]] = [[ [OO]int  ]]$ & By def. of substitution \\
          & $[[ [OO]aA  ]] = [[ [OO]([GG]aA)   ]] = [[ [OO]evar  ]]$ & By \Cref{lemma:subst_ext_invar} \\
          & $  [[ [OO]GG |- [OO]aA <~ [OO]aB    ]]  $ & Given \\
          & $[[  [OO]GG |- [OO]evar <~ [OO]int    ]]$ & By above equalities \\
          & $[[  GG |- evar <~~ int -| DD    ]]$ & By \Cref{thm:inst_complete} \\
          $\byhave$& $[[ DD --> OO'  ]]$ & Above \\
          $\byhave$& $[[ OO --> OO'  ]]$ & Above \\
          & $[[  GG |- evar <~ int -| DD    ]]$ & By \rref{as-instL} \\
          $\byhave$& $[[ GG |- [GG]aA <~ [GG]aB -| DD     ]]$ & By above equalities
        \end{longtable}
      \item Case \hl{2.AEx.UVar}. We have $[[ [GG]aB  ]] = [[b]]$. Similar to Case \hl{2.AEx.Int}.
      \item Case \hl{2.AEx.Arrow}. $[[ [GG]aB  ]] = [[ aB1 -> aB2   ]]$. We prove
        $[[evar]] \notin \textsc{fv}([[ [GG]aB  ]])$. Suppose for a
        contradiction, that $[[evar]] \in \textsc{fv}([[ [GG]aB ]])$, then
        $[[evar]]$ must be a subterm of $[[ [GG]aB  ]]$, so is
        $[[  [OO]evar  ]]$ a subterm of
        $[[ [OO]([GG]aB)     ]]$. The latter is equal to
        $[[ [OO]aB   ]]$, so $[[  [OO]evar  ]]$ is a subterm of
        $[[ [OO]aB   ]]$. Since $[[ [GG]aB  ]] = [[ aB1 -> aB2   ]]$, then
        $[[ [OO]aB   ]]$ must have the form $[[ aB1' -> aB2'  ]]$. Therefore
        $[[  [OO]evar   ]]$ must occur in either $[[aB1']]$ or $[[aB2']]$. But we
        have $[[ [OO]GG |- [OO]evar <~ [OO]aB       ]]$. That is , $[[  [OO]evar   ]]$
        cannot be a subterm of $[[ [OO]aB   ]]$. This is a contradiction.
        \begin{longtable}[l]{ll|l}
          &$[[evar]] \notin \textsc{fv}([[ [GG]aB  ]])$ & Proved above \\
          & $[[  GG --> OO  ]]$ & Given \\
          & $[[ [OO]aB   ]] = [[ [OO]([GG]aB)   ]]$ & By \Cref{lemma:subst_ext_invar} \\
          & $[[  [OO]GG |- [OO]evar <~ [OO]aB          ]]$ & Given \\
          & $[[  [OO]GG |- [OO]evar <~ [OO]([GG]aB)         ]]$ & By above equality \\
          & $[[ GG |- evar <~~ [GG]aB -| DD  ]]$ & By \Cref{thm:inst_complete} \\
          $\byhave$& $[[  DD --> OO' ]]$ & Above \\
          $\byhave$& $[[ OO --> OO' ]]$ & Above \\
          &$[[  GG |- evar <~ [GG]aB -| DD  ]]$ & By \rref{cs-instL} \\
          $\byhave$& $[[ GG |- [GG]aA <~ [GG]aB -| DD   ]]$ & By above equalities
        \end{longtable}
      \item Case \hl{2.AEx.S} and \hl{2.AEx.S}. Similar to Case \hl{2.AEx.Int}.
      \end{itemize}

    \item Case \hl{2.BEx.*}. $[[ [GG]aA  ]]$ is not polymorphic and
      $[[ [GG]aB   ]]$ is an existential variable: $[[ [GG]aB  ]] = [[evarb]]$. We split on the form of $[[ [GG]aA  ]]$.
      \begin{itemize}
      \item Case \hl{2.BEx.Int}. Similar to Case \hl{2.AEx.Unit}.
      \item Case \hl{2.BEx.UVar}. Similar to Case \hl{2.AEx.UVar}.
      \item Case \hl{2.BEx.Arrow}. Similar to Case \hl{2.AEx.Arrow}.
      \item Case \hl{2.BEx.S}. Similar to Case \hl{2.AEx.S}.
      \item Case \hl{2.BEx.G}. Similar to Case \hl{2.AEx.G}.
      \end{itemize}
      We use the second part of \Cref{thm:inst_complete} and apply \rref{as-instR}.

    \item Case \hl{2.Ints}. $[[ [GG]aA  ]] = [[ [GG]aB  ]] = [[int]] $:
      \begin{longtable}[l]{ll|l}
        $\byhave$&$[[  GG |- int <~ int -| GG ]]$& By \rref{as-int} \\
                 & $[[ GG --> OO   ]]$ & Given \\
        $\byhave$& $[[ DD --> OO'   ]]$ & $[[DD]] = [[GG]]$ \\
        $\byhave$& $[[ OO --> OO'  ]]$ & By \Cref{lemma:reflexivity} and $[[OO']] = [[OO]]$
      \end{longtable}

    \item Case \hl{2.UVars}. $[[ [GG]aA   ]] = [[ [GG]aB  ]] = [[a]]$:
      \begin{longtable}[l]{ll|l}
      $\byhave$&$[[ GG |-  a <~ a -| GG ]]$& By \rref{as-tvar} \\
                 & $[[ GG --> OO   ]]$ & Given \\
        $\byhave$& $[[ DD --> OO'   ]]$ & $[[DD]] = [[GG]]$ \\
        $\byhave$& $[[ OO --> OO'  ]]$ & By \Cref{lemma:reflexivity} and $[[OO']] = [[OO]]$
      \end{longtable}

    \item Case \hl{2.Arrows}. Let $[[ [GG]aA  ]] = [[ aA1 -> aA2  ]]$ and $[[ [GG]aB  ]] = [[ aB1 -> aB2   ]]$:
      \begin{longtable}[l]{ll|l}
        & $[[  GG --> OO  ]]$ & Given \\
        &$[[ [OO]aA  ]] = [[ [OO]([GG]aA)   ]] = [[ [OO]aA1 -> [OO]aA2  ]]$ & By \Cref{lemma:subst_ext_invar} \\
        &$[[ [OO]aB ]] = [[ [OO]([GG]aB)   ]] = [[ [OO]aB1 -> [OO]aB2  ]]$ & By \Cref{lemma:subst_ext_invar} \\
        & $[[ [OO]GG |- [OO]aA <~ [OO]aB   ]]$ & Given \\
        & $[[ [OO]GG |- [OO]aB1 <~ [OO]aA1   ]]$ & Premise \\
        & $[[ GG |- [GG]aB1 <~ [GG]aA1 -| TT  ]]$ & By i.h. \\
        & $[[ TT --> OO0  ]]$ & Above \\
        & $[[ OO --> OO0  ]]$ & Above \\
        & $[[ GG --> OO0  ]]$ & By \Cref{lemma:transitivity} \\ \\
        & $[[ [OO]GG   ]] = [[  [OO0]TT  ]]$ & By \Cref{lemma:confluence} \\
        & $[[ [OO]aA2   ]] = [[ [OO0]([GG]aA2)   ]]$ & By \Cref{lemma:subst_ext_invar} \\
        & $[[ [OO]aB2   ]] = [[ [OO0]([GG]aB2)   ]]$ & By \Cref{lemma:subst_ext_invar} \\
        & $[[ [OO]GG |- [OO]aA2 <~ [OO]aB2    ]]$ & Premise \\
        & $[[ [OO0]TT |- [OO0]([GG]aA2) <~ [OO0]([GG]aB2)    ]] $ & By above equalities \\
        & $[[ TT |- [TT]([GG]aA2) <~ [TT]([GG]aB2) -| DD   ]]$ & By i.h. \\
        $\byhave$& $[[ DD --> OO'   ]]$ & Above \\
        & $[[ OO0 --> OO'    ]]$ & Above \\ \\
        $\byhave$& $[[  GG |- [GG](aA1 -> aA2) <~ [GG](aB1 -> aB2) -| DD    ]]$ & By \rref{as-arrow} \\
        $\byhave$& $[[ OO --> OO'  ]]$ & By \Cref{lemma:transitivity}
      \end{longtable}

    \item Case \hl{2.S}: $[[ [GG]aA  ]] = [[ [GG]aB ]] = [[static]]$.
      \begin{longtable}[l]{ll|l}
        $\byhave$&$[[  GG |- static <~ static -| GG ]]$& By \rref{as-spar} \\
                 & $[[ GG --> OO   ]]$ & Given \\
        $\byhave$& $[[ DD --> OO'   ]]$ & $[[DD]] = [[GG]]$ \\
        $\byhave$& $[[ OO --> OO'  ]]$ & By \Cref{lemma:reflexivity} and $[[OO']] = [[OO]]$
      \end{longtable}
    \item Case \hl{2.G}. Similar to Case \hl{2.S}.

    \end{itemize}
  \end{itemize}
\end{proof}


\newpage

\section{Completeness of Typing}


\matchcomplete*
\begin{proof}
 By induction on the given derivation. We split on $[[ [GG]aA ]]   $.

 \begin{itemize}
 \item $[[  [GG]aA  ]] = [[ \/a. aA'   ]]$:
    \begin{longtable}[l]{ll|l}
      &$[[aA]] = [[ \/a . aA0]]   $& $[[GG]]$ is predicative \\
      & $[[aA']] = [[ [GG]aA0  ]]$ & $[[GG]]$ is predicative \\
      & $[[ [OO]aA   ]]  =  [[ \/a. [OO]aA0  ]]    $ & By def. of substitution \\
      & $[[  [OO]GG |- [OO]aA |> A1 -> A2  ]]$ & Given \\
      & $[[  [OO]GG |- \/a. [OO]aA0 |> A1 -> A2  ]]$ & By above equality \\
      & $[[  [OO]GG |- ([OO]aA0) [a ~> t]  |> A1 -> A2  ]]$ & By inversion \\
      & $[[  [OO]GG |- t   ]]$ & Above \\
      & $[[ GG --> OO   ]]$ & Given \\
      & $[[ GG, sa --> OO, sa   ]]$ & By def. of context extension \\ \\

      & $[[ [OO]GG  ]] = [[ [OO, sa = at](GG, sa)    ]]$ & By def. of context application \\
      & $[[ ([OO]aA0) [a ~> t]   ]] =  [[  [OO, sa = at]( aA0 [a ~> sa] )   ]]     $ & By def. of substitution \\
      & $[[  [OO, sa = at](GG, sa) |- [OO, sa = at]( aA0 [a ~> sa] )  |> A1 -> A2  ]]$ & By above equalities \\
      & $[[  GG, sa |- [GG, sa](aA0 [a ~> sa]) |> aA1' -> aA2' -| DD   ]]$ & By i.h. \\
      $\byhave$& $[[ DD --> OO'   ]]$ and $[[  OO, sa = at --> OO'    ]]$ & Above \\
      $\byhave$ & $[[A1]] = [[ [OO']aA1'  ]]$ and $[[A2 ]] = [[  [OO']aA2'  ]]$ & Above \\
      & $[[   [GG, sa](aA0 [a ~> sa])   ]] =  [[  ([GG]aA0) [a ~> sa]   ]]   $ & By def. of substitution \\
      & $[[   GG, sa |-   ([GG]aA0) [a ~> sa]   |> aA1' -> aA2' -| DD   ]]$ & By above equality \\
      & $[[ GG |- \/a . [GG]aA0 |> aA1' -> aA2' -| DD   ]]$ & By \rref{am-forallL} \\
      & $[[ [GG]aA ]] = [[ \/a. aA'  ]] = [[ \/a. [GG]aA0   ]]    $ & By above equalities \\
      $\byhave$ & $[[ GG |- [GG]aA |> aA1' -> aA2' -| DD  ]]$ & Above
    \end{longtable}



  \item   $[[  [GG]aA  ]] = [[ aA1' -> aA2'   ]]$:

    \begin{longtable}[l]{ll|l}
      & $[[ [OO]aA  ]] = [[ [OO]([GG]aA)  ]] = [[  [OO]aA1' -> [OO]aA2'  ]]$ & By \Cref{lemma:subst_ext_invar}   \\
      & $[[ [OO]GG |-  [OO]aA1' -> [OO]aA2' |> A1 -> A2  ]]$ & Given \\
      & Let $[[ DD  ]] = [[GG]]$ and $[[ OO' ]] = [[OO]]$ \\
      $\byhave$& $[[ [OO]aA1'   ]] = [[A1]]$ and $[[ [OO]aA2' ]] = [[A2]]$ \\
      $\byhave$ & $[[  GG |- aA1' -> aA2' |> aA1' -> aA2' -| GG  ]]$ & By \rref{am-arr} \\
      $\byhave$ & $[[  DD --> OO' ]]$ & Given $[[  GG --> OO  ]]$ \\
      $\byhave$ & $[[  OO --> OO' ]]$ & By \cref{lemma:reflexivity}

    \end{longtable}

  \item   $[[  [GG]aA  ]] = [[ unknown   ]]$:

    \begin{longtable}[l]{ll|l}
      & $[[ [OO]aA  ]] = [[ [OO]([GG]aA)  ]] = [[  unknown  ]]$ & By \Cref{lemma:subst_ext_invar}   \\
      & $[[ [OO]GG |-  unknown |> A1 -> A2  ]]$ & Given \\
      & Let $[[ DD  ]] = [[GG]]$ and $[[ OO' ]] = [[OO]]$ \\
      $\byhave$& $[[ A1   ]] = [[unknown]]$ and $[[ A2 ]] = [[unknown]]$ \\
      $\byhave$ & $[[  GG |- unknown |> unknown -> unknown -| GG  ]]$ & By \rref{am-unknown} \\
      $\byhave$ & $[[  DD --> OO' ]]$ & Given $[[  GG --> OO  ]]$ \\
      $\byhave$ & $[[  OO --> OO' ]]$ & By \cref{lemma:reflexivity}

    \end{longtable}

  \item   $[[  [GG]aA  ]] = [[ evar  ]]$:

    \begin{longtable}[l]{ll|l}
      & $[[  GG  ]] = [[ GG0[evar] ]]$ & Since $[[evar]] \in \textsc{unsolved}([[GG]]) $ \\
      & $[[ [OO]aA  ]] = [[ [OO]([GG]aA)  ]] = [[  [OO]evar  ]]$ & By \Cref{lemma:subst_ext_invar}   \\
      & $[[ [OO]GG |- [OO]evar |> A1 -> A2  ]]$ & Given \\
      & $[[ [OO]evar ]] = [[ t1 -> t2  ]]$ and $[[ A1 ]] = [[t1]]$ and $[[ A2 ]] = [[t2]]$ & $[[OO]]$ is predicate \\
      & $[[ OO ]] = [[  OO0[ evar = at' ]  ]]$ and $[[  [OO]at'   ]] = [[  t1 -> t2 ]]$ & Above \\
      & Let $[[DD]] = [[ GG0[evar1, evar2, evar = evar1 -> evar2]  ]]$ \\
      & Let $[[ OO'  ]] = [[ OO0[evar1 = at1, evar2 = at2, evar = evar1 -> evar2] ]] $ \\
      $\byhave$& $[[ DD --> OO'  ]]$ & By \Cref{lemma:paralell_admit} twice \\
      $\byhave$& $[[ OO --> OO'    ]]$ & By \Cref{lemma:paralell_ext_solu} and \Cref{lemma:paralell_admit} \\
      $\byhave$& $[[ GG0[evar] |- evar |> evar1 -> evar2 -| DD  ]]$ & By \rref{am-var} \\
      $\byhave$& $[[A1]] = [[t1]] = [[ [OO']evar1  ]]$ and $[[A2]] = [[t2]] = [[ [OO']evar2  ]]$ & Above
    \end{longtable}

 \end{itemize}


\end{proof}

\typingcomplete*
\begin{proof}
  By induction on the given derivation.
  \begin{itemize}
  \item Case \[     \ottaltinferrule{var}{}{ [[  (x : A) in [OO]GG ]] }{ [[ [OO]GG |- x : A]]  }  \]
    \begin{longtable}[l]{ll|l}
      & $ [[   (x : A) in [OO]GG  ]]  $  & Premise \\
      & $[[ GG --> OO  ]]$ & Given \\
      & $  [[ (x : aA') in GG  ]]  $ where $[[ [OO]aA'  ]] = [[  [OO]aA ]]$ & From def. of context application \\
      & Let $[[DD]] = [[GG]]$ and $[[OO']] = [[OO]]$. \\
      $\byhave$& $[[ GG --> OO ]]$ & Given \\
      $\byhave$& $[[ OO --> OO ]]$ & By \Cref{lemma:reflexivity} \\
      $\byhave$& $[[  GG |- x => aA' -| GG ]]$ & By \rref{inf-var} \\
      $\byhave$& $[[ [OO]aA' ]] = [[ [OO]aA ]] = [[A]]$ & A is well-formed in $[[  [OO]GG  ]]$ \\
      $\byhave$& $\erase{x} = \erase{x}$ & By def. of erasure
    \end{longtable}

  \item Case \[     \ottaltinferrule{int}{}{ }{ [[ [OO]GG |- n : int ]] }  \]

    \begin{longtable}[l]{ll|l}
      &Let $[[aA']] = [[int]]$ and $[[DD]] = [[GG]]$ and $[[OO']] = [[OO]]$. & \\
      $\byhave$& $[[ GG --> OO ]]$ & Given \\
      $\byhave$& $[[ OO --> OO ]]$ & By \Cref{lemma:reflexivity} \\
      $\byhave$& $[[ GG |- n => int -| GG ]]$ & By \rref{inf-int} \\
      $\byhave$& $[[ [OO]int ]] = [[int]]$ \\
      $\byhave$& $\erase{n} = \erase{n}$ & By def. of erasure
    \end{longtable}


  \item Case \[     \ottaltinferrule{lamann}{}{ [[ [OO]GG, x : A |- e : B  ]] }{ [[  [OO]GG |- \x : A . e : A -> B  ]] }  \]

    \begin{longtable}[l]{ll|l}
      & Let $[[OO0]] = [[ OO, x : aA]]$. \\
      & $[[ [OO0](GG, x : aA) ]] = [[ [OO]GG, x : aA  ]]$ & From def. of context application \\
      & $[[ [OO0](GG, x : aA) |- e : B   ]]$ & By above equality and premise \\
      & $[[ GG, x : aA |- ae' => aB0 -| DD0   ]]$ & By i.h. \\
      & $[[ DD0 --> OO' ]]$ & Above \\
      & $[[ OO0 --> OO' ]]$ & Above \\
      & $[[B]] = [[ [OO']aB0 ]]$ & Above \\
      & $\erase{e} = \erase{e'}$ & Above \\
      & $[[ GG, x : aA --> DD0  ]]$ & By \cref{lemma:typing_extension} \\
      & $[[ DD0 ]] = [[  DD1, x : aA' , DD2 ]]$ & By \cref{lemma:extension_order} \\
      & $[[ [DD1]aA  ]] = [[  [DD1] aA'  ]]$ & Above \\
      & $[[ GG --> DD1  ]] $ & Above \\
      & $[[ aA  ]] = [[  [DD1] aA'  ]]$ & $[[aA]]$ has no evar \\
      & $[[ GG, x : aA |- ae' => aB0 -| DD1, x : aA, DD2 ]]$ & By above equalities \\
      & $[[ GG, x : aA |- ae' <= aB0 -| DD1, x : aA, DD2 ]]$ & By \rref{chk-sub} \\ \\
      & $[[ DD1, x : aA', DD2 --> OO' ]]$ & By above equalities \\
      & $[[ OO' ]] = [[ OO1, x : aA'' , OO2 ]]$ & By \cref{lemma:extension_order} \\
      & $[[ [OO1]aA'  ]] = [[  [OO1] aA''  ]]$ & Above \\
      $\byhave$ & $[[ DD1 --> OO1  ]] $ & Above \\
      & $[[  OO, x : aA --> OO1, x : aA'', OO2 ]]$ & By above equalities \\
      $\byhave$& $[[  OO --> OO1  ]]$ & By \cref{lemma:extension_order} \\ \\

      & $[[  GG |- \x . ae' <= aA -> aB0 -| DD1  ]]$ & \rref{chk-lam} \\
      $\byhave$& $[[  GG |- (\x . ae') : aA -> aB0 => aA -> aB0 -| DD1  ]]$ & \rref{inf-anno} \\
      $\byhave$& $[[  [OO1](aA -> aB0)  ]] = [[ A -> [OO']aB0  ]] = [[ A -> B ]]$ & From above equality \\
      $\byhave$& $\erase{[[ \x : A . e  ]]} = \erlam{x}{\erase{e}} = \erlam{x}{\erase{e'}} = \erase{[[(\x . ae') : aA -> aB0]]}$ & By def. of erasure
    \end{longtable}

  \item Case \[     \ottaltinferrule{app}{}{ [[ [OO]GG |- e1 : A ]] \\ [[ [OO]GG |- A |> A1 -> A2]] \\ [[  [OO]GG |- e2 : A3  ]] \\ [[  [OO]GG |- A3 <~ A1  ]]  }{ [[  [OO]GG |- e1 e2 : A2  ]]  }  \]

    \begin{longtable}[l]{ll|l}
      & $[[  [OO]GG |- e1 : A  ]]$ & Premise \\
      & $[[ GG --> OO  ]]$ & Given \\
      & $[[ GG |- ae1' => aA' -| TT1  ]]$ & By i.h. \\
      & $[[TT1 --> OO0' ]]$ & Above \\
      & $[[ OO --> OO0'  ]]$ & Above \\
      & $[[A]] = [[ [OO0']aA'  ]]$ & Above \\
      & $\erase{[[e1]]} = \erase{[[ae1']]}$ & Above \\ \\

      & $[[ [OO]GG |- A |> A1 -> A2  ]]$ & Premise \\
      & $[[  [OO]GG  ]] = [[  [OO]OO  ]]$ & By \Cref{lemma:stable_complete_ctxt} \\
      & $ = [[ [OO0']OO0' ]]$ & By \Cref{lemma:finish_complete} \\
      & $ = [[ [OO0']GG   ]]$ & By \Cref{lemma:stable_complete_ctxt} \\
      & $ = [[ [OO0']TT1 ]]$ & By \Cref{lemma:confluence} \\
      & $[[ [OO0']TT1 |- [OO0']aA' |> A1 -> A2    ]]$ & By above equalities \\
      & $[[ TT1 |- [TT1]aA' |> aA1' -> aA2' -| TT2    ]]$ & By \Cref{thm:match_complete} \\
      & $[[ TT2 --> OO' ]]$ & Above \\
      & $[[ OO0' --> OO' ]]$ & Above \\
      & $[[A1]] = [[ [OO']aA1'  ]]$ & Above \\
      & $[[A2]] = [[ [OO']aA2'  ]]$ & Above \\ \\

      & $[[  [OO]GG |- e2 : A3   ]]$ & Premise \\
      & $ [[ [OO]GG  ]]   = [[  [OO]OO  ]]$ & By \Cref{lemma:stable_complete_ctxt} \\
      & $ = [[ [OO']OO'  ]]$ & By \Cref{lemma:finish_complete} \\
      & $ = [[ [OO']GG   ]]$ & By \Cref{lemma:stable_complete_ctxt} \\
      & $ = [[ [OO']TT2   ]]$ & By \Cref{lemma:confluence} \\
      & $[[ [OO']TT2 |- e2 : A3   ]]$ & By above equality \\
      & $[[  TT2 |- ae2' => aA3' -| TT3   ]]$ & By i.h. \\
      & $[[  TT3 --> OO1'   ]]$ & Above \\
      & $[[  OO' --> OO1'  ]]$ & Above \\
      & $[[A3]] = [[  [OO1']aA3'  ]]$ & Above \\
      & $\erase{[[e2]]} = \erase{[[ae2']]}$ & Above \\ \\

      & $[[ [OO]GG |- A3 <~ A1   ]]$ & Premise \\
      & $ [[ [OO]GG  ]]   = [[  [OO]OO  ]]$ & By \Cref{lemma:stable_complete_ctxt} \\
      & $ = [[ [OO1']OO1'  ]]$ & By \Cref{lemma:finish_complete} \\
      & $ = [[ [OO1']GG   ]]$ & By \Cref{lemma:stable_complete_ctxt} \\
      & $ = [[ [OO1']TT3   ]]$ & By \Cref{lemma:confluence} \\
      & $[[A3]] = [[  [OO1']aA3'  ]]$ & Above \\
      & $[[A1]] = [[ [OO']aA1'  ]] = [[ [OO1']aA1'  ]] $ & By \Cref{lemma:finish_types} \\
      & $[[ [OO1']TT3 |- [OO1']aA3' <~ [OO1']aA1'     ]]$ & By above equalities \\
      & $[[  TT3 |- [TT3]aA3' <~ [TT3]aA1' -| DD   ]]$ & By \Cref{thm:sub_completeness} \\
      & $[[ [TT3]aA1' ]] = [[ [TT3]([TT2]aA1)  ]]$ & By \cref{lemma:subst_ext_invar}\\
      & $[[  TT2 |- ae2' <= [TT2]aA1' -| DD  ]]$ & By \rref{chk-sub} \\
      $\byhave$& $[[  DD --> OO2'   ]]$ & Above \\
      & $[[ OO1' --> OO2'   ]]$ & Above \\
      $\byhave$& $[[  GG |- ae1' ae2' => aA2' -| DD  ]]$ & By \rref{inf-app} \\
      $\byhave$& $[[A2]] = [[ [OO']aA2' ]] = [[ [OO2']aA2'  ]]$ & \Cref{lemma:finish_types} \\
      $\byhave$& $[[  GG --> OO2' ]]$ & By \Cref{lemma:transitivity} \\
      $\byhave$& $\erase{[[e1 e2]]} = \erase{e_1} ~ \erase{e_2} = \erase{e_1'} ~ \erase{e_2'} = \erase{[[ ae1' ae2'  ]]}$ & By def. of erasure
    \end{longtable}


  \item Case \[     \ottaltinferrule{lam}{}{ [[ [OO]GG, x : t |- e : B  ]]  }{ [[  [OO]GG |- \x . e : t -> B  ]]   }  \]

    \begin{longtable}[l]{ll|l}
      & $[[ [OO]GG , x : t |- e : B   ]]$ & Given \\
      & $[[ [OO]GG, x : t  ]] = [[ [OO, x : at](GG , x : at)    ]]$ & By def. of context substitution \\
      & $[[ [OO, x : at](GG, x : at) |- e : B   ]]$ & By above equality \\
      & $[[ GG , x : at |- ae' => aB' -| DD'   ]]$ & By i.h., \\
      & $[[DD' --> OO']]$ & Above \\
      & $[[  OO, x : at --> OO'   ]]$ & Above \\
      & $[[B]] = [[ [OO']aB'  ]]$ & Above \\
      & $\erase{e} = \erase{e'}$ & Above \\
      & $[[ GG, x : at --> DD'   ]]$ & By \cref{lemma:typing_extension} \\
      & $[[DD']] = [[ DD, x : at, TT   ]]$ & By \cref{lemma:extension_order} \\
      & $[[ GG , x : at |- ae' => aB' -| DD, x : at, TT    ]]$ & By above equality \\
      $\byhave$& $[[  GG |- \x : at . ae' => at -> aB' -| DD   ]]$ & By \rref{inf-lamann} \\
      $\byhave$& $[[ DD --> OO'  ]]$ & By context extension \\
      $\byhave$& $[[ OO --> OO'   ]]$ & By context extension \\
      $\byhave$& $[[ t -> B  ]] = [[ at -> [OO']aB'   ]] = [[ [OO'](at -> aB')   ]]  $ & By def. of substitution \\
      $\byhave$& $\erase{[[\x . e]]} = \erlam{x}{\erase{e}} = \erlam{x}{\erase{e'}} = \erase{ [[\x : at . ae']] }$ & By def. of erasure
    \end{longtable}



  \item Case \[     \ottaltinferrule{gen}{}{ [[ [OO]GG, a |- e : A  ]]  }{ [[  [OO]GG |- e : \/a . A  ]]   }  \]

    \begin{longtable}[l]{ll|l}
      & $[[ [OO]GG, a |- e : A  ]]$ & Given \\
      & $[[ [OO]GG, a  ]] = [[  [OO, a](GG, a)   ]]$ & By def. of context substitution \\
      & $[[ [OO, a](GG, a) |- e : A    ]]$ & By above equality \\
      & $[[  GG, a |- ae' => aA' -| DD'  ]]$ & By i.h., \\
      & $[[ DD' --> OO'    ]]$ & Above \\
      & $[[ OO, a --> OO'   ]]$ & Above \\
      & $[[A]] = [[  [OO']aA'  ]]$ & Above \\
      $\byhave$& $\erase{e} = \erase{e'}$ & Above \\
      & $[[  GG, a --> DD'   ]]$ & By \cref{lemma:typing_extension} \\
      & $[[DD']] = [[DD, a, TT]]$ & By \cref{lemma:extension_order} \\
      $\byhave$& $[[  DD --> OO'   ]]$ & By context extension \\
      $\byhave$& $[[  OO --> OO'   ]]$ & By context extension \\
      & $[[  GG, a |- ae' => aA' -| DD, a , TT  ]]$ & By above equality \\
      & $ [[ DD, a, TT |- [DD, a, TT]aA' <~ [DD, a, TT]aA' -| DD, a, TT   ]]   $ & By reflexivity of consistent subtyping \\
      & $[[  GG, a |- ae' <= aA' -| DD, a, TT   ]]$ & By \rref{chk-sub} \\
      & $[[  GG |- ae' <= \/ a. aA' -| DD    ]]$ & By \rref{chk-gen} \\
      $\byhave$& $  [[  GG |- ae' : \/ a. aA' => \/ a. aA' -| DD    ]]  $ & By \rref{inf-anno} \\
      $\byhave$& $[[\/a. A]] = \forall a . \ctxsubst{\Omega'}{A'} = \ctxsubst{\Omega'}{([[\/a. aA']])}$ & By def. of substitution \\
    \end{longtable}



  \end{itemize}



\end{proof}



%%% Local Variables:
%%% mode: latex
%%% TeX-master: "../paper"
%%% org-ref-default-bibliography: ../paper.bib
%%% End:

\fi

\end{document}
