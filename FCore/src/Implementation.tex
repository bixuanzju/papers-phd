\begin{comment}
\section{Implementation}\label{sec:implementation}

We implemented\footnote{\name code repository: \url{https://github.com/hkuplg/fcore}}
 a compiler for \name based on the
representation and type-directed translation we described in
Sections~\ref{sec:fcore} and \ref{sec:tce}.  We wrote the compiler
in Haskell. This section overviews additional information about the
implementation.

\subsection{Syntax Extensions and Java Interoperability}

Our actual implementation of \name and Closure F extend the basic 
System F with other constructs. These include primitive
operations, types and literals, let bindings, conditional expressions,
tuples, and fixpoints. We also have different frontends, 
which extend \name with convenient language constructs that helped us in development.
We wrote our benchmark programs in one 
of these frontends. 
%\name and Closure F intermediate languages' value and type
%expressions with several useful constructs. 

We also added constructs for a basic interoperability mechanism with
Java and the JVM: creating instances of classes, calling methods,
accessing fields, and executing blocks of statements.  We type-check these constructs
using an ``oracle''.  In the implementation, the type-checker sends
queries about Java constructs to a server with a set classpath and the
server answers these queries using the Reflection
API. One advantage of this design is that it allows
us to target different Java-based platforms. Apart from supporting the
JVM, we can support the Android platform. The Android platform
uses a different set of libraries from the JVM SDK. 

Besides benchmark programs, we have a number of other programs written 
in \Name. These programs include functional programs for 
drawing fractals (such as the Mandelbrot set, the Burning Ship, or the
Sierpinski Triangle) which helped us to test the Java interoperability.
We also have a test suite with unit tests of the \name compiler.

%\subsection{Multi-argument Closure Optimization}

\subsection{Additional Optimizations}

Our compiler also performs standard optimizations that are not
accounted by the basic translations in Sections~\ref{sec:fcore} and
\ref{sec:tce}.  For unboxing, different \lstinline{Function} classes exist
with specialized primitive fields. The translation chooses among them
and generates variables with primitive types when possible. The
compiler inlines expressions of let bindings, value abstractions, and
fixpoints; and partially evaluates value applications, conditionals,
and primitive type operations.

An important optimization that our compiler also does is
multi-argument function optimization. This kind of optimization is
standard in FP~\cite{marlow06making} and provides a major boost in
time and memory performance. The straightforward
translation described in the previous section has one drawback when
translating multi-argument functions: for each single value
application, we generate a function object that requires an apply call
to execute the body. Using the additional information that Closure F
has about multi-argument binders, our compiler creates a single
multi-argument function, instead of multiple single argument
functions. 
\end{comment}
\begin{comment}
The idea employed in our implementation is similar 
to the alternative encoding of curried functions briefly presented 
in Section~\ref{subsec:ifos}.
\end{comment}

%In
% functional programming, we encounter different common scenarios, such
% as partial applied or higher-order functions, when this approach can
% become a major performance penalty. For this reason, we perform one
% optimization for these cases.  In this optimization, we augment the
% translation in two ways: we eliminate apply calls in partial
% applications and move the code blocks from apply methods to
% initialization blocks in the corresponding Closure objects. In order
% to do these two steps, we need to be able to detect partial
% applications.  Luckily, we can simply see this in types of our
% representation. When translating $E1~E2$, we have the following
% options for the E1's type $\Delta$ environment:
%
%\begin{enumerate}
%  \item $\forall \alpha.\Delta$: we continue checking next binders of $\Delta$.
%  \item $\forall (x: T).\Delta$: we know this is a partial application, hence we do not generate an apply call and move its body to the initialization block.
%  \item $\forall (x: T).\epsilon$: we know this is a full application, hence we generate code as usual.
%\end{enumerate}
