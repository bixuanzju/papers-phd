\section{\name and IFOs, Informally}\label{sec:overview}

This section informally presents \name programs and their IFO-based
encoding and how to deal with tail-call elimination.
Sections~\ref{sec:fcore} and \ref{sec:tce} present a formalized
compilation method for a subset of \name (System F) into Java, based on the ideas from this section.
Note that, for purposes of presentation, we show slightly simplified encodings in this
section compared to the formal compilation method.

\subsection{Encoding Functions with IFOs}\label{subsec:ifos}

\noindent In \Name, we compile all functions to classes extending the \lstinline{Function} class presented in
Section~\ref{sec:intro}. For example, consider a simple identity
function on integers. In \name or System F (extended with integers),
we represent it as follows:

\vspace{5pt}
$id \equiv \lambda (x : Int).~x$
\vspace{5pt}

\noindent We can manually encode this definition with an IFO in Java as follows:

\begin{lstlisting}
class Id extends Function {
    public void apply () {
        final Integer x = (Integer) this.arg;
        res = x;
    }
}
\end{lstlisting}

\noindent The \lstinline{arg} field encodes the argument of the
function, whereas the \lstinline{res} field encodes the result. Thus,
to create the identity function, all we need to do is to copy the
argument to the result. A function invocation such as
$id~3$ is encoded as follows:

\begin{lstlisting}
Function id = new Id();
id.arg = 3; // setting argument
id.apply();  // invoking apply()
\end{lstlisting}

\noindent The function application goes in two steps:
it first sets the \lstinline{arg} field to \lstinline{3} and then
invokes the \lstinline{apply()} method.

\subsubsection{Curried Functions.} IFOs can naturally define
curried functions, such as:

\vspace{5pt}
$constant \equiv \lambda (x : Int) .~\lambda (y : Int) .~x$
\vspace{5pt}

\noindent Given two integer arguments, this function will always return
the first one. Using IFOs, we can encode $constant$ in Java as follows:

\begin{lstlisting}
class Constant extends Function {
    public void apply () {
        final Integer x = (Integer) this.arg;
        class IConstant extends Function {
            public void apply() {
                final Integer y = (Integer) this.arg;
                res = x;
            }
        }
        res = new IConstant();
    }
}
\end{lstlisting}

\noindent Here, the first lambda function sets the second one as its result.
The definition of the second \lstinline{apply} method
sets the result of the function to the argument of the first lambda
function. The use of inner classes enforces the lexical scoping of functions.
 We encode an application such as $constant~3~4$ as:

\begin{lstlisting}
Function constant = new Constant();
constant.arg = 3;
constant.apply();
Function f = (Function) constant.res;
f.arg = 4;
f.apply();
\end{lstlisting}

\noindent We first set the argument of the \lstinline{constant} function to $3$. Then,
we invoke the \lstinline{apply} method and store the
resulting function to a variable \lstinline{f}. Finally, we set the
argument of \lstinline{f} to $4$ and invoke \lstinline{f}'s
\lstinline{apply} method.
Note that the alias \lstinline{x} for \lstinline{this.arg} is needed to prevent accidental
overwriting of arguments in partial applications. For example in
$constant~3~(constant~4~5)$, the inner application $constant~4~5$ would overwrite
$3$ to $4$ and the outer one would incorrectly return $4$.

\subsubsection{Partial Function Application.} With curried functions, we can encode partial application easily.
For example, consider the following
expression:
\vspace{5pt}
$three \equiv constant~3$.
\vspace{5pt}
\noindent The code for this partial application is simply:
\begin{lstlisting}
Function constant = new Constant();
constant.arg = 3;
constant.apply();
\end{lstlisting}
\begin{figure}[t!]
\begin{tabular}[t]{l r}
\begin{lstlisting}
// tail-call elimination
class Mutual {
	Function teven;
	Function todd;
	class TEven extends Function {
		public void apply () {
			final Integer n =
			  (Integer) this.arg;
			if (n == 0) {
				res = true;
			}
			else {
				todd.arg = n - 1;
				// tail call
				Next.next = todd;
			}
		}
	}
\end{lstlisting}&
\hspace{20pt}\begin{lstlisting}
	class TOdd extends Function {
		public void apply () {
			final Integer n =
			  (Integer) this.arg;
			if (n == 0) {
				res = false;
			}
			else {
				teven.arg = n - 1;
				// tail call
				Next.next = teven;
			}
		}
	}
	{ // initialization block
		todd = new TOdd();
		teven = new TEven();
	}
}
\end{lstlisting}
\end{tabular}

\caption{Functions \lstinline{even} and \lstinline{odd} using IFOs with
  tail-call elimination}

\label{fig:even}

\end{figure}
\subsubsection{Recursion.} \name supports simple recursion, as well as
mutual recursion. For example, consider the functions $even$ and $odd$
defined to be mutually recursive:

\vspace{5pt}
\noindent$even \equiv \lambda (n : Int).~{\bf if}~(n=0)~{\bf then}~true~{\bf  else}~odd(n-1)$ \\
\noindent$odd \equiv \lambda (n : Int).~{\bf if}~(n=0)~{\bf then}~false~{\bf else}~even(n-1)$
\vspace{5pt}

\noindent These two functions define a naive algorithm for detecting
whether a number is even or odd. We can encode recursion using Java's own
recursion: the Java references \lstinline{even} and
\lstinline{odd} are themselves mutually recursive (Figure~\ref{fig:even}).

\subsection{Tail-call Elimination}

The recursive calls in $even$ and $odd$ are tail calls.
IFOs present new ways for doing tail-call elimination in the JVM.
The key idea, inspired by Steele's work on encoding tail-call
elimination~\cite{Steele1978}, is to use a simple
auxiliary structure
\begin{lstlisting}
class Next {static Function next = null;}
\end{lstlisting}
\noindent that keeps track of the next call to be executed. Figure~\ref{fig:even} illustrates the use of the
\lstinline{Next} structure. This is where we make a fundamental use of the fact
that function application is divided into two parts with IFOs.
In tail calls, we set the arguments of the
function, but we delay the \lstinline{apply} method calls.
Instead, the \lstinline{next} field of \lstinline{Next} is set
to the function with the \lstinline{apply} method. The \lstinline{apply}
method is then invoked at the call-site of the functions. The code in Figure \ref{fig:trampoline} illustrates the call \lstinline{even 10}.
In JVM-style trampolines, each (method) call creates a Thunk. IFOs, however, are reused
throughout the execution.
\begin{figure}[t]
\begin{tabular}[t]{l l}
\begin{lstlisting}
// TCE with JVM-style trampolines
interface Thunk {
  Object apply();
}
...
  static Object teven(final int n) {
    if(n == 0) return true;
    else return new Thunk() {
      public Object apply() {
        return todd(n-1);
      }
    };
  }
  static Object todd(final int n) {
    if(n == 0) return false;
    else return new Thunk() {
      public Object apply() {
        return teven(n-1);
      }
    };
...
Object trampoline = even(10);
while(trampoline instanceof Thunk)
      trampoline =
      ((Thunk) trampoline).apply();
return (Boolean) trampoline;
\end{lstlisting}&
\hspace{10pt}\begin{lstlisting}
// TCE with IFOs + Next
Mutual m = new Mutual();
Function teven = m.teven;
...
teven.arg = 10;
Next.next = teven;
Function c;
Boolean res;
do {
  c = Next.next;
  Next.next = null;
  c.apply();
} while (Next.next != null);
res = (Boolean) c.res;
\end{lstlisting}
\end{tabular}

\caption{This figure contrasts the TCE approach with JVM-style trampolines (left, custom
implementation) and with IFOs and the Next handler (right, see Figure \ref{fig:even} for implementation).}

\label{fig:trampoline}
\vspace{-15pt}
\end{figure}
\noindent The idea is that a function call (which is not a tail-call)
has a loop that jumps back-and-forth into functions. The technique
is similar to some trampoline approaches in C-like languages. However,
an important difference to JVM-style trampolines is that utilization of heap space is not growing.
In other words, tail-calls do not create new objects for their execution, which
improves memory and time performance.
Note that this method is \emph{general}: it works for \emph{simple recursive
tail calls}, \emph{mutually recursive tail calls}, and \emph{non-recursive tail calls}.
