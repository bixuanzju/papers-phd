\section{Introduction}\label{sec:intro}

% Motivation
A runtime environment, such as the JVM, attracts both functional programming (FP) languages'
compiler writers and users: it enables cross-platform development and comes with a
large collection of libraries and tools. Moreover, FP languages give programmers
on the JVM other benefits: simple, concise and elegant ways to write different algorithms;
high code reuse via higher-order functions; and more opportunities for parallelism,
 by avoiding the overuse of side-effects (and shared mutable state) \cite{erlang}.
 Unfortunately, compilers for functional languages are hard to
implement efficiently in the JVM. FP promotes a
programming style where \emph{functions are first-class values} and \emph{recursion} is used
instead of mutable state and loops to define algorithms. The JVM is
not designed to deal with such programs.


The difficulty in optimizing FP in the JVM means that:
\emph{while FP in the JVM is possible today,
some compromises are still necessary for writing efficient programs.}
Existing JVM functional languages, including
Scala~\cite{Odersky2014b}
and Clojure~\cite{Hickey2008}, usually
work around the challenges imposed by the JVM. Those languages give
programmers alternatives to a FP
style. Therefore, performance-aware programmers avoid certain idiomatic
FP styles, which may be costly in those languages, and use
the available alternatives instead.

%\paragraph{JVM Challenges}
In particular, one infamous challenge when writing a
compiler for a functional language targeting the JVM is: How to eliminate and/or optimize tail calls?
Before tackling that, one needs to decide how to represent functions in the JVM.
There are two standard options: \emph{JVM methods} and
\emph{functions as objects} (FAOs).
Encoding first-class functions using only JVM methods directly is
limiting: JVM methods cannot encode currying and partial function application directly.
To support these features, the majority of functional languages or extensions (including Scala,
Clojure, and Java 8) adopt variants of the functions-as-objects
approach:

\begin{lstlisting}
interface FAO { Object apply(Object arg);}
\end{lstlisting}

\noindent With this representation,
we can encode curried functions, partial application and
pass functions as arguments.
However, neither FAOs nor JVM methods offer a good solution to
deal with \emph{general tail-call elimination}
(TCE)~\cite{Steele1977}. The JVM does not support proper tail calls.
In particular scenarios, such as single, tail-recursive calls, we can
easily achieve an optimal solution in the JVM.  Both Scala and Clojure
provide some support for tail-recursion~\cite{Odersky2014b,recur}.
However, for more general tail calls (such as mutually recursive
functions or non-recursive tail calls), existing solutions can worsen
the overall performance. For example, JVM-style trampolines~\cite{Schinz2001}
(which provide a general solution for tail calls) are significantly
slower than normal calls and consume heap memory for every tail call.

\subsubsection{Contributions.}
This paper presents a new JVM compilation technique for
functional programs, and creates an implementation of System
F~\cite{girard72dissertation,reynolds74towards} using the new
technique. The compilation technique builds on a new representation of
first-class functions in the JVM: \emph{imperative functional
  objects} (IFOs).
\emph{With IFOs it is possible to use a single
 representation of functions in the JVM and still achieve memory-efficient TCE}.
 As a first-class function representation, IFOs also support currying and
 partial function applications.

We represent an IFO by the following abstract class:

\begin{lstlisting}
abstract class Function {
   Object arg, res;
   abstract void apply();
}
\end{lstlisting}

\noindent With IFOs, we encode both the argument
(\lstinline{arg}) and the result of the functions (\lstinline{res})
as mutable fields.
We set the argument field before
invoking the \lstinline{apply()} method. At the end of the \lstinline{apply()} method, we set  the result field.
An important difference between the IFOs and FAOs encoding of
first-class functions is that, in IFOs, \emph{function application is
divided in two parts}: \emph{setting the argument field}; and \emph{invoking the
apply method}.
For example, if we have a function call
\lstinline{factorial 10}, the corresponding Java code using IFOs
is:

\begin{lstlisting}
factorial.arg = 10;  // setting argument
factorial.apply();      // invoking function
\end{lstlisting}

The fact that we can split function application into two parts is key
to enable new optimizations related to functions
in the JVM. In particular, the TCE
approach with IFOs does not require memory allocation for each tail
call and has less execution time overhead than the JVM-style trampolines used in languages
such as Clojure and Scala. Essentially, with IFOs, it is possible to
provide a straightforward TCE implementation, resembling Steele's ``UUO
handler''~\cite{Steele1978}, in the JVM.

Using IFOs and the TCE technique, we created
\Name: a JVM implementation of an extension of \emph{System
  F}.
\name aims to serve as an intermediate functional layer on top
of the JVM, which ML-style languages can target. According to our experimental results,
\name programs perform competitively against programs using regular JVM methods,
while still supporting TCE. Programs in \name tend to have
less execution time overhead
and use less memory than programs using conventional
JVM trampolines.

In summary, the contributions of this paper are:

\begin{itemize}

\item {\bf Imperative Functional Objects:} A new representation of
  first-class functions in the JVM, offering new ways to
  optimize functional programs.

\item {\bf A memory-efficient approach to tail-call elimination:} A way to
  implement TCE in the JVM using IFOs without allocating memory per each tail call.

\item \Name: An implementation of a System F-based
  intermediate language that can be used to target the JVM by FP
  compilers.

\item {\bf Formalization and empirical results:} Our basic compilation
  method from a subset of \name into Java is formalized. Our empirical
  results indicate that \name allows general TCE in constant memory space
  and with execution time comparable to regular JVM methods.

\end{itemize}
