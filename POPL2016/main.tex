% SIGPLAN template
\documentclass[preprint]{sigplanconf}

%% -- Packages Imports --

% AMS stuff
\usepackage{amsmath}
\usepackage{amssymb}
\usepackage{amsthm}
\usepackage{mathtools}
\usepackage{xspace}
\usepackage{comment}

% Language
\usepackage{csquotes}
\usepackage[british]{babel}
\MakeOuterQuote{"}


\newcommand{\name}{{\bf $\lambda_{\star}^{\mu}$}\xspace}
\newcommand{\coc}{{\bf $\lambda C$}\xspace}
\newcommand{\ecore}{$\lambda_{\star}$\xspace}
\newcommand{\cc}{$\lambda C$\xspace}
\newcommand{\sufcc}{{\bf Fun}\xspace}
\newcommand{\fold}{{\bf $\mathsf{fold}$}\xspace}
\newcommand{\unfold}{{\bf $\mathsf{unfold}$}\xspace}

\newcommand{\authornote}[3]{{\color{#2} {\sc #1}: #3}}
\newcommand\bruno[1]{\authornote{bruno}{red}{#1}}
\newcommand\jeremy[1]{\authornote{jeremy}{blue}{#1}}
\newcommand\linus[1]{\authornote{linus}{green}{#1}}
\newcommand{\fixme}[1]{{\color{red} #1}}

% Hyper links
\usepackage{hyperref}
\hypersetup{
   colorlinks,
   citecolor=black,
   filecolor=black,
   linkcolor=black,
   urlcolor=black
}

% Compact list
\usepackage{paralist}

% Figure import
\usepackage{graphicx}
\usepackage{float}

% Code highlighting
\usepackage{listings}
\usepackage{xcolor}
\newcommand{\hl}[2][gray!40]{\colorbox{#1}{#2}}
\newcommand{\hlmath}[2][gray!40]{%
  \colorbox{#1}{$\displaystyle#2$}}

% Theorem
\usepackage{amsthm}
\newtheorem{thm}{Theorem}[section]
\newtheorem{lem}[thm]{Lemma}
\newtheorem{dfn}[thm]{Definition}

%% Typesetting inference rules
\usepackage{styles/mathpartir}  % by Didier Rémy (http://gallium.inria.fr/~remy/latex/mathpartir.html

% Ott includes
\input{sections/expcore.ott.tex}
\input{sections/otthelper.tex}

% lhs2tex
\usepackage{mylhs2tex}

% Subsection style
\usepackage{titlesec}
\titleformat{\subsubsection}[runin]{\bfseries\itshape\normalsize}{}{0em}{}[$\;$]

% Table
\usepackage{booktabs}
\usepackage{tabularx}
\usepackage[para,online,flushleft]{threeparttable}

%% -- Packages Imports --

% Main
\begin{document}

% Page size - US Letter
%\special{papersize=8.5in,11in}
%\setlength{\pdfpageheight}{\paperheight}
%\setlength{\pdfpagewidth}{\paperwidth}

% Conference info
\conferenceinfo{CONF 'yy}{Month d--d, 20yy, City, ST, Country}
\copyrightyear{20yy}
\copyrightdata{978-1-nnnn-nnnn-n/yy/mm}
\doi{nnnnnnn.nnnnnnn}

% Title
\titlebanner{} % Only for preprint
\preprintfooter{} % Only for preprint

\title{Type-Level Computation One Step at a Time}
%\title{A Dependently-typed Intermediate Language with General Recursion}
%\subtitle{Or: Decidable Type-Checking in the presence of
%  Type-Level General Recursion}

\authorinfo{}{}{}
% \authorinfo{Foo \and Bar \and Baz}
%            {The University of Foo}
%            {\{foo,bar,baz\}@foo.edu}

\maketitle

% Abstract
\begin{abstract}
Many type systems support a conversion rule that allows type-level
computation. In such type systems ensuring the \emph{decidability} of
type checking requires type-level computation to terminate.
For calculi where the syntax of types and terms is the same, the
decidability of type checking is usually dependent on the strong normalization
of the calculus, which ensures termination. An unfortunate
consequence of this coupling between decidability and strong
normalization is that adding (unrestricted) general recursion to such
calculi is difficult.

This paper proposes an alternative to the conversion rule that allows
the same syntax for types and terms, type-level computation, and
preserves decidability of type checking under the presence of general
recursion. The key idea, which is inspired by the traditional
treatment of \emph{iso-recursive types}, is to make each type-level
computation step explicit. Each beta reduction or expansion at the
type level is introduced by type-safe cast construct. Such construct allows control
over the type-level computation and ensures decidability of
type checking even in the presence of non-terminating programs at the
type level.  We realize this idea by presenting \name: a variant of the
calculus of constructions with general recursion and recursive types.
Furthermore we show how many advanced programming language features of
state-of-the-art functional languages (such as Haskell) can be encoded
in our minimalistic core calculus.
\end{abstract}

% Category, terms & keywords
\category{D.3.1}{Programming Languages}{Formal Definitions and Theory}
\terms Languages, Design
\keywords Dependent types, Intermediate language

%% -- Starting Point -- 


\section{Introduction}
\label{sec:introduction}

Compositionality is a desirable property in programming
designs. Broadly defined, it is the principle that a
system should be built by composing smaller subsystems. For instance,
in the area of programming languages, compositionality is
a key aspect of \emph{denotational semantics}~\cite{scott1971toward, scott1970outline}, where
the denotation of a program is constructed from the denotations of its parts.
% For example, the semantics for a language of simple arithmetic expressions
% is defined as:
% 
% \[\begin{array}{lcl}
% \llbracket n \rrbracket_{E} & = & n \\
% \llbracket e_1 + e_2 \rrbracket_{E} & = & \llbracket e_1 \rrbracket_E + \llbracket  e_2 \rrbracket_E \\
% \end{array}\]
% 
% \bruno{Replace E by fancier symbol?}
% Here there are two forms of expressions: numeric literals and
% additions. The semantics of a numeric literal is just the numeric
% value denoted by that literal. The semantics of addition is the
% addition of the values denoted by the two subexpressions.
Compositional definitions have many benefits.
One is ease of reasoning: since compositional
definitions are recursively defined over smaller elements they
can typically be reasoned about using induction. Another benefit
is that compositional definitions are easy to extend,
without modifying previous definitions.
% For example, if we also wanted to support multiplication,
% we could simply define an extra case:
% 
% \[\begin{array}{lcl}
% \llbracket e_1 * e_2 \rrbracket_E & = & \llbracket e_1 \rrbracket_E * \llbracket  e_2 \rrbracket_E \\
% \end{array}\]

Programming techniques that support compositional
definitions include:
\emph{shallow embeddings} of
Domain Specific Languages (DSLs)~\cite{DBLP:conf/icfp/GibbonsW14}, \emph{finally
  tagless}~\cite{CARETTE_2009}, \emph{polymorphic embeddings}~\cite{hofer_polymorphic_2008} or
\emph{object algebras}~\cite{oliveira2012extensibility}. These techniques allow us to create
compositional definitions, which are easy to extend without
modifications. Moreover, when modeling semantics, both finally tagless and object algebras
support \emph{multiple interpretations} (or denotations) of
syntax, thus offering a solution to the well-known \emph{Expression Problem}~\cite{wadler1998expression}.
Because of these benefits these techniques have become
popular both in the functional and object-oriented
programming communities.

However, programming languages often only support simple compositional designs
well, while support for more sophisticated compositional designs is lacking.
For instance, once we have multiple interpretations of syntax, we may wish to
compose them. Particularly useful is a \emph{merge} combinator,
which composes two interpretations~\cite{oliveira2012extensibility,
oliveira2013feature, rendel14attributes} to form a new interpretation that,
when executed, returns the results of both interpretations. 

% For example, consider another pretty printing interpretation (or
% semantics) $\llbracket \cdot \rrbracket_P$ for arithmetic expressions, which
% returns the string that denotes the concrete syntax of the
% expression. Using merge we can compose the two interpretations to
% obtain a new interpretation that executes both printing and evaluation:
% \jeremy{Explain what is $E\,\&\,P$?}
% 
% \[\begin{array}{lcl}
% \llbracket \cdot \rrbracket_E \otimes \llbracket \cdot \rrbracket_P & = & \llbracket \cdot \rrbracket_{E\,\&\,P} \\
% \end{array}\]

The merge combinator can be manually defined in existing programming languages,
and be used in combination with techniques such as finally tagless or object
algebras. Moreover variants of the merge combinator are useful to
model more complex combinations
of interpretations. A good example are so-called \emph{dependent} interpretations,
where an interpretation does not depend \emph{only} on itself, but also on 
a different interpretation. These definitions with dependencies are quite
common in practice, and, although they are not orthogonal to the interpretation they
depend on, we would like to model them (and also mutually dependent interpretations)
in a modular and compositional style.

% For example consider the following two
% interpretations ($\llbracket \cdot \rrbracket_{\mathsf{Odd}}$ and
% $\llbracket \cdot \rrbracket_{\mathsf{Even}}$) over Peano-style natural numbers:

% \[\begin{array}{lclclcl}
% \llbracket 0 \rrbracket_{\mathsf{Even}}  & = & \mathsf{True} & ~~~~~~~~~~~~~~~~~~~~ & \llbracket 0 \rrbracket_{\mathsf{Odd}} & = & \mathsf{False} \\
% \llbracket S~e \rrbracket_{\mathsf{Even}} & = & \llbracket e \rrbracket_{\mathsf{Odd}} & ~~ & \llbracket S~e \rrbracket_{\mathsf{Odd}} & = & \llbracket e \rrbracket_{\mathsf{Even}}\\
% \end{array}\]

% \emph{Are these interpretations compositional or not?} Under
% a strict definition of compositionality they are not because
% the interpretation of the parts does not depend \emph{only} on the
% interpretation being defined. Instead both interpretations also depend
% on the other interpretation of the parts. In general,
% definitions with dependencies are quite common in practice.
% In this paper we consider these
% interpretations compositional, and we
% would like to model such dependent (or even mutually dependent)
% interpretations in a modular and compositional style.

Defining the merge combinator in existing
programming languages is verbose and cumbersome, requiring code for every
new kind of syntax. Yet, that code is essentially mechanical and ought to be
automated. 
While using advanced meta-programming techniques enables automating
the merge combinator to a large extent in existing programming
languages~\cite{oliveira2013feature, rendel14attributes}, those techniques have
several problems: error messages can be problematic, type-unsafe reflection
is needed in some approaches~\cite{oliveira2013feature} and
advanced type-level features are required in others~\cite{rendel14attributes}.
An alternative to the merge combinator that supports modular multiple
interpretations and works in OO languages with
support for some form of multiple inheritance and covariant
type-refinement of fields has also been recently
proposed~\cite{zhang19shallow}. 
While this approach is relatively simple, it still
requires a lot of manual boilerplate code for composition of interpretations.

This paper presents a calculus and polymorphic type system with
\emph{(disjoint) intersection types}~\cite{oliveira2016disjoint},
called \fnamee. \fnamee
supports our broader notion of compositional designs, and enables
the development of highly modular and reusable programs. \fnamee
has a built-in merge operator and a powerful subtyping relation that
are used to automate the composition of multiple (possibly dependent)
interpretations. In \fnamee subtyping is coercive and enables the
automatic generation of coercions in a \emph{type-directed} fashion. 
This process is similar to that of other type-directed code generation mechanisms
such as 
\emph{type classes}~\cite{Wadler89typeclasses}, which eliminate 
boilerplate code associated to the \emph{dictionary translation}~\cite{Wadler89typeclasses}.

\fnamee continues a line of
research on disjoint intersection types.
 Previous work on
\emph{disjoint polymorphism} (the \fname calculus)~\cite{alpuimdisjoint} studied the
combination of parametric polymorphism and disjoint intersection
types, but its subtyping relation does not support
BCD-style distributivity rules~\cite{Barendregt_1983} and the type system
also prevents unrestricted intersections~\cite{dunfield2014elaborating}. More recently the \name
calculus (or \namee)~\cite{bi_et_al:LIPIcs:2018:9227} introduced a system with \emph{disjoint
  intersection types} and BCD-style distributivity rules, but did not
account for parametric polymorphism. \fnamee is unique in that it
combines all three features in a single calculus:
\emph{disjoint intersection types} and a \emph{merge operator};
\emph{parametric (disjoint) polymorphism}; and a BCD-style subtyping
relation with \emph{distributivity rules}. The three features together
allow us to improve upon the finally tagless and object
algebra approaches and support advanced compositional designs.
Moreover previous work on disjoint intersection types has shown 
various other applications that are also possible in \fnamee, including: \emph{first-class
  traits} and \emph{dynamic inheritance}~\cite{bi_et_al:LIPIcs:2018:9214}, \emph{extensible records} and \emph{dynamic
  mixins}~\cite{alpuimdisjoint}, and \emph{nested composition} and \emph{family polymorphism}~\cite{bi_et_al:LIPIcs:2018:9227}. 


Unfortunately the combination of the three features has non-trivial
complications. The main technical challenge (like for most other
calculi with disjoint intersection types) is the proof of coherence
for \fnamee. Because of the presence of BCD-style distributivity
rules, our coherence proof is based on the recent approach employed in
\namee~\cite{bi_et_al:LIPIcs:2018:9227}, which uses a
\emph{heterogeneous} logical relation called \emph{canonicity}. To account for polymorphism,
which \namee's canonicity does not support, we originally wanted
to incorporate the relevant parts of System~F's logical relation~\cite{reynolds1983types}.
However, due to a mismatch between the two relations, this did not work. The
parametricity relation has been carefully set up with a delayed type
substitution to avoid ill-foundedness due to its impredicative polymorphism.
Unfortunately, canonicity is a heterogeneous relation and needs to account for
cases that cannot be expressed with the delayed substitution setup of the
homogeneous parametricity relation. Therefore, to handle those heterogeneous
cases, we resorted to immediate substitutions and 
% restricted \fnamee to
\emph{predicative instantiations}.
%other
%measures to avoid the ill-foundedness of impredicative instantiation.
%We have settled on restricting \fnamee to \emph{predicative polymorphism} to
%keep the coherence proof manageable. 
We do not believe that predicativity is a severe restriction in practice, since many source
languages (e.g., those based on the Hindley-Milner type system like Haskell and
OCaml) are themselves predicative and do not require the full generality of an
impredicative core language. Should impredicative instantiation be required,
we expect that step-indexing~\cite{ahmed2006step} can be used to recover well-foundedness, though
at the cost of a much more complicated coherence proof.

The formalization and metatheory of \fnamee are a significant advance over that of
\fname. Besides the support for distributive subtyping, \fnamee removes 
several restrictions imposed by the syntactic coherence
proof in \fname. In particular \fnamee supports unrestricted
intersections, which are forbidden in \fname. Unrestricted
intersections enable, for example, encoding certain forms of 
bounded quantification~\cite{pierce1991programming}.
Moreover the new proof method is more robust
with respect to language extensions. For instance, \fnamee supports the bottom
type without significant complications in the proofs, while it was a challenging
open problem in \fname.
A final interesting aspect is that \fnamee's type-checking is decidable. In the
design space of languages with polymorphism and subtyping, similar mechanisms
have been known to lead to undecidability. Pierce's seminal paper
``\emph{Bounded quantification is undecidable}''~\cite{pierce1994bounded} shows
that the contravariant subtyping rule for bounded quantification in
\fsub leads to undecidability of subtyping.  In \fnamee the
contravariant rule for disjoint quantification retains decidability. 
Since with unrestricted intersections \fnamee can express several
use cases of bounded quantification, \fnamee could be an interesting and
decidable alternative to \fsub.

\begin{comment}
Besides coherence, we show
several other important meta-theoretical results, such as type-safety, 
sound and complete algorithmic subtyping, and
decidability of the type system. Remarkably, unlike 
\fsub's \emph{bounded polymorphism}, disjoint polymorphism
in \fnamee supports decidable type-checking.
\end{comment}

In summary the contributions of this paper are:
\begin{itemize}

\item {\bf The \fnamee calculus,} which is the first calculus to combine 
disjoint intersection types, BCD-style distributive subtyping and 
disjoint polymorphism. We show several meta-theoretical results, such as \emph{type-safety}, \emph{sound and complete algorithmic subtyping},
\emph{coherence} and \emph{decidability} of the type system.
\fnamee includes the \emph{bottom type}, which was considered to be a
significant challenge in previous work on disjoint polymorphism~\cite{alpuimdisjoint}.

\item {\bf An extension of the canonicity relation with polymorphism,}
  which enables the proof of coherence of \fnamee. We show that the ideas of
  System F's \emph{parametricity} cannot be ported to
  \fnamee. To overcome the problem we use a technique based on
  immediate substitutions and a predicativity restriction.

% \item {\bf Disjoint intersection types in the presence of bottom:}
%   Our calculus includes the bottom type, which was considered to be a
% significant challenge in previous work on disjoint polymorphism~\cite{alpuimdisjoint}.

\item {\bf Improved compositional designs:} We show that \fnamee's combination of features
enables improved
compositional programming designs and supports automated composition
of interpretations in programming techniques like object algebras and
finally tagless.

\item {\bf Implementation and proofs:} All of the metatheory
  of this paper, except some manual proofs of decidability, has been
  mechanically formalized in Coq. Furthermore, \fnamee is
  implemented and all code presented in the paper is available. The
  implementation, Coq proofs and extended version with appendices can be found in
  \url{https://github.com/bixuanzju/ESOP2019-artifact}.

\end{itemize}

% \bruno{
% Still need to figure out how to integrate row types in the intro story
% Furthermore, we provide a detailed
% comparison between \emph{distributive disjoint polymorphism} and
% \emph{row types}.
% }

% Compositionality is a desirable property in programming
% designs. Broadly defined, compositionality is the principle that a
% system should be built by composing smaller subsystems.
% In the area of programming languages compositionality is
% a key aspect of \emph{denotational semantics}~\cite{scott1971toward, scott1970outline}, where
% the denotation of a program is constructed from denotations of its parts.
% For example, the semantics for a language of simple arithmetic expressions
% is defined as:
% 
% \[\begin{array}{lcl}
% \llbracket n \rrbracket_{E} & = & n \\
% \llbracket e_1 + e_2 \rrbracket_{E} & = & \llbracket e_1 \rrbracket_E + \llbracket  e_2 \rrbracket_E \\
% \end{array}\]
% 
% \bruno{Replace E by fancier symbol?}
% Here there are two forms of expressions: numeric literals and
% additions. The semantics of a numeric literal is just the numeric
% value denoted by that literal. The semantics of addition is the
% addition of the values denoted by the two subexpressions.
% Compositional definitions have many benefits.
% One is ease of reasoning: since compositional
% definitions are recursively defined over smaller elements they
% can typically be reasoned about using induction. Another benefit
% of compositional definitions is that they are easy to extend,
% without modifying previous definitions.
% For example, if we also wanted to support multiplication,
% we could simply define an extra case:
% 
% \[\begin{array}{lcl}
% \llbracket e_1 * e_2 \rrbracket_E & = & \llbracket e_1 \rrbracket_E * \llbracket  e_2 \rrbracket_E \\
% \end{array}\]
% 
% Programming techniques that support compositional
% definitions include:
% \emph{shallow embeddings} of
% Domain Specific Languages (DSLs)~\cite{DBLP:conf/icfp/GibbonsW14}, \emph{finally
%   tagless}~\cite{CARETTE_2009}, \emph{polymorphic embeddings}~\cite{} or
% \emph{object algebras}~\cite{oliveira2012extensibility}. All those techniques allow us to easily create
% compositional definitions, which are easy to extend without
% modifications. Moreover both finally tagless and object algebras
% support \emph{multiple interpretations} (or denotations) of
% the syntax, thus offering a solution to the infamous \emph{Expression Problem}~\cite{wadler1998expression}.
% Because of these benefits they have become
% popular both in the functional and object-oriented
% programming communities.
% 
% However, programming languages often only support simple
% compositional designs well, while language support for more sophisticated
% compositional designs is lacking. Once we have multiple
% interpretations of syntax, then we may wish to compose those
% interpretations. In particular, when multiple interpretations exist, a useful operation
% is a \emph{merge} combinator ($\otimes$) that composes two
% interpretations~\cite{oliveira2012extensibility, oliveira2013feature, rendel14attributes}, forming a
% new interpretation that, when executed, returns the results of both
% interpretations. For example, consider another pretty printing interpretation (or
% semantics) $\llbracket \cdot \rrbracket_P$ for arithmetic expressions, which
% returns the string that denotes the concrete syntax of the
% expression. Using merge we can compose the two interpretations to
% obtain a new interpretation that executes both printing and evaluation:
% \jeremy{Explain what is $E\,\&\,P$?}
% 
% \[\begin{array}{lcl}
% \llbracket \cdot \rrbracket_E \otimes \llbracket \cdot \rrbracket_P & = & \llbracket \cdot \rrbracket_{E\,\&\,P} \\
% \end{array}\]
% 
% Such merge combinator can be manually defined in existing programming 
% The merge combinator can be manually defined in existing programming
% languages, and be used in combination with techniques such as finally
% tagless or object algebras. Furthermore variants of the
% merge combinator can help express more complex combinations of multiple
% interpretations. For example consider the following two
% interpretations ($\llbracket \cdot \rrbracket_{\mathsf{Odd}}$ and
% $\llbracket \cdot \rrbracket_{\mathsf{Even}}$) over Peano-style natural numbers:
% 
% \[\begin{array}{lclclcl}
% \llbracket 0 \rrbracket_{\mathsf{Even}}  & = & \mathsf{True} & ~~~~~~~~~~~~~~~~~~~~ & \llbracket 0 \rrbracket_{\mathsf{Odd}} & = & \mathsf{False} \\
% \llbracket S~e \rrbracket_{\mathsf{Even}} & = & \llbracket e \rrbracket_{\mathsf{Odd}} & ~~ & \llbracket S~e \rrbracket_{\mathsf{Odd}} & = & \llbracket e \rrbracket_{\mathsf{Even}}\\
% \end{array}\]
% 
% \emph{Are these interpretations compositional or not?} Under
% a strict definition of compositionality they are not because
% the interpretation of the parts does not depend \emph{only} on the
% interpretation being defined. Instead both interpretations also depend
% on the other interpretation of the parts. In general,
% definitions with dependencies are quite common in practice.
% In this paper we consider these
% interpretations compositional, and we
% would like to model such dependent (or even mutually dependent)
% interpretations in a modular and compositional style.
% 
% However defining the merge combinator in existing programming
% languages is verbose and cumbersome, and requires code for every new
% kind of syntax. Yet, that code is essentially mechanical and
% ought to be automated. While using advanced meta-programming
% techniques enables automating the merge combinator to a large extent
% in existing programming languages~\cite{oliveira2013feature, rendel14attributes}, those techniques have
% several problems. For example, error messages can be problematic, some
% techniques rely on type-unsafe reflection, while other techniques
% require highly advanced type-level features.
% 
% This paper presents a calculus and polymorphic type system with
% \emph{(disjoint) intersection types}~\cite{oliveira2016disjoint}, called \fnamee, that
% supports our broader notion of compositional designs, and enables
% the development of highly modular and reusable programs. \fnamee
% has a built-in merge operator and a powerful subtyping relation that
% are used to automate the composition of multiple interpretations
% (including dependent interpretations). \fnamee continues a line of
% research on disjoint intersection types. Previous work on
% \emph{disjoint polymorphism} (the \fname calculus) studied the
% combination between parametric polymorphism and disjoint intersection
% types, but the subtyping relation did not support
% BCD-style distributivity rules~\cite{Barendregt_1983}. More recently the \name
% calculus (or \namee) studied a system with \emph{disjoint
%   intersection types} and BCD-style distributivity rules, but did not
% account for parametric polymorphism. \fnamee is unique in that it
% allows the combination of three useful features in a single calculus:
% \emph{disjoint intersection types} and a \emph{merge operator};
% \emph{parametric (disjoint) polymorphism}; and a BCD-style subtyping
% relation with \emph{distributivity rules}. All three features are
% necessary to use improved versions of finally tagless or object
% algebras that support improved compositional designs.
% 
% Unfortunatelly the combination of the three features has non-trivial
% complications. The main technical challenge (as often is the case for
% calculi with disjoint intersection types) is the proof of coherence
% for \fnamee. Because of the presence BCD-style distributivity
% rules, the proof of coherence is based on the approach using a
% \emph{heterogeneous} logical relation employed in
% \namee~\cite{bi_et_al:LIPIcs:2018:9227}. However the logical relation in
% \namee, which we call here \emph{canonicity}, does not
% account for polymorphism. To account for polymorphism we originally
% expected to simply borrow ideas from \emph{parametricity}~\cite{reynolds1983types} in
% System F~\cite{reynolds1974towards} and adapt them to fit with the canonicity relation.
% However, this did not work. The problem is partly due to the fact that
% canonicity (unlike parametricity) is an heterogenous relation and
% needs to account for heterogeneous cases that are not considered in an
% homogeneous relation such as parametricity. Those heterogeneous cases, combined
% with \emph{impredicative polymorphism}, resulted in an ill-founded logical
% relation. Fortunatelly it turns out that
% restricting the calculus to \emph{predicative polymorphism} and using
% an approach based on substitutions is
% sufficient to recover a well-founded canonicity relation.
% Therefore we
% adopted this approach in \fnamee.
% We do not view
% the predicativity restriction as being very severe in practice, since many
% practical languages have such restriction as well. For example languages based
% on Hindley-Milner style type systems (such as Haskell, OCaml or ML)
% \ningning{it's hard to say this is true. When we say Hindley-Milner type system,
%   or Haskell, we are referring to the source language. However, the core
%   language for, for example Haskell, which is System FC, is impredicative.
%   \fnamee is more close to a core language (which usually has explicit type
%   abstractions/applications). In this sense it's unfair to compare it with other
%   source languages.} all use predicative polymorphism. Furthermore with the
% predicativity restriction, the canonicity relation and corresponding proofs
% remain relatively simple and do not require emplying more complex approaches
% such as \emph{step-indexed logical relations}. \ningning{we should emphasize
%   that predicativity is not a restriction, rather it's choice we made in order
%   to prove coherence in Coq. Step-indexed logical relation might work for
%   impredicativity; it's just we don't know.}
% 
% In summary the contributions of this paper are:
% 
% \begin{itemize}
% 
% \item {\bf The \fnamee calculus,} which integrates disjoint intersection types,
%     distributivity and disjoint polymorphism. \fnamee
%     is the first calculus puts all three features together. The
%     combination is non-trivial, expecially with respect to the
%     coherence proof.
% 
% \bruno{improve text}
% \item {\bf The canonicity logical relation,} which enables the proof
%     of coherence of \fnamee. We show that the ideas of
%   System F's \emph{parametricity} cannot be ported to
%   \fnamee. To overcome the problem we develop a canonicity
%   relation that enables a proof of coherence.
% 
% \item {\bf Disjoint intersection types in the presence of bottom:}
%   Our calculus includes a bottom type, which was considered to be a
% significant challenge in previous work.
% 
% \item {\bf Improved compositional designs:} We show how \fnamee has all the
% features that enable improved
% compositional programming designs and support automated composition
% of interpretations in programming techniques like object algebras and
% finally tagless.
% 
% \item {\bf Implementation and proofs:} All proofs
% (including type-safety, coherence and decidability of the type system)
% are proved in the Coq theorem prover. Furthermore \sedel \ningning{where comes the name \sedel?} and
% \fnamee are implemented and all code presented in the paper is
% available. The implementation, proofs and examples can be found in:
% 
% \url{MISSING}
% 
% \end{itemize}
% 
% \bruno{
% Still need to figure out how to integrate row types in the intro story
% Furthermore, we provide a detailed
% comparison between \emph{distributive disjoint polymorphism} and
% \emph{row types}.
% }

% Local Variables:
% org-ref-default-bibliography: "../paper.bib"
% End:



\section{Overview}
\label{sec:overview}

This section aims at introducing first-class classes and traits, their possible
uses and applications, as well as the typing challenges that arise
from their use.
We start by describing a hypothetical JavaScript library for text editing
widgets, inspired and adapted from Racket's GUI
toolkit~\cite{DBLP:conf/oopsla/TakikawaSDTF12}. The example is illustrative of
typical uses of dynamic inheritance/composition, and also the typing challenges
in the presence of first-class classes/traits. Without diving into
technical details, we then give the corresponding typed version in
\name, and informally presents its salient features.

\subsection{First-Class Classes in JavaScript}

A class construct was officially added to JavaScript in the ECMAScript
2015 Language Specification~\cite{EcmaScript:15}. One purpose of
adding classes to JavaScript was to support a construct that is more
familiar to programmers who come from mainstream class-based languages,
such as Java or C++. However classes in JavaScript are
\emph{first-class} and support functionality not easily mimicked in
statically-typed class-based languages.

\subparagraph{Conventional Classes.}
Before diving into the more advanced features of JavaScript classes, we first
review the more conventional class declarations supported in JavaScript as well
as many other languages. Even for conventional classes there are some
interesting points to note about JavaScript that will be important when we move
into a typed setting. An example of a JavaScript class declaration is:
\begin{lstlisting}[language=JavaScript]
class Editor {
  onKey(key) { return "Pressing " + key; }
  doCut()    { return this.onKey("C-x") + " for cutting text"; }
  showHelp() { return "Version: " + this.version() + " Basic usage..."; }
};
\end{lstlisting}
This form of class definition is standard and very similar to declarations in
class-based languages (for example Java). The \lstinline{Editor} class
defines three methods: \lstinline{onKey} for handling key events,
\lstinline{doCut} for cutting text and \lstinline{showHelp} for displaying help
message. For the purpose of demonstration, we elide the actual implementation,
and replace it with plain messages.

We wish to bring the readers' attention to two points in the above class.
Firstly, note that the \lstinline{doCut} method is defined in terms of the
\lstinline{onKey} method via the keyword
\lstinline[language=JavaScript]{this}. In other words the call to
\lstinline{onKey} is enabled by the \emph{self} reference and is
\emph{dynamically dispatched} (i.e., the particular implementation of
\lstinline{onKey} will only be determined when the class or subclass
is instantiated). % Typically an
% OO programmer seeing this definition would expect the \lstinline{doCut} method
% to call the \lstinline{onKey} method of a subclass of \lstinline{Editor}, even though
% the subclass does not exist when the superclass \lstinline{Editor} is being
% defined.
Secondly, notice that there is no definition of
the \lstinline{version} method in the class body, but such method is used inside the
\lstinline{showHelp} method. In a untyped language, such as JavaScript, using
undefined methods is error prone -- accidentally instantiating \lstinline{Editor}
and then calling \lstinline{showHelp} will cause a runtime error!
Statically-typed languages usually provide some means to protect us from this
situation. For example, in Java, we would need an \textit{abstract} \lstinline{version}
method, which effectively makes \lstinline{Editor} an abstract class and
prevents it from being instantiated. As we will see, \name's treatment of
abstract methods is quite different from mainstream languages. In fact, \name
has a unified (typing) mechanism for dealing with both dynamic dispatch and abstract
methods. We will describe \name's mechanism for dealing with both features and
justify our design in \cref{sec:traits}.

% A couple of things worth pointing out in the above code snippet: (1) the class
% \lstinline{Editor} has no definition of the method
% \lstinline{version}, but such method
% is used in the body of the method \lstinline{showHelp}. In a strongly-typed OO
% language, such as Java, we would need to define an abstract method for
% \lstinline{version}. (2) The \lstinline{Editor} class requires
% \emph{dynamic dispatching}.
%  In the body of the method \lstinline{doCut} we invoke
% the method \lstinline{onKey} defined in the same class through the keyword
% \lstinline[language=JavaScript]{this}. This has the implication that when a
% subclass of \lstinline{Editor} overrides the method \lstinline{onKey}, a call to
% \lstinline{doCut} should invoke \lstinline{onKey} defined in the subclass
% instead of the original one.\bruno{punchline?}
%As we will see later, the type system of \name correctly handles it.

\subparagraph{First-Class Classes and Class Expressions.}
Another way to define a class in JavaScript is via a \emph{class expression}. This is where the class
model in JavaScript is very different from the traditional class model found in
many mainstream OO languages, such as Java, where classes are second-class
(static) entities. JavaScript embraces a dynamic class model that treats classes
as \emph{first-class} expressions: a function can take classes as arguments,
or return them as a result. First-class classes enable programmers to
abstract over patterns in the class hierarchy and to experiment with new forms of OOP
such as mixins and traits. In particular, mixins become programmer-defined
constructs. We illustrate this by presenting a simple mixin that adds
spell checking to an editor:
\begin{lstlisting}[language=JavaScript]
const spellMixin = Base => {
  return class extends Base {
    check()    { return super.onKey("C-c") + " for spell checking"; }
    onKey(key) { return "Process " + key + " on spell editor"; }
  }
};
\end{lstlisting}
In JavaScript, a mixin is simply a function with a superclass as input and a
subclass extending that superclass as an output. Concretely, \lstinline{spellMixin}
adds a method \lstinline{check} for spell checking. It also provides
a method \lstinline{onKey}.
The function \lstinline{spellMixin} shows the typical use of what we call \emph{dynamic inheritance}.
Note that \lstinline{Base}, which is supposed to be a superclass being inherited, is \emph{parameterized}.
Therefore \lstinline{spellMixin} can be applied to any base class at
\emph{runtime}. This is impossible to do, in a type-safe way, in
conventional statically-typed class-based languages like Java or
C++.\footnote{With C++ templates, it is possible to
  implement a so-called mixin pattern~\cite{DBLP:conf/gcse/SmaragdakisB00}, which enables extending
a parameterized class. However C++ templates defer type-checking until
instantiation, and such pattern still does not allow selection of the
base class at runtime (only at up to class instantiation time).}

It is noteworthy that not all applications of \lstinline{spellMixin} to base
classes are successful. Notice the use of the \lstinline{super} keyword in the
\lstinline{check} method. If the base class does not implement the
\lstinline{onKey} method, then mixin application fails with a runtime error. In
a typed setting, a type system must express this requirement (i.e., the presence of
the \lstinline{onKey} method) on the (statically unknown) base class that is
being inherited.


% The class expression inside the function body has no
% definition of the method \lstinline{version}, but which is used in the body of
% the method \lstinline{showHelp}. In a statically-typed OO language, such as Java,
% we would need an \emph{abstract method} for
% \lstinline{version}.


We invite the readers to pause for a while and think about what the type of
\lstinline{spellMixin} would look like. Clearly our type system should be
flexible enough to express this kind of dynamic pattern of composition in order
to accommodate mixins (or traits), but also not too lenient to allow any
composition.


\subparagraph{Mixin Composition and Conflicts.}
The real power of mixins is that \lstinline{spellMixin}'s functionality is not
tied to a particular class hierarchy and is composable with other features. For
example, we can define another mixin that adds simple modal editing -- as in Vim
-- to an arbitrary editor:
\begin{lstlisting}[language=JavaScript]
const modalMixin = Base => {
  return class extends Base {
    constructor() {
      super();
      this.mode = "command";
    }
    toggleMode() { return "toggle succeeded"; }
    onKey(key)   { return "Process " + key + " on modal editor"; }
  };
};
\end{lstlisting}
\lstinline{modalMixin} adds a \lstinline{mode} field that controls which
keybindings are active, initially set to the command mode, and a method
\lstinline{toggleMode} that is used to switch between modes. It also provides a method \lstinline{onKey}.

Now we can compose \lstinline{spellMixin} with \lstinline{modalMixin} to produce
a combination of functionality, mimicking some form of multiple inheritance:
\begin{lstlisting}[language=JavaScript]
class IDEEditor extends modalMixin(spellMixin(Editor)) {
  version() { return 0.2; }
}
\end{lstlisting}
The class \lstinline{IDEEditor} extends the base class \lstinline{Editor} with
modal editing and spell checking capabilities. It also defines the missing
\lstinline{version} method.

At first glance, \lstinline{IDEEditor} looks quite fine, but it has a subtle
issue. Recall that two mixins \lstinline{modalMixin} and \lstinline{spellMixin}
both provide a method \lstinline{onKey}, and the \lstinline{Editor} class also
defines an \lstinline{onKey} method of its own. Now we have a name clash. A
question arises as to which one gets picked inside the \lstinline{IDEEditor}
class. A typical mixin model resolves this issue by looking at the order of mixin applications. Mixins appearing later in the order
overrides \emph{all} the identically named methods of earlier mixins. So in our
case, \lstinline{onKey} in \lstinline{modalMixin} gets picked. If we
change the order of application to \lstinline{spellMixin(modalMixin(Editor))},
then \lstinline{onKey} in \lstinline{spellMixin} is inherited.

\subparagraph{Problem of Mixin Composition.}
From the above discussion, we can see that mixin are composed linearly: all the
mixins used by a class must be applied one at a time. However, when we wish to
resolve conflicts by selecting features from different mixins, we may not be
able to find a suitable order. For example, when we compose the two mixins to
make the class \lstinline{IDEEditor}, we can choose which of them comes first,
but in either order, \lstinline{IDEEditor} cannot access to the \lstinline{onKey}
method in the \lstinline{Editor} class.

\subparagraph{Trait Model.}
Because of the total ordering and the limited means for resolving conflicts imposed by the mixin model,
researchers have proposed a simple compositional model called
traits~\cite{scharli2003traits, Ducasse_2006}. Traits are lightweight entities and serve as
the primitive units of code reuse. Among others, the key difference from
mixins is that the order of trait composition is irrelevant, and conflicting
methods must be resolved \emph{explicitly}. This gives programmers
fine-grained control, when conflicts arise, of selecting desired features from
different components. Thus we believe traits are a better model for multiple
inheritance in statically-typed OO languages, and in \name we realize this
vision by giving traits a first-class status in the language,
achieving more expressive power compared with traditional (second-class) traits.


\subparagraph{Summary of Typing Challenges.}
From our previous discussion, we can identify the following typing challenges
for a type system to accommodate the programming patterns (first-class classes/mixins)
we have just seen in a typed setting:
\begin{itemize}
\item How to account for, in a typed way, abstract methods and dynamic dispatch.
\item What are the types of first-class classes or mixins.
\item How to type dynamic inheritance.
\item How to express constraints on method presence and absence (the use of
  \lstinline{super} clearly demands that).
% \item How to ensure that composition of mixins is going to be valid, i.e., how
%   to reflect linearity in a type system.
\item In the presence of first-class traits, how to detect conflicts statically,
  even when the traits involved are not statically known.
\end{itemize}
\name elegantly solves the above challenges in a unified way, as
we will see next.


% From a pragmatic point of view, this implicit conflict resolution
% sometimes give programmers more surprises than convenience. What if the compiler can alarm us when a
% potential conflict may occur. Because of the dynamic nature of JavaScript, we
% would not know before actually running the code that there is a conflict. We
% miss the guarantee that a static type system can provide: such conflict can be
% detected at compile-time.

% Given the flexibility of first-class classes in dynamically-typed languages, we
% -- being advocates of statically-typed languages -- were wondering how to
% incorporate this same expressive power into statically-typed
% languages. As it
% turns out, designing a sound type system that fully supports first-class classes
% is notoriously hard; there are only a few, quite sophisticated, languages that
% manage this~\cite{DBLP:conf/oopsla/TakikawaSDTF12, DBLP:conf/ecoop/LeeASP15}. We
% pushed it further: \name has support for typed first-class
% traits.\bruno{Better to say there's no work on typed first-class
%   traits, and little work on first-class classes/mixins, despite
%  many dynamic languages prominently supporting such features.}

\subsection{A Glance at Typed First-Class Traits in \name}

We now rewrite the above library in \name, but this time with types. The resulting code has the same functionality as the dynamic version, but is
statically typed. All code snippets in this and later sections are runnable in
our prototype implementation. Before proceeding, we ask the readers to bear in mind that in this section we are not using traits
in the most canonical way, i.e., we use traits as if they are classes (but with
built-in conflict detection). This is because we are trying to stay as close as possible
to the structure of the JavaScript version for ease of comparison. In
\cref{sec:traits} we will remedy this to make better use of traits.

\subparagraph{Simple Traits.}
Below is a simple trait \lstinline{editor}, which corresponds to the JavaScript
class \lstinline{Editor}. The \lstinline{editor} trait defines the same set of
methods: \lstinline{on_key}, \lstinline{do_cut} and \lstinline{show_help}:
\lstinputlisting[linerange=14-18]{../../examples/overview2.sl}% APPLY:linerange=OVERVIEW_EDITOR
The first thing to notice is that \name uses a syntax (similar to Scala's
self type annotations~\cite{odersky2004overview}) where we can give a type annotation to the
\lstinline{self} reference. In the type of \lstinline{self} we use
\lstinline{&} construct to create intersection types. \lstinline{Editor} and \lstinline{Version} are two record types:
\lstinputlisting[linerange=7-8]{../../examples/overview2.sl}% APPLY:linerange=OVERVIEW_EDITOR_TYPES
For the sake of conciseness, \name uses \lstinline{type} aliases to abbreviate types.

\subparagraph{Self-Types Encode Abstract Methods.}
Recall that in the JavaScript class \lstinline{Editor}, the \lstinline{version}
method is undefined, but is used inside \lstinline{showHelp}. How can we express
this in the typed setting, if not with an abstract method? In \name, self-types
play the role of trait requirements. As the first approximation, we
can justify the use of \lstinline{self.version} by noticing that (part of) the
type of \lstinline{self} (i.e., \lstinline{Version}) contains the declaration of
\lstinline{version}. An interesting aspect of \name's trait model is that there
is no need for abstract methods. Instead, abstract methods can be simulated as
requirements of a trait. Later, when the trait is composed with other
traits, \emph{all} requirements on the self-types must be
satisfied and one of the traits in the composition must provide an
implementation of the method \lstinline{version}.
%to this point in \cref{sec:traits}.

As in the JavaScript version, the \lstinline{on_key} method is invoked on
\lstinline{self} in the body of \lstinline{do_cut}. This is allowed as (part of)
the type of \lstinline{self} (i.e., \lstinline{Editor}) contains the signature
of \lstinline{on_key}. Comparing \lstinline{editor} to the JavaScript class
\lstinline{Editor}, almost everything stays the same, except that we now have
the typed version. As a side note, since \name is currently a pure functional OO
language, there is no difference between fields and methods, so we can omit
empty arguments and parameter parentheses.

\subparagraph{First-Class Traits and Trait Expressions.}

\name treats traits as first-class expressions, putting them in the same
syntactic category as objects, functions, and other primitive forms. To
illustrate this, we give the \name version of \lstinline{spellMixin}:
\lstinputlisting[linerange=22-29]{../../examples/overview2.sl}% APPLY:linerange=OVERVIEW_HELP
This looks daunting at first, but \lstinline{spell_mixin} has almost the same structure as
its JavaScript cousin \lstinline{spellMixin}, albeit with
some type annotations. In \name, we use capital letters (\lstinline{A}, \lstinline{B}, $\dots$) to denote type variables, and trait
expressions \lstinline$trait [self : ...] inherits ... => {...}$ to create
first-class traits. Trait expressions have trait
types of the form \lstinline{Trait[T1, T2]} where \lstinline{T1} and \lstinline{T2} denote trait requirements and functionality respectively.
We will explain trait types in \cref{sec:traits}. Despite the structural similarities, there are several significant
features that are unique to \name (e.g., the disjointness operator \lstinline{*}).
We discuss these in the following.



\subparagraph{Disjoint Polymorphism and Conflict Detection.}

\name uses a type system based on \emph{disjoint intersection types}~\cite{oliveira2016disjoint} and
\emph{disjoint polymorphism}~\cite{alpuimdisjoint}. Disjoint intersections
empower \name to detect conflicts statically when trying to compose two
traits with identically named features. For example composing two traits
\lstinline{a} and \lstinline{b} that both provide \lstinline{foo} gives a
type error (the overloaded \lstinline{&} operator denotes trait composition):
\begin{lstlisting}
trait a => { foo = 1 };
trait b => { foo = 2 };
trait c inherits a & b => {}; -- type error!
\end{lstlisting}
Disjoint polymorphism, as a more advanced mechanism, allows detecting conflicts
even in the presence of polymorphism -- for example when a trait is parameterized and its
full set of methods is not statically known. As can be seen,
\lstinline{spell_mixin} is actually a polymorphic function. Unlike ordinary
parametric polymorphism, in \name, a type variable can also have a disjointness
constraint. For instance, \lstinline{A * Spelling & OnKey}
means that \lstinline{A} can be instantiated to any type as long as it \emph{does not}
contain \lstinline{check} and \lstinline{on_key}. To mimic mixins, the
argument \lstinline{base}, which is supposed to be some trait, serves as the
``base'' trait that is being inherited. Notice that the type variable
\lstinline{A} appears in the type of \lstinline{base}, which essentially states
that \lstinline{base} is a trait that contains at least those methods specified
by \lstinline{Editor}, and possibly more (which we do not know statically).
% In summary, \lstinline{Trait[Editor & Version, Editor & A]} (the assigned type
% of \lstinline{base}) specifies that both method \emph{presence} and \emph{absence}.
Also note that leaving out the \lstinline{override} keyword will result in a
type error. The type system is forcing us to be very specific as to what is the
intention of the \lstinline{on_key} method because it sees the same method is
also declared in \lstinline{base}, and blindly inheriting \lstinline{base}
will definitely cause a method conflict. As a final note, the use of \lstinline{super}
inside \lstinline{check} is allowed because the ``super'' trait \lstinline{base}
implements \lstinline{on_key}, as can be seen from its type.


\subparagraph{Dynamic Inheritance.}

Disjoint polymorphism enables us to correctly type dynamic inheritance:
\lstinline{spell_mixin} is able to take any trait that conforms with its
assigned type, equips it with the \lstinline{check} method and overrides its
old \lstinline{on_key} method. As a side note, the use of disjoint polymorphism
is essential to correctly model the mixin semantics. From the type we know
\lstinline{base} has some features specified by \lstinline{Editor}, plus
something more denoted by \lstinline{A}. By inheriting \lstinline{base}, we are
guaranteed that the result trait will have everything that is already contained
in \lstinline{base}, plus more features. This is in some sense similar to row
polymorphism~\cite{wand1994type} in that the result trait is prohibited from
forgetting methods from the argument trait. As we will discuss in
\cref{sec:related}, disjoint polymorphism is more expressive than row
polymorphism.


\subparagraph{Typing Mixin Composition.}
Next we give the typed version of \lstinline{modalMixin} as follows:
\lstinputlisting[linerange=34-41]{../../examples/overview2.sl}% APPLY:linerange=OVERVIEW_MODAL
Now the definition of \lstinline{modal_mixin} should be self-explanatory.
Finally we can apply both ``mixins'' one by one to \lstinline{editor} to create
a concrete editor:
\lstinputlisting[linerange=46-49]{../../examples/overview2.sl}% APPLY:linerange=OVERVIEW_LINE
As with the JavaScript version, we need to fill in the missing
\lstinline{version} method. It is easy to verify that the \lstinline{on_key} method
in \lstinline{modal_mixin} is inherited. Compared with the untyped version,
here this behaviour is reasonable because in each mixin we specifically tags the
\lstinline{on_key} method to be an overriding method. Let us take a close look
at the mixin applications. Since \name is currently explicitly typed, we need to
provide concrete types when using \lstinline{modal_mixin} and \lstinline{spell_mixin}.
In the inner application (\lstinline{spell_mixin Top editor}), we use the top
type \lstinline{Top} to instantiate \lstinline{A} because the \lstinline{editor} trait
provides exactly those method specified by \lstinline{Editor} and nothing more
(hence \lstinline{Top}). In the outer application, we use \lstinline{Spelling}
to instantiate \lstinline{A}. This is where implicit conflict resolution of
mixins happens. We know the result of the inner application actually forms a
trait that provides both \lstinline{check} and \lstinline{on_key}, but the
disjointness constraint of \lstinline{A} requires the absence of \lstinline{on_key},
thus we cannot instantiate \lstinline{A} to \lstinline{Spelling & OnKey} for example
when applying \lstinline{modal_mixin}. Therefore the outer application effectively excludes
\lstinline{on_key} from \lstinline{spell_mixin}.
In summary, the order of mixin applications is reflected by the order
of function applications, and conflict resolution code is implicitly embedded.
Of course changing the mixin application order to \lstinline{spell_mixin ModalEdit (modal_mixin Top editor)} gives the expected behaviour.


Admittedly the typed version is unnecessarily complicated as we were
mimicking mixins by functions over traits. The final editor
\lstinline{ide_editor} suffers from the same problem as the class
\lstinline{IDEEditor}, since there is no obvious way to access the
\lstinline{on_key} method in the \lstinline{editor} trait.\footnote{In fact, as
  we will see in \cref{sec:traits}, we can still access \lstinline{on_key} in
  \lstinline{editor} by the forwarding operator.} \cref{sec:traits}
makes better use of traits to simplify the editor code.



% Note that the use of \lstinline{override} is valid because the type system knows the inherit clause contains \lstinline{on_key}.
% As a bonus, since \name guarantees that there are no potential conflicts in a program,
% we can reason that the version number in \lstinline{modal_editor} is
% \lstinline{0.1}.

%%% Local Variables:
%%% mode: latex
%%% TeX-master: "../paper"
%%% org-ref-default-bibliography: ../paper.bib
%%% End:


\input{sections/examples.tex}

\input{sections/explicitcore.tex}

\input{sections/recursion.tex}

\input{sections/datatypes.tex}


\section{Related Work}
\label{sec:related}

Along the way we discussed some of the most relevant work to motivate,
compare and
promote our gradual typing design. In what follows, we briefly discuss related
work on gradual typing and polymorphism.


\paragraph{Gradual Typing}

The seminal paper by \citet{siek2006gradual} is the first to propose gradual
typing, which enables programmers to mix static and dynamic typing in a program
by providing a mechanism to control which parts of a program are statically
checked. The original proposal extends the simply typed lambda calculus by
introducing the unknown type $\unknown$ and replacing type equality with type
consistency. Casts are introduced to mediate between statically and dynamically
typed code. Later \citet{siek2007gradual} incorporated gradual typing into a
simple object oriented language, and showed that subtyping and consistency are
orthogonal -- an insight that partly inspired our work. We show that subtyping
and consistency are orthogonal in a much richer type system with higher-rank
polymorphism. \citet{siek2009exploring} explores the design space of different
dynamic semantics for simply typed lambda calculus with casts and unknown types.
In the light of the ever-growing popularity of gradual typing, and its somewhat
murky theoretical foundations, \citet{siek2015refined} felt the urge to have a
complete formal characterization of what it means to be gradually typed. They
proposed a set of criteria that provides important guidelines for designers of
gradually typed languages. \citet{cimini2016gradualizer} introduced the
\emph{Gradualizer}, a general methodology for generating gradual type systems
from static type systems. Later they also develop an algorithm to generate
dynamic semantics~\cite{CiminiPOPL}. \citet{garcia2016abstracting} introduced
the AGT approach based on abstract interpretation. As we discussed, none of
these approaches instructed us how to define consistent subtyping for
polymorphic types.

There is some work on integrating gradual typing with rich type disciplines.
\citet{Ba_ados_Schwerter_2014} establish a framework to combine gradual typing and
effects, with which a static effect system can be transformed to a dynamic
effect system or any intermediate blend. \citet{Jafery:2017:SUR:3093333.3009865}
present a type system with \emph{gradual sums}, which combines refinement and
imprecision. We have discussed the interesting definition of \emph{directed
  consistency} in Section~\ref{sec:exploration}. \citet{castagna2017gradual} develop a gradual type system with
intersection and union types, with consistent subtyping defined by following
the idea of \citet{garcia2016abstracting}.
TypeScript~\citep{typescript} has a distinguished dynamic type, written {\color{blue} any}, whose fundamental feature is that any type can be
implicitly converted to and from {\color{blue} any}.
% They prove that the conversion
% definition (called \emph{assignment compatibility}) coincides with the
% restriction operator from \citet{siek2007gradual}.
Our treatment of the unknown type in \cref{fig:decl:conssub} is similar to their
treatment of {\color{blue} any}. However, their type system does not have
polymorphic types. Also, Unlike our consistent subtyping which inserts runtime
casts, in TypeScript, type information is erased after compilation so there are
no runtime casts, which makes runtime type errors possible.
% dynamic checks does not contribute to type safety.


\paragraph{Gradual Type Systems with Explicit Polymorphism}

\citet{Morris:1973:TS:512927.512938} dynamically enforces
parametric polymorphism and uses \emph{sealing} functions as the
dynamic type mechanism. More recent works on integrating gradual typing with
parametric polymorphism include the dynamic type of \citet{abadi1995dynamic} and
the \emph{Sage} language of \citet{gronski2006sage}. None of these has carefully
studied the interaction between statically and dynamically typed code.
\citet{ahmed2011blame} proposed \pbc that extends the blame
calculus~\cite{Wadler_2009} to incorporate polymorphism. The key novelty of
their work is to use dynamic sealing to enforce parametricity. As such, they end
up with a sophisticated dynamic semantics. Later, \citet{amal2017blame} prove
that with more restrictions, \pbc satisfies parametricity. Compared to their
work, our type system can catch more errors earlier since, as we argued, 
their notion of \emph{compatibility} is too permissive. For example, the
following is rejected (more precisely, the corresponding source program never
gets elaborated) by our type system:
\[
  (\blam x \unknown x + 1) : \forall a. a \to a \rightsquigarrow \cast {\unknown \to \nat}
  {\forall a. a \to a} (\blam x \unknown x + 1)
\]
while the type system of \pbc would accept the translation, though at runtime,
the program would result in a cast error as it violates parametricity.
% This does not imply, in any regard that \pbc is not well-designed; there are
% circumstances where runtime checks are \emph{needed} to ensure
% parametricity.
We emphasize that it is the combination of our powerful type system together
with the powerful dynamic semantics of \pbc that makes it possible to have
implicit higher-rank polymorphism in a gradually typed setting.
% without compromising parametricity.
\citet{devriese2017parametricity} proved that
embedding of System F terms into \pbc is not fully abstract. \citet{yuu2017poly}
also studied integrating gradual typing with parametric polymorphism. They
proposed System F$_G$, a gradually typed extension of System F, and System
F$_C$, a new polymorphic blame calculus. As has been discussed extensively,
their definition of type consistency does not apply to our setting (implicit
polymorphism). All of these approaches mix consistency with subtyping to some
extent, which we argue should be orthogonal. On a side note, it seems that our
calculus can also be safely translated to System F$_C$. However we do not
understand all the tradeoffs involved in the choice between \pbc and System
F$_C$ as a target.



\paragraph{Gradual Type Inference}
\citet{siek2008gradual} studied unification-based type inference for gradual
typing, where they show why three straightforward approaches fail to meet their
design goals. One of their main observations is
that simply ignoring dynamic types during unification does not work. Therefore,
their type system assigns unknown types to type variables and infers gradual
types, which results in a complicated type system and inference algorithm. In
our algorithm presented in \cref{sec:advanced-extension}, comparisons between
existential variables and unknown types are emphasized by the distinction
between static existential variables and gradual existential variables. By
syntactically refining unsolved gradual existential variables with unknown types, we gain a
similar effect as assigning unknown types, while keeping the algorithm relatively
simple.
\citet{garcia2015principal} presented a new approach where gradual type
inference only produces static types, which is adopted in our type system. They
also deal with let-polymorphism (rank 1 types). They proposed the distinction
between static and gradual type parameters, which inspired our extension to
restore the dynamic gradual guarantee. Although those existing works all involve
gradual types and inference, none of these works deal with higher-rank
implicit polymorphism.


\paragraph{Higher-rank Implicit Polymorphism}

\citet{odersky1996putting} introduced a type system for higher-rank implicit
polymorphic types. Based on that, \citet{jones2007practical} developed an
approach for type checking higher-rank predicative polymorphism.
\citet{dunfield2013complete} proposed a bidirectional account of higher-rank
polymorphism, and an algorithm for implementing the declarative system, which
serves as the main inspiration for our algorithmic system. The key difference,
however, is the integration of gradual typing.
% \citet{vytiniotis2012defer}
% defers static type errors to runtime, which is fundamentally different from
% gradual typing, where programmers can control over static or runtime checks by
% precision of the annotations.
As our work, those works are in a
\emph{predicative} setting, since complete type inference for higher-rank
types in an impredicative setting is undecidable. Still, there are many type
systems trying to infer some impredicative types, such as
\texttt{$ML^F$}~\citep{le2014mlf,remy2008graphic,le2009recasting}, the HML
system~\citep{leijen2009flexible}, the FPH system~\citep{vytiniotis2008fph} and
so on. Those type systems usually end up with non-standard System F types, and
sophisticated forms of type inference.

%%% Local Variables:
%%% mode: latex
%%% TeX-master: "../paper"
%%% org-ref-default-bibliography: "../paper.bib"
%%% End:



\section{Conclusion and Future Work}
\label{sec:conclusion}

We have proposed \fnamee, a type-safe and coherent calculus with disjoint
intersection types, BCD subtyping and parametric polymorphism. \fnamee improves
the state-of-art of compositional designs, and enables the development of highly
modular and reusable programs. One interesting and useful further extension
would be implicit polymorphism. For that we want to combine
Dunfield and Krishnaswami's approach~\cite{dunfield2013complete} with our bidirectional type system.
We would also like to study the parametricity of \fnamee. As we have seen in
\cref{sec:failed:lr}, it is not at all obvious how to extend the standard
logical relation of System F to account for disjointness, and avoid potential
circularity due to impredicativity. A promising solution is to use step-indexed
logical relations~\cite{ahmed2006step}. 
% TOM: This sentence is broken. Do we even need it?
% We have yet investigated further on that direction.


\section*{Acknowledgments}

We thank the anonymous reviewers and Yaoda Zhou for their helpful comments.
This work has been sponsored by the Hong Kong Research Grant
Council projects number 17210617 and 17258816, and by the Research Foundation -
Flanders.



%%% Local Variables:
%%% mode: latex
%%% TeX-master: "../paper"
%%% org-ref-default-bibliography: "../paper.bib"
%%% End:


%% -- References --

% \acks
% Thanks to Blah. This work is supported by Blah.

\newpage

\bibliographystyle{abbrvnat}
% \nocite{*}
\bibliography{main}

%% -- Appendix --
\clearpage
\appendix

\section{Some Proofs about the Declarative System}


\lemmaerase*
\begin{proof}
By straightforward induction on the typing derivation.
\end{proof}

\lemmacoherence*
\begin{proof}
According to Lemma~\ref{lemma:erase}, after erasure of types and casts,
$\mathcal{C}\{[[pe1]]\}$ and $\mathcal{C}\{[[pe2]]\}$ are equivalent. So if
$\mathcal{C}\{ [[pe1]] \} \reduce [[n]]$, it is impossible for $\mathcal{C}\{
[[pe2]] \}$ to reduce to a different integer according to the dynamic semantics.
\end{proof}

\proptop*
\begin{proof} \leavevmode
  \begin{itemize}
  \item From first to second: By induction on the derivation of consistent
    subtyping. We have extra case \rref{CS-Top} now, where $B = \top$.
    We can choose $C = A$, and
    $D$ by replacing the unknown types in $C$ by $\nat$. Namely, $D$ is a static
    type, so by \rref{S-Top} we are done.
  \item From second to first: By induction on the derivation of second
    subtyping. We have extra case \rref{S-Top} now, where
    $B = \top$, so $A \tconssub B$ holds by \rref{CS-Top}.
  \end{itemize}
\end{proof}

\propagttop*
\begin{proof} \leavevmode
  \begin{itemize}
  \item From left to right: By induction on the derivation of consistent
    subtyping. We have case \rref{CS-Top} now.
    It follows that for
    every static type $A_1 \in \gamma(A)$, we can derive $A_1 \tsub \top$ by
    \rref{S-Top}.
    We have $B_1 = B = \top$ and we are done.
  \item From right to left: By induction on the derivation of subtyping and
    inversion on the concretization. We have extra case \rref{S-Top} now, where
    $B$ is $\top$. So
    consistent subtyping directly holds.
  \end{itemize}
\end{proof}

\propparalpha*
\begin{proof}
Follows directly from the definition of Translation Pre-order.
\end{proof}

\begin{definition}[Measurements of Translation]
  There are three measurements of a translation $[[pe]]$,
  \begin{enumerate}
  \item $[[ ||pe||e]]$, the size of the expression 
  \item $[[ ||pe||s ]]$, the number of distinct static type parameters in $[[pe]]$
  \item $[[ ||pe||g ]]$, the number of distinct gradual type parameters in $[[pe]]$
  \end{enumerate}
  We use $[[ ||pe|| ]]$ to denote the lexicographical order of the triple
  $([[ ||pe||e ]], -[[ ||pe||s ]], -[[ ||pe||g ]])$.
\end{definition}

\begin{definition}[Size of types]

  \begin{align*}
    [[ || int ||  ]] &= 1 \\
    [[ || a ||  ]] &= 1 \\
    [[ || A -> B  ||  ]] &= [[ || A || ]] + [[ || B || ]] + 1 \\
    [[ || \/a . A ||  ]] &= [[ || A || ]] + 1 \\
    [[ || unknown ||  ]] &= 1 \\
    [[ || static ||  ]] &= 1 \\
    [[ || gradual ||  ]] &= 1
  \end{align*}

\end{definition}

\begin{definition}[Size of expressions]

  \begin{align*}
    [[ || x ||e  ]] &= 1 \\
    [[ || n ||e  ]] &= 1 \\
    [[ || \x : A . pe ||e  ]] &= [[ || A || ]] + [[ || pe ||e ]] + 1 \\
    [[ || /\ a. pe ||e  ]] &= [[ || pe ||e ]] + 1 \\
    [[ || pe1 pe2 ||e  ]] &= [[ || pe1 ||e ]] + [[  || pe2 ||e ]] + 1 \\
    [[ || < A `-> B> pe ||e  ]] &= [[ || pe ||e ]] + [[  || A || ]] + [[  || B || ]] + 1 \\
  \end{align*}

\end{definition}


\begin{lemma} \label{lemma:size_e}
  If $[[dd |- e : A ~~> pe]]$ then $[[ || pe ||e    ]] \geq [[ || e ||e   ]]  $.
\end{lemma}
\begin{proof}
  Immediate by inspecting each typing rule.
\end{proof}

\begin{corollary} \label{lemma:decrease_stop}
  If $[[dd |- e : A ~~> pe]]$ then $[[ || pe ||   ]] > ([[ || e ||e ]], -[[ || e ||e ]], -[[ || e ||e ]] )  $.
\end{corollary}
\begin{proof}
  By \cref{lemma:size_e} and note that $ [[ || pe ||e   ]] > [[  || pe ||s  ]] $ and $ [[ || pe ||e   ]] > [[  || pe ||g  ]] $
\end{proof}

\begin{lemma} \label{lemma:type_decrease}
  $[[ || A || ]] \leq [[ || S(A) || ]]  $.
\end{lemma}
\begin{proof}
  By induction on the structure of $[[A]]$. The interesting cases are $[[ A ]] = [[static]]$ and
  $[[ A ]] = [[gradual]]$. When $[[ A ]] = [[static]]$, $[[ S(A) ]] = [[t]]$
  for some monotype $[[t]]$ and it is immediate that $[[ || static ||  ]]  \leq [[ || t || ]] $
  (note that $[[ || static ||  ]] < [[ || gradual ||  ]] $ by definition).
\end{proof}

\begin{lemma}[Substitution Decreases Measurement]
  \label{lemma:subst_dec_measure}
  If $[[pe1]] \leq [[pe2]]$, then $ {[[ ||pe1|| ]]} \leq [[ ||pe2|| ]]$; unless
  $[[pe2]] \leq [[pe1]]$ also holds, otherwise we have $[[ ||pe1|| ]] < [[ ||pe2|| ]]$.
\end{lemma}
\begin{proof}
  Since $[[ pe1  ]] \leq [[  pe2  ]]$, we know $[[ pe2  ]] = [[ S(pe1)  ]]$ for some $[[S]]$. By induction on
  the structure of $[[pe1]]$.

  \begin{itemize}
  \item Case $[[pe1]] = [[  \x : A . pe ]]$. We have
    $[[ pe2  ]] = [[  \x : S(A) . S(pe)  ]]$. By \cref{lemma:type_decrease} we have $[[ || A || ]] \leq [[ || S(A) || ]]$.
    By i.h., we have $[[ || pe ||  ]] \leq [[ || S(pe) ||  ]]$. Therefore $[[ || \x : A . pe ||    ]] \leq [[ || \x : S(A) . S(pe) ||  ]]$.
  \item Case $[[pe1]] = [[ < A `-> B > pe  ]]$. We have
    $[[pe2]] = [[ < S(A) `-> S(B) > S(pe)  ]]$.  By \cref{lemma:type_decrease} we have $[[ || A || ]] \leq [[ || S(A) || ]]$
    and $[[ || B || ]] \leq [[ || S(B) || ]]$. By i.h., we have $[[ || pe ||  ]] \leq [[ || S(pe) ||  ]]$.
    Therefore $[[  || < A `-> B > pe ||  ]] \leq [[ || < S(A) `-> S(B) > S(pe)  ||   ]]$.

  \item The rest of cases are immediate.
  \end{itemize}
\end{proof}


\lemmareptyping*
\begin{proof}
We already know that at least one translation $[[pe]] = [[pe1]]$ exists
for every typing derivation. If $[[pe1]]$ is a representative translation then we
are done. Otherwise there exists another translation $[[pe2]]$ such that
$[[pe2]] \leq [[pe1]]$ and $ [[pe1]] \not \leq [[pe2]]$. By
\cref{lemma:subst_dec_measure}, we have $[[||pe2||]] < [[ ||pe1|| ]]$. We continue
with $[[pe]] = [[pe2]]$, and get a strictly decreasing sequence $[[ || pe1 ||  ]], [[ || pe2 || ]], \dots$.
By \cref{lemma:decrease_stop}, we know this sequence cannot be infinite long. Suppose it ends at $[[ || pej || ]]$,
by the construction of the sequence, we know that $[[pej]]$ is a representative translation of $[[e]]$.
\end{proof}


\newpage


\section{The Extended Algorithmic System}


\subsection{Syntax}


\begin{center}
\begin{tabular}{lrcl} \toprule
  Expressions & $[[ae]]$ & \syndef & $[[x]] \mid [[n]] \mid [[ \x : aA . ae ]]  \mid  [[\x . ae]] \mid [[ae1 ae2]] \mid [[ae : aA]] \mid [[ let x = ae1 in ae2  ]] $ \\
  Types & $[[aA]], [[aB]]$ & \syndef & $ [[int]] \mid [[a]] \mid [[evar]] \mid [[aA -> aB]] \mid [[\/ a. aA]] \mid [[unknown]] \mid [[static]] \mid [[gradual]] $ \\
  Monotypes & $[[at]], [[as]]$ & \syndef & $ [[int]] \mid [[a]] \mid [[evar]] \mid [[at -> as]] \mid [[static]] \mid [[gradual]]$ \\
  Existential variables & $[[evar]]$ & \syndef & $[[sa]]  \mid [[ga]]  $   \\
  Castable Types & $[[agc]]$ & \syndef & $ [[int]] \mid [[a]] \mid [[evar]] \mid [[agc1 -> agc2]] \mid [[\/ a. agc]] \mid [[unknown]] \mid [[gradual]] $ \\
  Castable Monotypes & $[[atc]]$ & \syndef & $ [[int]] \mid [[a]] \mid [[evar]] \mid [[atc1 -> atc2]] \mid [[gradual]]$ \\
  Algorithmic Contexts & $[[GG]], [[DD]], [[TT]]$ & \syndef & $[[empty]] \mid [[GG , x : aA]] \mid [[GG , a]] \mid [[GG , evar]]  \mid [[GG, sa = at]] \mid [[GG, ga = atc]] \mid [[ GG, mevar ]] $ \\
  Complete Contexts & $[[OO]]$ & \syndef & $[[empty]] \mid [[OO , x : aA]] \mid [[OO , a]] \mid [[OO, sa = at]] \mid [[OO, ga = atc]] \mid [[OO, mevar]] $ \\
  \bottomrule
\end{tabular}

\end{center}


\subsection{Type System}


\drules[as]{$ [[GG |- aA <~ aB -| DD ]] $}{Algorithmic Consistent Subtyping}{tvar, evar, int, arrow, forallR, forallLL, spar, gpar, unknownLL, unknownRR, instL, instR}

\drules[instl]{$ [[ GG |- evar <~~ aA -| DD   ]] $}{Instantiation I}{solveS, solveG, solveUS, solveUG, reachSGOne, reachSGTwo, reachOther, arr, forallR}

\drules[instr]{$ [[ GG |- aA <~~ evar -| DD   ]] $}{Instantiation II}{solveS, solveG, solveUS, solveUG, reachSGOne, reachSGTwo, reachOther, arr, forallLL}

\drules[inf]{$ [[ GG |- ae => aA -| DD ]] $}{Inference}{var, int, lamannTwo, lamTwo, anno, app, letTwo}

\drules[chk]{$ [[ GG |- ae <= aA -| DD ]] $}{Checking}{lam, gen, sub}

\drules[am]{$ [[ GG |- aA |> aA1 -> aA2 -| DD ]] $}{Algorithmic Matching}{forallL, arr, unknown, var}

\newpage

\section{Decidability} \label{app:decidable}

The decidability proofs mostly follow that of DK system. Whenever possible, we only show
the new cases; otherwise we provide full detailed proofs.

\subsection{Decidability of Instantiation}


\begin{lemma}[Left Unsolvedness Preservation]
  Let $[[GG]] = [[  GG0, evar, GG1 ]]$. If  $[[ GG |- evar <~~ aA -| DD    ]]$ or $[[  GG |- aA <~~ evar -| DD  ]]$, and $[[evarb]] \in \textsc{unsolved}([[GG0]])$,
  then  $[[DD]] = [[(DD0, evarb, DD1)]]$ or $[[DD]] = [[(DD0, evarb', evarb = evarb', DD1)]]$ where $[[evarb']]$ is a fresh unsolved existential.
\end{lemma}
\begin{proof}
  By induction on the given derivation. We show the new cases.

  \begin{itemize}
  \item Case \[     \ottaltinferrule{instl-solveUS}{}{  }{ [[  GG0, sa, GG1 |- sa <~~ unknown -| GG0 , ga, sa = ga, GG1 ]] }  \]
    First notice that $[[evarb]]$ cannot be $[[ga]]$. Then to the left of $[[sa]]$, the contexts $[[DD]]$ and $[[GG]]$ are the same $[[GG0]]$.
  \item Case \[  \drule{instl-solveUG}   \]
    Immediate, since to the left of $[[ga]]$, the contexts $[[DD]]$ and $[[GG]]$ are the same.
  \item Case \[  \drule{instl-reachSGOne}    \]
    First notice that $[[evarb]]$ cannot be $[[ga]]$. Then to the left of $[[sa]]$, the contexts $[[DD]]$ and $[[GG]]$ are the same.
  \item Case \[  \drule{instl-reachSGTwo}    \]
    If $[[evarb]] \neq [[sb]] $, immediate, since to the left of $[[ga]]$ ($[[evarb]]$ cannot be $[[gb]]$), the contexts $[[DD]]$ and $[[GG]]$ are the same.
    Otherwise, $[[sb]]$'s solution (i.e., $[[gb]]$) is a fresh unsolved existential that lies just before $[[sb]]$.
  \item Case \rref*{instr-solveUS} is similar to case \rref*{instl-solveUS}.
  \item Case \rref*{instr-solveUG} is similar to case \rref*{instl-solveUG}.
  \item Case \rref*{instr-reachSG1} is similar to case \rref*{instl-reachSG1}.
  \item Case \rref*{instr-reachSG2} is similar to case \rref*{instl-reachSG2}.
  \end{itemize}

\end{proof}


\begin{lemma}[Left Free Variable Preservation]
  Let $[[GG]] = [[  GG0, evar, GG1  ]]$. If $[[ GG |- evar <~~ aA -| DD    ]]$ or $[[  GG |- aA <~~ evar -| DD  ]]$, and $[[ GG |- aB ]]$
  and $[[ evar ]] \notin \textsc{fv}([[ [GG]aB  ]])$ and $[[evarb]] \in \textsc{unsolved}([[GG0]])  $
  and $[[evarb]]  \notin \textsc{fv}([[ [GG]aB  ]])  $,
  then $[[evarb]] \notin \textsc{fv}([[  [DD]aB ]])$.
\end{lemma}
\begin{proof}
  By induction on the given derivation. We show the new cases.
  \begin{itemize}
  \item Case \[  \drule{instl-solveUS}    \]
    Since $[[DD]]$ differs from $[[GG]]$ only in solving $[[sa]]$ to $[[ga]]$, and $[[ga]]$ is fresh,
    applying $[[DD]]$ to a type will not introduce $[[evarb]]$.  We have $[[evarb]]  \notin \textsc{fv}([[ [GG]aB  ]])  $, so
    $[[evarb]]  \notin \textsc{fv}([[ [DD]aB  ]])  $.

  \item Case \[  \drule{instl-solveUG}    \]
    Immediate, since $[[DD]]$ and $[[GG]]$ are the same.

  \item Case \[  \drule{instl-reachSGOne}    \]

    Since $[[DD]]$ differs from $[[GG]]$ only in solving $[[sa]]$ and $[[gb]]$ to $[[ga]]$, and $[[ga]]$ is fresh,
    applying $[[DD]]$ to a type will not introduce $[[evarb]]$.  We have $[[evarb]]  \notin \textsc{fv}([[ [GG]aB  ]])  $, so
    $[[evarb]]  \notin \textsc{fv}([[ [DD]aB  ]])  $.

  \item Case \[  \drule{instl-reachSGTwo}    \]

    Since $[[DD]]$ differs from $[[GG]]$ only in solving $[[sb]]$ and $[[ga]]$ to $[[gb]]$, and $[[gb]]$ is fresh,
    applying $[[DD]]$ to a type will not introduce $[[evarb]]$.  We have $[[evarb]]  \notin \textsc{fv}([[ [GG]aB  ]])  $, so
    $[[evarb]]  \notin \textsc{fv}([[ [DD]aB  ]])  $.

  \item Case \rref*{instr-solveUS} is similar to case \rref*{instl-solveUS}.
  \item Case \rref*{instr-solveUG} is similar to case \rref*{instl-solveUG}.
  \item Case \rref*{instr-reachSG1} is similar to case \rref*{instl-reachSG1}.
  \item Case \rref*{instr-reachSG2} is similar to case \rref*{instl-reachSG2}.


  \end{itemize}

\end{proof}


\begin{lemma}[Instantiation Size Preservation] \label{lemma:dec:instan}
  If $[[GG]] = [[ GG0, evar, GG1 ]]$ and $[[  GG |- evar  <~~ aA -| DD ]]$
  or $[[ GG |- aA <~~ evar -| DD  ]]$, and $[[GG |- aB]]$ and $[[evar notin fv( [GG]aB )]]$, then $ | [[   [GG]aB   ]] | = | [[   [DD]aB   ]] | $, where $| [[   aC  ]] | $ is the plain size of $[[aC]]$.
\end{lemma}
\begin{proof}
  By induction on the given derivation. We show the new cases.
  \begin{itemize}
  \item Case \[ \drule{instl-solveUS} \]
    Since $[[DD]]$ differs $[[GG]]$ only in solving $[[sa]]$, and we know $[[ sa notin fv([GG]aB)  ]]  $, we have $[[  [DD]aB  ]] = [[  [GG]aB ]]$, so
    $ | [[   [GG]aB   ]] | = | [[   [DD]aB   ]] | $.

  \item Case \[  \drule{instl-solveUG}   \]
    Immediate, since $[[DD]]$ and $[[GG]]$ are the same.

  \item Case \[   \drule{instl-reachSGOne}   \]
    Since $[[DD]]$ differs $[[GG]]$ only in solving $[[sa]]$ and $[[gb]]$, and we know $[[ sa notin fv([GG]aB)  ]]  $,
    even if $[[gb]]$ occurs in $[[  [GG]aB]]$, its solution is again an existential variable, so the size does not change,
    so $ | [[   [GG]aB   ]] | = | [[   [DD]aB   ]] | $.

  \item Case \[   \drule{instl-reachSGTwo}   \]
    Since $[[DD]]$ differs $[[GG]]$ only in solving $[[ga]]$ and $[[sb]]$, and we know $[[ ga notin fv([GG]aB)  ]]  $,
    even if $[[sb]]$ occurs in $[[  [GG]aB]]$, its solution is again an existential variable, so the size does not change,
    so $ | [[   [GG]aB   ]] | = | [[   [DD]aB   ]] | $.

  \item Case \rref*{instr-solveUS} is similar to case \rref*{instl-solveUS}.
  \item Case \rref*{instr-solveUG} is similar to case \rref*{instl-solveUG}.
  \item Case \rref*{instr-reachSG1} is similar to case \rref*{instl-reachSG1}.
  \item Case \rref*{instr-reachSG2} is similar to case \rref*{instl-reachSG2}.

  \end{itemize}
\end{proof}


\decInstan*
\begin{proof}
  By induction on the derivation of $[[  GG |- aA   ]]$. We show the new cases.
  \begin{itemize}
  \item Case \[  \drule{ad-unknown}   \]
    By \rref{instl-solveUS} or \rref{instl-solveUG}.

  \item Case \[  \drule{ad-static}   \]
    By \rref{instl-solveS}.

  \item Case \[  \drule{ad-gradual}   \]

    By \rref{instl-solveS} or \rref{instl-solveG}.


  \item Case \[    \ottaltinferrule{ad-evar}{}{  }{ [[  GG0, sa, GG1 |- ga  ]] }  \]
    If $[[ga]] \in [[GG0]]  $, then we have a derivation by \rref{instl-reachOther}. If $[[ga]] \in [[GG1]]$, then
    we have a derivation by \rref{instl-reachSG1}.


  \item Case \[    \ottaltinferrule{ad-evar}{}{  }{ [[  GG0, ga, GG1 |- sa  ]] }  \]
    If $[[sa]] \in [[GG0]]  $, then we have a derivation by \rref{instl-reachSG2}. If $[[sa]] \in [[GG1]]$, then
    we have a derivation by \rref{instl-reachOther}.

  \end{itemize}


\end{proof}



\subsection{Decidability of Algorithmic Consistent Subtyping}


\begin{lemma}[Monotypes Solve Variables] \label{lemma:mono_solve_var}
  If $[[ GG |- evar <~~ at -| DD ]]$ or $[[ GG |- at <~~ evar -| DD  ]]$, then if $[[  [GG]at   ]] = [[at]]$ and
  $[[evar]] \notin \textsc{fv}([[ [GG] at  ]])$, then $| \textsc{unsolved}([[GG]])   | = |  \textsc{unsolved}([[DD]])      | + 1$.
\end{lemma}
\begin{proof}
  By induction on the given derivation. Since our syntax of monotypes differ
  from DK only in having static and gradual parameters, we show only two affected cases.
  \begin{itemize}
  \item Case \[  \drule{instl-solveS}  \]
    It is immediate that $|  \textsc{unsolved}([[GG, sa, GG']])    | = |  \textsc{unsolved}([[GG, sa = at, GG']])    | + 1$.
  \item Case \[  \drule{instl-solveG}  \]
    It is immediate that $|  \textsc{unsolved}([[GG, ga, GG']])    | = |  \textsc{unsolved}([[GG, ga = atc, GG']])    | + 1$.
  \end{itemize}
\end{proof}


\begin{lemma}[Monotype Monotonicity] \label{lemma:mono_mono}
  If $[[GG |- at1 <~ at2 -| DD]]$  then $| \textsc{unsolved}([[DD]]) | \leq |  \textsc{unsolved}([[GG]])    |$.
\end{lemma}
\begin{proof}
  By induction on the derivation. We show the new cases.
  \begin{itemize}
  \item Case \rref*{as-spar,as-gpar}: In these rules, $[[DD]] = [[GG]]$, so $| \textsc{unsolved}([[DD]]) | = |  \textsc{unsolved}([[GG]])    |$.
  \end{itemize}

\end{proof}


\begin{lemma}[Substitution Decreases Size] \label{lemma:subst_decrease}
  If $[[GG |- aA]]$, then $|[[  GG |-   [GG]aA    ]]| \leq  |  [[GG |-aA]]      |   $.
\end{lemma}
\begin{proof}
  By induction on $| [[ GG |- aA  ]] |$. We show the new cases.
  \begin{itemize}
  \item $[[aA]] = [[unknown]]$, or $[[aA]] = [[static]]$, or $[[aA]] = [[gradual]]$ then $[[  [GG] aA ]] = [[aA]]$. Therefore
    $|[[  GG |-   [GG]aA    ]]| =  |  [[GG |-aA]]      |   $.
  \end{itemize}
\end{proof}


\begin{lemma}[Monotype Context Invariance] \label{lemma:mono_inva}
  If $[[GG |- at <~ at' -| DD]]$ where $[[  [GG]at   ]] = [[at]]$ and $[[  [GG]at'   ]] = [[at']]$ and
  $|   \textsc{unsolved}([[GG]])      | = |  \textsc{unsolved}([[DD]])    |$, then $[[DD]] = [[GG]]$.
\end{lemma}
\begin{proof}
  By induction on the derivation. We show the new cases.
  \begin{itemize}
  \item Cases \rref*{as-spar,as-gpar}: In these rules, the output context is the
    same as the input context, so the result is immediate.
  \item Case \[  \drule{as-instL}   \]
    By \cref{lemma:mono_solve_var}, $|\textsc{unsolved}([[ DD ]])| < | \textsc{unsolved}([[ GG[evar]  ]])   |$, which is contrary to what is given,
    so this case is impossible.

  \item Case \rref*{as-instR} is similar to \rref*{as-instL}.
  \end{itemize}

\end{proof}

\decsubtype*
\begin{proof}
  Let the judgment $[[ GG |- aA <~ aB -| DD   ]]$ be measured lexicographically by
 \begin{enumerate}[(M1)]
\item the number of $\forall$-quantifiers in $[[aA]]$ and $[[aB]]$;
\item the number of unknown types in $[[aA]]$ and $[[aB]]$;
\item $| \textsc{unsolved}([[GG]]) |$: the number of unsolved existential
  variables in $[[GG]]$;
\item $| [[GG |- aA]] | + | [[GG |- aB]]   |$.
\end{enumerate}

We focus on the interesting (and new) cases.

\begin{asparaitem}
\item Cases \rref*{as-spar,as-gpar,as-unknownLL,as-unknownRR} have no premises.
\item Case \[  \drule{as-arrow}   \]
  We discuss each premise separately:
  \begin{description}
    \item[First premise:]
      If $[[aA2]]$ or $[[aB2]]$ has a quantifier, then the first premise is smaller by (M1). Otherwise, if $[[aA2]]$ or $[[aB2]]$
      has a unknown type, then first premise is smaller by (M2). Otherwise, the first premise shares the same input context as the conclusion, so it
      has the same (M3), but the types $[[aB1]]$ and $[[aA1]]$ are subterms of the conclusion's types, so the first premise is smaller by (M4).
    \item[Second premise:] If $[[aB1]]$ or $[[aA1]]$ has a quantifier, then the second premise is smaller by (M1) because
      applying contexts will not introduce quantifiers. Otherwise,
      if $[[aB1]]$ or $[[aA1]]$ has a unknown type, then the second premise is smaller by (M2) because
      applying contexts will not introduce unknown types. Otherwise, at this point, we know
      $[[aB1]]$ and $[[aA1]]$ are monotypes, so by \cref{lemma:mono_mono} on the first premise,
      we have $| \textsc{unsolved}([[TT]])  | \leq | \textsc{unsolved}([[GG]])   |$.
      \begin{itemize}
      \item If $| \textsc{unsolved}([[TT]])  | < | \textsc{unsolved}([[GG]])   |$, then the second premise is smaller by (M3).
      \item If $| \textsc{unsolved}([[TT]])  | = | \textsc{unsolved}([[GG]])   |$, then we have the same (M3). By \cref{lemma:mono_inva}
        on the first premise, we know  $[[TT]] = [[GG]]$, so $| [[TT |- [TT]aA2]] | = | [[GG |- [GG]aA2]] |$.
        By \cref{lemma:subst_decrease} we know $| [[GG |- [GG]aA2]] | \leq  | [[ GG |- aA2  ]] |  $. Therefore we have
        \[
          | [[TT |- [TT]aA2]] | \leq  | [[ GG |- aA2  ]] |
        \]
        Same for $[[aB2]]$:
        \[
          | [[TT |- [TT]aB2]] | \leq  | [[ GG |- aB2  ]] |
        \]
        Therefore,
        \[
          | [[TT |- [TT]aA2]] | + | [[TT |- [TT]aB2]] | \leq | [[ GG |- aA2  ]] | +  | [[ GG |- aB2  ]] | < | [[ GG |- aA1 -> aA2  ]] | + | [[ GG |- aB1 -> aB2  ]] |
        \]
        and the second premise is smaller by (M4).
      \end{itemize}
  \end{description}
\end{asparaitem}

\end{proof}

\subsection{Decidability of Algorithmic Typing}

\decmatch*
\begin{proof}
  \Rref{am-arr,am-unknown,am-var} do not have premises. For \rref{am-forall},
  the size of $[[aA]]$ is decreasing in the premise.
\end{proof}


\dectyping*
\begin{proof}

  We consider the following measure:
  \begin{center}
    \begin{tabular}{lllll}
      \multirow{2}{*}{$ \Big \langle$} & \multirow{2}{*}{$[[ae]],$} & $[[=>]]$ & \multirow{2}{*}{$| [[GG |- aA]] |$} & \multirow{2}{*}{$\Big \rangle$} \\
                                       &                    & $[[<=]],$ &  &
    \end{tabular}
  \end{center}
  and show every inference/checking premise is smaller than the conclusion.
  \begin{itemize}
  \item \Rref{inf-var,inf-int} do not have premises.
  \item \Rref{inf-anno,inf-lamann,inf-lam,inf-let,chk-lam} all have strictly
    smaller $[[ae]]$ in the premises.
  \item \Rref{inf-app}: The first and third premises have strictly smaller
    $[[ae]]$. The second (matching) judgment is decidable by
    \cref{lemma:decmatch}.
  \item \Rref{chk-gen}: Both the premise and conclusion type the same term, and
    both are the checking judgments. However $| [[GG, a |- aA]] | < | [[GG |-  \/a. aA]] |$, so
    the premise is smaller.
  \item \Rref{chk-sub}: The first premise uses inference mode, so it is smaller.
    The second premise is decidable by \cref{thm:decsubtype}.
  \end{itemize}

\end{proof}

\newpage



\section{Properties of Consistent Subtyping}


\begin{clemma}[Consistent Subtyping is Reflexive] \label{lemma:sub_refl}%
  If $[[ dd |- A  ]]$ then $[[ dd |- A <~ A  ]]$.
\end{clemma}


\begin{clemma}[Monotype Equality]   \label{lemma:mono_equal}
  If  $[[ dd |- t <~ s  ]]$  then $[[t]] = [[s]]$.
\end{clemma}

\begin{lemma}[Invertibility]  \label{lemma:forall_invert}
  If $ [[dd |- A <~ \/b . B]] $ then $[[ dd, b |- A <~ B ]]$.
\end{lemma}
\begin{proof}
  By induction on the given derivation.
  \begin{itemize}
  \item \Rref{cs-arrow,cs-tvar,cs-int,cs-unknownRR,cs-spar,cs-gpar} are impossible since
    the supertype is not a forall type.
  \item Case \[  \drule{cs-forallR}   \] The premise is exactly what we need.
  \item Case \[  \drule{cs-forallL}  \] where $B = [[  \/b . B0 ]]$. By i.h.,
    we have $[[ dd, b |- A [a ~> t] <~ B0 ]]$. By \rref{cs-forallL} we have
    $[[  dd , b |- \/a . A <~ B0 ]]$.
  \item Case \[ \drule{cs-unknownLL}   \] where $[[gc]] = [[\/b . gc0 ]]$. By \rref{cs-unknownLL}
    we have $[[ dd , b |- unknown <~ gc0 ]]$.
  \end{itemize}
\end{proof}


\newpage


\section{Properties of Context Extension}

\subsection{Syntactic Properties}


Since the definition of the context extension judgment ($[[GG --> DD]]$,
\cref{fig:ctxt-extension}) is exactly the same as that of the DK system, we
refer the reader to their technical report~\citep{dunfield2013complete} for the proofs of the following
syntactic properties of context extension.



\begin{lemma}[Reverse Declaration Order Preservation]   \label{lemma:reverse_preserve}
  If $\Gamma \exto \Delta$ and $a$ and $b$ are both declared in $\Gamma$ and $a$
  is declared to the left of $b$ in $\Delta$, then $a$ is declared to the left
  of $b$ in $\Gamma$.
\end{lemma}



\begin{lemma}[Reflexivity]   \label{lemma:reflexivity}
  If $[[GG]]$ is well-formed then $[[GG --> GG]]$.
\end{lemma}


\begin{lemma}[Transitivity]   \label{lemma:transitivity}
  If $[[GG --> DD]] $ and $[[DD --> TT]]$ then $[[ GG --> TT  ]]$.
\end{lemma}

\begin{definition}[Softness]
  A context $[[TT]]$ is soft iff it consists only of $[[evar]]$ and $[[evar = at]]$ declarations.
\end{definition}


\begin{lemma}[Substitution Extension Invariance]  \label{lemma:subst_ext_invar}
  If $[[TT |- aA]]$ and $[[TT --> GG]]$ then $[[ [GG]aA ]] = [[ [GG] ([TT]aA)]]  $ and $ [[ [GG]aA ]] = [[ [TT] ([GG]aA)]]   $.
\end{lemma}


\begin{lemma}[Extension Order] \label{lemma:extension_order}%
  We have the following:
  \begin{enumerate}
  \item If $[[ GL , a, GR --> DD]]$ then $[[DD]]  = [[(DL, a, DR)]]  $ where
    $[[GL --> DL]]$. Moreover, if $[[GR]]$ is soft then $[[DR]]$ is soft.
  \item If $[[ GL , mevar, GR --> DD]]$ then $[[DD]]  = [[(DL, mevar, DR)]]  $ where
    $[[GL --> DL]]$. Moreover, if $[[GR]]$ is soft then $[[DR]]$ is soft.
  \item If $[[GL, evar, GR --> DD]]$ then $[[DD]]  = [[(DL, TT, DR)]]  $ where $[[GL --> DL]]$ and $[[TT]]$ is either $[[evar]]$ or $[[evar]] = [[at]]$ for some $[[at]]$.
  \item If $[[GL, evar = at, GR --> DD]]$ then $[[DD]]  = [[(DL, evar = at', DR)]]  $ where $[[GL --> DL]]$ and and $[[ [DL]at  ]] = [[ [DL] at' ]]$.
  \item If $[[GL , x : aA , GR --> DD]]$ then $[[DD]]  = [[(DL, x : aA', DR)]]  $ where $[[GL --> DL]]$ and $[[ [DL]aA  ]] = [[ [DL] aA' ]]$. Moreover, $[[GR]]$ is soft if and only if $[[DR]]$ is soft.
  \end{enumerate}
\end{lemma}


\begin{lemma}[Solution Admissibility for Extension] \label{lemma:solution_ext}
  If $[[ GL |- at  ]]$ then $[[  GL, evar, GR --> GL, evar = at, GR ]]$.
\end{lemma}


\begin{lemma}[Unsolved Variable Addition for Extension]   \label{lemma:unsolved_ext}
  We have that $ [[GL, GR --> GL, evar, GR]] $
\end{lemma}


\begin{lemma}[Parallel Admissibility]   \label{lemma:paralell_admit}
  If $\Gamma_L \exto \Delta_L$ and $\Gamma_L, \Gamma_R \exto \Delta_L, \Delta_R$ then:
  \begin{enumerate}
  \item $\Gamma_L, \genA, \Gamma_R \exto \Delta_L, \genA, \Delta_R$
  \item If $\Delta_L \vdash \tau'$ then $\Gamma_L, \genA, \Gamma_R \exto \Delta_L, \genA = \tau', \Delta_R$.
  \item If $\Gamma_L \vdash \tau$ and $\Delta_L \vdash \tau'$ and $\ctxsubst{\Delta_L}{\tau} = \ctxsubst{\Delta_L}{\tau'}$, then $\Gamma_L, \genA = \tau, \Gamma_R \exto \Delta_L, \genA = \tau', \Delta_R$.
  \end{enumerate}
\end{lemma}

\begin{lemma}[Parallel Extension Solution] \label{lemma:paralell_ext_solu}
  If $\Gamma_L, \genA, \Gamma_R \exto \Delta_L, \genA = \tau', \Delta_R$ and
  $\Gamma_L \vdash \tau$ and $\ctxsubst{\Delta_L}{\tau} =
  \ctxsubst{\Delta_L}{\tau'}$, then $\Gamma_L, \genA = \tau, \Gamma_R \exto
  \Delta_L, \genA = \tau', \Delta_R$.
\end{lemma}




\begin{lemma}[Drop Variable for Extension]   \label{lemma:drop_ext}
  If $[[ GG, evar --> DD  ]]$ then $[[ GG --> DD]]$.
\end{lemma}


\begin{lemma}[Finishing Types]
  If $\Omega \vdash A$ and $\Omega \exto \Omega'$ then $\ctxsubst{\Omega}{A} = \ctxsubst{\Omega'}{A}$.
  \label{lemma:finish_types}
\end{lemma}


\begin{lemma}[Finishing completions]
  If $\Omega \exto \Omega'$ then $\ctxsubst{\Omega}{\Omega} = \ctxsubst{\Omega'}{\Omega'}$.
  \label{lemma:finish_complete}
\end{lemma}



\begin{lemma}[Confluence of Completeness]   \label{lemma:confluence}
  If $[[  DD1 --> OO]]$ and $[[ DD2 --> OO]]$ then
  $[[ [OO]DD1 ]]  =  [[ [OO] DD2  ]]$.
\end{lemma}


\begin{lemma}[Variable Preservation]
  If $(x : A) \in \Delta$ or $(x : A) \in \Omega$ and $\Delta \exto \Omega$ then $(x : \ctxsubst{\Omega}{A}) \in \ctxsubst{\Omega}{\Delta}$.
  \label{lemma:variable_preservation}
\end{lemma}


\begin{lemma}[Softness Goes Away]
  If $\Delta, \Theta \exto \Omega, \Omega_Z$ where $\Delta \exto \Omega$ and $\Theta$ is soft, then $\ctxsubst{\Omega, \Omega_Z}{(\Delta, \Theta)} = \ctxsubst{\Omega}{\Delta}$.
  \label{lemma:subst_go_away}
\end{lemma}

\begin{lemma}[Stability of Complete Contexts]
  If $\Gamma \exto \Omega$ then $\ctxsubst{\Omega}{\Gamma} = \ctxsubst{\Omega}{\Omega}$.
  \label{lemma:stable_complete_ctxt}
\end{lemma}



\subsection{Instantiation Extends}

\begin{lemma}[Instantiation Extension]   \label{lemma:inst_extension}
 If $[[  GG |- evar <~~ aA  -| DD   ]]$ or $[[ GG |- aA <~~ evar -| DD  ]]$ then $[[ GG --> DD ]]$.
\end{lemma}
\begin{proof}
  By induction on the given instantiation derivation.
  \begin{itemize}
  \item \Rref{instl-solveS,instl-solveG,instl-reachOther,instr-solveS,instr-solveG,instr-reachOther}
    are immediate from \cref{lemma:solution_ext}.

  \item Case \[ \drule{instl-solveUS} \]
    By \cref{lemma:unsolved_ext} we have $[[ GG[sa] --> GG[ga, sa]  ]]$. By \cref{lemma:solution_ext} we have
    $[[ GG[ga, sa] --> GG[ga, sa = ga]  ]]$. By \cref{lemma:transitivity} we have $[[ GG[sa] --> GG[ga, sa=ga]   ]]$.

  \item Case \[ \drule{instl-solveUG} \] Immediate by \cref{lemma:reflexivity}.


  \item Case \[  \drule{instl-reachSGOne}  \]
    By \cref{lemma:unsolved_ext} we have $[[  GG[sa][gb] --> GG[ga,sa][gb]  ]]$. By applying \cref{lemma:solution_ext} twice,
    we have $[[ GG[ga,sa][gb] --> GG[ga,sa=ga][gb=ga]    ]]$. By \cref{lemma:transitivity} we have $[[ GG[sa][gb] --> GG[ga,sa=ga][gb=ga]  ]]$.

  \item Case \[  \drule{instl-reachSGTwo}   \] Same as the case for \rref{instl-reachSG1}.

  \item Case \[ \drule{instl-arr}\]
    By applying \cref{lemma:unsolved_ext} twice, we have $[[  GG[evar] --> GG[evar2, evar1, evar]  ]]$.
    By \cref{lemma:solution_ext} we have $[[ GG[evar2, evar1, evar] --> GG[evar2, evar1, evar = evar1 -> evar2]  ]]$.
    By i.h., we have $[[ GG[evar2, evar1, evar = evar1 -> evar2] --> TT  ]]$ and $[[ TT --> DD  ]]$. By \cref{lemma:transitivity} we
    have $[[  GG[evar] --> DD  ]]$.

  \item Case \[ \drule{instl-forallR}  \]
    By i.h., we have $[[ GG[evar], b --> DD , b , TT  ]]$. By \cref{lemma:extension_order} (1), we have $[[GG[evar] --> DD]]$.

  \item Case \[ \drule{instr-solveUS}  \] Same as the case for \rref{instl-solveUS}.

  \item Case \[  \drule{instr-solveUG}\] Same as the case for \rref{instl-solveUG}.

  \item Case \[  \drule{instr-reachSGOne}  \]  Same as the case for \rref{instl-reachSG1}.

  \item Case \[  \drule{instr-reachSGTwo}  \]  Same as the case for \rref{instl-reachSG1}.

  \item Case \[ \drule{instr-arr} \] Same as the case for \rref{instl-arr}.

  \item Case \[ \drule{instr-forallLL}   \]
    By i.h., we have $[[ GG[evar] , mevarb, sb --> DD, mevarb, TT ]]$. By \cref{lemma:extension_order}(2) we have $[[ GG[evar] --> DD  ]]$.

  \end{itemize}
\end{proof}

\subsection{Consistent Subtyping Extends}

\begin{lemma} \label{lemma:contamination_extension}
  If $[[GG |- aA ]]$ then $[[GG --> [aA] GG]]$.
\end{lemma}
\begin{proof}
  By induction on the structure of $[[GG]]$. The only interesting case is when $[[GG]] = [[GG' , sa]] $. By \cref{def:contamination},
  we have $[[ [aA] (GG' , sa) ]] = [[ [aA] GG' , ga, sa = ga]] $. By i.h., we have $[[ GG' --> [aA] GG'  ]]$.
  By definition of context extension we have $[[ GG' , sa --> [aA] GG' , sa  ]]$. By \cref{lemma:unsolved_ext} we have
  $[[ [aA] GG' , sa --> [aA] GG' , ga, sa ]]$. By \cref{lemma:solution_ext} we have $[[ [aA] GG' , ga, sa --> [aA] GG' , ga, sa = ga  ]]$.
  By \cref{lemma:transitivity} we have $[[ GG' , sa --> [aA] GG' , ga, sa = ga ]]$.
\end{proof}

\begin{lemma}[Consistent Subtyping Extension]   \label{lemma:sub_extension}
  If $ [[GG |- aA <~ aB -| DD]]  $ then $ [[GG --> DD]]$.
\end{lemma}
\begin{proof}
  By induction on the derivation of consistent subtyping.

  \begin{itemize}
  \item \Rref{as-tvar,as-evar,as-int,as-spar,as-gpar} are immediate from \cref{lemma:reflexivity}.

  \item Case \[ \drule{as-arrow} \]
    By i.h., we have $[[GG --> TT]]$ and $[[TT --> DD]]$. By \cref{lemma:transitivity}, we have $[[GG --> DD]]$.

  \item Case \[ \drule{as-forallR} \]
    By i.h., we have $[[GG , a --> DD , a , TT]]$. By \cref{lemma:extension_order} (1), we have $[[GG --> DD]]$.


  \item Case \[ \drule{as-forallLL} \]
    By i.h., we have $[[GG, mevar, sa --> DD, mevar, TT]]$. By \cref{lemma:extension_order} (2), we have $[[GG --> DD]]$.


  \item Case \[ \drule{as-unknownLL}  \] Immediate by \cref{lemma:contamination_extension}.

  \item Case \[ \drule{as-unknownRR}  \] Immediate by \cref{lemma:contamination_extension}.


  \item \Rref{as-instL,as-instR} are immediate.

  \end{itemize}

\end{proof}


\newpage

\section{Soundness of Consistent Subtyping}


\begin{definition}[Filling]   \label{def:filling}
  The filling of a context $[[ | GG | ]]$ solves all unsolved variables:
  \begin{align*}
    [[| empty |]] &= [[ empty ]] \\
    [[|GG , x : aA|]] &= [[|GG| , x : aA]] \\
    [[|GG, a|]] &= [[|GG| , a]] \\
    [[|GG, evar = at|]] &= [[|GG| , evar = at]] \\
    [[|GG, evar|]] &= [[|GG| , evar = int]] \\
    [[|GG, mevar|]] &= [[|GG| , mevar]]
  \end{align*}
\end{definition}


\begin{lemma}[Substitution Stability]
  For any well-formed complete context $(\Omega, \Omega_Z)$, if $\Omega \vdash A$
  then $\ctxsubst{\Omega}{A} = \ctxsubst{\Omega,\Omega_Z}{A}$.
  \label{lemma:subst_stable}
\end{lemma}
\begin{proof}
  By induction on $\Omega_Z$. If $\Omega_Z = [[empty]]$, the result is
  immediate. Otherwise use the i.h. and the fact that $\Omega \vdash A$ implies
  $\textsc{fv}(A) \cap \textsc{dom}(\Omega_Z) = \emptyset$.
\end{proof}



\begin{lemma}[Filling Completes]   \label{lemma:filling_completes}
  If $[[GG --> OO]]$ and $[[(GG, TT)]]$ is well-formed, then $[[GG , TT --> OO, | TT |]]$.
\end{lemma}
\begin{proof}
  By induction on $[[TT]]$, following \cref{def:filling} and applying the rules for context extension.
\end{proof}

\instsoundness*
\begin{proof}
  By induction on the given instantiation derivation.
  \begin{itemize}
  \item Case \[ \drule{instl-solveS} \] Immediate from \cref{lemma:sub_refl}.

  \item Case \[ \drule{instl-solveG} \] Immediate from \cref{lemma:sub_refl}.

  \item Case \[ \drule{instl-solveUS} \]
    We know $[[ [OO] sa ]] = [[tc]] $ for some castable monotype $[[tc]]$, and $[[tc]] \in [[gc]]$.
    By \rref{cs-unknownRR}, we have $[[  [OO] (GG[sa]) |- tc <~ unknown  ]]$


  \item Case \[  \drule{instl-solveUG} \] Similar to the case for \rref{instl-solveUS}.

  \item Case \[ \drule{instl-reachSGOne}  \]
    We know $[[ [OO] sa ]] = [[ [OO]ga]] =  [[tc]] $ and $[[ [OO] gb  ]] = [[ [OO]ga]] = [[tc]]$ for some castable monotype $[[tc]]$.
    By \cref{lemma:sub_refl} we have $[[  [OO](GG[sa][gb] ) |- tc <~ tc   ]]$.

  \item Case \[  \drule{instl-reachSGTwo}   \] Similar to the case for \rref{instl-reachSG1}.

  \item Case \[ \drule{instl-reachOther}  \]
    Let $[[DD]] = [[GG[evar][evarb]  ]]$, we have $[[ [OO]evar ]] = [[t]] $ and $[[  [OO] evarb  ]]  = [[  [OO] evar ]] = [[t]] $
    for some monotype $[[t]]$. By \cref{lemma:sub_refl} we have $[[ [OO]DD |- t <~ t  ]]$.

  \item Case \[ \drule{instl-arr}   \]
    Let $[[GG1]] = [[ GG[evar2, evar1, evar = evar1 -> evar2]  ]]    $:
    \begin{longtable}[l]{ll|l}
      & $[[TT |- evar2 <~~ [TT]aA2 -| DD]]$ & Premise \\
      & $[[ TT --> DD ]]$ & By \cref{lemma:inst_extension} \\
      & $[[ DD --> OO ]]$ & Given \\
      & $[[TT --> OO]]$ & By \cref{lemma:transitivity} \\
      & $[[ GG1 |- aA1 <~~ evar1 -| TT]]$ & Given \\
      & $[[  [OO]DD |- [OO]aA1 <~ [OO]evar1 ]]$ & By i.h. and \Cref{lemma:confluence} \\
      & $[[TT |- evar2 <~~ [TT]aA2 -| DD]]$& Premise \\
      & $[[ [OO]DD |- [OO]evar2 <~ [OO] ([TT]aA2)  ]]$ & By i.h. \\
      & $[[  TT --> DD ]]$ & Above \\
      & $[[ [OO]DD |- [OO]evar2 <~ [OO] aA2  ]]$ & By \cref{lemma:subst_ext_invar} \\
      & $[[  [OO] DD |- [OO] evar1 -> [OO] evar2 <~ [OO]aA1 -> [OO]aA2  ]]$ & By \rref{cs-arrow} \\
      & $[[  [OO] DD |- [OO] evar <~ [OO] (aA1 -> aA2)  ]] $ & By def. of substitution
    \end{longtable}

  \item Case \[ \drule{instl-forallR}  \]
    \begin{longtable}[l]{ll|l}
      & $[[DD , b, TT -->  OO , b , |TT|]]  $ & By \cref{lemma:filling_completes} \\
      & $[[GG[evar] , b |- evar <~~ aB -| DD , b , TT]]$ & Given \\
      & $[[ [OO , b , |TT|](DD , b , TT)  |- [OO , b , |TT|] evar <~ [OO , b , |TT|] aB    ]]$ & By i.h. \\
      & $[[ [OO , b , |TT|](DD , b , TT)  |- [OO , b ] evar <~ [OO , b ] aB    ]]$ & Free variables in $[[evar]]$ and $[[aB]]$ are declared in $[[(OO, b)]]$\\
      & $[[ [OO , b ](DD , b )  |- [OO , b ] evar <~ [OO , b ] aB    ]]$ & By context partitioning and thinning \\
      & $[[ [OO]DD , b   |- [OO] evar <~ [OO] aB    ]]$ & By context substitution \\
      & $[[ [OO]DD    |- [OO] evar <~ \/b . [OO] aB    ]]$ & By \rref{cs-forallR} \\
       & $[[ [OO]DD    |- [OO] evar <~ [OO] (\/b . aB)    ]]$ & By def. of substitution
    \end{longtable}

  \item Case \[ \drule{instr-forallLL}  \]

    \begin{longtable}[l]{ll}
      & $[[DD , mevarb, TT -->  OO , mevarb , |TT|]]  $ \qquad By \cref{lemma:filling_completes} \\
      & $[[GG[evar] , mevarb, sb |- aB[b ~> sb] <~~ evar -| DD, mevarb, TT]]$ \qquad Premise \\
      & $[[ [OO , mevarb , |TT|] (DD , mevarb, TT) |- [OO , mevarb , |TT|] (aB[b ~> sb]) <~ [OO , mevarb , |TT|]evar ]]$ \qquad By i.h. \\
      & $[[ [OO] DD |- ([OO]aB) [b ~> [OO , mevarb , |TT|] sb] <~ [OO]evar ]]$ \qquad By distributivity of substitution \\
      & $[[  [OO] DD |- [OO, mevarb, |TT|]sb  ]]$ \qquad Follows from def. of context application \\
      & $[[  [OO] DD |- \/ b . [OO] aB  <~ [OO] evar   ]]$ \qquad By \rref{cs-forallL} and $[[ [OO, mevarb, |TT| ] sb ]]$ is a monotype  \\
       & $[[  [OO] DD |- [OO]  (\/ b . aB)  <~ [OO] evar   ]]$ \qquad By def. of substitution
    \end{longtable}

  \item The rest of the cases are similar to the above cases.

  \end{itemize}
\end{proof}


\subsoundness*
\begin{proof}
  By induction on the derivation of consistent subtyping.

  \begin{itemize}
  \item Case \[ \drule{as-tvar}  \]

    \begin{longtable}[l]{ll|l}
      & $ [[a in GG[a] ]]  $ & Given \\
      & $ [[ a in [OO](GG[a])  ]]   $ & Follows from def. of context application \\
      & $ [[ [OO](GG[a]) |- a <~ a    ]]  $ & By \rref{cs-tvar} \\
      & $ [[ [OO](GG[a]) |- [OO]a <~ [OO]a    ]]     $ & By def. of substitution
    \end{longtable}


  \item Case \[  \drule{as-evar}   \]

    \begin{longtable}[l]{ll|l}
      & $ [[ [OO]evar  ]]  $ defined & Follows from def. of context application \\
      & $  [[ [OO]DD |- [OO]evar  ]]  $ & Follows from $ [[DD]] = [[ [GG]evar  ]]  $ \\
      & $  [[ [OO]DD |- [OO]evar <~ [OO]evar    ]]  $ & By \cref{lemma:sub_refl}
    \end{longtable}

  \item Case \[  \drule{as-int}  \] Immediate.


  \item Case \[  \drule{as-arrow}    \]

    \begin{longtable}[l]{ll|l}
      & $  [[ GG |- aB1 <~ aA1 -| TT   ]]  $ & Premise \\
      & $   [[ DD --> OO   ]]  $ & Given \\
      & $   [[ TT --> OO       ]]$ & By \cref{lemma:transitivity} \\
      & $[[  [OO] TT |- [OO]aB1 <~ [OO]aA1      ]]$ & By i.h. \\
      & $[[  [OO] DD |- [OO]aB1 <~ [OO]aA1      ]]$ & By \cref{lemma:confluence} \\
      & $ [[ TT |- [TT] aA2 <~ [TT] aB2 -| DD]]   $ & Premise \\
      & $ [[ [OO]DD |- [OO]([TT] aA2) <~ [OO]([TT] aB2) ]]   $ & By i.h. \\
      & $ [[ [OO]([TT] aA2)  ]] = [[ [OO]aA2  ]]  $ & By \cref{lemma:subst_ext_invar} \\
      & $ [[ [OO]([TT] aB2)  ]] = [[ [OO]aB2  ]]  $ & By \cref{lemma:subst_ext_invar} \\
      & $ [[ [OO]DD |- [OO]aA2 <~ [OO]aB2 ]]    $ & By above equalities \\
      & $  [[ [OO]DD |-[OO]aA1 -> [OO]aA2 <~[OO]aB1 -> [OO]aB2 ]]     $ & By \rref{cs-arrow} \\
      & $  [[ [OO]DD |-[OO](aA1 -> aA2)  <~[OO](aB1 -> aB2) ]]       $ & By def. of substitution
    \end{longtable}


  \item Case \[ \drule{as-forallR}  \]

      \begin{longtable}[l]{ll|l}
      & $  [[GG, a --> DD, a , TT]]   $ & By \cref{lemma:sub_extension} \\
      & $[[TT]]$ is soft & By \cref{lemma:extension_order} (1) where $[[GR]] = [[empty]]$ \\
      & $[[ DD --> OO ]]$ & Given \\
      & $\underbrace{[[DD , a, TT]]}_{[[DD']]} [[-->]] \underbrace{  [[OO , a, |TT| ]]   }_{[[OO']]}$ & By \cref{lemma:filling_completes} \\
      & $  [[ GG, a |- aA <~ aB -| DD, a, TT ]]   $ & Given \\
      & $ [[ [OO']DD' |- [OO']aA <~ [OO']aB  ]]  $ & By i.h. \\
      & $ [[ [OO']aA  ]] = [[ [OO , a] aA  ]]   $ & By \cref{lemma:subst_stable} \\
      & $ [[ [OO']aB  ]] = [[ [OO , a] aB  ]]   $ & By \cref{lemma:subst_stable} \\
      & $ [[ [OO']DD'   ]]  = [[ [OO, a](DD, a)   ]] $ & By \cref{lemma:subst_go_away} \\
      & $[[ [OO, a](DD, a) |- [OO, a]aA <~ [OO, a]aB  ]]$ & By above equalities \\
      & $ [[ [OO]DD, a |- [OO]aA <~ [OO]aB  ]]    $ & By def. of substitution \\
      & $ [[ [OO]DD |- [OO]aA <~ \/a. [OO]aB  ]]  $ & By \rref{cs-forallR} \\
      & $ [[ [OO]DD |- [OO]aA <~ [OO] (\/a. aB)  ]]   $ & By def. of substitution
      \end{longtable}


    \item Case \[  \drule{as-forallLL}  \]

      \begin{longtable}[l]{ll|l}
        & Let $[[OO']] = [[OO, msa, |TT|]]$ \\
        & $[[  DD --> OO  ]]$ & Given \\
        & $[[DD, msa, TT --> OO']]$  & By \cref{lemma:filling_completes} \\
        & $[[ GG, msa, sa |- aA [a ~> sa] <~ aB -| DD, msa, TT]]$ & Premise \\
        & $[[ [OO'](DD, msa, TT) |- [OO'](aA [a ~> sa]) <~ [OO']aB     ]]$ & By i.h. \\
        & $ [[ [OO']aB  ]] = [[ [OO , a] aB  ]]   $ & By \cref{lemma:subst_stable} \\
        & $ [[ [OO](DD, msa, TT) |-  [OO'] aA [a ~> [OO']sa] <~ [OO]aB     ]]  $ & By distributivity of substitution \\
        & $[[ [OO](DD, msa, TT) |- [OO']sa    ]]$ & Follows from def. of context application \\
        & $ [[ [OO](DD, msa, TT) |- \/a. [OO'] aA <~ [OO]aB     ]]  $ & By \rref{cs-forallL} \\
        & $ [[ [OO]DD |- \/a. [OO] aA <~ [OO]aB     ]]  $ & By context partitioning \\
        & $[[ [OO]DD |- [OO] (\/a. aA) <~ [OO]aB     ]] $ & By def. of substitution
      \end{longtable}

    \item Case \[  \drule{as-spar}  \] Immediate from \rref{cs-spar}.

    \item Case \[  \drule{as-gpar}  \] Immediate from \rref{cs-gpar}.


    \item Case \[ \drule{as-unknownLL}   \] Immediate from \rref{cs-unknownLL}.

    \item Case \[ \drule{as-unknownRR}   \] Immediate from \rref{cs-unknownRR}.

    \item Case \[  \drule{as-instL}     \]
      \begin{longtable}[l]{ll|l}
        & $[[ GG[evar] |- evar <~~ aA -| DD]]$ & Premise \\
        & $[[  [OO]DD |- [OO]evar <~ [OO]aA   ]]$ & By \cref{thm:inst_soundness}
      \end{longtable}

    \item Case \[  \drule{as-instR}     \] Similar to the case for \rref{as-instL}.

  \end{itemize}
\end{proof}


\newpage

\section{Soundness of Typing}

\textbf{Note:} We use $\byhave$ to improve readability when the conclusion has several parts.


\begin{lemma}[Matching Extension]   \label{lemma:match_extension}
  If $[[  GG |- aA |> aA1 -> aA2 -| DD]]$ then $ [[GG --> DD]]  $.
\end{lemma}
\begin{proof}
  By induction on the given derivation.
  \begin{itemize}
  \item Case \[ \drule{am-forallL} \]
    By i.h., we have $[[ GG, sa --> DD  ]]$. By \cref{lemma:drop_ext}, we have $[[GG --> DD]]$.

  \item Case \[  \drule{am-arr}  \] Immediate by \cref{lemma:reflexivity}.

  \item Case \[ \drule{am-unknown}   \] Immediate by \cref{lemma:reflexivity}.

  \item Case \[ \drule{am-var} \]
    By applying \cref{lemma:unsolved_ext} twice, we have
    $ [[ GG[evar] --> GG[evar1, evar2, evar]  ]]   $. By
    \cref{lemma:solution_ext}, we have $[[ GG[evar1, evar2, evar] --> GG[evar1, evar2, evar = evar1 -> evar2]  ]]$. By
    \cref{lemma:transitivity}, we have $[[ GG[evar] --> GG[evar1, evar2, evar = evar1 -> evar2]  ]]$.
  \end{itemize}

\end{proof}


\begin{lemma}[Typing Extension]   \label{lemma:typing_extension}
  If $ [[GG |- ae => aA -| DD]] $ or $ [[ GG |- ae <= aA -| DD  ]]  $ then $[[ GG --> DD]]$.
\end{lemma}
\begin{proof}
  By induction on the given derivation.

  \begin{itemize}
  \item Case \[ \drule{inf-var}  \] Immediate by \cref{lemma:reflexivity}.

  \item Case \[  \drule{inf-int}  \] Immediate by \cref{lemma:reflexivity}.

  \item Case \[ \drule{inf-lamTwo}   \]
    By i.h., we have $[[ GG, sa, sb, x : sa --> DD, x : sa, TT  ]]$. By \cref{lemma:extension_order}, we have
    $[[ GG, sa, sb --> DD  ]]$. By definition, we have $[[  GG --> GG, sa, sb  ]]$. By \cref{lemma:transitivity}
    we have $[[ GG --> DD ]]$.

  \item Case \[  \drule{inf-lamannTwo}  \]
    By i.h., we have $[[ GG, sb, x : aA --> DD, x : aA, TT  ]]   $. By \cref{lemma:extension_order},
    we have $[[GG --> DD]]$.

  \item Case \[ \drule{inf-app}   \]
    By i.h., we have $[[  GG --> TT1  ]]$, $[[ TT2 --> DD]]$. By \cref{lemma:match_extension},
    we have $[[ TT1 --> TT2  ]]$. By \cref{lemma:transitivity} , we have
    $[[GG --> DD]]$.


  \item Case \[ \drule{inf-anno}   \]
    By i.h., we have $[[ GG --> DD ]]$.

  \item Case \[ \drule{chk-gen}   \]
    By i.h., we have $[[  GG, a --> DD, a, TT  ]]$. By
    \cref{lemma:extension_order} we have $[[ GG --> DD]]$.

  \item Case \[ \drule{chk-lam} \]
    By i.h., we have $[[GG , x : aA --> DD, x : aA, TT]]$.
    By \cref{lemma:extension_order} we have $[[GG --> DD]]$.

  \item Case \[ \drule{chk-sub} \]
    By i.h., we have $[[GG --> TT]]$. By
    \cref{lemma:sub_extension} we have $[[TT --> DD]]$. By
    \cref{lemma:transitivity} we have $[[GG --> DD]]$.
  \end{itemize}

\end{proof}

\begin{theorem}[Matching Soundness]  \label{thm:match_soundness}%
  If $[[GG |- aA |> aA1 -> aA2 -| DD]]$ where $[[ [GG]aA = aA  ]]$ and $[[ DD --> OO ]]$
  then $[[  [OO]DD |- [OO]aA |> [OO]aA1 -> [OO]aA2 ]]$.
\end{theorem}
\begin{proof}
  By induction on the given derivation.

  \begin{itemize}
  \item Case \[ \drule{am-forallL}  \]
    \begin{longtable}[l]{ll|l}
      & $[[ GG , sa |- aA[a ~> sa] |> aA1 -> aA2 -| DD]]$ & Premise \\
      & $[[DD --> OO]]$ & Given \\
      & $[[ [OO]DD |- [OO](aA[a ~> sa]) |> [OO]aA1 -> [OO]aA2   ]]$ & By i.h. \\
      & $[[ [OO]DD |- [OO]aA [a ~> [OO]sa] |> [OO]aA1 -> [OO]aA2   ]]$ & By distributivity of substitution \\
      & $[[ [OO]DD |- [OO]sa   ]]$ & Follows from def. of context application \\
      & $[[ [OO]DD |- \/a. [OO]aA |> [OO]aA1 -> [OO]aA2   ]]$ & By \rref{m-forall} \\
      & $[[ [OO]DD |- [OO] (\/a. aA) |> [OO]aA1 -> [OO]aA2   ]]$ & By def. of substitution
    \end{longtable}


  \item Case \[  \drule{am-arr}  \] Immediate from \rref{m-arr}.

  \item Case \[  \drule{am-unknown}  \] Immediate from \rref{m-unknown}.

  \item Case \[  \drule{am-var}  \]
      \begin{longtable}[l]{ll|l}
        &$[[DD --> OO]]$& Given \\
        & $[[ [OO]evar  ]] = [[ [OO]evar1 -> [OO]evar2   ]]$ & By def. of context application \\
        & $[[  [OO]DD |- [OO]evar1 -> [OO]evar2 |> [OO]evar1 -> [OO]evar2  ]]$ & By \rref{m-arr}
      \end{longtable}

  \end{itemize}

\end{proof}


\typingsoundness*
\begin{proof}
  By induction on the given derivation.

  \begin{itemize}
  \item Case \[ \drule{inf-var}  \]
    \begin{longtable}[l]{ll|l}
      &$[[(x : aA) in GG]]$ & Premise \\
      &$[[(x : aA) in  DD]]$ & $[[DD]] = [[OO]]$ \\
      &$[[DD --> OO]]$ & Given \\
      &$[[ (x : [OO]aA) in [OO]GG  ]]$ & By \Cref{lemma:variable_preservation} \\
      $\byhave$&$[[ [OO]GG |- x : [OO]aA   ]]$ & By \rref{var} \\
      $\byhave$& $\erase{[[x]]} = \erase{[[x]]}$ & By def. of erasure
    \end{longtable}

  \item Case \[ \drule{inf-int}  \]
    \begin{longtable}[l]{ll|l}
      $\byhave$&$[[ [OO]GG |- n : int  ]]$ & By \rref{int} \\
      $\byhave$& $\erase{[[n]]} = \erase{[[n]]}$ & By def. of erasure
    \end{longtable}

  \item Case \[ \drule{inf-lamannTwo}  \]

    \begin{longtable}[l]{ll|l}
      & $[[  GG , sa, x : aA --> DD, x : aA , TT  ]]$ & By \Cref{lemma:typing_extension} \\
      & $[[TT]]$ is soft & By \Cref{lemma:extension_order} where $[[GR]] = [[empty]]$ \\
      & $[[ DD --> OO]]$ & Given \\
      & $\underbrace{[[DD , x : aA , TT]]}_{[[DD']]} [[-->]] \underbrace{[[OO, x : aA, |TT|]] }_{[[OO']]}$ & By \Cref{lemma:filling_completes} \\
      & $[[ GG, x: aA |- ae => aB -| DD, x : aA, TT ]]$ & Premise \\
      & $[[  [OO'] DD' |- e' : [OO']sb   ]]$ & By i.h. \\
      & $\erase{[[ae]]} = \erase{e'}$ & above \\
      & $[[ [OO']sb   ]] = [[ [OO, x : aA]sb  ]] = [[ [OO]sb  ]]$ & By \Cref{lemma:subst_stable} and def. of substitution \\
      & $[[ [OO']DD' ]]$ = $[[ [OO]DD , x : [OO]aA   ]]$ & By \Cref{lemma:subst_go_away} and def. of context substitution \\
      & $[[  [OO] DD, x : [OO]aA |- e' : [OO]sb   ]] $ & By above equalities \\
      & $[[  [OO] DD |- \ x : [OO]aA . e' : [OO]aA -> [OO]sb   ]] $ & By \rref{lamann} \\
      & $[[ [OO]aA  ]] = [[A]]$ & Type annotations cannot contain evars \\
      & $[[  [OO] DD |- \ x : A . e' :[OO]aA -> [OO]sb   ]]$ & By above equality \\
      $\byhave$& $ [[  [OO] DD |- \ x : A . e' :[OO](aA -> sb)   ]] $ & By def. of substitution \\
      $\byhave$& $\erase{[[\x : A . e']]} = \erlam{x}{\erase{e'}} = \erlam{x}{\erase{e}} = \erase{\blam{x}{A}{e}} $ & By def. of erasure
    \end{longtable}



  \item Case \[  \drule{inf-lam}  \]

    \begin{longtable}[l]{ll|l}
      &$[[ GG, sa, sb, x : sa --> DD, x : sa, TT   ]]$& By \cref{lemma:typing_extension} \\
      &$[[ GG , sa, sb --> DD  ]]$& By \cref{lemma:extension_order} \\
      &$[[TT]]$ is soft & Above \\ \\
      &$[[ DD --> OO  ]]$  & Given \\
      &$[[  DD , x : sa --> OO , x : [OO]sa  ]]$  & By def \\
      &$\underbrace{[[DD, x : sa, TT]]}_{[[DD']]} [[-->]] \underbrace{ [[OO, x : [OO]sa, |TT|]] }_{[[OO']]}$  & By \cref{lemma:filling_completes} \\ \\

      & $ [[  GG, sa, sb, x : sa |- ae <= sb -| DD, x : sa, TT ]] $ & Premise \\
      & $[[ [OO']DD' |- e' : [OO']sb   ]]$ & By i.h. \\
      & $\erase{[[ae]]} = \erase{[[e']]}$ & Above \\
      & $[[ [OO']sb  ]] = [[ [OO]sb  ]]$ & By def. of context substitution \\
      & $[[  [OO']DD'  ]] = [[ [OO]DD, x : [OO]sa  ]]  $ & By def. of context substitution \\
      & $[[ [OO]DD, x : [OO]sa |- e' : [OO]sb   ]]$ & By above equalities \\
      & $[[ [OO]sa ]]$ is a monotype & $\Omega$ is predicative \\
      & $[[ [OO]DD |-\x .  e' : [OO]sa -> [OO]sb   ]] $ & By \rref{lam} \\
      $\byhave$& $[[ [OO]DD |-\x .  e' : [OO](sa -> sb)  ]]$ & By def. of substitution \\
      $\byhave$& $\erase{\erlam{x}{e}} = \erlam{x}{\erase{e}} = \erlam{x}{\erase{e'}} = \erase{\erlam{x}{e'}}$ & By def. of erasure
    \end{longtable}



  \item Case \[  \drule{inf-app}  \]

    \begin{longtable}[l]{ll|l}
      & $[[ DD --> OO  ]]$ & Given \\
      & $[[TT1 --> OO]]$ & By \cref{lemma:match_extension,lemma:typing_extension,lemma:transitivity} \\
      & $[[ GG |- ae1 => aA -| TT1]]$ & Premise \\
      & $[[ [OO]TT1 |- e1' : [OO]aA    ]]$ & By i.h. \\
      & $\erase{[[e1']]} = \erase{[[ae1]]}$ & above \\
      & $[[ [OO]TT1  ]] = [[ [OO]DD  ]]$ & By \Cref{lemma:confluence} \\
      & $[[ [OO]DD |- e1' : [OO]aA    ]]$ & By above equality \\
      & $[[ TT2 |- ae2 <= [TT2]aA1 -| DD ]]$ & Premise \\
      & $[[ [OO] DD |- e2' : [OO]aA1   ]]$ & By i.h. \\
      & $\erase{[[e2']]} = \erase{[[ae2]]}$ & Above \\
      & $[[ TT1 |- [TT1] aA |> aA1 -> aA2 -| TT2]]$ & Premise \\
      & $[[ [OO]TT2 |- [OO]([TT1] aA) |> [OO]aA1 -> [OO]aA2     ]]$ & By \Cref{thm:match_soundness} \\
      & $[[ [OO]TT2  ]] =  [[ [OO]DD  ]]$ & By \Cref{lemma:confluence} \\
      & $[[ [OO]([TT1]aA)  ]] =  [[  [OO]aA  ]]$ & By \Cref{lemma:subst_ext_invar} \\
      & $[[ [OO]DD |- [OO]aA |> [OO]aA1 -> [OO]aA2     ]]$ & By above equalities \\
      & $[[ [OO]DD |- [OO]aA1 <~ [OO]aA1    ]]$ & By \cref{lemma:sub_refl} \\
      $\byhave$& $[[  [OO]DD |- e1' e2' : [OO]aA2   ]]$ & By \rref{app} \\
      $\byhave$& $\erase{e_1' ~ e_2'} = \erase{e_1'} ~ \erase{e_2'} = \erase{e_1} ~ \erase{e_2} = \erase{e_1 ~ e_2}$ & By def. of erasure
    \end{longtable}


  \item Case \[  \drule{inf-anno}  \]
    \begin{longtable}[l]{ll|l}
      & $[[  GG |- ae <= aA -| DD ]]$ & Premise \\
      $\byhave$ & $ [[  [OO]DD |- e' : [OO]aA  ]]  $ & By i.h., \\
      & $\erase{[[e]]} = \erase{[[e']]}$ & Above \\
      $\byhave$ & $\erase{e : A} = \erase{e} = \erase{e'}$ & By above equality and the def. of erasure
    \end{longtable}



  \item Case \[ \drule{chk-gen}  \]

    \begin{longtable}[l]{ll|l}
      & $[[ DD --> OO ]]$ & Given \\
      & $[[ DD , a --> OO , a]]$ & By def \\
      & $[[ GG , a --> DD , a, TT  ]]$ & By \cref{lemma:typing_extension} \\
      & $[[TT]]$ is soft & By \cref{lemma:extension_order} \\
      & $\underbrace{[[DD, a, TT]]}_{[[DD']]} [[-->]] \underbrace{[[OO, a, |TT|]]}_{[[OO']]}$ & By \cref{lemma:filling_completes} \\
      & $[[ GG, a |- ae <= aA -| DD , a , TT]]$  & Premise \\
      & $[[ [OO']DD' |- e' : [OO']aA  ]]$  & By i.h., \\
      $\byhave$& $\erase{[[ae]]} = \erase{[[e']]}$ & Above \\
      & $[[  [OO']aA  ]] = [[ [OO]aA ]]$ & By \cref{lemma:subst_stable} \\
      & $[[ [OO']DD'  ]]$ = $[[ [OO]DD, a  ]]$ & By \Cref{lemma:subst_go_away} and def. of context substitution \\
      & $[[ [OO]DD, a |- e' : [OO]aA  ]]$  & By above equalities \\
      & $[[  [OO]DD |- e' : \/a. [OO]aA  ]]$  & By \rref{gen} \\
      $\byhave$& $[[ [OO]DD |- e' : [OO] (\/a. aA)  ]]$  & By def. of substitution

    \end{longtable}


  \item Case \[ \drule{chk-lam}  \]

    \begin{longtable}[l]{ll|l}
      & $[[  DD --> OO  ]]$ & Given \\
      & $[[ DD, x : aA --> OO, x : [OO]aA   ]]$ & By def \\
      & $[[ GG , x : aA --> DD, x : aA, TT   ]]$ & By \cref{lemma:typing_extension} \\
      & $[[TT]]$ is soft & By \cref{lemma:extension_order} \\
      & $\underbrace{[[DD, x : aA, TT]]}_{[[DD']]} [[-->]] \underbrace{[[OO, x : [OO]aA, |TT| ]]}_{[[OO']]}$ & By \cref{lemma:filling_completes} \\
      & $[[ GG, x : aA |- ae <= aB -| DD, x : aA, TT]]$ & Premise \\
      & $[[ [OO']DD' |- e' : [OO']aB ]]$ & By i.h., \\
      & $\erase{[[ae]]} = \erase{[[e']]}$ & Above \\
      & $[[ [OO']aB  ]] = [[ [OO]aB  ]]$ & By \cref{lemma:subst_stable} \\
      & $[[ [OO']DD'  ]] = [[ [OO]DD, x : [OO]aA  ]] $ & By \Cref{lemma:subst_go_away} and def. of context substitution \\
      & $[[ [OO]DD, x : [OO]aA |- e' : [OO]aB ]]$ & By above equalities \\
      & $[[ [OO]DD |- \x : [OO]aA . e' : [OO]aA -> [OO]aB ]] $ & By \rref{lamann} \\
      $\byhave$ & $[[ [OO]DD |- \x : [OO]aA . e' : [OO](aA -> aB) ]]$ & By def. of substitution \\
      $\byhave$ & $\erase{\erlam x e} = \erlam x {\erase{e}} = \erlam x {\erase{e'}} = \erase{[[\x : [OO]aA . e']]}$ & By the def. of erasure
    \end{longtable}


  \item Case \[ \drule{chk-sub}  \]

    \begin{longtable}[l]{ll|l}
      & $[[ TT |- [TT]aA <~ [TT]aB -| DD]]$ & Premise \\
      & $[[TT --> DD]]$ & By \cref{lemma:sub_extension} \\
      & $[[DD -->OO ]]$ & Given \\
      & $[[TT --> OO]]$ & By \cref{lemma:transitivity} \\
      & $[[ GG |- ae => aA -| TT]]$ & Premise \\
      & $[[ [OO]TT |- e' : [OO]aA   ]]$ & By i.h., \\
      & $\erase{[[ae]]} = \erase{[[e']]}$ & Above \\
      & $[[ [OO]TT   ]] = [[ [OO]DD  ]]$ & By \cref{lemma:confluence} \\
      & $[[ [OO]DD |- e' : [OO]aA   ]] $ & By above equality \\
      & $[[ [OO]DD |- [OO]([TT]aA) <~ [OO]([TT]aB) ]] $ & By \Cref{thm:sub_soundness} \\
      & $[[ [OO]([TT]aA)  ]] = [[ [OO]aA ]]$ & By \cref{lemma:subst_ext_invar} \\
      & $[[ [OO]([TT]aB)  ]] = [[ [OO]aB ]]$ & By \cref{lemma:subst_ext_invar} \\
      & $[[ [OO]DD |- [OO]aA <~ [OO]aB ]]$ & By above equalities \\
      $\byhave$& $\ctxsubst{\Omega}{\Delta} \vdash (e' : [[ [OO]aB   ]]) : [[ [OO]aB   ]]$ & By def. annotation \\
      $\byhave$& $\erase{(e' : [[ [OO]aB  ]])} = \erase{e'} = \erase{e}$ & By def. erasure
    \end{longtable}



  \end{itemize}


\end{proof}

\newpage

\section{Completeness of Consistent Subtyping}

\instcomplete*
\begin{proof}
  By mutual induction on the given derivation.
  \begin{enumerate}
  \item We have $ [[  [OO]GG |- [OO]evar <~ [OO]aA  ]]   $. We case analyze the shape of $[[aA]]$.
    \begin{itemize}
    \item Case $[[aA]] = [[unknown]], [[evar = sa]]$:
      \begin{longtable}[l]{ll|l}
        & $[[  [OO]GG |- [OO]sa <~ [OO]unknown  ]]$ & Given \\
        & $[[ [OO]unknown  ]] = [[unknown]]$ \\
        & $[[  [OO]GG |- [OO]sa <~ unknown  ]]$ & By above equality \\
        & $[[sa]] \notin \textsc{unsolved}([[GG]])$ & Given \\
        & $[[ GG  ]] = [[ GL, sa, GR   ]]$ & Above \\
        & $[[ GL, sa, GR --> OO   ]]$ & Given \\
        & $[[ OO ]] = [[ OL, sa = atc, OR  ]]   $ & By \cref{lemma:extension_order} and $[[OO]]$ is complete and $[[ [OO]sa ]] \in [[agc]]$ \\
        & $[[ GL --> OL   ]]$ & Above \\
        & Let $[[DD]] = [[ GL, ga, sa = ga, GR ]]$ \\
        & and $[[OO']] = [[OL, ga = atc, sa = atc, OR]]$ \\
        $\byhave$& $[[ GG |- sa <~~ unknown -| DD    ]]$ & By \rref{instl-solveUS} \\
        $\byhave$& $\Delta \exto \Omega'$ & By \cref{lemma:paralell_admit,lemma:paralell_ext_solu} \\
        $\byhave$& $\Omega \exto \Omega'$ & By \cref{lemma:unsolved_ext,lemma:solution_ext}
      \end{longtable}
    \item Case $A = [[unknown]], [[evar = ga]]$:
      \begin{longtable}[l]{ll|l}
        & $[[  [OO]GG |- [OO]ga <~ [OO]unknown  ]]$ & Given \\
        & $[[ [OO]unknown  ]] = [[unknown]]$ \\
        & $[[  [OO]GG |- [OO]ga <~ unknown  ]]$ & By above equality \\
        & $[[ga]] \notin \textsc{unsolved}([[GG]])$ & Given \\
        & $[[ GG  ]] = [[ GG0[ga]   ]]$ & Above \\
        & Let $[[DD]] = [[ GG0[ga] ]]$ and $[[OO']] = [[OO]]$\\
        $\byhave$& $[[ GG |- ga <~~ unknown -| DD ]]$ & By \rref{instl-solveUG} \\
        $\byhave$& $[[ DD --> OO' ]]$ & Given \\
        $\byhave$& $[[ OO --> OO']]$ & By \cref{lemma:reflexivity}
      \end{longtable}

    \item Case $[[aA]] = [[gb]], [[evar]] = [[sa]]$:
      \begin{longtable}[l]{ll|l}
        & $[[ [OO]GG |- [OO]sa <~ [OO]gb  ]]$ & Given \\
        & $[[ [OO]GG |- t <~ tc   ]]$ & Let $[[ [OO]sa]] = [[t]]$ and $[[ [OO]gb  ]] = [[tc]]$ and $[[OO]]$ is predicative \\
        & $[[t]] = [[tc]]$ & By \cref{lemma:mono_equal} \\ \\
        & $[[ [GG]gb ]] = [[gb]]$ & Given \\
        & $[[gb]] \in \textsc{unsolved}{([[GG]])}$ & Above
      \end{longtable}
      Now consider whether $[[sa]]$ is declared to the left of $[[gb]]$.
      \begin{itemize}
      \item Case $[[GG]] = [[  GG0, sa, GG1, gb, GG2 ]]$
        \begin{longtable}[l]{ll|l}
          & Let $[[DD]] = [[ GG0, ga, sa = ga, GG1, gb = ga, GG2     ]]$ & \\
          $\byhave$& $[[  GG |- sa <~~ gb -| DD ]]$ & By \rref{instl-reachSG1} \\
          & $ [[GG --> OO]]   $ & Given \\
          & $[[OO]] = [[ OO0, sa = atc, OO1, gb = atc, OO2     ]]$ & By \cref{lemma:extension_order}   \\
          & Let $[[OO']] = [[ OO0, ga = atc, sa = atc, OO1, gb = atc, OO2     ]]$  \\
          $\byhave$& $[[OO --> OO']]$ & By \cref{lemma:unsolved_ext,lemma:solution_ext} \\
          $\byhave$& $[[DD --> OO' ]]$ & By \cref{lemma:paralell_admit,lemma:paralell_ext_solu}
        \end{longtable}
      \item Case $[[GG]] = [[  GG0, gb, GG1, sa, GG2 ]]$
        \begin{longtable}[l]{ll|l}
          & Let $[[DD]] = [[  GG0, gb, GG1, sa = gb, GG2  ]]$ & \\
          $\byhave$& $ [[GG |- sa <~~ gb -| DD]] $ & By \rref{instl-reachOther} \\
          $\byhave$& $[[DD --> OO]]$ & By \Cref{lemma:paralell_ext_solu} \\
          $\byhave$& $[[OO --> OO]]$ & By \Cref{lemma:reflexivity}
        \end{longtable}
      \end{itemize}
    \item Case $[[aA]] = [[sb]]$ is similar to the above case.
    \item Case $A = a$:
      \begin{longtable}[l]{ll|l}
        &$[[  [OO]GG |- [OO]evar <~ [OO]a  ]]$& Given \\
        &$[[  [OO]GG |- [OO]evar <~ a  ]]$ & From $[[  [OO]a ]] = [[a]]$ \\
        & $[[ [OO]evar ]] = [[a]]$ & By inversion of \rref{cs-tvar} \\
        & $[[a]]$ is declared to the left of $[[evar]]$ in $[[OO]]$ & $[[OO]]$ is well-formed \\
        & $[[GG --> OO]]$ & Given \\
        & $[[a]]$ is declared to the left of $[[evar]]$ in $[[GG]]$ & By \Cref{lemma:reverse_preserve} \\
        & Let $[[GG]] = [[ GG0[a][evar]   ]]$ \\
        & Let $[[DD]] = [[ GG0[a][evar = a] ]]$ \\
        $\byhave$& $[[ GG |- evar <~~ a -| DD ]]$ & By \rref{instl-solveS} or \rref{instl-solveG} \\
        $\byhave$& $\Delta \exto \Omega$ & By \Cref{lemma:paralell_ext_solu} \\
        $\byhave$& $\Omega \exto \Omega$ & By \Cref{lemma:reflexivity}
      \end{longtable}
    \item Case $[[aA]] = [[aA1 -> aA2]]$:
      \begin{longtable}[l]{ll|l}
        & $[[  [OO]GG |- [OO]evar <~ [OO]aA1 -> [OO]aA2  ]]$ & Given \\
        & $[[ [OO]evar  ]] = [[ t1 -> t2 ]]$ & $\Omega$ is predicative \\
        & $[[  [OO]GG |- [OO]aA1 <~ t1  ]]$ & By inversion of \rref{cs-arrow} \\
        & $[[  [OO]GG |- t2 <~ [OO]aA2  ]]$ & Above \\
        & $[[GG]] = [[ GG0[evar] ]]$ & From $[[evar]] \in \textsc{unsolved}{([[GG]])}$ \\
        & $[[GG0[evar] ]] [[-->]] \underbrace{[[  GG0[evar2, evar1, evar = evar1 -> evar2 ]   ]]}_{[[GG1]]}$ \\
        & $[[ GG --> OO ]]$ & Given \\
        & $[[OO]] = [[ OO0[evar = at0] ]]$ & From $\genA \in \textsc{unsolved}{(\Gamma)}$ \\
        & $[[ OO0[evar = at0] ]] [[-->]] \underbrace{[[ OO0[evar2 = at2, evar1 = at1, evar = evar1 -> evar2]  ]]}_{[[OO1]]}$ \\ \\
        & $[[ [OO]GG   ]] = [[ [OO1]GG1 ]]$ & By \Cref{lemma:finish_complete} \\
        & $[[ [OO]aA1  ]] = [[ [OO1]aA1  ]]$ & By \Cref{lemma:finish_types} \\
        & $[[t1]] = [[ [OO1]evar1 ]]$ & From def. of $[[OO1]]$ \\
        & $[[  [OO1]GG1 |- [OO1]aA1 <~ [OO1]evar1  ]]$ & By above equalities \\
        & $[[ GG1 |- aA1 <~~ evar1 -| DD2   ]]$ & By i.h. \\
        & $[[ DD2 --> OO2]]$ and $[[ OO1 --> OO2 ]]$ & Above \\ \\
        & $[[ [OO]GG   ]] = [[ [OO2]GG2 ]]$ & By \Cref{lemma:finish_complete} \\
        & $[[ [OO]aA2  ]] = [[ [OO2]aA2  ]] = [[ [OO2]([DD2]aA2)  ]]$ & By \Cref{lemma:finish_types} \\
        & $[[t2]] = [[ [OO2]evar2 ]]$ & By  $[[OO1 --> OO2]]$ \\
        & $[[ [OO2]DD2 |- [OO2]evar2 <~ [OO2]([DD2]aA2)   ]]$ & By above equalities \\
        & $[[  DD2 |-  evar2 <~~ [DD2]aA2 -| DD ]]$ & By i.h. \\
        & $[[ OO2 --> OO' ]]$ & Above \\
        $\byhave$& $[[ DD --> OO'   ]]$ & Above \\
        $\byhave$& $[[ GG0[evar] |- evar <~~ aA1 -> aA2 -| DD  ]]$ & By \rref{instl-arr} \\
        $\byhave$& $[[OO --> OO' ]]$ & By \Cref{lemma:transitivity}
      \end{longtable}
    \item Case $[[A]] = [[int]]$:
      \begin{longtable}[l]{ll|l}
        & $[[  [OO]GG |- [OO]evar <~ [OO]int   ]]$ & Given \\
        & $[[ [OO]int  ]] = [[int]]$ \\
        & $[[  [OO]GG |- [OO]evar <~ int   ]]$ & By above equality \\
        & $[[ [OO]evar ]] = [[int]]$ & $\Omega$ is predicative \\
        & $[[evar]] \in \textsc{unsolved}{(\Gamma)}$ & Given \\
        & $[[GG]] = [[GG0[evar] ]]$ & Above \\
        & Let $[[DD]] = [[ GG0[evar = int]  ]]$ and $[[OO']] = [[OO]]$\\
        & $[[ GG0[evar] |- evar <~~ int -| DD  ]]$ & By \rref{instl-solveS} or \rref{instl-solveG} \\
        & $[[GG --> OO]]$ & Given \\
        & $[[  GG0[evar = int] --> OO  ]]$ & By \Cref{lemma:paralell_ext_solu}
      \end{longtable}
    \item Case $[[aA]] = [[  \/b . aB  ]]$:
      \begin{longtable}[l]{ll|l}
        & $[[ [OO]GG |- [OO]evar <~ \/b. [OO]aB  ]]$ & Given \\
        & $[[ [OO]evar ]]$ cannot be a quantifier & $[[OO]]$ is predicative \\
        & $[[ [OO]GG, b |- [OO]evar <~ [OO]aB  ]] $ & By inversion of \rref{cs-forallR} \\ \\
        & $[[ [OO]GG, b   ]] = [[   [OO, b](GG, b)  ]]$ & By def. of context substitution \\
        & $[[ [OO]evar  ]] = [[  [OO,b]evar ]]$ & By def. of substitution \\
        & $[[ [OO]aB  ]] = [[  [OO, b]aB  ]]$ & By def. of substitution \\
        & $ [[ [OO, b](GG, b) |- [OO, b]evar <~ [OO, b]aB  ]]  $ & By above equalities \\
        & $ [[GG, b |- evar <~~ aB -| DD0 ]]  $ & By i.h. \\
        & $[[ DD0 --> OO' ]]$ & Above \\
        & $[[  OO, b --> OO'  ]]$ & Above \\
        $\byhave$& $[[  OO --> OO'  ]]$ & By \cref{lemma:drop_ext}\\  \\
        & $[[ GG, b --> DD0 ]]$ & By \Cref{lemma:inst_extension} \\
        & $[[DD0]] = [[DD, b, DD']]$ & By \Cref{lemma:extension_order} \\
        & $[[  GG --> DD ]]$ & Above \\
        $\byhave$& $[[  DD --> OO'  ]]$ \\
        $\byhave$& $ [[ GG |- evar <~~ \/b. aB -| DD  ]]  $ & By \rref{instl-forallR}
      \end{longtable}
    \end{itemize}
  \item Now we have $[[ [OO]GG |- [OO]aA <~ [OO]evar ]]$. These cases are mostly symmetric. The one exception is when $[[ aA  ]] = [[  \/b. aB ]]$.
    \begin{itemize}
    \item Case $[[aA]] = [[\/b . aB]]$:
      \begin{longtable}[l]{ll}
        & $[[ [OO]GG |- \/b . [OO]aB <~ [OO]evar   ]]$ \qquad Given \\
        & $[[ [OO]evar  ]]$ cannot be a quantifier \qquad $[[OO]]$ is predicative \\
        & $[[  [OO]GG |- t  ]]$ \qquad By inversion of \rref{cs-forallL} \\
        & $[[ [OO]GG |- ([OO]aB) [b ~> t] <~ [OO]evar    ]]$ \qquad Above \\ \\
        & $[[ [OO]GG ]] = [[  [OO, msb, sb = at] (GG, msb, sb)  ]]$ \qquad By def. of context application \\
        & $[[ ([OO]B) [b ~> t] ]] = [[  [OO, msb, sb = at] (aB [b ~> sb])  ]]$ \qquad by def. of substitution \\
        & $[[  [OO]evar  ]] = [[ [OO, msb, sb = at]evar  ]]$ \qquad By def. of substitution \\
        & $[[   [OO, msb, sb = at] (GG, msb, sb)   |-   [OO, msb, sb = at] (aB [b ~> sb])   <~ [OO, sb = at ]evar    ]]$ \qquad By above equalities \\
        & $[[ GG, msb, sb |- aB [b ~> sb] <~~ evar -| DD   ]]$ \qquad By i.h. \\
        &$[[GG, msb, sb --> DD]]$ \qquad By \cref{lemma:inst_extension} \\
        & $[[DD]] = [[DL, msb, DR]]$ \qquad By \cref{lemma:extension_order} \\
        & $[[ GG --> DL ]]$ \qquad Above \\
        & $[[ OO, msb, sb = at --> OO']]$ \qquad Above \\
        & $[[OO']] = [[OL, msb, OR]]$ \qquad By \cref{lemma:extension_order} \\
        $\byhave$& $[[ OO --> OL ]]$ \qquad Above \\
        $\byhave$& $[[ DL --> OL ]]$ \qquad \cref{lemma:transitivity} \\
        $\byhave$ & $[[ GG |- \/b . aB <~~ evar -| DL  ]]$ \qquad By \rref{instr-forallLL}
      \end{longtable}
    \end{itemize}
  \end{enumerate}
\end{proof}


\begin{landscape}
\begin{table}
  \centering
  \begin{footnotesize}
\begin{tabular}{cccccccccc} \toprule
&                 & \multicolumn{8}{c}{ $[[ [GG]aB  ]]$  }  \\
&                 & $\forall b. B'$  & $\int$      & $a$         & $\genB$     & $\unknown$          & $B_1 \to B_2$ & $[[static]]$ & $[[gradual]]$ \\ \cmidrule{3-10}
   \multirow{6}{*}{$\ctxsubst{\Gamma}{A}$} & $\forall a. A'$ & \hl{1 (B poly)}  & \hl{2.Poly} & \hl{2.Poly} & \hl{2.Poly} & \hl{1 (B unknown)}         &  \hl{2.Poly} & \hl{2.Poly}  & \hl{2.Poly}  \\ \cmidrule{3-10}
& $\int$          & \hl{1 (B poly)}  &  \hl{2.Ints}           &  \textit{Impossible} & \hl{2.BEx.Int}            & \hl{1 (B unknown)}  &  \textit{Impossible} &  \textit{Impossible} &  \textit{Impossible} \\ \cmidrule{3-10}
& $a$             &  \hl{1 (B poly)} & \textit{Impossible} &  \hl{2.UVars}           &  \hl{2.BEx.UVar}           &  \hl{1 (B unknown)} &  \textit{Impossible}  &  \textit{Impossible} &  \textit{Impossible} \\ \cmidrule{3-10}
& \multirow{2}{*}{$\genA$}         & \multirow{2}{*}{\hl{1 (B poly)}}   & \multirow{2}{*}{\hl{2.AEx.Int}} & \multirow{2}{*}{\hl{2.AEx.UVar}} & \hl{2.AEx.SameEx}  & \multirow{2}{*}{\hl{1 (B unknown)}}  & \multirow{2}{*}{\hl{2.AEx.Arrow}} & \multirow{2}{*}{\hl{2.AEx.S}} & \multirow{2}{*}{\hl{2.AEx.G}}  \\
& & & & & \hl{2.AEx.OtherEx} &   &   \\ \cmidrule{3-10}
& $\unknown$      & \hl{1 (B poly)}  & \hl{2.Unknown} &  \hl{2.Unknown}  &  \hl{2.Unknown} & \hl{1 (B unknown)}  & \hl{2.Unknown} & \textit{Impossible} & \hl{2.Unknown}  \\ \cmidrule{3-10}
  & $A_1 \to A_2$     & \hl{1 (B poly)}  & \textit{Impossible} &  \textit{Impossible}  & \hl{2.BEx.Arrow}  & \hl{1 (B unknown)}  & \hl{2.Arrows} &  \textit{Impossible} &  \textit{Impossible} \\ \cmidrule{3-10}
  & $[[static]]$     & \hl{1 (B poly)}  & \textit{Impossible} &  \textit{Impossible}  & \hl{2.BEx.S} & \textit{Impossible} & \textit{Impossible} & \hl{2.S} &  \textit{Impossible}  \\ \cmidrule{3-10}
  & $[[gradual]]$     & \hl{1 (B poly)}  & \textit{Impossible} &  \textit{Impossible}  & \hl{2.BEx.G} & \hl{1 (B unknown)} & \textit{Impossible} & \textit{Impossible} & \hl{2.G} \\ \bottomrule
\end{tabular}
  \end{footnotesize}
\caption{List of cases} \label{table:complete}
\end{table}
\end{landscape}

\subcomplete*
\begin{proof}

  By induction on the given declarative derivation. We list all the possible cases in \cref{table:complete}.

We first split on$\ctxsubst{\Gamma}{B}$.
  \begin{itemize}
  \item Case \hl{1 (B poly)}: $[[ [GG]aB  ]]$ is polymorphic: $[[  [GG]aB  ]] = [[ \/b. aB'   ]]$:
    \begin{longtable}[l]{ll|l}
      &$[[aB]] = [[ \/b . aB0]]   $& $[[GG]]$ is predicative \\
      & $[[aB']] = [[ [GG]aB0  ]]$ & $[[GG]]$ is predicative \\
      & $[[ [OO]aB   ]]  =  [[ \/b. [OO]aB0  ]]    $ & By def. of substitution \\
      & $[[  [OO]GG |- [OO]aA <~ [OO]aB   ]]$ & Premise \\
      & $[[  [OO]GG |- [OO]aA <~ \/b . [OO]aB0   ]]$ & By above equality \\
      & $[[  [OO]GG, b |- [OO]aA <~ [OO]aB0   ]]$ & By \Cref{lemma:forall_invert} \\
      & $[[ [OO]GG, b  ]] = [[ [OO, b] (GG, b)     ]]$ & By def. of substitution \\
      & $[[  [OO]aA  ]] = [[ [OO, b]aA  ]]$ & By def. of substitution \\
      & $[[  [OO]aB  ]] = [[ [OO, b]aB  ]]$ & By def. of substitution \\
      & $[[  [OO,b](GG, b) |- [OO,b]aA <~ [OO,b]aB0   ]]$ & By above equalities \\
      & $[[  GG, b |- [GG, b]aA <~ [GG, b]aB0 -| DD'  ]]$ & By i.h. \\
      & $[[  DD' --> OO0'  ]]$ & Above \\
      & $[[ OO, b --> OO0'   ]]$ & Above \\
      & $[[  GG, b |- [GG]aA <~ [GG]aB0 -| DD'  ]]$ & By def. of substitution \\ \\
      & $[[ GG, b --> DD'   ]]$ & By \Cref{lemma:inst_extension} \\
      & $[[ DD'  ]] = [[ DD, b, TT  ]]$ & By \Cref{lemma:extension_order} \\
      & $[[ GG --> DD   ]]$ & Above \\
      & $[[ DD, b, TT --> OO0'    ]]$ & By $[[DD' --> OO0']]$ and above equality \\
      & $[[ OO0'  ]] = [[ OO', b, OR  ]]$ & By \Cref{lemma:extension_order} \\
      $\byhave$& $[[ DD --> OO'  ]]$ & Above \\
      & $[[ OO, b --> OO', b , OR  ]]$ & By above equality \\
      $\byhave$& $[[ OO --> OO'  ]]$ & By \Cref{lemma:extension_order} \\ \\
      & $[[  GG, b |- [GG]aA <~ [GG]aB0 -| DD, b, TT  ]]$ & By above equality \\
      & $[[  GG |- [GG]aA <~ \/b. [GG]aB0 -| DD  ]]$ & By \rref{as-forallR} \\
      $\byhave$& $[[  GG |- [GG]aA <~ \/b. aB' -| DD  ]]$ & By above equality
    \end{longtable}

  \item Case \hl{1 (B unknown)}: $\ctxsubst{\Gamma}{B} = \unknown$:
        \begin{longtable}[l]{ll|l}
          & $[[ [OO]aB]] = [[unknown]]$ \\
          & $[[ [OO]GG |- [OO]aA <~ unknown  ]]$ & Given \\
          & $[[ [OO]aA ]] \in [[gc]] $ & Above \\
          &$[[ GG --> OO  ]]$& Given \\
          &$[[ [GG]aA  ]] \in [[agc]]  $ & Above \\
          & $[[  GG |- [GG]aA <~ unknown -| [ [GG]aA ] GG ]]$ & By \rref{as-unknownRR} \\
          & There exists $[[OO']]$ such that $[[ [ [GG]aA ] GG --> OO'    ]]$  and $[[  OO --> OO'  ]]$
        \end{longtable}

 \item Case 2.*: $[[ [GG]aB   ]]$ is not polymorphic. We split on the form of $[[  [OO]aA  ]]$.
    \begin{itemize}
    \item Case \hl{2.Poly}: $[[ [OO]aA  ]]$ is polymorphic: $[[ [GG]aA  ]] = [[ \/a . aA'  ]]$:
      \begin{longtable}[l]{ll|l}
        & $[[aA]] = [[\/a. aA0]]$& $[[GG]]$ is predicative \\
        & $[[aA']] = [[ [GG]aA0  ]]$ & $[[GG]]$ is predicative \\
        & $[[ [OO]aA  ]] = [[ \/a . [OO]aA0 ]]$ & By def. of substitution \\
        & $[[ [OO]GG |- [OO]aA <~ [OO]aB   ]]$ & Premise \\
        & $[[ [OO]GG |- \/a. [OO]aA0 <~ [OO]aB   ]]$ & By above equality \\
        & $[[ [OO]GG |- ([OO]aA0) [a ~> t] <~ [OO]aB   ]]$ & By inversion on \rref{cs-forallL} \\
        & $[[ [OO]GG |- t  ]]$ & Above \\ \\
        & $[[ [OO]GG  ]] = [[  [OO, evar = at](GG, evar)   ]]$ & By def. of substitution \\
        & $[[ ([OO]aA0) [a ~> t]  ]] = [[  [OO, evar = at](aA0 [ a ~> evar ])  ]]$ & By def. of substitution \\
        & $[[ [OO]aB ]] = [[ [OO, evar = at]aB   ]]$ & By def. of substitution \\
        & $[[ [OO, evar = at](GG, evar) |- [OO, evar = at](aA0 [ a ~> evar ]) <~ [OO, evar = at]aB   ]] $ & By above equalities \\
        & $[[  GG, evar |- [GG, evar] (aA0 [ a ~> evar ]) <~ [GG, evar]aB -| DD   ]]$ & By i.h. \\
        $\byhave$& $[[ DD --> OO'   ]]$ & Above \\
        & $[[  OO, evar = at --> OO'   ]]$ & Above \\
        $\byhave$& $[[ OO --> OO'  ]]$ & By \cref{lemma:drop_ext} \\
        & $[[ [GG, evar] (aA0 [ a ~> evar ])  ]] = [[  ([GG]aA0) [a ~> evar]   ]]$ & By def. of substitution \\
        & $[[ [GG, evar]aB   ]] = [[ [GG]aB   ]]$ & By def. of substitution \\
        & $[[  GG, evar |- ([GG]aA0) [a ~> evar] <~ [GG]aB -| DD   ]]$ & By above equality \\
        & $[[  GG |- \/a. ([GG]aA0) <~ [GG]aB -| DD   ]]$ & By \rref{as-forallLL} \\
        $\byhave$& $[[  GG |- \/a. aA' <~ [GG]aB -| DD   ]]$ & By above equality \\
      \end{longtable}

    \item Case \hl{2.Unknown}: $[[ [GG]aA  ]] = [[unknown]]$:
        \begin{longtable}[l]{ll|l}
          & $[[ [OO] aA]] = [[unknown]]$ & Obviously, what else? \\
          & $[[ [OO]GG |- unknown <~ [OO]aB  ]]$ & Given \\
          & $[[ [OO]aB ]] \in [[gc]] $ & Above \\
          &$[[ GG --> OO  ]]$& Given \\
          &$[[ [GG]aB  ]] \in [[agc]]  $ & Above \\
          & $[[  GG |- unknown <~ [GG]aB -| [ [GG]aB ] GG ]]$ & By \rref{as-unknownLL} \\
          & There exists $[[OO']]$ such that $[[ [ [GG]aB ] GG --> OO'    ]]$ and $[[  OO --> OO'  ]]$
        \end{longtable}

    \item Case \hl{2.AEx.*}: $[[ [GG]aA ]]$ is an existential variable: $[[ [GG]aA  ]] = [[evar]]$. We split on the form of $[[ [GG]aB  ]]$.
      \begin{itemize}
      \item Case \hl{2.AEx.SameEx}. $[[ [GG]aB ]]$ is the same existential variable $[[ [GG]aB  ]] = [[evar]]$:
        \begin{longtable}[l]{ll|l}
          &$[[  GG |- evar <~ evar -| GG   ]]$& By \rref{as-evar} \\
          $\byhave$& $[[  GG |- [GG]aA <~ [GG]aB -| GG   ]]$ & By above equality \\
          $\byhave$& $[[ DD --> OO   ]]$ & $[[DD]] = [[GG]]$ \\
          $\byhave$& $[[ OO --> OO'  ]]$ & By \Cref{lemma:reflexivity} and $[[OO']] = [[OO]]$
        \end{longtable}
      \item Case \hl{2.AEx.OtherEx}. $[[ [GG]aB  ]]$ is a different existential variable $[[ [GG]aB ]] = [[evarb]]$ where $[[evarb]] \neq [[evar]]$:
        \begin{longtable}[l]{ll|l}
          &$[[ [OO]aA  ]] = [[ [OO]([GG]aA)   ]] = [[ [OO]evar  ]]$& By \Cref{lemma:subst_ext_invar} \\
          &$[[ [OO]aB  ]] = [[ [OO]([GG]aB)   ]] = [[ [OO]evarb  ]]$& By \Cref{lemma:subst_ext_invar} \\
          & $ [[ [OO]GG |- [OO]aA <~ [OO]aB  ]]   $ & Given \\
          & $[[ [OO]GG |- [OO]evar <~ [OO]evarb  ]] $ & By above equalities \\
          & $[[ GG |- evar <~~ evarb -| DD   ]]$ & By \Cref{thm:inst_complete} \\
          $\byhave$& $[[ DD --> OO'  ]]$ & Above \\
          $\byhave$& $[[ OO --> OO'   ]]$ & Above \\
          & $[[ GG |- evar <~ evarb -| DD     ]]$ & By \rref{as-instL} \\
          $\byhave$& $[[ GG |- [GG]aA <~ [GG]aB -| DD     ]]$ & By above equalities
        \end{longtable}
      \item Case \hl{2.AEx.Int}. We have $[[ [GG]aB  ]] = [[int]]$:
        \begin{longtable}[l]{ll|l}
          &$[[ GG --> OO  ]]$& Given \\
          & $ [[ [OO] aB ]] = [[int]] = [[ [OO]int  ]]$ & By def. of substitution \\
          & $[[ [OO]aA  ]] = [[ [OO]([GG]aA)   ]] = [[ [OO]evar  ]]$ & By \Cref{lemma:subst_ext_invar} \\
          & $  [[ [OO]GG |- [OO]aA <~ [OO]aB    ]]  $ & Given \\
          & $[[  [OO]GG |- [OO]evar <~ [OO]int    ]]$ & By above equalities \\
          & $[[  GG |- evar <~~ int -| DD    ]]$ & By \Cref{thm:inst_complete} \\
          $\byhave$& $[[ DD --> OO'  ]]$ & Above \\
          $\byhave$& $[[ OO --> OO'  ]]$ & Above \\
          & $[[  GG |- evar <~ int -| DD    ]]$ & By \rref{as-instL} \\
          $\byhave$& $[[ GG |- [GG]aA <~ [GG]aB -| DD     ]]$ & By above equalities
        \end{longtable}
      \item Case \hl{2.AEx.UVar}. We have $[[ [GG]aB  ]] = [[b]]$. Similar to Case \hl{2.AEx.Int}.
      \item Case \hl{2.AEx.Arrow}. $[[ [GG]aB  ]] = [[ aB1 -> aB2   ]]$. We prove
        $[[evar]] \notin \textsc{fv}([[ [GG]aB  ]])$. Suppose for a
        contradiction, that $[[evar]] \in \textsc{fv}([[ [GG]aB ]])$, then
        $[[evar]]$ must be a subterm of $[[ [GG]aB  ]]$, so is
        $[[  [OO]evar  ]]$ a subterm of
        $[[ [OO]([GG]aB)     ]]$. The latter is equal to
        $[[ [OO]aB   ]]$, so $[[  [OO]evar  ]]$ is a subterm of
        $[[ [OO]aB   ]]$. Since $[[ [GG]aB  ]] = [[ aB1 -> aB2   ]]$, then
        $[[ [OO]aB   ]]$ must have the form $[[ aB1' -> aB2'  ]]$. Therefore
        $[[  [OO]evar   ]]$ must occur in either $[[aB1']]$ or $[[aB2']]$. But we
        have $[[ [OO]GG |- [OO]evar <~ [OO]aB       ]]$. That is , $[[  [OO]evar   ]]$
        cannot be a subterm of $[[ [OO]aB   ]]$. This is a contradiction.
        \begin{longtable}[l]{ll|l}
          &$[[evar]] \notin \textsc{fv}([[ [GG]aB  ]])$ & Proved above \\
          & $[[  GG --> OO  ]]$ & Given \\
          & $[[ [OO]aB   ]] = [[ [OO]([GG]aB)   ]]$ & By \Cref{lemma:subst_ext_invar} \\
          & $[[  [OO]GG |- [OO]evar <~ [OO]aB          ]]$ & Given \\
          & $[[  [OO]GG |- [OO]evar <~ [OO]([GG]aB)         ]]$ & By above equality \\
          & $[[ GG |- evar <~~ [GG]aB -| DD  ]]$ & By \Cref{thm:inst_complete} \\
          $\byhave$& $[[  DD --> OO' ]]$ & Above \\
          $\byhave$& $[[ OO --> OO' ]]$ & Above \\
          &$[[  GG |- evar <~ [GG]aB -| DD  ]]$ & By \rref{cs-instL} \\
          $\byhave$& $[[ GG |- [GG]aA <~ [GG]aB -| DD   ]]$ & By above equalities
        \end{longtable}
      \item Case \hl{2.AEx.S} and \hl{2.AEx.S}. Similar to Case \hl{2.AEx.Int}.
      \end{itemize}

    \item Case \hl{2.BEx.*}. $[[ [GG]aA  ]]$ is not polymorphic and
      $[[ [GG]aB   ]]$ is an existential variable: $[[ [GG]aB  ]] = [[evarb]]$. We split on the form of $[[ [GG]aA  ]]$.
      \begin{itemize}
      \item Case \hl{2.BEx.Int}. Similar to Case \hl{2.AEx.Unit}.
      \item Case \hl{2.BEx.UVar}. Similar to Case \hl{2.AEx.UVar}.
      \item Case \hl{2.BEx.Arrow}. Similar to Case \hl{2.AEx.Arrow}.
      \item Case \hl{2.BEx.S}. Similar to Case \hl{2.AEx.S}.
      \item Case \hl{2.BEx.G}. Similar to Case \hl{2.AEx.G}.
      \end{itemize}
      We use the second part of \Cref{thm:inst_complete} and apply \rref{as-instR}.

    \item Case \hl{2.Ints}. $[[ [GG]aA  ]] = [[ [GG]aB  ]] = [[int]] $:
      \begin{longtable}[l]{ll|l}
        $\byhave$&$[[  GG |- int <~ int -| GG ]]$& By \rref{as-int} \\
                 & $[[ GG --> OO   ]]$ & Given \\
        $\byhave$& $[[ DD --> OO'   ]]$ & $[[DD]] = [[GG]]$ \\
        $\byhave$& $[[ OO --> OO'  ]]$ & By \Cref{lemma:reflexivity} and $[[OO']] = [[OO]]$
      \end{longtable}

    \item Case \hl{2.UVars}. $[[ [GG]aA   ]] = [[ [GG]aB  ]] = [[a]]$:
      \begin{longtable}[l]{ll|l}
      $\byhave$&$[[ GG |-  a <~ a -| GG ]]$& By \rref{as-tvar} \\
                 & $[[ GG --> OO   ]]$ & Given \\
        $\byhave$& $[[ DD --> OO'   ]]$ & $[[DD]] = [[GG]]$ \\
        $\byhave$& $[[ OO --> OO'  ]]$ & By \Cref{lemma:reflexivity} and $[[OO']] = [[OO]]$
      \end{longtable}

    \item Case \hl{2.Arrows}. Let $[[ [GG]aA  ]] = [[ aA1 -> aA2  ]]$ and $[[ [GG]aB  ]] = [[ aB1 -> aB2   ]]$:
      \begin{longtable}[l]{ll|l}
        & $[[  GG --> OO  ]]$ & Given \\
        &$[[ [OO]aA  ]] = [[ [OO]([GG]aA)   ]] = [[ [OO]aA1 -> [OO]aA2  ]]$ & By \Cref{lemma:subst_ext_invar} \\
        &$[[ [OO]aB ]] = [[ [OO]([GG]aB)   ]] = [[ [OO]aB1 -> [OO]aB2  ]]$ & By \Cref{lemma:subst_ext_invar} \\
        & $[[ [OO]GG |- [OO]aA <~ [OO]aB   ]]$ & Given \\
        & $[[ [OO]GG |- [OO]aB1 <~ [OO]aA1   ]]$ & Premise \\
        & $[[ GG |- [GG]aB1 <~ [GG]aA1 -| TT  ]]$ & By i.h. \\
        & $[[ TT --> OO0  ]]$ & Above \\
        & $[[ OO --> OO0  ]]$ & Above \\
        & $[[ GG --> OO0  ]]$ & By \Cref{lemma:transitivity} \\ \\
        & $[[ [OO]GG   ]] = [[  [OO0]TT  ]]$ & By \Cref{lemma:confluence} \\
        & $[[ [OO]aA2   ]] = [[ [OO0]([GG]aA2)   ]]$ & By \Cref{lemma:subst_ext_invar} \\
        & $[[ [OO]aB2   ]] = [[ [OO0]([GG]aB2)   ]]$ & By \Cref{lemma:subst_ext_invar} \\
        & $[[ [OO]GG |- [OO]aA2 <~ [OO]aB2    ]]$ & Premise \\
        & $[[ [OO0]TT |- [OO0]([GG]aA2) <~ [OO0]([GG]aB2)    ]] $ & By above equalities \\
        & $[[ TT |- [TT]([GG]aA2) <~ [TT]([GG]aB2) -| DD   ]]$ & By i.h. \\
        $\byhave$& $[[ DD --> OO'   ]]$ & Above \\
        & $[[ OO0 --> OO'    ]]$ & Above \\ \\
        $\byhave$& $[[  GG |- [GG](aA1 -> aA2) <~ [GG](aB1 -> aB2) -| DD    ]]$ & By \rref{as-arrow} \\
        $\byhave$& $[[ OO --> OO'  ]]$ & By \Cref{lemma:transitivity}
      \end{longtable}

    \item Case \hl{2.S}: $[[ [GG]aA  ]] = [[ [GG]aB ]] = [[static]]$.
      \begin{longtable}[l]{ll|l}
        $\byhave$&$[[  GG |- static <~ static -| GG ]]$& By \rref{as-spar} \\
                 & $[[ GG --> OO   ]]$ & Given \\
        $\byhave$& $[[ DD --> OO'   ]]$ & $[[DD]] = [[GG]]$ \\
        $\byhave$& $[[ OO --> OO'  ]]$ & By \Cref{lemma:reflexivity} and $[[OO']] = [[OO]]$
      \end{longtable}
    \item Case \hl{2.G}. Similar to Case \hl{2.S}.

    \end{itemize}
  \end{itemize}
\end{proof}


\newpage

\section{Completeness of Typing}


\matchcomplete*
\begin{proof}
 By induction on the given derivation. We split on $[[ [GG]aA ]]   $.

 \begin{itemize}
 \item $[[  [GG]aA  ]] = [[ \/a. aA'   ]]$:
    \begin{longtable}[l]{ll|l}
      &$[[aA]] = [[ \/a . aA0]]   $& $[[GG]]$ is predicative \\
      & $[[aA']] = [[ [GG]aA0  ]]$ & $[[GG]]$ is predicative \\
      & $[[ [OO]aA   ]]  =  [[ \/a. [OO]aA0  ]]    $ & By def. of substitution \\
      & $[[  [OO]GG |- [OO]aA |> A1 -> A2  ]]$ & Given \\
      & $[[  [OO]GG |- \/a. [OO]aA0 |> A1 -> A2  ]]$ & By above equality \\
      & $[[  [OO]GG |- ([OO]aA0) [a ~> t]  |> A1 -> A2  ]]$ & By inversion \\
      & $[[  [OO]GG |- t   ]]$ & Above \\
      & $[[ GG --> OO   ]]$ & Given \\
      & $[[ GG, sa --> OO, sa   ]]$ & By def. of context extension \\ \\

      & $[[ [OO]GG  ]] = [[ [OO, sa = at](GG, sa)    ]]$ & By def. of context application \\
      & $[[ ([OO]aA0) [a ~> t]   ]] =  [[  [OO, sa = at]( aA0 [a ~> sa] )   ]]     $ & By def. of substitution \\
      & $[[  [OO, sa = at](GG, sa) |- [OO, sa = at]( aA0 [a ~> sa] )  |> A1 -> A2  ]]$ & By above equalities \\
      & $[[  GG, sa |- [GG, sa](aA0 [a ~> sa]) |> aA1' -> aA2' -| DD   ]]$ & By i.h. \\
      $\byhave$& $[[ DD --> OO'   ]]$ and $[[  OO, sa = at --> OO'    ]]$ & Above \\
      $\byhave$ & $[[A1]] = [[ [OO']aA1'  ]]$ and $[[A2 ]] = [[  [OO']aA2'  ]]$ & Above \\
      & $[[   [GG, sa](aA0 [a ~> sa])   ]] =  [[  ([GG]aA0) [a ~> sa]   ]]   $ & By def. of substitution \\
      & $[[   GG, sa |-   ([GG]aA0) [a ~> sa]   |> aA1' -> aA2' -| DD   ]]$ & By above equality \\
      & $[[ GG |- \/a . [GG]aA0 |> aA1' -> aA2' -| DD   ]]$ & By \rref{am-forallL} \\
      & $[[ [GG]aA ]] = [[ \/a. aA'  ]] = [[ \/a. [GG]aA0   ]]    $ & By above equalities \\
      $\byhave$ & $[[ GG |- [GG]aA |> aA1' -> aA2' -| DD  ]]$ & Above
    \end{longtable}



  \item   $[[  [GG]aA  ]] = [[ aA1' -> aA2'   ]]$:

    \begin{longtable}[l]{ll|l}
      & $[[ [OO]aA  ]] = [[ [OO]([GG]aA)  ]] = [[  [OO]aA1' -> [OO]aA2'  ]]$ & By \Cref{lemma:subst_ext_invar}   \\
      & $[[ [OO]GG |-  [OO]aA1' -> [OO]aA2' |> A1 -> A2  ]]$ & Given \\
      & Let $[[ DD  ]] = [[GG]]$ and $[[ OO' ]] = [[OO]]$ \\
      $\byhave$& $[[ [OO]aA1'   ]] = [[A1]]$ and $[[ [OO]aA2' ]] = [[A2]]$ \\
      $\byhave$ & $[[  GG |- aA1' -> aA2' |> aA1' -> aA2' -| GG  ]]$ & By \rref{am-arr} \\
      $\byhave$ & $[[  DD --> OO' ]]$ & Given $[[  GG --> OO  ]]$ \\
      $\byhave$ & $[[  OO --> OO' ]]$ & By \cref{lemma:reflexivity}

    \end{longtable}

  \item   $[[  [GG]aA  ]] = [[ unknown   ]]$:

    \begin{longtable}[l]{ll|l}
      & $[[ [OO]aA  ]] = [[ [OO]([GG]aA)  ]] = [[  unknown  ]]$ & By \Cref{lemma:subst_ext_invar}   \\
      & $[[ [OO]GG |-  unknown |> A1 -> A2  ]]$ & Given \\
      & Let $[[ DD  ]] = [[GG]]$ and $[[ OO' ]] = [[OO]]$ \\
      $\byhave$& $[[ A1   ]] = [[unknown]]$ and $[[ A2 ]] = [[unknown]]$ \\
      $\byhave$ & $[[  GG |- unknown |> unknown -> unknown -| GG  ]]$ & By \rref{am-unknown} \\
      $\byhave$ & $[[  DD --> OO' ]]$ & Given $[[  GG --> OO  ]]$ \\
      $\byhave$ & $[[  OO --> OO' ]]$ & By \cref{lemma:reflexivity}

    \end{longtable}

  \item   $[[  [GG]aA  ]] = [[ evar  ]]$:

    \begin{longtable}[l]{ll|l}
      & $[[  GG  ]] = [[ GG0[evar] ]]$ & Since $[[evar]] \in \textsc{unsolved}([[GG]]) $ \\
      & $[[ [OO]aA  ]] = [[ [OO]([GG]aA)  ]] = [[  [OO]evar  ]]$ & By \Cref{lemma:subst_ext_invar}   \\
      & $[[ [OO]GG |- [OO]evar |> A1 -> A2  ]]$ & Given \\
      & $[[ [OO]evar ]] = [[ t1 -> t2  ]]$ and $[[ A1 ]] = [[t1]]$ and $[[ A2 ]] = [[t2]]$ & $[[OO]]$ is predicate \\
      & $[[ OO ]] = [[  OO0[ evar = at' ]  ]]$ and $[[  [OO]at'   ]] = [[  t1 -> t2 ]]$ & Above \\
      & Let $[[DD]] = [[ GG0[evar1, evar2, evar = evar1 -> evar2]  ]]$ \\
      & Let $[[ OO'  ]] = [[ OO0[evar1 = at1, evar2 = at2, evar = evar1 -> evar2] ]] $ \\
      $\byhave$& $[[ DD --> OO'  ]]$ & By \Cref{lemma:paralell_admit} twice \\
      $\byhave$& $[[ OO --> OO'    ]]$ & By \Cref{lemma:paralell_ext_solu} and \Cref{lemma:paralell_admit} \\
      $\byhave$& $[[ GG0[evar] |- evar |> evar1 -> evar2 -| DD  ]]$ & By \rref{am-var} \\
      $\byhave$& $[[A1]] = [[t1]] = [[ [OO']evar1  ]]$ and $[[A2]] = [[t2]] = [[ [OO']evar2  ]]$ & Above
    \end{longtable}

 \end{itemize}


\end{proof}

\typingcomplete*
\begin{proof}
  By induction on the given derivation.
  \begin{itemize}
  \item Case \[     \ottaltinferrule{var}{}{ [[  (x : A) in [OO]GG ]] }{ [[ [OO]GG |- x : A]]  }  \]
    \begin{longtable}[l]{ll|l}
      & $ [[   (x : A) in [OO]GG  ]]  $  & Premise \\
      & $[[ GG --> OO  ]]$ & Given \\
      & $  [[ (x : aA') in GG  ]]  $ where $[[ [OO]aA'  ]] = [[  [OO]aA ]]$ & From def. of context application \\
      & Let $[[DD]] = [[GG]]$ and $[[OO']] = [[OO]]$. \\
      $\byhave$& $[[ GG --> OO ]]$ & Given \\
      $\byhave$& $[[ OO --> OO ]]$ & By \Cref{lemma:reflexivity} \\
      $\byhave$& $[[  GG |- x => aA' -| GG ]]$ & By \rref{inf-var} \\
      $\byhave$& $[[ [OO]aA' ]] = [[ [OO]aA ]] = [[A]]$ & A is well-formed in $[[  [OO]GG  ]]$ \\
      $\byhave$& $\erase{x} = \erase{x}$ & By def. of erasure
    \end{longtable}

  \item Case \[     \ottaltinferrule{int}{}{ }{ [[ [OO]GG |- n : int ]] }  \]

    \begin{longtable}[l]{ll|l}
      &Let $[[aA']] = [[int]]$ and $[[DD]] = [[GG]]$ and $[[OO']] = [[OO]]$. & \\
      $\byhave$& $[[ GG --> OO ]]$ & Given \\
      $\byhave$& $[[ OO --> OO ]]$ & By \Cref{lemma:reflexivity} \\
      $\byhave$& $[[ GG |- n => int -| GG ]]$ & By \rref{inf-int} \\
      $\byhave$& $[[ [OO]int ]] = [[int]]$ \\
      $\byhave$& $\erase{n} = \erase{n}$ & By def. of erasure
    \end{longtable}


  \item Case \[     \ottaltinferrule{lamann}{}{ [[ [OO]GG, x : A |- e : B  ]] }{ [[  [OO]GG |- \x : A . e : A -> B  ]] }  \]

    \begin{longtable}[l]{ll|l}
      & Let $[[OO0]] = [[ OO, x : aA]]$. \\
      & $[[ [OO0](GG, x : aA) ]] = [[ [OO]GG, x : aA  ]]$ & From def. of context application \\
      & $[[ [OO0](GG, x : aA) |- e : B   ]]$ & By above equality and premise \\
      & $[[ GG, x : aA |- ae' => aB0 -| DD0   ]]$ & By i.h. \\
      & $[[ DD0 --> OO' ]]$ & Above \\
      & $[[ OO0 --> OO' ]]$ & Above \\
      & $[[B]] = [[ [OO']aB0 ]]$ & Above \\
      & $\erase{e} = \erase{e'}$ & Above \\
      & $[[ GG, x : aA --> DD0  ]]$ & By \cref{lemma:typing_extension} \\
      & $[[ DD0 ]] = [[  DD1, x : aA' , DD2 ]]$ & By \cref{lemma:extension_order} \\
      & $[[ [DD1]aA  ]] = [[  [DD1] aA'  ]]$ & Above \\
      & $[[ GG --> DD1  ]] $ & Above \\
      & $[[ aA  ]] = [[  [DD1] aA'  ]]$ & $[[aA]]$ has no evar \\
      & $[[ GG, x : aA |- ae' => aB0 -| DD1, x : aA, DD2 ]]$ & By above equalities \\
      & $[[ GG, x : aA |- ae' <= aB0 -| DD1, x : aA, DD2 ]]$ & By \rref{chk-sub} \\ \\
      & $[[ DD1, x : aA', DD2 --> OO' ]]$ & By above equalities \\
      & $[[ OO' ]] = [[ OO1, x : aA'' , OO2 ]]$ & By \cref{lemma:extension_order} \\
      & $[[ [OO1]aA'  ]] = [[  [OO1] aA''  ]]$ & Above \\
      $\byhave$ & $[[ DD1 --> OO1  ]] $ & Above \\
      & $[[  OO, x : aA --> OO1, x : aA'', OO2 ]]$ & By above equalities \\
      $\byhave$& $[[  OO --> OO1  ]]$ & By \cref{lemma:extension_order} \\ \\

      & $[[  GG |- \x . ae' <= aA -> aB0 -| DD1  ]]$ & \rref{chk-lam} \\
      $\byhave$& $[[  GG |- (\x . ae') : aA -> aB0 => aA -> aB0 -| DD1  ]]$ & \rref{inf-anno} \\
      $\byhave$& $[[  [OO1](aA -> aB0)  ]] = [[ A -> [OO']aB0  ]] = [[ A -> B ]]$ & From above equality \\
      $\byhave$& $\erase{[[ \x : A . e  ]]} = \erlam{x}{\erase{e}} = \erlam{x}{\erase{e'}} = \erase{[[(\x . ae') : aA -> aB0]]}$ & By def. of erasure
    \end{longtable}

  \item Case \[     \ottaltinferrule{app}{}{ [[ [OO]GG |- e1 : A ]] \\ [[ [OO]GG |- A |> A1 -> A2]] \\ [[  [OO]GG |- e2 : A3  ]] \\ [[  [OO]GG |- A3 <~ A1  ]]  }{ [[  [OO]GG |- e1 e2 : A2  ]]  }  \]

    \begin{longtable}[l]{ll|l}
      & $[[  [OO]GG |- e1 : A  ]]$ & Premise \\
      & $[[ GG --> OO  ]]$ & Given \\
      & $[[ GG |- ae1' => aA' -| TT1  ]]$ & By i.h. \\
      & $[[TT1 --> OO0' ]]$ & Above \\
      & $[[ OO --> OO0'  ]]$ & Above \\
      & $[[A]] = [[ [OO0']aA'  ]]$ & Above \\
      & $\erase{[[e1]]} = \erase{[[ae1']]}$ & Above \\ \\

      & $[[ [OO]GG |- A |> A1 -> A2  ]]$ & Premise \\
      & $[[  [OO]GG  ]] = [[  [OO]OO  ]]$ & By \Cref{lemma:stable_complete_ctxt} \\
      & $ = [[ [OO0']OO0' ]]$ & By \Cref{lemma:finish_complete} \\
      & $ = [[ [OO0']GG   ]]$ & By \Cref{lemma:stable_complete_ctxt} \\
      & $ = [[ [OO0']TT1 ]]$ & By \Cref{lemma:confluence} \\
      & $[[ [OO0']TT1 |- [OO0']aA' |> A1 -> A2    ]]$ & By above equalities \\
      & $[[ TT1 |- [TT1]aA' |> aA1' -> aA2' -| TT2    ]]$ & By \Cref{thm:match_complete} \\
      & $[[ TT2 --> OO' ]]$ & Above \\
      & $[[ OO0' --> OO' ]]$ & Above \\
      & $[[A1]] = [[ [OO']aA1'  ]]$ & Above \\
      & $[[A2]] = [[ [OO']aA2'  ]]$ & Above \\ \\

      & $[[  [OO]GG |- e2 : A3   ]]$ & Premise \\
      & $ [[ [OO]GG  ]]   = [[  [OO]OO  ]]$ & By \Cref{lemma:stable_complete_ctxt} \\
      & $ = [[ [OO']OO'  ]]$ & By \Cref{lemma:finish_complete} \\
      & $ = [[ [OO']GG   ]]$ & By \Cref{lemma:stable_complete_ctxt} \\
      & $ = [[ [OO']TT2   ]]$ & By \Cref{lemma:confluence} \\
      & $[[ [OO']TT2 |- e2 : A3   ]]$ & By above equality \\
      & $[[  TT2 |- ae2' => aA3' -| TT3   ]]$ & By i.h. \\
      & $[[  TT3 --> OO1'   ]]$ & Above \\
      & $[[  OO' --> OO1'  ]]$ & Above \\
      & $[[A3]] = [[  [OO1']aA3'  ]]$ & Above \\
      & $\erase{[[e2]]} = \erase{[[ae2']]}$ & Above \\ \\

      & $[[ [OO]GG |- A3 <~ A1   ]]$ & Premise \\
      & $ [[ [OO]GG  ]]   = [[  [OO]OO  ]]$ & By \Cref{lemma:stable_complete_ctxt} \\
      & $ = [[ [OO1']OO1'  ]]$ & By \Cref{lemma:finish_complete} \\
      & $ = [[ [OO1']GG   ]]$ & By \Cref{lemma:stable_complete_ctxt} \\
      & $ = [[ [OO1']TT3   ]]$ & By \Cref{lemma:confluence} \\
      & $[[A3]] = [[  [OO1']aA3'  ]]$ & Above \\
      & $[[A1]] = [[ [OO']aA1'  ]] = [[ [OO1']aA1'  ]] $ & By \Cref{lemma:finish_types} \\
      & $[[ [OO1']TT3 |- [OO1']aA3' <~ [OO1']aA1'     ]]$ & By above equalities \\
      & $[[  TT3 |- [TT3]aA3' <~ [TT3]aA1' -| DD   ]]$ & By \Cref{thm:sub_completeness} \\
      & $[[ [TT3]aA1' ]] = [[ [TT3]([TT2]aA1)  ]]$ & By \cref{lemma:subst_ext_invar}\\
      & $[[  TT2 |- ae2' <= [TT2]aA1' -| DD  ]]$ & By \rref{chk-sub} \\
      $\byhave$& $[[  DD --> OO2'   ]]$ & Above \\
      & $[[ OO1' --> OO2'   ]]$ & Above \\
      $\byhave$& $[[  GG |- ae1' ae2' => aA2' -| DD  ]]$ & By \rref{inf-app} \\
      $\byhave$& $[[A2]] = [[ [OO']aA2' ]] = [[ [OO2']aA2'  ]]$ & \Cref{lemma:finish_types} \\
      $\byhave$& $[[  GG --> OO2' ]]$ & By \Cref{lemma:transitivity} \\
      $\byhave$& $\erase{[[e1 e2]]} = \erase{e_1} ~ \erase{e_2} = \erase{e_1'} ~ \erase{e_2'} = \erase{[[ ae1' ae2'  ]]}$ & By def. of erasure
    \end{longtable}


  \item Case \[     \ottaltinferrule{lam}{}{ [[ [OO]GG, x : t |- e : B  ]]  }{ [[  [OO]GG |- \x . e : t -> B  ]]   }  \]

    \begin{longtable}[l]{ll|l}
      & $[[ [OO]GG , x : t |- e : B   ]]$ & Given \\
      & $[[ [OO]GG, x : t  ]] = [[ [OO, x : at](GG , x : at)    ]]$ & By def. of context substitution \\
      & $[[ [OO, x : at](GG, x : at) |- e : B   ]]$ & By above equality \\
      & $[[ GG , x : at |- ae' => aB' -| DD'   ]]$ & By i.h., \\
      & $[[DD' --> OO']]$ & Above \\
      & $[[  OO, x : at --> OO'   ]]$ & Above \\
      & $[[B]] = [[ [OO']aB'  ]]$ & Above \\
      & $\erase{e} = \erase{e'}$ & Above \\
      & $[[ GG, x : at --> DD'   ]]$ & By \cref{lemma:typing_extension} \\
      & $[[DD']] = [[ DD, x : at, TT   ]]$ & By \cref{lemma:extension_order} \\
      & $[[ GG , x : at |- ae' => aB' -| DD, x : at, TT    ]]$ & By above equality \\
      $\byhave$& $[[  GG |- \x : at . ae' => at -> aB' -| DD   ]]$ & By \rref{inf-lamann} \\
      $\byhave$& $[[ DD --> OO'  ]]$ & By context extension \\
      $\byhave$& $[[ OO --> OO'   ]]$ & By context extension \\
      $\byhave$& $[[ t -> B  ]] = [[ at -> [OO']aB'   ]] = [[ [OO'](at -> aB')   ]]  $ & By def. of substitution \\
      $\byhave$& $\erase{[[\x . e]]} = \erlam{x}{\erase{e}} = \erlam{x}{\erase{e'}} = \erase{ [[\x : at . ae']] }$ & By def. of erasure
    \end{longtable}



  \item Case \[     \ottaltinferrule{gen}{}{ [[ [OO]GG, a |- e : A  ]]  }{ [[  [OO]GG |- e : \/a . A  ]]   }  \]

    \begin{longtable}[l]{ll|l}
      & $[[ [OO]GG, a |- e : A  ]]$ & Given \\
      & $[[ [OO]GG, a  ]] = [[  [OO, a](GG, a)   ]]$ & By def. of context substitution \\
      & $[[ [OO, a](GG, a) |- e : A    ]]$ & By above equality \\
      & $[[  GG, a |- ae' => aA' -| DD'  ]]$ & By i.h., \\
      & $[[ DD' --> OO'    ]]$ & Above \\
      & $[[ OO, a --> OO'   ]]$ & Above \\
      & $[[A]] = [[  [OO']aA'  ]]$ & Above \\
      $\byhave$& $\erase{e} = \erase{e'}$ & Above \\
      & $[[  GG, a --> DD'   ]]$ & By \cref{lemma:typing_extension} \\
      & $[[DD']] = [[DD, a, TT]]$ & By \cref{lemma:extension_order} \\
      $\byhave$& $[[  DD --> OO'   ]]$ & By context extension \\
      $\byhave$& $[[  OO --> OO'   ]]$ & By context extension \\
      & $[[  GG, a |- ae' => aA' -| DD, a , TT  ]]$ & By above equality \\
      & $ [[ DD, a, TT |- [DD, a, TT]aA' <~ [DD, a, TT]aA' -| DD, a, TT   ]]   $ & By reflexivity of consistent subtyping \\
      & $[[  GG, a |- ae' <= aA' -| DD, a, TT   ]]$ & By \rref{chk-sub} \\
      & $[[  GG |- ae' <= \/ a. aA' -| DD    ]]$ & By \rref{chk-gen} \\
      $\byhave$& $  [[  GG |- ae' : \/ a. aA' => \/ a. aA' -| DD    ]]  $ & By \rref{inf-anno} \\
      $\byhave$& $[[\/a. A]] = \forall a . \ctxsubst{\Omega'}{A'} = \ctxsubst{\Omega'}{([[\/a. aA']])}$ & By def. of substitution \\
    \end{longtable}



  \end{itemize}



\end{proof}



%%% Local Variables:
%%% mode: latex
%%% TeX-master: "../paper"
%%% org-ref-default-bibliography: ../paper.bib
%%% End:


%% -- The end --

\end{document}

%%% Local Variables:
%%% mode: latex
%%% TeX-master: t
%%% End:
