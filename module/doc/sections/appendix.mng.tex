\newcommand{\ruleform}[1]{\vspace{7mm} \leftline{\fbox{#1}} \vspace{3mm}}
\newcommand{\typeinfer}[3]{\[\inferrule*[Right=#1]{ #2 }{#3}\]}

\newcommand{\sigwrapper}[1]{\kw{sig}\,\{#1\}}
\newcommand{\modwrapper}[1]{\kw{mod}\,\{#1\}}
\newcommand{\trans}[1]{\colorbox{gray!30}{$\rightsquigarrow{} #1$}}
\newcommand{\coerf}[1]{\colorbox{orange!30}{$\hookrightarrow{} #1$}}
\newcommand{\transenv}[1]{\colorbox{cyan!30}{$ \mapsto #1$}}
\newcommand{\subst}[2]{[#1/#2]}
\newcommand{\letin}[3]{\kw{let}\, #1 = #2\,\kw{in}\,#3}
\newcommand{\open}[2]{\kw{open}\,#1\,\kw{in}\,#2}
\newcommand{\projk}{\mathsf{proj}_\mathsf{k}\,}
\newcommand{\intersec}[2]{#1\,\&\,#2}


\section{Source Language}

\subsection{Syntax}
% \gram{\ottes\ottinterrule{}
%       % \ottpp\ottinterrule{}
%       % \ottd\ottinterrule{}
%       % \ottb\ottinterrule{}
%       % \ottG\ottinterrule{}
%     }


% $T \{X_i \mapsto Y_i \mid X_i \in S \}$ means for each $X_i$ in the
% set $S$, substitute $X_i$ for $Y_i$ in $T$.


% \subsection{Syntax Sugar}

% \[
% \begin{array}{ll}
% % \kw{let}\,B\,\kw{in}\,E &\triangleq \modwrapper{B, x = E}.x \\
% E:T &\triangleq (\lambda X : T . X)\, E
% \end{array}
% \]

\subsection{Type System}

% Context Formation Rules
\ruleform{$ \vdash\Gamma $}

\jeremy{should check $X_1 \leq X_2 \coerf{f}$ in the enviroment?}

\typeinfer {Env-Empty} { } {\vdash\varnothing}

\typeinfer {Env-Var} {\vdash\Gamma \\ \Gamma\vdash T :\star}
{\vdash\Gamma,X:T \transenv{e}}

% \typeinfer {Env-Def} {\vdash\Gamma \\ \Gamma \vdash T:\star \\ \Gamma \vdash{}
%   E:T} {\vdash\Gamma,X:T=E \transenv{e}}

% \typeinfer {Env-Sub} {\vdash \Gamma \\ \Gamma \vdash X_1:\star \\ \Gamma \vdash
%   X_2:\star} {\vdash \Gamma, X_1 \leq X_2 \coerf{f}}


% Subtyping Rules
\ruleform{$\Gamma\vdash{}T_1\leq{}T_2 \coerf{c}$}

\jeremy{subtyping merge?}

\paragraph{Notes}

\begin{enumerate}
\item Unlike \cite{chen-2003-coerc}, the coercion functions in our
  calculus operate on translated types, which means coercive subtyping
  and translation are intimately tangled. This could potentially
  complicate the meta-theoretic study of the system.
\item Only well-typed terms have subtyping relations. This is
  guaranteed by the typing rules.
\item In the coercive subtyping relation, sometimes we write $\Gamma
  \vdash T \trans{t}$ to mean that $T$ is well-typed and translates to
  $t$.
\end{enumerate}

\typeinfer {S-Var} { \Gamma \vdash X_1:\star \\ \Gamma \vdash X_2:\star \\ X_1 \leq
  X_2 \coerf{c} \in \Gamma } { \Gamma \vdash X_1 \leq X_2 \coerf{c} }

\typeinfer {S-Eq} {\Gamma \vdash T_1 = T_2 \\ \Gamma \vdash T_1
  \trans{t}} {\Gamma \vdash T_1 \leq T_2 \coerf{\lambda x:t . x}}

% \typeinfer {S-Tran} {\Gamma \vdash T_1 \leq T_2 \coerf{f} \\ \Gamma
%   \vdash T_2 \leq T_3 \coerf{g}} {\Gamma \vdash T_1 \leq T_3 \coerf{g \circ f}}

\typeinfer {S-Pi} {\Gamma \vdash T_2 \leq T_1 \coerf{c_1} \\ \Gamma,
  X_1:T_1 \transenv{c_1 x}, X_2:T_2 \transenv{x}, X_1 \leq X_2
  \coerf{\lambda y:c_1 x .  \mathsf{cast}_{\downarrow} y} \\ \vdash
  T_1' \leq T_2' \coerf{c_2} \\
  \Gamma \vdash \Pi X_1:T_1.T'_1 \trans{t_1} \\ \Gamma \vdash T_2
  \trans{t_2}}
{\Gamma \vdash \Pi X_1:T_1.T'_1 \leq \Pi X_2:T_2.T'_2 \\
  \coerf{\lambda f:t_1 . \lambda x:t_2 . c_2(f(c_1\,x))}}

\typeinfer{S-Rec}{ \Gamma \vdash \mu X_1:\star . E_1 \trans{t_1} \\ \Gamma \vdash \mu
  X_2:\star . E_2 \trans{t_2} \\ \Gamma, X_1:\star \transenv{t_1}, X_2:\star \transenv{t_2}, X_1
  \leq X_2 \coerf{c} \vdash E_1 \leq E_2 \coerf{f} }{ \Gamma \vdash \mu X_1:\star . E_1 \leq \mu X_2:\star .
  E_2 \\ \coerf{\mu (c:t_1 \rightarrow t_2) . \lambda y:t_1 .
    \mathsf{cast}^{\uparrow}[t_2] f(\mathsf{cast}_{\downarrow} y)} }

\huang{S-Lam breaks the coercion property}

\typeinfer {S-Lam} { \Gamma \vdash T \trans{t} \\ \Gamma \vdash E \trans{e} \\
  \Gamma, X : T \transenv{x}; S \vdash A \leq B \coerf{c} }
{\Gamma; E,S \vdash \lambda X:T . A \leq \lambda X:T . B \coerf{(\lambda x:t . c) e}}

\typeinfer {S-App} {\Gamma;E,S \vdash A \leq B \coerf{c}} {\Gamma;S \vdash
  A E \leq B E \coerf{c}}

\typeinfer {S-App-E} {\Gamma \vdash A E \trans{\tau} \\ \Gamma \vdash B E
  \trans{u} \\ \Gamma \vdash A : \Pi \overline{(X:T)}^n . \star \\ \Gamma;E \vdash
  A \leq B \coerf{c}} {\Gamma;\emptyset \vdash A E \leq B E \coerf{\lambda
    y:\tau . \mathsf{cast}^{\uparrow}_n [u] (c (\mathsf{cast}_{\downarrow}^n y))}}

\typeinfer {S-And} { \Gamma \vdash T_1 \trans{t_1} \\ \Gamma \vdash T_1 \leq T_2 \coerf{c_1} \\ \Gamma
  \vdash T_1 \leq T_3 \coerf{c_2} } { \Gamma \vdash T_1 \leq
  \intersec{T_2}{T_3} \coerf{\lambda x : t_1 . (c_1\,x, c_2\,x)}}

\typeinfer {S-And$_{\mathsf{k}}$} { \Gamma \vdash \intersec{T_1}{T_2}
  \trans{t} \\ \Gamma \vdash T_\mathsf{k} \leq T_3 \coerf{c} } {
  \Gamma \vdash \intersec{T_1}{T_2} \leq T_3 \coerf{\lambda x : t
    . c(\projk x)} }

% \typeinfer {S-Sig} {\Gamma \vdash T \trans{t} \\ \Gamma \vdash
%   \sigwrapper{X:T, \overline{D} } \trans{\sigma_1} \\ \Gamma \vdash
%   \sigwrapper{X:T',\overline{D'} } \trans{\sigma_2} \\ \Gamma \vdash T \leq
%   T' \coerf{f} \\ \Gamma,X:T \transenv{x} \vdash \sigwrapper{ \overline{D} } \leq
%   \sigwrapper{ \overline{D'} } \coerf{g} }
% {\Gamma \vdash \sigwrapper{X:T, \overline{D} } \leq \sigwrapper{X:T',
%     \overline{D'} } \\ \coerf{\lambda m:\sigma_1 . (f (\pi_1 m), (\lambda x:t.g) (\pi_1 m)
%     (\pi_2 m))\ \kw{as}\ \sigma_2} }

% \typeinfer {S-Thin} {\Gamma \vdash \sigwrapper{D, \overline{D_1} }
%   \trans{t} } {\Gamma \vdash \sigwrapper{D, \overline{D_1} } \leq
%   \sigwrapper{\overline{D_1}} \coerf{\lambda x:t . \pi_2 x}}

% \typeinfer {S-Proj} { ? } { ? }



\paragraph{Properties of Coercive Substyping}

\begin{description}
\item[{Reflexivity}] if $\Gamma \vdash T : \star$, then $\exists c$
  such that $\Gamma \vdash T \leq T \hookrightarrow c$
\item[{Transitivity}] if $\Gamma \vdash T_1 : \star$ and $\Gamma
  \vdash T_2 : \star$ and $\Gamma \vdash T_3 : \star$ and $\Gamma
  \vdash T_1 \leq T_2 \hookrightarrow c_1$ and $\Gamma \vdash T_2 \leq T_3
  \hookrightarrow c_2$, then $\exists c$ such that $\Gamma \vdash T_1
  \leq T_3 \hookrightarrow c$
\item[{Subtyping rules produce type-correct coercions}] if $\Gamma
  \vdash T_1 : \star \rightsquigarrow e_1$ and $\Gamma \vdash T_2 :
  \star \rightsquigarrow e_2$ and $\Gamma \vdash T_1 \leq T_2
  \hookrightarrow c$ and $\Gamma \rightsquigarrow \Gamma'$, then
  $\Gamma' \vdash c : e_1 \rightarrow e_2$
% \item[{Coherence}] if $\Gamma \vdash T_1 : \star$ and $\Gamma \vdash
%   T_2 : \star$ and $\Gamma \vdash T_1 \leq T_2 \hookrightarrow c_1$
%   and $\Gamma \vdash T_1 \leq T_2 \hookrightarrow c_2$, then $c_1
%   =_{\alpha\eta} c_2$
\end{description}



% Type Equality Rules
% \ruleform{$ \Gamma \vdash T_1 = T_2 $}

% \paragraph{Note}

% By restricting module projections only on variables, we avoid equating
% arbitrate module expressions (which, in general, is undecidable), thus
% paths are equated syntactically.

% \typeinfer {E-REFL} { } {\Gamma \vdash T = T }

% \typeinfer {E-SYM} {\Gamma \vdash T_2 = T_1} {\Gamma \vdash T_1 = T_2}

% \typeinfer {E-Tran} {\Gamma \vdash T_1 = T_2 \\ T_2 = T_3} {\Gamma \vdash T_1 = T_3}

% \typeinfer {E-Pi} {\Gamma \vdash T_1 = T_2 \\ \Gamma, X:T_1 \vdash
%   T'_1 = T'_2} {\Gamma \vdash \Pi X:T_1.T'_1 = \Pi X:T_2.T'_2}

% \typeinfer {E-Sig} {\Gamma \vdash D = D' \\ \Gamma, D \vdash \sigwrapper{\overline{D_1}} = \sigwrapper{ \overline{D'_1} } } {\Gamma \vdash
%   \sigwrapper{D, \overline{D_1} } = \sigwrapper{D', \overline{D'_1} }}

% \typeinfer {E-Lam} {\Gamma \vdash T_1 = T_2 \\ \Gamma, X:T_1 \vdash T'_1 = T'_2}
% {\Gamma \vdash \lambda X:T_1.T'_1 = \lambda X:T_2.T'_2}

% \typeinfer {E-Mu} {\Gamma \vdash T_1 = T_2 \\ \Gamma, X:T_1 \vdash T'_1 = T'_2} {\Gamma \vdash \mu X:T_1.T'_1 = \mu X:T_2.T'_2}

% \typeinfer {E-App} {\Gamma \vdash E_1 = E'_1 \\ \Gamma \vdash E_2 = E'_2} {\Gamma \vdash E_1\,E_2 = E'_1\,E'_2}

% \typeinfer {E-And} {\Gamma \vdash T_1 = T'_1 \\ \Gamma \vdash
%   T_2 = T'_2} {\Gamma \vdash \intersec{T_1}{T_2} = \intersec{T'_1}{T'_2}}

% \typeinfer {E-Merge} {\Gamma \vdash E_1 = E'_1 \\ \Gamma \vdash E_2 = E'_2} {\Gamma \vdash E_1,,E_2 = E'_1,,E'_2}


% \typeinfer {E-Def} {X : T = E \in \Gamma} {\Gamma \vdash X = E}


% Term Formation Rules
\ruleform{$ \Gamma \vdash  E:T \trans{e} $}

% \paragraph{Identifiers}

% In order to handle identifiers as simply as possible, we make the
% following assumptions. First, we assume the identifier $x$ in module
% language can be injectively mapped to the variable $x$ in the target
% language. Second, we assume that this mapping is applied implicitly,
% and thus we use module-language identifiers as if they were
% target-language variables.


\typeinfer {T-Ax} {\vdash\Gamma} {\Gamma\vdash\star:\star \trans{\star}}

\typeinfer {T-Var} {\vdash\Gamma \\ X:T\transenv{e} \in \Gamma} {\Gamma\vdash X:T \trans{e}}

\typeinfer {T-App} {\Gamma\vdash E_1: (\Pi X:T_2 . T_1) \trans{e_1} \\
  \Gamma\vdash E_2:T_2 \trans{e_2}} {\Gamma\vdash E_1 E_2:T_1 \subst{E_2}{X}
  \trans{e_1 e_2}}

\typeinfer {T-Lam} {\Gamma\vdash T_1:\star \trans{t_1} \\
  \Gamma, X:T_1 \transenv{x} \vdash E:T_2 \trans{e} \\
  \Gamma\vdash(\Pi X:T_1 . T_2):\star} {\Gamma\vdash(\lambda X:T_1 . E): (\Pi X:
  T_1 . T_2) \trans{\lambda x:t_1 . e}}

\typeinfer {T-Pi} {\Gamma \vdash T_1 : \star \trans{t_1} \\
  \Gamma, X:T_1 \transenv{x} \vdash T_2:\star \trans{t_2}} {\Gamma \vdash (\Pi X:T_1 . T_2) :
  \star \trans{\Pi x:t_1 . t_2}}

\typeinfer {T-Mu} {\Gamma,X:T \transenv{x} \vdash E:T \trans{e} \\ \Gamma \vdash
  T:\star \trans{t}} {\Gamma\vdash(\mu X:T . E) : T \trans{\mu x:t . e}}

% \typeinfer {T-SigUnit} { \vdash \Gamma } {\Gamma \vdash \sigwrapper{} : \star \trans{\kw{Unit}}}

% \typeinfer {T-Sig} {\Gamma,X:T\transenv{x} \vdash \sigwrapper{\overline{D}} :
%   \star \trans{e} \\ \Gamma \vdash T:\star \trans{t}} {\Gamma \vdash
%   \sigwrapper{X:T,\overline{D}} : \star \trans{\Sigma x:t . e}}

% \typeinfer {T-ModUnit} { } {\Gamma \vdash \modwrapper{ } : \sigwrapper{} \trans{\kw{unit}}}

% \typeinfer {T-Mod} {\Gamma \vdash E:T \trans{e_1} \\
%   \Gamma \vdash T:\star \trans{t_1} \\ \Gamma, X:T=E\transenv{e_1} \vdash
%   \modwrapper{\overline{B}} : \sigwrapper{\overline{D}} \trans{e_2} \\
%   \Gamma, X:T=E\transenv{x} \vdash \sigwrapper{\overline{D}} : \star \trans{t_2}}
%   {\Gamma \vdash \modwrapper{X\,[: T] = E,\overline{B}} :
%     \sigwrapper{X:T,\overline{D}} \\ \trans{(e_1, e_2)\ \kw{as}\ \Sigma x:t_1 . t_2}}

% \typeinfer {T-Seal} {\Gamma\vdash E:T \trans{e}} {\Gamma\vdash (E : T) : T \trans{e}}

% \typeinfer {T-Path} {\Gamma \vdash P : \sigwrapper{\overline{D_1}^n, X
%     : T, \overline{D_2}} \trans{e}} {\Gamma \vdash P.X : T \{Y \mapsto P.Y
%   \mid Y \in \overline{D_1}\} \trans{\underbrace{\pi_2 (\dots ( \pi_2}_n
%     (\pi_1 e)))}}

% \typeinfer {T-Let} {\Gamma\vdash E_1 : T_1 \trans{e_1} \\ \Gamma, x : T_1 = E_1 \vdash E_2 : T_2 \trans{e_2}}
% {\Gamma\vdash \letin{x}{E_1}{E_2} : T_2 \trans{\letin{x}{e_1}{e_2}}}


\typeinfer{T-And}{\Gamma \vdash T_1 : \star \trans{t_1} \\
  \Gamma \vdash T_2 : \star \trans{t_2}}{\Gamma \vdash
  \intersec{T_1}{T_2} : \star \trans{ t_1 \times t_2 }}

\typeinfer{T-Merge}{\Gamma \vdash E_1 : T_1 \trans{e_1} \\ \Gamma
  \vdash E_2 : T_2 \trans{e_2}}{\Gamma \vdash E_1,,E_2 :
  \intersec{T_1}{T_2} \trans{(e_1,e_2)}}

\typeinfer {T-Subs} {\Gamma\vdash E:T' \trans{e} \\ \Gamma \vdash T' \leq
  T \coerf{c}} {\Gamma \vdash E : T \trans{c\,e}}


\paragraph{Properties of Elaboration}

\begin{description}
\item[{Equivalence of elaborating and source typing}] if $\Gamma
  \vdash E : T \trans{e}$ then $\Gamma \vdash E : T$, and vice versa.
\item[{Elaboration produces well-typed terms}] if $\Gamma \vdash E : T
  \trans{e}$, $\Gamma \vdash T : \star \trans{t}$, and the
  translations of $\Gamma$ is $\Gamma'$ \jeremy{formalize?}, then
  $\Gamma' \vdash e : t$.

\end{description}



% \subsection{Typing Rules (Algorithmic)}

% \ruleform{$ \Gamma \vdash  E:T \trans{e} $}

% \paragraph{Note}

% Omit unaffected rules.

% % \typeinfer {T-Ax} { } { }

% % \typeinfer {T-Var} { } { }

% % \typeinfer {T-App} { } { }

% % \typeinfer {T-Lam} { } { }

% % \typeinfer {T-Pi} { } { }

% % \typeinfer {T-Mu} { } { }

% % \typeinfer {T-SigUnit} { } { }

% % \typeinfer {T-Sig} { } { }

% % \typeinfer {T-ModUnit} { } { }

% \typeinfer {T-Mod} {\Gamma \vdash E:T' \trans{e_1} \\ \Gamma \vdash T'
%   \leq T \coerf{f} \\ \Gamma , x:T=E \vdash \modwrapper{\overline{B}} :
%   \sigwrapper{\overline{D}} \trans{e_2\ \kw{as}\ t_2}} { \Gamma \vdash
% \modwrapper{x:T=E, \overline{B}} : \sigwrapper{x:T, \overline{D}} \\
% \trans{(f e_1, e_2\ \kw{as}\ t_2 [x \mapsto f e_1])\ \kw{as}\ \Sigma x:t_1 . t_2} }

% % \typeinfer {T-Seal} {\Gamma \vdash E:T' \trans{e} \\ \Gamma \vdash T' \leq T \coerf{f}}
% % {\Gamma \vdash (E : T) : T \trans{f e}}

% % \typeinfer {T-Path} { } { }

% % \typeinfer {T-Let} { } { }


\newpage{}

\section{Target Language}

\subsection{Syntax}
% \gram{\ottet{}}

\subsection{Type System}

\ruleform{$ \Gamma\vdash{}e:\tau $}
\typeinfer {T-CastUp} {\Gamma\vdash{}e:\tau_2 \\ \Gamma\vdash\tau_1:\star
\\ \tau_1\longrightarrow\tau_2} {\Gamma\vdash(\mathsf{cast}^{\uparrow}[\tau_1]e):\tau_1}

\typeinfer {T-CastDown} {\Gamma\vdash{}e:\tau_1 \\
  \Gamma\vdash\tau_2:\star \\ \tau_1\longrightarrow\tau_2} {\Gamma\vdash(\mathsf{cast}_{\downarrow}e):\tau_2}

% \typeinfer {T-Sum} {\Gamma\vdash\tau:\star \\
%   \Gamma,x:\tau\vdash{}\sigma:\star} {\Gamma\vdash(\Sigma{}x:\tau.\sigma):\star}

\typeinfer {T-Product} {\Gamma \vdash \tau_1 : \star \\
  \Gamma \vdash \tau_2 : \star} {\Gamma \vdash \tau_1 * \tau_2 : \star}

% \typeinfer {T-Pair} {\Gamma \vdash e_1 : \tau \\  \Gamma\vdash{}e_2:\sigma[x\mapsto{}e_1]}
% {\Gamma\vdash(e_1,e_2)\ \kw{as} \ (\Sigma{}x:\tau.\sigma) : \Sigma{}x:\tau.\sigma}

\typeinfer {T-Pair} {\Gamma \vdash e_1 : \tau_1 \\  \Gamma \vdash
  e_2:\tau_2 }
{\Gamma \vdash(e_1,e_2) : \tau_1 * \tau_2}

% \typeinfer {T-Proj-1} {\Gamma\vdash{}e:\Sigma{}x:\tau.\sigma}
% {\Gamma\vdash\pi_1 e:\tau}

% \typeinfer {T-Proj-2} {\Gamma\vdash{}e:\Sigma{}x:\tau.\sigma}
% {\Gamma\vdash\pi_2 e:\sigma\subst{\pi_1 e}{x}}

\typeinfer {T-Proj} {\Gamma \vdash e: \tau_1 * \tau_2}
{\Gamma \vdash \projk e : \tau_\mathsf{k}}


\typeinfer {T-Let} {\Gamma\vdash e_1 : \tau_1 \\ \Gamma, x : \tau_1 = e_1 \vdash e_2 : \tau_2}
{\Gamma\vdash \letin{x}{e_1}{e_2} : \tau_2}


\newpage{}

\section{New Source}

\jeremy{2 sets of syntax, one annotation-free}

\subsection{Subtyping}

\ruleform{$ T_1 \leq T_2 \coerf{c} $}

\typeinfer{S-Var} { } { X \leq X \coerf{\lambda x.x} }

\typeinfer{S-Star} { } { \star \leq \star \coerf{\lambda x.x} }

\typeinfer{S-Pi} { A' \leq A \coerf{c_1} \\ B \leq B' \coerf{c_2} } { \Pi x:A .
  B \leq \Pi x:A' . B' \coerf{\lambda f. \lambda x . c_2 (f (c_1 x))} }

\typeinfer{S-Lam} { A \leq B \coerf{c} } { \lambda x . A \leq \lambda x . B
  \coerf{\lambda x . \mathsf{cast}^{\uparrow} (c (\mathsf{cast}_{\downarrow} x))} }

\typeinfer{S-App} { A \leq B \coerf{c} } { A E \leq B E \coerf{c} }

\typeinfer{S-CastUp} { A \leq B \coerf{c} } { \mathsf{cast}^{\uparrow} A \leq
  \mathsf{cast}^{\uparrow} B \coerf{c} }

\typeinfer{S-CastDown} { A \leq B \coerf{c} } { \mathsf{cast}_{\downarrow} A \leq
  \mathsf{cast}_{\downarrow} B \coerf{c} }

\typeinfer{S-Inter-1} { A \leq C \coerf{c} } { A \& B \leq C \coerf{\lambda x . c
  (\mathsf{proj}_1 x)} }

\typeinfer{S-Inter-2} { B \leq C \coerf{c} } { A \& B \leq C \coerf{\lambda x . c
  (\mathsf{proj}_2 x)} }

\typeinfer{S-Inter-3} { A \leq B \coerf{c_1} \\ A \leq C \coerf{c_2} } { A \leq
  B \& C \coerf{\lambda x . (c_1 x, c_2 x)} }

\typeinfer{S-Merge} { } { A ,, B \leq A ,, B \coerf{\lambda x.x} }

\typeinfer{S-Anno} { A \leq B \coerf{c} } { A :: T \leq B :: T \coerf{c} }

\subsection{Type Checking}

\newcommand{\gao}[1]{\colorbox{gray!30}{$\hookrightarrow{} #1$}}

\ruleform{$ \Gamma \vdash E:T \gao{ (E', e) } $}

\typeinfer{T-Star} { } { \Gamma \vdash \star : \star \gao{(\star, \star)} }

\typeinfer{T-Var} { X : T \in \Gamma } { \Gamma \vdash X : T \gao{(X, x)} }

\typeinfer{T-App} { \Gamma \vdash E_1 : \Pi X:T_1 . T_2 \gao{(E'_1, e_1)} \\
  \Gamma \vdash E_2 : T_1 \gao{(E'_2, e_2)} } { \Gamma \vdash E_1 E_2 : T_2 [X \mapsto
  E_2] \gao{(E'_1 E'_2, e_1 e_2)} }

\typeinfer{T-Pi} { \Gamma \vdash T_1 : \star \gao{(T'_1, t_1)} \\ \Gamma,
  X:T_1 \vdash T_2 : \star \gao{(T'_2, t_2)} } { \Gamma \vdash \Pi X:T_1 .
  T_2 : \star \gao{(\Pi X:T'_1 . T'_2, \Pi x : t_1 . t_2)} }

\typeinfer{T-Inter} { \Gamma \vdash T_1 : \star \gao{(T'_1, t_1)} \\ \Gamma
  \vdash T_2 : \star \gao{(T'_2, t_2)} } { \Gamma \vdash T_1 \& T_2 : \star \gao{(T'_1 \& T'_2, t_1 \times t_2)} }

\typeinfer{T-Merge} { \Gamma \vdash E_1 : T_1 \gao{(E'_1, e_1)} \\ \Gamma
  \vdash E_2 : T_2 \gao{(E'_2, e_2)} } { \Gamma \vdash E_1 ,, E_2 : T_1 \& T_2
  \gao{(E'_1 ,, E'_2, (e_1, e_2))} }

\typeinfer{T-CastUp} { \Gamma \vdash E : T_1 \gao{(E', e)} \\ \Gamma \vdash T_1 :
  \star \gao{(T'_1, t_1)} \\ \Gamma \vdash T_2 : \star \gao{(T'_2, t_2)} \\ t_2
  \longrightarrow t_1 } { \Gamma \vdash \mathsf{cast}^{\uparrow} E : T_2
  \gao{((\mathsf{cast}^{\uparrow} E') :: T'_2, \mathsf{cast}^{\uparrow}e)} }

\typeinfer{T-CastDown} { \Gamma \vdash E : T_1 \gao{(E', e)} \\ \Gamma \vdash T_1 :
  \star \gao{(T'_1, t_1)} \\ \Gamma \vdash T_2 : \star \gao{(T'_2, t_2)} \\ t_1
  \longrightarrow t_2 } { \Gamma \vdash \mathsf{cast}_{\downarrow} E : T_2
  \gao{((\mathsf{cast}_{\downarrow} E') :: T_2', \mathsf{cast}_{\downarrow}e)} }

\typeinfer{T-Lam} { \Gamma, X:T_1 \vdash E : T_2 \gao{(E',e)} \\ \Gamma \vdash \Pi
  X:T_1 . T_2 : \star \gao{(T, t)}} { \Gamma \vdash \lambda X.E : \Pi X:T_1 .
  T_2 \gao{((\lambda X.E') :: T, \lambda x. e)} }

\typeinfer{T-Sub} { \Gamma \vdash E : T \gao{(E', e)} \\ \Gamma \vdash T : \star
  \gao{(T'', t)} \\ T \leq T' \coerf{c}} {
  \Gamma \vdash E : T' \gao{(E' :: T'', c\,e)} }

\subsection{Bi-directional Type Checking}

\ruleform{$ \Gamma \vdash E \Rightarrow T \trans{e} $}

\typeinfer{T-Star} { } { \Gamma \vdash \star \Rightarrow \star \trans{\star} }

\typeinfer{T-Var} { X : T \in \Gamma } { \Gamma \vdash X \Rightarrow T
  \trans{x} }

\typeinfer{T-Anno} {  \Gamma \vdash
  E \Leftarrow T \trans{e} } { \Gamma \vdash E :: T \Rightarrow T \trans{e} }

\typeinfer{T-App} { \Gamma \vdash E_1 \Rightarrow \Pi X:T_1 . T_2 \trans{e_1} \\
  \Gamma \vdash E_2 \Leftarrow T_1 \trans{e_2} } { \Gamma \vdash E_1 E_2
  \Rightarrow T_2 [X \mapsto E_2] \trans{e_1 e_2} }

\typeinfer{T-Pi} { \Gamma \vdash T_1 \Leftarrow \star \trans{t_1} \\ \Gamma,
  X:T_1 \vdash T_2 \Leftarrow \star \trans{t_2} } { \Gamma \vdash \Pi X:T_1 .
  T_2 \Rightarrow \star \trans{\Pi x:t_1 . t_2} }

\typeinfer{T-Inter} { \Gamma \vdash T_1 \Leftarrow \star \trans{t_1} \\ \Gamma
  \vdash T_2 \Leftarrow \star \trans{t_2} } { \Gamma \vdash T_1 \& T_2
  \Rightarrow \star \trans{t_1 \times t_2} }

\typeinfer{T-Merge} { \Gamma \vdash E_1 \Rightarrow T_1 \trans{e_1} \\ \Gamma
  \vdash E_2 \Rightarrow T_2 \trans{e_2} } { \Gamma \vdash E_1 ,, E_2
  \Rightarrow T_1 \& T_2 \trans{(e_1, e_2)} }

\ruleform{$ \Gamma \vdash E \Leftarrow T \trans{e} $}

\typeinfer{T-CastUp} { \Gamma \vdash E \Rightarrow  T_1 \trans{e} \\ \Gamma \vdash T_1 \Leftarrow
  \star \trans{t_1} \\ \Gamma \vdash T_2 \Leftarrow \star \trans{t_2} \\ t_2
  \longrightarrow t_1 } { \Gamma \vdash \mathsf{cast}^{\uparrow} E \Leftarrow T_2
  \trans{\mathsf{cast}^{\uparrow} e} }

\typeinfer{T-CastDown} { \Gamma \vdash E \Rightarrow T_1 \trans{e} \\ \Gamma \vdash T_1 \Leftarrow
  \star \trans{t_1} \\ \Gamma \vdash T_2 \Leftarrow \star \trans{t_2} \\ t_1
  \longrightarrow t_2 } { \Gamma \vdash \mathsf{cast}_{\downarrow} E \Leftarrow T_2
  \trans{\mathsf{cast}_{\downarrow} e} }

\typeinfer{T-Lam} { \Gamma, X:T_1 \vdash E \Leftarrow T_2 \trans{e} } { \Gamma
  \vdash \lambda X.E \Leftarrow \Pi X:T_1 . T_2 \trans{\lambda x.e} }

\typeinfer{T-Sub} { \Gamma \vdash E \Rightarrow T' \trans{e} \\ T'
  \leq T \coerf{c} } { \Gamma \vdash E \Leftarrow T \trans{c\, e} }

% \huang{Already included in T-Sub?}
% \typeinfer{T-Check} { \Gamma \vdash E \Rightarrow T \trans{e} } { \Gamma \vdash
%   E \Leftarrow T \trans{e} }

\newpage{}

\section{Test}

\newpage{}

\section{New Subtyping + Nondependent Intersection}

\subsection{Subtyping}

\newcommand{\subty}[4]{#1 \triangleleft\ #2 \leq #3\ \triangleright #4}
\newcommand{\castdown}[1]{\mathsf{cast}_{\downarrow} #1}
\newcommand{\castup}[1]{\mathsf{cast}^{\uparrow} #1}
\newcommand{\proj}[1]{\mathsf{proj}_\mathsf{#1}}
\newcommand{\etaconv}[1]{\eta_{\downarrow}(#1)}
\newcommand{\optcast}[1]{\lceil #1 \rceil}
\newcommand{\optpair}[1]{\lfloor #1 \rfloor}

\ruleform{$ \etaconv{e} = e' $}

$$
\etaconv{e} =
\begin{cases}
f & e = \lambda x . f x \land x \notin fv(f) \\
e & otherwise
\end{cases}
$$

\ruleform{$ \optcast{e} = e' $}

$$
\optcast{e} =
\begin{cases}
e' & e = \castup \castdown e' \\
e & otherwise
\end{cases}
$$

\ruleform{$ \optpair{e} = e' $}

$$
\optpair{e} =
\begin{cases}
e' & e = (\proj{1} e', \proj{2} e') \\
e & otherwise
\end{cases}
$$

\ruleform{$ \subty{e}{T}{T'}{e'} $}

\typeinfer{S-Var} { } { \subty{e}{x}{x}{e} }

\typeinfer{S-Star} { } { \subty{e}{\star}{\star}{e} }

\typeinfer{S-Pi}
{ \subty{x}{A'}{A}{e_1} \\ \subty{e\ e_1}{B}{B'}{e_2} }
{ \subty{e}{\Pi y:A.B}{\Pi y:A'.B'}{\etaconv{\lambda x . e_2}} }

\typeinfer{S-Lam}
{ \subty{\castdown e}{A}{B}{e'} }
{ \subty{e}{\lambda x.A}{\lambda x.B}{\optcast{\castup e'}} }

\typeinfer{S-App}
{ \subty{e}{A}{B}{e'} }
{ \subty{e}{A\ E}{B\ E}{e'} }

\typeinfer{S-CastUp}
{ \subty{e}{A}{B}{e'} }
{ \subty{e}{\castup A}{\castup B}{e'} }

\typeinfer{S-CastDown}
{ \subty{e}{A}{B}{e'} }
{ \subty{e}{\castdown A}{\castdown B}{e'} }

\typeinfer{S-AndL1}
{ \subty{\proj{1} e}{A}{C}{e'} }
{ \subty{e}{A \& B}{C}{e'} }

\typeinfer{S-AndL2}
{ \subty{\proj{2} e}{B}{C}{e'} }
{ \subty{e}{A \& B}{C}{e'} }

\typeinfer{S-AndR}
{ \subty{e}{A}{B}{e_1} \\ \subty{e}{A}{C}{e_2} }
{ \subty{e}{A}{B \& C}{\optpair{(e_1, e_2)}} }