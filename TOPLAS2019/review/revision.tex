\documentclass[dvipsnames]{article}

\usepackage{parskip}
\usepackage{pdfpages}
\usepackage{listings}
\usepackage{amsmath}
\usepackage{amssymb}

\title{Revision Log}

\author{}


\usepackage[round]{natbib}
\usepackage[colorlinks=true, urlcolor=blue, citecolor=blue]{hyperref}
\usepackage{longtable}
\bibliographystyle{plainnat}

\newcommand\mynote[3]{\textcolor{#2}{#1: #3}}
\newcommand\bruno[1]{\mynote{Bruno}{red}{#1}}
\newcommand\tom[1]{\mynote{Tom}{blue}{#1}}
\newcommand\ningning[1]{\mynote{Ningning}{orange}{#1}}
\newcommand\jeremy[1]{\mynote{Jeremy}{gray}{#1}}
\newcommand\reply[1]{\textcolor{ForestGreen}{Reply: #1}}

% Code highlighting
\usepackage{listings}
\lstset{%
  basicstyle=\ttfamily\small, % the size of the fonts that are used for the code
  keywordstyle=\sffamily\bfseries,
  captionpos=none,
  columns=flexible,
  lineskip=-1pt,
  keepspaces=true,
  showspaces=false,               % show spaces adding particular underscores
  showstringspaces=false,         % underline spaces within strings
  showtabs=false,                 % show tabs within strings adding particular underscores
  breaklines=true,                % sets automatic line breaking
  breakatwhitespace=true,         % sets if automatic breaks should only happen at whitespace
  escapeinside={(*}{*)},
  sensitive=true
}

\lstdefinelanguage{myhaskell}{
  language=haskell,
  morekeywords = {def},
  commentstyle=\color{red}\textit,
  literate={->}{{$\rightarrow$}}1 {Top}{{$\top$}}1 {?}{{$\star$}}1 {=>}{{$\Rightarrow$}}1 {forall}{{$\forall$}}1 {/\\}{{$\Lambda$}}1,
  xleftmargin  = 3mm,
}

\lstset{language=myhaskell}

\newcommand{\lst}[1]{\text{\lstinline$#1$}}

% Table
\usepackage{multirow}
\usepackage{tabularx}
\newcolumntype{Y}{>{\centering\arraybackslash}X}
\newcolumntype{Z}{>{\raggedleft\arraybackslash}X}

% Infer rules
\usepackage{mathpartir}
\newcommand{\rname}[1]{{\,\text{\scriptsize \textsc{#1}}}}
\newcommand{\rul}[1]{\textsc{#1}}

% Extra symbols
\usepackage{stmaryrd}

% Macros for math typesetting

%% Names
% \newcommand{\name}{{\bf $\lambda_{\mu}^{\eq}$}\xspace}

%% Symbols
\newcommand{\syndef}{$::=$}
\newcommand{\synor}{$\mid$}
\newcommand{\syneq}{$\triangleq$}
\newcommand{\header}[1]{\multicolumn{1}{l}{$\boxed{#1}$}}
\newcommand{\headercap}[2]{\multicolumn{1}{l}{$\boxed{#1}$\quad{#2}}}
\newcommand{\headercapm}[2]{\vspace{1pt}\raggedright \framebox{\mbox{$#1$}} \quad
  #2}
\newcommand{\headercapt}[2]{\framebox{\mbox{$#1$}} \quad #2}
\newcommand{\marker}[1]{\blacktriangleright_{#1}}

%% Arrows
\newcommand{\To}{\Rightarrow}
\newcommand{\Chk}{\Downarrow}
\newcommand{\Inf}{\Uparrow}
\newcommand{\Inst}{{inst}}
\newcommand{\Gen}{{gen}}
\newcommand{\redto}{\hookrightarrow}
\newcommand{\redton}{\hookrightarrow^*}
\newcommand{\eq}{\sim}
\newcommand{\lt}{\sqsubseteq}
\newcommand{\sugar}{\triangleq}
\newcommand{\trto}[1]{\rightsquigarrow{#1}}
\newcommand{\opt}[1]{}
\newcommand{\trtop}{\rightsquigarrow}

%% Styles
\newcommand{\kw}[1]{\operatorname{\mathbf{#1}}}
\newcommand{\var}{\mathit}
\newcommand{\fun}{\mathsf}

%% Constructs
\newcommand{\bind}[3]{#1 #2:#3.~}
\newcommand{\blam}{\bind \lambda}
\newcommand{\bmu}{\bind \mu}
\newcommand{\barr}[2]{(#1:#2) \to}

\newcommand{\bindv}[4][]{#2\,\overline{#3:#4}^{#1}.~}
\newcommand{\blamv}[3][]{\bindv[#1] \lambda {#2} {#3}}
\newcommand{\bmuv}[3][]{\bindv[#1] \mu {#2} {#3}}
\newcommand{\barrv}[3][]{\overline{#2:#3}^{#1} \to}

\newcommand{\eqlam}[2]{\lambda_{\eq}({#1} \eq {#2}).~}
\newcommand{\eqty}[2]{({#1} \eq {#2})\Rightarrow}
\newcommand{\eqapp}[2][]{\langle {#2} \rangle^{#1}}
\newcommand{\eqlamv}[3][]{\lambda_{\eq}\overline{{#2} \eq {#3}}^{#1}.~}
\newcommand{\eqtyv}[3][]{\overline{({#2} \eq {#3})}^{#1}\Rightarrow}

\newcommand{\bpi}{\bind \Pi}
\newcommand{\bpiv}[3][]{\bindv[#1] \Pi {#2} {#3}}

\newcommand{\fold}{\fun{fold}}
\newcommand{\unfold}{\fun{unfold}}

\newcommand{\cast}[2]{\langle #1 \hookrightarrow #2 \rangle ~}
\newcommand{\blame}[2]{\kw{blame}_{#1} {#2}}
\newcommand{\subst}[2]{[#1 \mapsto #2]}
\newcommand{\ctxsubst}[2]{[#1]#2}
\newcommand{\Subst}[2]{[#1 \Mapsto #2]}
\newcommand{\substv}[3][]{\overline{[#2 \mapsto #3]}^{#1}}

\newcommand{\triv}{\_}
\newcommand{\trivtm}{\bullet}
\newcommand{\er}[1]{|{#1}|}
\newcommand{\erf}[1]{\|{#1}\|}
\newcommand{\erlam}[1]{\lambda {#1}.~}
\newcommand{\ermu}[1]{\mu {#1}.~}
\newcommand{\ereqlam}{\lambda_\eq.~}

\newcommand{\genvar}{\widehat}
\newcommand{\genA}{\genvar{a}}
\newcommand{\genB}{\genvar{b}}
\newcommand{\genC}{\genvar{c}}
\newcommand{\typA} {\alpha}
\newcommand{\typB} {\beta}
\newcommand{\varA}{\alpha}
\newcommand{\varB}{\beta}

%% Context
\newcommand{\dctx}{\Psi}
\newcommand{\tctx}{\Gamma}
\newcommand{\sctx}{\Psi}
\newcommand{\sctxb}{\Psi'}
\newcommand{\ctxsplit}{\shortmid}
\newcommand{\ctxinit}{\varnothing}
\newcommand{\ctxl}{\Theta}
\newcommand{\ctxr}{\Delta}
\newcommand{\cctx}{\Omega}
\newcommand{\byuni}{\vdash}
\newcommand{\byinf}{\vdash}
\newcommand{\byminf}{\vdash_m}
\newcommand{\bylessp}{\vdash}
\newcommand{\byoinf}{\vdash^\lambda}
\newcommand{\byhinf}{\vdash^{\mathit{OL}}}
\newcommand{\byfinf}{\vdash^F}
\newcommand{\bypinf}{\vdash^{\mathit{B}}}
\newcommand{\infto}{\Rightarrow}
\newcommand{\bychk}{\vdash}
\newcommand{\byochk}{\vdash^\lambda}
\newcommand{\chkby}{\Leftarrow}
\newcommand{\byapp}{\vdash_\bullet}
\newcommand{\byall}{\vdash_\delta}
\newcommand{\bytar}{\vdash}
\newcommand{\byinst}{\vdash_\Inst}
\newcommand{\bygen}{\vdash_\Gen}
\newcommand{\bysub}{\vdash}
\newcommand{\bydsub}{\vdash}
\newcommand{\bycg}{\vdash}
\newcommand{\byrf}{\vdash}
\newcommand{\bywf}{\vdash}
\newcommand{\bywt}{\vDash}
\newcommand{\toctx}{\dashv \ctxl}
\newcommand{\toctxo}{\dashv \tctx}
\newcommand{\toctxr}{\dashv \ctxr}
\newcommand{\dpreinf}[1][]{\dctx {#1} \byinf}
\newcommand{\dprechk}[1][]{\dctx {#1} \bychk}
\newcommand{\dpreall}[1][]{\dctx {#1} \byall}
\newcommand{\dpreapp}[1][]{\dctx {#1} \byapp}
\newcommand{\dpreuni}[1][]{\dctx {#1} \byuni}
\newcommand{\dpretar}[1][]{\dtctx {#1} \bytar}
\newcommand{\dpreinst}[1][]{\dctx {#1} \byinst}
\newcommand{\dpregen}[1][]{\dctx {#1} \bygen}
\newcommand{\dprecg}[1][]{\dctx {#1} \bycg}
\newcommand{\dprewf}[1][]{\dctx {#1} \bywf}
\newcommand{\dprewt}[1][]{\dctx {#1} \bywt}
\newcommand{\tpreinf}[1][]{\dctx {#1} \byinf}
\newcommand{\fpreinf}[1][]{\tctx {#1} \byfinf}
\newcommand{\tprechk}[1][]{\tctx {#1} \bychk}
\newcommand{\tpreall}[1][]{\tctx {#1} \byall}
\newcommand{\tpreapp}[1][]{\tctx {#1} \byapp}
\newcommand{\tpreuni}[1][]{\tctx {#1} \byuni}
\newcommand{\tpretar}[1][]{\ttctx {#1} \bytar}
\newcommand{\tpreinst}[1][]{\tctx {#1} \byinst}
\newcommand{\tpregen}[1][]{\tctx {#1} \bygen}
\newcommand{\tprecg}[1][]{\tctx {#1} \bycg}
\newcommand{\tprewf}[1][]{\dctx {#1} \bywf}
\newcommand{\tprewt}[1][]{\tctx {#1} \bywt}
\newcommand{\tpresub}[1][]{\dctx {#1} \bysub}
\newcommand{\tpreconssub}[1][]{\dctx {#1} \bysub}
\newcommand{\tprematch}[1][]{\dctx {#1} \vdash}
\newcommand{\tpreglb}[1][]{\tctx {#1} \vdash}
\newcommand{\presub}[1][]{{#1} \bysub}
\newcommand{\dpresub}[1][]{\dctx {#1} \bydsub}
\newcommand{\wc}{\ \var{ctx}\ }
\newcommand{\exto}{\longrightarrow}
\newcommand{\cgto}{\longmapsto}
\newcommand{\rfto}{\rightsquigarrow}
\newcommand{\aeq}{\equiv_\alpha}
\newcommand{\uni}{\leqq}
\newcommand{\dsub}{\leq}
\newcommand{\tsub}{<:}
\newcommand{\tsuper}{{\color{BrickRed}~:>~}}
\newcommand{\tvarinst}{:\leqq}
\newcommand{\tsubeither}{<:_m}
\newcommand{\elet}[3]{\kw{let} {#1} = {#2} \kw{in} {#3}}
\newcommand{\gen}[2]{\overbar{{#1}({#2})}}
\newcommand{\agen}[2]{{#1}_{agen}({#2})}
\newcommand{\tgen}[2]{{#1}_{gen}({#2})}

\newcommand{\erase}[1]{\lfloor{#1}\rfloor}
\newcommand{\etaeq}{\rightsquigarrow_{\eta id}}

\newcommand{\match}{\triangleright}
\newcommand{\glb}{\sqcap}
\newcommand{\tconssub}{\lesssim}
\newcommand{\unif}{\lessapprox}
\newcommand{\mask}[2]{{#1}|_{#2}}
\newcommand{\bymask}{\vdash}
% \newcommand{\appto}{\Rightarrow\mathrel{\mkern-12mu}\Rightarrow}
\newcommand{\lessp}{\sqsubseteq}
\newcommand{\lesspp}{\sqsubseteq^{\mathit{B}}}
\newcommand{\pbccons}{\prec}
\newcommand{\unknown}{\star}

% algorithm subsitution
\newcommand{\as}{S}
% algorithm name supply
\newcommand{\an}{N}
\newcommand{\byalgo}{\vdash}
\newcommand{\byarrow}{\vdash^\to}
\newcommand{\ato}{\hookrightarrow}
\newcommand{\acons}{\cdot}
\newcommand{\ex}[1]{\setminus_{#1}}
\newcommand{\san}[3]{(\mathcal{S}_#1, \mathcal{A}_#2#3)}

% Primitives
\newcommand{\nat}{\mathsf{Int}}
\newcommand{\bool}{\mathsf{Bool}}
\newcommand{\float}{\mathsf{Float}}
\newcommand{\truee}{\mathsf{true}}
\newcommand{\tope}{\mathsf{Top}}
\newcommand{\pbc}{$\lambda\mathsf{B}$\xspace}


\newcommand{\overbar}[1]{\mkern 1.5mu\overline{\mkern-1.5mu#1\mkern-1.5mu}\mkern 1.5mu}
\newcommand{\agtconssub}{~\widetilde{\tsub}~}


\newcommand{\byhave}{\blacklozenge}

\newcommand{\byget}{\blacksquare}


\newcommand{\obb}{$\mathbf{FOb}^{?}_{<:}$}

\newcommand{\revision}[1]{{\color{Red}{#1}}}

\newcommand{\blamev}{\mathsf{blame}}
\newcommand{\diverge}{\Uparrow}
\newcommand{\reduce}{\Downarrow}

\newcommand{\static}{\mathcal{S}}
\newcommand{\gradual}{\mathcal{G}}
\newcommand{\gsubst}{S^\gradual}
\newcommand{\ssubst}{S^\static}
\newcommand{\psubst}{S^\mathcal{P}}
\newcommand{\erasetp}[1]{\lceil{#1}\rceil}

\newcommand{\ctxeq}[3]{#1 \backsimeq_{ctx} #2 : #3}
\newcommand{\ctxappro}[3]{#1 \preceq_{ctx} #2 : #3}
\newcommand{\defeq}{\triangleq}

\newcommand\subsetsim{\mathrel{\substack{
      \textstyle\sqsubset\\[-0.2ex]\textstyle\sim}}}

\begin{document}


\maketitle

\section{README}

\textbf{Replies written to reviewers are in \reply{green}. Polish anything in
  the paper/reply as you see appropriate.}


\section{Associate Editor}

From Associate Editor Jeffrey Foster: 

Recommendation \#1: Major Revision 

Associate Editor

Comments for Author:

Thank you for your submission to TOPLAS. The reviewers all feel this is a
strong, well-written paper that makes important contributions. However, they do
point out a range of issues that should be addressed before the paper can be
accepted. Hence I am recommending you complete a Major Revision of this work.

In preparing your revision, please be sure to take all of the reviewers'
detailed comments into account. Please be especially sure to address the key
concerns, including Reviewer 1's comments about Section 4.4 and about being more
clear in acknowledging the relationship of the proof to prior work. Also be sure
to address Reviewer 2's comment about one of the examples.

\section{Referee: 1}

\subsection{Summary}

The paper presents new results in gradual typing, with one contribution being
consistent subtyping for implicit polymorphism, another being type inference for
gradual typing with higher-rank polymorphism. Some results depend on a
conjecture, but even if the conjecture turns out to be false, the work done
under that assumption is likely to be valuable.

To address the main TOPLAS criteria:

\begin{enumerate}
\item This is one of the best papers I have seen in the last year. Either the
  first half of the paper or the second might be worth consideration alone, but
  the combination is certainly substantial.
\item Gradual typing is "hot", and the specific research direction is
  well-motivated.
\item The presentation is largely effective. There is a substantial inaccuracy
  in section 4.4, some inconsistent notation, and at least one missing
  description of notation. Moreover, the paper needs to clarify that some of the
  manual proofs (in the appendix) are adapted (sometimes with no real
  differences) from a previous paper; that previous paper is amply cited as
  inspiration, but the authors need to point out the extent to which they have
  relied on others' work for both the structure of the metatheory (statements of
  lemmas and the lemmas' names) and some of the proofs themselves.
\end{enumerate}

The remainder of my review is structured as follows:
\begin{enumerate}
\item some thoughts on section 4.4;
\item relatively major issues that need to be addressed;
\item minor comments (where the authors need not make the changes if they prefer the current version).
\end{enumerate}

\subsection{Thoughts on section 4.4}

In Jafery and Dunfield 2017, A $\lessp$ B means A is more precise (or "less
imprecise") than B, not less. Section 4.4 flips the meaning; the paper should
point this out (or make the notation consistent).

While it is true that their definition of directed consistency *looks* somewhat
like the submission's Definition 4.1, the directions don't match up. 4.1 says
you can move to a supertype (which, given a reasonable subtyping relation, is
always safe: it should never immediately cause a risky cast), then "blur" the
type via consistency, then move to a supertype. Jafery and Dunfield's directed
consistency says you can gain precision (which might fail), move to a supertype,
and then lose precision.

Maybe the reference to 4.1 at p. 14 / line 26 was meant to be a reference to 4.4
(consistent subtyping in AGT)?

Even if so, the discussion in the next paragraph (27-31) is clearly about 4.1.
While it is reasonable to ask which relation is "more fundamental", that has
little to do with the superficial fact that both directed consistency and 4.1
"look[] very similar".

\reply{
  \begin{enumerate}
  \item We added a note to clarify that the definition of directed consistency
    is adapted to match our interpretation of the notation $\lessp$, for the
    sake of consistency with Cimini and Siek 2016 throughout the paper. That is, $A \lessp B$ means $A$ is
    less precise than $B$. This definition of $\lessp$ was given in appendix,
    and is now moved to the main text (see Fig. 10).
  \item We deleted ``two definitions look similar''. We agree with the reviewer
    that this superficial fact is misleading.
  \item We rephrased the sentence of the reference to 4.1 that was at p. 14 /
    line 26 to make it more clear. As the definition of directed consistency
    coincides with Definition 4.4 (Consistent Subtyping in AGT), by Corollary
    4.5 (Equivalence to AGT on Simple Types), the definition coincides with our
    definition of consistent subtyping, i.e. Definition 4.1 (Consistent
    Subtyping).
  \end{enumerate}
  }

At 40-41 the paper says "It may seem that precision is a more atomic
relation..." I agree. But then the paper claims (okay, implies, but I am not
sure how else to read the last sentence "Thus precision can be difficult to
extend...") that, because it is (I would say, has often been) easier to define
consistency exnihilo than precision, precision is really less atomic. That
doesn't follow.

I see two research questions here.

The first question is "moral": in a gradual language, what relation(s) "should"
come first? That is a conceptual question.

The second is a pragmatic question: in a gradual language, what relation(s)
should come first to make the designer's life easier?

Ideally, the questions would have the same answer, but there is no particular
reason to believe they do. Lines 40-48 are maybe an answer to the second
question, but the phrase "more atomic" connotes (to me) the first question.

\reply{We don't have a general answer to this question. As far as we are
  concerned, it is easier for type consistency to come first and to make the
  designer's life easier, for exactly the reason we stated in the paragraph.
  though we cannot claim it for sure without a deeper analysis.}


\subsection{Relatively major issues}

11, footnote 5: how is your simplified version different?

\reply{We extended the explanation in the footnote. This simplified version
  is presented as one of the key ideas of the design of type consistency, which
  is later amended with labels.}

13, footnote 6: at least in this section, there is only one place that $\rightarrow$ is used
as a set-level operator (in Def. 4.3); it would be more clear to write out
\[
  \gamma(A \rightarrow B) = \{  S_A \rightarrow S_B | S_A \in \gamma(A) \quad \text{and} \quad S_B \in \gamma(B)    \}
\]
% \begin{verbatim}
% {γ(A → B) = {S_A → S_B | S_A ∈ γ(A) and S_B ∈ γ(B)}.
% \end{verbatim}
As it is, the reader has to find the definition of lifted operators in the AGT
paper, and figure out that you are writing $\rightarrow$ instead of $\tilde{\rightarrow}$.

\reply{We made the change as suggested.}

In Def. 4.4 it would be good to mention that the $A_1$, $B_1$ in the
concretizations are *static* types (this is explicit in the AGT paper, because
it uses different meta-variables), though this is implicit in your explanation
of concretization.

\reply{We made the change as suggested.}

14, lines 22-23: "In their setting, precision is defined for type constructors
and subtyping [is defined] for static types." The first part is slightly
misleading, and the second part is not accurate:

1. Jafery and Dunfield define precision for type constructors, and then
straightforwardly lift it to types.

\reply{We clarified the sentence as suggested.}

2. They define subtyping over *all* gradual types. Subtyping over sums (which
are the only gradual types in their system) is defined using subtyping over "sum
constructors". Arguably, their system mixes up precision and subtyping, because
the subtyping relation is not defined purely over static types. It seems likely
that could be disentangled, leading to a directed consistency relation that has
a static relation (subtyping over static types, not gradual types) bracketed by
(im)precision, but that's not what they did.

\reply{We clarified the sentence as suggested.}

Section 5.3 appears to mix up (opposite) meanings of the symbol $\lessp$:


Meaning 1: O
the same program except that e2 has more unknown types". This is the reading of
$\lessp$ as "more precise than" or "less imprecise than": e1 is more precise, e2 is
less precise (because it has more unknown types).

Meaning 2: Within Lemma 4 and Definition 5.4, this is reversed. In Definition
5.4, e' $\lessp$ e means that e is more precise. If the "gradually typed program
evaluates to a value" (that value being v), then *less* precise annotations
(those on e') translate to s', which evaluates to v', where v' is "less
precise".

My personal preference is strongly for Meaning 1, which is consistent with the
original statement of the gradual guarantee (Siek et al. 2015). The paper should
at least be internally consistent.

\reply{This is indeed a typo. It should be ``e1 $\lessp$ e2 means that e1 and e2
  are the same program except that e1 has more unknown types''. That is, Meaning
  2. This interpretation matches our definition of $\lessp$, which was in the
  appendix and is now given in Fig 10.}

In 6.0, various notations are explained, including one-hole notation for
contexts, but the two-hole notation $\Gamma[\theta_1][\theta_2]$ is not
explained---even though at 23 line 40, the notation is used with "Recall
that...".

\reply{We added the missing explanation.}

In Def. 9.11, the case for $(\Gamma, a_S)_A$ where $a_S$ does *not* occur in A
is missing. Should it be the following?

\[
  (\Gamma, \widehat{a}_S) \ \Gamma_A, \widehat{a}_S  \quad \text{if $\widehat{a}_S$ occurs in $A$}
\]

\reply{Yes. We added the missing case.}

Arguing for the decidability of the algorithmic system (page 25) only by citing
Dunfield and Krishnaswami (2013) is somewhat unsatisfying: since decidability is
not obvious, it would be better to show decidability outright for the actual
system at hand. If their proof strategy continues to work, doing the proof
should be easy. (If this proof was mechanized, then just mention that. But my
impression is that your proofs in this section were not mechanized.)

\reply{We added detailed proofs showing decidability of our extended algorithmic
  system in appendix C. We significantly expanded section 6.4 to highlight the
  key lemmas of decidability. Due to our gradual setting, we have a revised
  measure where we need to count the number of unknown types when proving
  decidability of consistent subtyping. This is a key difference from the
  original decidability proofs in the DK system. The adoption of matching
  judgment also simplifies the reasoning a bit when proving decidability of typing.
}



% \bruno{Here we should say more. Firstly we should mention a
%   significantly expanded Section 6.4. Then we should say something
%   like: In the paper we extended our discussion on decidability and
%   discussed the key ideas, including the revised measure used in the
%   decidability argument... The differences to the original
%   decidability (by DK)
%   proof are ...}

32 line 44: "the dynamic gradual guarantee is a story of the good choices":
Please explain what you mean by a "story of the good choices". I don't know what
a "story of choices" would be.

\reply{We rephrased the sentence to make it clear. ``\emph{...so we can
    restrict the discussion of dynamic gradual guarantee to representative translations.}''}

37 line 8: "principle" should be "principal".

\reply{Fixed.}

37 lines 32-34: Saying that no previous approaches "can instruct us how to
define..." is very strong; it would be better to say they did not instruct you,
or that they do not directly lead to a definition of consistent subtyping for
polymorphism.

\reply{We weakened the argument as suggested. ``\emph{...none of these
    approaches instructed us how to define consistent subtyping for polymorphic
    types.}''}

38 line 23: Remind the reader what Siek and Vachharajani's observation was.

\reply{We added explanations to make the description more clear.}

38 line 34: "a sole inspiration": "Sole" means "only", so "a" does not make
sense; did you mean "the sole inspiration"? But it was probably not the sole
inspiration, rather the main or primary inspiration. Unless you meant that the
declarative system is the sole inspiration for the algorithmic system, in which
case you should rephrase (and say "basis" rather than "inspiration").

\reply{We changed it to ``\emph{...the main inspiration}...''}

Appendix:

Much of the infrastructure (at least for these manual proofs for the relevant
part of the paper) is identical to Dunfield and Krishnaswami (2013), up to and
including the names of many lemmas (even their odder names, such as "Softness
Goes Away" and "Confluence of Completeness"). This is good evidence that their
proofs can feasibly be adapted and extended (unlike many papers that don't give
the metatheory in enough detail), but the paper does not make the extent of this
reuse clear (I don't think it was mentioned at all). In at least some cases, the
proofs are repeated almost verbatim, so appropriate credit is mandatory.

\reply{We added credits to DK in appendix E.1: basically all the syntactic
  properties of context extensions are copied from theirs verbatim because we
  have the same context extension rules. However, the rest of lemmas and
  theorems are provided with detailed proofs, with necessary changes to
  support our new gradual type system.}


\subsection{Minor comments}

In the abstract, including citation years would be helpful.

\reply{It is our personal preference to avoid citation years in the abstract. We
  slightly revised the abstraction by adding more texts to disambiguate when
  there can be confusion.}

The phrase "armed with a coherence conjecture" is odd, since (being a
conjecture) it is not an effective weapon. "Assuming a coherence conjecture"?

\reply{We made the change as suggested.}

Quotations (e.g. p. 2 line 10-11) should not be in italics.

\reply{Fixed.}

p. 2 line 41: "same flavor" should be "some flavor"

\reply{Fixed.}

p. 3: Footnote 2 should go in the main text.

\reply{Thanks for the advice. We state it explicitly in the main text now.
  \emph{``The second goal of this paper is to present the design of GPC,
    which stands for Gradually Polymorphic Calculus...''}.}

p. 9: "dispersed with": "littered with" (or say that "toDyn and fromDyn are
dispersed throughout the code")

\reply{Fixed.}

p. 10 line 41: "Importantly" is redundant with "crucial".

\reply{Fixed.}

p. 10, p. 13: "so-called" has a connotation that the name given is not good
(e.g. writing "so-called \emph{naive subtyping}" if I think the term "naive
subtyping" is misleading). Using italics for the term itself, without
"so-called", is sufficient.

\reply{Fixed.}

p. 19 line 45: "a static program that is typeable...if and only if...": delete
"that".

\reply{Fixed.}

Fig. 13: The "Instantiation I" and "Instantiation II" judgments are not
distinguished by syntax:
\[
  \tctx \vdash \widehat{a} \lessapprox \widehat{a} \dashv \Delta
\]
could mean either "Instantiation I" with $\widehat{a}$ as the right-hand A, or
"Instantiation II" with $\widehat{a}$ as the left-hand A. It's not actually
broken, because AS-EVAR handles the situation when it would otherwise arise
"from" Fig. 12, and within Fig. 13 it is clear (e.g. in INSTR-ARR the first
premise is clearly switching to "Instantiation I" because \verb|A_1| is on the
right). But it is potentially confusing. Even if the authors do not want to
change the notation, this issue should be mentioned.

\reply{The above judgment does not hold, because both in INSTL-SOLVE and
  INSTR-SOLVE, it requires $\tau$ to be well-typed in the context before
  $\widehat{a}$, which is impossible for this case. And exactly, this case is
  handled in AS-EVAR.}

20 line 43: "henceforth DK system": the abbreviation is not actually used later.

\reply{We replaced some of the citations with DK system to reduce repetitive
  citations.}

22 lines 45-46: Could mention that as-forallR is invertible, so in an
implementation the choice is easy.

\reply{We made the change as suggested.}

23 line 30: "Two twin judgments" is redundant: "twin" implies two. I would say
"Two symmetric judgments".

\reply{Fixed.}

In 6.2, it seems worth mentioning that the idea implemented by INST(L,R)-SOLVE
is due to Cardelli (see citation in Dunfield and Krishnaswami 2013).

\reply{We added citation to Cardelli 1993 in Section 6.2.}

29 line 20: hyphen in "syntax directed", line 21: "combing" should be
"combining"

\reply{Fixed.}

32 Def. 9.7: the last \verb|\Rightarrow| should probably be \verb|\Longrightarrow|,

\reply{Fixed.}

but I would find the definition easier to parse in English: "if ... then for all
C such that ...implies ...". Mixing object- and meta-level symbols (like $\forall$) is
confusing.

\reply{We replaced object-level $\forall$ with the text ``for all'' in Definition
  5.1 and 9.7 to make it more clear.}

34 line 24: add space before "to"

\reply{Fixed.}

The notation \verb|$\Gamma_{\mathbb C}$| is maybe too lightweight: it looks like
a meta-variable. Using a superscript would be somewhat better, but I would
prefer a more function-like notation \verb|("contaminate(\Gamma, C)"|?

\reply{We made the change as suggested.}

34 line 45: Emphasize that substituting in the input contexts is not just
unfortunate, but would effectively turn the input contexts into something else
entirely: the point of the input context is that it is input, with necessary
changes (like solving existentials) appearing in the output context.

\reply{We added this clarification accordingly.}

34 line 49: "every static existential variables": change "every" to "all"

\reply{Fixed.}

36 line 21: "with static" should be "with a static" 

\reply{Fixed.}

36 line 44: "later" should be "latter" 

\reply{Fixed.}

38 line 36: "higher rank" needs a hyphen 

\reply{Fixed.}

line 38: "as such" should be "such as" 

\reply{Fixed.}

line 50: "insight in" should be "insight into" 

\reply{Fixed.}

39 line 9: "real world" needs a hyphen 

\reply{Fixed.}

Appendix: 

E.6: "if any only if" should be "if and only if". 

\reply{Fixed.}

Definition F.1 is missing the case for the marker $\blacktriangleright_{ \widehat{a} }$.

\reply{Fixed.}

In Lemma F.2, the first \verb|\emptyset| should be the empty context.

\reply{Fixed.}


\section{Referee: 2 }

\subsection{Comments to the Author}

The paper presents a gradually typed language with
Curry-style (implicit) higher-rank polymorphism, focusing on syntactic aspects,
especially the static "refined" criteria of gradual typing and correctness of a
bidirectional type-checker in the style of Dunfield-Krishnaswami. A shorter
paper has previously been presented at ESOP and the main additions in the
submission seem to be sections 3, 8.2 and 9.

The contributions of the paper are strong. The authors improve over previous
work considerably in the syntax of a gradual polymorphic source language: the
declarative and algorithmic systems are both straightforward, a sign of a good
design.

I think the most confused point in the paper is about instantiation of a forall
type with a dynamic type. In the main body of the paper, adapted from the short
version, the system does not allow this because the dynamic type is not a
monotype. However, the last motivating example (heterogeneous containers)
instantiates the polymorphic Scott-encoded `nil` and `cons` functions with the
dynamic type. Finally, in section 9, they present a system with "static and
gradual type parameters", where an otherwise un-constrained gradual type
parameter can be instantiated with the dynamic type. My understanding is that
the heterogeneous container example is based on this final system, which is also
the one followed by the implementation, but this needs to be explained more
clearly in the text, preferably with a small representative example.

\reply{We added a note for the heterogeneous container in Section 3 to clarify
  that the heterogeneous list can only get the dynamic type in the extended type
  system presented in Section 9. }

\ningning{I don't understand here about the ``a small representative example''.
  An example for what?}

Finally, before my more targeted comments, a philosophical question. One of the
main contributions is a new formulation of consistent subtyping that shows that
the definition of Siek and Taha is incorrect. We can see in retrospect that the
presentation in Siek-Taha was a coincidence based on the simplicity of their the
types involved. While the new characterization is clearly an improvement,
without a *semantic* criterion it's not clear why this is correct. Why should we
believe *in principle* that the correct definition of consistent subtyping is
"$A \lesssim B$ iff $A <: A' \sim B' <: B$" and that this will be work in the
future? The authors show it also works for a top type, but it would be nice to
have a more conceptual explanation.

\reply{We don't have a general answer to this question.\\
  We don't claim that the definition of Siek and Taha is incorrect. Their
  definition works well in gradual typing for objects. Nor do we claim that
  our definition is \emph{the} correct one. We merely claim that it
  generalizes and subsumes the definition from Siek and Taha, and it works for
  polymorphic types and Top types.\\
  It would be interesting to investigate whether our notion of consistent
  subtyping has a more foundational conceptual explanation, for example, whether
  it would coincide with AGT on polymorphic types. We leave that as future
  work.}


\subsection{More Targeted Technical Comments}

\begin{enumerate}
\item I suggest you downgrade the status of Corollary 5.3 to Lemma rather than
  corollary, because it is generally quite difficult to make these contextual
  equivalence results completely precise. In particular, the proposed reason for
  the corollary to be true ("the only role of types and casts is to emit cast
  errors") is a major component of the dynamic gradual guarantee.

\reply{We made the change as suggested.}

\item It is worth pointing out that even with parametricity, Scott encodings are
  not correct in CBV with effects, and so the use of polymorphism does not
  alleviate the need for algebraic datatypes from the language. For instance,
  ($\lambda n. \lambda c. \Omega$) is a valid inhabitant of the Scott encoding
  but does not correspond to any (strict) list. In a language like Haskell
  though you are probably fine.

\reply{We made a note as suggested.}

\item Section 5.2 can be improved in its explanation of contextual
  approximation/equivalence. Specifically, approximation is only introduced as
  an intermediate notion for presenting equivalence, so it would be clearer to
  present the following direct characterization of equivalence, that has the
  benefit of being almost a direct translation into math of "equivalence up to
  error on either side":
\begin{verbatim}
s1 =~ s2 iff forall C : (...).
    C[s1] ->* blame \/ C[s2] ->* blame
    \/ (C[s1],C[s2] diverge) \/ (C[s1],C[s2] ->* n)
\end{verbatim}
  similarly Definition 9.7 could be removed and replaced with equivalence
  defined as:
\begin{verbatim}
s1 =~ s2 iff forall C : (...).
    (C[s1],C[s2] ->* blame)
    \/ (C[s1],C[s2] diverge) \/ (C[s1],C[s2] ->* n)
\end{verbatim}
  % \ningning{TODO. I would prefer the current definition.}
  \reply{ Thank you for the suggestion. We would like to stick to the current
    definition for personal preference, which is also the form adopted by Ahmed
    et al. 2017.
  }

\item Page 14 "recall that consistency is in fact an equivalence relation
  lifted...": this interpretation is inherent to AGT, but you are not using the
  AGT approach. It's not clear at all where in the paper this should be recalled
  from.

  \reply{Fixed. We now mentioned this observation in Section 4.1 where we define
    consistency.}

 \item Page 14 you present precision for the first time and as I'm sure the
   authors are aware the directionality of this relation is fraught in the
   literature, so please note for the reader when it is introduced that you are
   picking the opposite convention from most of the work you are citing,
   including the rule you are presenting.

  \reply{We added a note for more explanations, and the full definition of
    precision is now in the main text (see Fig. 10)}

\item The authors show that their definition of consistent subtyping is
  "correct" for a top type and a predicative forall type, and that the Siek-Taha
  definition fails. Both of these can be understood as *intersection types*: the
  top type is the intersection of the empty set and a forall type is an
  infinitary intersection. Does a similar property hold for a *binary*
  intersection? A discussion of this would improve 8.1.

  \reply{We added a simple discussion in Section 8.1 as suggested.}

\item In the statement of the dynamic gradual guarantee using representative
  translations, shouldn't "then for some B and r' with \verb|e' : B ~~> r'|" be
  replaced with the stronger "for any B and r' with \verb|e' : B ~~> r'|"? They are
  equivalent if 9.8 is correct?

  \reply{No, they are not. We require that $B \lessp A$. Yet, due to the generalization rule
    an expression can have infinitely many different types which don't
    necessarily satisfy that constraint. For example, $1$ can have both types $\nat$ and $\forall a. \nat$. Indeed,
    $\bullet \dashv 1 : \nat \rightsquigarrow 1$ and
    $\bullet \dashv 1 : \forall a. \nat \rightsquigarrow \Lambda a.
    1$.}

\end{enumerate}


\subsection{Writing/Typos }

\begin{enumerate}
\item Page 12 observations 1 and 2 can be summarized by saying that
  consistent subtyping is a *bimodule* with respect to subtyping on
  each side, or more explicitly, a $<:,<:$-bimodule. Then definition
  4.1 says that consistent subtyping is the least subtyping bimodule
  extending consistency.

  \reply{The reviewer is right, this is an accurate characterization. We added a
    note to point it out for readers who are potentially familiar with category
    theory.}

\item Page 14: "at first sight their definition looks very similar (replacing
  $\lessp$ with $<:$ and ...)". Watch out: this is at best a misleading
  oversimplification: notice that you need to flip the precision on one side.
  This definition says that consistent subtyping is a [=,=] bimodule, as all AGT
  lifted relations are. This should be fixed to avoid confusing the reader.

  \reply{We removed this misleading sentence.}

\item Page 21: When introducing the bidirectional system for the first time, it
  would be very helpful to clearly identify the *modes* for all of the judgments
  when they are introduced. So for example when algorithmic consistent subtyping
  is introduced I wasn't sure at first if the types were inputs or outputs until
  I read through the definition.

  \reply{We added explanation in every algorithmic judgment form as suggested.}

\item page 23 line 45 "rule rule"

  \reply{Fixed.}

\item Page 30 the sentence starting "In order to account..." doesn't make
  grammatical sense

  \reply{Fixed.}

\item Page 37 "There is not much work on integrating gradual typing with
  parametric polymorphism": I would not say that this is true at all, especially
  if you broaden it to include the broader class of works on the dynamic
  enforcement of parametric polymorphism which go back as early as Morris 1973
  "Types are not Sets" (pre-dating the formalization of parametricity!).

  \reply{We rephrased the sentence to make it clear and included a simple
    discussion of Morris 1973 in related work.}

\item Finally the last paragraph of page 39 "As future work,..." is so vague as
  to be useless. If you have something more specific to say, like specific
  features say it.

  \reply{We made the change as suggested.}
\end{enumerate}


\section{Referee: 3 }

\subsection{Comments to the Author}


This paper makes an important contribution to the theory of gradual typing in
showing how to combine gradual typing with implicit, higher-rank polymorphism.
The paper generalizes the notion of consistent subtyping to handle polymorphic
subtyping, defines both a declarative and algorithmic version of the type
system, and gives a translation to the polymorphic blame calculus. The paper's
treatment of consistent subtyping is particularly nice in the way it further
demonstrates the orthogonality of consistency and subtyping. The paper proves
that the declarative and algorithm versions of the type system are in agreement
and it proves that the system satisfies the static aspects of the refined
criteria for gradual typing. The paper shows that their language satisfies the
dynamic gradual guarantee provided that the polymorphic blame calculus does and
that a conjecture about coherence holds. Most of the definitions and proofs were
verified in Coq.



\subsection{Issues to be fixed}

p. 5 

"To compose subtyping and consistency, Siek and Taha defined consistent 
subtyping (written $\lesssim$) in two equivalent ways:"

That is not the definition of consistent subtyping. That is Proposition 2 of
Siek and Taha (2007). The definition, as you later admit, is based on the
restriction operator. In several later points in the paper (page 11, 28) you
make the same mistake, calling this the definition, which it is not. In general,
the discussion of Siek and Taha (2007) should focus more on the actual
definition (and the restriction operator) and less on Prop. 2. Section 4.5 does
a better job in this regard.

\reply{We rewrote this subsection to make it clear. Now we state explicitly in
  the text that the restriction operator is the definition, which we refer to as
  the \emph{algorithmic} consistent subtyping. $A \sim B <: C$ and $A <:C \sim
  B$ come from the proposition, and we refer to them as \emph{declarative}
  consistent subtyping, as it servers as a good guideline on superimposing
  consistency and subtyping.}

p. 16 

The treatment of the dynamic type in the definition of consistent subtyping of
Figure 8 is not novel. The first published version of consistent subtyping that
takes this approach is due to Bierman and Abadi. (See the bibtex entry below, in
particular, see section 6 about the prodFTS language.) The novelty of Figure 8
is in dealing with universal types.

\begin{verbatim}
@incollection{Bierman:2014aa,
Author = {Bierman, Gavin and Abadi, Mart{'\i}n and Torgersen, Mads},
Booktitle = {ECOOP 2014 -- Object-Oriented Programming},
Editor = {Jones, Richard},
Pages = {257-281},
Publisher = {Springer Berlin Heidelberg},
Series = {Lecture Notes in Computer Science},
Title = {Understanding {TypeScript}},
Volume = {8586},
Year = {2014}}
\end{verbatim}

\reply{We added the discussion with this paper in related work (see the end of
  the paragraph titled `` Gradual Typing.''). Basically there are two
differences between TypeScript and our system. First, as the reviewer mentioned,
they don't have polymorphic types. Second, while our
consistent subtyping inserts run-time casts, in TypeScript, type information is
erased after compilation so there are no runtime casts, which makes runtime type
errors possible.}

% \ningning{Please double check what I wrote in related work.}

p. 31 

"therefore it is best to instantiate a with $\unknown$" "We interpret G as
$\unknown$ since any monotype could possibly lead to a cast error."

It seems that the upshot is that you get similar results (or even the same?) as
the inference algorithm of Siek and Vachharajani (2008). The discussion of Siek
and Vachharajani (2008) in Section 10 (Related Work) needs to be updated to
state the similarities/differences in the inference results. That is, answer the
question: if you remove polymorphism from your system, how do the inference
solutions compare to those of Siek and Vachharajani? Are they identical?

\reply{We updated Related Work to include a simple discussion. While we believe
  we get similar results as theirs, without rigorous proof we cannot claim it
  for sure. Our algorithm is inspired by Garcia and Cimini 2015, where they
  claim that their algorithm have principal types. In this sense, our algorithm
  should produce similar results as Garcia and Cimini 2015, which is at least as
  general a type as that of Siek and Vachharaiani 2008 in inferring simple
  types.}

\ningning{Please double check here.}


\subsection{Minor comments and edits}

p. 2 

"As Siek and Taha [2007] put it this shows that"

\verb|==>| insert a comma 

"As Siek and Taha [2007] put it, this shows that" 

\reply{Fixed.}


p. 3 

"and does not require the ad-hoc restriction operator of Siek and Taha [2007]"

What makes the restriction operator "ad-hoc"? On what basis do you make this
derogatory remark?

\reply{We removed this remark.}

p. 6 

In section 2.2 it would be helpful to say something about the two subtyping
rules for universal types.

\reply{We added explanations for those subtyping rules.}


p. 9 

"The drawback is that the code is dispersed with toDyn and fromDyn, 
which obscure the program logic"

\verb|==>| change "disperse" to "littered" to avoid non-idiomatic phrase
"The drawback is that the code is littered with toDyn and fromDyn,
which obscure the program logic" 

\reply{Fixed.}


p. 25 

"the proof strategy employed by them for decidability can be easily 
adapted to our setting"

\verb|==>| "can be" is weak and vague
"we easily adapted the proof strategy employed by them for decidability 
to our setting" 

\reply{We added detailed proofs showing decidability of our extended algorithmic
  system (Appendix C). We refer to our relevant reply to reviewer 1. }
% \bruno{Expand or refer to reply to reviewer 1.}


p. 28 

The "naive design" is painfully naive. Furthermore, the V-App1 rule would be
wrong regardless of the V-Sub rule. I recommend deleting this part.

\reply{We have decided to keep this design because we were explicitly asked by a reviewer of our ESOP
  version if it would work. This shows that this design is not so
  obvious for everyone.\\
  We slightly revised the design's explanation to emphasize from
  the beginning that it is wrong.}

p. 29 

In the proper declarative design, why not using matching and just one
application rule, following more modern presentation of gradual typing?

\reply{We made the change as suggested. We updated the Coq proof as well.}


p. 39 

"have the dynamic guarantee"

\verb|==>|
"have the dynamic gradual guarantee"

\reply{Fixed.}

\end{document}
