
\section{Related Work}
\label{sec:related}

Along the way we discussed some of the most relevant work to motivate,
compare and
promote our gradual typing design. In what follows, we briefly discuss related
work on gradual typing and polymorphism.


\paragraph{Gradual Typing}

The seminal paper by \citet{siek2006gradual} is the first to propose gradual
typing, which enables programmers to mix static and dynamic typing in a program
by providing a mechanism to control which parts of a program are statically
checked. The original proposal extends the simply typed lambda calculus by
introducing the unknown type $\unknown$ and replacing type equality with type
consistency. Casts are introduced to mediate between statically and dynamically
typed code. Later \citet{siek2007gradual} incorporated gradual typing into a
simple object oriented language, and showed that subtyping and consistency are
orthogonal -- an insight that partly inspired our work. We show that subtyping
and consistency are orthogonal in a much richer type system with higher-rank
polymorphism. \citet{siek2009exploring} explores the design space of different
dynamic semantics for simply typed lambda calculus with casts and unknown types.
In the light of the ever-growing popularity of gradual typing, and its somewhat
murky theoretical foundations, \citet{siek2015refined} felt the urge to have a
complete formal characterization of what it means to be gradually typed. They
proposed a set of criteria that provides important guidelines for designers of
gradually typed languages. \citet{cimini2016gradualizer} introduced the
\emph{Gradualizer}, a general methodology for generating gradual type systems
from static type systems. Later they also develop an algorithm to generate
dynamic semantics~\cite{CiminiPOPL}. \citet{garcia2016abstracting} introduced
the AGT approach based on abstract interpretation. As we discussed, none of
these approaches instructed us how to define consistent subtyping for
polymorphic types.

There is some work on integrating gradual typing with rich type disciplines.
\citet{Ba_ados_Schwerter_2014} establish a framework to combine gradual typing and
effects, with which a static effect system can be transformed to a dynamic
effect system or any intermediate blend. \citet{Jafery:2017:SUR:3093333.3009865}
present a type system with \emph{gradual sums}, which combines refinement and
imprecision. We have discussed the interesting definition of \emph{directed
  consistency} in Section~\ref{sec:exploration}. \citet{castagna2017gradual} develop a gradual type system with
intersection and union types, with consistent subtyping defined by following
the idea of \citet{garcia2016abstracting}.
TypeScript~\citep{typescript} has a distinguished dynamic type, written {\color{blue} any}, whose fundamental feature is that any type can be
implicitly converted to and from {\color{blue} any}.
% They prove that the conversion
% definition (called \emph{assignment compatibility}) coincides with the
% restriction operator from \citet{siek2007gradual}.
Our treatment of the unknown type in \cref{fig:decl:conssub} is similar to their
treatment of {\color{blue} any}. However, their type system does not have
polymorphic types. Also, Unlike our consistent subtyping which inserts runtime
casts, in TypeScript, type information is erased after compilation so there are
no runtime casts, which makes runtime type errors possible.
% dynamic checks does not contribute to type safety.


\paragraph{Gradual Type Systems with Explicit Polymorphism}

\citet{Morris:1973:TS:512927.512938} dynamically enforces
parametric polymorphism and uses \emph{sealing} functions as the
dynamic type mechanism. More recent works on integrating gradual typing with
parametric polymorphism include the dynamic type of \citet{abadi1995dynamic} and
the \emph{Sage} language of \citet{gronski2006sage}. None of these has carefully
studied the interaction between statically and dynamically typed code.
\citet{ahmed2011blame} proposed \pbc that extends the blame
calculus~\cite{Wadler_2009} to incorporate polymorphism. The key novelty of
their work is to use dynamic sealing to enforce parametricity. As such, they end
up with a sophisticated dynamic semantics. Later, \citet{amal2017blame} prove
that with more restrictions, \pbc satisfies parametricity. Compared to their
work, our type system can catch more errors earlier since, as we argued, 
their notion of \emph{compatibility} is too permissive. For example, the
following is rejected (more precisely, the corresponding source program never
gets elaborated) by our type system:
\[
  (\blam x \unknown x + 1) : \forall a. a \to a \rightsquigarrow \cast {\unknown \to \nat}
  {\forall a. a \to a} (\blam x \unknown x + 1)
\]
while the type system of \pbc would accept the translation, though at runtime,
the program would result in a cast error as it violates parametricity.
% This does not imply, in any regard that \pbc is not well-designed; there are
% circumstances where runtime checks are \emph{needed} to ensure
% parametricity.
We emphasize that it is the combination of our powerful type system together
with the powerful dynamic semantics of \pbc that makes it possible to have
implicit higher-rank polymorphism in a gradually typed setting.
% without compromising parametricity.
\citet{devriese2017parametricity} proved that
embedding of System F terms into \pbc is not fully abstract. \citet{yuu2017poly}
also studied integrating gradual typing with parametric polymorphism. They
proposed System F$_G$, a gradually typed extension of System F, and System
F$_C$, a new polymorphic blame calculus. As has been discussed extensively,
their definition of type consistency does not apply to our setting (implicit
polymorphism). All of these approaches mix consistency with subtyping to some
extent, which we argue should be orthogonal. On a side note, it seems that our
calculus can also be safely translated to System F$_C$. However we do not
understand all the tradeoffs involved in the choice between \pbc and System
F$_C$ as a target.



\paragraph{Gradual Type Inference}
\citet{siek2008gradual} studied unification-based type inference for gradual
typing, where they show why three straightforward approaches fail to meet their
design goals. One of their main observations is
that simply ignoring dynamic types during unification does not work. Therefore,
their type system assigns unknown types to type variables and infers gradual
types, which results in a complicated type system and inference algorithm. In
our algorithm presented in \cref{sec:advanced-extension}, comparisons between
existential variables and unknown types are emphasized by the distinction
between static existential variables and gradual existential variables. By
syntactically refining unsolved gradual existential variables with unknown types, we gain a
similar effect as assigning unknown types, while keeping the algorithm relatively
simple.
\citet{garcia2015principal} presented a new approach where gradual type
inference only produces static types, which is adopted in our type system. They
also deal with let-polymorphism (rank 1 types). They proposed the distinction
between static and gradual type parameters, which inspired our extension to
restore the dynamic gradual guarantee. Although those existing works all involve
gradual types and inference, none of these works deal with higher-rank
implicit polymorphism.


\paragraph{Higher-rank Implicit Polymorphism}

\citet{odersky1996putting} introduced a type system for higher-rank implicit
polymorphic types. Based on that, \citet{jones2007practical} developed an
approach for type checking higher-rank predicative polymorphism.
\citet{dunfield2013complete} proposed a bidirectional account of higher-rank
polymorphism, and an algorithm for implementing the declarative system, which
serves as the main inspiration for our algorithmic system. The key difference,
however, is the integration of gradual typing.
% \citet{vytiniotis2012defer}
% defers static type errors to runtime, which is fundamentally different from
% gradual typing, where programmers can control over static or runtime checks by
% precision of the annotations.
As our work, those works are in a
\emph{predicative} setting, since complete type inference for higher-rank
types in an impredicative setting is undecidable. Still, there are many type
systems trying to infer some impredicative types, such as
\texttt{$ML^F$}~\citep{le2014mlf,remy2008graphic,le2009recasting}, the HML
system~\citep{leijen2009flexible}, the FPH system~\citep{vytiniotis2008fph} and
so on. Those type systems usually end up with non-standard System F types, and
sophisticated forms of type inference.

%%% Local Variables:
%%% mode: latex
%%% TeX-master: "../paper"
%%% org-ref-default-bibliography: "../paper.bib"
%%% End:
