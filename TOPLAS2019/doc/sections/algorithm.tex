\section{Algorithmic Type System}
\label{sec:algorithm}

% The declarative type system in \cref{sec:type-system} serves as a good
% specification for how typing should behave. It remains to see whether this
% specification delivers an algorithm. The main challenge lies in the rules \rref{CS-ForallL} in
% \cref{fig:decl:conssub} and rule \rref{M-Forall} in
% \cref{fig:decl-typing}, which both need to guess a monotype.

% \bruno{why are we not highlightinh the differences in gray anymore?}
In this section we give a bidirectional account of the algorithmic type system
that implements the declarative specification.
The algorithm is largely inspired
by the algorithmic bidirectional system of \citet{dunfield2013complete}
(henceforth DK system). However our algorithmic system differs from theirs in
three aspects: (1) the addition of the unknown type $\unknown$; (2) the use of
the matching judgment; and 3) the approach of \emph{gradual inference only
  producing static types}~\citep{garcia2015principal}. We then prove that our
algorithm is both sound and complete with respect to the declarative type
system. Full proofs can be found in the appendix. We also provide an
implementation, which can be found in the supplementary materials.\footnote{Note that the proofs in the appendix and the implementation are for
  the extended system in \cref{sec:advanced-extension}, which subsumes the
  algorithmic system presented in this section.}

\paragraph{Algorithmic Contexts.}

\begin{figure}[t]
  \centering
  \begin{small}
\begin{tabular}{lrcl} \toprule
  Expressions & $[[ae]]$ & \syndef & $[[x]] \mid [[n]] \mid [[ \x : aA . ae ]]  \mid  [[\x . ae]] \mid [[ae1 ae2]] \mid [[ae : aA]] \mid [[ let x = ae1 in ae2  ]] $ \\
  Types & $[[aA]], [[aB]]$ & \syndef & $ [[int]] \mid [[a]] \mid [[evar]] \mid [[aA -> aB]] \mid [[\/ a. aA]] \mid [[unknown]] $ \\
  Monotypes & $[[at]], [[as]]$ & \syndef & $ [[int]] \mid [[a]] \mid [[evar]] \mid [[at -> as]] $ \\
  Algorithmic Contexts & $[[GG]], [[DD]], [[TT]]$ & \syndef & $[[empty]] \mid [[GG , x : aA]] \mid [[GG , a]] \mid [[GG , evar]] \mid [[GG, evar = at]] \mid [[ GG , mevar]] $ \\
  Complete Contexts & $[[OO]]$ & \syndef & $[[empty]] \mid [[OO , x : aA]] \mid [[OO , a]] \mid [[OO, evar = at]] \mid [[ OO, mevar]]   $ \\ \bottomrule
\end{tabular}
  \end{small}
\caption{Syntax of the algorithmic system}
\label{fig:algo-syntax}
\end{figure}


\Cref{fig:algo-syntax} shows the syntax of the algorithmic system. A
noticeable difference are the
algorithmic contexts $[[GG]]$, which are represented as an
\emph{ordered} list containing declarations of type variables $[[a]]$ and term
variables $[[x : aA]]$. Unlike declarative contexts, algorithmic contexts also
contain declarations of existential type variables $[[evar]]$, which can be either
unsolved (written $[[evar]]$) or solved to some monotype (written $[[evar = at]]$). Finally, algorithmic contexts
include a \textit{marker} $[[ mevar  ]]$ (read ``marker $[[evar]]$'' ), which is used to delineate existential variables created by the algorithm. We will have more
to say about markers when we examine the rules.
Complete contexts $[[OO]]$ are the same as contexts, except that they contain no unsolved variables.

Apart from expressions in the declarative system, we add annotated expressions
$[[ ae : aA ]]$. The well-formedness judgments for types and contexts are shown in \cref{fig:well-formedness}.

\begin{figure}[t]
  \centering
  \begin{small}

\drules[ad]{$ [[GG |- aA ]] $}{Well-formedness of types}{int, unknown, tvar, evar, solved, arrow, forall}

\drules[wf]{$ [[ |- GG ]] $}{Well-formedness of algorithmic contexts}{empty, var, tvar, evar, solved, marker}

  \end{small}
  \caption{Well-formedness of types and contexts in the algorithmic system}
  \label{fig:well-formedness}
\end{figure}

\paragraph{Notational convenience}
Following DK system, we use contexts as substitutions on
types. We write $[[ [GG]aA  ]]$ to mean $[[GG]]$ applied as a
substitution to type $[[aA]]$. We also use a hole notation, which is useful when
manipulating contexts by inserting and replacing declarations in the middle. The
hole notation is used extensively in proving soundness and completeness. For
example, $[[GG]] [ [[TT]]   ]$ means $[[GG]]$ has the form $[[GL]], [[TT]], [[GR]]$;
if we have $[[ GG[evar]  ]] = [[(GL, evar, GR)]]$, then
$[[  GG[evar = at]  ]] = [[(GL, evar = at, GR)]] $. Occasionally,
we will see a context with two \emph{ordered} holes, e.g., $[[GG]] = [[ GG0[TT1][TT2] ]]$ means $[[GG]]$ has the form
$[[ GL, TT1, GM, TT2, GR ]]$.

\paragraph{Input and output contexts}
The algorithmic system, compared with the declarative system, includes similar
judgment forms, except that we replace the declarative context $[[dd]]$ with an
algorithmic context $[[GG]]$ (the \emph{input context}), and add an
\emph{output context} $[[DD]]$ after a backward turnstile, e.g.,
$[[  GG |- aA <~ aB -| DD  ]]$ is the judgment form for the
algorithmic consistent subtyping. All algorithmic rules manipulate input and
output contexts in a way that is consistent with the notion of
\emph{context extension}, which will be described in \cref{sec:ctxt:extension}.

We start with the explanation of the algorithmic consistent subtyping as it
involves manipulating existential type variables explicitly (and solving them if
possible).

\subsection{Algorithmic Consistent Subtyping}
\label{sec:algo:subtype}

\begin{figure}[t]
  \centering
  \begin{small}

\drules[as]{$ [[GG |- aA <~ aB -| DD ]] $}{Under input context $[[GG]]$, $[[aA]]$ is a consistent subtype of $[[aB]]$, with output context $[[DD]]$}{tvar, int, evar, unknownL, unknownR, arrow, forallR, forallL,  instL, instR}

  \end{small}
  \caption{Algorithmic consistent subtyping}
  \label{fig:algo:subtype}
\end{figure}

\Cref{fig:algo:subtype} presents the rules of algorithmic consistent subtyping
$[[GG |- aA <~ aB -| DD ]]$, which says that under input context $[[GG]]$,
$[[aA]]$ is a consistent subtype of $[[aB]]$, with output context $[[DD]]$. The
first five rules do not manipulate contexts, but illustrate how contexts are
propagated.

\Rref{as-tvar,as-int} do not involve existential variables, so the output
contexts remain unchanged. \Rref{as-evar} says that any unsolved existential
variable is a consistent subtype of itself. The output is still the same as the
input context as the rule gives no clue as to what is the solution of that
existential variable. \Rref{as-unknownL,as-unknownR} are the 
counterparts of \rref{cs-unknownL,cs-unknownR}.

\Rref{as-arrow} is a natural extension of its declarative counterpart. The
output context of the first premise is used by the second premise, and the
output context of the second premise is the output context of the conclusion.
Note that we do not simply check $[[ A2 <~ B2 ]]$, but apply $[[TT]]$ (the input
context of the second premise) to both types (e.g., $[[ [TT]aA2 ]]$). This is to
maintain an important invariant: whenever $[[ GG |- aA <~ aB -| DD ]]$ holds, the types $[[aA]]$ and $[[aB]]$ are fully applied under input context
$[[GG]]$ (they contain no existential variables already solved in $[[GG]]$). The
same invariant applies to every algorithmic judgment.


\Rref{as-forallR}, similar to the declarative rule \rref*{cs-forallR}, adds $[[a]]$ to the input context.
Note that the output context of the premise allows additional existential variables to appear
after the type variable $[[a]]$, in a trailing context $[[TT]]$. These existential variables
could depend on $[[a]]$; since $[[a]]$
goes out of scope in the conclusion, we need to drop them from
the concluding output, resulting in $[[DD]]$.
The next rule is essential to eliminating the guessing work. Instead of
guessing a monotype $[[t]]$ out of thin air, \rref{as-forallL} generates a fresh
existential variable $[[evar]]$, and replaces $[[a]]$ with $[[evar]]$ in the
body $[[aA]]$. The new existential variable $[[evar]]$ is then added to the input context,
just before the
marker $[[mevar]]$. The output context ($[[ DD, mevar, TT   ]]$) allows additional existential variables to
appear after $[[mevar]]$ in $[[TT]]$.
For the same reasons as in \rref{as-forallR}, we drop them from the output context.
A central idea behind these two rules is that we defer the decision of picking a
monotype for a type variable, and hope that it could be solved later when we
have more information at hand. As a side note, when both types are universal quantifiers,
then either \rref{as-forallR} or \rref*{as-forallL} could be tried.
In practice, one can apply \rref{as-forallR} eagerly as it is invertible.


The last two rules (\rref*{as-instL,as-instR}) are specific to the algorithm,
thus having no counterparts in the declarative version. They both check
consistent subtyping with an unsolved existential variable on one side and an
arbitrary type on the other side. Apart from checking that the existential
variable does not occur in the type $A$, both rules do not directly solve the
existential variables, but leave the real work to the instantiation judgment.

\subsection{Instantiation}
\label{sec:algo:instantiate}

\begin{figure}[t]
  \centering
  \begin{small}

\drules[instl]{$ [[ GG |- evar <~~ aA -| DD   ]] $}{Under input context $[[GG]]$, instantiate $[[evar]]$ such that $[[evar <~ aA]]$, with output context $[[DD]]$}{solve, solveU, reach, forallR, arr}

\drules[instr]{$ [[ GG |- aA <~~ evar -| DD   ]] $}{Under input context $[[GG]]$, instantiate $[[evar]]$ such that $[[aA <~ evar]]$, with output context $[[DD]]$}{solve, solveU, reach, forallL, arr}

  \end{small}
  \caption{Algorithmic instantiation}
  \label{fig:algo:instantiate}
\end{figure}

% A central idea of the algorithmic system is to defer the decision of picking a
% monotype to as late as possible.
Two symmetric judgments $[[ GG |- evar <~~ aA -| DD ]]$ and $[[ GG |- aA <~~ evar -| DD ]]$
defined in \cref{fig:algo:instantiate} instantiate unsolved
existential variables. They read ``under input context $[[GG]]$, instantiate $[[evar]]$ to a consistent
subtype (or supertype) of $[[aA]]$, with output context $[[DD]]$''. The judgments are extended naturally from
DK system, whose original inspiration comes from
\citet{cardelli1993implementation}. Since these two judgments are mutually
defined, we discuss them together.
% , and omit symmetric rules when convenient.

\Rref{instl-solve} is the simplest one -- when an existential variable meets a
monotype -- where we simply set the solution of $[[evar]]$ to the monotype
$[[at]]$ in the output context. We also need to check that the monotype $[[at]]$
is well-formed under the prefix context $[[GG]]$.

\Rref{instl-solveU} is similar to \rref{as-unknownR} in that we put no
constraint\footnote{As we will see in \cref{sec:advanced-extension} where we
  present a more refined system, the ``no constraint'' statement is not entirely
  true. } on $\genA$ when it meets the unknown type $\unknown$. This design
decision reflects the point that type inference only produces static
types~\citep{garcia2015principal}.


\Rref{instl-reach} deals with the situation where two existential variables
meet. Recall that $[[ GG[evar][evarb]  ]]$ denotes a context where some unsolved existential
variable $[[evar]]$ is declared before $[[evarb]]$. In this situation, the only logical
thing we can do is to set the solution of one existential variable to the other
one, depending on which one is declared before. For example, in the output
context of \rref{instl-reach}, we have $[[evarb]] = [[evar]]$ because in the input
context, $[[evar]]$ is declared before $[[evarb]]$.

\Rref{instl-forallR} is the instantiation version of \rref{as-forallR}.
Since our system is predicative, $[[evar]]$ cannot be instantiated to $[[\/b. B]]$,
but we can decompose $[[ \/b. B   ]]$ in the same way as in \rref{as-forallR}.
\Rref{instr-forallL} is the instantiation version of \rref{as-forallL}.

\Rref{instl-arr} applies when $[[evar]]$ meets an arrow type. It
follows that the solution must also be an arrow type.
This is why, in the first premise, we generate two fresh existential variables
$[[evar1]]$ and $[[evar2]]$, and insert them just before $[[evar]]$
in the input context, so that we can solve $[[evar]]$ to $[[ evar1 -> evar2 ]]$.
Note that the first premise $[[ aA1 <~~ evar1 ]]$ switches to the other instantiation judgment.


% \paragraph{Example}

% We show a derivation of $\Gamma[\genA] \vdash \forall b. b \to \unknown \unif
% \genA$ to demonstrate the interplay between instantiation, quantifiers and the
% unknown type:
% \[
%   \inferrule*[right=InstRAllL]
%       {
%         \inferrule*[right=InstRArr]
%         {
%           \inferrule*[right=InstLReach]{ }{\Gamma', \genB \vdash \genA_1 \unif \genB \dashv \Gamma' , \genB = \genA_1} \\
%           \inferrule*[right=InstRSolveU]{ }{\Gamma', \genB = \genA_1 \vdash \unknown \unif \genA_2 \dashv \Gamma', \genB = \genA_1}
%         }
%         {
%           \Gamma[\genA], \genB \vdash \genB \to \unknown \unif \genA \dashv \Gamma', \genB = \genA_1
%         }
%       }
%       {
%         \Gamma[\genA] \vdash \forall b. b \to \unknown \unif \genA \dashv \Gamma', \genB = \genA_1
%       }
% \]
% where $\Gamma' = \Gamma[\genA_2, \genA_1, \genA = \genA_1 \to \genA_2]$. Note
% that in the output context, $\genA$ is solved to $\genA_1 \to \genA_2$, and
% $\genA_2$ remains unsolved because the unknown type $\unknown$ puts no
% constraint on it. Essentially this means that the solution of $\genA$ can be any
% function, which is intuitively correct since $\forall b. b \to \unknown$ can be
% interpreted, from the parametricity point of view, as any function.

\subsection{Algorithmic Typing}
\label{sec:algo:typing}

\begin{figure}[t]
  \centering
  \begin{small}


\drules[inf]{$ [[ GG |- ae => aA -| DD ]] $}{Under input context $[[GG]]$, $[[ae]]$ infers output type $[[aA]]$, with output context $[[DD]]$}{var, int, anno, lamann, lam, let, app}

\drules[chk]{$ [[ GG |- ae <= aA -| DD ]] $}{Under input context $[[GG]]$, $[[ae]]$ checks against input type $[[aA]]$, with output context $[[DD]]$}{lam, gen, sub}

\drules[am]{$ [[ GG |- aA |> aA1 -> aA2 -| DD ]] $}{Under input context $[[GG]]$, $[[aA]]$ matches output type $[[aA1 -> aA2]]$, with output context $[[DD]]$}{forall, arr, unknown, var}

  \end{small}
  \caption{Algorithmic typing}
  \label{fig:algo:typing}
\end{figure}

We now turn to the algorithmic typing rules in \cref{fig:algo:typing}. Because
general type inference for System F is undecidable~\citep{WELLS1999111}, our algorithmic
system uses bidirectional type checking to accommodate (first-class)
polymorphism. Traditionally, two modes are employed in bidirectional systems:
the checking mode $[[ GG |- ae <= aA -| TT ]]$, which takes a term $[[ae]]$ and a type $[[aA]]$ as
input, and ensures that the term $[[ae]]$ checks against $[[aA]]$; the
inference mode $[[ GG |- ae => aA -| TT ]]$, which takes a term $[[ae]]$ and
produces a type $[[aA]]$. We first discuss rules in the inference mode.

\Rref{inf-var,inf-int} do not generate any new information and simply propagate
the input context. \Rref{inf-anno} is standard, switching to the checking
mode in the premise.

In \rref{inf-lamann}, we generate a fresh existential variable $[[evarb]]$ for
the function codomain, and check the function body against $[[evarb]]$. Note that
it is tempting to write $[[ GG, x : aA |- ae => aB -| DD, x : aA, TT ]]$ as the
premise (in the hope of better matching its declarative counterpart \rref{lamann}), which has
a subtle consequence. Consider the expression $[[ \x : int. \y . y ]]$. Under
the new premise, this is untypable because of
$[[ empty |- \x : int. \y . y => int -> evar -> evar -| empty ]]$ where $[[evar]]$ is not found
in the output context. This explains why we put $[[evarb]]$ \emph{before} $ [[ x : aA ]]  $
so that it remains in the output context $[[DD]]$.
\Rref{inf-lam}, which corresponds to \rref{lam}, one of the
guessing rules, is similar to \rref{inf-lamann}. As with the other algorithmic
rules that eliminate guessing, we create new existential variables $[[evar]]$
(for function domain) and $[[evarb]]$ (for function codomain) and check the function body against $[[evarb]]$.
\Rref{inf-let} is similar to \rref{inf-lamann}.

\paragraph{Algorithmic Matching}
\Rref{inf-app} (which differs significantly from that of \cite{dunfield2013complete}) deserves
attention. It relies on the algorithmic matching judgment $[[ GG |- aA |> aA1 ->
aA2 -| DD ]]$. The matching judgment algorithmically synthesizes an arrow type from an
arbitrary type. \Rref{am-forall} replaces $a$ with a fresh existential variable
$[[evar]]$, thus eliminating guessing. \Rref{am-arr,am-unknown} correspond
directly to the declarative rules. \Rref{am-var}, which has no corresponding
declarative version, is similar to \rref{instl-arr}/\rref*{instr-arr}: we create
$[[evar1]]$ and $[[evar2]]$ and solve $[[evar]] $ to $ [[evar1 -> evar2]]$ in the output context.

Back to the \rref{inf-app}. This rule first infers the type of $[[e1]]$, producing an output
context $[[TT1]]$. Then it applies $[[TT1]]$ to $A$ and goes into the matching
judgment, which delivers an arrow type $[[aA1 -> aA2]]$ and another output
context $[[TT2]]$. $[[TT2]]$ is used as the input context when checking $[[e2]]$
against $[[ [TT2]aA1 ]]$, where we go into the checking mode.


Rules in the checking mode are quite standard. \Rref{chk-lam} checks against
$[[aA -> aB]]$. \Rref{chk-gen}, like the declarative \rref{gen},
adds a type variable $[[a]]$ to the input context. % In both rules, the trailing
% context is discarded, since $[[x]]$ and $[[a]]$ go out of scope in the
% conclusion.
\Rref{chk-sub} uses the algorithmic consistent subtyping judgment.



\subsection{Decidability}

Our algorithmic system is decidable. It is not at all obvious to see
why this is the case, as many rules
are not strictly structural (e.g., many rules have $[[ [GG]aA ]]$ in the
premises). This implies that we need a more sophisticated measure to
support the argument. Since the typing rules (\cref{fig:algo:typing}) depend on
the consistent subtyping rules (\cref{fig:algo:subtype}), which in turn depends
on the instantiation rules (\cref{fig:algo:instantiate}), to show the
decidability of the typing judgment, we need to show that the instantiation and
consistent subtyping judgments are decidable. The proof strategy mostly follows
that of the DK system. Here only highlights of the proofs are given; the full proofs can be found
in \cref{app:decidable}.



\paragraph{Decidability of Instantiation}

The basic idea is that we need to show $[[aA]]$ in the instantiation judgments
$[[GG |- evar <~~ aA -| DD ]]$ and $[[GG |- aA <~~ evar -| DD]]$ always gets
smaller. Most of the rules are structural and thus easy to verify (e.g.,
\rref{instl-forallR}); the non-trivial cases are \rref{instl-arr,instr-arr}
where context applications appear in the premises. The key observation there
is that the instantiation rules preserve the size of (substituted) types.
The formal statement of decidability of instantiation needs a few
pre-conditions: assuming $[[evar]]$ is unsolved in the input context
$[[GG]]$, that $[[aA]]$ is well-formed under the context $[[GG]]$, that $[[aA]]$
is fully applied under the input context $[[GG]]$ ($[[ [GG]aA ]] = [[aA]]$), and
that $[[evar]]$ does not occur in $[[aA]]$. Those conditions are actually met
when instantiation is invoked: \rref{chk-sub} applies the input context, and the
subtyping rules apply input context when needed.

\begin{restatable}[Decidability of Instantiation]{theorem}{decInstan}
  If $[[GG]]  = [[ GG0[evar]  ]]  $ and $[[  GG |- aA  ]]$ such that $[[ [GG]aA  ]] = [[aA]]$ and $[[ evar notin fv(aA)   ]]$ then:
  \begin{enumerate}
  \item Either there exists $[[DD]]$ such that $[[GG |- evar <~~ aA -| DD]]$, or not.
  \item Either there exists $[[DD]]$ such that $[[GG |- aA <~~ evar -| DD]]$, or not.
  \end{enumerate}
\end{restatable}


\paragraph{Decidability of Algorithmic Consistent Subtyping}

Proving decidability of the algorithmic consistent subtyping is a bit more
involved, as the induction measure consists of several parts.
We measure the judgment $[[GG |- aA <~ aB -| DD]]$ lexicographically by
\begin{enumerate}[(M1)]
\item the number of $\forall$-quantifiers in $[[aA]]$ and $[[aB]]$;
\item \hl{the number of unknown types in $[[aA]]$ and $[[aB]]$;}
\item $| \textsc{unsolved}([[GG]]) |$: the number of unsolved existential
  variables in $[[GG]]$;
\item $| [[GG |- aA]] | + | [[GG |- aB]]   |$.
\end{enumerate}
Notice that because of our gradual setting, we also need to measure the number
of unknown types (M2). This is a key difference from the DK system.
We refer the reader to \cref{app:decidable} for more details.
For (M4), we
use \emph{contextual size}---the size of well-formed types under certain
contexts, which penalizes solved variables (*).

\begin{definition}[Contextual Size] \leavevmode
  \begin{center}
  \begin{tabular}{llll}
    $|  [[GG |- int]] |$     &=&  $1$ \\
    $| [[GG |- unknown]]    |$  &=&  $1$ \\
    $|  [[GG |- a ]]  |$    &=&  $1$ \\
    $|  [[GG |- evar]]  |$    &=&  $1$ \\
    $|  [[ GG[evar = at] |- evar ]] |$    &=&  $1 + | [[  GG[evar = at] |- at   ]]   | $ & $(*)$ \\
    $|  [[ GG |- \/a . aA ]] |$    &=&  $1 + | [[  GG, a |- aA ]]   | $ \\
    $|  [[ GG |- aA -> aB ]] |$    &=&  $1 + | [[  GG |- aA ]]   | + | [[ GG |- aB ]] | $ \\
  \end{tabular}
\end{center}
\end{definition}


\begin{restatable}[Decidability of Algorithmic Consistent Subtyping]{theorem}{decsubtype} \label{thm:decsubtype}
  Given a context $[[GG]]$ and types $[[aA]]$, $[[aB]]$ such that $[[GG |- aA]]$ and
  $[[ GG |- aB ]]$ and $[[  [GG]aA  ]] = [[aA]]$ and $[[ [GG]aB ]] = [[aB]]$, it is decidable whether there exists $[[DD]]$ such that $[[ GG |- aA <~ aB -| DD    ]]$.
\end{restatable}


\paragraph{Decidability of Algorithmic Typing}


Similar to proving decidability of algorithmic consistent subtyping, the key is
to come up with a correct measure. Since the typing rules depend on the matching
judgment, we first show decidability of the algorithmic matching.

\begin{restatable}[Decidability of Algorithmic Matching]{lemma}{decmatch} \label{lemma:decmatch}%

  Given a context $[[GG]]$ and a type $[[aA]]$ it is decidable whether there
  exist types $[[aA1]]$, $[[aA2]]$ and a context $[[DD]]$ such that $[[ GG |- aA |> aA1 -> aA2 -| DD ]]$.
\end{restatable}

Now we are ready to show decidability of typing. The proof is obtained by
induction on the lexicographically ordered triple: size of $[[ae]]$, typing
judgment (where the inference mode $[[=>]]$ is considered smaller than the
checking mode $[[<=]]$) and contextual size.
\begin{center}
\begin{tabular}{lllll}
  \multirow{2}{*}{$ \Big \langle$} & \multirow{2}{*}{$[[ae]],$} & $[[=>]]$ & \multirow{2}{*}{$| [[GG |- aA]] |$} & \multirow{2}{*}{$\Big \rangle$} \\
                                   &                    & $[[<=]],$ &  &
\end{tabular}
\end{center}
The above measure is much simpler than the corresponding one in the DK system,
where they also need to consider the application judgment together with the
inference and checking judgments. This shows another benefit (besides the
independence from typing) of adopting the matching judgment.

\begin{restatable}[Decidability of Algorithmic Typing]{theorem}{dectyping} \leavevmode
  \begin{enumerate}
  \item Inference: Given  a context $[[GG]]$ and a term $[[ae]]$, it is decidable whether
    there exist a type $[[aA]]$ and a context $[[DD]]$ such that $[[GG |- ae => aA -| DD]]$.
  \item Checking: Given a context $[[GG]]$, a term $[[ae]]$ and a type $[[aB]]$
    such that $[[GG |- aB]]$, it is decidable whether there exists a context
    $[[DD]]$ such that $[[  GG |- ae <= aB -| DD  ]]$.
  \end{enumerate}
\end{restatable}


\section{Soundness and Completeness}
\label{sec:sound:complete}

To be confident that our algorithmic type system and the declarative type system
agree with each other, we need to prove that the algorithmic rules are sound and
complete with respect to the declarative specification. Before we give the
formal statements of the soundness and completeness theorems, we need a
meta-theoretical device, called \emph{context
  extension}~\cite{dunfield2013complete}, to capture a notion of
information increase from input contexts to output contexts.

\subsection{Context Extension}
\label{sec:ctxt:extension}


A context extension judgment $\Gamma \exto \Delta$ reads ``$\Gamma$ is extended
by $\Delta$''. Intuitively, this judgment says that $\Delta$ has at least as
much information as $\Gamma$: some unsolved existential variables in $[[GG]]$
may be solved in $[[DD]]$. The full inductive definition can be found
\cref{fig:ctxt-extension}. We refer the reader to \citet[][Section
4]{dunfield2013complete} for further explanation of context extension.


\begin{figure}[t]
  \centering

  \begin{small}
\drules[ext]{$ [[ GG --> DD  ]] $}{Context extension}{id, var, tvar, evar, solved, solve, add, addSolve, marker}
  \end{small}
  

\caption{Context extension}
\label{fig:ctxt-extension}
\end{figure}


% \subsection{Completeness and Soundness}

% We prove that the algorithmic rules are sound and complete with
% respect to the declarative specifications. We need an auxiliary judgment
% $\Gamma \exto \Delta$ that captures a notion of information increase from input
% contexts $\Gamma$ to output contexts $\Delta$~\citep{dunfield2013complete}.

\subsection{Soundness}

Roughly speaking, soundness of the algorithmic system says
that given a derivation of an algorithmic judgment with input context
$[[GG]]$, output context $[[DD]]$, and a complete context $[[OO]]$
that extends $[[DD]]$, applying $[[OO]]$ throughout the given algorithmic judgment
should yield a derivable declarative judgment. For example, let us consider
an algorithmic typing judgment $[[ empty |- \x . x => evar -> evar -| evar  ]]$, and any complete context, say, $[[OO]] = ([[ evar = int ]])$,
then applying $[[OO]]$ to the above judgment yields $[[ empty |- \x . x : int -> int  ]]$, which is derivable in the declarative system.

However there is one complication: applying $[[OO]]$ to the algorithmic expression does not necessarily yield a typable declarative expression.
For example, by \rref{chk-lam} we have $[[ \x .x   ]] \Leftarrow [[ (\/a. a -> a) -> (\/a. a -> a)   ]]   $,
but $[[ \x . x]]$ itself cannot have type
$[[ (\/a. a -> a) -> (\/a. a -> a)   ]]$ in the declarative system. To
circumvent that, we add an annotation to the lambda abstraction, resulting in
$[[ \x : (\/a . a -> a) . x   ]]$, which is typeable in the declarative system with
the same type. To relate $[[ \x . x   ]]$ and $[[ \x : (\/a. a -> a) . x   ]]$, we
erase all annotations on both expressions. 
\begin{definition}[Type annotation erasure] The erasure function is denoted as
  $\erase{\cdot}$, and defined as follows:
\begin{center}
\begin{tabular}{p{5cm}l}
    $\erase{x} = x $&
    $\erase{n} = n $\\
    $\erase{\blam{x}{A}{e}} = \erlam{x}{\erase{e}} $&
    $\erase{\erlam{x}{e}} = \erlam{x}{\erase{e}} $\\
    $\erase{e_1 ~ e_2} = \erase{e_1} ~ \erase{e_2} $&
    $\erase{e : A } = \erase{e}$
\end{tabular}
\end{center}
\end{definition}



\begin{restatable}[Instantiation Soundness]{theorem}{instsoundness} \label{thm:inst_soundness}%
  Given $[[ DD --> OO ]]$ and $[[ [GG]aA = aA ]]$ and  $[[evar notin fv(aA)]]$:

  \begin{enumerate}
  \item If $[[GG |- evar <~~  aA -| DD ]]$ then $[[  [OO]DD |- [OO]evar <~ [OO]aA  ]] $.
  \item If $[[GG |- aA <~~ evar -| DD ]]$ then $[[  [OO]DD |- [OO]aA <~ [OO]evar  ]] $.
  \end{enumerate}
\end{restatable}

Notice that the declarative judgment uses $[[ [OO]DD   ]]$, an
operation that applies a complete context $[[OO]]$ to the algorithmic context
$[[DD]]$, essentially plugging in all known solutions and removing all
declarations of existential variables (both solved and unsolved), resulting in a
declarative context.

With instantiation soundness, next we show that the algorithmic consistent
subtyping is sound:

\begin{restatable}[Soundness of Algorithmic Consistent Subtyping]{theorem}{subsoundness} \label{thm:sub_soundness}%
  If $[[  GG |- aA <~ aB -| DD ]]$ where $[[ [GG]aA = aA  ]]$ and $[[  [GG] aB = aB  ]]$ and $[[  DD --> OO ]]$ then
  $[[  [OO]DD |- [OO]aA <~ [OO]aB   ]]$.
\end{restatable}

Finally the soundness theorem of algorithmic typing is:

\begin{restatable}[Soundness of Algorithmic Typing]{theorem}{typingsoundness} \label{thm:type_sound}%
  Given $[[DD --> OO]]$:
  \begin{enumerate}
  \item If $[[  GG |- ae => aA -| DD  ]]$ then $\exists [[e']]$ such that $ [[  [OO]DD |- e' : [OO] aA  ]]   $ and $\erase{[[ae]]} = \erase{[[e']]}$.
  \item If $[[  GG |- ae <= aA -| DD  ]]$ then $\exists [[e']]$ such that $ [[  [OO]DD |- e' : [OO] aA  ]]   $ and $\erase{[[ae]]} = \erase{[[e']]}$.
  \end{enumerate}
\end{restatable}

\subsection{Completeness}

Completeness of the algorithmic system is the reverse of soundness: given a
declarative judgment of the form $[[  [OO]GG  ]] \vdash
\ctxsubst{\Omega} \dots $, we want to get an algorithmic derivation of $\Gamma
\vdash \dots \dashv \Delta$. It turns out that completeness is a bit trickier to
state in that the algorithmic rules generate existential variables on the fly,
so $\Delta$ could contain unsolved existential variables that are not found in
$\Gamma$, nor in $\Omega$. Therefore the completeness proof must produce another
complete context $\Omega'$ that extends both the output context $\Delta$, and
the given complete context $\Omega$. As with soundness, we need erasure to
relate both expressions.


\begin{restatable}[Instantiation Completeness]{theorem}{instcomplete} \label{thm:inst_complete}
  Given $[[GG --> OO]]$ and $[[aA = [GG]aA]]$ and $[[evar]] \notin \textsc{unsolved}([[GG]]) $ and $[[  evar notin fv(aA)  ]]$:
  \begin{enumerate}
  \item If $[[ [OO]GG |- [OO] evar <~ [OO]aA   ]]$ then there are $[[DD]], [[OO']]$ such that $[[OO --> OO']]$
    and $[[DD --> OO']]$ and $[[GG |- evar <~~ aA -| DD]]$.
  \item If $[[ [OO]GG |- [OO]aA  <~ [OO] evar  ]]$ then there are $[[DD]], [[OO']]$ such that $[[OO --> OO']]$
    and $[[DD --> OO']]$ and $[[GG |- aA <~~ evar -| DD]]$.
  \end{enumerate}

\end{restatable}

Next is the completeness of consistent subtyping:

\begin{restatable}[Generalized Completeness of Consistent Subtyping]{theorem}{subcomplete} \label{thm:sub_completeness}
  If $[[ GG --> OO  ]]$ and $[[ GG |- aA  ]]$ and $[[ GG |- aB  ]]$ and $[[  [OO]GG |- [OO]aA <~ [OO]aB  ]]$ then
  there exist $[[DD]]$ and $[[OO']]$ such that $[[DD --> OO']]$ and $[[OO --> OO']]$ and $[[  GG |- [GG]aA <~ [GG]aB -| DD ]]$.
\end{restatable}

We prove that the algorithmic matching is complete with respect to the
declarative matching:


\begin{restatable}[Matching Completeness]{theorem}{matchcomplete} \label{thm:match_complete}%
  Given $[[ GG --> OO  ]]$ and $[[ GG |- aA  ]]$, if
  $[[ [OO]GG |- [OO]aA |> A1 -> A2  ]]$
  then there exist $[[DD]]$, $[[OO']]$, $[[aA1']]$ and $[[aA2']]$ such that $[[ GG |- [GG]aA |> aA1' -> aA2' -| DD   ]]$
  and $[[ DD --> OO'  ]]$ and $[[ OO --> OO'  ]]$ and $[[A1]] = [[ [OO']aA1'  ]]$ and $[[A2]] = [[ [OO']aA2'  ]]$.
\end{restatable}


Finally here is the completeness theorem of the algorithmic typing:

\begin{restatable}[Completeness of Algorithmic Typing]{theorem}{typingcomplete}
  \label{thm:type_complete}
  Given $[[GG --> OO]]$ and $[[GG |- aA]]$, if $[[ [OO]GG |- e : A ]]$ then there exist $[[DD]]$, $[[OO']]$, $[[aA']]$ and $[[ae']]$
  such that $[[DD --> OO']]$ and $[[OO --> OO']]$ and $[[  GG |- ae' => aA' -| DD  ]]$ and $[[A]] = [[ [OO']aA'  ]]$ and $\erase{[[e]]} = \erase{[[ae']]}$.
\end{restatable}



%%% Local Variables:
%%% mode: latex
%%% TeX-master: "../paper"
%%% org-ref-default-bibliography: "../paper.bib"
%%% End:
