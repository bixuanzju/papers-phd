
\section{Conclusion}
\label{sec:conclusion}

In this paper, we have presented a generalized definition of consistent subtyping
that works for polymorphic types.
Based on this new
definition, we have developed \name: a gradually typed calculus
with predicative implicit higher-rank polymorphism, and corresponding
algorithms that can be used to
implement the calculus.

As far as we know, our work is the first to integrate gradual typing with
implicit (higher-rank) polymorphism, which we believe is a major step
towards gradualizing modern functional languages, such as Haskell. Moreover, our
extension with type parameters and the extensive discussion of related
properties (e.g., representative translations) provides insight into the
dynamic semantics for gradual languages with implicit polymorphism.
With respect to the dynamic gradual guarantee, we discuss an extension
of the calculus with static and gradual type parameters. We propose a
variant of the dynamic gradual guarantee with representative
translations. Then we show that our calculus supports this property
if:
1) \pbc does indeed have the dynamic gradual guarantee (which is unknown at the
time of writing); and 2) our coherence conjecture can be proved.

As future work, we want to investigate whether our notion of consistent
subtyping has a more fundamental conceptual explanation, for example, whether it
coincides with AGT on polymorphic types. It is also interesting to see whether
our results can scale to real-world languages (e.g. Haskell) and other
programming language features, such as recursive types, union types and
intersection types. Recent work by \citet{castagna2017gradual} on gradual typing
with union and intersection types in a simply typed setting may shed some light
on this direction.


%%% Local Variables:
%%% mode: latex
%%% TeX-master: "../paper"
%%% org-ref-default-bibliography: "../paper.bib"
%%% End:
