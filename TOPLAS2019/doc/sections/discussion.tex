
\section{Simple Extensions and Variants}
\label{sec:discussion}

This section considers two simple variations on the presented system.
The first variation extends the system with a top type, while the
second variation considers a more declarative formulation using a
subsumption rule.

% In this section we consider top types to further
% demonstrate the applicability of our definition of consistent subtyping. We also
% discuss the design decisions involved in the declarative/algorithmic system.
% and show how it
% helps address the issue arising from the ambiguity of the type-directed
% translation.

\subsection{Top Types}
\label{subsec:extension-top}

%To demonstrate that our definition of consistent subtyping (\cref{def:decl-conssub}) is
%applicable to other features,
We argued that our definition of consistent subtyping (\cref{def:decl-conssub})
generalizes the original definition by \citet{siek2007gradual}.
We have shown its applicability to polymorphic types, for which 
\citet{siek2007gradual} approach cannot
be extended naturally. To strengthen our argument,
we show how to extend our approach to $\tope$ types, which 
is also not supported by \citet{siek2007gradual} approach. 

%To aid comparison, we also
%show how to adapt the AGT approach to support $\tope$ and verify that these two
%approaches, though rooted in different foundations, coincide again on
%\emph{simple types}. However, \cref{def:old-decl-conssub} of



\paragraph{Consistent Subtyping with $\top$}
In order to preserve the orthogonality between subtyping and consistency, we
require $\top$ to be a common supertype of all static types, as shown in 
\rref{S-Top}. This rule might seem strange at first glance, since even
if we remove
the requirement $A~\mathsf{static}$, the rule still seems reasonable.
However, an important point is that, because of the orthogonality between
subtyping and consistency, subtyping itself should not contain a potential
information loss!
Therefore, subtyping instances such as $\unknown \tsub \top$ are not allowed.
For consistency, we add the rule that $\top$ is consistent with $\top$, which is
actually included in the original reflexive rule $A \sim A$. For consistent
subtyping, every type is a consistent subtype of $\top$, for example, $\nat \to
\unknown \tconssub \top$.
\begin{mathpar}
  \SubTop \and \CTop \and \CSTop
\end{mathpar}
It is easy to verify that \cref{def:decl-conssub} is still equivalent to that in
\cref{fig:decl:conssub} extended with \rref{CS-Top}. That is,
\cref{lemma:properties-conssub} holds:
\begin{restatable}[Extension with $\top$]{proposition}{proptop}
  \label{prop:top}
  $\tpreconssub A \tconssub B \Leftrightarrow \tpresub A \tsub C$, $C \sim D$, $\tpresub D \tsub B$, for some $C, D$.
\end{restatable}
% \begin{proof}\leavevmode
%   \begin{itemize}
%   \item From first to second: By induction on the derivation of consistent
%     subtyping. We have extra case \rref{CS-Top} now, where $B = \top$.
%     We can choose $C = A$, and
%     $D$ by replacing the unknown types in $C$ by $\nat$. Namely, $D$ is a static
%     type, so by \rref{S-Top} we are done.
%   \item From second to first: By induction on the derivation of second
%     subtyping. We have extra case \rref{S-Top} now, where
%     $B = \top$, so $A \tconssub B$ holds by \rref{CS-Top}.
%   \end{itemize}
% \end{proof}

% \paragraph{Extending AGT}

We extend the definition of concretization (\cref{def:concret}) with $\top$
by adding another equation $\gamma(\top) = \{\top\}$. Note that
\citet{castagna2017gradual} also have this equation in their calculus.
It is easy to verify that \cref{lemma:coincide-agt} still holds:
\begin{restatable}[Equivalent to AGT on $\top$]{proposition}{propagttop}
  \label{prop:agt-top}
  $A \tconssub B$ if only if $A \agtconssub B$.
\end{restatable}

% \begin{proof}\leavevmode
%   \begin{itemize}
%   \item From left to right: By induction on the derivation of consistent
%     subtyping. We have case \rref{CS-Top} now.
%     It follows that for
%     every static type $A_1 \in \gamma(A)$, we can derive $A_1 \tsub \top$ by
%     \rref{S-Top}.
%     We have $B_1 = B = \top$ and we are done.
%   \item From right to left: By induction on the derivation of subtyping and
%     inversion on the concretization. We have extra case \rref{S-Top} now, where
%     $B$ is $\top$. So
%     consistent subtyping directly holds.
%   \end{itemize}
% \end{proof}

\paragraph{\citeauthor{siek2007gradual}'s Definition of Consistent Subtyping Does Not Work for $\top$.}

As with the analysis in \cref{subsec:towards-conssub}, $\nat \to \unknown
\tconssub \top$ only holds when we first apply consistency, then subtyping.
However we cannot find a type $A$ such that
$\nat \to \unknown \tsub A$ and $A \sim \top$. The
following diagram depicts the situation:
\begin{center}
  \begin{tikzpicture}
   \matrix (m) [matrix of math nodes,row sep=3em,column sep=4em,minimum width=2em]
   {
     \emptyset & \top \\
     \nat \to \unknown &
     \nat \to \nat \\};

   \path[-stealth]
   (m-2-1) edge node [left] {$\tsub$} (m-1-1)
   (m-2-2) edge node [left] {$\tsub$} (m-1-2);

   \draw
   (m-1-1) edge node [above] {$\sim$} (m-1-2)
   (m-2-1) edge node [below] {$\sim$} (m-2-2);
 \end{tikzpicture}
\end{center}

\noindent Additionally we have a similar problem in
extending the restriction operator: \emph{non-structural} masking between
$\nat \to \unknown$ and $\top$ cannot be easily achieved.

Note that both the top and universally quantified types can be seen as special
cases of intersection types. Indeed, top is the intersection of the empty set,
while a universally quantified type is the intersection of the infinite set of
its instantiations \citep{Davies:2000:ITC:351240.351259}. Recall from
\cref{subsec:agt} that \citet{castagna2017gradual} shows that consistent
subtyping from AGT works well for intersection types, and our definition
coincides with AGT (\cref{lemma:coincide-agt} and \cref{prop:agt-top}). We thus
believe that our definition is compatible with conventional binary intersection
types as well. Yet, a rigorous formalization would be needed to substantiate
this belief.


\subsection{A More Declarative Type System}
\label{subsec:variant}

In \cref{sec:type-system} we present our declarative system in terms of the
matching and consistent subtyping judgments. The rationale behind this design
choice is that the resulting declarative system combines subtyping and type
consistency in the application rule, thus making it easier to design an
algorithmic system accordingly. Still, one may wonder if it is possible to
design a more declarative specification. For example, even though we mentioned
that the subsumption rule is incompatible with consistent subtyping, it might be
possible to accommodate a subsumption rule for normal subtyping (instead of
consistent subtyping). In this section, we discuss an alternative for the
design of the declarative system.

\paragraph{Wrong Design}
A naive design that does not work is to replace \rref{App} in
\cref{fig:decl-typing} with the following two rules:
\begin{mathpar}
  \inferrule{
          \dctx \byinf e : A 
       \\ A \tsub B
          }{
          \dctx \byinf e : B
          }\rname{V-Sub}

  \inferrule{
          \dctx \byinf e_1 : A 
       \\ \dctx \byinf e_2 : A_1
       \\ A \sim A_1 \to A_2
          }{
          \dctx \byinf e_1 ~ e_2 : A_2
          }\rname{V-App1}
\end{mathpar}
\Rref{V-Sub} is the standard subsumption rule: if an expression $e$ has type
$A$, then it can be assigned some type $B$ that is a supertype of $A$. \Rref{V-App1}
first infers that $e_1$ has type $A$, and $e_2$ has type $A_1$, then
it finds some $A_2$ so that $A$ is consistent with $A_1 \to A_2$.

There would be two obvious benefits of this variant if it did work:
firstly this approach closely resembles
the traditional declarative type systems for calculi with subtyping; secondly it
saves us from discussing various forms of $A$ in \rref{V-App1}, leaving the job
to the consistency judgment.

The design is wrong because of the information loss caused by the choice of
$A_2$ in \rref{V-App1}. Suppose we have $\dpreinf \mathsf{plus} : \nat \to \nat
\to \nat$, then we can apply it to $1$ to get
\[
  \inferrule{\dpreinf \mathsf{plus} : \nat \to \nat \to \nat
    \\ \dpreinf 1 : \nat
    \\ \nat \to \nat \to \nat \sim \nat \to \unknown \to \nat}
    {\dpreinf \mathsf{plus}~1 : \unknown \to \nat}\rname{V-App1}
\]
Further applying it to $\truee$ we get
\[
  \inferrule
  {\dpreinf \mathsf{plus}~1 \infto \unknown \to \nat
    \\ \dpreinf \truee : \bool
  \\ \unknown \to \nat \sim \bool \to \nat}
    {\dpreinf \mathsf{plus}~1~\truee : \nat}\rname{V-App1}
\]
which is obviously wrong! The type consistency
in \rref{V-App1} causes information loss for both the argument type $A_1$ and
the return type $A_2$. The problem is that information of $A_2$ can get lost again if it
appears in further applications.
The moral of the story is that we should be very
careful when using type consistency. We hypothesize that it is inevitable to do
case analysis for the type of the function in an application (i.e., $A$ in
\rref{V-App1}).


\paragraph{Proper Declarative Design}

The proper design refines the first variant by using a matching judgment to
carefully distinguish two cases for the typing result of $e_1$ in \rref{V-App1}:
(1) when it is an arrow type, and (2) when it is an unknown type. This
variant replaces \rref{App} in \cref{fig:decl-typing} with the following
rules:
\begin{mathpar}
  \inferrule{
          \tctx \byinf e : A 
       \\ A \tsub B
          }{
          \tctx \byinf e : B
          }\rname{V-Sub}
\and
  \inferrule{
          [[dd |- e : A]]
       \\ [[dd |- A |> A1 -> A2]]
       \\ [[dd |- e2 : A3]]
       \\ [[A1 ~ A3]]
          }{
          [[dd |- e1 e2 : A2]]
          }\rname{V-App2}
\and
\inferrule{
          }{
          [[dd |- A1 -> A2 |> A1 -> A2]]
          }
\and
\inferrule{
          }{
          [[dd |- unknown |> unknown -> unknown]]
          }
\end{mathpar}
\Rref{V-Sub} is the same as in the first variant. In \rref{V-App2}, we infer
that $e_1$ has type $A$, and use the matching judgment to get an arrow type $[[A1 ->
A2]]$. Then we need to ensure that the argument type $A_3$ is \emph{consistent
  with} (rather than a consistent subtype of) $A_1$, and use $A_2$ as the
result type of the application. The matching judgment only deals with two cases,
as polymorphic types are handled by \rref{V-Sub}. These rules are closely related
to the ones in \citet{siek2006gradual} and \citet{siek2007gradual}.

The more declarative nature of this system also implies that it is highly non-syntax-directed, 
and it does not offer any insight into combining subtyping and
consistency. We have proved in Coq the following lemmas to establish soundness
and completeness of this system with respect to our original system (to avoid
ambiguity, we use the notation $\byminf$ to indicate the more declarative
version):

\begin{clemma}[Completeness of $\byminf$]
  If $\tctx \byinf e : A$, then $\tctx \byminf e : A$.
\end{clemma}

\begin{clemma}[Soundness of $\byminf$]
  If $\tctx \byminf e : A$, then there exists some $B$, such that $\tctx \byinf e : B$ and $\tctx \bysub B \tsub A$.
\end{clemma}

% In the
% following we discuss the relation between this new type system and the one we
% presented in \cref{sec:type-system}.




\begin{comment}
\paragraph{Do soundness and completeness still hold?}

The reader may ask if the declarative system produces types such as $\unknown
\to \nat$ and $\unknown \to \bool$, but the algorithmic system computes the type
$\unknown \to \unknown$, does it imply that the algorithmic system is no longer
sound and complete with respect to the declarative system? The answer is no.
First of all, note that \cref{thm:type_complete} reads ``$\dots$ there exist
$\Delta$, $\Omega'$ and $A'$ such that $\dots$ $A = \ctxsubst{\Omega'}{A'}$''.
Now if $A'$ (which is produced by the algorithmic system) contains some unsolved
existential variables, it is up to us to pick solutions in $\Omega'$ to match up
with $A$ (which is produced by the declarative system). More concretely, let us
assume that the declarative system produces type $\unknown \to \nat$, and let
$\Omega = f : \forall a. a \to a$ and $\Delta = f : \forall a. a \to a, \genA$,
the completeness theorem asks if we can find a complete context $\Omega'$ that
extends both $\Delta$ and $\Omega$ such that $\ctxsubst{\Omega'}{(\unknown \to
  \genA)} = \unknown \to \nat$. It is obvious that $\Omega' = f : \forall a. a
\to a, \genA = \nat$ is one such complete context, and we have
$\ctxsubst{\Omega'}{(\unknown \to \genA)} = \unknown \to \nat$. So the
algorithmic system is still complete. Similar arguments apply to the soundness
theorem. 
Secondly, an observation (which follows from the soundness and
completeness theorems) is: if an
expression is typeable with many types in the declarative system, the same expression must be
typeable with a single type that contains some unsolved existential variables in the algorithmic
algorithmic system.
In that case, replacing them with $\unknown$ is the best
strategy for the sake of execution.

\paragraph{Existential variables do not indicate parametricity}

Another reading of the above example may suggest that the result type $\unknown
\to \genA$ implies parametricity, with the implication that $\genA$ can be
changed arbitrarily without affecting the runtime behaviour of the
program. A similar phenomenon is discussed by \citet{siek2008gradual}, where they
argue that we cannot simply ``ignore dynamic types during unification''. In
their view, a type signature with a type variable indicates parametricity, but
this type does not. We agree that type variables do indicate parametricity, but
\emph{existential variables} do not! An \emph{unsolved} existential variable
indicates that a value's only constraint is that it may be cast to and from
$\unknown$, thus may introduce runtime casts. A similar observation is also
found in \citet{garcia2015principal}, where they have to distinguish between
\emph{static polymorphism} and \emph{gradual polymorphism}, and in addition
to gradual type parameters, they have to introduce the so-called \emph{static
  type parameters}. We argue that our language design is much simpler, and our
algorithmic system naturally embraces this distinction.

Now the question is can we really do that? The syntax in \cref{fig:algo-syntax}
specifies that the solution of a existential variable can only be a monotype.
This is true if the existential variable has a solution. Reading of the output
context reveals that $\genA$ does not have any solution at all. What is more,
our target language (i.e., \pbc) has a nice property, the so-called
``Jack-of-All-Trade Principle''~\cite{ahmed2011blame} that says if instantiating
a type parameter to any given type yields an answer then instantiating that type
parameter to $\unknown$ yields the same answer. In light of these, no type is
more suitable than $\unknown$.

We need to note that this does not mean we should always instantiate a type
parameter to $\unknown$, as is the case in \pbc. One of our design principles is
that we should extract as much information as possible from the static aspects of
the type system, until there is nothing more we can know, then leave the job to
the runtime checks.
\end{comment}



%%% Local Variables:
%%% mode: latex
%%% TeX-master: "../paper"
%%% org-ref-default-bibliography: ../paper.bib
%%% End:
