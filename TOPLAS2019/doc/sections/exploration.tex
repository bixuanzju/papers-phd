\section{Revisiting Consistent Subtyping}
\label{sec:exploration}

In this section we explore the design space of consistent subtyping.
% Now
% metavariables $A,B$ in \cref{fig:original-typing} also range over the unknown type
% $\unknown$.
We start with the definitions of consistency and subtyping for
polymorphic types, and compare with some relevant work.
% (in particular the compatibility relation by \citet{ahmed2011blame} and type
% consistency by \citet{yuu2017poly}).
We then discuss the design decisions
involved in our new definition of consistent subtyping, and justify the new
definition by demonstrating its equivalence with that of \citet{siek2007gradual}
and the AGT approach~\citep{garcia2016abstracting} on simple types.

% \begin{figure}[t]
%   \centering
%   \begin{small}
% \begin{tabular}{lrcl} \toprule
%   % Expressions & $e$ & \syndef & $x \mid n \mid
%   %                        \blam x A e \mid \erlam x e \mid e~e $ \\
% %%                         \mid \erlam x e \equiv \blam x \unknown e $ \\

%   Types & $A, B$ & \syndef & $ \nat \mid a \mid A \to B \mid \forall a. A \mid \unknown$ \\
%   Monotypes & $\tau, \sigma$ & \syndef & $ \nat \mid a \mid \tau \to \sigma$ \\

%   Contexts & $\dctx$ & \syndef & $\ctxinit \mid \dctx,x: A \mid \dctx, a$ \\
%   % Syntactic Sugar & $e : A$ & $\equiv$ & $(\blam x A x) ~ e$ \\
%   \bottomrule
% \end{tabular}
%   \end{small}
% \caption{Polymorphic types added by $\unknown$.}
% \label{fig:decl-types}
% \end{figure}

% \paragraph{Types}
The syntax of types is given at the top of
\cref{fig:decl:subtyping}.
% Meta-variable $e$ ranges over expressions.
% Expressions are either variables $x$, integers $n$, annotated lambda
% abstractions $\blam x A e$, un-annotated lambda abstractions $\erlam x e$ or
% applications $e_1 ~ e_2$.
We write $[[A]], [[B]]$ for types. Types are either the integer type $[[int]]$, type
variables $[[a]]$, functions types $[[A -> B]]$, universal quantification $[[ \/a . A ]]$, or
the unknown type $[[unknown]]$. Though we only have one base type $[[int]]$, we
also use $[[bool]]$ in examples. Note that monotypes $[[t]]$ contain all
types other than the universal quantifier and the unknown type $[[unknown]]$. % Note that
% $\unknown$ is only added to the category of polymorphic types $A, B$.
We will discuss this restriction when we present the subtyping rules.
Contexts $[[dd]]$ are \emph{ordered} lists of type variable declarations and term variables.



\subsection{Consistency and Subtyping}
\label{subsec:consistency-subtyping}

We start by giving the definitions of consistency and subtyping for polymorphic
types, and comparing our definitions with the compatibility relation by
\citet{ahmed2011blame} and type consistency by \citet{yuu2017poly}.

\begin{figure}[t]
  \centering
  \begin{small}
    \begin{tabular}{lrcl} \toprule
      Types & $[[A]], [[B]]$ & $\Coloneqq$ & $[[int]] \mid [[a]] \mid [[A -> B]] \mid [[\/ a. A]] \mid [[unknown]]$ \\
      Monotypes & $[[t]], [[s]]$ & $\Coloneqq$ & $ [[int]] \mid [[a]] \mid [[t -> s]] $ \\
      Contexts & $[[dd]]$ & $\Coloneqq$ & $ [[empty]]  \mid [[dd , x : A]]  \mid [[dd, a]]$ \\
      \bottomrule
    \end{tabular}

\renewcommand\ottaltinferrule[4]{
  \inferrule*
    {#3}
    {#4}
}

    \drules[dconsist]{$ [[ A ~ B ]] $}{Type Consistency}{refl, unknownR, unknownL, arrow, forall}

\renewcommand\ottaltinferrule[4]{
  \inferrule*[right=\scriptsize{#1}]
    {#3}
    {#4}
}
    \drules[s]{$ [[dd |- A <: B ]] $}{Subtyping}{tvar, int, arrow, forallL, forallR, unknown}

\renewcommand\ottaltinferrule[4]{
  \inferrule*
    {#3}
    {#4}
}
    \drules[d]{$[[ dd |- A ]]$}{Well-formedness of types}{int, unknown,
      tvar, arrow, forall}

  \end{small}
  \caption{Syntax of types, consistency, subtyping, well-formedness of types in the declarative system.}
  \label{fig:decl:subtyping}
\end{figure}

\paragraph{Consistency.}
The key observation here is that consistency is mostly a structural relation,
except that the unknown type $\unknown$ can be regarded as any type. In other
words, consistency is an equivalence relation lifted from static types to
gradual types~\citep{garcia2016abstracting}. Following this observation, we
naturally extend the definition from \cref{fig:objects} with polymorphic types,
as shown in the middle of \cref{fig:decl:subtyping}. In particular a polymorphic
type $[[ \/a . A]]$ is consistent with another polymorphic type $[[ \/a. B ]]$
if $[[A]]$ is consistent with $[[B]]$.

\paragraph{Subtyping.}

We express the fact that one type is a polymorphic generalization of another by
means of the subtyping judgment $[[ dd |- A <: B ]]$. Compared with the
subtyping rules of \citet{odersky1996putting} in \cref{fig:original-typing}, the
only addition is the neutral subtyping of $[[unknown]]$. Notice that, in
\rref{s-forallL}, the universal quantifier is only allowed to be instantiated
with a \emph{monotype}. The judgment $[[ dd |- A ]]$ checks whether all the type
variables in $[[A]]$ are bound in the context $[[dd]]$. According to the syntax
in \cref{fig:decl:subtyping}, monotypes do not include the unknown type
$[[unknown]]$. This is because if we were to allow the unknown type to be used
for instantiation, we could have $[[ \/a . a -> a <: unknown -> unknown ]]$ by
instantiating $[[a]]$ with $[[unknown]]$. Since $[[ unknown -> unknown ]]$ is
consistent with any functions $[[ A -> B ]]$, for instance, $[[ int -> bool ]]$,
this means that we could provide an expression of type $[[ \/a . a -> a ]]$ to a
function where the input type is supposed to be $[[ int -> bool ]]$. However, as
we know, $[[ \/a . a -> a ]]$ is definitely not compatible with $[[ int -> bool
]]$. Indeed, this does not hold in any polymorphic type systems without gradual
typing. So the gradual type system should not accept it either. (This is the
\emph{conservative extension} property that will be made precise in
\cref{sec:criteria}.)

Importantly there is a subtle distinction between a type variable and the
unknown type, although they both represent a kind of ``arbitrary'' type. The
unknown type stands for the absence of type information: it could be \emph{any
  type} at \emph{any instance}. Therefore, the unknown type is consistent with
any type, and additional type-checks have to be performed at runtime. On the
other hand, a type variable indicates \emph{parametricity}.
% and is subject to global constraints
In other words, a
type variable can only be instantiated to a single type. For example, in the
type $[[ \/a . a -> a ]]$, the two occurrences of $a$ represent an arbitrary but
single type (e.g., $[[ int -> int  ]]$, $[[ bool -> bool ]]$), while $[[unknown -> unknown]]$
could be an arbitrary function (e.g., $[[ int -> bool  ]]$) at runtime.

\paragraph{Comparison with Other Relations.}

In other polymorphic gradual calculi, consistency and subtyping are often mixed
up to some extent. In \pbc~\citep{ahmed2011blame}, the compatibility relation for polymorphic types
is defined as follows:
\begin{mathpar}
  \CompAllR \and \CompAllL
\end{mathpar}
Notice that, in \rref{Comp-AllL}, the universal quantifier is \emph{always}
instantiated to $[[unknown]]$. However, this way, \pbc allows $[[\/a . a -> a]] \pbccons [[int -> bool]]$,
which as we discussed before might not be what we
expect. Indeed \pbc relies on sophisticated runtime checks to rule out such
instances of the compatibility relation a posteriori.

\citet{yuu2017poly} introduced the so-called \emph{quasi-polymorphic} types
for types that may be used where a $\forall$-type is expected, which is
important for their purpose of conservativity over System F. Their type
consistency relation, involving polymorphism, is defined as
follows\footnote{This is a simplified version. These two rules are presented in
  Section~3.1 in their paper as one of the key ideas of the design of type
  consistency, which are later amended with \emph{labels}.}:
\begin{mathpar}
  \inferrule{[[A ~ B]] }{[[ \/a . A ~ \/a . B  ]]}
  \and
  \inferrule{ [[A ~ B]] \\ [[B]] \neq [[\/a. B']] \\ [[unknown]] \in \mathsf{Types}([[B]])  }{ [[\/a. A ~ B]]   }
  % \inferrule{A \sim B \\ B \neq \forall a. B' \\ \unknown \in \mathsf{Types}(B)}
  % {\forall a. A \sim B}
\end{mathpar}
Compared with our consistency definition in \cref{fig:decl:subtyping},
their first rule is the same as ours. The second rule says that a non
$\forall$-type can be consistent with a $\forall$-type only if it contains
$[[unknown]]$. In this way, their type system is able to reject $[[ \/a . a -> a ~ int -> bool  ]]$.
However, in order to keep conservativity, they also reject
$[[  \/a. a -> a ~ int -> int ]]$, which is perfectly sensible in their
setting of explicit polymorphism. However with implicit polymorphism, we
would expect $[[ \/a. a -> a  ]]$ to be related with $[[int -> int]]$,
since $[[a]]$ can be instantiated to $[[int]]$.

Nonetheless, when it comes to interactions between dynamically typed and
polymorphically typed terms, both relations allow $[[ \/a. a -> int  ]]$
to be related with $[[ unknown -> int ]]$ for example, which in our view, is a kind of
(implicit) polymorphic subtyping combined with type consistency, and
that should be derivable by the more primitive notions in the type system
(instead of inventing new relations). One of our design principles is that
subtyping and consistency are \emph{orthogonal}, and can be naturally
superimposed, echoing the opinion of \citet{siek2007gradual}.

\subsection{Towards Consistent Subtyping}
\label{subsec:towards-conssub}

With the definitions of consistency and subtyping, the question now is how to
compose the two relations so that two types can be compared in a way that takes
both relations into account.

Unfortunately, the strawman version of consistent subtyping
(\cref{def:old-decl-conssub}) does not work well with our definitions of
consistency and subtyping for polymorphic types. Consider two types: $[[ (\/a. a
-> int) -> int ]]$, and $[[ (unknown -> int) -> int ]]$. The first type can only
reach the second type in one way (first by applying consistency, then
subtyping), but not the other way, as shown in \cref{fig:example:a}. We use
$\emptyset$ to mean that we cannot find such a type. Similarly, there are
situations where the first type can only reach the second type by the other way
(first applying subtyping, and then consistency), as shown in
\cref{fig:example:b}.

\begin{small}
\begin{figure}[t]
  \begin{subfigure}[b]{.5\linewidth}
    \centering
      \begin{tikzpicture}
        \matrix (m) [matrix of math nodes,row sep=2.5em,column sep=4em,minimum width=2em]
        {
          \emptyset & (\unknown \to \nat) \to \nat \\
          (\forall a. a \to \nat) \to \nat & (\forall a. \unknown \to \nat) \to \nat \\};

        \path[-stealth]
        (m-2-1) edge node [left] {$\tsub$} (m-1-1)
        (m-2-2) edge node [left] {$\tsub$} (m-1-2);

        \draw
        (m-1-1) edge node [above] {$\sim$} (m-1-2)
        (m-2-1) edge node [below] {$\sim$} (m-2-2);
      \end{tikzpicture}
      \caption{}
      \label{fig:example:a}
  \end{subfigure}
  \begin{subfigure}[b]{.4\linewidth}
    \centering
    \begin{tikzpicture}
      \matrix (m) [matrix of math nodes,row sep=2.5em,column sep=4em,minimum width=2em]
      {
        \nat \to \nat & \nat \to \unknown \\
        \forall a. a & \emptyset \\};

      \path[-stealth]
      (m-2-1) edge node [left] {$\tsub$} (m-1-1)
      (m-2-2) edge node [left] {$\tsub$} (m-1-2);

      \draw
      (m-1-1) edge node [above] {$\sim$} (m-1-2)
      (m-2-1) edge node [below] {$\sim$} (m-2-2);
    \end{tikzpicture}
    \caption{}
    \label{fig:example:b}
  \end{subfigure}
  \begin{subfigure}[b]{.8\linewidth}
    \centering
    \begin{tikzpicture}
      \matrix (m) [matrix of math nodes,row sep=2.5em,column sep=1em,minimum width=2em]
      {
        \emptyset &
        (((\unknown \to \nat)\to \nat) \to \bool) \to (\nat \to \unknown)  \\
        (((\forall a. a \to \nat) \to \nat) \to \bool) \to (\forall a. a) &
        \emptyset \\};

      \path[-stealth]
      (m-2-1) edge node [left] {$\tsub$} (m-1-1)
      (m-2-2) edge node [left] {$\tsub$} (m-1-2);

      \draw
      (m-1-1) edge node [above] {$\sim$} (m-1-2)
      (m-2-1) edge node [below] {$\sim$} (m-2-2);
    \end{tikzpicture}
    \caption{}
    \label{fig:example:c}
  \end{subfigure}
  \caption{Examples that break the original definition of consistent subtyping.}
  \label{fig:example}
\end{figure}
\end{small}



What is worse, if those two examples are composed in a way that those types all
appear co-variantly, then the resulting types cannot reach each other
in either
way. For example, \cref{fig:example:c} shows two such types by putting a
$[[bool]]$ type in the middle, and neither definition of consistent subtyping
works. % But these two types ought to be related somehow!

\paragraph{Observations on Consistent Subtyping Based on Information Propagation.}

In order to develop a correct definition of consistent subtyping for
polymorphic types, we need to understand how consistent subtyping works.
We first review two important properties of subtyping: (1) subtyping induces the
subsumption rule: if $[[ A <: B  ]]$, then an expression of type $[[A]]$ can be used
where $[[B]]$ is expected; (2) subtyping is transitive: if $[[ A <: B ]]$, and $[[B <: C]]$,
then $[[A <: C]]$. Though consistent subtyping takes the unknown type into
consideration, the subsumption rule should also apply: if $[[ A <~ B ]]$, then
an expression of type $[[A]]$ can also be used where $[[B]]$ is expected, given that
there might be some information lost by consistency. A crucial difference from
subtyping is that consistent subtyping is \emph{not} transitive because
information can only be lost once (otherwise, any two types are a consistent
subtype of each other). Now consider a situation where we have both $[[ A <: B ]]$,
and $[[ B <~ C ]]$, this means that $[[A]]$ can be used where $[[B]]$ is expected, and
$[[B]]$ can be used where $[[C]]$ is expected, with possibly some loss of information. In
other words, we should expect that $[[A]]$ can be used where $[[C]]$ is expected, since
there is at most one-time loss of information.

\begin{observation}
  If $[[A <: B]]$, and $[[B <~ C]]$, then $[[A <~ C]]$.
\end{observation}

This is reflected in \cref{fig:obser:a}. A symmetrical
observation is given in \cref{fig:obser:b}:


\begin{observation}
  If $[[C <~ B]]$, and $[[B <: A]]$, then $[[C <~ A]]$.
\end{observation}

\begin{figure}[t]
  \centering
  \begin{subfigure}[b]{.4\linewidth}
    \centering
    \begin{tikzpicture}
      \matrix (m) [matrix of math nodes,row sep=2.5em,column sep=4em,minimum width=2em]
      {
        T_1 & C \\
        B   & T_2 \\
        A & \\};

      \path[-stealth]
      (m-3-1) edge node [left] {$\tsub$} (m-2-1)
      (m-2-2) edge node [left] {$\tsub$} (m-1-2)
      (m-2-1) edge node [left] {$\tsub$} (m-1-1);

      \draw
      (m-2-1) edge node [below] {$\sim$} (m-2-2)
      (m-1-1) edge node [above] {$\sim$} (m-1-2);

      \draw [dashed, ->]
      (m-2-1) edge node [above] {$\tconssub$} (m-1-2);

      \path [dashed, ->, out=0, in=0, looseness=2]
      (m-3-1) edge node [right] {$\tconssub$} (m-1-2);
    \end{tikzpicture}
    \caption{}
    \label{fig:obser:a}
  \end{subfigure}
  \centering
  \begin{subfigure}[b]{.4\linewidth}
    \centering
    \begin{tikzpicture}
      \matrix (m) [matrix of math nodes,row sep=2.5em,column sep=4em,minimum width=2em]
      {
        & A \\
        T_1 & B \\
        C   & T_2 \\};

      \path[-stealth]
      (m-3-1) edge node [left] {$\tsub$} (m-2-1)
      (m-3-2) edge node [left] {$\tsub$} (m-2-2)
      (m-2-2) edge node [left] {$\tsub$} (m-1-2);

      \draw
      (m-2-1) edge node [above] {$\sim$} (m-2-2)
      (m-3-1) edge node [below] {$\sim$} (m-3-2);

      \draw [dashed, ->]
      (m-3-1) edge node [above] {$\tconssub$} (m-2-2);

      \path [dashed, ->, out=135, in=180, looseness=2]
      (m-3-1) edge node [left] {$\tconssub$} (m-1-2);
    \end{tikzpicture}
    \caption{}
    \label{fig:obser:b}
  \end{subfigure}
  \caption{Observations of consistent subtyping}
  \label{fig:obser}
\end{figure}


From the above observations, we see what the problem is with the original
definition. In \cref{fig:obser:a}, if $B$ can reach $C$ by $T_1$, then by
subtyping transitivity, $A$ can reach $C$ by $T_1$. However, if $B$ can only reach $C$ by
$T_2$, then $A$ cannot reach $C$ through the original definition. A similar
problem is shown in \cref{fig:obser:b}.
% : if $C$ can only reach $B$ by $T_1$, then $C$ cannot reach $A$ through the original definition.

It turns out that these two problems can be fixed using the same strategy:
instead of taking one-step subtyping and one-step consistency, our definition of
consistent subtyping allows types to take \emph{one-step subtyping, one-step
consistency, and one more step subtyping}. Specifically, $A \tsub B \sim T_2
\tsub C$ (in \cref{fig:obser:a})
and $C \tsub T_1 \sim B \tsub A$ (in \cref{fig:obser:b}) have the same relation chain: subtyping,
consistency, and subtyping.

\paragraph{Definition of Consistent subtyping.} From the above discussion, we are
ready to modify \cref{def:old-decl-conssub}, and adapt it to our
notation\footnote{For readers who are familiar with category theory,
  this defines consistent subtyping as the least subtyping bimodule
  extending consistency.}:
\begin{definition}[Consistent Subtyping]
  \label{def:decl-conssub}
  % \[
  %   \inferrule{
  %      \tpresub A \tsub C
  %      \\ C \sim D
  %      \\ \tpresub D \tsub B
  %   }{
  %     \tpresub A \tconssub B
  %   }
  % \]
  $[[dd |- A <~ B]] $ if and only if  $ [[dd |- A <: A']]$, $[[  A' ~ B'  ]]$ and $[[ dd |- B' <: B ]]$ for some $[[A']]$ and $[[B']]$.
\end{definition}
\noindent With \cref{def:decl-conssub}, \Cref{fig:example:c:fix}
illustrates the correct relation chain for the broken example shown in
\cref{fig:example:c}.

At first sight, \cref{def:decl-conssub}
seems worse than the original: we need to guess \emph{two} types! It turns out
that \cref{def:decl-conssub} is a generalization of
\cref{def:old-decl-conssub}, and they are equivalent in the system of
\citet{siek2007gradual}. However, more generally, \cref{def:decl-conssub}
% We argue that this is a \emph{general} definition of
% consistent subtyping, and
is compatible with polymorphic types.

\begin{figure}[t]
  \centering
  \begin{subfigure}[b]{.4\linewidth}
  \begin{tikzpicture}
    \matrix (m) [matrix of math nodes,row sep=2.5em,column sep=6em,minimum width=2em]
    {
      A_2 &
      A_3
      \\
      A_1
      &
      A_4  \\
      };

    \path[-stealth]
    (m-2-1) edge node [left] {$\tsub$} (m-1-1)
    (m-1-2) edge node [left] {$\tsub$} (m-2-2);
    \path[dashed, ->, out=315, in=225, looseness=0.45]
    (m-2-1) edge node [above] {$\tconssub$} (m-2-2);

    \draw
    (m-1-1) edge node [above] {$\sim$} (m-1-2);
  \end{tikzpicture}
  \end{subfigure}
  \begin{subfigure}[b]{.4\linewidth}
  \begin{align*}
  A_1 &=(((\forall a. a \to \nat) \to \nat) \to \bool) \to (\forall a. a) \\
  A_2 &= ((\forall a. a \to \nat) \to \nat) \to \bool) \to (\nat \to \nat) \\
  A_3 &= ((\forall a. \unknown \to \nat) \to \nat) \to \bool) \to (\nat \to \unknown) \\
  A_4 &= (((\unknown \to \nat) \to \nat) \to \bool) \to (\nat \to \unknown)
  \end{align*}
  \end{subfigure}
  \caption{Example that is fixed by the new definition of consistent subtyping.}
  \label{fig:example:c:fix}
\end{figure}

\begin{proposition}[Generalization of Declarative Consistent Subtyping]\leavevmode
  \label{prop:subsumes}
\begin{itemize}
  \item \cref{def:decl-conssub} subsumes
    \cref{def:old-decl-conssub}.
    \\In \cref{def:decl-conssub},
    by choosing $D=B$, we have $A\tsub C$ and $C \sim B$; by choosing $C=A$, we have
    $A \sim D$, and $D \tsub B$.
  \item \cref{def:old-decl-conssub} is equivalent to
    \cref{def:decl-conssub} in the system of \citeauthor{siek2007gradual}.
    \\If $A \tsub C$, $C \sim D$, and $D \tsub
    B$, by \cref{def:old-decl-conssub},
    $A \sim C'$, $C' \tsub D$ for some $C'$. By subtyping
    transitivity, $C' \tsub B$. So $A \tconssub B$ by $A \sim C'$ and $C'
    \tsub B$.
  \end{itemize}
\end{proposition}


\subsection{Abstracting Gradual Typing}
\label{subsec:agt}

\citet{garcia2016abstracting} presented a new foundation for gradual typing that
they call the \emph{Abstracting Gradual Typing} (AGT) approach. In the AGT
approach, gradual types are interpreted as sets of static types, where static
types refer to types containing no unknown types. In this interpretation,
predicates and functions on static types can then be lifted to apply to gradual
types. Central to their approach is the so-called \emph{concretization}
function. For simple types, a concretization $\gamma$ from gradual types to a
set of static types is defined as follows:

\begin{definition}[Concretization]
  \label{def:concret}
  \begin{mathpar}
    \gamma(\nat) = \{\nat\} \and
    \gamma(A \to B) = \{A' \to B' \mid A' \in \gamma(A), B' \in \gamma(B)\} \and
    \gamma(\unknown) = \{\text{All static types}\}
  \end{mathpar}
\end{definition}

Based on the concretization function, subtyping between static types can be
lifted to gradual types, resulting in the consistent subtyping relation:
\begin{definition}[Consistent Subtyping in AGT]
  \label{def:agt-conssub}
  $A \agtconssub B$ if and only if $A_1 \tsub B_1$ for some \emph{static types} $A_1$ and $B_1$ such that
   $A_1 \in \gamma(A)$ and $B_1 \in \gamma(B)$.
\end{definition}

Later they proved that this definition of consistent subtyping coincides with
that of \cref{def:old-decl-conssub}. By
\cref{prop:subsumes}, we can directly conclude that our definition coincides with AGT:

\begin{corollary}[Equivalence to AGT on Simple Types]
  \label{lemma:coincide-agt}
  $A \tconssub B$ if and only if $A \agtconssub B$.
\end{corollary}

However, AGT does not show how to deal with polymorphism (e.g. the
interpretation of type variables) yet.
% As noted by \citet{garcia2016abstracting} in the conclusion, extending
% AGT to deal with polymorphism remains as an open question.
Still, as noted by \citet{garcia2016abstracting},
it is a promising line of future work for
AGT, and the question remains whether our definition would coincide with it.

Another note related to AGT is that the definition is later adopted by
\citet{castagna2017gradual} in a gradual type system with union and intersection
types, where the static types $A_1, B_1$ in Definition~\ref{def:agt-conssub} can
be algorithmically computed by also accounting for top and bottom types.

\subsection{Directed Consistency}


\emph{Directed consistency}~\citep{Jafery:2017:SUR:3093333.3009865} is defined
in terms of precision and subtyping:
\[
  \inferrule{
       A' \lessp A
    \\ A \tsub B
    \\ B' \lessp B
  }{
    A' \tconssub B'
  }
\]
The judgment $A \lessp B$ is read ``$A$ is less precise than $B$''.\footnote{\citeauthor{Jafery:2017:SUR:3093333.3009865} actually read  $A \lessp$ B as ``$A$ is \emph{more precise} than $B$''. We, however, use the ``less precise'' notation (which is also adopted by \citet{cimini2016gradualizer} ) throughout the paper. The full rules can be found in \cref{fig:lessp}.}
In their setting,
precision is first defined for type constructors and then lifted to gradual
types, and subtyping is defined for gradual types. If we interpret this
definition from the AGT point of view, finding a more precise static type
has the same effect as concretization. Namely, $A' \lessp A $ implies $A \in
\gamma(A')$ and $B' \lessp B$ implies $B \in \gamma(B')$ if $A $ and $ B$ are static
types. Therefore we consider this definition as AGT-style. From this
perspective, this definition naturally coincides with \cref{def:agt-conssub},
and by \cref{lemma:coincide-agt}, it coincides with \cref{def:decl-conssub}.

The value of their definition is that consistent subtyping is derived
compositionally from \emph{gradual subtyping} and \emph{precision}.
Arguably, gradual types play a role in both definitions, which is different from
\cref{def:decl-conssub} where subtyping is neutral to unknown types. Still, the
definition is interesting as it takes precision into consideration, rather than
consistency. Then a question arises as to \emph{how are consistency and
  precision related}.

\paragraph{Consistency and Precision.}

Precision is a partial order (anti-symmetric and transitive), while consistency
is symmetric but not transitive. Recall that consistency is in fact an
equivalence relation lifted from static types to gradual
types~\citep{garcia2016abstracting}, which embodies the key role of gradual types
in typing. Therefore defining consistency independently is straightforward, and
it is theoretically viable to validate the definition of consistency directly.
On the other hand, precision is usually connected with the gradual criteria
\citep{siek2015refined}, and finding a correct partial order that adheres to the
criteria is not always an easy task. For example, \citet{yuu2017poly} argued
that term precision for gradual System F is actually nontrivial, leaving the
gradual guarantee of the semantics as a conjecture. Thus precision can be
difficult to extend to more sophisticated type systems, e.g. dependent types.

Nonetheless, in our system, precision and consistency can be related by the
following lemma \footnote{Lemmas with $\mathcal{L}$ are those proved in Coq. The
  same applies to $\mathcal{T}$heorems.}:

\begin{clemma}[Consistency and Precision]\leavevmode
  \label{lemma:consistency-precision}
  \begin{itemize}
  \item If $[[A ~ B]]$,
    then there exists (static) $C$,
    such that $A \lessp C$,
    and $B \lessp C$.
  \item If for some (static) $C$,
    we have $A \lessp C$,
    and $B \lessp C$,
    then we have $[[A ~ B]]$.
  \end{itemize}
\end{clemma}

% Thus, it is interesting that those two definitions illustrate the
% correspondence of different foundations (on simple types): one is defined
% directly on gradual types, and the other stems from AGT, which is based on
% static subtyping.

\subsection{Consistent Subtyping Without Existentials}
\label{subsec:conssub-wo-exist}

\cref{def:decl-conssub} serves as a fine specification of how consistent
subtyping should behave in general. But it is inherently non-deterministic
because of the two intermediate types $C$ and $D$. As
\cref{def:algo-old-decl-conssub}, we need a combined relation to directly compare two
types. A natural attempt is to try to extend the restriction operator for
polymorphic types. Unfortunately, as we show below, this does not work. However
it is possible to devise an equivalent inductive definition instead.

\paragraph{Attempt to Extend the Restriction Operator.}
Suppose that we try to extend \cref{def:algo-old-decl-conssub} to account for polymorphic
types. The original restriction operator is structural, meaning that it works
for types of similar structures. But for polymorphic types, two input types
could have different structures due to universal quantifiers, e.g, $[[ \/a. a -> int  ]]$
and $[[ (int -> unknown) -> int ]]$. If we try to mask the first type
using the second, it seems hard to maintain the information that $a$ should be
instantiated to a function while ensuring that the return type is masked. There
seems to be no satisfactory way to extend the restriction operator in order to
support this kind of non-structural masking.

\paragraph{Interpretation of the Restriction Operator and Consistent Subtyping.}
If the restriction operator cannot be extended naturally, it is useful to
take a step back and revisit what the restriction operator actually does. For
consistent subtyping, two input types could have unknown types in different
positions, but we only care about the known parts. What the restriction
operator does is (1) erase the type information in one type if the corresponding
position in the other type is the unknown type; and (2) compare the resulting types
using the normal subtyping relation. The example below shows the
masking-off procedure for the types $[[ int -> unknown -> bool  ]]$ and $[[int -> int -> unknown]]$.
Since the known parts have the relation that $[[ int -> unknown -> unknown <: int -> unknown -> unknown ]]$, we conclude that
$[[ int -> unknown -> bool <~ int -> int -> unknown  ]]$.
\begin{center}
  \begin{tikzpicture}
    \tikzstyle{column 5}=[anchor=base west, nodes={font=\tiny}]
    \matrix (m) [matrix of math nodes,row sep=0.5em,column sep=0em,minimum width=2em]
    {
      \nat \to & \unknown & \to & \bool & \mid \nat \to \nat \to \unknown &
      = \nat \to \unknown \to \unknown
      \\
       \nat \to & \nat & \to & \unknown & \mid \nat \to \unknown \to \bool &
      = \nat \to \unknown \to \unknown \\};

    \path[-stealth, ->, out=0, in=0]
    (m-1-6) edge node [right] {$\tsub$} (m-2-6);

    \draw
    (m-1-2.north west) rectangle (m-2-2.south east)
    (m-1-4.north west) rectangle (m-2-4.south east);
  \end{tikzpicture}
\end{center}
Here differences of the types in boxes are erased because of the restriction
operator. Now if we compare the types in boxes directly instead of through the
lens of the restriction operator, we can observe that the \emph{consistent
  subtyping relation always holds between the unknown type and an arbitrary
  type.} We can interpret this observation directly from
\cref{def:decl-conssub}: the unknown type is neutral to subtyping ($[[ unknown <: unknown  ]]$),
the unknown type is consistent with any type ($[[unknown ~ A]]$),
and subtyping is reflexive ($[[A <: A]]$). Therefore,
\emph{the unknown type is a consistent subtype of any type ($[[  unknown <~ A  ]]$), and vice versa ($[[ A <~ unknown ]]$).}
Note that this interpretation provides a
general recipe for lifting a (static) subtyping relation to a (gradual)
consistent subtyping relation, as discussed below.

\paragraph{Defining Consistent Subtyping Directly.}

From the above discussion, we can define the consistent subtyping relation
directly, \emph{without} resorting to subtyping or consistency at all. The key
idea is that we replace $<:$ with $[[<~]]$ in
\cref{fig:decl:subtyping}, get rid of \rref{s-unknown} and add two
extra rules concerning $[[unknown]]$, resulting in the rules of consistent
subtyping in \cref{fig:decl:conssub}. Of particular interest are the rules
\rref*{cs-unknownL} and \rref*{cs-unknownR}, both of which correspond to what we
just said: the unknown type is a consistent subtype of any type, and vice versa.
\begin{figure}[t]
  \begin{small}
\drules[cs]{$ [[dd |- A <~ B ]] $}{Consistent Subtyping}{tvar, int, arrow, forallR, forallL, unknownL, unknownR}
  \end{small}
  \caption{Consistent Subtyping for implicit polymorphism.}
  \label{fig:decl:conssub}
\end{figure}
From now on, we use the symbol $\tconssub$ to refer to the consistent subtyping
relation in \cref{fig:decl:conssub}. What is more, we can prove that the two definitions
are equivalent.

\begin{ctheorem}   \label{lemma:properties-conssub}
  $[[dd |- A <~ B]]  \Leftrightarrow [[ dd |- A <: A'  ]]$, $[[ A' ~ B' ]]$, $[[ dd |- B' <: B   ]]$ for some $A', B'$.
\end{ctheorem}

% \noindent Not surprisingly, consistent subtyping is reflexive:

% \begin{clemma}[Consistent Subtyping is Reflexive] \label{lemma:sub_refl}%
%   If $\Psi \vdash A$ then $\Psi \vdash A \tconssub A$.
% \end{clemma}





%%% Local Variables:
%%% mode: latex
%%% TeX-master: "../paper"
%%% org-ref-default-bibliography: "../paper.bib"
%%% End:
