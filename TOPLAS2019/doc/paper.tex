\documentclass[format=acmsmall, review=false, screen=true]{acmart}


% Metadata Information
\acmJournal{TWEB}
\acmVolume{9}
\acmNumber{4}
\acmArticle{39}
\acmYear{2010}
\acmMonth{3}
\copyrightyear{2009}
%\acmArticleSeq{9}

% Copyright
%\setcopyright{acmcopyright}
\setcopyright{acmlicensed}
%\setcopyright{rightsretained}
%\setcopyright{usgov}
%\setcopyright{usgovmixed}
%\setcopyright{cagov}
%\setcopyright{cagovmixed}

% DOI
\acmDOI{0000001.0000001}

% Paper history
% \received{February 2007}
% \received[revised]{March 2009}
% \received[accepted]{June 2009}



%% Some recommended packages.
\usepackage{booktabs}   %% For formal tables:
                        %% http://ctan.org/pkg/booktabs
\usepackage{subcaption} %% For complex figures with subfigures/subcaptions
                        %% http://ctan.org/pkg/subcaption

%% Bibliography style
\bibliographystyle{ACM-Reference-Format}
\citestyle{acmauthoryear}   %% For author/year citations
% \setcitestyle{aysep={}}


% AMS packages
\usepackage{amsmath}
\usepackage{amssymb}

\usepackage{mathtools}
\usepackage{mdwlist}
\usepackage{pifont}


% Miscellaneous
\usepackage{paralist}
\usepackage{graphicx}
\usepackage{epstopdf}
\usepackage{float}
\usepackage{longtable}
\usepackage{multirow}
\usepackage{lscape}


% Revision tools
\usepackage{xspace}
\usepackage{comment}
\newcommand\mynote[3]{}
% \newcommand\bruno[1]{\mynote{Bruno}{red}{#1}}
% \newcommand\tom[1]{\mynote{Tom}{blue}{#1}}
% \newcommand\ningning[1]{\mynote{Ningning}{orange}{#1}}
% \newcommand\jeremy[1]{\mynote{Jeremy}{gray}{#1}}

\newcommand{\hl}[2][gray!40]{\colorbox{#1}{#2}}
\newcommand{\hlmath}[2][gray!40]{\colorbox{#1}{$\displaystyle#2$}}
\newcommand{\otthl}[2][gray!40]{ \colorbox{#1}{$\displaystyle#2$}}


\newcommand{\name}{\textsf{GPC}\xspace}


% Graphs
\usepackage{tikz}
\usetikzlibrary{matrix}
\usetikzlibrary{arrows,automata}
\usetikzlibrary{positioning}


% Hyper links
\usepackage{url}
\usepackage{
  nameref,%\nameref
  hyperref,%\autoref
}
\usepackage[capitalise]{cleveref}

\usepackage{thmtools, thm-restate}

\usepackage[misc]{ifsym}

\declaretheorem[name=$\mathcal{L}$emma,
  % numberwithin=section,
  refname={$\mathcal{L}$emma,$\mathcal{L}$emmas},
  Refname={$\mathcal{L}$emma,$\mathcal{L}$emmas}]{clemma}

\declaretheorem[name=$\mathcal{T}$heorem,
  % numberwithin=section,
  refname={$\mathcal{T}$heorem,$\mathcal{T}$heorems},
  Refname={$\mathcal{T}$heorem,$\mathcal{T}$heorems}]{ctheorem}

\declaretheorem[name=Observation]{observation}

\declaretheorem[name=Proof]{Proof}


\usepackage{rotating}

\usepackage{ottalt}

\renewcommand\ottaltinferrule[4]{
  \inferrule*[right=\scriptsize{#1}]
    {#3}
    {#4}
}


\inputott{ott-rules}
\input{paper_utility.tex}

% ------------------------------------------------------
% ORIGINAL TYPING
% ------------------------------------------------------

\newcommand*{\OVar}{\inferrule{
            x : A \in \tctx
            }{
            \tctx \byoinf x \infto A
            }\rname{Var}}

\newcommand*{\ONat}{\inferrule{
            }{
            \tctx \byoinf n \infto \nat
            }\rname{Nat}}

\newcommand*{\OLamAnnA}{\inferrule{
            \tctx, x: A \byoinf e \infto B
            }{
            \tctx \byoinf (\blam x A e) \infto A \to B
            }\rname{LamAnn-I}}

\newcommand*{\OLamAnnB}{\inferrule{
            \tctx, x: A \byoinf e \chkby B
            }{
            \tctx \byoinf (\blam x A e) \chkby A \to B
            }\rname{LamAnn-C}}

\newcommand*{\OApp}{\inferrule{
            \tctx \byoinf e_1 \infto A_1 \to A_2
         \\ \tctx \byochk e_2 \chkby A_1
            }{
            \tctx \byoinf e_1 ~ e_2 \infto A_2
            }\rname{App}}

\newcommand*{\OLamB}{\inferrule{
            \tctx, x: A \byochk e \chkby B
            }{
            \tctx \byochk \erlam x e \chkby A \to B
            }\rname{Lam-C}}

\newcommand*{\OSub}{\inferrule{
            \tctx \byoinf e \infto A
            }{
            \tctx \byochk e \chkby A
            }\rname{Sub}}

% ------------------------------------------------------
% TYPING
% ------------------------------------------------------

\newcommand*{\Var}{\inferrule{
            x : A \in \tctx
            }{
            \tctx \byinf x \infto A
            \trto{x}
            }\rname{Var}}

\newcommand*{\Nat}{\inferrule{
            }{
            \tctx \byinf n \infto \nat
            \trto{n}
            }\rname{Nat}}

\newcommand*{\LamAnnA}{\inferrule{
            \tctx, x: A \byinf e \infto B \trto {e'}
            }{
            \tctx \byinf (\blam x A e) \infto A \to B
            \trto{\blam x A {e'}}
            }\rname{LamAnn-I}}

\newcommand*{\LamAnnB}{\inferrule{
            C \match A_1 \to B
         \\ A = A_1 \glb A_2
         \\ \tctx, x: A \byinf e \chkby B \trto {e'}
            }{
            \tctx \bychk (\blam x {A_2} e) \chkby C
            \trto{\cast{A \to B}{C}(\blam x {A} {e'})}
            }\rname{LamAnn-C}}

\newcommand*{\App}{\inferrule{
            \tctx \byinf e_1 \infto A \trto{e_1'}
         \\ A \match A_1 \to A_2
         \\ \tctx \bychk e_2 \chkby A_1 \trto{e_2'}
            }{
            \tctx \byinf e_1 ~ e_2 \infto A_2
            \trto {(\cast{A}{A_1 \to A_2} e_1') ~ e_2'}
            }\rname{App}}

\newcommand*{\LamB}{\inferrule{
            C \match A \to B
         \\ \tctx, x: A \bychk e \chkby B \trto{e_1'}
            }{
            \tctx \bychk \erlam x e \chkby C
            \trto{\cast{A \to B}{\erlam x e}}
            }\rname{Lam-C}}

\newcommand*{\Sub}{\inferrule{
            e \neq (\blam x C e')
         \\ \tctx \byinf e \infto B \trto{e'}
         \\ B \sim A
            }{
            \tctx \bychk e \chkby A
            \trto{\cast B A e'}
            }\rname{Sub}}

% ------------------------------------------------------
% CAST CALCULUS
% ------------------------------------------------------

\newcommand*{\CaVar}{\inferrule{
            x : A \in \tctx
            }{
            \tpreinf x : A
            }\rname{C-Var}}

\newcommand*{\CaNat}{\inferrule{
            }{
            \tctx \byinf n : \nat
            }\rname{Nat}}

\newcommand*{\CaLam}{\inferrule{
            \tctx, x: A \byinf e : B
            }{
            \tpreinf \blam x A e : A \to B
            }\rname{C-Lam}}

\newcommand*{\CaApp}{\inferrule{
            \tpreinf e_1 : A \to B
         \\ \tpreinf e_2 : A
            }{
            \tpreinf e_1 ~ e_2 : B
            }\rname{C-App}}

\newcommand*{\CaCast}{\inferrule{
            \tpreinf e : A
         \\ B \sim A
            }{
            \tpreinf \cast A B e : B
            }\rname{C-Cast}}

\newcommand*{\CaBlame}{\inferrule{
            }{
            \tpreinf \blame A l : A
            }\rname{C-Blame}}

% ------------------------------------------------------
% Matching
% ------------------------------------------------------

\newcommand*{\MA}{\inferrule{}{
            (A_1 \to A_2) \match (A_1 \to A_2)
            }}

\newcommand*{\MB}{\inferrule{}{
            \unknown \match \unknown \to \unknown
            }}

\newcommand*{\MMA}{\inferrule{ }{
            \tprematch (A_1 \to A_2) \match (A_1 \to A_2)
            }\rname{M-Arr}}

\newcommand*{\MMB}{\inferrule{ }{
            \tprematch \unknown \match \unknown \to \unknown
            }\rname{M-Unknown}}

\newcommand*{\MMC}{\inferrule{
            \tprewf \tau
         \\ \tprematch A \subst a \tau \match A_1 \to A_2
            }{
            \tprematch \forall a. A \match A_1 \to A_2
            }\rname{M-Forall}}


% ------------------------------------------------------
% Matching (Algorithmic)
% ------------------------------------------------------

\newcommand*{\AMMA}{\inferrule{ }{
            \Gamma \vdash (A_1 \to A_2) \match (A_1 \to A_2) \toctxo
            }\rname{AM-Arr}}

\newcommand*{\AMMB}{\inferrule{ }{
            \Gamma \vdash \unknown \match \unknown \to \unknown \toctxo
            }\rname{AM-Unknown}}

\newcommand*{\AMMC}{\inferrule{ \tctx, \genA \vdash A \subst a \genA \match A_1 \to A_2 \toctxr
            }{
            \Gamma \vdash \forall a. A \match A_1 \to A_2 \toctxr
            }\rname{AM-Forall}}

\newcommand*{\AMMD}{\inferrule{ }{
            \tctx[\genC] \vdash \genC \match \genA \to \genB \dashv \tctx[\genA, \genB, \genC = \genA \to \genB]
            }\rname{AM-Var}}



% ------------------------------------------------------
% Instantiation
% ------------------------------------------------------

\newcommand*{\InstLSolve}{\inferrule{ \tctx \bywf \tau}
            {\tctx, \genA, \tctx' \vdash \genA \unif \tau \dashv \tctx, \genA = \tau, \tctx'
            }\rname{InstLSolve}}

\newcommand*{\InstLSolveU}{\inferrule{ }
            {\tctx[\genA] \vdash \genA \unif \unknown \dashv \tctx[\genA]
            }\rname{InstLSolveU}}

\newcommand*{\InstLReach}{\inferrule{ }
            {\tctx[\genA][\genB] \vdash \genA \unif \genB \dashv \tctx[\genA][\genB = \genA]
            }\rname{InstLReach}}

\newcommand*{\InstLArr}{\inferrule{ \tctx[\genA_2, \genA_1, \genA = \genA_1 \to \genA_2] \vdash A_1 \unif \genA_1 \toctx \\
             \ctxl \vdash \genA_2 \unif \ctxsubst{\ctxl}{A_2} \toctxr
            }
            {\tctx[\genA] \vdash \genA \unif A_1 \to A_2 \toctxr
            }\rname{InstLArr}}

\newcommand*{\InstLAllR}{\inferrule{ \tctx[\genA], b \vdash \genA \unif B \toctxr, b, \Delta'
            }
            {\tctx[\genA] \vdash \genA \unif \forall b . B \toctxr
            }\rname{InstLAllR}}


\newcommand*{\InstRSolve}{\inferrule{ \tctx \bywf \tau}
            {\tctx, \genA, \tctx' \vdash \tau \unif \genA \dashv \tctx, \genA = \tau, \tctx'
            }\rname{InstRSolve}}

\newcommand*{\InstRSolveU}{\inferrule{ }
            {\tctx[\genA] \vdash \unknown \unif \genA \dashv \tctx[\genA]
            }\rname{InstRSolveU}}

\newcommand*{\InstRReach}{\inferrule{ }
            {\tctx[\genA][\genB] \vdash \genB \unif \genA \dashv \tctx[\genA][\genB = \genA]
            }\rname{InstRReach}}

\newcommand*{\InstRArr}{\inferrule{ \tctx[\genA_2, \genA_1, \genA = \genA_1 \to \genA_2] \vdash \genA_1 \unif A_1 \toctx \\
             \ctxl \vdash \ctxsubst{\ctxl}{A_2}  \unif \genA_2  \toctxr
            }
            {\tctx[\genA] \vdash A_1 \to A_2  \unif \genA \toctxr
            }\rname{InstRArr}}

\newcommand*{\InstRAllL}{\inferrule{ \tctx[\genA], \genB \vdash B \subst b \genB \unif \genA \toctxr
            }
            {\tctx[\genA] \vdash \forall b . B \unif \genA  \toctxr
            }\rname{InstRAllL}}



% ------------------------------------------------------
% Consistency
% ------------------------------------------------------

\newcommand*{\CA}{\inferrule{}{
            \gcastable \sim \unknown
            }}

\newcommand*{\CB}{\inferrule{}{
            \unknown \sim \gcastable
            }}

\newcommand*{\CC}{\inferrule{
            A_1 \sim B_1
         \\ A_2 \sim B_2
            }{
            A_1 \to A_2 \sim B_1 \to B_2
            }}

\newcommand*{\CD}{\inferrule{}{
            A \sim A
            }}

\newcommand*{\CE}{\inferrule{
            A \sim B
            }{
            \forall a. A \sim \forall a. B
            }}

% ------------------------------------------------------
% GREATEST LOWER BOUND
% ------------------------------------------------------

\newcommand*{\GA}{\inferrule{}{
            A \glb A = A
            }}

\newcommand*{\GB}{\inferrule{}{
            A \glb \unknown = \unknown \glb A = A
            }}

\newcommand*{\GC}{\inferrule{}{
            (A_1 \to A_2) \glb (B_1 \to B_2) = (A_1 \glb B_1) \to (A_2 \glb B_2)
            }}

\newcommand*{\GGA}{\inferrule{
            }{
            \tpreglb A \glb A = A
            }}

\newcommand*{\GGB}{\inferrule{
            }{
            \tpreglb A \glb \unknown = A
            }}

\newcommand*{\GGF}{\inferrule{
            A ~ is ~ G
            }{
            \tpreglb \unknown \glb A = A
            }}

\newcommand*{\GGG}{\inferrule{
            A ~ isnot~ G
            }{
            \tpreglb \unknown \glb A = \unknown
            }}

\newcommand*{\GGC}{\inferrule{
            \tpreglb[,a] A \glb B = C
            }{
            \tpreglb A \glb \forall a. B = C
            }}

\newcommand*{\GGD}{\inferrule{
         \\ \tpreglb[, a] A \glb B = C
            }{
            \tpreglb \forall a. A \glb B = \forall a. C
            }}

\newcommand*{\GGE}{\inferrule{
            \tpreglb A_1 \glb B_1 = C_1
         \\ \tpreglb A_2 \glb B_2 = C_2
            }{
            \tpreglb A_1 \to A_2 \glb B_1 \to B_2 = C_1 \to C_2
            }}

% ------------------------------------------------------
% MASK
% ------------------------------------------------------

\newcommand*{\MSUnknownL}{\inferrule{
            }{
            \tctx \bymask \mask \unknown B = \unknown
            }\rname{Mask-UnknownL}}

\newcommand*{\MSUnknownR}{\inferrule{
            }{
            \tctx \bymask \mask A \unknown = \unknown
            }\rname{Mask-UnknownR}}

\newcommand*{\MSForallL}{\inferrule{
            \tctx, a \bymask \mask A B = C
            }{
            \tctx \bymask \mask {\forall a. A} B  = \forall a. C
            }\rname{Mask-ForallL}}

\newcommand*{\MSForallR}{\inferrule{
            \tctx, b \bymask \mask A B = C
            }{
            \tctx \bymask \mask A {\forall b. B}  = C
            }\rname{Mask-ForallR}}

\newcommand*{\MSArrow}{\inferrule{
            \tctx \bymask \mask {A_1} {B_1} = {C_1}
         \\ \tctx \bymask \mask {A_2} {B_2} = {C_2}
            }{
            \tctx \bymask \mask {A_1 \to A_2} {B_1 \to B_2} = C_1 \to C_2
            }\rname{Mask-Arrow}}

\newcommand*{\MSNat}{\inferrule{
            }{
            \tctx \bymask \mask \nat \nat = \nat
            }\rname{Mask-Int}}

% ------------------------------------------------------
% CONSISTENT SUBTYPING
% ------------------------------------------------------

\newcommand*{\CSForallR}{\inferrule{
            \tpresub[,a] A \tconssub B
            }{
            \tpresub A \tconssub \forall a. B
            }\rname{CS-ForallR}}

\newcommand*{\CSForallL}{\inferrule{
            \dctx \bywf \tau
         \\ \tpreconssub A \subst a \tau \tconssub B
            }{
            \tpresub \forall a. A \tconssub B
            }\rname{CS-ForallL}}

\newcommand*{\CSFun}{\inferrule{
            \tpreconssub B_1 \tconssub A_1
         \\ \tpreconssub A_2 \tconssub B_2
            }{
            \tpreconssub A_1 \to A_2 \tconssub B_1 \to B_2
            }\rname{     CS-Fun}}

\newcommand*{\CSTVar}{\inferrule{
            a \in \dctx
            }{
            \tpreconssub a \tconssub a
            }\rname{CS-TVar}}

\newcommand*{\CSInt}{\inferrule{
            }{
            \tpreconssub \nat \tconssub \nat
            }\rname{CS-Int}}

\newcommand*{\CSUnknownL}{\inferrule{
            }{
            \tpreconssub \unknown \tconssub \gcastable
            }\rname{CS-UnknownL}}

\newcommand*{\CSUnknownR}{\inferrule{
            }{
            \tpreconssub \gcastable \tconssub \unknown
            }\rname{CS-UnknownR}}

\newcommand*{\CSSVar}{\inferrule{
            }{
            \tpreconssub \static \tconssub \static
            }\rname{CS-SVar}}

\newcommand*{\CSGVar}{\inferrule{
            }{
            \tpreconssub \gradual \tconssub \gradual
            }\rname{CS-GVar}}

% ------------------------------------------------------
% CONSISTENT SUBTYPING (Algorithmic)
% ------------------------------------------------------

\newcommand*{\ACSForallR}{\inferrule{ \tctx, a \vdash A \tconssub B \toctxr, a, \ctxl
            }{
            \Gamma \vdash A \tconssub \forall a. B \toctxr
            }\rname{ACS-ForallR}}

\newcommand*{\ACSForallL}{\inferrule{ \tctx, \genA \vdash A \subst a \genA \tconssub B \toctxr
            }{
            \Gamma \vdash \forall a. A \tconssub B \toctxr
            }\rname{ACS-ForallL}}

\newcommand*{\ACSFun}{\inferrule{\Gamma \vdash B_1 \tconssub A_1 \toctx \\
             \ctxl \vdash \ctxsubst{\ctxl}{A_2} \tconssub \ctxsubst{\ctxl}{B_2} \toctxr
            }{
            \Gamma \vdash A_1 \to A_2 \tconssub B_1 \to B_2 \toctxr
            }\rname{ACS-Fun}}

\newcommand*{\ACSTVar}{\inferrule{
            }{
            \tctx[a] \vdash a \tconssub a \dashv \tctx[a]
            }\rname{ACS-TVar}}

\newcommand*{\ACSExVar}{\inferrule{
            }{
            \tctx[\genA] \vdash \genA \tconssub \genA \dashv \tctx[\genA]
            }\rname{ACS-ExVar}}


\newcommand*{\ACSInt}{\inferrule{
            }{
            \Gamma \vdash \nat \tconssub \nat \toctxo
            }\rname{ACS-Int}}

\newcommand*{\ACSUnknownL}{\inferrule{
            }{
            \Gamma \vdash \unknown \tconssub \gcastable \toctxo
            }\rname{ACS-UnknownL}}

\newcommand*{\ACSUnknownR}{\inferrule{
            }{
            \Gamma \vdash \gcastable \tconssub \unknown \toctxo
            }\rname{ACS-UnknownR}}

\newcommand*{\AInstantiateL}{\inferrule{ \genA \notin \mathit{fv}(A) \\
             \tctx[\genA] \vdash \genA \unif A \toctxr
            }{
            \tctx[\genA] \vdash \genA \tconssub A \toctxr
            }\rname{ACS-InstL}}

\newcommand*{\AInstantiateR}{\inferrule{ \genA \notin \mathit{fv}(A) \\
             \tctx[\genA] \vdash  A \unif \genA \toctxr
            }{
            \tctx[\genA] \vdash A \tconssub \genA  \toctxr
            }\rname{ACS-InstR}}

\newcommand*{\ACSSVar}{\inferrule{
            }{
            \Gamma \vdash \static \tconssub \static \toctxo
            }\rname{ACS-SVar}}

\newcommand*{\ACSGVar}{\inferrule{
            }{
            \Gamma \vdash \gradual \tconssub \gradual \toctxo
            }\rname{ACS-GVar}}


% ------------------------------------------------------
% LESS PRECISION
% ------------------------------------------------------

\newcommand*{\LUnknown}{\inferrule{
            }{
            \unknown \lessp A
            }\rname{L-Unknown}}

\newcommand*{\LNat}{\inferrule{
            }{
            \nat \lessp \nat
            }\rname{L-Nat}}

\newcommand*{\LArrow}{\inferrule{
            A_1 \lessp B_1
         \\ A_2 \lessp B_2
            }{
            A_1 \to A_2 \lessp B_1 \to B_2
            }\rname{L-Arrow}}

\newcommand*{\LTVar}{\inferrule{
            }{
            a \lessp a
            }\rname{L-TVar}}

\newcommand*{\LForall}{\inferrule{
            A \lessp B
            }{
            \forall a. A \lessp \forall a. B
            }\rname{L-Forall}}

% Term level

\newcommand*{\LRefl}{\inferrule{
            }{
            e \lessp e
            }\rname{L-Refl}}

\newcommand*{\LAbsAnn}{\inferrule{
            A_1 \lessp A_2
         \\ e_1 \lessp e_2
            }{
            \blam x {A_1} {e_1} \lessp \blam x {A_2} {e_2}
            }\rname{L-LamAnn}}

\newcommand*{\LApp}{\inferrule{
            e_1 \lessp e_3
         \\ e_2 \lessp e_4
            }{
            e_1 ~ e_2 \lessp e_3 ~ e_4
            }\rname{L-App}}

% PBC Term level

\newcommand*{\LVar}{\inferrule{
            x : A \in \dctx_1
         \\ x : B \in  \dctx_2
            }{
            \dctx_1 \ctxsplit \dctx_2 \bylessp x \lesspp x
            }\rname{L-Var}}

\newcommand*{\LNatP}{\inferrule{
            }{
            \dctx_1 \ctxsplit \dctx_2 \bylessp n \lesspp n
            }\rname{L-Nat}}

\newcommand*{\LAbsAnnP}{\inferrule{
            A_1 \lessp A_2
         \\ \dctx_1, x: A_1 \ctxsplit \dctx_2, x: A_2 \bylessp e_1 \lesspp e_2
            }{
            \dctx_1 \ctxsplit \dctx_2 \bylessp \blam x {A_1} {e_1} \lesspp \blam x {A_2} {e_2}
            }\rname{L-LamAnn}}

\newcommand*{\LAppP}{\inferrule{
            \dctx_1 \ctxsplit \dctx_2 \bylessp e_1 \lesspp e_3
         \\ \dctx_1 \ctxsplit \dctx_2 \bylessp e_2 \lesspp e_4
            }{
            \dctx_1 \ctxsplit \dctx_2 \bylessp e_1 ~ e_2 \lesspp e_3 ~ e_4
            }\rname{L-App}}

\newcommand*{\LCast}{\inferrule{
            A_1 \lessp B_1
         \\ A_2 \lessp B_2
         \\ \dctx_1 \ctxsplit \dctx_2 \bylessp e_1 \lesspp e_2
            }{
            \dctx_1 \ctxsplit \dctx_2 \bylessp \cast {A_1} {A_2} {e_1} \lesspp \cast{B_1} {B_2} {e_2}
            }\rname{L-Cast}}

\newcommand*{\LCastL}{\inferrule{
            \dctx_1 \ctxsplit \dctx_2 \bylessp e_1 \lesspp e_2
         \\ \dctx_2 \bypinf e_2 : B
         \\ A_1 \lessp B
         \\ A_2 \lessp B
            }{
            \dctx_1 \ctxsplit \dctx_2 \bylessp \cast {A_1} {A_2} {e_1} \lesspp {e_2}
            }\rname{L-CastL}}

\newcommand*{\LCastR}{\inferrule{
            \dctx_1 \ctxsplit \dctx_2 \bylessp e_1 \lesspp e_2
         \\ \dctx_1 \bypinf e_1 : A
         \\ A \lessp B_1
         \\ A \lessp B_2
            }{
            \dctx_1 \ctxsplit \dctx_2 \bylessp e_1 \lesspp \cast {B_1} {B_2} {e_2}
            }\rname{L-CastR}}

% Env level

\newcommand*{\LERefl}{\inferrule{
            }{
            \tctx \lessp \tctx
            }\rname{L-ERefl}}

\newcommand*{\LEPush}{\inferrule{
            \tctx_1 \lessp \tctx_2
         \\ A_1 \lessp A_2
            }{
            \tctx_1, x: A_1 \lessp \tctx_2, x:A_2
            }\rname{L-EPush}}

% ------------------------------------------------------
% ORIGINAL HIGHER-RANKED TYPE
% ------------------------------------------------------

\newcommand*{\HVar}{\inferrule{
            x : A \in \tctx
            }{
            \tctx \byinf x \infto A
            }\rname{Var}}

\newcommand*{\HNat}{\inferrule{
            }{
            \tpreinf n \infto \nat
            }\rname{Nat}}

\newcommand*{\HLamAnnA}{\inferrule{
            \tctx, x: A \byinf e \infto B
            }{
            \tctx \byinf (\blam x A e) \infto A \to B
            }\rname{LamAnn-I}}

\newcommand*{\HLamAnnB}{\inferrule{
            B \tsub A
         \\ \tctx, x: A \bychk e \chkby C
            }{
            \tctx \byinf (\blam x A e) \chkby B \to C
            }\rname{LamAnn-C}}

\newcommand*{\HApp}{\inferrule{
            \tctx \byinf e_1 \infto A
         \\ \tctx \byinf A \bullet e \appto B
            }{
            \tctx \byinf e_1 ~ e_2 \infto B
            }\rname{App}}

\newcommand*{\HAppPoly}{\inferrule{
            \tctx \byinf A \subst a \tau \bullet e \appto B
            }{
            \tctx \byinf \forall a. A \bullet e \appto B
            }\rname{AppPoly}}

\newcommand*{\HAppFun}{\inferrule{
            \tprechk e \chkby A_1
            }{
            \tctx \byinf A_1 \to A_2 \bullet e \appto A_2
            }\rname{AppFun}}

\newcommand*{\HSub}{\inferrule{
            \tctx \byinf e \infto A
         \\ \tpresub A \tsub B
            }{
            \tctx \bychk e \chkby B
            }\rname{Sub}}

\newcommand*{\HAll}{\inferrule{
            \tctx, a \bychk e \chkby A
            }{
            \tctx \bychk e \chkby \forall a. A
            }\rname{Forall}}

\newcommand*{\SForallR}{\inferrule{
            \tctx, a \bysub A \tsub B
            }{
            A \tsub \forall a. B
            }\rname{S-ForallR}}

\newcommand*{\SForallL}{\inferrule{
            \tctx \bywf \tau
         \\ \tctx \bysub A \subst a \tau \tsub B
            }{
            \tpresub \forall a. A \tsub B
            }\rname{S-ForallL}}

\newcommand*{\SFun}{\inferrule{
            \tpresub B_1 \tsub A_1
         \\ \tpresub A_2 \tsub B_2
            }{
            \tpresub A_1 \to A_2 \tsub B_1 \to B_2
            }\rname{S-Fun}}

\newcommand*{\STVar}{\inferrule{
            a \in \tctx
            }{
            \tctx \bysub a \tsub a
            }\rname{S-TVar}}

\newcommand*{\SInt}{\inferrule{
            }{
            \tctx \bysub \nat \tsub \nat
            }\rname{S-Int}}

% ------------------------------------------------------
% GRADUAL HIGHER-RANKED TYPE
% ------------------------------------------------------

\newcommand*{\HRVar}{\inferrule{
            x : A \in \tctx
            }{
            \tctx \byinf x \infto A
            }\rname{Var}}

\newcommand*{\HRNat}{\inferrule{
            }{
            \tpreinf n \infto \nat
            }\rname{Nat}}

\newcommand*{\HRLamAnnA}{\inferrule{
            \tctx, x: A \byinf e \infto B
            }{
            \tctx \byinf \blam x A e \infto A \to B
            }\rname{LamAnn-I}}

\newcommand*{\HRLamAnnB}{\inferrule{
            C \match A_1 \to B
         \\ \tpresub A_1 \tconssub A_2
         \\ A = A_2 \glb A_1
         \\ \tctx, x: A \bychk e \chkby B
            }{
            \tctx \byinf \blam x {A_2} e \chkby C
            }\rname{LamAnn-C}}

\newcommand*{\HRApp}{\inferrule{
            \tctx \byinf e_1 \infto A
         \\ A \match A_1 \to A_2
         \\ \tctx \bychk e_2 \chkby A_1
            }{
            \tctx \byinf e_1 ~ e_2 \infto A_2
            }\rname{App}}

\newcommand*{\HRSub}{\inferrule{
            e \neq (\blam x C e')
         \\ \tctx \byinf e \infto A
         \\ \tpresub A \tconssub B
            }{
            \tctx \bychk e \chkby B
            }\rname{Sub}}

\newcommand*{\HRAll}{\inferrule{
            \tctx, a \bychk e \chkby A
            }{
            \tctx \bychk e \chkby \forall a. A
            }\rname{Forall}}

% -- SUBTYPING

\newcommand*{\HSForallR}{\inferrule{
            \tpresub[,a] A \tsub B
            }{
            \tpresub A \tsub \forall a. B
            }\rname{S-ForallR}}

\newcommand*{\HSForallL}{\inferrule{
            \dctx \bywf \tau
         \\ \tpresub A \subst a \tau \tsub B
            }{
            \tpresub \forall a. A \tsub B
            }\rname{S-ForallL}}

\newcommand*{\HSFun}{\inferrule{
            \tpresub B_1 \tsub A_1
         \\ \tpresub A_2 \tsub B_2
            }{
            \tpresub A_1 \to A_2 \tsub B_1 \to B_2
            }\rname{S-Fun}}

\newcommand*{\HSTVar}{\inferrule{
            a \in \dctx
            }{
            \tpresub a \tsub a
            }\rname{S-TVar}}

\newcommand*{\HSInt}{\inferrule{
            }{
            \tpresub \nat \tsub \nat
            }\rname{S-Int}}

\newcommand*{\HSUnknown}{\inferrule{
            }{
            \tpresub \unknown \tsub \unknown
            }\rname{S-Unknown}}

\newcommand*{\HSSVar}{\inferrule{
            }{
            \tpresub \static \tsub \static
            }\rname{S-SVar}}

\newcommand*{\HSGVar}{\inferrule{
            }{
            \tpresub \gradual \tsub \gradual
            }\rname{S-GVar}}

% ------------------------------------------------------
% GRADUAL HIGHER-RANKED TYPING : DECLARATIVE
% ------------------------------------------------------

\newcommand*{\DVar}{\inferrule{
            x : A \in \dctx
            }{
            \dctx \byinf x : A
            \trto {x}
            }\rname{Var}}

\newcommand*{\DNat}{\inferrule{
            }{
            \tpreinf n : \nat
            \trto {n}
            }\rname{Nat}}

\newcommand*{\DLam}{\inferrule{
            \dctx, x: \tau \byinf e : B
            \trto {s}
            }{
            \dctx \byinf \erlam x e : \tau \to B
            \trto {\blam x \tau s}
            }\rname{Lam}}

\newcommand*{\DLamAnnA}{\inferrule{
            \dctx, x: A \byinf e : B
            \trto {s}
            }{
            \dctx \byinf \blam x A e : A \to B
            \trto {\blam x A s}
            }\rname{LamAnn}}

\newcommand*{\DApp}{\inferrule{
            \dctx \byinf e_1 : A
            \trto {s_1}
         \\ \dctx \byinf A \match A_1 \to A_2
         \\ \dctx \byinf e_2 : A_3
            \trto {s_2}
         \\ \tpreconssub A_3 \tconssub A_1
            }{
            \dctx \byinf e_1 ~ e_2 : A_2
            \trto {(\cast A {A_1 \to A_2} s_1) ~
            (\cast {A_3} {A_1} s_2)
            }
            }\rname{App}}

\newcommand*{\DGen}{\inferrule{
            \dctx, a \byinf e : A \trto {s}
            }{
            \dctx \byinf e : \forall a. A
            \trto {\Lambda a. s}
            }\rname{Gen}}

% ------------------------------------------------------
% GRADUAL HIGHER-RANKED TYPING : Algorithmic
% ------------------------------------------------------

\newcommand*{\AVar}{\inferrule{
            (x : A) \in \tctx
            }{
            \Gamma \vdash x \infto A \toctxo
            }\rname{AVar}}

\newcommand*{\ANat}{\inferrule{
            }{
            \Gamma \vdash n \infto \nat \toctxo
            }\rname{ANat}}

\newcommand*{\ALamU}{\inferrule{
             \tctx, \genA, \genB, x : \genA \bychk e \chkby \genB \toctxr, x : \genA, \ctxl
            }{
            \tctx \byinf \erlam x e \infto \genA \to \genB \toctxr
            }\rname{ALamU}}


\newcommand*{\ALamAnnA}{\inferrule{
            \tctx, x: A \byinf e \infto B \toctxr,  x : A , \ctxl
            }{
            \tctx \byinf \blam x A e \infto A \to B \toctxr
            }\rname{ALamAnnA}}

\newcommand*{\ALam}{\inferrule{
            \tctx, x: A \byinf e \chkby B \toctxr,  x : A , \ctxl
            }{
            \tctx \byinf \erlam x e \chkby A \to B \toctxr
            }\rname{ALam}}

\newcommand*{\AGen}{\inferrule{
            \tctx, a \bychk e \chkby A \toctxr, a, \ctxl
            }{
            \tctx \bychk e \chkby \forall a. A \toctxr
            }\rname{AGen}}

\newcommand*{\AAnno}{\inferrule{
            \tctx \vdash A
            \\
            \tctx \bychk e \chkby A \toctxr
            }{
            \tctx \bychk e : A \infto A \toctxr
            }\rname{AAnno}}


\newcommand*{\AApp}{\inferrule{
            \Gamma \vdash e_1 \infto A \dashv \ctxl_1
         \\ \ctxl_1 \byinf \ctxsubst{\ctxl_1}{A} \match A_1 \to A_2 \dashv \ctxl_2
         \\ \ctxl_2 \byinf e_2 \chkby \ctxsubst{\ctxl_2}{A_1} \dashv \ctxr
            }{
            \Gamma \vdash e_1 ~ e_2 \infto A_2 \toctxr
            }\rname{AApp}}

\newcommand*{\ASub}{\inferrule{
            \tctx \byinf e \infto A \toctx
         \\ \ctxl \bysub \ctxsubst{\ctxl} A \tconssub \ctxsubst{\ctxl} B \toctxr
            }{
            \tctx \bychk e \chkby B \toctxr
            }\rname{ASub}}


% ------------------------------------------------------
% Context extension
% ------------------------------------------------------


\newcommand*{\ExtID}{\inferrule{
            }{
            \ctxinit \exto \ctxinit
            }\rname{ExtID}}

\newcommand*{\ExtVar}{\inferrule{
              \Gamma \exto \Delta \\
              \ctxsubst{\Delta}{A} = \ctxsubst{\Delta}{A'}
            }{
            \Gamma, x : A \exto \Delta, x : A'
            }\rname{ExtVar}}

\newcommand*{\ExtUVar}{\inferrule{
              \Gamma \exto \Delta
            }{
            \Gamma, a \exto \Delta, a
            }\rname{ExtUVar}}

\newcommand*{\ExtEVar}{\inferrule{
              \Gamma \exto \Delta
            }{
            \Gamma, \genA \exto \Delta, \genA
            }\rname{ExtEVar}}

\newcommand*{\ExtSolved}{\inferrule{
              \Gamma \exto \Delta \\
              \ctxsubst{\Delta}{\tau} = \ctxsubst{\Delta}{\tau'}
            }{
            \Gamma, \genA = \tau \exto \Delta, \genA = \tau'
            }\rname{ExtSolved}}

\newcommand*{\ExtSolve}{\inferrule{
              \Gamma \exto \Delta
            }{
            \Gamma, \genA \exto \Delta, \genA = \tau
            }\rname{ExtSolve}}

\newcommand*{\ExtAdd}{\inferrule{
              \Gamma \exto \Delta
            }{
            \Gamma \exto \Delta, \genA
            }\rname{ExtAdd}}

\newcommand*{\ExtAddS}{\inferrule{
              \Gamma \exto \Delta
            }{
            \Gamma \exto \Delta, \genA = \tau
            }\rname{ExtAddSolved}}



% ------------------------------------------------------
% HIGHER-RANKED TYPING : NON-BI
% ------------------------------------------------------

\newcommand*{\NVar}{\inferrule{
            x : A \in \dctx
            }{
            \dctx \byhinf x : A
            }\rname{Var}}

\newcommand*{\NNat}{\inferrule{
            }{
            \dctx \byhinf n : \nat
            }\rname{Nat}}

\newcommand*{\NLam}{\inferrule{
            \dctx, x: \tau \byhinf e : B
            }{
            \dctx \byhinf \erlam x e : \tau \to B
            }\rname{Lam}}

\newcommand*{\NGen}{\inferrule{
            \dctx, a \byhinf e : A
            }{
            \dctx \byhinf e : \forall a. A
            }\rname{Gen}}

\newcommand*{\NLamAnnA}{\inferrule{
            \dctx, x: A \byhinf e : B
            }{
            \dctx \byhinf \blam x A e : A \to B
            }\rname{LamAnn}}

\newcommand*{\NApp}{\inferrule{
            \dctx \byhinf e_1 : A_1 \to A_2
         \\ \dctx \byhinf e_2 : A_1
            }{
            \dctx \byhinf e_1 ~ e_2 : A_2
            }\rname{App}}

\newcommand*{\NSub}{\inferrule{
            \dctx \byhinf e : A_1
         \\ \tpresub A_1 \tsub A_2
            }{
            \dctx \byhinf e : A_2
            }\rname{Sub}}

\newcommand*{\NForallR}{\inferrule{
            \tpresub[,a] A \tsub B
            }{
            \tpresub A \tsub \forall a. B
            }\rname{ForallR}}

\newcommand*{\NForallL}{\inferrule{
            \dctx \bywf \tau
         \\ \tpresub A \subst a \tau \tsub B
            }{
            \tpresub \forall a. A \tsub B
            }\rname{ForallL}}

\newcommand*{\NFun}{\inferrule{
            \tpresub B_1 \tsub A_1
         \\ \tpresub A_2 \tsub B_2
            }{
            \tpresub A_1 \to A_2 \tsub B_1 \to B_2
            }\rname{CS-Fun}}

\newcommand*{\NTVar}{\inferrule{
            a \in \dctx
            }{
            \tpresub a \tsub a
            }\rname{CS-TVar}}

\newcommand*{\NSInt}{\inferrule{
            }{
            \tpresub \nat \tsub \nat
            }\rname{CS-Int}}


% ------------------------------------------------------
% MASK OFF
% ------------------------------------------------------

\newcommand*{\FA}{\inferrule{
            }{
            \mask A \unknown = \unknown
            }\rname{F-StarR}}

\newcommand*{\FB}{\inferrule{
            }{
            \mask \unknown A = \unknown
            }\rname{F-StarL}}

\newcommand*{\FC}{\inferrule{
            }{
            \mask {\forall a. A} B = \mask A B
            }\rname{F-ForallL}}

\newcommand*{\FD}{\inferrule{
            }{
            \mask A {\forall a. B} = \mask A B
            }\rname{F-ForallR}}

\newcommand*{\FE}{\inferrule{
            }{
            \mask {A_1 \to A_2} {B_1 \to B_2} = \mask {A_1} {B_1} \to \mask {A_2} {B_2}
            }\rname{F-Fun}}

% ------------------------------------------------------
% PBC
% ------------------------------------------------------

\newcommand*{\PBCVar}{\inferrule{
            x : A \in \tctx
            }{
            \tctx \bypinf x \infto A
            }\rname{PBC-Var}}

\newcommand*{\PBCNat}{\inferrule{
            }{
            \tctx \bypinf n \infto \nat
            }\rname{PBC-Var}}

\newcommand*{\PBCApp}{\inferrule{
            \tctx \bypinf e_1 \infto A_1 \to A_2
         \\ \tctx \bypinf e_2 \infto A_1
            }{
            \tctx \bypinf e_1 ~ e_2 \infto A_2
            }\rname{PBC-App}}

\newcommand*{\PBCLam}{\inferrule{
            \tctx, x: A \bypinf e \infto B
            }{
            \tctx \bypinf \blam x A e \infto A \to B
            }\rname{PBC-Lam}}

\newcommand*{\PBCBLam}{\inferrule{
            \tctx, X \bypinf e \infto A
            }{
            \tctx \bypinf \Lambda X. e \infto \forall X. A
            }\rname{PBC-BLam}}

\newcommand*{\PBCTApp}{\inferrule{
            \tctx, X \bypinf e \infto  \forall X. A
            }{
            \tctx \bypinf e ~ [B] \infto A \subst X B
            }\rname{PBC-TApp}}

\newcommand*{\PBCCast}{\inferrule{
            \tctx \bypinf e \infto A
         \\ A \pbccons B
            }{
            \tctx \bypinf \cast A B e \infto B
            }\rname{PBC-TApp}}

% ------------------------------------------------------
% PBC Compatibility
% ------------------------------------------------------

\newcommand*{\CompRefl}{\inferrule{
            }{
            A \pbccons A
            }\rname{Comp-Refl}}

\newcommand*{\CompUnknownR}{\inferrule{
            }{
            A \pbccons \unknown
            }\rname{Comp-UnknownR}}

\newcommand*{\CompUnknownL}{\inferrule{
            }{
            \unknown \pbccons A
            }\rname{Comp-UnknownL}}

\newcommand*{\CompArrow}{\inferrule{
            A_1 \pbccons B_1
         \\ A_2 \pbccons B_2
            }{
            A_1 \to A_2 \pbccons B_1 \to B_2
            }\rname{Comp-Arrow}}

\newcommand*{\CompAllR}{\inferrule{
            A \pbccons B
            }{
            A \pbccons \forall X. B
            }\rname{Comp-AllR}}

\newcommand*{\CompAllL}{\inferrule{
            A \subst X \star \pbccons B
            }{
            \forall X. A \pbccons B
            }\rname{Comp-AllL}}

% ------------------------------------------------------
% EXTENSION
% ------------------------------------------------------

\newcommand*{\SubTop}{\inferrule{
            A ~ static
            }{
            \dctx \bysub A \tsub \top
            }\rname{S-Top}}

\newcommand*{\CTop}{\inferrule{}{
            \top \sim \top
            }}

\newcommand*{\CSTop}{\inferrule{
            }{
            \dctx \bysub A \tconssub \top
            }\rname{CS-Top}}


% ------------------------------------------------------
% Well-formedess of type under declarative context
% ------------------------------------------------------

\newcommand*{\DeclVarWF}{\inferrule{
              a \in \dctx
            }{
            \dctx \vdash a
            }\rname{DeclVarWF}}

\newcommand*{\DeclIntWF}{\inferrule{
            }{
            \dctx \vdash \nat
            }\rname{DeclIntWF}}

\newcommand*{\DeclUnknownWF}{\inferrule{
            }{
            \dctx \vdash \unknown
            }\rname{DeclUnknownWF}}


\newcommand*{\DeclFunWF}{\inferrule{
              \dctx \vdash A \\ \dctx \vdash B
            }{
            \dctx \vdash A \to B
            }\rname{DeclFunWF}}

\newcommand*{\DeclForallWF}{\inferrule{
              \dctx, a \vdash A
            }{
            \dctx \vdash \forall a. A
            }\rname{DeclForallWF}}

% ------------------------------------------------------
% Well-formedess of type under algorithmic context
% ------------------------------------------------------

\newcommand*{\VarWF}{\inferrule{
            }{
            \Gamma[a] \vdash a
            }\rname{VarWF}}

\newcommand*{\IntWF}{\inferrule{
            }{
            \Gamma \vdash \nat
            }\rname{IntWF}}

\newcommand*{\UnknownWF}{\hlmath{\inferrule{
            }{
            \Gamma \vdash \unknown
            }\rname{UnknownWF}}}

\newcommand*{\FunWF}{\inferrule{
              \Gamma \vdash A \\ \Gamma \vdash B
            }{
            \Gamma \vdash A \to B
            }\rname{FunWF}}

\newcommand*{\ForallWF}{\inferrule{
              \Gamma, a \vdash A
            }{
            \Gamma \vdash \forall a. A
            }\rname{ForallWF}}

\newcommand*{\EVarWF}{\inferrule{
            }{
            \Gamma[\genA] \vdash \genA
            }\rname{EVarWF}}

\newcommand*{\SolvedEVarWF}{\inferrule{
            }{
            \Gamma[\genA = \tau] \vdash \genA
            }\rname{SolvedEVarWF}}

% ------------------------------------------------------
% OBJECTS: SUBTYPING
% ------------------------------------------------------

\newcommand*{\ObSInt}{\inferrule{}
            {
            \nat \tsub \nat
            }}

\newcommand*{\ObSBool}{\inferrule{}
            {
            \bool \tsub \bool
            }}

\newcommand*{\ObSFloat}{\inferrule{}
            {
            \float \tsub \float
            }}

\newcommand*{\ObSIntFloat}{\inferrule{}
            {
            \nat \tsub \float
            }}

\newcommand*{\ObFun}{\inferrule{B_1 \tsub A_1 \\ A_2 \tsub B_2}
            {
            A_1 \to A_2 \tsub B_1 \to B_2
            }}

\newcommand*{\ObSUnknown}{\inferrule{}
            {
            \unknown \tsub \unknown
            }}

\newcommand*{\ObSRecord}{\inferrule{}
            {
            [l_i : A_i^{i \in 1...n+m}] \tsub
            [l_i : A_i^{i \in 1...n}]
            }}

% ------------------------------------------------------
% OBJECTS: CONSISTENCY
% ------------------------------------------------------

\newcommand*{\ObCRefl}{\inferrule{}
            {
            A \sim A
            }}

\newcommand*{\ObCUnknownR}{\inferrule{}
            {
            A \sim \unknown
            }}

\newcommand*{\ObCUnknownL}{\inferrule{}
            {
            \unknown \sim A
            }}

\newcommand*{\ObCC}{\inferrule{
            A_1 \sim B_1
         \\ A_2 \sim B_2
            }{
            A_1 \to A_2 \sim B_1 \to B_2
            }}


\newcommand*{\ObCRecord}{\inferrule{
            A_i \sim B_i
            }
            {
            [l_i: A_i] \sim [l_i:B_i]
            }}

\begin{document}

%% Title information

\title{Consistent Subtyping for All}


\author{Ningning Xie}
\affiliation{%
  \institution{The University of Hong Kong}
  \city{Hong Kong}
  \country{China}}
\email{nnxie@cs.hku.hk}
\author{Xuan Bi}
\affiliation{%
  \institution{The University of Hong Kong}
  \city{Hong Kong}
  \country{China}
}
\email{xbi@cs.hku.hk}
\author{Bruno C. d. S. Oliveira}
\affiliation{%
 \institution{The University of Hong Kong}
 \city{Hong Kong}
 \country{China}}
\email{bruno@cs.hku.hk}
\author{Tom Schrijvers}
\affiliation{%
 \institution{KU Leuven}
 \city{Leuven}
 \country{Belgium}}
\email{tom.schrijvers@cs.kuleuven.be}




\begin{abstract}
  Consistent subtyping is employed in some gradual type systems to validate type
  conversions. The original definition by \citeauthor{siek2007gradual} serves as a guideline for designing gradual type
  systems with subtyping. Polymorphic types \`a la System F also induce a
  subtyping relation that relates polymorphic types to their instantiations.
  However \citeauthor{siek2007gradual}'s definition is not adequate for
  polymorphic subtyping.
% , including the subtyping
% relation arising from implicit polymorphism.
The first goal of this paper is to propose a generalization of consistent
subtyping that is adequate for polymorphic subtyping, and subsumes the original
definition by \citeauthor{siek2007gradual}. The new definition of consistent
subtyping provides novel insights with respect to previous polymorphic gradual
type systems, which did not employ consistent subtyping.
% For instance both
% \citeauthor{ahmed2011blame} (in the Polymorphic Blame Calculus) and
% \citeauthor{yuu2017poly} use, respectively,
% notions of \emph{compatibility} and \emph{type consistency} instead
% of consistent subtyping to validate casts.
% We argue that, for implicit
% polymorphism, \citeauthor{ahmed2011blame}'s notion of
% compatibility is too permissive (i.e. too many programs are allowed to
% type-check), and that \citeauthor{yuu2017poly}'s notion of type consistency is too
% conservative (i.e. programs that should type check are rejected).
% To further validate our
% generalized notion of consistent subtyping we also study the addition
% of top types to gradual type systems. Like polymorphism, the notion of
% consistent subtyping proposed by \citeauthor{siek2007gradual} does not work for top
% types, but our revised definition can account for top types.
The second goal of this paper is to present a gradually typed calculus for
implicit (higher-rank) polymorphism that uses our new notion of consistent
subtyping. We develop both declarative and (bidirectional) algorithmic versions
for the type system.
The algorithmic version employs techniques developed by
\citeauthor{dunfield2013complete} for higher-rank polymorphism to deal with instantiation.
We prove that the
new calculus satisfies all static aspects of the refined criteria for gradual
typing. 
We also study an extension of the type system with static and gradual
type parameters, in an attempt to support a variant of the dynamic
criterion for gradual typing.
Assuming a coherence conjecture for the extended calculus,
we show that the dynamic gradual guarantee of our source
language can be reduced to that of \pbc, which, at the time of writing, is
still an open question.
Most of the metatheory of this paper, except some manual proofs for the algorithmic type
system and extensions, has been mechanically formalized using the Coq proof assistant.
\end{abstract}


%
% The code below should be generated by the tool at
% http://dl.acm.org/ccs.cfm
% Please copy and paste the code instead of the example below.
%
% \begin{CCSXML}
% <ccs2012>
%  <concept>
%   <concept_id>10010520.10010553.10010562</concept_id>
%   <concept_desc>Computer systems organization~Embedded systems</concept_desc>
%   <concept_significance>500</concept_significance>
%  </concept>
%  <concept>
%   <concept_id>10010520.10010575.10010755</concept_id>
%   <concept_desc>Computer systems organization~Redundancy</concept_desc>
%   <concept_significance>300</concept_significance>
%  </concept>
%  <concept>
%   <concept_id>10010520.10010553.10010554</concept_id>
%   <concept_desc>Computer systems organization~Robotics</concept_desc>
%   <concept_significance>100</concept_significance>
%  </concept>
%  <concept>
%   <concept_id>10003033.10003083.10003095</concept_id>
%   <concept_desc>Networks~Network reliability</concept_desc>
%   <concept_significance>100</concept_significance>
%  </concept>
% </ccs2012>
% \end{CCSXML}

% \ccsdesc[500]{Computer systems organization~Embedded systems}
% \ccsdesc[300]{Computer systems organization~Redundancy}
% \ccsdesc{Computer systems organization~Robotics}
% \ccsdesc[100]{Networks~Network reliability}

%
% End generated code
%


% \keywords{Wireless sensor networks, media access control,
% multi-channel, radio interference, time synchronization}


\maketitle



%% -- Starting Point --

\section{Introduction}

Modern statically typed functional languages (such as ML, Haskell,
Scala or OCaml) have increasingly expressive type systems. Often these
large source languages are translated into a much smaller typed core
language. The choice of the core language is essential to ensure that
all the features of the source language can be encoded. For a simple
polymorphic functional language it is possible to pick a
variant of System $F$~\cite{systemfw,Reynolds:1974} as a core
language. However, the desire for more expressive type system features
puts pressure on the core languages, often requiring them to be
extended to support new features.  For example, if the source language
supports \emph{higher-kinded types} or \emph{type-level functions}
then System $F$ is not expressive enough and can no longer be used as
the core language. Instead another core language that does provide
support for higher-kinded types, such as
System~$F_{\omega}$~\cite{systemfw}, needs to be used. Of course the
drive to add more and more advanced type-level features means that
eventually the core language needs to be extended again. Indeed modern
functional languages like Haskell use specially crafted core
languages, such as System $F_{C}$~\cite{fc}, that provide support for all
modern features of Haskell. Although \emph{extensions} of System
$F_{C}$~\cite{fc:pro,Eisenberg:2014} satisfy the current needs of
modern Haskell, it is very likely to be extended again in the
future~\cite{fc:kind}. Moreover System $F_{C}$ has grown to be a relatively
large and complex language, with multiple syntactic levels, and dozens
of language constructs.

\begin{comment}
However System~$F_{\omega}$ is
significantly more complex than System F and thus harder to
maintain. If later a new feature, such as \emph{kind polymorphism}, is
desired the core language may need to be changed again to account for
the new feature, introducing at the same time new sources of
complexity. Indeed the core language for modern versions of 
functional languages are quite complex, having multiple syntactic 
sorts (such as terms, types and kinds), as well as dozens of 
language constructs~\cite{}\bruno{$F_{C}$}. 
\end{comment}

The more expressive type (and kind) systems become, the more types become similar
to the terms. Therefore a natural idea is to unify terms and
types. There are obvious benefits in this approach: only one syntactic
level (terms) is needed; and there are much less language constructs,
making the core language easier to reason, implement and maintain. At the same
time the core language becomes more expressive, giving us for free
many useful language features. Moreover, due to the inherent
expressiveness, extensions are less likely to be required.
\emph{Pure type systems} (PTS)~\cite{handbook} build
on such observations and show how a whole family of type systems
(including System $F$ and System $F_{\omega}$) can be implemented
using just a single syntactic form. With the added expressiveness it
is even possible to have type-level programs expressed using the same
syntax as terms, as well as dependently typed programs~\cite{coc}.
Because the idea of using a unified syntax is so appealing several
researchers have in the past considered such an
option for implementing functional languages~\cite{cayenne, typeintype, pts:henk}.

However having the same syntax for types and terms can also be
problematic. Usually type systems based on PTS have a conversion rule
to support type-level computation.  In such type systems ensuring the
\emph{decidability} of type checking requires type-level computation
to terminate. When the syntax of types and terms is the same, the
decidability of type checking is usually dependent on the strong
normalization of the calculus. An example is the proof of decidability
of type checking for the \emph{calculus of constructions}~\cite{coc}
(and other normalizing PTS), which depends on strong normalization
~\cite{pts:normalize}. Modern dependently
typed languages such as Idris~\cite{idris} and Agda~\cite{agda}, which are also
built on a unified syntax for types and terms, require strong
normalization as well: all recursive programs must pass a termination
checker.  An unfortunate consequence of coupling
decidability of type checking and strong normalization is that adding
(unrestricted) general recursion to such calculi is difficult. Indeed
past work on using a simple PTS-like calculi to model functional languages
with unrestricted general recursion, had to give up on decidability of
type-checking~\cite{cayenne, typeintype}.
%There
%is a clear tension between decidability of type checking and allowing
%general recursion in calculi with unified syntax.

This paper proposes \name: a simple yet expressive call-by-name
variant of the calculus of constructions, which has a fraction of the
language constructs of existing core languages. The key challenge
solved in this work is how to define a calculus comparable in
simplicity to the calculus of constructions, while featuring both
general recursion and decidable type checking. The main idea, 
inspired by the traditional treatment of \emph{iso-recursive
  types}~\cite{tapl}, is to recover decidable type-checking by making each
type-level computation step explicit, i.e., each beta reduction or
expansion at the type level is controlled by a \emph{type-safe}
cast. Since single computation steps are trivially terminating, decidability
of type checking is possible even in the presence of non-terminating
programs at the type level.  At the same time term-level programs
using general recursion work as in any conventional functional
languages, and can even be non-terminating.

\begin{comment}
For example, if a type-level program requires two beta reductions to
reach normal form, then two casts are needed in the program. If a
non-terminating program is used at the type level, it is not possible
to cause non-termination in the type checker, because that would
require a program with an infinite number of casts. Therefore, since
single beta-steps are trivially terminating, decidability of type
checking is possible even in the presence of non-terminating programs
at the type level.  At the same time term-level programs using general
recursion work as in any conventional functional languages, and can
even be non-terminating.
\end{comment}

Our motivation to develop \name is to use it as a simpler alternative
to existing core languages for functional programming. We focus on traditional
functional languages like ML or Haskell extended with many interesting
type-level features, but perhaps not the \emph{full power} of
dependent types.  The paper shows how many of programming language
features of Haskell, including some of the latest extensions, can be
encoded in \name via a surface language. The surface
language supports \emph{algebraic datatypes}, \emph{higher-kinded
  types}, \emph{nested datatypes}~\cite{nesteddt}, \emph{kind
  polymorphism}~\cite{fc:pro} and \emph{datatype
  promotion}~\cite{fc:pro}.  This result is interesting because \name
is a minimal calculus with only 8 language constructs and a single
syntactic sort. In contrast the latest versions of System $F_{C}$
(Haskell's core language) have multiple syntactic sorts and dozens of
language constructs.
%Even if support for equality and
%coercions, which constitutes a significant part of System $F_{C}$,
%would be removed the resulting language would still be significantly
%larger and more complex than \name.

It is worth emphasizing that \name does sacrifice having an expressive form
of type equality to gain the ability of doing arbitrary general
recursion at the term level.  Nevertheless, 
the core language (System $F_{C}$) of Haskell also comes with a similarly weak
notion of type equality.  In both System $F_{C}$ and \name, type
equality in \name is purely syntactic (modulo alpha-conversion).

A non-goal of the current work (although a worthy avenue for future
work) is to use \name as a core language for modern dependently typed
languages like Agda or Idris. In contrast to \name, those languages
use a more powerful notion of equality. In particular \name
currently lacks full-reduction and it is unable to exploit injectivity 
properties when comparing two types for equality. Moreover,
\name (and also System $F_{C}$) lack \emph{logical consistency}:
that is ensuring the soundness of proofs written as programs.
This is in contrast to dependently typed languages, where logical
consistency is typically ensured.
Various researchers~\cite{zombie:popl14,zombie:thesis,Swamy2011} have been investigating how to combine logical
consistency, general recursion and dependent types. However, this is
usually done by having the type system carefully control the total and
partial parts of computation, making those calculi significantly more
complex than \name or the calculus of constructions. In
\name, logical consistency is traded by the simplicity of the system.

\begin{comment}
In particular
the treatment of type-level computation in \name shares similar ideas
with Haskell. Although Haskell's surface language provides a rich set
of mechanisms to do type-level computation~\cite{}, the core language
lacks fundamental mechanisms todo type-level computation. Type
equality in System $F_{C}$ is, like in \name, purely syntactic (modulo
alpha-conversion).
\end{comment}

\begin{comment}
 and there is no type-level
abstraction. In other words in Haskell, mechanisms such as type
classes and type families

Although it may seem that forcing each step of computation 
at the type-level to be explicit will prevent convinient use of 
type-level computation.

Point about the treatment of type-level computation in Haskell. Haskell's
core language has type applications, but no type-level lambda. Equality 
is syntactic modulo alpha-conversion. This design choice was rooted in the 
desire to support Hindley-Milner type-inference... 
\end{comment}

In summary, the contributions of this work are:

\begin{itemize}
\item {\bf The \name calculus:} A simple core calculus for functional programming, that collapses terms, types and
  kinds into the same hierarchy and supports general recursion. \name
  is type-safe and the type system is decidable.

\item {\bf One-step casts and a generalization of iso-recursive types:} \name 
 generalizes iso-recursive types by making all type-level computation
 steps explicit via \emph{one-step casts}. In \name the combination of
  one-step casts and recursion subsumes iso-recursive types.

\item {\bf An expressive surface language}, built on top of \name,
  that supports datatypes, pattern matching and various advanced
  language extensions of Haskell. The type safety of the type-directed
  translation to \name is proved.

\item {\bf A prototype implementation:} The implementation of \name is
  available\footnote{\url{https://github.com/bixuanzju/full-version}}.
\end{itemize}

\begin{comment}
\begin{enumerate}[a)]
\item Motivations:

\begin{itemize}

\item Because of the reluctance to introduce dependent
  types\footnote{This might be changed in the near future. See
    \url{https://ghc.haskell.org/trac/ghc/wiki/DependentHaskell/Phase1}.},
  the current intermediate language of Haskell, namely System $F_C$
  \cite{fc}, separates expressions as terms, types and kinds, which
  brings complexity to the implementation as well as further
  extensions \cite{fc:pro,fc:kind}.

\item Popular full-spectrum dependently typed languages, like Agda,
  Coq, Idris, have to ensure the termination of functions for the
  decidability of proofs. No general recursion and the limitation of
  enforcing termination checking make such languages impractical for
  general-purpose programming.

\item We would like to introduce a simple and compiler-friendly
  dependently typed core language with only one hierarchy, which
  supports general recursion at the same time.

\end{itemize}

\item Contribution:

\begin{itemize}

\item A core language based on Calculus of Constructions (CoC) that
  collapses terms, types and kinds into the same hierarchy.

\item General recursion by introducing recursive types for both terms
  and types by the same $\mu$ primitive.

\item Decidable type checking and managed type-level computation by
  replacing implicit conversion rule of CoC with generalized
  \textsf{fold}/\textsf{unfold} semantics.

\item First-class equality by coercion, which is used for encoding
  GADTs or newtypes without runtime overhead.

\item Surface language that supports datatypes, pattern matching and
  other language extensions for Haskell, and can be encoded into the
  core language.

\end{itemize}


\end{enumerate}
\end{comment}

\section{Background}
\label{sec:background}

In this section we review a simple gradually typed language with
objects~\cite{siek2007gradual}, to introduce the concept of consistency
subtyping. We also briefly talk about the Odersky-L{\"a}ufer type system for
higher-rank types~\cite{odersky1996putting}, which serves as the original
language on which our gradually typed calculus with implicit
higher-rank polymorphism is based.


\subsection{Gradual Subtyping}

\begin{figure}[t]
  \begin{small}
  \begin{mathpar}
    \framebox{$A \tsub B$}\\
    \ObSInt \and \ObSBool \and \ObSFloat \and
    \ObSIntFloat \\ \ObFun \and
    \ObSRecord \and \ObSUnknown
  \end{mathpar}

  \begin{mathpar}
    \framebox{$A \sim B$}\\
    \ObCRefl \and \ObCUnknownR \and
    \ObCUnknownL \and \ObCC \and \ObCRecord
  \end{mathpar}

  \end{small}

  \caption{Subtyping and type consistency in \obb}
  \label{fig:objects}
\end{figure}

\citet{siek2007gradual} developed a gradual typed system for object-oriented
languages that they call \obb. Central to gradual typing is the concept of
\textit{consistency} (written $\sim$) between gradual types, which are types
that may involve the unknown type $\unknown$. The intuition is that consistency
relaxes the structure of a type system to tolerate unknown positions in a
gradual type. They also defined the subtyping relation in a way that static type
safety is preserved. Their key insight is that the unknown type $\unknown$ is
neutral to subtyping, with only $\unknown \tsub \unknown$. Both relations are
found in \Cref{fig:objects}.

A primary contribution of their work is to show that type consistency and
subtyping are orthogonal, and can be superimposed. To compose subtyping and
consistency, \citeauthor{siek2007gradual} defined \textit{consistent subtyping}
(written $\tconssub$) in multiple equivalent ways:

\begin{mdef}[Consistent Subtyping \`a la \citet{siek2007gradual}]\leavevmode
\label{def:old-decl-conssub}
\begin{itemize}
\item $A \tconssub B$ if and only if $A \sim C$ and $C \tsub B$ for some $C$.
\item $A \tconssub B$ if and only if $A \tsub C$ and $C \sim B$ for some $C$.
\end{itemize}
\end{mdef}

Both definitions are non-deterministic because of the intermediate type $C$. To
remove non-determinism, they came up with a so-called \textit{restriction
  operator}, written $\mask A B$ that masks off the parts of a type $A$ that are
unknown in a type $B$. The definition of the restriction operator is given
below:
\begin{small}
\begin{align*}
  \mask A B & =  \kw{case}~(A, B)~\kw{of}\\
               & \mid (-, \unknown) \Rightarrow \unknown\\
               & \mid ([l_1: A_1,...,l_n:A_n], [l_1:B_1,...,l_m:B_m]) \quad \kw{where} n \leq m \Rightarrow\\
               & \qquad [l_1 : \mask {A_1} {B_1}, ..., l_n : \mask {A_n} {B_n}]\\
               & \mid ([l_1: A_1,...,l_n:A_n], [l_1:B_1,...,l_m:B_m]) \quad \kw{where} n > m \Rightarrow\\
               & \qquad [l_1 : \mask {A_1} {B_1}, ..., l_m : \mask {A_m} {B_m},...,l_n:A_n ]\\
               & \mid (-, -) \Rightarrow A\\
  \mask {A_1 \to A_2} {B_1 \to B_2} & =  \mask {A_1} {B_1} \to \mask {A_2} {B_2}
\end{align*}
\end{small}
With the restriction operator, consistent subtyping is simply defined
as $A \tconssub B \equiv \mask A B \tsub \mask B A$. And they proved that this
definition is equivalent to \Cref{def:old-decl-conssub}.


\subsection{The Odersky-L{\"a}ufer Type System}

\begin{figure}[t]
  \begin{small}

    \begin{tabular}{lrcl} \toprule
      Expressions & $e$ & \syndef & $x \mid n \mid
                                    \blam x A e \mid e~e$ \\

      Types & $A, B$ & \syndef & $ \nat \mid a \mid A \to B \mid \forall a. A$ \\
      Monotypes & $\tau, \sigma$ & \syndef & $ \nat \mid a \mid \tau \to \sigma$ \\

      Contexts & $\dctx$ & \syndef & $\ctxinit \mid \dctx,x: A \mid \dctx, a$ \\  \bottomrule
    \end{tabular}

  \begin{mathpar}
    \framebox{$\dctx \byhinf e : A$}\\
    \NVar \and \NNat \and \NLamAnnA \and
    \NApp \and \NSub
  \end{mathpar}

  \begin{mathpar}
    \framebox{$\dctx \bysub A \tsub B$}\\
    \NForallL \and \NForallR \and
    \NFun \and \NTVar \and \NSInt
  \end{mathpar}

  \end{small}
  \caption{Syntax and static semantics of the Odersky-L{\"a}ufer type system.}
  \label{fig:original-typing}
\end{figure}


The calculus we are combining gradual typing with is the fully annotated version
of the well-established type system for higher-rank types proposed by
\citet{odersky1996putting}. One difference is that, for simplicity, we do not account 
for a let expression and, consequently, we do not have
let-generalization. However, there is already existing work about gradual type
systems with let expressions and let generalization (for example, see
\citep{garcia2015principal}). Similar techniques can
be applied to our calculus to enable let generalization.

The syntax of the type system, along with the typing and subtyping judgments is
given in \Cref{fig:original-typing}. We save the explanations for the
static semantics to \Cref{sec:type-system}, where we present our
gradually typed version of the calculus.

%%% Local Variables:
%%% mode: latex
%%% TeX-master: "../paper"
%%% org-ref-default-bibliography: "../paper.bib"
%%% End:
\section{Motivation and Applications}
\label{sec:motivation}

In this section we motivate why the combination of gradual typing and implicit
polymorphism is useful. We then illustrate two concrete applications related to
algebraic datatypes. The first application shows how gradual typing helps in
defining Scott encodings of algebraic datatypes~\citep{curry1958combinatory,
  parigot1992recursive}, which are impossible to encode in plain System F. The
second application shows how gradual typing makes it easy to model and use
heterogeneous containers.

% \subsection{Motivation}

% In this section we discuss why we are interested in integrating gradual
% typing and implicit polymorphism.

% \paragraph{Why bring gradual types to implicit polymorphism} The Glasgow
% Haskell Compiler (GHC) is a state-of-art compiler that is able to deal with many
% advanced features, including predicative implicit polymorphism. However,
% consider the program \lstinline{(\f. (f 1, f 'a')) id} that cannot type check in
% current GHC, where \lstinline{id} stands for the identity function. The problem
% with this expression is that type inference fails to infer a polymorphic type
% for $f$, given there is no explicit annotation for it. However, from dynamic
% point of view, this program would run smoothly. Also, requesting programmers to
% add annotations for each polymorphic types could become annoying when the
% program scales up and the annotation is long and complicated.

% Gradual types provides a simple solution for it. Rewriting the above expression
% to \lstinline{(\f: *. (f 1, f 'a')) id} produces the value \lstinline{(1, 'a')}.
% Namely, gradual typing provides an alternative that defers typing errors (if
% exists) into dynamic time. Also, without losing type safety, it allows
% programmer to be sloppy about annotations to be agile for program development,
% and later refine the typing annotations to regain the power of static typing.

% \paragraph{Why bring implicit polymorphism to gradual typing} There are several
% existing work about integrating explicit polymorphism into gradual type systems,
% with or without explicit casts~\citep{ahmed2011blame, yuu2017poly}, yet no work
% investigates to add the expressive power of implicit polymorphism into a
% source language.
% % In implicit
% % polymorphism, type applications can be reconstructed by the type checker, for
% % example, instead of \lstinline{id Int 3}, one can directly write \lstinline{id 3}.
% Implicit polymorphism is a hallmark of functional programming, and
% modern functional languages (such as Haskell) employ sophisticated
% type-inference algorithms that, aided by type annotations, can deal with
% higher-rank polymorphism. Therefore as a step towards gradualizing such type
% systems, this paper develops both declarative and algorithmic versions for a
% type system with implicit higher-rank polymorphism.

\subsection{Motivation: Gradually Typed Higher-Rank Polymorphism}

Our work combines implicit (higher-rank) polymorphism with gradual typing. As
is well known, a gradually typed language supports both fully static and fully
dynamic checking of program properties, as well as the continuum between these
two extremes. It also offers programmers fine-grained control over the
static-to-dynamic spectrum, i.e., a program can be evolved by introducing more
or less precise types as needed~\citep{garcia2016abstracting}. 

Haskell is a language renowned for its advanced type system, but it does not feature
gradual typing. Of particular interest to us is its support for implicit
higher-rank polymorphism, which is supported via explicit type
annotations. 
In Haskell some programs that are safe at run-time may
be rejected due to the conservativity of the type system. For example, consider
the following Haskell program adapted from \citet{jones2007practical}:
\begin{lstlisting}
foo :: ([Int], [Char])
foo = let f x = (x [1, 2] , x ['a', 'b']) in f reverse
\end{lstlisting}
This program is rejected by Haskell's type checker because Haskell implements
the Damas-Milner \citep{hindley69principal, damas1982principal} rule that a
lambda-bound argument (such as \lstinline{x}) can only have a monotype, i.e.,
the type checker can only assign \lstinline{x} the type
\lstinline{[Int] -> [Int]}, or \lstinline{[Char] -> [Char]},
but not \lstinline$foralla. [a] -> [a]$.
Finding such manual polymorphic annotations can be non-trivial, especially
when the program scales up and the annotation is long and complicated.

Instead
of rejecting the program outright, due to missing type annotations, gradual
typing provides a simple alternative by giving \lstinline$x$ the unknown type
$\unknown$. With this type the same program type-checks and produces
\lstinline$([2, 1], ['b', 'a'])$. By running the program, programmers can gain
more insight about its run-time behaviour. Then, with this insight, they can
also give \lstinline$x$ a more precise type (\lstinline$foralla. [a] -> [a]$) a
posteriori so that the program continues to type-check via implicit polymorphism
and also grants more static safety. In this paper, we envision such a language
that combines the benefits of both implicit higher-rank polymorphism and gradual
typing.

%-------------------------------------------------------------------------------
\subsection{Application: Efficient (Partly) Typed Encodings of ADTs}

Our calculus does not provide built-in support for algebraic datatypes (ADTs).
Nevertheless, the calculus is expressive enough to support efficient
function-based encodings of (optionally polymorphic) ADTs\footnote{In a type
  system with impure features, such as non-termination or exceptions, the encoded
  types can have valid inhabitants with side-effects, which means we only get
  the \textit{lazy} version of those datatypes.}.
This offers an immediate way to model algebraic
datatypes in our calculus without requiring extensions to our calculus or, more
importantly, to its target---the polymorphic blame calculus. While we believe
that such extensions are possible, they would likely require non-trivial
extensions to the polymorphic blame calculus. Thus the alternative of being able
to model algebraic datatypes without extending \pbc is appealing. The encoding
also paves the way to provide built-in support for algebraic datatypes in the
source language, while elaborating them via the encoding into \pbc.

\paragraph{Church and Scott Encodings}
It is well-known
that polymorphic calculi such as System F can encode datatypes via
Church encodings. However these encodings have well-known drawbacks. 
In particular, some operations are hard to define, and they can have a time
complexity that is greater than that of the corresponding functions for built-in
algebraic datatypes. A good example is the definition of
the predecessor function for Church numerals~\citep{church1941calculi}. Its
definition requires significant ingenuity (while it is trivial with 
built-in algebraic datatypes), and it has \emph{linear} time
complexity (versus the \emph{constant} time complexity for a definition 
using built-in algebraic datatypes). 

An alternative to Church encodings are the so-called Scott
encodings~\citep{curry1958combinatory}. They address the two drawbacks of Church
encodings: they allow simple definitions that directly correspond to programs
implemented with built-in algebraic datatypes, and those definitions have the same time
complexity to programs using algebraic datatypes.

Unfortunately, Scott encodings, or more precisely, their typed
variant~\citep{parigot1992recursive}, cannot be expressed in System F: in the
general case they require recursive types, which System F does not support.
However, with gradual typing, we can remove the need for recursive types, thus
enabling Scott encodings in our calculus.

\paragraph{A Scott Encoding of Parametric Lists}
Consider for instance the typed
Scott encoding of parametric lists in a system with implicit polymorphism:
\begin{align*}
   [[ List a ]] &\triangleq [[  mu L . \/b. b -> (a -> L -> b) -> b       ]] \\
   [[nil]] &\triangleq [[  fold [ List a ] (\n . \c . n ): \/ a . List a    ]] \\
   [[cons]] & \triangleq [[ \x . \xs . fold [List a]  (\n . \c. c x xs) :  \/a . a -> List a -> List a  ]]
\end{align*}
This encoding requires both polymorphic and recursive types\footnote{Here we use
iso-recursive types, but equi-recursive types can be used too.}. 
Like System F, our calculus 
only supports the former, but not the latter. Nevertheless, gradual types still
allow us to use the Scott encoding in a partially typed fashion.
The trick is to omit the recursive type binder $\mu L$ and replace the recursive
occurrence of $L$ by the unknown type $\unknown$:
\begin{align*}
   [[ Listu a  ]] &\triangleq [[\/ b. b -> (a -> unknown -> b) -> b]]
\end{align*}
As a consequence, we need to replace the term-level witnesses of the
iso-recursion by explicit type annotations to respectively forget or recover the type structure of
the recursive occurrences:
\begin{align*}
  [[ fold [Listu a] ]] & \triangleq [[\x . x : (\/b . b -> (a -> Listu a -> b) -> b) -> Listu a  ]] \\
  [[ unfold [Listu a] ]] & \triangleq [[ \x . x : Listu a -> (\/b . b -> (a -> Listu a -> b) -> b)     ]]
\end{align*}
With the reinterpretation of $[[fold]]$ and $[[unfold]]$ as functions instead of
built-in primitives, we have exactly the same definitions of $[[nilu]]$ and
$[[consu]]$.

Note that when we elaborate our calculus into the polymorphic blame calculus, the above
type annotations give rise to explicit casts. For
instance, after elaboration $[[ fold [Listu a] e   ]]$ results in the cast 
$ [[< (\/b . b -> (a -> Listu a -> b) -> b) `-> Listu a > pe]] $ where $[[pe]]$ is the elaboration of $[[e]]$.

In order to perform recursive traversals on lists, e.g., to compute their
length, we need a fixpoint combinator like the Y combinator. Unfortunately, this combinator
cannot be assigned a type in the simply typed lambda calculus or System F.
Yet, we can still provide a gradual typing for it in our system.
\begin{align*}
[[fix]] &\triangleq [[  \ f . (\x : unknown . f (x x)) (\x : unknown . f (x x)) : \/ a. (a -> a) -> a ]]
\end{align*}
This allows us for instance to compute the length of a list.
\begin{align*}
\mathsf{length} &\triangleq [[  fix ( \len . \l . zerou (\xs . succu (len xs)))  ]]
\end{align*}
Here $[[ zerou : natu  ]]$ and $[[ succu : natu -> natu    ]]$
are the encodings of the constructors for natural numbers $[[ natu
]]$. In practice, 
for performance reasons, we could extend our
language with a $\mathbf{letrec}$ construct in a standard way to
support general recursion, instead of defining a fixpoint combinator.

% length :: forall a. List a -> Nat
% length = fix (\len -> \l -> l zero (\xs -> succ (len xs)))

Observe that the gradual typing of lists still enforces that all
elements in the list are
of the same type. For instance, a heterogeneous list like
$[[  consu zerou (consu trueu nilu)    ]]$,
is rejected because $[[ zerou : natu    ]]$ and $[[ trueu : boolu  ]]$ have different types.

\paragraph{Heterogeneous Containers}
Heterogeneous containers are datatypes that can store data of different types,
which is very useful in various scenarios. One typical application is that an
XML element is heterogeneously typed. Moreover, the result of a SQL query
contains heterogeneous rows.

In statically typed languages, there are several ways to obtain heterogeneous lists. For example, in Haskell, one option is
to use \emph{dynamic types}. Haskell's library \textbf{Data.Dynamic} provides the
special type \textbf{Dynamic} along with its injection \textbf{toDyn} and
projection \textbf{fromDyn}. The drawback is that the code is littered with
\textbf{toDyn} and \textbf{fromDyn}, which obscures the program logic.
One can also use the \textsc{HList} library
\citep{kiselyov2004strongly}, which provides strongly typed data structures for
heterogeneous collections. The library requires several Haskell extensions, such as multi-parameter classes~\citep{jones1997type} and
functional dependencies~\citep{jones2000type}.
With fake dependent
types~\citep{mcbride2002faking}, heterogeneous vectors are also possible with
type-level constructors.

In our type system, with explicit type annotations that set the element types to
the unknown type we can disable the homogeneous typing discipline for the
elements and get gradually typed heterogeneous lists\footnote{This argument is
  based on the extended type system in \cref{sec:advanced-extension}.}. Such
gradually typed heterogeneous lists are akin to Haskell's approach with Dynamic
types, but much more convenient to use since no injections and projections are
needed, and the $[[unknown]]$ type is built-in and natural to use.

An example of such gradually typed heterogeneous collections is:
\[
l \triangleq [[consu (zerou : unknown) (consu (trueu : unknown) nilu)]]
\]
Here we annotate each element with type annotation $\unknown$ and the type
system is happy to type-check that $[[ l : Listu unknown ]]$.
Note that we are being meticulous about the syntax, but with proper
implementation of the source language, we could write more succinct programs
akin to Haskell's syntax, such as \lstinline{[0, True]}.

% \begin{itemize}
% \item Scott encodings of simple first-order ADTs (e.g. naturals)
% \item Parigot encodings improves Scott encodings with recursive schemes, but
%   occupies exponential space, whereas Church encoding only occupies linear
%   space.
% \item An alternative encoding which retains constant-time destructors but also
%   occupies linear space.
% \item Parametric ADTs also possible?
% \item Typing rules
% \end{itemize}
% 
% \begin{example}[Scott Encoding of Naturals]
% \begin{align*}
%   [[nat]] &\triangleq [[  \/a. a -> (unknown -> a) -> a ]] \\
%   \mathsf{zero} &\triangleq [[ \x . \f . x  ]] \\
%   \mathsf{succ} &\triangleq [[ \y : nat . \x . \f . f y ]]
% \end{align*}
% \end{example}
% Scott encodings give constant-time destructors (e.g., predecessor), but one has to
% get recursion somewhere. Since our calculus admits untyped lambda calculus, we
% could use a fixed point combinator.
% 
% \begin{example}[Parigot Encoding of Naturals]
% \begin{align*}
%   [[nat]] &\triangleq [[  \/a. a -> (unknown -> a -> a) -> a ]] \\
%   \mathsf{zero} &\triangleq [[ \x . \f . x  ]] \\
%   \mathsf{succ} &\triangleq [[ \y : nat . \x . \f . f y (y x f) ]]
% \end{align*}
% \end{example}
% Parigot encodings give primitive recursion, apart form constant-time
% destructors, but at the cost of exponential space complexity (notice in
% $\mathsf{succ}$ there are two occurances of $[[y]]$).
% 
% Both Scott and Parigot encodings are typable in System F with positive recursive
% types, which is strong normalizing.
% 
% \begin{example}[Alternative Encoding of Naturals]
% \begin{align*}
%   [[nat]] &\triangleq [[  \/a. a -> (unknown -> (unknown -> a) -> a) -> a ]] \\
%   \mathsf{zero} &\triangleq [[ \x . \f . x  ]] \\
%   \mathsf{succ} &\triangleq [[ \y : nat . \x . \f .  f y (\g . g x f) ]]
% \end{align*}
% \end{example}
% This encoding enjoys constant-time destructors, linear space complexity, and
% primitive recursion.
% The static version is $[[ mu b . \/ a . a -> (b -> (b -> a) -> a) -> a ]]$,
% which can only be expressed in System F with
% general recursive types (notice the second $[[b]]$ appears in a negative position).

%%% Local Variables:
%%% mode: latex
%%% TeX-master: "../paper"
%%% org-ref-default-bibliography: "../paper.bib"
%%% End:

\section{Revisiting Consistent Subtyping}
\label{sec:exploration}

In this section we explore the design space of consistent subtyping. In addition
to the unknown type $\unknown$, we have the same syntax of types as in
\Cref{fig:original-typing}. We start with the definitions of consistency and
subtyping for polymorphic types, and compare with some relevant work (in
particular the compatibility relation by \citet{ahmed2011blame} and type
consistency by \citet{yuu2017poly}). We then discuss the design decisions
involved towards our new definition of consistent subtyping, and justify the new
definition by demonstrating its equivalence with that of \citet{siek2007gradual}
and the AGT approach~\cite{garcia2016abstracting} on simple types.

% \subsection{Language Overview}

% \begin{figure}[t]
%   \centering
%   \begin{small}
% \begin{tabular}{lrcl}
%   Expressions & $e$ & \syndef & $x \mid n \mid
%                          \blam x A e \mid e~e$ \\
% %%                         \mid \erlam x e \equiv \blam x \unknown e $ \\

%   Types & $A, B$ & \syndef & $ \nat \mid a \mid A \to B \mid \forall a. A \mid \unknown$ \\
%   Monotypes & $\tau, \sigma$ & \syndef & $ \nat \mid a \mid \tau \to \sigma$ \\

%   Contexts & $\dctx$ & \syndef & $\ctxinit \mid \dctx,x: A \mid \dctx, a$ \\
%   Syntactic Sugar & $\erlam x e$ & $\equiv$ & $\blam x \unknown e$ \\
%                   & $e : A$ & $\equiv$ & $(\blam x A x) ~ e$
% \end{tabular}
%   \end{small}
% \caption{Syntax of the declarative type system}
% \label{fig:decl-syntax}
% \end{figure}
%%\bruno{Do not use the notation for sugar in the lambda expression: it
%%  is confusing. Explain it in text.}


%  The syntax of our language is given in \Cref{fig:decl-syntax}.
% Compared
% with the Odersky-L{\"a}ufer type system, the only addition is the unknown type
% $\unknown$. We use the meta-variable $e$ to range over expressions. There are
% variables $x$, integers $n$, annotated lambda abstraction $\blam x A e$, and
% application $e_1 ~ e_2$. We write $A$, $B$ for types. They are the integer type
% $\nat$, type variables $a$, functions $A \to B$, universal quantification
% $\forall a. A$, and the unknown type $\unknown$. Monotypes $\tau$ contain all
% types other than the universal quantifier and the unknown type. Contexts $\dctx$
% map term variables to their types, and record all type variables with the
% expected well-formdness condition. Following \citet{siek2006gradual}, if a
% lambda binder is without annotation, it is automatically annotated with
% $\unknown$. As a convenience, the language also provides type ascription $e :
% A$, which is simulated as $(\blam x A x) ~ e$.

\subsection{Consistency and Subtyping}
\label{subsec:consistency-subtyping}

We start by giving the definitions of consistency and subtyping for polymorphic
types, and showing the differences of our definitions from other works, in
particular the compatibility relation by \citet{ahmed2011blame} and type
consistency by \citet{yuu2017poly}.

\begin{figure}[t]
  \begin{small}
  \begin{mathpar}
    \framebox{$A \sim B$} \\
    \CD \and \CA \and \CB \and \CC \and \CE
  \end{mathpar}

  \begin{mathpar}
    \framebox{$\dctx \bywf A $} \\
    \DeclVarWF \and \DeclIntWF \and \DeclUnknownWF \\ \DeclFunWF \and \DeclForallWF
  \end{mathpar}

  \begin{mathpar}
    \framebox{$\tpresub A \tsub B$} \\
    \HSForallR \and \HSForallL \and \HSFun \and
    \HSTVar \and \HSInt \and \HSUnknown
  \end{mathpar}
  \end{small}
  \caption{Consistency, well-formedness of types and subtyping in the declarative system.}
  \label{fig:decl:subtyping}
\end{figure}

\paragraph{Consistency}
The key observation here is that consistency is mostly a structural relation,
except that the unknown type $\unknown$ can be regarded as any type. Following
this observation, we naturally extend the definition from
\Cref{fig:objects} with polymorphic types, as shown at the top of
\Cref{fig:decl:subtyping}. In particular a polymorphic type $\forall a. A$
is consistent with another polymorphic type $\forall a. B$ if $A \sim B$.

\paragraph{Subtyping}

We express the fact that one type is a polymorphic generalization of another by
means of the subtyping judgment $\Psi \vdash A \tsub B$. Compared with the
subtyping rules of \citet{odersky1996putting} in
\Cref{fig:original-typing}, the only addition is the neutral subtyping of
$\unknown$, given at the bottom of \Cref{fig:decl:subtyping}. Notice
that in the rule
\rul{S-ForallL}, the universal quantifier is only allowed to be instantiated
with a \emph{monotype}. According to the syntax in \Cref{fig:original-typing},
monotypes do not contain the unknown type $\unknown$. This is because if we were
to allow the unknown type to be used for instantiation, we could have the
following subtyping relation
\[
  \forall a . a \to a \tsub \unknown \to \unknown
\]
by instantiating $a$ with $\unknown$. Since $\unknown \to \unknown$ is
consistent with any functions $A \to B$, for instance, $\nat \to \bool$, this
means that we could provide an expression of type $\forall a. a \to a$ to a
function where the input type is supposed to be $\nat \to \bool$. However, as we
might expect, $\forall a. a \to a$ is definitely not compatible with $\nat \to
\bool$. This does not hold in any polymorphic type systems without gradual
typing. So the gradual type system should not accept it either. (This is the
so-called \textit{conservative extension} property that will be made precise in
\Cref{sec:criteria}.)

Importantly there is a subtle but crucial distinction between a type variable
and the unknown type, although they all represent a kind of ``arbitrary'' type.
The unknown type stands for the absence of type information: it could be
\textit{any type} at \textit{any instance}. Because of the absence of type
information, the unknown type is consistent with any type, and additional
type checks may have to be performed at runtime. On the other hand, a type
variable denotes some instantiation of a universal quantifier, and is subject to
global constraints. In other words, a type variable can only be instantiated to
a single type. For example, in the type $\forall a. a \to a$, the two
occurrences of $a$ represent an arbitrary but single type (e.g., $\nat \to
\nat$, $\bool \to \bool$), while $\unknown \to \unknown$ could be an arbitrary
function (e.g., $\nat \to \bool$) at runtime.

\paragraph{Comparison with Other Relations}

In other polymorphic gradual calculi, consistency and subtyping are often mixed
up to some extent. In the Polymorphic Blame Calculus
(\pbc)~\citep{ahmed2011blame}, the compatibility relation for polymorphic types
is defined as follows:
\begin{mathpar}
  \CompAllR \and \CompAllL
\end{mathpar}
Notice that, in rule \rul{Comp-AllL}, the universal quantifier is \textit{always}
instantiated to $\unknown$. However, in this way, \pbc allows $\forall a. a \to a
\pbccons \nat \to \bool$, which as we discussed before might not be what we
expect. Indeed \pbc relies on sophisticated runtime checks to rule out such
instances of the compatibility relation \`a posteriori.

\citet{yuu2017poly} introduced the so-called
\textit{quasi-polymorphic} types for types that may be used where a
$\forall$-type is expected, which is important for their purpose of
conservativity over System F. Their type consistency relation, involving polymorphism, is
defined as follows\footnote{This is a simplified version.}:
\begin{mathpar}
  \inferrule{A \sim B }{\forall a. A \sim \forall a. B}
  \and
  \inferrule{A \sim B \\ B \neq \forall a. B' \\ \unknown \in \mathsf{Types}(B)}
  {\forall a. A \sim B}
\end{mathpar}
Compared with our consistency definition in \Cref{fig:decl:subtyping},
their first rule is the same as ours. The second rule says that a non
$\forall$-type can be consistent with a $\forall$-type only if it contains
$\unknown$. In this way, their type system is able to reject $\forall a. a \to a
\sim \nat \to \bool$. However, in order to keep conservativity, they also reject
$\forall a. a \to a \sim \nat \to \nat$, which is perfectly sensible in their
setting (i.e., explicit polymorphism). However with implicit polymorphism, we
would expect $\forall a. a \to a$ to be related with $\nat \to \nat$, which is
exactly the case in our subtyping relation since $a$ can be instantiated to
$\nat$.

Nonetheless, when it comes to interactions between dynamically typed and
polymorphically typed terms, both relations allow for example $(\forall a. a) \to
\nat$ to be related with $\unknown \to \nat$, which in our view, is some sort of
(implicit) polymorphic subtyping, and that should be achievable by the more
primitive notions in the type system (instead of inventing new relations). One
of our design principles is that, subtyping and consistency should be
\textit{orthogonal}, and can be naturally superimposed, echoing the same opinion
by \citet{siek2007gradual}.

\subsection{Towards Consistent Subtyping}
\label{subsec:towards-conssub}

With the definitions of consistency and subtyping, the question now is how to
compose these two relations so that two types can be compared in a way that takes
these two relations into account.

Unfortunately, the original definition of \citet{siek2007gradual}
(\Cref{def:old-decl-conssub}) does not work well with our definitions of
consistency and subtyping for polymorphic types. Consider two types: $(\forall
a. a) \to \nat$, and $\unknown \to \nat$. The first type can only reach the
second type in one way (first by applying consistency, then subtyping), but not the
other way, as shown in \Cref{fig:example:a}. We use $\bot$ to mean that we
cannot find such a type. Similarly, there are situations where the first type
can only reach the second type using the other way (first applying
subtyping, and then
consistency), as shown in \Cref{fig:example:b}.

\begin{figure}
  \begin{subfigure}[b]{.4\linewidth}
    \centering
      \begin{tikzpicture}
        \matrix (m) [matrix of math nodes,row sep=3em,column sep=4em,minimum width=2em]
        {
          \bot & \unknown \to \nat \\
          (\forall a. a) \to \nat & (\forall a. \unknown) \to \nat \\};

        \path[-stealth]
        (m-2-1) edge node [left] {$\tsub$} (m-1-1)
        (m-2-2) edge node [left] {$\tsub$} (m-1-2);

        \draw
        (m-1-1) edge node [above] {$\sim$} (m-1-2)
        (m-2-1) edge node [below] {$\sim$} (m-2-2);
      \end{tikzpicture}
      \caption{}
      \label{fig:example:a}
  \end{subfigure}
  \begin{subfigure}[b]{.4\linewidth}
    \centering
    \begin{tikzpicture}
      \matrix (m) [matrix of math nodes,row sep=3em,column sep=4em,minimum width=2em]
      {
        \nat \to \nat & \nat \to \unknown \\
        \forall a. a & \bot \\};

      \path[-stealth]
      (m-2-1) edge node [left] {$\tsub$} (m-1-1)
      (m-2-2) edge node [left] {$\tsub$} (m-1-2);

      \draw
      (m-1-1) edge node [above] {$\sim$} (m-1-2)
      (m-2-1) edge node [below] {$\sim$} (m-2-2);
    \end{tikzpicture}
    \caption{}
    \label{fig:example:b}
  \end{subfigure}
  \begin{subfigure}[b]{.8\linewidth}
    \centering
    \begin{tikzpicture}
      \matrix (m) [matrix of math nodes,row sep=3em,column sep=4em,minimum width=2em]
      {
        \bot &
        ((\unknown \to \nat) \to \bool) \to (\nat \to \unknown)  \\
        (((\forall a. a) \to \nat) \to \bool) \to (\forall a. a) &
        \bot \\};

      \path[-stealth]
      (m-2-1) edge node [left] {$\tsub$} (m-1-1)
      (m-2-2) edge node [left] {$\tsub$} (m-1-2);

      \draw
      (m-1-1) edge node [above] {$\sim$} (m-1-2)
      (m-2-1) edge node [below] {$\sim$} (m-2-2);
    \end{tikzpicture}
    \caption{}
    \label{fig:example:c}
  \end{subfigure}
  \caption{Examples that break the original definition of consistent subtyping.}
  \label{fig:example}
\end{figure}

What is worse, if those two examples are composed in a way that those types all
appear co-variantly, then the resulting types cannot reach each other by either
way. For example, \Cref{fig:example:c} shows such two types by putting a
$\bool$ type in the middle, and neither definition of consistent subtyping
works. But such types ought to be related somehow!

\paragraph{Observations on consistent subtyping}

In order to develop the correct definition of consistent subtyping for
polymorphic types, we need to understand how consistent subtyping works.
We first review two important properties of subtyping: 1) subtyping induces the
subsumption rule: if $A \tsub B$, then an expression of type $A$ can be used
where $B$ is expected; 2) subtyping is transitive: if $A \tsub B$, and $B \tsub
C$, then $A \tsub C$. Though consistent subtyping takes the unknown type into
consideration, the subsumption rule should also apply: if $A \tconssub B$, then
an expression of type $A$ can also be used where $B$ is expected, given that
there might be some information lost by consistency. A crucial difference from
subtyping is that consistent subtyping is \textit{not} transitive because
information can only be lost once (otherwise, any two types are a consistent
subtype of each other). Now consider a situation where we have both $A \tsub B$,
and $B \tconssub C$, this means that $A$ can be used where $B$ is expected, and
$B$ can be used where $C$ is expected, with possibly some loss of information. In
other words, we should expect that $A$ can be used where $C$ is expected, since
there is at most one-time loss of information.

\begin{observation}
  If $A \tsub B$, and $B \tconssub C$, then $A \tconssub C$.
\end{observation}

This is reflected in \Cref{fig:obser:a}. A similar and symmetrical
observation is given in \Cref{fig:obser:b}:

\begin{observation}
  If $C \tconssub B$, and $B \tsub A$, then $C \tconssub A$.
\end{observation}

\begin{figure}[t]
  \centering
  \begin{subfigure}[b]{.4\linewidth}
    \centering
    \begin{tikzpicture}
      \matrix (m) [matrix of math nodes,row sep=3em,column sep=4em,minimum width=2em]
      {
        T_1 & C \\
        B   & T_2 \\
        A & \\};

      \path[-stealth]
      (m-3-1) edge node [left] {$\tsub$} (m-2-1)
      (m-2-2) edge node [left] {$\tsub$} (m-1-2)
      (m-2-1) edge node [left] {$\tsub$} (m-1-1);

      \draw
      (m-2-1) edge node [above] {$\sim$} (m-2-2)
      (m-1-1) edge node [above] {$\sim$} (m-1-2);

      \draw [dashed, ->]
      (m-2-1) edge node [above] {$\tconssub$} (m-1-2);

      \path [dashed, ->, out=0, in=0, looseness=2]
      (m-3-1) edge node [right] {$\tconssub$} (m-1-2);
    \end{tikzpicture}
    \caption{}
    \label{fig:obser:a}
  \end{subfigure}
  \centering
  \begin{subfigure}[b]{.4\linewidth}
    \centering
    \begin{tikzpicture}
      \matrix (m) [matrix of math nodes,row sep=3em,column sep=4em,minimum width=2em]
      {
        & A \\
        T_1 & B \\
        C   & T_2 \\};

      \path[-stealth]
      (m-3-1) edge node [left] {$\tsub$} (m-2-1)
      (m-3-2) edge node [left] {$\tsub$} (m-2-2)
      (m-2-2) edge node [left] {$\tsub$} (m-1-2);

      \draw
      (m-2-1) edge node [above] {$\sim$} (m-2-2)
      (m-3-1) edge node [below] {$\sim$} (m-3-2);

      \draw [dashed, ->]
      (m-3-1) edge node [above] {$\tconssub$} (m-2-2);

      \path [dashed, ->, out=135, in=180, looseness=2]
      (m-3-1) edge node [left] {$\tconssub$} (m-1-2);
    \end{tikzpicture}
    \caption{}
    \label{fig:obser:b}
  \end{subfigure}
  \caption{Observations of consistent subtyping}
  \label{fig:obser}
\end{figure}


From the above observations, we can see what the problem is with the original
definition. In \Cref{fig:obser:a}, if $B$ can reach $C$ by $T_1$, then
according to the transitivity of subtyping, $A$ can reach $C$ by $T_1$. However,
if $B$ can only reach $C$ by $T_2$, then $A$ cannot reach $C$ through the
original definition. A similar problem is shown in \Cref{fig:obser:b}: if $C$ can
only reach $B$ by $T_1$, then $C$ cannot reach $A$ through the original definition.

However, it turns out that those two problems can be fixed by the same strategy:
instead of taking one-step subtyping and one-step consistency, our definition of
consistent subtyping allows types to take one-step subtyping, one-step
consistency, and one more step subtyping. Specifically, $A \tsub B \sim T_2 \tsub C$
and $C \tsub T_1 \sim B \tsub A$ have the same relation chain: subtyping,
consistency, and subtyping.

\paragraph{Definition of consistent subtyping} From the above discussion, we are
ready to modify \Cref{def:old-decl-conssub}, and adapt it to our notation:

\begin{mdef}[Consistent Subtyping]
  \label{def:decl-conssub}
  $\tpresub A \tconssub B$, if and only if $\tpresub A \tsub C$, $C \sim D$, and
  $\tpresub D \tsub B$ for
  some $C, D$.
\end{mdef}

\noindent With \Cref{def:decl-conssub}, \Cref{fig:example:c:fix}
illustrates the correct relation chain for the broken example shown in
\Cref{fig:example:c}.

At first sight, \Cref{def:decl-conssub}
seems worse than the original: we need to guess \textit{two} types! It turns out
that \Cref{def:decl-conssub} is a generalization of
\Cref{def:old-decl-conssub}, and they are equivalent in the system by
\citet{siek2007gradual}. We argue that this is the \textit{general} definition of
consistent subtyping, independent of language features, and in particular is
compatible with polymorphic types.


\begin{figure}[t]
  \centering
  \begin{tikzpicture}
    \matrix (m) [matrix of math nodes,row sep=3em,column sep=4em,minimum width=2em]
    {
      ((\forall a. a) \to \nat) \to \bool) \to (\nat \to \nat) &
      ((\forall a. \unknown) \to \nat) \to \bool) \to (\nat \to \unknown)\\
      (((\forall a. a) \to \nat) \to \bool) \to (\forall a. a) &
      ((\unknown \to \nat) \to \bool) \to (\nat \to \unknown)  \\
      };

    \path[-stealth]
    (m-2-1) edge node [left] {$\tsub$} (m-1-1)
    (m-1-2) edge node [left] {$\tsub$} (m-2-2)
    (m-2-1) edge node [above] {$\tconssub$} (m-2-2);

    \draw
    (m-1-1) edge node [above] {$\sim$} (m-1-2);
  \end{tikzpicture}
  \caption{Example that is fixed by the new definition of consistent subtyping.}
  \label{fig:example:c:fix}
\end{figure}


\begin{mprop}\leavevmode
  \label{prop:subsumes}
\begin{itemize}
  \item \Cref{def:decl-conssub} subsumes
    \Cref{def:old-decl-conssub}:
    in \Cref{def:decl-conssub},
    by choosing $D=B$, we have $A\tsub C$ and $C \sim B$; by choosing $C=A$, we have
    $A \sim D$, and $D \tsub B$.
  \item \Cref{def:old-decl-conssub} is equivalent to
    \Cref{def:decl-conssub} in the system by~\citet{siek2007gradual}:
    if $A \tsub C$, $C \sim D$, and $D \tsub
    B$, by \Cref{def:old-decl-conssub},
    we have $A \sim C'$, $C' \tsub D$ for some $C'$. By subtyping
    transitivity, we have $C' \tsub B$. So we have $A \tconssub B$ by $A \sim C'$, and $C'
    \tsub B$.
  \end{itemize}
\end{mprop}

\subsection{Consistent Subtyping Without Existentials}

\Cref{def:decl-conssub} serves as a fine specification of how consistent
subtyping should behave in general. But it is inherently non-deterministic
because of the two intermediate types $C$ and $D$. As with
\citet{siek2007gradual}'s definition,
we need a combined relation to directly compare two types. A first,
and natural attempt is to try to extend the restriction operator for
polymorphic types. 
Unfortunately this does not work, but it is possible to devise a
simple and elegant inductive definition instead.

\paragraph{Attempt on extending the restriction operator}
Suppose we try to extend the restriction operator to account for polymorphic
types. The original restriction operator is structural, meaning that it works
for types of similar structures. But for polymorphic types, two input types
could have different structures due to universal quantifiers, e.g, $\forall a. a
\to \nat$ and $(\nat \to \unknown) \to \nat$. If we try to mask the first type
using the second, it seems hard to maintain the information that $a$ should be
instantiated to a function while ensuring that the return type is masked. There
seems to be no satisfactory way to extend the restriction operator in order to
support this kind of non-structural masking.

\paragraph{Interpretation of the restriction operator and consistent subtyping}
If the restriction operator cannot be extended naturally, it is useful to
take a step back and revisit what the restriction operator actually does. For
consistent subtyping, two input types could have unknown types in different
positions, but we only care about the known parts. To do that, the restriction
operator is used to: 1) erase the type information in one type if the corresponding
position in the other type is the unknown type; and 2) compare the resulting types 
using the normal subtyping relation. The example below shows the
masking-off procedure for the types $\nat \to \unknown \to \bool$ and $\nat \to
\nat \to \unknown$. Since the known parts have the relation that $\nat \to
\unknown \to \unknown \tsub \nat \to \unknown \to \unknown$, we conclude that
$\nat \to \unknown \to \bool \tconssub \nat \to \nat \to \unknown$.
\begin{center}
  \begin{tikzpicture}
    \tikzstyle{column 5}=[anchor=base west, nodes={font=\tiny}]
    \matrix (m) [matrix of math nodes,row sep=1em,column sep=0em,minimum width=2em]
    {
      \nat \to & \unknown & \to & \bool & \mid \nat \to \nat \to \unknown &
      = \nat \to \unknown \to \unknown
      \\
       \nat \to & \nat & \to & \unknown & \mid \nat \to \unknown \to \bool &
      = \nat \to \unknown \to \unknown \\};

    \path[-stealth, ->, out=0, in=0]
    (m-1-6) edge node [right] {$\tsub$} (m-2-6);

    \draw
    (m-1-2.north west) rectangle (m-2-2.south east)
    (m-1-4.north west) rectangle (m-2-4.south east);
  \end{tikzpicture}
\end{center}
Here differences of the types in boxes are erased because of the
restriction operator. Now if we compare the types in boxes directly instead of
through the lens of the restriction operator, we can observe that the
\textit{consistent subtyping relation always holds between the unknown type and
  an arbitrary type.} We can interpret this observation directly using
\Cref{def:decl-conssub}: the unknown type is neutral to subtyping
($\unknown \tsub \unknown$), the unknown type is consistent with any type
($\unknown \sim A$), and subtyping is reflexive ($A \tsub A$). Therefore,
\textit{the unknown type is a consistent subtype of any type ($\unknown
  \tconssub A$), and vice versa ($A \tconssub \unknown$).}

\paragraph{Defining consistent subtyping directly}

From the above discussion, we can define the consistent subtyping relation
directly, \textit{without} resorting to subtyping or consistency at all. The key
idea is that we replace $\tsub$ with $\tconssub$ in
\Cref{fig:decl:subtyping}, get rid of rule \rul{S-Unknown} and add two
extra rules concerning $\unknown$, resulting in the rules of consistent
subtyping in \Cref{fig:decl:conssub}. Of particular interest are the rules
\rul{CS-UnknownL} and \rul{CS-UnknownR}, both of which correspond to what we
just said: the unknown type is a consistent subtype of any type, and vice versa.
\begin{figure}[t]
  \begin{small}
  \begin{mathpar}
    \framebox{$\tpresub A \tconssub B$} \\
    \CSForallR \and \CSForallL \and \CSFun \and
    \CSTVar \and \CSInt \and \CSUnknownL \and \CSUnknownR
  \end{mathpar}
  \end{small}
  \caption{Consistent Subtyping for implicit polymorphism.}
  \label{fig:decl:conssub}
\end{figure}
From now on, we use the symbol $\tconssub$ to refer to the consistent subtyping
relation in \Cref{fig:decl:conssub}, instead of the one in
\Cref{def:decl-conssub}. What is more, we can prove that those two are
equivalent\footnote{Note to reviewers: Theorems with $\mathcal{T}$ are those
  proved in Coq. The same applies to $\mathcal{L}$emmas.}:

\begin{ctheorem} The following definitions are equivalent:
  \label{lemma:properties-conssub}
  \begin{itemize}
  \item  $\tpreconssub A \tconssub B$.
  \item  $\tpresub A \tsub C$, $C \sim D$, $\tpresub D \tsub B$, for some $C, D$.
  \end{itemize}
\end{ctheorem}

\noindent Not surprisingly, consistent subtyping is reflexive:

\begin{clemma}[Consistent Subtyping is Reflexive] \label{lemma:sub_refl}%
  If $\Psi \vdash A$ then $\Psi \vdash A \tconssub A$.
\end{clemma}

\subsection{Abstracting Gradual Typing}
\label{subsec:agt}

\citet{garcia2016abstracting} presented a new foundation for gradual typing that
they call the \textit{Abstracting Gradual Typing} (AGT) approach. In the AGT
approach, gradual types are interpreted as sets of static types, where static
types refer to types containing no unknown types. In this interpretation,
predicates and functions on static types can then be lifted to apply to gradual
types. Central to their approach is the so-called \textit{concretization}
function. For simple types, a concretization $\gamma$ from gradual types to a
set of static types\footnote{For simplification, we directly regard type
  constructor $\to$ as a set-level operator.} is defined as follows:
\begin{mdef}[Concretization]
  \label{def:concret}
  \begin{mathpar}
    \gamma(\nat) = \{\nat\} \and \gamma(A \to B) = \gamma(A) \to \gamma(B) \and
    \gamma(\unknown) = \{\text{All static types}\}
  \end{mathpar}
\end{mdef}

Based on the concretization function, subtyping between static types can be
lifted to gradual types, resulting in the consistent subtyping relation:
\begin{mdef}[Consistent Subtyping in AGT]
  \label{def:agt-conssub}
  $A \agtconssub B$ if and only if $A_1 \tsub B_1$ for some $A_1 \in \gamma(A)$, $B_1 \in \gamma(B)$.
\end{mdef}

\noindent Later they proved that this definition of consistent subtyping coincides with
that of \citet{siek2007gradual} (\Cref{def:old-decl-conssub}).

It seems that the AGT approach of consistent subtyping is quite different from
ours: theirs is defined purely in terms of static subtyping; we directly define
consistent subtyping on gradual types (\Cref{fig:decl:conssub}).
Nonetheless, the two approaches coincide on \textit{simple types}:

\begin{mprop}[Equivalence to AGT on Simple Types]
  \label{lemma:coincide-agt}
  $A \tconssub B$ if only if $A \agtconssub B$.
\end{mprop}
% \begin{proof}\leavevmode
%   \begin{itemize}
%   \item From left to right: By induction on the derivation of consistent
%     subtyping. In cases \rul{CS-UnknownL} and \rul{CS-UnknownR}, since the
%     static set of $\unknown$ contains all static types, it follows that for
%     every static type $A_1 \in \gamma(A)$, we can always find $A_1 \in
%     \gamma(\unknown)$, and by the reflexivity of subtyping, we are done. The
%     rest are trivial cases.
%   \item From right to left: By induction on the derivation of subtyping and
%     inversion on the concretization. If $A$ or $B$ is a unknown type, then
%     consistent subtyping directly holds. Other cases are trivial.
%   \end{itemize}
% \end{proof}

% Further more, the coincidence between our definition and AGT on consistent
% subtyping for objects
% can be proved by showing both are coincided with
% \citet{siek2007gradual}.

This proposition is rather trivial:
in \Cref{fig:decl:conssub},
rule \rul{CS-UnknownL} and \rul{CS-UnknownR} correspond directly to the
concretization of $\unknown$, which contains all static types. Nonetheless,
this reveals two points. First, it validates our definition of consistent
subtyping by another interpretation. Second, as noted by
\citet{garcia2016abstracting}, it shows that consistent subtyping can be derived
from two quite different foundations: one is defined directly on gradual types,
the other is defined purely in terms of static subtyping.

However, as noted by \citet{garcia2016abstracting} in the conclusion, extending
AGT to deal with polymorphism still remains as an open question. The difficulty
possibly stems from the fundamental conflicts between the set-theoretic
interpretation of gradual types and the parametric interpretation of
polymorphic types~\cite{reynolds1983types} (in particular, the interpretation of
type variables). This shows one advantage of our approach in that it is
independent of other concepts. Still, it is a promising line of future work for
AGT, and the question remains whether our definition would coincide with it.




%%% Local Variables:
%%% mode: latex
%%% TeX-master: "../paper"
%%% org-ref-default-bibliography: "../paper.bib"
%%% End:
\section{A Type System with Gradually Typed Implicit Polymorphism}
\label{sec:type-system}

In \Cref{sec:exploration} we have introduced the consistent
subtyping relation that naturally extends to polymorphic types. In
this section we continue with the development by giving a declarative
type system for implicit polymorphism that employs the consistent
subtyping relation. The declarative system itself is already quite
interesting as it is equipped with both higher-rank polymorphism and
the unknown type. Moreover, unlike non-gradual type systems with
higher-rank polymorphism, guessed types affect runtime behaviour if
used by the implicit casts, which raises concerns with respect to
coherency. Our response to those concerns is given in \Cref{subsec:algo:discuss},
after we give a simple
algorithm that implements the declarative system
(\Cref{sec:algorithm}) and discuss soundness and completeness.

% Later in \Cref{sec:algorithm} we give a simple
%algorithm that implements the declarative system.

\subsection{Language Overview}

\begin{figure}[t]
  \centering
  \begin{small}
\begin{tabular}{lrcl} \toprule
  Expressions & $e$ & \syndef & $x \mid n \mid
                         \blam x A e \mid e~e$ \\
%%                         \mid \erlam x e \equiv \blam x \unknown e $ \\

  Types & $A, B$ & \syndef & $ \nat \mid a \mid A \to B \mid \forall a. A \mid \unknown$ \\
  Monotypes & $\tau, \sigma$ & \syndef & $ \nat \mid a \mid \tau \to \sigma$ \\

  Contexts & $\dctx$ & \syndef & $\ctxinit \mid \dctx,x: A \mid \dctx, a$ \\
  Syntactic Sugar & $\erlam x e$ & $\equiv$ & $\blam x \unknown e$ \\
              & $e : A$ & $\equiv$ & $(\blam x A x) ~ e$ \\ \bottomrule
\end{tabular}
  \end{small}
\caption{Syntax of the declarative type system}
\label{fig:decl-syntax}
\end{figure}

The complete syntax of the declarative system is given in
\Cref{fig:decl-syntax}. We use the meta-variable $e$ to range over expressions.
Expressions are either variables $x$, integers $n$, annotated lambda
abstractions $\blam x A e$, or applications $e_1 ~ e_2$. We write $A$, $B$ for
types. Types are either the integer type $\nat$, type variables $a$, functions
types $A \to B$, universal quantification $\forall a. A$, or the unknown type
$\unknown$. Though we only have one base type $\nat$, we also use $\bool$ for
the purpose of illustration. Monotypes $\tau$ contain all types other than the
universal quantifier and the unknown type. Contexts $\dctx$ map term variables
to their types, and record all type variables with the expected well-formedness
condition. Following \citet{siek2006gradual}, if a lambda binder does not have
an annotation, it is automatically annotated with $\unknown$. As a convenience,
the language also provides type ascription $e : A$, which is simulated by
$(\blam x A x) ~ e$.

\subsection{Typing in Detail}

\Cref{fig:decl-typing} gives the typing rules for our declarative system
(the reader is advised to ignore the gray-shaded parts for now). Rule \rul{Var}
extracts the type of the variable from the typing context. Rule \rul{Nat} always
infers integer types. Rule \rul{LamAnn} puts $x$ with type annotation $A$ into
the context, and continues type checking the body $e$. Rule \rul{App} first
infers the type of $e_1$, then the matching judgment $\tprematch A \match A_1
\to A_2$ extracts the domain type $A_1$ and the codomain type $A_2$ from type
$A$. The type $A_3$ of the argument $e_2$ is then compared with $A_1$ using the
consistent subtyping judgment.

\renewcommand{\trto}[1]{\hlmath{\rightsquigarrow{#1}}}
\begin{figure}[t]
  \begin{small}
  \begin{mathpar}
    \framebox{$\tpreinf e : A \trto s$} \\
    \DVar \and \DNat \and \DLamAnnA \and \DApp
  \end{mathpar}

  \begin{mathpar}
    \framebox{$\tprematch A \match A_1 \to A_2$} \\
    \MMC \\ \MMA \and \MMB
  \end{mathpar}

  \end{small}
  \caption{Declarative typing}
  \label{fig:decl-typing}
\end{figure}

\paragraph{Matching} It turns out that matching~\cite{siek2015refined} can be
extended to polymorphic types naturally. In \rul{M-Forall}, a monotype $\tau$ is
guessed to instantiate the universal quantifier $a$. This natural extension is
also inspired by the \textit{application judgment} $\tpreinf A \bullet e \infto
C$ by \citet{dunfield2013complete}, which says that if we apply a term of type
$A$ to an argument $e$, we get something of type $C$. If $A$ is a polymorphic
type, the judgment works by guessing instantiations of polymorphic quantifiers
until it reaches an arrow type. Rule \rul{M-Arr} and \rul{M-Unknown} are the
same as \citet{siek2015refined}.


\renewcommand{\trto}[1]{\rightsquigarrow{#1}}
\subsection{Type-directed Translation}
\label{sec:type:trans}

We give the dynamic semantics of our language by translating it to
\pbc~\cite{ahmed2011blame}. Below we show a subset of the terms in \pbc that are
used in the translation:
\[
  \text{Terms}\quad s ::= x \mid n \mid \blam x A s \mid s~s \mid \cast A B s
\]
A cast $\cast A B {s}$ converts the value of term $s$ from type $A$ to type $B$.
A cast from $A$ to $B$ is permitted only if the types are \textit{compatible},
written $A \pbccons B$, as briefly mentioned in
\Cref{subsec:consistency-subtyping}. The syntax of types in \pbc is the
same as ours.

The translation is given in the gray-shaded parts in \Cref{fig:decl-typing}. The
only interesting case here is to insert explicit casts in the application rule.
Note that there is no need to translate matching or consistent subtyping,
instead we insert the source and target types of a cast directly in the
translated expressions, thanks to the following two lemmas:

\begin{clemma}[Compatibility of Matching]
  \label{lemma:comp-match}
  If $\tprematch A \match A_1 \to A_2$, then $A \pbccons A_1 \to A_2$.
\end{clemma}

\begin{clemma}[Compatibility of Consistent Subtyping]
  \label{lemma:comp-conssub}
  If $\tpreconssub A \tconssub B$, then $A \pbccons B$.
\end{clemma}

In order to show the correctness of the translation, we prove that our
translation always produces well-typed expressions in \pbc. By
\Cref{lemma:comp-match,lemma:comp-conssub}, we have the following theorem:

\begin{ctheorem}[Type Safety]
  \label{lemma:type-safety}
  If $\tpreinf e : A \trto s$, then $\dctx \bypinf s : A$.
\end{ctheorem}

\paragraph{Parametricity} An important semantic property of polymorphic types is
\textit{relational parametricity}~\cite{reynolds1983types}. The parametricity
property says that all instances of a parametrically polymorphic function should
behave \textit{uniformly}. In other words, functions cannot inspect into a type
variable, and act differently for different instances of the type variable. A
classic example is a function with the type $\forall a . a \to a$. The
parametricity property guarantees that a value of this type must be either the
identity function (i.e., $\lambda x . x$) or the undefined function (one which
never returns a value). However, with the addition of the unknown type
$\unknown$, careful measures are to be taken to ensure parametricity. This is
exactly the circumstance that \pbc was designed to address. \citet{amal2017blame}
proved that \pbc satisfies relational parametricity. Based on their result, and
by \Cref{lemma:type-safety}, parametricity is preserved in our system.

\paragraph{Guessed types affect runtime behaviour}

However, the translation does not always produce a unique target expression.
This is because when we guess a monotype $\tau$ in rule \rul{M-Forall} and
\rul{CS-ForallL}, we could have different choices, which inevitably leads to
different types. Unlike (non-gradual) polymorphic type systems
\citep{jones2007practical, dunfield2013complete}, the guessed types affect
runtime behaviour of the translated programs, since they could appear inside the
explicit casts. For example, the following shows two possible translations for
the same source expression $\blam x \unknown {f ~ x}$, where $f$ is
instantiated to $\nat \to \nat$ and $\bool \to \bool$, respectively:
\begin{align*}
  f: \forall a. a \to a &\byinf (\blam x \unknown {f ~ x})
                          : \unknown \to \nat \\
                          &\trto (\blam x \unknown (\cast {\forall a. a \to a} {\nat \to \nat} f) ~
                          (\hlmath{\cast \unknown \nat} x))
  \\
  f: \forall a. a \to a &\byinf (\blam x \unknown {f ~ x})
                          : \unknown \to \bool \\
                          &\trto (\blam x \unknown (\cast {\forall a. a \to a} {\bool \to \bool} f) ~
                          (\hlmath{\cast \unknown \bool} x))
\end{align*}
If we apply $\blam x \unknown {f ~ x}$ to $3$ for example, which should be fine
since the function can take any input, the first translation runs smoothly in
\pbc, while the second one will raise a cast error ($\nat$ cannot be cast to
$\bool$). Similarly, if we apply it to $\truee$, then the second succeeds while
the first fails. The culprit lies in the highlighted parts where any
instantiation of $a$ would be put inside the explicit cast. More generally, any
choice introduces an explicit cast to that type in the translation, which causes
a runtime cast error if the function is applied to a value whose type does not
match the guessed type. Note that this does not compromise the type safety of
the translated expressions, since cast errors are part of the type safety
guarantees.

\paragraph{Coherency}

The ambiguity of translation seems to imply that the
declarative is \textit{incoherent}. Coherence is a desired
property for a semantics. A semantic is coherent if any \textit{valid program}
has exactly one meaning~\cite{Reynolds_coherence}. We argue that the declarative
system is still coherent in the sense that if a program produces a value, this
value is unique. In the above example, whatever the translation might be,
applying $\blam x \unknown {f ~ x}$ to $3$ either results in a cast error, or
produces $3$, and not any other values.

This discrepancy is due to the guessing nature of the \textit{declarative}
system. As far as the declarative system is concerned, both $\nat \to \nat$ and
$\bool \to \bool$ are equally acceptable. But this is not the case at runtime.
The acute reader may have found that the \textit{only} appropriate choice is to
instantiate $f$ to $\unknown \to \unknown$. However, as specified by rule
\rul{M-Forall} in \Cref{fig:decl-typing}, we can only instantiate type variables
to monotypes, but $\unknown$ is \textit{not} a monotype! We will get back to
this issue in \Cref{subsec:algo:discuss} after we present the corresponding
algorithmic system in \Cref{sec:algorithm}.


\subsection{Correctness Criteria}
\label{sec:criteria}

\citet{siek2015refined} present a set of properties that a well-designed gradual
typing calculus must have, which they call refined criteria. Among all the
criteria, those related to the static aspects of gradual typing are well
summarized by \citet{cimini2016gradualizer}. Here we review those criteria and
adapt them to our notation. We have proved in Coq that our type system satisfies
all of these criteria.

\begin{clemma}[Correctness Criteria]\leavevmode
  \begin{itemize}
  \item \textbf{Conservative extension:}
    for all static $\dctx$, $e$, and $A$,
    $\dctx \byhinf e : A $ if and only if $\dctx \byinf e : A$.
  \item \textbf{Monotonicity w.r.t. precision:}
    for all $\dctx, e, e', A$,
    if $\dctx \byinf e : A$,
    and $e' \lessp e$,
    then $\dctx \byinf e' : B$,
    and $B \lessp A$ for some B.
  \item \textbf{Type Preservation of cast insertion:}
    for all $\dctx, e, A$,
    if $\dctx \byinf e : A$,
    then $\dctx \byinf e : A \trto s$,
    and $\dctx \bypinf s : A$ for some $s$.
  \item \textbf{Monotonicity of cast insertion:}
    for all $\dctx, e_1, e_2, e_1', e_2', A$,
    if $\dctx \byinf e_1 : A \trto e_1'$,
    and $\dctx \byinf e_2 : A \trto e_2'$,
    and $e_1 \lessp e_2$,
    then $\dctx \ctxsplit \dctx \bylessp e_1' \lesspp e_2'$.
  \end{itemize}
\end{clemma}

\begin{figure}[t]
  \begin{small}
  \begin{mathpar}
    \framebox{$A \lessp B$}{\quad \text{Type precision}} \\
    \LUnknown \and \LNat \and \LArrow \and \LTVar
    \and \LForall
  \end{mathpar}

  \begin{mathpar}
    \framebox{$e_1 \lessp e_2$}{\quad \text{Term precision}} \\
    \LRefl \and \LAbsAnn \and \LApp
  \end{mathpar}

  \begin{mathpar}
    \framebox{$\dctx_1 \ctxsplit \dctx_2 \bylessp e_1 \lesspp e_2$}
    {\quad \text{Term less precision in \pbc}} \\
    \LVar \and \LNatP \and \LAbsAnnP \and
    \LAppP \and \LCast \and \LCastL \and
    \LCastR
  \end{mathpar}
  \end{small}
  \caption{Less Precision}
  \label{fig:lessp}
\end{figure}


The first criterion states that the gradual type system should be a conservative
extension of the original system (i.e., the Odersky-L{\"a}ufer type system in
our case). In other words, a \textit{static} program that is typeable in the
original type system should remain typeable in the gradual type
system. A static program is one that does not contain any type $\unknown$. It also
ensures that ill-typed programs of the original language remain so in the
gradual type system.

The second criterion states that if a typeable expression loses some type
information, it remains typeable. This criterion depends on the definition of
the precision relation, written $A \lessp B$, which is given in the top of
\Cref{fig:lessp}. The relation intuitively captures a notion of types containing
more or less unknown types ($\unknown$). The precision relation over types lifts
to programs, i.e., $e_1 \lessp e_2$ means that $e_1$ and $e_2$ are the same
program except that $e_2$ has more unknown types.

The first two criteria are fundamental to gradual typing. They explain for
example why these two programs $(\blam x \nat {x + 1})$ and $(\blam x \unknown
{x + 1})$ are typeable, as the former is typeable in the Odersky-L{\"a}ufer type
system and the latter is a less-precise version of it.

The last two criteria relate to the compilation to the cast calculus. The
third criterion is essentially the same as \Cref{lemma:type-safety}, given that
a target expression should always exist, which can be easily seen from
\Cref{fig:decl-typing}. The last criterion ensures that the translation
must be monotonic over the precision relation $\lessp$. (The definition of the
precision relation $\lesspp$ for \pbc is found in the bottom of
\Cref{fig:lessp}.)


%%% Local Variables:
%%% mode: latex
%%% TeX-master: "../paper"
%%% org-ref-default-bibliography: "../paper.bib"
%%% End:
\section{Algorithmic Type System}
\label{sec:algorithm}

\begin{figure}[t]
  \centering
  \begin{small}
\begin{tabular}{lrcl} \toprule
  Expressions & $e$ & \syndef & $x \mid n \mid
                         \blam x A e \mid \erlam x e \mid e~e \mid e : A $ \\
  Types & $A, B$ & \syndef & $ \nat \mid a \mid \genA \mid A \to B \mid \forall a. A \mid \unknown$ \\
  Monotypes & $\tau, \sigma$ & \syndef & $ \nat \mid a \mid \genA \mid \tau \to \sigma$ \\
  Contexts & $\Gamma, \Delta, \Theta$ & \syndef & $\ctxinit \mid \tctx,x: A \mid \tctx, a \mid \tctx, \genA \mid \tctx, \genA = \tau$ \\
  Complete Contexts & $\Omega$ & \syndef & $\ctxinit \mid \Omega,x: A \mid \Omega, a \mid \Omega, \genA = \tau$ \\ \bottomrule
\end{tabular}
  \end{small}
\caption{Syntax of the algorithmic system}
\label{fig:algo-syntax}
\end{figure}


% The declarative type system in \cref{sec:type-system} serves as a good
% specification for how typing should behave. It remains to see whether this
% specification delivers an algorithm. The main challenge lies in the rules \rul{CS-ForallL} in
% \cref{fig:decl:conssub} and rule \rul{M-Forall} in
% \cref{fig:decl-typing}, which both need to guess a monotype.

% \bruno{why are we not highlightinh the differences in gray anymore?}
In this section we give a bidirectional account of the algorithmic type system
that implements the declarative specification. The algorithm is largely inspired
by the algorithmic bidirectional system of \citet{dunfield2013complete}
(henceforth DK system). However our algorithmic system differs from theirs in
three aspects: 1) the addition of the unknown type $\unknown$; 2) the use of the
matching judgment; and 3) the approach of \textit{gradual inference only
  producing static types}~\citep{garcia2015principal}. We then prove that our
algorithm is both sound and complete with respect to the declarative type
system. Full proofs can be found in the appendix.

\paragraph{Algorithmic Contexts.}

The algorithmic context $\Gamma$ is an
\textit{ordered} list containing declarations of type variables $a$ and term
variables $x : A$. Unlike declarative contexts, algorithmic contexts also
contain declarations of existential type variables $\genA$, which can be either
unsolved (written $\genA$) or solved to some monotype (written $\genA = \tau$).
Complete contexts $\Omega$ are those that contain no unsolved existential type
variables. \Cref{fig:algo-syntax} shows the syntax of the algorithmic system.
Apart from expressions in the declarative system, we have annotated expressions
$e : A$.

% \paragraph{Notational convenience}
% Following \citet{dunfield2013complete}, we use contexts as substitutions on
% types. We write $\ctxsubst{\Gamma}{A}$ to mean $\Gamma$ applied as a
% substitution to type $A$. We also use a hole notation, which is useful when
% manipulating contexts by inserting and replacing declarations in the middle. The
% hole notation is used extensively in proving soundness and completeness. For
% example, $\Gamma[\Theta]$ means $\Gamma$ has the form $\Gamma_L, \Theta,
% \Gamma_R$; if we have $\Gamma[\genA] = (\Gamma_L, \genA, \Gamma_R)$, then
% $\Gamma[\genA = \tau] = (\Gamma_L, \genA = \tau, \Gamma_R)$.

% \paragraph{Input and output contexts}
% The algorithmic system, compared with the declarative system, includes similar
% judgment forms, except that we replace the declarative context $\Psi$ with an
% algorithmic context $\Gamma$ (the \textit{input context}), and add an
% \textit{output context} $\Delta$ after a backward turnstile. For example,
% $\Gamma \vdash A \tconssub B \dashv \Delta$ is the judgment form for the
% algorithmic consistent subtyping, and so on. All rules manipulate input and
% output contexts in a way that is consistent with the notion of \textit{context
%   extension}, which is described in \cref{sec:ctxt:extension}.

% We start with the explanation of the algorithmic consistent subtyping as it
% involves manipulating existential type variables explicitly (and solving them if
% possible).

\subsection{Algorithmic Consistent Subtyping and Instantiation}
\label{sec:algo:subtype}

\begin{figure}[t]
  \centering
  \begin{small}
  %   \begin{mathpar}
  % \framebox{$\Gamma \vdash A$} \\
  % \VarWF \and \IntWF \and \UnknownWF \and \FunWF \and \ForallWF \and \EVarWF
  % \and \SolvedEVarWF
  %   \end{mathpar}

\begin{mathpar}
  \framebox{$\Gamma \vdash A \tconssub B \toctxr$} \\
  \ACSTVar \and \ACSExVar \and \ACSInt \quad \ACSUnknownL \quad \ACSUnknownR \and
  \ACSFun \and \ACSForallR \and \ACSForallL \and \AInstantiateL \quad \AInstantiateR
\end{mathpar}
  \end{small}
  \caption{Algorithmic consistent subtyping}
  \label{fig:algo:subtype}
\end{figure}

\Cref{fig:algo:subtype} shows the algorithmic consistent subtyping rules.
The first five rules do not manipulate contexts. % Rules \rul{ACS-TVar} and
% \rul{ACS-Int} do not involve existential variables, so the output context
% remains unchanged. Rule \rul{ACS-ExVar} says that any unsolved existential
% variable is a consistent subtype of itself. The output is still the same as the
% input context as this gives no clue as to what is the solution of that
% existential variable.
% Rules \rul{ACS-UnknownL} and \rul{ACS-UnknownR} are the verbatim
% correspondences of rule \rul{CS-UnknownL} and \rul{CS-UnknownR}.
Rule \rul{ACS-Fun} is a natural extension of its declarative counterpart. The
output context of the first premise is used by the second premise, and the
output context of the second premise is the output context of the conclusion.
Note that we do not simply check $A_2 \tconssub B_2$, but apply $\Theta$
% (the input context of the second premise)
to both types (e.g., $\ctxsubst{\Theta}{A_2} $). This is
to maintain an important invariant that types
% : whenever we try to derive $\Gamma \vdash A \tconssub B \dashv \Delta$, the types $A$ and $B$
are fully applied
under input context $\Gamma$ (they contain no existential variables already solved in
$\Gamma$). The same invariant applies to every algorithmic judgment.
Rule \rul{ACS-ForallR} looks similar to its declarative counterpart, except that
we need to drop the trailing context $a, \Theta$ from the concluding output
context since they become out of scope.
% again, bears a similarity with the declarative
% version. Note that the output context of its premise allows additional elements
% to appear after the type variable $a$, in a trailing context $\Theta$. Since $a$
% becomes out of scope in the conclusion, we need to drop the trailing context
% $\Theta$ together with $a$ from the concluding output context, resulting in
% $\Delta$.
% The next rule is essential to eliminating the guessing work, thus appears
% significantly different from its declarative version. Instead of guessing a
% monotype $\tau$ out of thin air,
Rule \rul{ACS-ForallL} generates a fresh
existential variable $\genA$, and replaces $a$ with $\genA$ in the body $A$. The
new existential variable $\genA$ is then added to the premise's input context.
% Unlike rule \rul{ACS-ForallR}, the output context $\Delta$ of the premise
% remains unchanged in the conclusion.
% A central idea behind this rule is that we
% defer the decision of choosing a monotype for a type variable, and hope that it
% could be solved later when we have more information at hand.
As a side note, when both types are quantifiers, then either \rul{ACS-ForallR}
or \rul{ACS-ForallR} could be tried. In practice, one can apply
\rul{ACS-ForallR} eagerly.
The last two rules % are specific to the algorithm, thus having no counterparts in
% the declarative version. They
together check consistent subtyping with an
unsolved existential variable on one side and an arbitrary type on the other
side by the help of the instantiation judgment. % Apart from checking that the existential variable does not occur in the
% type $A$, both of the rules do not directly solve the existential variables, but
% leave the real work to the instantiation judgment.

% \subsection{Instantiation}
% \label{sec:algo:instantiate}

\begin{figure}[t]
  \centering
  \begin{small}
\begin{mathpar}
  \framebox{$\tctx \vdash \genA \unif A \toctxr$} \\
  % {\quad \text{Under input context $\Gamma$, instantiate $\genA$ such that
  %     $\genA \tconssub A$, with output context $\Delta$ }} \\
  \InstLSolve \and \InstLReach \and \InstLSolveU   \and \InstLAllR \and \InstLArr
\end{mathpar}

% \begin{mathpar}
%   \framebox{$\tctx \vdash A \unif \genA  \toctxr$} \\
%   % {\quad \text{Under input context $\Gamma$, instantiate $\genA$ such that
%   %     $A \tconssub \genA$, with output context $\Delta$}} \\
%   \InstRSolve \and \InstRReach \and \InstRSolveU  \and \InstRAllL \and \InstRArr
% \end{mathpar}

  \end{small}
  \caption{Algorithmic instantiation}
  \label{fig:algo:instantiate}
\end{figure}

% A central idea of the algorithmic system is to defer the decision of picking a
% monotype to as late as possible.
The judgment $\Gamma \vdash \genA \unif A \dashv \Delta$ defined in
\cref{fig:algo:instantiate} instantiates unsolved existential variables.
Judgment $\genA \unif A$ reads ``instantiate $\genA$ to a consistent subtype of
$A$''. For space reasons, we omit its symmetric judgement $\Gamma \vdash A \unif
\genA \dashv \Delta$.
% Since these two are mutually defined, we
% discuss them together, and omit symmetric rules when convenient.
Rule \rul{InstLSolve} and rule \rul{InstLReach} set $\genA$ to
$\tau$ and $\genB$ in the output context, respectively.
% is the simplest
% one -- when an existential variable meets a monotype. In that case, we simply
% set the solution of $\genA$ to the monotype $\tau$ in the output context. We
% also need to check that the monotype $\tau$ is well-formed under the prefix
% context $\Gamma$.
Rule \rul{InstLSolveU} is similar to \rul{ACS-UnknownR} in that we put no
constraint on $\genA$ when it meets the unknown type $\unknown$. This design
decision reflects the point that type inference only produces static
types~\citep{garcia2015principal}. We will get back to this point in
\cref{subsec:algo:discuss}.
% Rule \rul{InstLReach} deals with the situation where two existential variables
% meet. Note that $\Gamma[\genA][\genB]$ denotes a context where some unsolved existential
% variable $\genA$ is declared before $\genB$. In this situation, the only logical
% thing we can do is to set the solution of one existential variable to the other
% one, depending on which is declared before which. For example, in the output
% context of rule \rul{InstLReach}, we have $\genB = \genA$ because in the input
% context, $\genA$ is declared before $\genB$.
Rule \rul{InstLAllR} is the instantiation version of rule \rul{ACS-ForallR}.
% Since our system is predicative, $\genA$ cannot be instantiated to $\forall b.
% B$, but we can decompose $\forall b. B$ in the same way as in \rul{ACS-ForallR}.
% Rule \rul{InstRAllL} is the instantiation version of rule \rul{ACS-ForallL}.
The last rule \rul{InstLArr} applies when $\genA$ meets a function type. It
follows that the solution must also be a function type.
% looks a bit complicated, but it is actually very
% intuitive: what does the solution of $\genA$ look like when $A$ is a function
% type? The solution must also be a function type!
That is why, in the first premise, we generate two fresh existential variables
$\genA_1$ and $\genA_2$, and insert them just before $\genA$ in the input
context, so that the solution of $\genA$ can mention them. Note that $A_1 \unif
\genA_1$ switches to the other instantiation judgment.


% \paragraph{Example}

% We show a derivation of $\Gamma[\genA] \vdash \forall b. b \to \unknown \unif
% \genA$ to demonstrate the interplay between instantiation, quantifiers and the
% unknown type:
% \[
%   \inferrule*[right=InstRAllL]
%       {
%         \inferrule*[right=InstRArr]
%         {
%           \inferrule*[right=InstLReach]{ }{\Gamma', \genB \vdash \genA_1 \unif \genB \dashv \Gamma' , \genB = \genA_1} \\
%           \inferrule*[right=InstRSolveU]{ }{\Gamma', \genB = \genA_1 \vdash \unknown \unif \genA_2 \dashv \Gamma', \genB = \genA_1}
%         }
%         {
%           \Gamma[\genA], \genB \vdash \genB \to \unknown \unif \genA \dashv \Gamma', \genB = \genA_1
%         }
%       }
%       {
%         \Gamma[\genA] \vdash \forall b. b \to \unknown \unif \genA \dashv \Gamma', \genB = \genA_1
%       }
% \]
% where $\Gamma' = \Gamma[\genA_2, \genA_1, \genA = \genA_1 \to \genA_2]$. Note
% that in the output context, $\genA$ is solved to $\genA_1 \to \genA_2$, and
% $\genA_2$ remains unsolved because the unknown type $\unknown$ puts no
% constraint on it. Essentially this means that the solution of $\genA$ can be any
% function, which is intuitively correct since $\forall b. b \to \unknown$ can be
% interpreted, from the parametricity point of view, as any function.

\subsection{Algorithmic Typing}
\label{sec:algo:typing}

\begin{figure}[t]
  \centering
  \begin{small}
\begin{mathpar}
  \framebox{$\Gamma \vdash e \Rightarrow A \toctxr $} \\
  % {\quad \text{Under input context $\Gamma$, $e$ synthesizes output type $A$,
  %     with output context $\Delta$}} \\
  \AVar \and \ANat \and \ALamU \and \ALamAnnA \and \AAnno \and \AApp
\end{mathpar}
\begin{mathpar}
  \framebox{$\Gamma \vdash e \Leftarrow A \toctxr $} \\
  % {\quad \text{Under input context $\Gamma$, $e$ synthesizes output type $A$,
  %     with output context $\Delta$}} \\
  \ALam \and \AGen \and \ASub
\end{mathpar}
\begin{mathpar}
  \framebox{$\Gamma \vdash A \match A_1 \to A_2 \toctxr$} \\
  % {\quad \text{Under input context $\Gamma$, $A$ synthesizes output type $A_1
  %     \to A_2$, with output context $\Delta$}} \\
  \AMMC \quad \AMMA \and \AMMB \and \AMMD
\end{mathpar}
  \end{small}
  \caption{Algorithmic typing}
  \label{fig:algo:typing}
\end{figure}

We now turn to the algorithmic typing rules in \cref{fig:algo:typing}. The
algorithmic system uses bidirectional type checking to accommodate polymorphism.
Most of them are quite standard.
% All of them are direct analogies of their declarative counterparts. Rules \rul{AVar}
% and \rul{ANat} do not generate any new information, thus the output context is
% the same as the input context. Rule \rul{ALamAnnA} infers the type of a lambda
% abstraction. It does so by pushing $x : A$ into the input context and continues
% to infer the type of the body $B$. The output context in the premise has
% additional declarations in the trailing context $\Theta$, which is discarded in
% the concluding output context.
Perhaps rule \rul{AApp} (which differs significantly from that in the DK system)
deserves attention. It relies on the algorithmic matching judgment $\Gamma
\vdash A \match A_1 \to A_2 \dashv \Delta$.
% The matching judgment
% algorithmically synthesizes a function type from an arbitrary type.
Rule
\rul{AM-ForallL} replaces $a$ with a fresh existential variable $\genA$, thus
eliminating guessing. Rule \rul{AM-Arr} and \rul{AM-Unknown} correspond
directly to the declarative rules.
% self-explanatory. Rule
% \rul{AM-Unknown} says that the unknown type $\unknown$ can be split into a
% function type $\unknown \to \unknown$.
Rule \rul{AM-Var}, which has no
corresponding declarative version, is similar to \rul{InstRArr}/\rul{InstLArr}:
we create $\genA$ and $\genB$ and add $\genC = \genA \to \genB$ to the context.

% Back to \rul{AApp}. This rule first infers the type of $e_1$, producing a output
% context $\Theta_1$. Then it applies $\Theta_1$ to $A$ and goes into the matching
% judgment, which delivers a function type $A_1 \to A_2$ and another output
% context $\Theta_2$. $\Theta_2$ is used as the input context when inferring the
% type of $e_2$. The last premise algorithmically checks if
% $\ctxsubst{\Theta_3}{A_3}$ is a consistent subtype of
% $\ctxsubst{\Theta_3}{A_1}$. $A_2$ and $\Delta$ are the concluding output type
% and the concluding output context, respectively.


% \section{Soundness and Completeness}
% \label{sec:sound:complete}

% To be confident that our algorithmic type system and the declarative type system
% accept exactly the same programs, we need to prove that the algorithmic rules
% are sound and complete with respect to the declarative specifications. Before we
% give the formal statements of the soundness and completeness theorems, we need a
% meta-theoretical device, called \textit{context extension}~\cite{dunfield2013complete}, to help capture a notion of
% information increase from input contexts to output contexts.

% \subsection{Context Extension}
% \label{sec:ctxt:extension}


% A context extension judgment $\Gamma \exto \Delta$ reads ``$\Gamma$ is extended
% by $\Delta$''. Intuitively, this judgment says that $\Delta$ has at least as
% much information as $\Gamma$: some unsolved existential variables in $\Gamma$
% may be solved in $\Delta$. (The full inductive definition can be found in the
% supplementary material. We refer the reader to \citet[][Section
% 4]{dunfield2013complete} for further explanations of context extension.)

\subsection{Completeness and Soundness}

We prove that the algorithmic rules are sound and complete with
respect to the declarative specifications. We need an auxiliary judgment
$\Gamma \exto \Delta$ that captures a notion of information increase from input
contexts $\Gamma$ to output contexts $\Delta$~\citep{dunfield2013complete}.

\paragraph{Soundness.} Roughly speaking, soundness of the algorithmic system says
that given an expression $e$ that type checks in the algorithmic system, there exists
a corresponding expression $e'$ that type checks in the declarative system.
However there is one complication: $e$ does not necessarily have more annotations
than $e'$. For example, by \rul{ALam} we have $\erlam{x}{x} \chkby (\forall a.
a) \rightarrow (\forall a . a)$, but $\erlam{x}{x}$ itself cannot have type
$(\forall a. a) \rightarrow (\forall a . a)$ in the declarative system. To
circumvent that, we add an annotation to the lambda abstraction, resulting in
$\blam{x}{(\forall a . a)}{x}$, which is typeable in the declarative system with
the same type. To relate $\erlam{x}{x}$ and $\blam{x}{(\forall a . a)}{x}$, we
erase all annotations on both expressions. The definition of erasure $\erase{\cdot}$ is
standard and thus omitted.

% \jeremy{mention erasure and why (talk about \rul{ALam} and \rul{ASub})}


% \begin{restatable}[Instantiation Soundness]{mtheorem}{instsoundness} \label{thm:inst_soundness}%
%   Given $\Delta \exto \Omega$ and $\ctxsubst{\Gamma}{A} = A$ and $\genA \notin \mathit{fv}(A)$:
%   \begin{itemize}
%   \item If $\Gamma \vdash \genA \unif A \dashv \Delta$ then $\ctxsubst{\Omega}{\Delta} \vdash \ctxsubst{\Omega}{\genA} \tconssub \ctxsubst{\Omega}{A}$.
%   \item If $\Gamma \vdash A \unif \genA \dashv \Delta$ then $\ctxsubst{\Omega}{\Delta} \vdash \ctxsubst{\Omega}{A} \tconssub \ctxsubst{\Omega}{\genA}$.
%   \end{itemize}
% \end{restatable}

% Notice that the declarative judgment uses $\ctxsubst{\Omega}{\Delta}$, a
% operation that applies a complete context $\Omega$ to the algorithmic context
% $\Delta$, essentially plugging in all known solutions and removing all
% declarations of existential variables (both solved and unsolved), resulting in a
% declarative context.

% With instantiation soundness, next we show that the algorithmic consistent
% subtyping is sound:

% \begin{restatable}[Soundness of Algorithmic Consistent Subtyping]{mtheorem}{subsoudness} \label{thm:sub_soundness}%
%   If $\Gamma \vdash A \tconssub B \toctxr$ where $\ctxsubst{\tctx}{A} = A$ and
%   $\ctxsubst{\tctx}{B} = B$ and $\ctxr \exto \cctx$ then
%   $\ctxsubst{\cctx}{\Delta} \vdash \ctxsubst{\cctx}{A} \tconssub
%   \ctxsubst{\cctx}{B}$.
% \end{restatable}

% At this point, we are ``two thirds of the way'' to proving the ultimate theorem.
% The remaining third concerns with the soundness of matching:

% \begin{restatable}[Matching Soundness]{mtheorem}{matchsoundness}  \label{thm:match_soundness}%
%   If $\Gamma \vdash A \match A_1 \to A_2 \dashv \Delta$ where
%   $\ctxsubst{\Gamma}{A} = A$ and $\Delta \exto \Omega$ then
%   $\ctxsubst{\Omega}{\Delta} \vdash \ctxsubst{\Omega}{A} \match
%   \ctxsubst{\Omega}{A_1} \to \ctxsubst{\Omega}{A_2}$.
% \end{restatable}


% Finally the soundness theorem of algorithmic typing is:

\begin{restatable}[Soundness of Algorithmic Typing]{mtheorem}{typingsoundness} \label{thm:type_sound}
  Given $\ctxr \exto \cctx$,

  \begin{enumerate}
  \item If $\Gamma \vdash e \infto A \toctxr$ then $\exists e'$ such
    that $\ctxsubst{\cctx}{\Delta} \vdash e' : \ctxsubst{\cctx}{A}$ and
    $\erase{e} = \erase{e'}$.
  \item If $\Gamma \vdash e \chkby A \toctxr$ then $\exists e'$ such
    that $\ctxsubst{\cctx}{\Delta} \vdash e' : \ctxsubst{\cctx}{A}$ and
    $\erase{e} = \erase{e'}$.
  \end{enumerate}


\end{restatable}


\paragraph{Completeness.}
Completeness of the algorithmic system is the reverse of soundness: given a
declarative judgment of the form $\ctxsubst{\Omega}{\Gamma} \vdash
\ctxsubst{\Omega} \dots $, we want to get an algorithmic derivation of $\Gamma
\vdash \dots \dashv \Delta$. It turns out that completeness is a bit trickier to
state in that the algorithmic rules generate existential variables on the fly,
so $\Delta$ could contain unsolved existential variables that are not found in
$\Gamma$, nor in $\Omega$. Therefore the completeness proof must produce another
complete context $\Omega'$ that extends both the output context $\Delta$, and
the given complete context $\Omega$. As with soundness, we need erasure to
relate both expressions.

% \jeremy{talk about \rul{Gen}}

% \begin{restatable}[Instantiation Completeness]{mtheorem}{instcomplete}  \label{thm:inst_complete}%
%   Given $\Gamma \exto \Omega$ and $A = \ctxsubst{\Gamma}{A}$ and $\genA \in
%   \mathit{unsolved}(\Gamma)$ and $\genA \notin \mathit{fv}(A)$:
%   \begin{enumerate}
%   \item If $\ctxsubst{\Omega}{\Gamma} \vdash \ctxsubst{\Omega}{\genA} \tconssub
%     \ctxsubst{\Omega}{A}$ then there exist $\Delta$, $\Omega'$ such that $\Omega \exto
%     \Omega'$ and $\Delta \exto \Omega'$ and $\Gamma \vdash \genA \unif A \dashv \Delta$.
%   \item If $\ctxsubst{\Omega}{\Gamma} \vdash \ctxsubst{\Omega}{A} \tconssub
%     \ctxsubst{\Omega}{\genA}$ then there exist $\Delta$, $\Omega'$ such that $\Omega \exto
%     \Omega'$ and $\Delta \exto \Omega'$ and $\Gamma \vdash A \unif \genA \dashv \Delta$.
%   \end{enumerate}
% \end{restatable}


% Next is the completeness of consistent subtyping:

% \begin{restatable}[Generalized Completeness of Subtyping]{mtheorem}{subcomplete}  \label{thm:sub_completeness}%
%   If $\Gamma \exto \Omega$ and $\Gamma \vdash A$ and $\Gamma \vdash B$ and
%   $\ctxsubst{\Omega}{\Gamma} \vdash \ctxsubst{\Omega}{A} \tconssub
%   \ctxsubst{\Omega}{B}$ then there exist $\Delta$, $\Omega'$ such that $\Delta
%   \exto \Omega'$ and $\Omega \exto \Omega'$ and $\Gamma \vdash
%   \ctxsubst{\Gamma}{A} \tconssub \ctxsubst{\Gamma}{B \dashv \Delta}$.
% \end{restatable}


% We prove that the algorithmic matching is complete with respect to the
% declarative matching:

% \begin{restatable}[Matching Completeness]{mtheorem}{matchcomplete} \label{thm:match_complete}%
%   Given $\Gamma \exto \Omega$ and $\Gamma \vdash A$, if
%   $\ctxsubst{\Omega}{\Gamma} \vdash \ctxsubst{\Omega}{A} \match A_1 \to A_2$
%   then there exist $\Delta$, $\Omega'$, $A_1'$ and $A_2'$ such that $\Gamma
%   \vdash \ctxsubst{\Gamma}{A} \match A_1' \to A_2' \dashv \Delta$ and $\Delta \exto \Omega'$ and
%   $\Omega \exto \Omega'$ and $A_1 = \ctxsubst{\Omega'}{A_1'}$ and $A_2 =
%   \ctxsubst{\Omega'}{A_2'}$.
% \end{restatable}


% Finally here is the completeness theorem of the algorithmic typing:

\begin{restatable}[Completeness of Algorithmic Typing]{mtheorem}{typingcomplete}  \label{thm:type_complete}
  Given $\Gamma \exto \Omega$ and $\Gamma \vdash A $, if
  $\ctxsubst{\Omega}{\Gamma} \vdash e : A$ then there exist $\Delta$,
  $\Omega'$, $A'$ and $e'$ such that $\Delta \exto \Omega'$ and $\Omega \exto \Omega'$
  and $\Gamma \vdash e' \infto A' \dashv \Delta$ and $A = \ctxsubst{\Omega'}{A'}$ and $\erase{e} = \erase{e'}$.
\end{restatable}





%%% Local Variables:
%%% mode: latex
%%% TeX-master: "../paper"
%%% org-ref-default-bibliography: "../paper.bib"
%%% End:


\section{Discussion}
\label{sec:discuss}

Discuss the limitations/issues of \name. The differences from the original trait model.

\begin{itemize}
\item Override: Form a new trait by layering additional methods over an existing
  trait. This operation is an asymmetric sum.
\item Exclusion: forms a new trait by removing a method from an existing trait
\end{itemize}


\begin{itemize}
\item Traits have no proper notation of inheritance relationship (no super keyword)
\item Traits have no nice syntax of redefining
\item traits allow annotation of type, then term declarations don't need to
\end{itemize}
\section{Restoring the Dynamic Gradual Guarantee with Type Parameters}
\label{sec:advanced-extension}

In \cref{sec:type:trans} we have seen an example where a single source expression could
produce two different target expressions with different runtime behaviors. As we
explained, this is due to the guessing nature of the declarative system, and,
from the (source) typing point of view, no guessed type is particularly better than 
any other. As a consequence, this breaks the dynamic gradual guarantee as discussed in \cref{sec:criteria}.

To alleviate this situation, we introduce \textit{static type parameters}, which
are placeholders for monotypes, and \textit{gradual type parameters}, which are
placeholders for monotypes that are consistent with the unknown type. The
concept of static type parameters and gradual type parameters in the context of
gradual typing was first introduced by \citet{garcia2015principal}, and later
played a central role in the work of \citet{yuu2017poly}. In our type system,
type parameters mainly help capture the notion of \textit{representative
  translations}, and should not appear in a source program.
% As far as we know,
% we are the first to employ type parameters in the (implicit) polymorphic
% setting.
With them we are able to recast the dynamic gradual guarantee in terms
of representative translations, and to prove that every well-typed source expression
possesses at least one representative translation. With a
coherence conjecture regarding representative translations, the dynamic gradual
guarantee of our extended source language now can be reduced to that of \pbc,
which, at the time of writing, is still an open question.


% \jeremy{emphasize type parameters are just analysis tool for the purpose of
%   discussing dynamic gradual guarantee, they don't actually appear in program text. }


% The crucial difference
% between them is that the former correspond to existential variables without any
% constraints, while the latter correspond to existential variables with the only
% constraint that they are compared with an unknown type. With static and gradual
% type parameters in place, we are able to reason about dynamic semantics in more
% depth.


\subsection{Declarative Type System}
\label{sec:type-param}

The new syntax of types is given at the top of \cref{fig:exd:type}, with the differences
highlighted. In addition to the types of \cref{fig:decl:subtyping}, we add \emph{static type parameters} $[[static]]$,
and \emph{gradual type parameters} $[[gradual]]$. Both kinds of type parameters are monotypes. The addition of type parameters,
however, leads to two new syntactic categories of types. \emph{Castable types} $[[gc]]$
represent types that can be cast from or to $[[unknown]]$. It includes all
types, except those that contain static type parameters. \emph{Castable monotypes}
$[[tc]]$ are those castable types that are also monotypes.

\begin{figure}[t]
  \centering
  \begin{small}
    \begin{tabular}{lrcl} \toprule
      Types & $[[A]], [[B]]$ & \syndef & $[[int]] \mid [[a]] \mid [[A -> B]] \mid [[\/ a. A]] \mid [[unknown]] \mid \hlmath{[[static]] \mid [[gradual]]} $ \\
      Monotypes & $[[t]], [[s]]$ & \syndef & $ [[int]] \mid [[a]] \mid [[t -> s]] \mid \hlmath{[[static]] \mid [[gradual]]}$ \\
      \hl{Castable Types} & $[[gc]]$ & \syndef & $ [[int]] \mid [[a]] \mid [[gc1 -> gc2]] \mid [[\/ a. gc]] \mid [[unknown]] \mid [[gradual]] $ \\
      \hl{Castable Monotypes} & $[[tc]]$ & \syndef & $ [[int]] \mid [[a]] \mid [[tc1 -> tc2]] \mid [[gradual]]$ \\

      \bottomrule
    \end{tabular}

    \begin{drulepar}[cs]{$ [[dd |- A <~ B ]] $}{Consistent Subtyping}{}
      \drule{tvar}
      \drule{int}
      \drule{arrow}
      \drule{forallR}
      \drule{forallL}
      \hlmath{\drule{unknownLL}} \and
      \hlmath{\drule{unknownRR}} \and
      \hlmath{\drule{spar}} \and
      \hlmath{\drule{gpar}}
    \end{drulepar}
  \end{small}
  \caption{Syntax of types, and consistent subtyping in the extended declarative
  system.}
  \label{fig:exd:type}
\end{figure}


\paragraph{Consistent Subtyping.}
The new definition of consistent subtyping is given at the bottom of
\cref{fig:exd:type}, again with the differences highlighted. Now the unknown type is only a consistent subtype of all
castable types, rather than of all types (\rref{cs-unknownLL}), and vice versa
(\rref{cs-unknownRR}). Moreover, the static type parameter $[[static]]$ is a consistent
subtype of itself (\rref{cs-spar}), and similarly for the gradual type parameter
(\rref{cs-gpar}). From this definition it follows immediately that 
$[[unknown]]$ is incomparable with types that contain static type parameters $[[static]]$,
such as $[[static -> int]]$.

\paragraph{Typing and Translation.}

Given these extensions to types and consistent subtyping, the typing process
remains the same as in \cref{fig:decl-typing}. To account for the
changes in the translation, if we extend \pbc with type parameters as in
\citet{garcia2015principal}, then the translation remains the same as well.

\subsection{Substitutions and Representative Translations}

As we mentioned, type parameters serve as placeholders for monotypes. As a
consequence, wherever a type parameter is used, any \emph{suitable} monotype
could appear just as well. To formalize this observation, we define substitutions for type
parameters as follows:

\begin{definition}[Substitution] Substitutions for type parameters are defined as:
  \begin{enumerate}
    \item Let $\ssubst : [[static]] \to [[t]]$ be a total function mapping static type
      parameters to monotypes. 
    \item Let $\gsubst : [[gradual]] \to [[tc]]$ be a total function mapping gradual type
      parameters to castable monotypes.
    \item Let $\psubst = \gsubst \cup \ssubst$ be a union of $\ssubst$ and $\gsubst$ mapping static and gradual
      type parameters accordingly.
  \end{enumerate}
\end{definition}
\noindent Note that since $[[gradual]]$ might be compared with $[[unknown]]$, only
castable monotypes are suitable substitutes, whereas $[[static]]$
can be replaced by any monotypes. Therefore, we can substitute $[[gradual]]$ for $[[static]]$,
but not the other way around.


Let us go back to our example and its two translations in \cref{sec:type:trans}. The
problem with those translations is that neither $[[int -> int]]$ nor $[[bool ->
int]]$ is general enough. With type parameters, however, we can state a more
\textit{general} translation that covers both through substitution:
\begin{align*}
  f: \forall a. a \to \nat &\byinf (\blam x \unknown {f ~ x})
                          : \unknown \to \nat \\
                          &\trto (\blam x \unknown (\cast {\forall a. a \to \nat} {\gradual \to \nat} f) ~
                          (\hlmath{\cast \unknown \gradual} x))
\end{align*}
The advantage of type parameters is that they help reasoning
about the dynamic semantics. Now we are not limited to a particular choice, such
as $[[int -> int]]$ or $[[bool -> int]]$, which might or might not emit a cast
error at runtime. Instead we have a general choice $[[gradual ->
int]]$. 

What does the more general choice with type parameters tell us? First, we know that in this case, there is no concrete
constraint on $[[a]]$, so we can instantiate it with a type parameter. Second,
the fact that the general choice uses $[[gradual]]$ rather than
$[[static]]$ indicates that any chosen instantiation needs to be a castable type.
It follows that any concrete instantiation will have an impact on the
runtime behavior; therefore it is best to instantiate $[[a]]$ with
$[[unknown]]$. However, type inference cannot instantiate $[[a]]$ with
$[[unknown]]$, and substitution cannot replace $[[gradual]]$ with $[[unknown]]$ either.
This means that we need a syntactic refinement process of the translated programs in
order to replace type parameters with allowed gradual types.

\paragraph{Syntactic Refinement.}

We define syntactic refinement of the translated expressions as follows. As
$[[static]]$ denotes no constraints at all, substituting it with any monotype
would work. Here we arbitrarily use $[[int]]$. We interpret
$[[gradual]]$ as $[[unknown]]$ since any monotype could possibly lead to a cast
error.

\begin{definition}[Syntactic Refinement] The syntactic refinement of a
  translated expression $[[pe]]$ is denoted by $\erasetp s$, and defined as follows:
  \begin{center}
\begin{tabular}{lllllll} \toprule
  $\erasetp{\nat}$ &$=$ & $ \nat$ &  &   $\erasetp{a} $ & $ = $ & $a $ \\
  $\erasetp{A \to B}$ &$=$ & $ \erasetp{A} \to \erasetp{B}$ &  &   $\erasetp {\forall a. A} $ & $ = $ & $ \forall a . \erasetp{A} $ \\
  $\erasetp{\unknown}$ &$=$ & $\unknown$ &  &   $\erasetp {\static} $ & $ = $ & $\nat$ \\
  $\erasetp{\gradual}$ &$=$ & $\unknown$ &  \\ \bottomrule
\end{tabular}

  \end{center}
\end{definition}
% \bruno{Can we align the ``='' and the types?}
\noindent Applying the syntactic refinement to the translated
expression, we get
  \[
    (\blam x \unknown (\cast {\forall a. a \to \nat} { \hlmath[blue!40]{\unknown} \to \nat} f) ~
    (\cast \unknown {\hlmath[blue!40]{\unknown}} x))
  \]
where two $[[gradual]]$ are refined by $[[unknown]]$ as highlighted.
It is easy to verify that both applying this expression to $3$ and to
$\mathit{true}$ now results in a translation that evaluates to
a value.

\paragraph{Representative Translations.}
To decide whether one translation is more general than the other, we define a preorder
between translations.

\begin{definition}[Translation Pre-order]
  Suppose $[[dd |- e : A ~~> pe1]]$ and $[[dd |- e : A ~~> pe2]]$,
  we define $[[pe1]] \leq [[pe2]]$ to mean $[[pe2]] \aeq [[S(pe1)]]$ for
  some $[[S]]$.
\end{definition}

\begin{restatable}[]{proposition}{propparalpha}
  \label{prop:parameter:alpha}
  If $[[ pe1 ]] \leq [[pe2]]$ and $[[ pe2 ]] \leq [[pe1]]$, then $[[pe1]]$ and
  $[[pe2]]$ are $\alpha$-equivalent (i.e., equivalent up to renaming of type parameters).
\end{restatable}

The preorder between translations gives rise to a notion of
what we call \textit{representative translations}:

\begin{definition}[Representative Translation]
  A translation $[[pe]]$ is said to be a representative translation of a typing
  derivation $[[dd |- e : A ~~> pe]]$ if and only if for any other translation
  $[[dd |- e : A ~~> pe']]$ such that $[[pe']] \leq [[pe]]$, we have $[[pe]]
  \leq [[pe']]$. From now on we use $[[rpe]]$ to denote a representative
  translation.
\end{definition}

An important property of representative translations, which we conjecture for
the lack of rigorous proof, is that if there exists any translation of an
expression that (after syntactic refinement) can reduce to a value, so can a
representative translation of that expression. Conversely, if a
representative translation runs into a blame, then no translation of that
expression can reduce to a value.

\begin{conjecture}[Property of Representative Translations]\label{lemma:repr}
  For any expression $[[e]]$ such that $[[ dd |- e : A ~~> pe ]]$ and $[[ dd |- e : A ~~> rpe ]]$ and
  $\forall [[CC]].\, [[CC : (dd |- A) ~~> (empty |- int) ]]   $, we have
  \begin{itemize}
  \item If $  [[CC]] \{  \erasetp{[[pe]]} \}  [[==>]] [[n]]$, then $ [[CC]] \{   \erasetp{[[rpe]]}   \} [[==>]] [[n]]$.
  \item If $[[CC]] \{ \erasetp {[[rpe]]}   \} [[==>]] [[blame]]$, then $ [[CC]] \{ \erasetp {[[pe]]} \}  [[==>]] [[blame]]$.
  \end{itemize}
\end{conjecture}

Given this conjecture, we can state a stricter coherence property (without the
``up to casts'' part) between any two representative translations. We first
strengthen \cref{conj:coher} following \citet{amal2017blame}:

\begin{definition}[Contextual Approximation \`a la \citet{amal2017blame}] \leavevmode
  \begin{center}
  \begin{tabular}{lll}
$[[dd]] \vdash \ctxappro{[[pe1]]}{[[pe2]]}{[[A]]}$ & $\defeq$ & $[[ dd |- pe1 : A  ]] \land [[dd |- pe2 : A ]] \ \land $ \\
                                                   & & for all $\mathcal{C}.\, [[ CC : (dd |- A) ~~> (empty |- int) ]] \Longrightarrow$ \\
                                                   & &  $\quad (\mathcal{C}\{ \erasetp{[[pe1]]} \}   \Downarrow [[n]] \Longrightarrow  \mathcal{C} \{ \erasetp{[[ pe2 ]]}  \}  \reduce [[n]]) \ \land$ \\
                                                   & & $\quad (\mathcal{C} \{ \erasetp{[[ pe1 ]]} \} \reduce \blamev \Longrightarrow \mathcal{C} \{ \erasetp{[[ pe2 ]]}  \}  \reduce \blamev)$

  \end{tabular}
  \end{center}
\end{definition}
The only difference is
that now when a program containing $[[pe1]]$ reduces to a value, so does one
containing $[[pe2]]$.


From \cref{lemma:repr}, it follows that coherence holds between
two representative translations of the same expression.

\begin{corollary}[Coherence for Representative Translations]
  For any expression $[[e]]$
  such that $[[ dd |- e : A ~~> rpe1    ]]$ and $[[ dd |- e : A ~~> rpe2    ]]$, we have
  $[[ dd ]] \vdash \ctxeq{[[rpe1]]}{[[rpe2]]}{[[A]]} $.
\end{corollary}

We have proved that for every typing derivation, at least one representative translation exists.

\begin{restatable}[Representative Translation for Typing]{lemma}{lemmareptyping}
  \label{lemma:rep:typing}
  For any typing derivation $[[dd |- e : A]]$ there exists at least one
  representative translation $r$ such that $[[dd |- e : A ~~> rpe]]$.
\end{restatable}

For our example, $(\blam x \unknown (\cast {\forall a. a \to \nat} {\gradual \to
  \nat} f) ~ (\cast \unknown \gradual x))$ is a representative translation,
while the other two are not.


\subsection{Dynamic Gradual Guarantee, Reloaded}

Given the above propositions, we are ready to revisit the dynamic gradual
guarantee. The nice thing about representative translations is that the
dynamic gradual guarantee of our source language is essentially that of \pbc,
our target language. However, the dynamic gradual guarantee for \pbc is still an
open question. According to \citet{yuu2017poly}, the difficulty lies in the
definition of term precision that preserves the semantics. We leave it here as a
conjecture as well. From a declarative point of view, we cannot prevent the
system from picking undesirable instantiations, but we know that some choices
are better than the others, so we can restrict the discussion of dynamic gradual
guarantee to representative translations.

\begin{conjecture}[Dynamic Gradual Guarantee in terms of Representative Translations]
  Suppose $e' \lessp e$,
  \begin{enumerate}
  \item If $[[empty |- e : A ~~> rpe]]$, $\erasetp {r} \Downarrow v$,
    then for some $B$ and $r'$, we have $[[ empty |- e' : B ~~> rpe']]$,
    and $B \lessp A$,
    and $\erasetp {r'} \Downarrow v'$,
    and $v' \lessp v$.
  \item If $[[empty |- e' : B ~~> rpe']]$, $\erasetp {r'} \Downarrow v'$,
    then for some $A$ and $[[rpe]]$, we have $ [[empty |- e : A ~~> rpe]]$,
    and $B \lessp A$. Moreover,
    $\erasetp r \Downarrow v$ and $v' \lessp v$,
    or $\erasetp r \Downarrow \blamev$.
  \end{enumerate}
\end{conjecture}

For the example in \cref{sec:criteria}, now we know that the representative
translation of the right one will evaluate to $1$ as well.
\begin{mathpar}
  (\blam{f}{\forall a. a \to \nat}{\blam{x}{\nat}{f~x}})~(\lambda x .\, 1)~3 \and
  (\blam{f}{\forall a. a \to \nat}{\blam{x}{\unknown}{f~x}})~(\lambda x .\, 1)~3
\end{mathpar}

More importantly, in what follows, we show that our extended algorithm is able to find those representative translations.


\subsection{Extended Algorithmic Type System}
\label{subsec:exd-algo}

\begin{figure}[t]
  \centering
  \begin{small}
    \begin{tabular}{lrcl} \toprule
      Types & $[[aA]], [[aB]]$ & \syndef & $ [[int]] \mid [[a]] \mid [[evar]] \mid [[aA -> aB]] \mid [[\/ a. aA]] \mid [[unknown]] \mid \hlmath{[[static]] \mid [[gradual]]} $ \\
      Monotypes & $[[at]], [[as]]$ & \syndef & $ [[int]] \mid [[a]] \mid [[evar]] \mid [[at -> as]] \mid \hlmath{[[static]] \mid [[gradual]]}$ \\
      \hl{Existential variables} & $[[evar]]$ & \syndef & $[[sa]]  \mid [[ga]]  $   \\
      \hl{Castable Types} & $[[agc]]$ & \syndef & $ [[int]] \mid [[a]] \mid [[evar]] \mid [[agc1 -> agc2]] \mid [[\/ a. agc]] \mid [[unknown]] \mid [[gradual]] $ \\
      \hl{Castable Monotypes} & $[[atc]]$ & \syndef & $ [[int]] \mid [[a]] \mid [[evar]] \mid [[atc1 -> atc2]] \mid [[gradual]]$ \\
      Algorithmic Contexts & $[[GG]], [[DD]], [[TT]]$ & \syndef & $[[empty]] \mid [[GG , x : aA]] \mid [[GG , a]] \mid [[GG , evar]]  \mid \hlmath{[[GG, sa = at]] \mid [[GG, ga = atc]]} \mid [[ GG, mevar ]] $ \\
      Complete Contexts & $[[OO]]$ & \syndef & $[[empty]] \mid [[OO , x : aA]] \mid [[OO , a]] \mid \hlmath{[[OO, sa = at]] \mid [[OO, ga = atc]]} \mid [[OO, mevar]] $ \\
      \bottomrule
    \end{tabular}
  \end{small}
  \caption{Syntax of types, contexts and consistent subtyping in the extended algorithmic system.}
  \label{fig:exd:algo:type}
\end{figure}


% \jeremy{the example is wrong, we need a new example to motivate}
To understand the design choices involved in the new algorithmic system, we
consider the following algorithmic typing example:
\[
  f: [[ \/a .  a -> int  ]], x : [[unknown]] \byinf [[ f x ]] \infto [[int]]  \dashv f : [[\/a . a -> int]], x : [[unknown]], \genA
\]
Compared with the declarative typing, where we have many choices (e.g., $[[int -> int]]$, $[[bool -> int]]$, and so on)
to instantiate $[[\/ a. a -> int]]$, the algorithm computes
the instantiation $[[ evar -> int ]]$ with $[[evar]]$ unsolved in the output context.
What can we know from the algorithmic typing? First we know that, here $[[evar]]$
is \textit{not constrained} by the typing problem. Second, and more importantly,
$[[evar]]$ has been compared with an unknown type (when typing $([[ f x ]])$).
Therefore, it is possible to make a more refined distinction
between different kinds of existential variables. The first
kind of existential variables are those that indeed have no constraints at all,
as they do not affect the dynamic semantics; while the second kind (as in this example) are
those where the only constraint is that
\textit{the variable was once compared with an unknown type}~\citep{garcia2015principal}.

The syntax of types is shown in \cref{fig:exd:algo:type}. A notable
difference, apart from the addition of static and gradual parameters, is that we
further split existential variables $[[evar]]$ into static existential variables
$[[ sa ]]$ and gradual existential variables $[[ga]]$.
Depending on whether an existential variable has been
compared with $[[unknown]]$ or not, its solution space changes. More
specifically, static existential variables can be solved to a monotype
$[[at]]$, whereas gradual existential variables can only be solved to a
castable monotype $[[atc]]$, as can be seen in the changes of algorithmic
contexts and complete contexts. As a result, the typing result for the above example
now becomes
\[
  f: [[ \/a .  a -> int  ]], x : [[unknown]] \byinf [[ f x ]] \infto [[int]]  \dashv f : [[\/a . a -> int]], x : [[unknown]], \hlmath{[[ga]]}
\]
since we can solve any unconstrained $[[ga]]$ to $[[gradual]]$, it is easy to
verify that the resulting translation is indeed a representative translation.

Our extended algorithm is novel in the following aspects. We naturally extend
the concept of existential variables~\citep{dunfield2013complete} to deal with
comparisons between existential variables and unknown types. Unlike
\citet{garcia2015principal}, where they use an extra set to store types that
have been compared with unknown types, our two kinds of existential variables emphasize
the type distinction better, and correspond more closely to the two kinds of type parameters,
as we can solve $[[sa]]$ to $ [[static]]$ and $[[ga]] $ to $ [[gradual]]$.

The implementation of the algorithm can be found in the supplementary materials.


\paragraph{Extended Algorithmic Consistent Subtyping}


\begin{figure}[t]
  \centering
  \begin{small}
   \begin{drulepar}[as]{$ [[GG |- aA <~ aB -| DD ]] $}{Algorithmic Consistent Subtyping}
     \drule{tvar}
     \drule{int}
     \drule{evar} \and
     \hlmath{\drule{spar}} \and
     \hlmath{\drule{gpar}} \and
     \hlmath{\drule{unknownLL}}
     \hlmath{\drule{unknownRR}} \and
     \drule{arrow}
     \drule{forallR} \and
     \hlmath{\drule{forallLL}} \and
     \drule{instL}
     \drule{instR}
   \end{drulepar}
  \end{small}
  \caption{Extended algorithmic consistent subtyping}
  \label{fig:exd:algo:sub}
\end{figure}

While the changes in the syntax seem negligible, the addition of static and
gradual type parameters changes the algorithmic judgments in a significant way.
We first discuss the algorithmic consistent subtyping, which is shown in \cref{fig:exd:algo:sub}.
For notational convenience, when static and
gradual existential variables have the same rule form, we compress them into one rule. For
example, \rref{as-evar} is really two rules $[[ GG[sa] |- sa <~ sa -| GG[sa] ]]$
and $[[ GG[ga] |- ga <~ ga -| GG[ga] ]]$; same for \rref{as-instL,as-instR}.

\Rref{as-spar,as-gpar} are direct analogies of \rref{cs-spar,cs-gpar}. Though
looking simple, \rref{as-unknownLL,as-unknownRR} deserve much explanation. To
understand what the output context $[[ [agc]GG ]]$ is for, let us first see why
this seemingly intuitive rule $[[ GG |- unknown <~ agc -| GG ]]$ (like
\rref{as-unknownL} in the original algorithmic system) is wrong. Consider the
judgment $[[ sa |- unknown <~ sa -> sa -| sa ]]$, which seems fine. If this
holds, then -- since $[[sa]]$ is unsolved in the output context -- we can solve
it to $[[ static ]]$ for example (recall that $[[sa]]$ can be solved to some
monotype), resulting in $[[ unknown <~ static -> static ]]$. However, this is in
direct conflict with \rref{cs-unknownLL} in the declarative system precisely
because $[[ static -> static ]]$ is not a castable type! A possible solution
would be to transform all static existential variables to gradual existential
variables within $[[agc]]$ whenever it is being compared to $[[ unknown ]]$:
while $[[ sa |- unknown <~ sa -> sa -| sa ]]$ does not hold, $[[ ga |- unknown
<~ ga -> ga -| ga ]]$ does. While substituting static existential variables with
gradual existential variables seems to be intuitively correct, it is rather hard
to formulate---not only do we need to perform substitution in $[[agc]]$, we also
need to substitute accordingly in both the input and output contexts in order to
ensure that no existential variables become unbound. However, making such changes is
at odds with the interpretation of input contexts: they are ``input'', which
evolve into output contexts with more variables solved. Therefore, in line with
the use of input contexts, a simple solution is to generate a
new gradual existential variable and solve the static existential variable to it
in the output context, without touching $[[agc]]$ at all. So we have $[[ sa |- unknown <~ sa -> sa -| ga, sa = ga ]]$.

Based on the above discussion, the following defines $[[ [aA]GG ]]$:
\begin{definition}$[[ [aA]GG ]]$ is defined inductively as follows  \label{def:contamination} %
  \begin{center}
    \begin{tabular}{llll} \toprule
     $[[ [aA] empty    ]]$ & = &  $[[empty]]$  & \\
    $[[ [aA] (GG, x : aA)  ]]$ &=& $[[ [aA] GG , x : aA     ]]$ & \\
    $[[ [aA] (GG, a)  ]]$ &=& $[[ [aA] GG , a     ]]$ & \\
    $[[ [aA] (GG, sa)  ]]$ &=& $[[ [aA] GG , ga , sa = ga  ]]$  & if $[[sa]]$ occurs in $[[aA]]$     \\
    $[[ [aA] (GG, sa)  ]]$ &=& $[[ [aA] GG , sa     ]]$     & if $[[sa]]$ does not occur in $[[aA]]$  \\
    $[[ [aA] (GG, ga)  ]]$ &=& $[[ [aA] GG , ga     ]]$ & \\
    $[[ [aA] (GG, evar = at)  ]]$ &=& $[[ [aA] GG , evar = at     ]]$ & \\
    $[[ [aA] (GG, mevar)  ]]$ &=& $[[ [aA] GG , mevar     ]]$ & \\ \bottomrule
    \end{tabular}
  \end{center}
\end{definition}
\noindent $[[ [aA]GG ]]$ solves all static existential variables found within $[[aA]]$ to fresh
gradual existential variables in $[[GG]]$. Notice the case for $[[ [aA] (GG, sa)]]$
is exactly what we have just described.

\Rref{as-forallLL} is slightly different from \rref{as-forallL} in the original
algorithmic system in that we replace $[[a]]$ with a new static existential
variable $[[sa]]$. Note that $[[sa]]$ might be solved to a gradual existential
variable later. The rest of the rules are the same as those in the original system.


\begin{figure}[t]
  \centering
  \begin{small}

   \begin{drulepar}[instl]{$ [[ GG |- evar <~~ aA -| DD   ]] $}{Instantiation I}
     \hlmath{\drule{solveS}} \and
     \hlmath{\drule{solveG}} \and
     \hlmath{\drule{solveUS}} \and
     \hlmath{\drule{solveUG}} \and
     \hlmath{\drule{reachSGOne}} \and
     \hlmath{\drule{reachSGTwo}} \and
     \hlmath{\drule{reachOther}} \and
     \drule{forallR}
     \drule{arr}
   \end{drulepar}

   \begin{drulepar}[instr]{$ [[ GG |- aA <~~ evar -| DD   ]] $}{Instantiation II, excerpt}
     \hlmath{\drule{forallLL}} \and
   \end{drulepar}
  \end{small}

  \caption{Instantiation in the extended algorithmic system}
  \label{fig:exd:inst}

\end{figure}

\paragraph{Extended Instantiation}

The instantiation judgments shown in \cref{fig:exd:inst} also change
significantly. The complication comes from the fact that now we have two different
kinds of existential variables, and the relative order they appear in the
context affects their solutions.


\Rref{instl-solveS, instl-solveG} are the refinement to \rref{instl-solve} in
the original system. The next two rules deal with situations where one side is
an existential variable and the other side is an unknown type.
\Rref{instl-solveUS} is a special case of \rref{as-unknownRR} where we create a
new gradual existential variable $[[ga]]$ and set the solution of $[[sa]]$ to be
$[[ga]]$ in the output context. \Rref{instl-solveUG} is the same as
\rref{instl-solveU} in the original system and simply propagates the input
context. The next two rules \rref*{instl-reachSG1,instl-reachSG2} are a bit
involved, but they both answer to the same question: how to solve a gradual
existential variable when it is declared after some static existential variable.
More concretely, in \rref{instl-reachSG1}, we feel that we need to solve
$[[gb]]$ to another existential variable. However, simply setting $[[ gb = sa]]$ and leaving $[[sa]]$ untouched
in the output context is wrong. The reason is that $[[gb]]$ could be a gradual existential
variable created by \rref{as-unknownLL}/\rref*{as-unknownRR} and solving $[[gb]]$ to a static existential
variable would result in the same problem as we have discussed. Instead, we create another new gradual
existential variable $[[ga]]$ and set the solutions of both $[[sa]]$ and $[[gb]]$ to it; similarly in \rref{instl-reachSG2}.
\Rref{instl-reachOther} deals with the other cases (e.g., $[[ sa <~~ sb  ]]$, $[[ ga <~~ gb  ]]$ and so on).
In those cases, we employ the same strategy as in the original system.

As for the other instantiation judgment, most of the rules are symmetric and thus omitted.
The only interesting rule is \rref*{instr-forallLL}, which is similar to what we did for \rref{as-forallLL}.



\paragraph{Algorithmic Typing and Metatheory}

Fortunately, the changes in the algorithmic bidirectional system are minimal: we replace
every existential variable with a static existential variable.
Furthermore, we proved that the extended
algorithmic system is sound and complete with respect to the extended
declarative system. The proofs can be found in the appendix.



\paragraph{Do We Really Need Type Parameters in the Algorithmic System?}

As we mentioned earlier, type parameters in the declarative system are merely an
analysis tool, and in practice, type parameters are inaccessible to
programmers. For the sake of proving soundness and completeness, we have to
endow the algorithmic system with type parameters. However, the algorithmic
system already has static and gradual existential variables, which can serve the same
purpose. In that regard, we could directly solve every \textit{unsolved} static and
gradual existential variable in the output context to $[[int]]$ and
$[[unknown]]$, respectively.


% \jeremy{example of showing finding the representative translation?}
% \ningning{Include a simple discussion?: since type parameters are used to help
%   with reasoning, in practice, programmers are actually not allowed to write
%   them. Therefore, the algorithm could directly set unsolved static existential
%   to integers and gradual existential to unknowns after type checking as
%   algorithmic refinement process, without even knowing type parameters. }

% \subsection{Discussion}

\subsection{Restricted Generalization}

In \cref{sec:type:trans}, we discussed the issue that the translation produces
multiple target expressions due to the different choices for instantiations, and
those translations have different dynamic semantics. Besides that, there is
another cause for multiple translations: redundant generalization during
translation by \rref{gen}. Consider the simple expression $[[(\x:int. x) 1]]$,
the following shows two possible translations:
\begin{align*}
  [[empty |- (\x : int . x) 1 : int ]] &[[~~>]] [[ (\x : int . x) (<int `-> int> 1)]]
  \\
  [[empty |- (\x : int . x) 1 : int ]] &[[~~>]]  [[ (\x : int . x) (<\/ a. int `-> int> (/\ a. 1))]]
\end{align*}
The difference comes from the fact that in the second translation, we apply
\rref{gen} while typing $1$ to get $[[empty |- 1 : \/ a. int]]$. As a consequence, the translation of $1$
is accompanied by a cast from $[[\/ a. int]]$ to $[[int]]$ since the former is a
consistent subtype of the latter. This difference is harmless, because obviously
these two expressions will reduce to the same value in \pbc, thus preserving
coherence (up to cast error). While it is not going to break coherence,
it does result in multiple representative translations for one
expression (e.g., the above two translations are both the representative translations).

There are several ways to make the translation process more deterministic. For
example, we can restrict generalization to happen only in let expressions and
require let expressions to include annotations, as $[[ let x : A = e1 in e2 ]]$.
Another feasible option would be to give a declarative, bidirectional system as
the specification (instead of the type assignment one), in the same spirit of
\citet{dunfield2013complete}. Then we can restrict generalization to be
performed through annotations in checking mode.

With restricted generalization, we hypothesize that now each expression has exactly
one representative translation (up to renaming of fresh type parameters).
Instead of calling it a \textit{representative} translation, we can say it is a
\textit{principal} translation. Of course the above is only a sketch; we have
not defined the corresponding rules, nor studied metatheory.


\begin{comment}
\subsubsection{Interpretation of Type Parameters}
\label{subsec:type-par}

% \jeremy{If I understand it correctly, we actually used these two interpretations
%   in the extended declarative system. Def 8.1 (substitutions) is the first
%   interpretation; and Def 8.2 (syntactic refinement) is the second
%   interpretation in that $[[static]]$ is irrelevant to program execution so we
%   can replace it with any type, whereas $[[gradual]]$ is relevant so we replace
%   it with unknown }

In \cref{sec:type-param}, we introduced type parameters into our type system. It turns
out that type parameters are a useful tool to help us identify
representative translations and reason about the dynamic semantics of the
type system. But what are type parameters, exactly? Below we provide two plausible
interpretations.

The first interpretation of type parameters (the one we adopted) is that they are placeholders for
monotypes. This is to say, their meaning is given by substitution, and replacing
type parameters with other monotypes should not break typing:

\begin{proposition}
  If $[[dd |- e : A]]$, then $\psubst ([[dd]]) \vdash \psubst ([[e]]) : \psubst ([[A]])$.
\end{proposition}

\jeremy{See Proposition 1 of Principle scheme for gradual programs, where they also have exactly the same proposition, but they call it
type polymorphism! how this compare to the second interpretation?}

In practice, we should not allow programmers to write type parameters explicitly
in a program, as type parameters are only generated during typing process, and
get refined before evaluation. As a result, we can hide all the details of type
parameters from programmers and internally replace them with suitable concrete
types when necessary. This also reflects the point we discussed in the end of
\cref{subsec:exd-algo}.

On the other hand, we can interpret type parameters using \textit{polymorphism}.
In this sense, both of them can be extracted to generate type abstractions.
However, there is one subtle difference. That is, static type parameter
indicates \textit{parametric polymorphism} in the traditional sense, which is
irrelevant to program execution; while gradual type parameter indicates
\textit{gradual polymorphism}, which means it has no typing constraints but is
relevant to program execution \citep{garcia2015principal}. This interpretation
suggests that we might need a more refined distinction between type
abstractions, such as \citet{yuu2017poly}.

We argue that the extension of type parameters is \textit{a} feasible way to
reason about the dynamic semantics in a implicit polymorphic language, but it is
not necessarily \textit{the} only way. Still, it remains to see if
our discussion sheds lights on the study of dynamic semantics for
gradual languages with implicit polymorphism.

\end{comment}



%%% Local Variables:
%%% mode: latex
%%% TeX-master: "../paper"
%%% org-ref-default-bibliography: ../paper.bib
%%% End:

\section{Related Work}
\label{sec:related}

% \bruno{I think (part of) this text can be discussed in here instead:


There are multiple flavours of inheritance. To avoid confusion, since the same
terminology is often used in the literature to mean different things, we use the
following 3 terms when comparing related work with ours.

\begin{itemize}
\item{{\bf Static inheritance:}} Static inheritance refers to what the typical
  model of inheritance in class-based languages. The inheritance model is said
  to be static because when using class extension, the extended classes are
  statically known at compile-time.
\item{{\bf Mutable Inheritance:}} Prototype-based languages allow another model
  of inheritance, which we call \emph{mutable inheritance}. In this inheritance
  model, self-references are mutable and changeable at any point.
\item{{\bf Dynamic Inheritance:}} Dynamic inheritance is a less well-known model
  which stands in between static and mutable inheritance. Unlike the static
  inheritance model, with dynamic inheritance objects can inherit from other
  objects which are not statically known. However, unlike mutable inheritance,
  the self-reference is not mutable and cannot be arbitrarily changed at
  run-time.
\end{itemize}

Figure~\ref{fig:comparision} shows the comparison between \name and various
similar languages that follow \citeauthor{cook1989inheritance}'s ``Inheritance is not
Subtyping'' (i.e. the flexible model), as we will explain below.

\begin{figure}[t]
  \centering
  \begin{small}
  \begin{tabular}{|l||c|c|c|c|}
    \hline
    & \bf{Statically typed} & \bf{Polymorphism} & \bf{Meta-theory} & \bf{Inheritance}  \\
    \hline
    \name & \cmark & \cmark & \cmark & Dynamic \\
    \hline
    \textsc{Self} & \xmark & \xmark & \xmark & Mutable \\
    \hline
    Cecil & \cmark & \cmark & \xmark & Static \\
    \hline
    Cook's Modula-3 & \cmark & \xmark & \xmark & Static \\
    \hline
    IFJ & \cmark & \xmark & \cmark & Dynamic \\
    \hline
    \textsc{Darwin} & \cmark & \xmark & \xmark & Dynamic \\
    \hline
  \end{tabular}
  \end{small}
  \caption{Comparison between \name and various similar languages that
  adopt the \emph{flexible model}.}
  \label{fig:comparision}
\end{figure}



% \paragraph{Dynamically-typed Languages with Delegation Mechanism}

% \begin{itemize}
% \item Clojure Protocols
%   % http://www.ibm.com/developerworks/library/j-clojure-protocols/
% \item Ruby mixin
% \item JS mixin
% \end{itemize}

% They are all dynamically typed.


\paragraph{Delegation-based languages}

\citet{lieberman1986using} is the first to promote the use of prototypes and
delegation as the mechanism to code sharing between objects. Since then many
researchers have studied the mechanisms of
delegation~\cite{wegner1987dimensions,malenfant1995semantic,goldberg1989smalltalk}.
\textsc{Self}~\cite{ungar1988self} is a dynamically typed, prototype-based
language with a simple and uniform object model. \textsc{Self}'s inheritance
model is typical of what we call mutable inheritance, because an object's parent
slots may be assigned new values at run-time. Mutable inheritance is rather
unstructured, and oftentimes access to any clashing methods will generate a
``messageAmbiguous'' error at run-time. Although \name's dynamic inheritance is
not as powerful as mutable inheritance, its static type system can guarantee
that no such errors occur at run-time.

There is not much work on statically-typed, delegation-based languages.
\citet{kniesel1999type} provides a good overview of problems when combining
delegation with a static type discipline. Cecil~\cite{chambers1992object,
  chambers1993cecil} is a prototype-based language, where delegation is the
mechanism for method call and code reuse. Cecil supports a polymorphic static
type system, although no meta-theory of any kind is given. Its type system is
able to detect statically when a message might be ambiguously defined as a
result of multiple inheritance or multiple dispatching. However, one major
omission of Cecil, which is also one of the interesting features of \name, is
dynamic inheritance. There are other
works~\cite{fisher1995delegation,anderson2003can} on delegation in a
statically-typed setting, but none of them provide means (such as the merge
construct, disjointness constraints, etc.) that are needed for extensible
designs.

\citet{cook1989inheritance} were the first to propose a typed model of
inheritance where subtyping and inheritance are two separate concepts. In
particular, they introduce the notion of \textit{type inheritance} and show that
inherited objects have inherited types, not subtypes. An interesting aspect of
their calculus is the \textbf{with} construct, used to join two records. This is
somewhat similar to our merge construct. However two major differences are worth
pointing out: 1) the \textbf{with} construct operates only on records; and 2) it
is a biased operator, favoring values from its right argument. This biased
operator is good for modelling mixins, but not traits. The
\textbf{with} construct seems to be unable to merge two arbitrary (and possible
polymorphic) values, since this seems to require something like
\emph{row polymorphism}~\cite{wand1987complete,wand1989type}, which is not available in their language.
The \textit{onion} construct in the Big Bang
language~\cite{palmer2015building,menon2012big} has a similar bias problem -- it is a
left-associative operator which gives rightmost precedence to one
implementation when conflicts exist.

\paragraph{Mixin-based inheritance}

Mixins have become very popular in many OO languages
~\cite{flatt1998classes,bono1999core, ancona2003jam}. \citeauthor{bracha1990mixin}'s
seminal paper~\citep{bracha1990mixin} extends Modula-3 with mixins. Mixins are subclasses parameterized
over a superclass, and used to produce a variety of classes with the same
functionality and behaviour. Mixin-based inheritance requires that mixins be
composed linearly, and as such, conflicts are resolved implicitly (mixins
appearing later overwrite all the identically named features of earlier mixins).
In comparison, the trait model in \name requires conflicts be resolved
explicitly. We want to emphasize that this conflict detection is essential in
expressing composition operators for Object Algebras, without running
into ambiguities.


\paragraph{Trait-based inheritance}

The seminar paper by \citet{scharli2003traits} introduced the ideas behind
traits, where they also documented an implementation of the trait
mechanism in a dynamically typed version of Smalltalk. Since then many
formalizations of traits have been
proposed~\cite{scharli2003traitsformal,ducasse2006traits,bettini2010prototypical}.
For example \citet{fisher2004typed} presented a statically-typed calculus that
models traits. Conflict detection is the hallmark of trait-based
inheritance, compared with mixin-based inheritance. One important difference
with \name is that those systems support \textit{classes} in addition to traits,
and consider the interaction between them, whereas \name is 
delegation based and the mechanism for code reuse is purely traits
(i.e., there are no classes in \name). The
deviation from traditional class-based models is not only because of its
simplicity, but also because we need a very \textit{dynamic} form of
inheritance, as has been elaborated throughout the paper.

Compared to the traditional trait mode, traits in \name have the following
differences: 1) traditional traits cannot be instantiated but only composed with
a class, whereas traits in \name can be instantiated directly; 2) traditional
traits cannot take constructor parameters whereas ours can; 3) the trait system
in \name lacks a proper notation of inheritance relationship. For example in the
traditional trait model, if the same method (i.e., from the same trait) is
obtained more than once via different paths, there is no conflict. This is not
the case in \name; and 4) traits in \name support dynamic
inheritance. 
%In the
%traditional trait model, when it comes to inheritance, the traits being
%inherited must be statically known.




% \citet{flatt1998classes} proposed MIXEDJAVA, an extension to a subset of
% sequential Java called CLASSICJAVA with mixins. In their model, mixins
% completely subsume the role of classes (classes are mixins that do not inherit
% any services). One interesting aspect in their system is that two identically
% named methods are allowed to coexist, and are resolved at run-time with run-time
% context information provided by the current \textit{view} of an object. In
% comparison, conflicts in \name are detected statically, and resolved by the
% programmers. Like \name, their model also enforces the distinction between
% implementation inheritance and subtyping.

% \citet{bono1999core} develop an imperative class-based calculus that provides a
% formal model for both single and mixin inheritance. Objects are represented by
% records and produced by instantiating classes. In their calculus, the class
% construct is extensible but not subtypable, while objects are subtypable but not
% extensible. Like \name, their system has a clean separation between subtyping
% and inheritance. Also, their type system does not have polymorphism.

% \citet{ancona2003jam} extends the Java language to support mixins, called Jam.
% Since Jam is an upward-compatible extension of Java 1.0, it is inheritantly a
% covariant mode. Unlike MIXEDJAVA, mixins can be only instantiated on classes,
% and there is no notion of mixin composition.


\begin{comment}

\begin{itemize}


\item ``Object-Oriented Multi-Methods in Cecil''

\item ``Dimensions of Object-Based Language Design''

\item ``On the Semantic Diversity of Delegation-Based Programming Languages''

\item ``Self: The power of simplicity''

\item ``Type-safe delegation for run-time component adaptation''

\item ``A delegation-based object calculus with subtyping''

\item ``Can Addresses be Types? (a case study: objects with delegation)''

\item ``Inheritance is not subtyping''


Mixins

\item ``mixin-based inheritance''

\item ``Classes and mixins''

\item ``A core calculus of classes and mixins''

\item ``A core calculus of higher-order mixins and classes''

\item ``Jam—Designing a Java Extension with Mixins''



\end{itemize}

Do they have polymorphic type systems? Do they support mutable self reference?

\end{comment}


\paragraph{Class-based languages with more advanced forms of inheritance}

Incomplete Featherweight Java (IFJ), proposed by \citet{bettini2008type}, is a
conservative extension of Featherweight Java with incomplete objects. Besides
standard classes, programmers can also define incomplete classes, whose
instances are incomplete objects. Incomplete objects can be composed (by object
composition) with complete objects, yielding new complete objects at run-time,
while ensuring statically that the composition is type-safe. Incomplete objects
are quite flexible, and support dynamic inheritance. However, object composition
in IFJ is quite restrictive, compared to \name, in that it can only compose an
incomplete object with a complete object. In that regard, and also because IFJ's
type system is not polymorphic, IFJ is unable to encode composition operators of
Object Algebras. \citet{kniesel1999type} showed that type-safe integration of
delegation with subtyping into a class-based model is possible, resulting in the
\textsc{Darwin} model. In \textsc{Darwin}, the type of the parent object must be
a declared class and this limits the flexibility of dynamic composition.
\citeauthor{ostermann2002dynamically}'s delegation
layers~\citep{ostermann2002dynamically} use delegation for doing dynamic
composition in a system with virtual classes. This is in contrast with most
other approaches that use class-based composition, but closer to the dynamic
composition that we use in \name.

There are many other class-based OO languages that are equipped with more
advanced forms of
inheritance~\cite{meyer1987eiffel,buchi2000generic,ostermann2001object}. Most of
them are heavyweight and are specific to classes. \name is object-centered, more
lightweight, and is dedicated to express extensible designs in a simpler way.


% Eiffel~\cite{meyer1987eiffel} is a class-based language that is based on the
% identification of classes with types and of inheritance with subtyping. Eiffel
% supports multiple inheritance, with the restriction that name collisions are
% considered programming errors, and ambiguities must be resolved explicitly by
% the programmer (by means of renaming). In this regard, \name is quite like
% Eiffel. However, the type system in \name is more lenient in that two
% identically named methods with different signatures can coexist without any
% problems.

% \citet{kniesel1999type} is the first to show that type-safe integration of
% delegation with subtyping into a class-based model is possible, resulting in the
% DARWIN model. In the DARWIN model, the type of the parent object must be a
% declared class and this limits the flexibility of dynamic composition, whereas
% in \name, the merge operator can merge/compose any objects. Another difference
% with \name lies in the conflict resolution, where DARWIN relies on method
% overriding with the assumption that the author of the overriding method is aware
% of the effect.

% Generic wrappers~\cite{buchi2000generic} supports aggregating objects at
% run-time. In their model, once a ``wrappee'' is assigned to a ``wrapper'', the
% wrappee is fixed. GBETA~\cite{ernst2000gbeta} has some dynamic features that are
% related to delegation. Like Generic wrappers, parents in GBETA are fixed at
% run-time.

% \citet{ostermann2001object} proposed compound references (CR) as a abstraction
% for object references, which provides explicit linguistic support for combining
% different composition properties on-demand. The model is statically typed, and
% decouples subtype declaration from implementation reuse.


% \citet{ostermann2002dynamically} proposed delegation layers as an approach to
% decompose a collaboration into layers and compose these layers dynamically at
% run-time. This combines and generalizes delegation and virtual classes concepts.

% \citet{ostermann2008nominal} compared the nominal and structural subtyping
% mechanisms. They argue nominal subtyping gives more safety guarantee, whereas
% structural subtyping is more flexible from a component-based perspective. The
% type system of \name chooses structural subtyping.

\paragraph{Intersection types, polymorphism and the merge construct}

There is a large body of work on intersection types. Here we only talk about
work that have direct influences on ours. \citet{dunfield2014elaborating} shows
significant expressiveness of type systems with intersection types and a merge
construct. However his calculus lacks coherence. The limitation was addressed
by~\citet{oliveira2016disjoint}, where they introduced the notion of
disjointness to ensure coherence. The combination of intersection types, a merge
construct and parametric polymorphism, while achieving coherence was first
studied in the \bname calculus~\cite{alpuimdisjoint}, where they proposed the
notion of disjoint polymorphism. \bname serves as the theoretical foundation of
\name.


\begin{comment}

\begin{itemize}

\item Eiffel

\item ``Delegation by object composition'' (IFJ) and ``Type safe dynamic object
  delegation in class-based languages''

\item ``Dynamically composable collaborations with delegation layers''

\item ``Generic wrappers''

\item ``Object-Oriented Composition Untangled''

\item ``gbeta - a language with virtual attributes, Block Structure, and Propagating, Dynamic Inheritance''

\item ``Nominal and Structural Subtyping in Component-Based Programming''

\item ``Engineering a programming language: The type and class system of Sather ''

\item ``Big Bang Designing a Statically-Typed Scripting Language''

\item ``Building a Typed Scripting Language''



\end{itemize}

\end{comment}


\section{Conclusions and Future Work}
\label{sec:conclusion}

We have proposed \name, a type-safe and coherent calculus with disjoint
intersection types, and support for nested composition/subtyping. \name
improves upon earlier work with a more
flexible notion of disjoint intersection types, which leads to
a clean and elegant formulation of the type system. Due to the added
flexibility we have had to employ a more powerful proof method based on logical
relations to rigorously prove coherence.
We also show how \name supports essential features of family
polymorphism, such as nested composition. We believe \name provides insights into family polymorphism, and
has potential for practical applications for extensible software designs.

A natural direction for future work is to enrich \name with parametric
polymorphism. There is abundant literature on logical relations for parametric
polymorphism~\citep{reynolds1983types} and we foresee no fundamental
difficulties in extending our proof method.\footnote{
Our prototype
  implementation already supports polymorphism, but we
  are still in the process of extending our Coq development with polymorphism. } The resulting calculus will be
more expressive than \fname. An interesting application that we intend to investigate
is native support for \textit{object algebras}~\citep{oliveira2012extensibility}
(or the finally tagless approach~\citep{CARETTE_2009}). For example, we can
define the object algebra interfaces for the Expression Problem example in
\cref{sec:overview} as follows:
\lstinputlisting[linerange=75-76]{../../impl/examples/overview.sl}% APPLY:linerange=LANG_EXT_INTER
By instantiating \lstinline{E} with \lstinline{IPrint}, i.e.,
\lstinline{ExpAlg[IPrint]}, we get the interface of the \lstinline{Lang} family.
In that sense, object algebra interfaces can be viewed as family interfaces.
Moreover, combing algebras implementing \lstinline{ExpAlg[IPrint]} and
\lstinline{ExpAlg[IEval]} to form \lstinline{ExpAlg[IPrint & IEval]} is trivial
with nested composition. Polymorphism also improves code reuse across expressions in the
base and extended languages. For example, the following creates two expressions,
one in the base language, the other in the extended language:
\lstinputlisting[linerange=81-82]{../../impl/examples/overview.sl}% APPLY:linerange=LANG_EXT
Notice how we can  reuse \lstinline{e1} of the base language in the definition
of \lstinline{e2}.



% \jeremy{creating expressions using base and extended expressions, and show more reuse}

% \jeremy{future work} \jeremy{mention in passing this rule is unsound with
%   effects, see ``Intersection types and computational effects''}

% Local Variables:
% mode: latex
% TeX-master: "../paper"
% End:
\section*{Acknowledgements}

We thank Ronald Garcia, Dustin Jamner, and the anonymous reviewers for their
helpful comments. This work has been sponsored by the Hong Kong Research Grant
Council projects number 17210617 and 17258816, and by the Research Foundation -
Flanders.

%%% Local Variables:
%%% mode: latex
%%% TeX-master: "../paper"
%%% org-ref-default-bibliography: "../paper.bib"
%%% End:



%% Acknowledgments

%% Bibliography
\bibliography{paper}

% \ifdefined\submitoption
\newpage
\appendix
\section{Full Specification of Core Language}

\subsection{Syntax}
\gram{\otte\ottinterrule
        \ottG\ottinterrule
        \ottv}
\\[2.0mm]
Syntactic Sugar\\
\resizebox{\columnwidth}{!}{$\ottcoresugar$} % defined in otthelper.mng.tex

\subsection{Operational Semantics}
\ottdefnstep{}
\ottusedrule{\ottdruleSXXMu{}}

\subsection{Typing}
\ottdefnctx{}\ottinterrule
\ottdefnexpr{}
\ottusedrule{\ottdruleTXXMu{}}

\section{Proofs about Core Language}
\subsection{Properties}
We follow the naming of lemmas and proofs of properties 
for Pure Type System from \cite{handbook}. Some lemmas have other well-known names, like
Lemma \ref{lem:appendix:thin} is often called \emph{Weakening} and 
Lemma \ref{lem:appendix:gen} is often called \emph{Inversion}.

\begin{comment}
\begin{lem}[Free Variable]\label{lem:appendix:free}
    If $[[G |- e:t]]$, then $\FV(e) \subseteq \dom([[G]])$ and $\FV([[t]])
\subseteq \dom([[G]])$.
\end{lem}

\begin{proof}
    By induction on the derivation of $[[G |- e:t]]$. We only treat cases
\ruleref{T\_Mu}, \ruleref{T\_CastUp} and \ruleref{T\_CastDown} (since proofs of
other cases are the same as \cc \cite{handbook}):
    \begin{description}
        \item[Case \ruleref{T\_Mu}:] From premises of $[[G |- (mu x:t.e1) :
t]]$, by the induction hypothesis, we have $\FV(e_1) \subseteq \dom([[G]]) \cup
\{[[x]]\}$ and $\FV(\tau) \subseteq \dom([[G]])$. Thus the result follows by
$\FV([[mu x:t.e1]])=\FV(e_1) \setminus \{[[x]]\} \subseteq \dom([[G]])$ and
$\FV(\tau) \subseteq \dom([[G]])$.
        \item[Case \ruleref{T\_CastUp}:] Since $\FV([[castup [t]
e1]])=\FV([[e1]])$, the result follows directly by the induction hypothesis.
        \item[Case \ruleref{T\_CastDown}:] Since $\FV([[castdown
e1]])=\FV([[e1]])$, the result follows directly by the induction hypothesis.
    \end{description}
\end{proof}
\end{comment}

\begin{lem}[Thinning]\label{lem:appendix:thin}
    Let $[[G]]$ and $[[G']]$ be legal contexts such that $[[G]] \subseteq
[[G']]$. If $[[G |- e : t]]$ then $[[G' |- e : t]]$.
\end{lem}

\begin{proof}
    By trivial induction on the derivation of $[[G |- e : t]]$.
\end{proof}

\begin{lem}[Substitution]\label{lem:appendix:subst}
	If $[[G1, x:T, G2 |- e1:t]]$ and $[[G1 |- e2:T]]$, then $[[G1, G2 [x |-> e2]
|- e1[x |-> e2]  : t[x |-> e2] ]]$.
\end{lem}

\begin{proof}
    By induction on the derivation of $[[G1, x:T, G2 |- e1:t]]$. We use the notation $[[e* == e
[x |-> e2] ]]$ to denote the substitution for short. Then the result can be written as \[ [[G1, G2* |- e1*  : t* ]]\]
We only treat cases \ruleref{T\_Mu}, \ruleref{T\_CastUp} and
\ruleref{T\_CastDown} since other cases can be easily followed by the proof for PTS in \cite{handbook}.
Consider the last step of derivation of the following
cases:
    \begin{description}
        \item[Case \ruleref{T\_Mu}:] $\inferrule{[[G1, x:T, G2, y:t |- e1:t]] \\
[[G1, x:T, G2 |- t:s]]}{[[G1, x:T, G2 |- (mu y:t.e1): t]]}$ 
        
        By the induction hypothesis, we have $[[G1, G2*, y:t* |- e1* : t*]]$ and $[[G1,
G2* |- t* : star]]$. Then by the derivation rule, $[[G1, G2* |- (mu
y:t*.e1*):t*]]$. Thus we can conclude $[[G1, G2* |- (mu y:t.e1)*:t*]]$.
        \item[Case \ruleref{T\_CastUp}:] $\inferrule{[[G1, x:T, G2 |- e1:t2]]
\\ [[G1, x:T, G2 |- t1:s]] \\ [[t1 --> t2]]}{[[G1, x:T, G2 |- (castup [t1]
e1):t1]]}$ 
        
        By the induction hypothesis, we have $[[G1, G2* |- e1*:t2*]]$, $[[G1, G2*
|- t1*:star]]$ and $[[t1 --> t2]]$. By the definition of substitution, we can
obtain $[[t1* --> t2*]]$ by $[[t1 --> t2]]$. Then by the derivation rule, $[[G1,
G2* |- (castup [t1*] e1*):t1*]]$. Thus we can conclude $[[G1, G2* |- (castup [t1]
e1)*:t1*]]$.
        \item[Case \ruleref{T\_CastDown}:] $\inferrule{[[G1, x:T, G2 |- e1:t1]]
\\ [[G1, x:T, G2 |- t2:s]] \\ [[t1 --> t2]]}{[[G1, x:T, G2 |- (castdown
e1):t2]]}$ 
        
        By the induction hypothesis, we have $[[G1, G2* |- e1*:t1*]]$, $[[G1, G2*
|- t2*:star]]$ and $[[t1 --> t2]]$ thus $[[t1* --> t2*]]$. Then by the
derivation rule, $[[G1, G2* |- (castdown e1*):t2*]]$. Thus we can conclude $[[G1, G2* |-
(castdown e1)*:t2*]]$.
    \end{description}
\end{proof}

\begin{lem}[Generation]\label{lem:appendix:gen}
If the alpha equivalence is witnessed by notation $[[=a]]$, we have the following results:
\begin{enumerate}[(1)]
	\item If $[[G |- x:T]]$, then there exist an expression $[[t]]$ such that $[[t
=a T]]$, $[[G |- t:s]]$ and $[[x:t elt G]]$.
	\item If $[[G |- e1 e2:T]]$, then there exist expressions $[[t1]]$ and
$[[t2]]$ such that $[[G |- e1 : (Pi x:t2.t1)]]$, $[[G |- e2:t2]]$ and $[[T =a
t1[x |-> e2] ]]$.
	\item If $[[G |- (\x:t1.e):T]]$, then there exist an expression $[[t2]]$ such
that $[[T =a Pi x:t1.t2]]$ where $[[G |- (Pi x:t1.t2):s]]$ and $[[G,x:t1 |-
e:t2]]$.
    \item If $[[G |- (Pi x:t1.t2):T]]$, then $[[T == s]]$, $[[G |- t1:s]]$ and
$[[G, x:t1 |- t2:s]]$.
	\item If $[[G |- (mu x:t.e):T]]$, then $[[G |- t:s]]$, $[[T =a t]]$ and $[[G,
x:t|-e:t]]$.
	\item If $[[G |- (castup [t1] e):T]]$, then there exist an expression $[[t2]]$
such that $[[G |- e:t2]]$, $[[G |- t1:s]]$, $[[t1 --> t2]]$ and $[[T =a t1]]$.
	\item If $[[G |- (castdown e):T]]$, then there exist expressions
$[[t1]],[[t2]]$ such that $[[G |- e:t1]]$, $[[G |- t2:s]]$, $[[t1 --> t2]]$ and
$[[T =a t2]]$.
\end{enumerate}
\end{lem}

\begin{proof}
    Consider a derivation of $[[G |- e:T]]$ for one of cases in the lemma. We
follow the process of derivation until expression $[[e]]$ is introduced the
first time. The last step of derivation can be done by
    \begin{itemize}
        \item rule \ruleref{T\_Var} for case 1;
        \item rule \ruleref{T\_App} for case 2;
        \item rule \ruleref{T\_Lam} for case 3;
        \item rule \ruleref{T\_Pi} for case 4;
        \item rule \ruleref{T\_Mu} for case 5;
        \item rule \ruleref{T\_CastUp} for case 6;
        \item rule \ruleref{T\_CastDown} for case 7.
    \end{itemize}
    In each case, assume the conclusion of the rule is $[[G' |- e : t']]$ where
$[[G']] \subseteq [[G]]$ and $[[t' =a T]]$. Then by inspection of used
derivation rules and Lemma \ref{lem:appendix:thin}, it can be shown that the
statement of the lemma holds and is the only possible case.
\end{proof}

\begin{lem}[Correctness of Types]\label{lem:appendix:corrtyp}
    If $[[G |- e:t]]$ then $[[t == s]]$ or $[[G |- t : s]]$.
\end{lem}

\begin{proof}
    Trivial induction on the derivation of $[[G |- e:t]]$ using Lemma
\ref{lem:appendix:gen}.
\end{proof}

\subsection{Decidability of Type Checking}
\begin{lem}[Decidability of One-step Reduction]\label{lem:appendix:unired}
	The one-step reduction $[[-->]]$ is called decidable if 
given $[[e]]$ there is a unique $[[e']]$ such that $[[e --> e']]$ or there is no such $[[e']]$.
\end{lem}

\begin{proof}
	By induction on the structure of $[[e]]$:
	\begin{description}
        \item[Case $[[e=x]]$:] $[[e]]$ is a variable which does not match any rules of $[[-->]]$. 
        Thus there is no $[[e]]'$ such that $[[e-->e']]$.
		\item[Case $[[e=v]]$:] $[[e]]$ is a value that has one of the following forms:
		\begin{inparaenum}[(1)]
		    \item $[[star]]$,
			\item $[[\x:t.e]]$,
			\item $[[Pi x:t1.t2]]$,
			\item $[[castup [t] e]]$.
		\end{inparaenum}
		Thus, it does not match any rules of $[[-->]]$. Then there is no $[[e]]'$ such that $[[e-->e']]$.
		\item[Case $[[e]]=[[(\x:t.e1) e2]]$:] Since the first term $[[\x:t.e1]]$ is a value, rule \ruleref{S\_App} does not apply to this case. Thus, only rule \ruleref{S\_Beta} can be applied and there is a unique $[[e']]=[[ e1[x|->e2] ]]$.
		\item[Case $[[e]]=[[castdown (castup [t] e1)]]$:] Since the inner term $[[castup [t] e1]]$ is a value, rule \ruleref{S\_CastDown} does not apply to this case. Thus, only rule \ruleref{S\_CastDownUp} can be applied and there is a unique $[[e']]=[[e1]]$.
		\item[Case $[[e]]=[[mu x:t.e1]]$:] Only rule \ruleref{S\_Mu} can be applied. Thus, there is a unique $[[e]]'=[[e1[x|->mu x:t.e1] ]]$.
		\item[Case $[[e]]=[[e1 e2]]$ and $[[e1]]$ is not a $\lambda$-term:] If
$[[e1]]=v$ and is not a $\lambda$-term, there is no rule to reduce $[[e]]$. 
Then there is no $[[e1']]$ such that $[[e1 --> e1']]$, which does not satisfy the premise of 
rule \ruleref{S\_App}. Thus, there is no $[[e]]'$ such that $[[e-->e']]$.

		Otherwise, if $[[e1]]$ is not a value, there exists some $[[e1']]$ such that $[[e1 --> e1']]$. By the
induction hypothesis, $[[e1']]$ is the unique reduction of $[[e1]]$. Thus, by rule
\ruleref{S\_App}, $[[e]]'=[[e1' e2]]$ is the unique reduction of $[[e]]$.
		\item[Case $[[e]]=[[castdown e1]]$ and $[[e1]]$ is not a $[[castup]]$-term:] If
$[[e1]]=v$ and is not a $[[castup]]$-term, there is no rule to reduce $[[e]]$. 
Then there is no $[[e1']]$ such that $[[e1 --> e1']]$, which does not satisfy the premise of 
rule \ruleref{S\_CastDown}. Thus, there is no $[[e]]'$ such that $[[e-->e']]$.

        Otherwise, if $[[e1]]$ is not a value, there exists some $[[e1']]$ such that $[[e1 --> e1']]$. By the
induction hypothesis, $[[e1']]$ is the unique reduction of $[[e1]]$. Thus, by rule
\ruleref{S\_CastDown}, $[[e]]'=[[castdown e1']]$ is the unique reduction of $[[e]]$.
	\end{description}
\end{proof}

\begin{thm}[Decidability of Type Checking]
	There is an algorithm which given $[[G]], [[e]]$ computes the unique
$[[t]]$ such that $[[G |- e:t]]$ or reports there is no such $[[t]]$.
\end{thm}

\begin{proof}
	By induction on the structure of $[[e]]$:
	\begin{description}
	    \item[Case $[[e=star]]$:] Trivial by applying \ruleref{T\_Ax} and $[[t ==
star]]$.
		\item[Case $[[e=x]]$:] Trivial by rule \ruleref{T\_Var}. If $[[x:t elt G]]$, then $[[t]]$ is the
unique type of $[[x]]$ such that $[[G |- x : t]]$. Otherwise, if $[[x]] \not \in \dom([[G]])$, there is no such $[[t]]$.
		\item[Case $[[e]]=[[e1 e2]]$:] By rule \ruleref{T\_App} and induction
hypothesis, there exist unique $[[t1]]$ and $[[t2]]$ such that $[[G
|- e1 : (Pi x:t1.t2)]]$, $[[G |- e2:t1]]$. Thus, $[[t2[x |-> e2] ]]$ is the unique type of $[[e]]$ such that $[[G |- e : t2[x |-> e2] ]]$.
		\item[Case $[[e=\x:t1.e1]]$:] By rule \ruleref{T\_Lam} and induction
hypothesis, there exist unique $[[t2]]$ such that $[[G |- (Pi
x:t1.t2):s]]$ and $[[G,x:t1 |- e:t2]]$. Thus, $[[Pi x:t1.t2 ]]$ is the unique type of $[[e]]$ such that $[[G |- e : Pi x:t1.t2  ]]$.
		\item[Case $[[e=Pi x:t1.t2]]$:] By rule \ruleref{T\_Pi} and induction
hypothesis, we have $[[G |- t1:s]]$ and $[[G, x:t1 |- t2:s]]$. Thus, $[[s]]$ is the unique type of $[[e]]$ such that $[[G |- e : s  ]]$.
		\item[Case $[[e=mu x:t.e1]]$:] By rule \ruleref{T\_Mu} and induction
hypothesis, we have $[[G |- t:s]]$ and $[[G, x:t|-e:t]]$. Thus, $[[t]]$ is the unique type of $[[e]]$ such that $[[G |- e : t]]$.
		\item[Case $[[e]]=[[castup [t1] e1]]$:] From the premises of rule
\ruleref{T\_CastUp}, by the induction hypothesis, we can derive the type of
$[[e1]]$ as $[[t2]]$ by $[[G |- e1:t2]]$, and check whether $[[t1]]$ is legal by $[[G |- t1:star]]$. 
For a legal $[[t1]]$, by Lemma \ref{lem:appendix:unired}, there is
a unique $[[t1']]$ such that $[[t1 --> t1']]$ or there is no such $[[t1']]$. 
If such $[[t1']]$ does not exist, then we report type checking fails. 

Otherwise, we examine if $[[t1']]$ is syntactically equal to $[[t2]]$, 
i.e., $[[t1' =a t2]]$. If the equality
holds, we conclude the unique type of $[[e]]$ is $[[t1]]$, i.e., $[[G |- e:t1]]$. Otherwise, we
report $[[e]]$ fails to type check.
		\item[Case $[[e]]=[[castdown e1]]$:] From the premises of rule
\ruleref{T\_CastDown}, by the induction hypothesis, we can derive the type of
$[[e1]]$ as $[[t1]]$ by $[[G |- e1:t1]]$. By Lemma \ref{lem:appendix:unired}, there is a unique
$[[t2]]$ such that $[[t1 --> t2]]$ or such $[[t2]]$ does not exist. 

If such $[[t2]]$ exists and its sorts is
$[[star]]$, we find the unique type of $[[e]]$ is $[[t2]]$ and can conclude $[[G |- e:t2]]$. Otherwise, we
report $[[e]]$ fails to type check.
	\end{description}
\end{proof}

\subsection{Type Safety}
\begin{dfn}[Multi-step reduction]
    The relation $[[->>]]$ is the transitive and reflexive closure of
$[[-->]]$.
\end{dfn}

\begin{dfn}[$n$-step reduction]
    The $n$-step reduction is denoted by $[[e0]] [[-->>]] [[en]]$, if
    there exists a sequence of one-step reductions $[[e0]] [[-->]]
    [[e1]] [[-->]] [[e2]] [[-->]] \dots [[-->]] [[en]]$, where $n$ is
    a positive integer and $[[ei]]\,(i=0,1,\dots,n)$ are valid
    expressions.
\end{dfn}

\begin{thm}[Subject Reduction]
If $[[G |- e:T]]$ and $[[e]] [[->>]] e'$ then $[[G |- e':T]]$.
\end{thm}

\begin{proof}
    We prove the case for one-step reduction, i.e., $[[e --> e']]$. The theorem
follows by induction on the number of one-step reductions of $[[e]] [[->>]]
[[e']]$.
    The proof is by induction with respect to the definition of one-step
reduction $[[-->]]$ as follows:
    \begin{description}
        \item[Case $\ottdruleSXXBeta{}$:] $\quad$ \\
        Suppose $[[G |- (\x:t1.e1)e2 :T]]$ and $[[G |- e1 [x |-> e2] :T']]$. By
Lemma \ref{lem:appendix:gen}(2), there exist expressions $[[t1']]$ and $[[t2]]$
such that 
        \begin{align}
            &[[G |- (\x:t1.e1):(Pi x:t1'.t2)]] \label{equ:lam} \\
            &[[G |- e2:t1']] \nonumber \\
            &[[T =a t2 [x |-> e2] ]] \nonumber
        \end{align}
        By Lemma \ref{lem:appendix:gen}(3), the judgement (\ref{equ:lam})
implies that there exists an expression $[[t2']]$ such that
        \begin{align}
            &[[Pi x:t1'.t2 =a Pi x:t1.t2']] \label{equ:lameq}\\
            &[[G, x:t1 |- e1:t2']] \nonumber
        \end{align}
        Hence, by (\ref{equ:lameq}) we have $[[t1 =a t1']]$ and $[[t2 =a
t2']]$. Then we can obtain $[[G, x:t1 |- e1:t2]]$ and $[[G |- e2:t1]]$. By
Lemma \ref{lem:appendix:subst}, we have $[[G |- e1[x |-> e2] : t2[x |-> e2]
]]$. Therefore, we conclude with $[[T' =a t2[x |-> e2] ]] [[=a]] [[T]]$.
        
        \item[Case $\ottdruleSXXApp{}$:] $\quad$ \\
        Suppose $[[G |- e1 e2 :T]]$ and $[[G |- e1' e2 :T']]$. By Lemma
\ref{lem:appendix:gen}(2), there exist expressions $[[t1]]$ and $[[t2]]$ such
that 
        \begin{align*}
            &[[G |- e1:(Pi x:t1.t2)]] \\
            &[[G |- e2:t1]]\\
            &[[T =a t2 [x |-> e2] ]]
        \end{align*}
        By the induction hypothesis, we have $[[G |- e1':(Pi x:t1.t2)]]$. By rule
\ruleref{T\_App}, we obtain $[[G |- e1' e2 : t2[x |-> e2] ]]$. Therefore, $[[T'
=a t2[x |-> e2] ]] [[=a]] [[T]]$.
        
        \item[Case $\ottdruleSXXCastDown{}$:] $\quad$ \\
        Suppose $[[G |- castdown e :T]]$ and $[[G |- castdown e' :T']]$. By
Lemma \ref{lem:appendix:gen}(7), there exist expressions $[[t1]], [[t2]]$ such
that 
        \begin{align*}
            &[[G |- e:t1]] \qquad [[G |- t2:s]] \\
            &[[t1 --> t2]] \qquad [[T =a t2 ]]
        \end{align*}
        By the induction hypothesis, we have $[[G |- e':t1]]$. By rule
\ruleref{T\_CastDown}, we obtain $[[G |- castdown e' : t2 ]]$. Therefore, $[[T'
=a t2]] [[=a]] [[T]]$.
        
        \item[Case $\ottdruleSXXCastDownUp{}$:] $\quad$ \\
        Suppose $[[G |- castdown (castup [t1] e) :T]]$ and $[[G |- e :T']]$. By
Lemma \ref{lem:appendix:gen}(7), there exist expressions $[[t1']], [[t2]]$ such
that 
        \begin{align}
            &[[G |- (castup [t1] e):t1']] \label{equ:fold} \\
            &[[t1' --> t2]] \label{equ:foldeq1} \\
            &[[T =a t2 ]] \label{equ:foldeq4}
        \end{align}
        By Lemma \ref{lem:appendix:gen}(6), the judgement (\ref{equ:fold})
implies that there exists an expression $[[t2']]$ such that
        \begin{align}
            &[[G |- e:t2']] \label{equ:foldr} \\
            &[[t1 --> t2']] \label{equ:foldeq2} \\
            &[[t1' =a t1]] \label{equ:foldeq3}
        \end{align}
        By (\ref{equ:foldeq1}, \ref{equ:foldeq2}, \ref{equ:foldeq3}) and Lemma
\ref{lem:appendix:unired} we obtain $[[t2 =a t2']]$. From (\ref{equ:foldr}) we
have $[[T' =a t2' ]]$. Therefore, by (\ref{equ:foldeq4}), $[[T' =a t2' ]]
[[=a]] [[t2 =a T]]$.
        
        \item[Case $\ottdruleSXXMu{}$:] $\quad$ \\
        Suppose $[[G |- (mu x:t.e) :T]]$ and $[[G |- e[x |-> mu x:t.e] :T']]$.
By Lemma \ref{lem:appendix:gen}(5), we have $[[T =a t]]$ and $[[G, x:t |-
e:t]]$. Then we obtain $[[G |- (mu x:t.e) : t]]$. Thus by Lemma
\ref{lem:appendix:subst}, we have $[[G |- e[x |-> mu x:t.e] : t[x |-> mu x:t.e]
]]$.
        
        Note that $[[x]]:[[t]]$, i.e., the type of $[[x]]$ is $[[t]]$, then
$[[x]] \notin \FV([[t]])$ holds implicitly. Hence, by the definition of
substitution, we obtain $[[t[x |-> mu x:t.e] == t]]$. Therefore, $[[T' =a t[x
|-> mu x:t.e] ]] [[==]] [[t =a T]]$.
    \end{description}
\end{proof}

\begin{thm}[Progress]
If $[[empty |- e:T]]$ then either $[[e]]$ is a value $v$ or there exists $[[e]]'$
such that $[[e --> e']]$.
\end{thm}

\begin{proof}
    By induction on the derivation of $[[empty |- e:T]]$ as follows:
    \begin{description}
        \item[Case $[[e=x]]$:] Impossible, because the context is empty.
        \item[Case $[[e=v]]$:] Trivial, since $[[e]]$ is already a value that
has one of the following forms:
		\begin{inparaenum}[(1)]
		    \item $[[star]]$,
			\item $[[\x:t.e]]$,
			\item $[[Pi x:t1.t2]]$,
			\item $[[castup [t] e]]$.
		\end{inparaenum}
		\item[Case $[[e]]=[[e1 e2]]$:] By Lemma \ref{lem:appendix:gen}(2), there
exist expressions $[[t1]]$ and $[[t2]]$ such that $[[empty |- e1:(Pi x:t1.t2)]]$ and
$[[empty |-e2:t1]]$. Consider whether $[[e1]]$ is a value:
    		\begin{itemize}
    		    \item If $[[e1]]=v$, by Lemma \ref{lem:appendix:gen}(3), it must be a
$\lambda$-term such that $[[e1 == \x:t1.e1']]$ for some $[[e1']]$ satisfying
$[[empty |- e1':t2]]$. Then by rule \ruleref{S\_Beta}, we have $[[(\x:t1.e1') e2 -->
e1' [x |-> e2] ]]$. Thus, there exists $[[e' == e1' [x |-> e2] ]]$ such that
$[[e --> e']]$.
    		    \item Otherwise, by the induction hypothesis, there exists $[[e1']]$ such
that $[[e1 --> e1']]$. Then by rule \ruleref{S\_App}, we have $[[e1 e2 --> e1'
e2]]$. Thus, there exists $[[e' == e1' e2]]$ such that $[[e --> e']]$.
    		\end{itemize}
		\item[Case $[[e]]=[[castdown e1]]$:] By Lemma \ref{lem:appendix:gen}(7),
there exist expressions $[[t1]]$ and $[[t2]]$ such that $[[empty |- e1:t1]]$ and
$[[t1 --> t2]]$. Consider whether $[[e1]]$ is a value:
		     \begin{itemize}
    		    \item If $[[e1]]=v$, by Lemma \ref{lem:appendix:gen}(6), it must be a
$[[castup]]$-term such that $[[e1 == castup [t1] e1']]$ for some $[[e1']]$
satisfying $[[empty |- e1':t2]]$. Then by rule \ruleref{S\_CastDownUp}, we can obtain
$[[castdown (castup [t1] e1') --> e1']]$. Thus, there exists $[[e' == e1']]$
such that $[[e --> e']]$.
    		    \item Otherwise, by the induction hypothesis, there exists $[[e1']]$ such
that $[[e1 --> e1']]$. Then by rule \ruleref{S\_CastDown}, we have $[[castdown
e1 --> castdown e1']]$. Thus, there exists $[[e' == castdown e1']]$ such that
$[[e --> e']]$.
    		\end{itemize}
		\item[Case $[[e]]=[[mu x:t.e1]]$:] By rule \ruleref{S\_Mu}, there always
exists $[[e' == e1[x |-> mu x:t.e1] ]]$.
    \end{description}
\end{proof}

\section{Full Specification of Surface Language}
\subsection{Syntax}
See Figure \ref{fig:appendix:syntax}.
\begin{figure*}
\centering
\gram{\ottpgm\ottinterrule
\ottdecl\ottinterrule
\ottu\ottinterrule
\ottp\ottinterrule
\ottE\ottinterrule
\ottGs}
\begin{align*}
&\text{Syntactic Sugar} \\
&\ottsurfsugar % defined in otthelper.mng.tex
\end{align*}
\caption{Syntax of the surface language}
\label{fig:appendix:syntax}
\end{figure*}

\subsection{Expression Typing}
See Figure \ref{fig:appendix:typing}.

\subsection{Translation to the Core}
See Figure \ref{fig:appendix:translate}.

\section{Proofs about Surface Language}
\subsection{Type Safety of the Translation}

\begin{thm}[Type Safety of Expression Translation]
Given a surface language expression $[[E]]$ and context $[[Gs]]$, 
if $[[Gs |- E:A ~> e]]$, $[[Gs |- A:star ~> t]]$ and $[[|- Gs ~> G]]$, then
$[[G |- e:t]]$.
\end{thm}

\begin{proof}
    By induction on the derivation of $[[Gs |- E : A ~> e]]$. Suppose there is
a core language context $[[G]]$ such that $[[|- Gs ~> G]]$.
    \begin{description}
        \renewcommand{\hlmath}[1]{#1}
        \item[Case $\ottdruleTRXXAx{}$:] $\quad$ \\ Trivial. $[[e]] = [[t]] = [[star]]$ and
$[[G |- star:star]]$ holds by rule \ruleref{T\_Ax}.
        \item[Case $\ottdruleTRXXVar{}$:] $\quad$ \\ Trivial. By rule \ruleref{T\_Var}, we
have $[[|- Gs ~> G]]$, then $[[x]]:[[t]] [[elt]] [[G]]$ where $[[Gs |-
A:star~>t]]$.
        \item[Case \resizebox{.9\columnwidth}{!}{$\ottdruleTRXXApp{}$}:] $\quad$ \\ Suppose
            \[\begin{array}{l}
            [[Gs |- E1 E2 : A1[x |-> E2] ~> e1 e2]] \\
            [[Gs |- A1[x |-> E2] : star ~> t1 [x |-> e2] ]].
            \end{array} \]
            By induction
            hypothesis, we have 
            $
            [[G |- e1 : (Pi x:t2.t1)]],
            [[G |- e2:t2]],
            $
            where
            \[\begin{array}{l}
             [[Gs |- E1 : (Pi x:A2.A1) ~> e1]] \\
              [[Gs |- (Pi x:A2.A1) : star ~> (Pi x:t2.t1)]] \\
              [[Gs |- E2 : A2 ~> e2]] \\
              [[Gs |- A2 : star ~> t2]].
            \end{array}\] Thus by rule \ruleref{T\_App}, we can conclude $[[G |- e1 e2 : t1 [x |-> e2] ]]$.
        \item[Case $\ottdruleTRXXLam{}$:] $\quad$ \\ Suppose
            \[\begin{array}{l}
            [[Gs |- (\x:A1.E):(Pi x:A1.A2) ~> \x:t1.e]] \\ 
            [[Gs |- Pi x:A1.A2 : star ~> Pi x:t1.t2]].
            \end{array} \]
            By the induction hypothesis, we have 
            $
            [[G, x : t1 |- e:t2]],
            [[G |- Pi x:t1.t2 : star]]
            $
            where 
            \[
            \begin{array}{ll}
            [[Gs, x : A1 |- E : A2 ~> e]] & \\
            [[Gs |- A1 : star ~> t1]] & [[Gs |- A2 : star ~> t2]] \\
            [[Gs |- (Pi x:A1.A2) : s ~> Pi x:t1.t2]] &
            \end{array}
            \]
            Thus by rule \ruleref{T\_Lam}, we can conclude $[[G |- (\x:t1.e):(Pi x:t1.t2)]]$.
        \item[Case $\ottdruleTRXXPi{}$:] $\quad$ \\ Suppose 
                \[ [[Gs |- (Pi x:A1.A2):r ~> Pi x:t1.t2]]. \] 
            By the induction hypothesis, we have 
            $
                [[G |- t1 : star]], [[G, x : t1 |- t2 : star]]
            $
            where
            $
                [[Gs |- A1 : s ~> t1]], [[Gs, x: A1 |- A2 : r ~> t2]]
            $
            Thus by rule \ruleref{T\_Pi} we can conclude $[[G |- (Pi x:t1.t2) : star]]$.
        \item[Case $\ottdruleTRXXMu{}$:] $\quad$ \\ Suppose 
                \[\begin{array}{l}
                    [[Gs |- (mu x:A . E):A ~> mu x:t.e]] \\
                    [[Gs |- A : star ~> t]]. 
                \end{array}\]
            By the induction hypothesis, we have 
                \[ [[G, x : t |- e : t]],\text{ where }[[Gs, x:A |- E:A ~> e]]. \] 
            Thus by rule \ruleref{T\_Mu}, we can conclude $[[G |- (mu x:t.e) : t]]$.
        \item[Case \resizebox{.9\columnwidth}{!}{$\ottdruleTRXXCase{}$}:] $\quad$ \\ Suppose 
            \[\begin{array}{l}
                [[Gs |- case E1 of << p => E2>> : B ~> (unfoldnp e1) T <<e2>>]] \\
                [[Gs |- B : star ~> T]].
            \end{array}\]
            By the induction hypothesis, we have 
            \[\begin{array}{ll}
                [[Gs |- E1 : D@<<U>>n ~> e1]] &
                [[Gs |- D@<<U>>n : star ~> t1]] \\
                [[G |- e1 : t1]] &
                [[<< Gs |- p => E2 : D@<<U>>n -> B ~> e2 >>]]            
            \end{array}\]
            By rule \ruleref{TRpat\_Alt}, we have
            \begin{align*}
                [[p]] &[[==]] [[K <<x:A[<< u |-> U >>]>>]] \\
                [[<<e2>>]] &[[==]] [[<<\ <<x:t'>> .e>>]]
            \end{align*}
            where
            \[\begin{array}{ll}
                [[<<Gs |- E2 : B ~> e>>]] &
                [[<<G |- e : T>>]] \\
                [[<<Gs |- U : star ~> uu'>>]] &
                [[<<Gs |- A[<< u |-> U >>]:star ~> t[<<uu |-> uu'>>]>>]] \\
                [[t']] [[==]] [[ t[<<uu |-> uu'>>] ]]
            \end{array}\]
            By rule \ruleref{TRdecl\_Data}, we have $[[D]]  [[ == ]] \ottdeclD$. Thus,
            \[ [[t1]] [[==]] [[D]] [[<<uu'>>]]^n,\text{ where }[[<<G |- uu' : ro>>]].\] 
            Note that by operational semantics of \name, the following reduction sequence follows for $[[t1]]$:
            \begin{align*}
                [[D]] [[<<uu'>>]]^n~
                &[[-->]]~ \mathscale[0.7]{[[(\ <<u:ro>>n . (bb:star) -> << ((<<x : t[D |-> X][X |-> D]>>) -> bb) >> -> bb) ]][[<<uu'>>]]^n}\\
                &[[-->>]]~ [[(bb:star) -> << (<<x:t'>>) -> bb >> -> bb]]
            \end{align*}
            Then by
            rule \ruleref{T\_CastDown} and the definition of $n$-step cast operator, the
            type of $[[unfoldnp e1]]$ is \[ [[(bb:star) -> << (<<x:t'>>) -> bb >> -> bb]].\] Note
            that by rule \ruleref{T\_Lam}, $[[G |- e2 : (<<x:t'>>) -> T]]$. Therefore, by rule
            \ruleref{T\_App}, we can conclude $[[G |- (unfoldnp e1) T <<e2>> : T]]$.
    \end{description}
\end{proof}

\begin{figure*}
\renewcommand{\hlmath}[1]{}
\renewcommand{\ottdrulename}[1]{\textsc{\replace{#1}{TR}{TS}}}
\renewcommand{\ottcom}[1]{\text{\replace{#1}{translation}{typing}}}
\ottdefnctxtrans{}\ottinterrule
\ottdefnpgmtrans{}\ottinterrule
\ottdefndecltrans{}\ottinterrule % defined in otthelper.mng.tex
\ottdefnpattrans{}\ottinterrule
\ottdefnexprtrans{}
\caption{Typing rules of the surface language}
\label{fig:appendix:typing}
\end{figure*}

\begin{figure*}
\ottdefnctxtrans{}\ottinterrule
\ottdefnpgmtrans{}\ottinterrule
\ottdefndecltrans{}
\[\hlmath{\ottdecltrans}\]\ottinterrule % defined in otthelper.mng.tex
\ottdefnpattrans{}\ottinterrule
\ottdefnexprtrans{}
\caption{Translation rules of the surface language}
\label{fig:appendix:translate}
\end{figure*}


% \fi

\end{document}
