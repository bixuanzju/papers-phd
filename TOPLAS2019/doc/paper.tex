\documentclass[format=acmsmall, review=false, screen=true]{acmart}


% Metadata Information
\acmJournal{TWEB}
\acmVolume{9}
\acmNumber{4}
\acmArticle{39}
\acmYear{2010}
\acmMonth{3}
\copyrightyear{2009}
%\acmArticleSeq{9}

% Copyright
%\setcopyright{acmcopyright}
\setcopyright{acmlicensed}
%\setcopyright{rightsretained}
%\setcopyright{usgov}
%\setcopyright{usgovmixed}
%\setcopyright{cagov}
%\setcopyright{cagovmixed}

% DOI
\acmDOI{0000001.0000001}

% Paper history
% \received{February 2007}
% \received[revised]{March 2009}
% \received[accepted]{June 2009}



%% Some recommended packages.
\usepackage{booktabs}   %% For formal tables:
                        %% http://ctan.org/pkg/booktabs
\usepackage{subcaption} %% For complex figures with subfigures/subcaptions
                        %% http://ctan.org/pkg/subcaption

%% Bibliography style
\bibliographystyle{ACM-Reference-Format}
\citestyle{acmauthoryear}   %% For author/year citations
% \setcitestyle{aysep={}}


% Basics
\usepackage{fixltx2e}
\usepackage{url}
\usepackage{fancyvrb}
\usepackage{mdwlist}  % Miscellaneous list-related commands
\usepackage{xspace}   % Smart spacing
\usepackage{supertabular}

% https://www.nesono.com/?q=book/export/html/347
% Package for inserting TODO statements in nice colorful boxes - so that you
% won’t forget to fix/remove them. To add a todo statement, use something like
% \todo{Find better wording here}.
\usepackage{todonotes}

%% Math
\usepackage{bm}       % Bold symbols in maths mode

% http://tex.stackexchange.com/questions/114151/how-do-i-reference-in-appendix-a-theorem-given-in-the-body
\usepackage{thmtools, thm-restate}

%% Theoretical computer science
\usepackage{stmaryrd}
\usepackage{mathtools}  % For "::=" ( \Coloneqq )

%% Font
% \usepackage[euler-digits,euler-hat-accent]{eulervm}


%% Some recommended packages.
\usepackage{booktabs}   %% For formal tables:
                        %% http://ctan.org/pkg/booktabs
\usepackage{subcaption} %% For complex figures with subfigures/subcaptions
                        %% http://ctan.org/pkg/subcaption


\usepackage{ottalt}

\usepackage{comment}

% Hyper links
\usepackage{url}
\usepackage{
  nameref,%\nameref
  hyperref,%\autoref
}
\usepackage[capitalise]{cleveref}
% \hypersetup{
%    colorlinks,
%    citecolor=black,
%    filecolor=black,
%    linkcolor=blue,
%    urlcolor=black
% }


% Code highlighting
\usepackage{listings}

\lstset{%
  backgroundcolor=\color{white},
  basicstyle=\small\ttfamily,
  keywordstyle=\sffamily\bfseries,
  captionpos=none,
  columns=flexible,
  lineskip=-1pt,
  keepspaces=true,
  showspaces=false,               % show spaces adding particular underscores
  showstringspaces=false,         % underline spaces within strings
  showtabs=false,                 % show tabs within strings adding particular underscores
  breaklines=true,                % sets automatic line breaking
  breakatwhitespace=true,         % sets if automatic breaks should only happen at whitespace
  escapeinside={(*}{*)},
  literate={->}{{$\rightarrow$}}1 {Top}{{$\top$}}1 {=>}{{$\Rightarrow$}}1 {/\\}{{$\Lambda$}}1,
  tabsize=2,
  commentstyle=\color{purple}\ttfamily,
  stringstyle=\color{red}\ttfamily,
  sensitive=false
}

\lstdefinelanguage{sedel}{
  keywords={Int, String, this, trait, inherits, super, type, Trait, override, self, new, if, then, else, let, in},
  identifierstyle=\color{black},
  morecomment=[l]{--},
  morecomment=[l]{//},
  morestring=[b]",
  xleftmargin  = 3mm,
  morestring=[b]'
}

\lstdefinelanguage{gbeta}{%
  language     = java,
  morekeywords = {virtual,refine},
  xleftmargin  = 3mm
}

\lstset{language=sedel}

\theoremstyle{remark}
\newtheorem{observation}{Observation}

\inputott{ott-rules}

% Code highlighting
\usepackage{listings}
\lstset{%
  basicstyle=\ttfamily\small, % the size of the fonts that are used for the code
  keywordstyle=\sffamily\bfseries,
  captionpos=none,
  columns=flexible,
  lineskip=-1pt,
  keepspaces=true,
  showspaces=false,               % show spaces adding particular underscores
  showstringspaces=false,         % underline spaces within strings
  showtabs=false,                 % show tabs within strings adding particular underscores
  breaklines=true,                % sets automatic line breaking
  breakatwhitespace=true,         % sets if automatic breaks should only happen at whitespace
  escapeinside={(*}{*)},
  sensitive=true
}

\lstdefinelanguage{myhaskell}{
  language=haskell,
  morekeywords = {def},
  commentstyle=\color{red}\textit,
  literate={->}{{$\rightarrow$}}1 {Top}{{$\top$}}1 {?}{{$\star$}}1 {=>}{{$\Rightarrow$}}1 {forall}{{$\forall$}}1 {/\\}{{$\Lambda$}}1,
  xleftmargin  = 3mm,
}

\lstset{language=myhaskell}

\newcommand{\lst}[1]{\text{\lstinline$#1$}}

% Table
\usepackage{multirow}
\usepackage{tabularx}
\newcolumntype{Y}{>{\centering\arraybackslash}X}
\newcolumntype{Z}{>{\raggedleft\arraybackslash}X}

% Infer rules
\usepackage{mathpartir}
\newcommand{\rname}[1]{{\,\text{\scriptsize \textsc{#1}}}}
\newcommand{\rul}[1]{\textsc{#1}}

% Extra symbols
\usepackage{stmaryrd}

% Macros for math typesetting

%% Names
% \newcommand{\name}{{\bf $\lambda_{\mu}^{\eq}$}\xspace}

%% Symbols
\newcommand{\syndef}{$::=$}
\newcommand{\synor}{$\mid$}
\newcommand{\syneq}{$\triangleq$}
\newcommand{\header}[1]{\multicolumn{1}{l}{$\boxed{#1}$}}
\newcommand{\headercap}[2]{\multicolumn{1}{l}{$\boxed{#1}$\quad{#2}}}
\newcommand{\headercapm}[2]{\vspace{1pt}\raggedright \framebox{\mbox{$#1$}} \quad
  #2}
\newcommand{\headercapt}[2]{\framebox{\mbox{$#1$}} \quad #2}
\newcommand{\marker}[1]{\blacktriangleright_{#1}}

%% Arrows
\newcommand{\To}{\Rightarrow}
\newcommand{\Chk}{\Downarrow}
\newcommand{\Inf}{\Uparrow}
\newcommand{\Inst}{{inst}}
\newcommand{\Gen}{{gen}}
\newcommand{\redto}{\hookrightarrow}
\newcommand{\redton}{\hookrightarrow^*}
\newcommand{\eq}{\sim}
\newcommand{\lt}{\sqsubseteq}
\newcommand{\sugar}{\triangleq}
\newcommand{\trto}[1]{\rightsquigarrow{#1}}
\newcommand{\opt}[1]{}
\newcommand{\trtop}{\rightsquigarrow}

%% Styles
\newcommand{\kw}[1]{\operatorname{\mathbf{#1}}}
\newcommand{\var}{\mathit}
\newcommand{\fun}{\mathsf}

%% Constructs
\newcommand{\bind}[3]{#1 #2:#3.~}
\newcommand{\blam}{\bind \lambda}
\newcommand{\bmu}{\bind \mu}
\newcommand{\barr}[2]{(#1:#2) \to}

\newcommand{\bindv}[4][]{#2\,\overline{#3:#4}^{#1}.~}
\newcommand{\blamv}[3][]{\bindv[#1] \lambda {#2} {#3}}
\newcommand{\bmuv}[3][]{\bindv[#1] \mu {#2} {#3}}
\newcommand{\barrv}[3][]{\overline{#2:#3}^{#1} \to}

\newcommand{\eqlam}[2]{\lambda_{\eq}({#1} \eq {#2}).~}
\newcommand{\eqty}[2]{({#1} \eq {#2})\Rightarrow}
\newcommand{\eqapp}[2][]{\langle {#2} \rangle^{#1}}
\newcommand{\eqlamv}[3][]{\lambda_{\eq}\overline{{#2} \eq {#3}}^{#1}.~}
\newcommand{\eqtyv}[3][]{\overline{({#2} \eq {#3})}^{#1}\Rightarrow}

\newcommand{\bpi}{\bind \Pi}
\newcommand{\bpiv}[3][]{\bindv[#1] \Pi {#2} {#3}}

\newcommand{\fold}{\fun{fold}}
\newcommand{\unfold}{\fun{unfold}}

\newcommand{\cast}[2]{\langle #1 \hookrightarrow #2 \rangle ~}
\newcommand{\blame}[2]{\kw{blame}_{#1} {#2}}
\newcommand{\subst}[2]{[#1 \mapsto #2]}
\newcommand{\ctxsubst}[2]{[#1]#2}
\newcommand{\Subst}[2]{[#1 \Mapsto #2]}
\newcommand{\substv}[3][]{\overline{[#2 \mapsto #3]}^{#1}}

\newcommand{\triv}{\_}
\newcommand{\trivtm}{\bullet}
\newcommand{\er}[1]{|{#1}|}
\newcommand{\erf}[1]{\|{#1}\|}
\newcommand{\erlam}[1]{\lambda {#1}.~}
\newcommand{\ermu}[1]{\mu {#1}.~}
\newcommand{\ereqlam}{\lambda_\eq.~}

\newcommand{\genvar}{\widehat}
\newcommand{\genA}{\genvar{a}}
\newcommand{\genB}{\genvar{b}}
\newcommand{\genC}{\genvar{c}}
\newcommand{\typA} {\alpha}
\newcommand{\typB} {\beta}
\newcommand{\varA}{\alpha}
\newcommand{\varB}{\beta}

%% Context
\newcommand{\dctx}{\Psi}
\newcommand{\tctx}{\Gamma}
\newcommand{\sctx}{\Psi}
\newcommand{\sctxb}{\Psi'}
\newcommand{\ctxsplit}{\shortmid}
\newcommand{\ctxinit}{\varnothing}
\newcommand{\ctxl}{\Theta}
\newcommand{\ctxr}{\Delta}
\newcommand{\cctx}{\Omega}
\newcommand{\byuni}{\vdash}
\newcommand{\byinf}{\vdash}
\newcommand{\byminf}{\vdash_m}
\newcommand{\bylessp}{\vdash}
\newcommand{\byoinf}{\vdash^\lambda}
\newcommand{\byhinf}{\vdash^{\mathit{OL}}}
\newcommand{\byfinf}{\vdash^F}
\newcommand{\bypinf}{\vdash^{\mathit{B}}}
\newcommand{\infto}{\Rightarrow}
\newcommand{\bychk}{\vdash}
\newcommand{\byochk}{\vdash^\lambda}
\newcommand{\chkby}{\Leftarrow}
\newcommand{\byapp}{\vdash_\bullet}
\newcommand{\byall}{\vdash_\delta}
\newcommand{\bytar}{\vdash}
\newcommand{\byinst}{\vdash_\Inst}
\newcommand{\bygen}{\vdash_\Gen}
\newcommand{\bysub}{\vdash}
\newcommand{\bydsub}{\vdash}
\newcommand{\bycg}{\vdash}
\newcommand{\byrf}{\vdash}
\newcommand{\bywf}{\vdash}
\newcommand{\bywt}{\vDash}
\newcommand{\toctx}{\dashv \ctxl}
\newcommand{\toctxo}{\dashv \tctx}
\newcommand{\toctxr}{\dashv \ctxr}
\newcommand{\dpreinf}[1][]{\dctx {#1} \byinf}
\newcommand{\dprechk}[1][]{\dctx {#1} \bychk}
\newcommand{\dpreall}[1][]{\dctx {#1} \byall}
\newcommand{\dpreapp}[1][]{\dctx {#1} \byapp}
\newcommand{\dpreuni}[1][]{\dctx {#1} \byuni}
\newcommand{\dpretar}[1][]{\dtctx {#1} \bytar}
\newcommand{\dpreinst}[1][]{\dctx {#1} \byinst}
\newcommand{\dpregen}[1][]{\dctx {#1} \bygen}
\newcommand{\dprecg}[1][]{\dctx {#1} \bycg}
\newcommand{\dprewf}[1][]{\dctx {#1} \bywf}
\newcommand{\dprewt}[1][]{\dctx {#1} \bywt}
\newcommand{\tpreinf}[1][]{\dctx {#1} \byinf}
\newcommand{\fpreinf}[1][]{\tctx {#1} \byfinf}
\newcommand{\tprechk}[1][]{\tctx {#1} \bychk}
\newcommand{\tpreall}[1][]{\tctx {#1} \byall}
\newcommand{\tpreapp}[1][]{\tctx {#1} \byapp}
\newcommand{\tpreuni}[1][]{\tctx {#1} \byuni}
\newcommand{\tpretar}[1][]{\ttctx {#1} \bytar}
\newcommand{\tpreinst}[1][]{\tctx {#1} \byinst}
\newcommand{\tpregen}[1][]{\tctx {#1} \bygen}
\newcommand{\tprecg}[1][]{\tctx {#1} \bycg}
\newcommand{\tprewf}[1][]{\dctx {#1} \bywf}
\newcommand{\tprewt}[1][]{\tctx {#1} \bywt}
\newcommand{\tpresub}[1][]{\dctx {#1} \bysub}
\newcommand{\tpreconssub}[1][]{\dctx {#1} \bysub}
\newcommand{\tprematch}[1][]{\dctx {#1} \vdash}
\newcommand{\tpreglb}[1][]{\tctx {#1} \vdash}
\newcommand{\presub}[1][]{{#1} \bysub}
\newcommand{\dpresub}[1][]{\dctx {#1} \bydsub}
\newcommand{\wc}{\ \var{ctx}\ }
\newcommand{\exto}{\longrightarrow}
\newcommand{\cgto}{\longmapsto}
\newcommand{\rfto}{\rightsquigarrow}
\newcommand{\aeq}{\equiv_\alpha}
\newcommand{\uni}{\leqq}
\newcommand{\dsub}{\leq}
\newcommand{\tsub}{<:}
\newcommand{\tsuper}{{\color{BrickRed}~:>~}}
\newcommand{\tvarinst}{:\leqq}
\newcommand{\tsubeither}{<:_m}
\newcommand{\elet}[3]{\kw{let} {#1} = {#2} \kw{in} {#3}}
\newcommand{\gen}[2]{\overbar{{#1}({#2})}}
\newcommand{\agen}[2]{{#1}_{agen}({#2})}
\newcommand{\tgen}[2]{{#1}_{gen}({#2})}

\newcommand{\erase}[1]{\lfloor{#1}\rfloor}
\newcommand{\etaeq}{\rightsquigarrow_{\eta id}}

\newcommand{\match}{\triangleright}
\newcommand{\glb}{\sqcap}
\newcommand{\tconssub}{\lesssim}
\newcommand{\unif}{\lessapprox}
\newcommand{\mask}[2]{{#1}|_{#2}}
\newcommand{\bymask}{\vdash}
% \newcommand{\appto}{\Rightarrow\mathrel{\mkern-12mu}\Rightarrow}
\newcommand{\lessp}{\sqsubseteq}
\newcommand{\lesspp}{\sqsubseteq^{\mathit{B}}}
\newcommand{\pbccons}{\prec}
\newcommand{\unknown}{\star}

% algorithm subsitution
\newcommand{\as}{S}
% algorithm name supply
\newcommand{\an}{N}
\newcommand{\byalgo}{\vdash}
\newcommand{\byarrow}{\vdash^\to}
\newcommand{\ato}{\hookrightarrow}
\newcommand{\acons}{\cdot}
\newcommand{\ex}[1]{\setminus_{#1}}
\newcommand{\san}[3]{(\mathcal{S}_#1, \mathcal{A}_#2#3)}

% Primitives
\newcommand{\nat}{\mathsf{Int}}
\newcommand{\bool}{\mathsf{Bool}}
\newcommand{\float}{\mathsf{Float}}
\newcommand{\truee}{\mathsf{true}}
\newcommand{\tope}{\mathsf{Top}}
\newcommand{\pbc}{$\lambda\mathsf{B}$\xspace}


\newcommand{\overbar}[1]{\mkern 1.5mu\overline{\mkern-1.5mu#1\mkern-1.5mu}\mkern 1.5mu}
\newcommand{\agtconssub}{~\widetilde{\tsub}~}


\newcommand{\byhave}{\blacklozenge}

\newcommand{\byget}{\blacksquare}


\newcommand{\obb}{$\mathbf{FOb}^{?}_{<:}$}

\newcommand{\revision}[1]{{\color{Red}{#1}}}

\newcommand{\blamev}{\mathsf{blame}}
\newcommand{\diverge}{\Uparrow}
\newcommand{\reduce}{\Downarrow}

\newcommand{\static}{\mathcal{S}}
\newcommand{\gradual}{\mathcal{G}}
\newcommand{\gsubst}{S^\gradual}
\newcommand{\ssubst}{S^\static}
\newcommand{\psubst}{S^\mathcal{P}}
\newcommand{\erasetp}[1]{\lceil{#1}\rceil}

\newcommand{\ctxeq}[3]{#1 \backsimeq_{ctx} #2 : #3}
\newcommand{\ctxappro}[3]{#1 \preceq_{ctx} #2 : #3}
\newcommand{\defeq}{\triangleq}

\newcommand\subsetsim{\mathrel{\substack{
      \textstyle\sqsubset\\[-0.2ex]\textstyle\sim}}}

% ------------------------------------------------------
% ORIGINAL TYPING
% ------------------------------------------------------

\newcommand*{\OVar}{\inferrule{
            x : A \in \tctx
            }{
            \tctx \byoinf x \infto A
            }\rname{Var}}

\newcommand*{\ONat}{\inferrule{
            }{
            \tctx \byoinf n \infto \nat
            }\rname{Nat}}

\newcommand*{\OLamAnnA}{\inferrule{
            \tctx, x: A \byoinf e \infto B
            }{
            \tctx \byoinf (\blam x A e) \infto A \to B
            }\rname{LamAnn-I}}

\newcommand*{\OLamAnnB}{\inferrule{
            \tctx, x: A \byoinf e \chkby B
            }{
            \tctx \byoinf (\blam x A e) \chkby A \to B
            }\rname{LamAnn-C}}

\newcommand*{\OApp}{\inferrule{
            \tctx \byoinf e_1 \infto A_1 \to A_2
         \\ \tctx \byochk e_2 \chkby A_1
            }{
            \tctx \byoinf e_1 ~ e_2 \infto A_2
            }\rname{App}}

\newcommand*{\OLamB}{\inferrule{
            \tctx, x: A \byochk e \chkby B
            }{
            \tctx \byochk \erlam x e \chkby A \to B
            }\rname{Lam-C}}

\newcommand*{\OSub}{\inferrule{
            \tctx \byoinf e \infto A
            }{
            \tctx \byochk e \chkby A
            }\rname{Sub}}

% ------------------------------------------------------
% TYPING
% ------------------------------------------------------

\newcommand*{\Var}{\inferrule{
            x : A \in \tctx
            }{
            \tctx \byinf x \infto A
            \trto{x}
            }\rname{Var}}

\newcommand*{\Nat}{\inferrule{
            }{
            \tctx \byinf n \infto \nat
            \trto{n}
            }\rname{Nat}}

\newcommand*{\LamAnnA}{\inferrule{
            \tctx, x: A \byinf e \infto B \trto {e'}
            }{
            \tctx \byinf (\blam x A e) \infto A \to B
            \trto{\blam x A {e'}}
            }\rname{LamAnn-I}}

\newcommand*{\LamAnnB}{\inferrule{
            C \match A_1 \to B
         \\ A = A_1 \glb A_2
         \\ \tctx, x: A \byinf e \chkby B \trto {e'}
            }{
            \tctx \bychk (\blam x {A_2} e) \chkby C
            \trto{\cast{A \to B}{C}(\blam x {A} {e'})}
            }\rname{LamAnn-C}}

\newcommand*{\App}{\inferrule{
            \tctx \byinf e_1 \infto A \trto{e_1'}
         \\ A \match A_1 \to A_2
         \\ \tctx \bychk e_2 \chkby A_1 \trto{e_2'}
            }{
            \tctx \byinf e_1 ~ e_2 \infto A_2
            \trto {(\cast{A}{A_1 \to A_2} e_1') ~ e_2'}
            }\rname{App}}

\newcommand*{\LamB}{\inferrule{
            C \match A \to B
         \\ \tctx, x: A \bychk e \chkby B \trto{e_1'}
            }{
            \tctx \bychk \erlam x e \chkby C
            \trto{\cast{A \to B}{\erlam x e}}
            }\rname{Lam-C}}

\newcommand*{\Sub}{\inferrule{
            e \neq (\blam x C e')
         \\ \tctx \byinf e \infto B \trto{e'}
         \\ B \sim A
            }{
            \tctx \bychk e \chkby A
            \trto{\cast B A e'}
            }\rname{Sub}}

% ------------------------------------------------------
% CAST CALCULUS
% ------------------------------------------------------

\newcommand*{\CaVar}{\inferrule{
            x : A \in \tctx
            }{
            \tpreinf x : A
            }\rname{C-Var}}

\newcommand*{\CaNat}{\inferrule{
            }{
            \tctx \byinf n : \nat
            }\rname{Nat}}

\newcommand*{\CaLam}{\inferrule{
            \tctx, x: A \byinf e : B
            }{
            \tpreinf \blam x A e : A \to B
            }\rname{C-Lam}}

\newcommand*{\CaApp}{\inferrule{
            \tpreinf e_1 : A \to B
         \\ \tpreinf e_2 : A
            }{
            \tpreinf e_1 ~ e_2 : B
            }\rname{C-App}}

\newcommand*{\CaCast}{\inferrule{
            \tpreinf e : A
         \\ B \sim A
            }{
            \tpreinf \cast A B e : B
            }\rname{C-Cast}}

\newcommand*{\CaBlame}{\inferrule{
            }{
            \tpreinf \blame A l : A
            }\rname{C-Blame}}

% ------------------------------------------------------
% Matching
% ------------------------------------------------------

\newcommand*{\MA}{\inferrule{}{
            (A_1 \to A_2) \match (A_1 \to A_2)
            }}

\newcommand*{\MB}{\inferrule{}{
            \unknown \match \unknown \to \unknown
            }}

\newcommand*{\MMA}{\inferrule{ }{
            \tprematch (A_1 \to A_2) \match (A_1 \to A_2)
            }\rname{M-Arr}}

\newcommand*{\MMB}{\inferrule{ }{
            \tprematch \unknown \match \unknown \to \unknown
            }\rname{M-Unknown}}

\newcommand*{\MMC}{\inferrule{
            \tprewf \tau
         \\ \tprematch A \subst a \tau \match A_1 \to A_2
            }{
            \tprematch \forall a. A \match A_1 \to A_2
            }\rname{M-Forall}}


% ------------------------------------------------------
% Matching (Algorithmic)
% ------------------------------------------------------

\newcommand*{\AMMA}{\inferrule{ }{
            \Gamma \vdash (A_1 \to A_2) \match (A_1 \to A_2) \toctxo
            }\rname{AM-Arr}}

\newcommand*{\AMMB}{\inferrule{ }{
            \Gamma \vdash \unknown \match \unknown \to \unknown \toctxo
            }\rname{AM-Unknown}}

\newcommand*{\AMMC}{\inferrule{ \tctx, \genA \vdash A \subst a \genA \match A_1 \to A_2 \toctxr
            }{
            \Gamma \vdash \forall a. A \match A_1 \to A_2 \toctxr
            }\rname{AM-Forall}}

\newcommand*{\AMMD}{\inferrule{ }{
            \tctx[\genC] \vdash \genC \match \genA \to \genB \dashv \tctx[\genA, \genB, \genC = \genA \to \genB]
            }\rname{AM-Var}}



% ------------------------------------------------------
% Instantiation
% ------------------------------------------------------

\newcommand*{\InstLSolve}{\inferrule{ \tctx \bywf \tau}
            {\tctx, \genA, \tctx' \vdash \genA \unif \tau \dashv \tctx, \genA = \tau, \tctx'
            }\rname{InstLSolve}}

\newcommand*{\InstLSolveU}{\hlmath{\inferrule{ }
            {\tctx[\genA] \vdash \genA \unif \unknown \dashv \tctx[\genA]
            }\rname{InstLSolveU}}}

\newcommand*{\InstLReach}{\inferrule{ }
            {\tctx[\genA][\genB] \vdash \genA \unif \genB \dashv \tctx[\genA][\genB = \genA]
            }\rname{InstLReach}}

\newcommand*{\InstLArr}{\inferrule{ \tctx[\genA_2, \genA_1, \genA = \genA_1 \to \genA_2] \vdash A_1 \unif \genA_1 \toctx \\
             \ctxl \vdash \genA_2 \unif \ctxsubst{\ctxl}{A_2} \toctxr
            }
            {\tctx[\genA] \vdash \genA \unif A_1 \to A_2 \toctxr
            }\rname{InstLArr}}

\newcommand*{\InstLAllR}{\inferrule{ \tctx[\genA], b \vdash \genA \unif B \toctxr, b, \Delta'
            }
            {\tctx[\genA] \vdash \genA \unif \forall b . B \toctxr
            }\rname{InstLAllR}}


\newcommand*{\InstRSolve}{\inferrule{ \tctx \bywf \tau}
            {\tctx, \genA, \tctx' \vdash \tau \unif \genA \dashv \tctx, \genA = \tau, \tctx'
            }\rname{InstRSolve}}

\newcommand*{\InstRSolveU}{\hlmath{\inferrule{ }
            {\tctx[\genA] \vdash \unknown \unif \genA \dashv \tctx[\genA]
            }\rname{InstRSolveU}}}

\newcommand*{\InstRReach}{\inferrule{ }
            {\tctx[\genA][\genB] \vdash \genB \unif \genA \dashv \tctx[\genA][\genB = \genA]
            }\rname{InstRReach}}

\newcommand*{\InstRArr}{\inferrule{ \tctx[\genA_2, \genA_1, \genA = \genA_1 \to \genA_2] \vdash \genA_1 \unif A_1 \toctx \\
             \ctxl \vdash \ctxsubst{\ctxl}{A_2}  \unif \genA_2  \toctxr
            }
            {\tctx[\genA] \vdash A_1 \to A_2  \unif \genA \toctxr
            }\rname{InstRArr}}

\newcommand*{\InstRAllL}{\inferrule{ \tctx[\genA], \genB \vdash B \subst b \genB \unif \genA \toctxr
            }
            {\tctx[\genA] \vdash \forall b . B \unif \genA  \toctxr
            }\rname{InstRAllL}}



% ------------------------------------------------------
% Consistency
% ------------------------------------------------------

\newcommand*{\CA}{\inferrule{}{
            A \sim \unknown
            }}

\newcommand*{\CB}{\inferrule{}{
            \unknown \sim A
            }}

\newcommand*{\CC}{\inferrule{
            A_1 \sim B_1
         \\ A_2 \sim B_2
            }{
            A_1 \to A_2 \sim B_1 \to B_2
            }}

\newcommand*{\CD}{\inferrule{}{
            A \sim A
            }}

\newcommand*{\CE}{\inferrule{
            A \sim B
            }{
            \forall a. A \sim \forall a. B
            }}

% ------------------------------------------------------
% GREATEST LOWER BOUND
% ------------------------------------------------------

\newcommand*{\GA}{\inferrule{}{
            A \glb A = A
            }}

\newcommand*{\GB}{\inferrule{}{
            A \glb \unknown = \unknown \glb A = A
            }}

\newcommand*{\GC}{\inferrule{}{
            (A_1 \to A_2) \glb (B_1 \to B_2) = (A_1 \glb B_1) \to (A_2 \glb B_2)
            }}

\newcommand*{\GGA}{\inferrule{
            }{
            \tpreglb A \glb A = A
            }}

\newcommand*{\GGB}{\inferrule{
            }{
            \tpreglb A \glb \unknown = A
            }}

\newcommand*{\GGF}{\inferrule{
            A ~ is ~ G
            }{
            \tpreglb \unknown \glb A = A
            }}

\newcommand*{\GGG}{\inferrule{
            A ~ isnot~ G
            }{
            \tpreglb \unknown \glb A = \unknown
            }}

\newcommand*{\GGC}{\inferrule{
            \tpreglb[,a] A \glb B = C
            }{
            \tpreglb A \glb \forall a. B = C
            }}

\newcommand*{\GGD}{\inferrule{
         \\ \tpreglb[, a] A \glb B = C
            }{
            \tpreglb \forall a. A \glb B = \forall a. C
            }}

\newcommand*{\GGE}{\inferrule{
            \tpreglb A_1 \glb B_1 = C_1
         \\ \tpreglb A_2 \glb B_2 = C_2
            }{
            \tpreglb A_1 \to A_2 \glb B_1 \to B_2 = C_1 \to C_2
            }}

% ------------------------------------------------------
% MASK
% ------------------------------------------------------

\newcommand*{\MSUnknownL}{\inferrule{
            }{
            \tctx \bymask \mask \unknown B = \unknown
            }\rname{Mask-UnknownL}}

\newcommand*{\MSUnknownR}{\inferrule{
            }{
            \tctx \bymask \mask A \unknown = \unknown
            }\rname{Mask-UnknownR}}

\newcommand*{\MSForallL}{\inferrule{
            \tctx, a \bymask \mask A B = C
            }{
            \tctx \bymask \mask {\forall a. A} B  = \forall a. C
            }\rname{Mask-ForallL}}

\newcommand*{\MSForallR}{\inferrule{
            \tctx, b \bymask \mask A B = C
            }{
            \tctx \bymask \mask A {\forall b. B}  = C
            }\rname{Mask-ForallR}}

\newcommand*{\MSArrow}{\inferrule{
            \tctx \bymask \mask {A_1} {B_1} = {C_1}
         \\ \tctx \bymask \mask {A_2} {B_2} = {C_2}
            }{
            \tctx \bymask \mask {A_1 \to A_2} {B_1 \to B_2} = C_1 \to C_2
            }\rname{Mask-Arrow}}

\newcommand*{\MSNat}{\inferrule{
            }{
            \tctx \bymask \mask \nat \nat = \nat
            }\rname{Mask-Int}}

% ------------------------------------------------------
% CONSISTENT SUBTYPING
% ------------------------------------------------------

\newcommand*{\CSForallR}{\inferrule{
            \tpresub[,a] A \tconssub B
            }{
            \tpresub A \tconssub \forall a. B
            }\rname{CS-ForallR}}

\newcommand*{\CSForallL}{\inferrule{
            \dctx \bywf \tau
         \\ \tpreconssub A \subst a \tau \tconssub B
            }{
            \tpresub \forall a. A \tconssub B
            }\rname{CS-ForallL}}

\newcommand*{\CSFun}{\inferrule{
            \tpreconssub B_1 \tconssub A_1
         \\ \tpreconssub A_2 \tconssub B_2
            }{
            \tpreconssub A_1 \to A_2 \tconssub B_1 \to B_2
            }\rname{     CS-Fun}}

\newcommand*{\CSTVar}{\inferrule{
            a \in \dctx
            }{
            \tpreconssub a \tconssub a
            }\rname{CS-TVar}}

\newcommand*{\CSInt}{\inferrule{
            }{
            \tpreconssub \nat \tconssub \nat
            }\rname{CS-Int}}

\newcommand*{\CSUnknownL}{\inferrule{
            }{
            \tpreconssub \unknown \tconssub A
            }\rname{CS-UnknownL}}

\newcommand*{\CSUnknownR}{\inferrule{
            }{
            \tpreconssub A \tconssub \unknown
            }\rname{CS-UnknownR}}


% ------------------------------------------------------
% CONSISTENT SUBTYPING (Algorithmic)
% ------------------------------------------------------

\newcommand*{\ACSForallR}{\inferrule{ \tctx, a \vdash A \tconssub B \toctxr, a, \ctxl
            }{
            \Gamma \vdash A \tconssub \forall a. B \toctxr
            }\rname{ACS-ForallR}}

\newcommand*{\ACSForallL}{\inferrule{ \tctx, \genA \vdash A \subst a \genA \tconssub B \toctxr
            }{
            \Gamma \vdash \forall a. A \tconssub B \toctxr
            }\rname{ACS-ForallL}}

\newcommand*{\ACSFun}{\inferrule{\Gamma \vdash B_1 \tconssub A_1 \toctx \\
             \ctxl \vdash \ctxsubst{\ctxl}{A_2} \tconssub \ctxsubst{\ctxl}{B_2} \toctxr
            }{
            \Gamma \vdash A_1 \to A_2 \tconssub B_1 \to B_2 \toctxr
            }\rname{ACS-Fun}}

\newcommand*{\ACSTVar}{\inferrule{
            }{
            \tctx[a] \vdash a \tconssub a \dashv \tctx[a]
            }\rname{ACS-TVar}}

\newcommand*{\ACSExVar}{\inferrule{
            }{
            \tctx[\genA] \vdash \genA \tconssub \genA \dashv \tctx[\genA]
            }\rname{ACS-ExVar}}


\newcommand*{\ACSInt}{\inferrule{
            }{
            \Gamma \vdash \nat \tconssub \nat \toctxo
            }\rname{ACS-Int}}

\newcommand*{\ACSUnknownL}{\hlmath{\inferrule{
            }{
            \Gamma \vdash \unknown \tconssub A \toctxo
            }\rname{ACS-UnknownL}}}

\newcommand*{\ACSUnknownR}{\hlmath{\inferrule{
            }{
            \Gamma \vdash A \tconssub \unknown \toctxo
            }\rname{ACS-UnknownR}}}

\newcommand*{\AInstantiateL}{\inferrule{ \genA \notin \mathit{fv}(A) \\
             \tctx[\genA] \vdash \genA \unif A \toctxr
            }{
            \tctx[\genA] \vdash \genA \tconssub A \toctxr
            }\rname{ACS-InstantiateL}}

\newcommand*{\AInstantiateR}{\inferrule{ \genA \notin \mathit{fv}(A) \\
             \tctx[\genA] \vdash  A \unif \genA \toctxr
            }{
            \tctx[\genA] \vdash A \tconssub \genA  \toctxr
            }\rname{ACS-InstantiateR}}



% ------------------------------------------------------
% LESS PRECISION
% ------------------------------------------------------

\newcommand*{\LUnknown}{\inferrule{
            }{
            \unknown \lessp A
            }\rname{L-Unknown}}

\newcommand*{\LNat}{\inferrule{
            }{
            \nat \lessp \nat
            }\rname{L-Nat}}

\newcommand*{\LArrow}{\inferrule{
            A_1 \lessp B_1
         \\ A_2 \lessp B_2
            }{
            A_1 \to A_2 \lessp B_1 \to B_2
            }\rname{L-Arrow}}

\newcommand*{\LTVar}{\inferrule{
            }{
            a \lessp a
            }\rname{L-TVar}}

\newcommand*{\LForall}{\inferrule{
            A \lessp B
            }{
            \forall a. A \lessp \forall a. B
            }\rname{L-Forall}}

% Term level

\newcommand*{\LRefl}{\inferrule{
            }{
            e \lessp e
            }\rname{L-Refl}}

\newcommand*{\LAbsAnn}{\inferrule{
            A_1 \lessp A_2
         \\ e_1 \lessp e_2
            }{
            \blam x {A_1} {e_1} \lessp \blam x {A_2} {e_2}
            }\rname{L-LamAnn}}

\newcommand*{\LApp}{\inferrule{
            e_1 \lessp e_3
         \\ e_2 \lessp e_4
            }{
            e_1 ~ e_2 \lessp e_3 ~ e_4
            }\rname{L-App}}

% PBC Term level

\newcommand*{\LVar}{\inferrule{
            x : A \in \dctx_1
         \\ x : B \in  \dctx_2
            }{
            \dctx_1 \ctxsplit \dctx_2 \bylessp x \lesspp x
            }\rname{L-Var}}

\newcommand*{\LNatP}{\inferrule{
            }{
            \dctx_1 \ctxsplit \dctx_2 \bylessp n \lesspp n
            }\rname{L-Nat}}

\newcommand*{\LAbsAnnP}{\inferrule{
            A_1 \lessp A_2
         \\ \dctx_1, x: A_1 \ctxsplit \dctx_2, x: A_2 \bylessp e_1 \lesspp e_2
            }{
            \dctx_1 \ctxsplit \dctx_2 \bylessp \blam x {A_1} {e_1} \lesspp \blam x {A_2} {e_2}
            }\rname{L-LamAnn}}

\newcommand*{\LAppP}{\inferrule{
            \dctx_1 \ctxsplit \dctx_2 \bylessp e_1 \lesspp e_3
         \\ \dctx_1 \ctxsplit \dctx_2 \bylessp e_2 \lesspp e_4
            }{
            \dctx_1 \ctxsplit \dctx_2 \bylessp e_1 ~ e_2 \lesspp e_3 ~ e_4
            }\rname{L-App}}

\newcommand*{\LCast}{\inferrule{
            A_1 \lessp B_1
         \\ A_2 \lessp B_2
         \\ \dctx_1 \ctxsplit \dctx_2 \bylessp e_1 \lesspp e_2
            }{
            \dctx_1 \ctxsplit \dctx_2 \bylessp \cast {A_1} {A_2} {e_1} \lesspp \cast{B_1} {B_2} {e_2}
            }\rname{L-Cast}}

\newcommand*{\LCastL}{\inferrule{
            \dctx_1 \ctxsplit \dctx_2 \bylessp e_1 \lesspp e_2
         \\ \dctx_2 \bypinf e_2 : B
         \\ A_1 \lessp B
         \\ A_2 \lessp B
            }{
            \dctx_1 \ctxsplit \dctx_2 \bylessp \cast {A_1} {A_2} {e_1} \lesspp {e_2}
            }\rname{L-CastL}}

\newcommand*{\LCastR}{\inferrule{
            \dctx_1 \ctxsplit \dctx_2 \bylessp e_1 \lesspp e_2
         \\ \dctx_1 \bypinf e_1 : A
         \\ A \lessp B_1
         \\ A \lessp B_2
            }{
            \dctx_1 \ctxsplit \dctx_2 \bylessp e_1 \lesspp \cast {B_1} {B_2} {e_2}
            }\rname{L-CastR}}

% Env level

\newcommand*{\LERefl}{\inferrule{
            }{
            \tctx \lessp \tctx
            }\rname{L-ERefl}}

\newcommand*{\LEPush}{\inferrule{
            \tctx_1 \lessp \tctx_2
         \\ A_1 \lessp A_2
            }{
            \tctx_1, x: A_1 \lessp \tctx_2, x:A_2
            }\rname{L-EPush}}

% ------------------------------------------------------
% ORIGINAL HIGHER-RANKED TYPE
% ------------------------------------------------------

\newcommand*{\HVar}{\inferrule{
            x : A \in \tctx
            }{
            \tctx \byinf x \infto A
            }\rname{Var}}

\newcommand*{\HNat}{\inferrule{
            }{
            \tpreinf n \infto \nat
            }\rname{Nat}}

\newcommand*{\HLamAnnA}{\inferrule{
            \tctx, x: A \byinf e \infto B
            }{
            \tctx \byinf (\blam x A e) \infto A \to B
            }\rname{LamAnn-I}}

\newcommand*{\HLamAnnB}{\inferrule{
            B \tsub A
         \\ \tctx, x: A \bychk e \chkby C
            }{
            \tctx \byinf (\blam x A e) \chkby B \to C
            }\rname{LamAnn-C}}

\newcommand*{\HApp}{\inferrule{
            \tctx \byinf e_1 \infto A
         \\ \tctx \byinf A \bullet e \appto B
            }{
            \tctx \byinf e_1 ~ e_2 \infto B
            }\rname{App}}

\newcommand*{\HAppPoly}{\inferrule{
            \tctx \byinf A \subst a \tau \bullet e \appto B
            }{
            \tctx \byinf \forall a. A \bullet e \appto B
            }\rname{AppPoly}}

\newcommand*{\HAppFun}{\inferrule{
            \tprechk e \chkby A_1
            }{
            \tctx \byinf A_1 \to A_2 \bullet e \appto A_2
            }\rname{AppFun}}

\newcommand*{\HSub}{\inferrule{
            \tctx \byinf e \infto A
         \\ \tpresub A \tsub B
            }{
            \tctx \bychk e \chkby B
            }\rname{Sub}}

\newcommand*{\HAll}{\inferrule{
            \tctx, a \bychk e \chkby A
            }{
            \tctx \bychk e \chkby \forall a. A
            }\rname{Forall}}

\newcommand*{\SForallR}{\inferrule{
            \tctx, a \bysub A \tsub B
            }{
            A \tsub \forall a. B
            }\rname{S-ForallR}}

\newcommand*{\SForallL}{\inferrule{
            \tctx \bywf \tau
         \\ \tctx \bysub A \subst a \tau \tsub B
            }{
            \tpresub \forall a. A \tsub B
            }\rname{S-ForallL}}

\newcommand*{\SFun}{\inferrule{
            \tpresub B_1 \tsub A_1
         \\ \tpresub A_2 \tsub B_2
            }{
            \tpresub A_1 \to A_2 \tsub B_1 \to B_2
            }\rname{S-Fun}}

\newcommand*{\STVar}{\inferrule{
            a \in \tctx
            }{
            \tctx \bysub a \tsub a
            }\rname{S-TVar}}

\newcommand*{\SInt}{\inferrule{
            }{
            \tctx \bysub \nat \tsub \nat
            }\rname{S-Int}}

% ------------------------------------------------------
% GRADUAL HIGHER-RANKED TYPE
% ------------------------------------------------------

\newcommand*{\HRVar}{\inferrule{
            x : A \in \tctx
            }{
            \tctx \byinf x \infto A
            }\rname{Var}}

\newcommand*{\HRNat}{\inferrule{
            }{
            \tpreinf n \infto \nat
            }\rname{Nat}}

\newcommand*{\HRLamAnnA}{\inferrule{
            \tctx, x: A \byinf e \infto B
            }{
            \tctx \byinf \blam x A e \infto A \to B
            }\rname{LamAnn-I}}

\newcommand*{\HRLamAnnB}{\inferrule{
            C \match A_1 \to B
         \\ \tpresub A_1 \tconssub A_2
         \\ A = A_2 \glb A_1
         \\ \tctx, x: A \bychk e \chkby B
            }{
            \tctx \byinf \blam x {A_2} e \chkby C
            }\rname{LamAnn-C}}

\newcommand*{\HRApp}{\inferrule{
            \tctx \byinf e_1 \infto A
         \\ A \match A_1 \to A_2
         \\ \tctx \bychk e_2 \chkby A_1
            }{
            \tctx \byinf e_1 ~ e_2 \infto A_2
            }\rname{App}}

\newcommand*{\HRSub}{\inferrule{
            e \neq (\blam x C e')
         \\ \tctx \byinf e \infto A
         \\ \tpresub A \tconssub B
            }{
            \tctx \bychk e \chkby B
            }\rname{Sub}}

\newcommand*{\HRAll}{\inferrule{
            \tctx, a \bychk e \chkby A
            }{
            \tctx \bychk e \chkby \forall a. A
            }\rname{Forall}}

\newcommand*{\HSForallR}{\inferrule{
            \tpresub[,a] A \tsub B
            }{
            \tpresub A \tsub \forall a. B
            }\rname{S-ForallR}}

\newcommand*{\HSForallL}{\inferrule{
            \dctx \bywf \tau
         \\ \tpresub A \subst a \tau \tsub B
            }{
            \tpresub \forall a. A \tsub B
            }\rname{S-ForallL}}

\newcommand*{\HSFun}{\inferrule{
            \tpresub B_1 \tsub A_1
         \\ \tpresub A_2 \tsub B_2
            }{
            \tpresub A_1 \to A_2 \tsub B_1 \to B_2
            }\rname{S-Fun}}

\newcommand*{\HSTVar}{\inferrule{
            a \in \dctx
            }{
            \tpresub a \tsub a
            }\rname{S-TVar}}

\newcommand*{\HSInt}{\inferrule{
            }{
            \tpresub \nat \tsub \nat
            }\rname{S-Int}}

\newcommand*{\HSUnknown}{\inferrule{
            }{
            \tpresub \unknown \tsub \unknown
            }\rname{S-Unknown}}

% ------------------------------------------------------
% GRADUAL HIGHER-RANKED TYPING : DECLARATIVE
% ------------------------------------------------------

\newcommand*{\DVar}{\inferrule{
            x : A \in \dctx
            }{
            \dctx \byinf x : A
            \trto {x}
            }\rname{Var}}

\newcommand*{\DNat}{\inferrule{
            }{
            \tpreinf n : \nat
            \trto {n}
            }\rname{Nat}}

\newcommand*{\DLamAnnA}{\inferrule{
            \dctx, x: A \byinf e : B
            \trto {s}
            }{
            \dctx \byinf \blam x A e : A \to B
            \trto {\blam x A s}
            }\rname{LamAnn}}

\newcommand*{\DApp}{\inferrule{
            \dctx \byinf e_1 : A
            \trto {s_1}
         \\ \dctx \byinf A \match A_1 \to A_2
         \\ \dctx \byinf e_2 : A_3
            \trto {s_2}
         \\ \tpreconssub A_3 \tconssub A_1
            }{
            \dctx \byinf e_1 ~ e_2 : A_2
            \trto {(\cast A {A_1 \to A_2} s_1) ~
            (\cast {A_3} {A_1} s_2)
            }
            }\rname{App}}


% ------------------------------------------------------
% GRADUAL HIGHER-RANKED TYPING : Algorithmic
% ------------------------------------------------------

\newcommand*{\AVar}{\inferrule{
            (x : A) \in \tctx
            }{
            \Gamma \vdash x \infto A \toctxo
            }\rname{AVar}}

\newcommand*{\ANat}{\inferrule{
            }{
            \Gamma \vdash n \infto \nat \toctxo
            }\rname{ANat}}

\newcommand*{\ALamAnnA}{\inferrule{
            \tctx, x: A \byinf e \infto B \toctxr,  x : A , \ctxl
            }{
            \tctx \byinf \blam x A e \infto A \to B \toctxr
            }\rname{ALamAnnA}}

\newcommand*{\AApp}{\inferrule{
            \Gamma \vdash e_1 \infto A \dashv \ctxl_1
         \\ \hlmath{\ctxl_1 \byinf \ctxsubst{\ctxl_1}{A} \match A_1 \to A_2 \dashv \ctxl_2}
         \\ \ctxl_2 \byinf e_2 \infto A_3 \dashv \ctxl_3
         \\ \ctxl_3 \bysub \ctxsubst{\ctxl_3}{A_3} \tconssub \ctxsubst{\ctxl_3}{A_1} \toctxr
            }{
            \Gamma \vdash e_1 ~ e_2 \infto A_2 \toctxr
            }\rname{AApp}}


% ------------------------------------------------------
% Context extension
% ------------------------------------------------------


\newcommand*{\ExtID}{\inferrule{
            }{
            \ctxinit \exto \ctxinit
            }\rname{ExtID}}

\newcommand*{\ExtVar}{\inferrule{
              \Gamma \exto \Delta \\
              \ctxsubst{\Delta}{A} = \ctxsubst{\Delta}{A'}
            }{
            \Gamma, x : A \exto \Delta, x : A'
            }\rname{ExtVar}}

\newcommand*{\ExtUVar}{\inferrule{
              \Gamma \exto \Delta
            }{
            \Gamma, a \exto \Delta, a
            }\rname{ExtUVar}}

\newcommand*{\ExtEVar}{\inferrule{
              \Gamma \exto \Delta
            }{
            \Gamma, \genA \exto \Delta, \genA
            }\rname{ExtEVar}}

\newcommand*{\ExtSolved}{\inferrule{
              \Gamma \exto \Delta \\
              \ctxsubst{\Delta}{\tau} = \ctxsubst{\Delta}{\tau'}
            }{
            \Gamma, \genA = \tau \exto \Delta, \genA = \tau'
            }\rname{ExtSolved}}

\newcommand*{\ExtSolve}{\inferrule{
              \Gamma \exto \Delta
            }{
            \Gamma, \genA \exto \Delta, \genA = \tau
            }\rname{ExtSolve}}

\newcommand*{\ExtAdd}{\inferrule{
              \Gamma \exto \Delta
            }{
            \Gamma \exto \Delta, \genA
            }\rname{ExtAdd}}

\newcommand*{\ExtAddS}{\inferrule{
              \Gamma \exto \Delta
            }{
            \Gamma \exto \Delta, \genA = \tau
            }\rname{ExtAddSolved}}



% ------------------------------------------------------
% HIGHER-RANKED TYPING : NON-BI
% ------------------------------------------------------

\newcommand*{\NVar}{\inferrule{
            x : A \in \dctx
            }{
            \dctx \byhinf x : A
            }\rname{Var}}

\newcommand*{\NNat}{\inferrule{
            }{
            \dctx \byhinf n : \nat
            }\rname{Nat}}

\newcommand*{\NLamAnnA}{\inferrule{
            \dctx, x: A \byhinf e : B
            }{
            \dctx \byhinf \blam x A e : A \to B
            }\rname{LamAnn}}

\newcommand*{\NApp}{\inferrule{
            \dctx \byhinf e_1 : A_1 \to A_2
         \\ \dctx \byhinf e_2 : A_1
            }{
            \dctx \byhinf e_1 ~ e_2 : A_2
            }\rname{App}}

\newcommand*{\NSub}{\inferrule{
            \dctx \byhinf e : A_1
         \\ \tpresub A_1 \tsub A_2
            }{
            \dctx \byhinf e : A_2
            }\rname{Sub}}

\newcommand*{\NForallR}{\inferrule{
            \tpresub[,a] A \tsub B
            }{
            \tpresub A \tsub \forall a. B
            }\rname{ForallR}}

\newcommand*{\NForallL}{\inferrule{
            \dctx \bywf \tau
         \\ \tpresub A \subst a \tau \tsub B
            }{
            \tpresub \forall a. A \tsub B
            }\rname{ForallL}}

\newcommand*{\NFun}{\inferrule{
            \tpresub B_1 \tsub A_1
         \\ \tpresub A_2 \tsub B_2
            }{
            \tpresub A_1 \to A_2 \tsub B_1 \to B_2
            }\rname{CS-Fun}}

\newcommand*{\NTVar}{\inferrule{
            a \in \dctx
            }{
            \tpresub a \tsub a
            }\rname{CS-TVar}}

\newcommand*{\NSInt}{\inferrule{
            }{
            \tpresub \nat \tsub \nat
            }\rname{CS-Int}}


% ------------------------------------------------------
% MASK OFF
% ------------------------------------------------------

\newcommand*{\FA}{\inferrule{
            }{
            \mask A \unknown = \unknown
            }\rname{F-StarR}}

\newcommand*{\FB}{\inferrule{
            }{
            \mask \unknown A = \unknown
            }\rname{F-StarL}}

\newcommand*{\FC}{\inferrule{
            }{
            \mask {\forall a. A} B = \mask A B
            }\rname{F-ForallL}}

\newcommand*{\FD}{\inferrule{
            }{
            \mask A {\forall a. B} = \mask A B
            }\rname{F-ForallR}}

\newcommand*{\FE}{\inferrule{
            }{
            \mask {A_1 \to A_2} {B_1 \to B_2} = \mask {A_1} {B_1} \to \mask {A_2} {B_2}
            }\rname{F-Fun}}

% ------------------------------------------------------
% PBC
% ------------------------------------------------------

\newcommand*{\PBCVar}{\inferrule{
            x : A \in \tctx
            }{
            \tctx \bypinf x \infto A
            }\rname{PBC-Var}}

\newcommand*{\PBCNat}{\inferrule{
            }{
            \tctx \bypinf n \infto \nat
            }\rname{PBC-Var}}

\newcommand*{\PBCApp}{\inferrule{
            \tctx \bypinf e_1 \infto A_1 \to A_2
         \\ \tctx \bypinf e_2 \infto A_1
            }{
            \tctx \bypinf e_1 ~ e_2 \infto A_2
            }\rname{PBC-App}}

\newcommand*{\PBCLam}{\inferrule{
            \tctx, x: A \bypinf e \infto B
            }{
            \tctx \bypinf \blam x A e \infto A \to B
            }\rname{PBC-Lam}}

\newcommand*{\PBCBLam}{\inferrule{
            \tctx, X \bypinf e \infto A
            }{
            \tctx \bypinf \Lambda X. e \infto \forall X. A
            }\rname{PBC-BLam}}

\newcommand*{\PBCTApp}{\inferrule{
            \tctx, X \bypinf e \infto  \forall X. A
            }{
            \tctx \bypinf e ~ [B] \infto A \subst X B
            }\rname{PBC-TApp}}

\newcommand*{\PBCCast}{\inferrule{
            \tctx \bypinf e \infto A
         \\ A \pbccons B
            }{
            \tctx \bypinf \cast A B e \infto B
            }\rname{PBC-TApp}}

% ------------------------------------------------------
% PBC Compatibility
% ------------------------------------------------------

\newcommand*{\CompRefl}{\inferrule{
            }{
            A \pbccons A
            }\rname{Comp-Refl}}

\newcommand*{\CompUnknownR}{\inferrule{
            }{
            A \pbccons \unknown
            }\rname{Comp-UnknownR}}

\newcommand*{\CompUnknownL}{\inferrule{
            }{
            \unknown \pbccons A
            }\rname{Comp-UnknownL}}

\newcommand*{\CompArrow}{\inferrule{
            A_1 \pbccons B_1
         \\ A_2 \pbccons B_2
            }{
            A_1 \to A_2 \pbccons B_1 \to B_2
            }\rname{Comp-Arrow}}

\newcommand*{\CompAllR}{\inferrule{
            A \pbccons B
            }{
            A \pbccons \forall X. B
            }\rname{Comp-AllR}}

\newcommand*{\CompAllL}{\inferrule{
            A \subst X \star \pbccons B
            }{
            \forall X. A \pbccons B
            }\rname{Comp-AllL}}

% ------------------------------------------------------
% EXTENSION
% ------------------------------------------------------

\newcommand*{\SubTop}{\inferrule{
            A ~ static
            }{
            \dctx \bysub A \tsub \top
            }\rname{S-Top}}

\newcommand*{\CTop}{\inferrule{}{
            \top \sim \top
            }}

\newcommand*{\CSTop}{\inferrule{
            }{
            \dctx \bysub A \tconssub \top
            }\rname{CS-Top}}


% ------------------------------------------------------
% Well-formedess of type under declarative context
% ------------------------------------------------------

\newcommand*{\DeclVarWF}{\inferrule{
              a \in \dctx
            }{
            \dctx \vdash a
            }\rname{DeclVarWF}}

\newcommand*{\DeclIntWF}{\inferrule{
            }{
            \dctx \vdash \nat
            }\rname{DeclIntWF}}

\newcommand*{\DeclUnknownWF}{\inferrule{
            }{
            \dctx \vdash \unknown
            }\rname{DeclUnknownWF}}


\newcommand*{\DeclFunWF}{\inferrule{
              \dctx \vdash A \\ \dctx \vdash B
            }{
            \dctx \vdash A \to B
            }\rname{DeclFunWF}}

\newcommand*{\DeclForallWF}{\inferrule{
              \dctx, a \vdash A
            }{
            \dctx \vdash \forall a. A
            }\rname{DeclForallWF}}

% ------------------------------------------------------
% Well-formedess of type under algorithmic context
% ------------------------------------------------------

\newcommand*{\VarWF}{\inferrule{
            }{
            \Gamma[a] \vdash a
            }\rname{VarWF}}

\newcommand*{\IntWF}{\inferrule{
            }{
            \Gamma \vdash \nat
            }\rname{IntWF}}

\newcommand*{\UnknownWF}{\hlmath{\inferrule{
            }{
            \Gamma \vdash \unknown
            }\rname{UnknownWF}}}

\newcommand*{\FunWF}{\inferrule{
              \Gamma \vdash A \\ \Gamma \vdash B
            }{
            \Gamma \vdash A \to B
            }\rname{FunWF}}

\newcommand*{\ForallWF}{\inferrule{
              \Gamma, a \vdash A
            }{
            \Gamma \vdash \forall a. A
            }\rname{ForallWF}}

\newcommand*{\EVarWF}{\inferrule{
            }{
            \Gamma[\genA] \vdash \genA
            }\rname{EVarWF}}

\newcommand*{\SolvedEVarWF}{\inferrule{
            }{
            \Gamma[\genA = \tau] \vdash \genA
            }\rname{SolvedEVarWF}}

% ------------------------------------------------------
% OBJECTS: SUBTYPING
% ------------------------------------------------------

\newcommand*{\ObSInt}{\inferrule{}
            {
            \nat \tsub \nat
            }}

\newcommand*{\ObSBool}{\inferrule{}
            {
            \bool \tsub \bool
            }}

\newcommand*{\ObSFloat}{\inferrule{}
            {
            \float \tsub \float
            }}

\newcommand*{\ObSIntFloat}{\inferrule{}
            {
            \nat \tsub \float
            }}

\newcommand*{\ObFun}{\inferrule{B_1 \tsub A_1 \\ A_2 \tsub B_2}
            {
            A_1 \to A_2 \tsub B_1 \to B_2
            }}

\newcommand*{\ObSUnknown}{\inferrule{}
            {
            \unknown \tsub \unknown
            }}

\newcommand*{\ObSRecord}{\inferrule{}
            {
            [l_i : A_i^{i \in 1...n+m}] \tsub
            [l_i : A_i^{i \in 1...n}]
            }}

% ------------------------------------------------------
% OBJECTS: CONSISTENCY
% ------------------------------------------------------

\newcommand*{\ObCRefl}{\inferrule{}
            {
            A \sim A
            }}

\newcommand*{\ObCUnknownR}{\inferrule{}
            {
            A \sim \unknown
            }}

\newcommand*{\ObCUnknownL}{\inferrule{}
            {
            \unknown \sim A
            }}

\newcommand*{\ObCC}{\inferrule{
            A_1 \sim B_1
         \\ A_2 \sim B_2
            }{
            A_1 \to A_2 \sim B_1 \to B_2
            }}


\newcommand*{\ObCRecord}{\inferrule{
            A_i \sim B_i
            }
            {
            [l_i: A_i] \sim [l_i:B_i]
            }}

\begin{document}

\title{Consistent Subtyping for All}


\author{Ningning Xie}
\affiliation{%
  \institution{The University of Hong Kong}
  \city{Hong Kong}
  \country{China}}
\email{nnxie@cs.hku.hk}
\author{Xuan Bi}
\affiliation{%
  \institution{The University of Hong Kong}
  \city{Hong Kong}
  \country{China}
}
\email{xbi@cs.hku.hk}
\author{Bruno C. d. S. Oliveira}
\affiliation{%
 \institution{The University of Hong Kong}
 \city{Hong Kong}
 \country{China}}
\email{bruno@cs.hku.hk}
\author{Tom Schrijvers}
\affiliation{%
 \institution{KU Leuven}
 \city{Leuven}
 \country{Belgium}}
\email{tom.schrijvers@cs.kuleuven.be}




\begin{abstract}
  Consistent subtyping is employed in some gradual type systems to validate type
  conversions. The original definition by \citeauthor{siek2007gradual} serves as
  a guideline for designing gradual type systems with subtyping. Polymorphic
  types \`a la System F also induce a subtyping relation that relates
  polymorphic types to their instantiations. However
  \citeauthor{siek2007gradual}'s definition is not adequate for polymorphic
  subtyping. The first goal of this paper is to propose a generalization of
  consistent subtyping that is adequate for polymorphic subtyping, and subsumes
  the original definition by \citeauthor{siek2007gradual}. The new definition of
  consistent subtyping provides novel insights with respect to previous
  polymorphic gradual type systems, which did not employ consistent subtyping.
  The second goal of this paper is to present a gradually typed calculus for
  implicit (higher-rank) polymorphism that uses our new notion of consistent
  subtyping. We develop both declarative and (bidirectional) algorithmic
  versions for the type system. The algorithmic version employs techniques
  developed by \citeauthor{dunfield2013complete} for higher-rank polymorphism to
  deal with instantiation. We prove that the new calculus satisfies all static
  aspects of the refined criteria for gradual typing. We also study an extension
  of the type system with static and gradual type parameters, in an attempt to
  support a variant of the dynamic criterion for gradual typing. Assuming a
  coherence conjecture for the extended calculus, we show that the dynamic
  gradual guarantee of our source language can be reduced to that of \pbc,
  which, at the time of writing, is still an open question. Most of the
  metatheory of this paper, except some manual proofs for the algorithmic type
  system and extensions, has been mechanically formalized using the Coq proof
  assistant.
\end{abstract}


% The code below should be generated by the tool at
% http://dl.acm.org/ccs.cfm
% Please copy and paste the code instead of the example below.

\begin{CCSXML}
<ccs2012>
<concept>
<concept_id>10011007.10011006.10011008.10011009.10011012</concept_id>
<concept_desc>Software and its engineering~Functional languages</concept_desc>
<concept_significance>500</concept_significance>
</concept>
<concept>
<concept_id>10011007.10011006.10011008.10011024.10011025</concept_id>
<concept_desc>Software and its engineering~Polymorphism</concept_desc>
<concept_significance>500</concept_significance>
</concept>
<concept>
<concept_id>10011007.10011006.10011039.10011311</concept_id>
<concept_desc>Software and its engineering~Semantics</concept_desc>
<concept_significance>500</concept_significance>
</concept>
</ccs2012>
\end{CCSXML}

\ccsdesc[500]{Software and its engineering~Functional languages}
\ccsdesc[500]{Software and its engineering~Polymorphism}
\ccsdesc[500]{Software and its engineering~Semantics}

% End generated code


\keywords{Gradual typing, implicit polymorphism, consistent subtyping, dynamic gradual guarantee}


\maketitle


%% -- Starting Point --


\section{Introduction}
\label{sec:introduction}

Compositionality is a desirable property in programming
designs. Broadly defined, it is the principle that a
system should be built by composing smaller subsystems. For instance,
in the area of programming languages, compositionality is
a key aspect of \emph{denotational semantics}~\cite{scott1971toward, scott1970outline}, where
the denotation of a program is constructed from the denotations of its parts.
% For example, the semantics for a language of simple arithmetic expressions
% is defined as:
% 
% \[\begin{array}{lcl}
% \llbracket n \rrbracket_{E} & = & n \\
% \llbracket e_1 + e_2 \rrbracket_{E} & = & \llbracket e_1 \rrbracket_E + \llbracket  e_2 \rrbracket_E \\
% \end{array}\]
% 
% \bruno{Replace E by fancier symbol?}
% Here there are two forms of expressions: numeric literals and
% additions. The semantics of a numeric literal is just the numeric
% value denoted by that literal. The semantics of addition is the
% addition of the values denoted by the two subexpressions.
Compositional definitions have many benefits.
One is ease of reasoning: since compositional
definitions are recursively defined over smaller elements they
can typically be reasoned about using induction. Another benefit
is that compositional definitions are easy to extend,
without modifying previous definitions.
% For example, if we also wanted to support multiplication,
% we could simply define an extra case:
% 
% \[\begin{array}{lcl}
% \llbracket e_1 * e_2 \rrbracket_E & = & \llbracket e_1 \rrbracket_E * \llbracket  e_2 \rrbracket_E \\
% \end{array}\]

Programming techniques that support compositional
definitions include:
\emph{shallow embeddings} of
Domain Specific Languages (DSLs)~\cite{DBLP:conf/icfp/GibbonsW14}, \emph{finally
  tagless}~\cite{CARETTE_2009}, \emph{polymorphic embeddings}~\cite{hofer_polymorphic_2008} or
\emph{object algebras}~\cite{oliveira2012extensibility}. These techniques allow us to create
compositional definitions, which are easy to extend without
modifications. Moreover, when modeling semantics, both finally tagless and object algebras
support \emph{multiple interpretations} (or denotations) of
syntax, thus offering a solution to the well-known \emph{Expression Problem}~\cite{wadler1998expression}.
Because of these benefits these techniques have become
popular both in the functional and object-oriented
programming communities.

However, programming languages often only support simple compositional designs
well, while support for more sophisticated compositional designs is lacking.
For instance, once we have multiple interpretations of syntax, we may wish to
compose them. Particularly useful is a \emph{merge} combinator,
which composes two interpretations~\cite{oliveira2012extensibility,
oliveira2013feature, rendel14attributes} to form a new interpretation that,
when executed, returns the results of both interpretations. 

% For example, consider another pretty printing interpretation (or
% semantics) $\llbracket \cdot \rrbracket_P$ for arithmetic expressions, which
% returns the string that denotes the concrete syntax of the
% expression. Using merge we can compose the two interpretations to
% obtain a new interpretation that executes both printing and evaluation:
% \jeremy{Explain what is $E\,\&\,P$?}
% 
% \[\begin{array}{lcl}
% \llbracket \cdot \rrbracket_E \otimes \llbracket \cdot \rrbracket_P & = & \llbracket \cdot \rrbracket_{E\,\&\,P} \\
% \end{array}\]

The merge combinator can be manually defined in existing programming languages,
and be used in combination with techniques such as finally tagless or object
algebras. Moreover variants of the merge combinator are useful to
model more complex combinations
of interpretations. A good example are so-called \emph{dependent} interpretations,
where an interpretation does not depend \emph{only} on itself, but also on 
a different interpretation. These definitions with dependencies are quite
common in practice, and, although they are not orthogonal to the interpretation they
depend on, we would like to model them (and also mutually dependent interpretations)
in a modular and compositional style.

% For example consider the following two
% interpretations ($\llbracket \cdot \rrbracket_{\mathsf{Odd}}$ and
% $\llbracket \cdot \rrbracket_{\mathsf{Even}}$) over Peano-style natural numbers:

% \[\begin{array}{lclclcl}
% \llbracket 0 \rrbracket_{\mathsf{Even}}  & = & \mathsf{True} & ~~~~~~~~~~~~~~~~~~~~ & \llbracket 0 \rrbracket_{\mathsf{Odd}} & = & \mathsf{False} \\
% \llbracket S~e \rrbracket_{\mathsf{Even}} & = & \llbracket e \rrbracket_{\mathsf{Odd}} & ~~ & \llbracket S~e \rrbracket_{\mathsf{Odd}} & = & \llbracket e \rrbracket_{\mathsf{Even}}\\
% \end{array}\]

% \emph{Are these interpretations compositional or not?} Under
% a strict definition of compositionality they are not because
% the interpretation of the parts does not depend \emph{only} on the
% interpretation being defined. Instead both interpretations also depend
% on the other interpretation of the parts. In general,
% definitions with dependencies are quite common in practice.
% In this paper we consider these
% interpretations compositional, and we
% would like to model such dependent (or even mutually dependent)
% interpretations in a modular and compositional style.

Defining the merge combinator in existing
programming languages is verbose and cumbersome, requiring code for every
new kind of syntax. Yet, that code is essentially mechanical and ought to be
automated. 
While using advanced meta-programming techniques enables automating
the merge combinator to a large extent in existing programming
languages~\cite{oliveira2013feature, rendel14attributes}, those techniques have
several problems: error messages can be problematic, type-unsafe reflection
is needed in some approaches~\cite{oliveira2013feature} and
advanced type-level features are required in others~\cite{rendel14attributes}.
An alternative to the merge combinator that supports modular multiple
interpretations and works in OO languages with
support for some form of multiple inheritance and covariant
type-refinement of fields has also been recently
proposed~\cite{zhang19shallow}. 
While this approach is relatively simple, it still
requires a lot of manual boilerplate code for composition of interpretations.

This paper presents a calculus and polymorphic type system with
\emph{(disjoint) intersection types}~\cite{oliveira2016disjoint},
called \fnamee. \fnamee
supports our broader notion of compositional designs, and enables
the development of highly modular and reusable programs. \fnamee
has a built-in merge operator and a powerful subtyping relation that
are used to automate the composition of multiple (possibly dependent)
interpretations. In \fnamee subtyping is coercive and enables the
automatic generation of coercions in a \emph{type-directed} fashion. 
This process is similar to that of other type-directed code generation mechanisms
such as 
\emph{type classes}~\cite{Wadler89typeclasses}, which eliminate 
boilerplate code associated to the \emph{dictionary translation}~\cite{Wadler89typeclasses}.

\fnamee continues a line of
research on disjoint intersection types.
 Previous work on
\emph{disjoint polymorphism} (the \fname calculus)~\cite{alpuimdisjoint} studied the
combination of parametric polymorphism and disjoint intersection
types, but its subtyping relation does not support
BCD-style distributivity rules~\cite{Barendregt_1983} and the type system
also prevents unrestricted intersections~\cite{dunfield2014elaborating}. More recently the \name
calculus (or \namee)~\cite{bi_et_al:LIPIcs:2018:9227} introduced a system with \emph{disjoint
  intersection types} and BCD-style distributivity rules, but did not
account for parametric polymorphism. \fnamee is unique in that it
combines all three features in a single calculus:
\emph{disjoint intersection types} and a \emph{merge operator};
\emph{parametric (disjoint) polymorphism}; and a BCD-style subtyping
relation with \emph{distributivity rules}. The three features together
allow us to improve upon the finally tagless and object
algebra approaches and support advanced compositional designs.
Moreover previous work on disjoint intersection types has shown 
various other applications that are also possible in \fnamee, including: \emph{first-class
  traits} and \emph{dynamic inheritance}~\cite{bi_et_al:LIPIcs:2018:9214}, \emph{extensible records} and \emph{dynamic
  mixins}~\cite{alpuimdisjoint}, and \emph{nested composition} and \emph{family polymorphism}~\cite{bi_et_al:LIPIcs:2018:9227}. 


Unfortunately the combination of the three features has non-trivial
complications. The main technical challenge (like for most other
calculi with disjoint intersection types) is the proof of coherence
for \fnamee. Because of the presence of BCD-style distributivity
rules, our coherence proof is based on the recent approach employed in
\namee~\cite{bi_et_al:LIPIcs:2018:9227}, which uses a
\emph{heterogeneous} logical relation called \emph{canonicity}. To account for polymorphism,
which \namee's canonicity does not support, we originally wanted
to incorporate the relevant parts of System~F's logical relation~\cite{reynolds1983types}.
However, due to a mismatch between the two relations, this did not work. The
parametricity relation has been carefully set up with a delayed type
substitution to avoid ill-foundedness due to its impredicative polymorphism.
Unfortunately, canonicity is a heterogeneous relation and needs to account for
cases that cannot be expressed with the delayed substitution setup of the
homogeneous parametricity relation. Therefore, to handle those heterogeneous
cases, we resorted to immediate substitutions and 
% restricted \fnamee to
\emph{predicative instantiations}.
%other
%measures to avoid the ill-foundedness of impredicative instantiation.
%We have settled on restricting \fnamee to \emph{predicative polymorphism} to
%keep the coherence proof manageable. 
We do not believe that predicativity is a severe restriction in practice, since many source
languages (e.g., those based on the Hindley-Milner type system like Haskell and
OCaml) are themselves predicative and do not require the full generality of an
impredicative core language. Should impredicative instantiation be required,
we expect that step-indexing~\cite{ahmed2006step} can be used to recover well-foundedness, though
at the cost of a much more complicated coherence proof.

The formalization and metatheory of \fnamee are a significant advance over that of
\fname. Besides the support for distributive subtyping, \fnamee removes 
several restrictions imposed by the syntactic coherence
proof in \fname. In particular \fnamee supports unrestricted
intersections, which are forbidden in \fname. Unrestricted
intersections enable, for example, encoding certain forms of 
bounded quantification~\cite{pierce1991programming}.
Moreover the new proof method is more robust
with respect to language extensions. For instance, \fnamee supports the bottom
type without significant complications in the proofs, while it was a challenging
open problem in \fname.
A final interesting aspect is that \fnamee's type-checking is decidable. In the
design space of languages with polymorphism and subtyping, similar mechanisms
have been known to lead to undecidability. Pierce's seminal paper
``\emph{Bounded quantification is undecidable}''~\cite{pierce1994bounded} shows
that the contravariant subtyping rule for bounded quantification in
\fsub leads to undecidability of subtyping.  In \fnamee the
contravariant rule for disjoint quantification retains decidability. 
Since with unrestricted intersections \fnamee can express several
use cases of bounded quantification, \fnamee could be an interesting and
decidable alternative to \fsub.

\begin{comment}
Besides coherence, we show
several other important meta-theoretical results, such as type-safety, 
sound and complete algorithmic subtyping, and
decidability of the type system. Remarkably, unlike 
\fsub's \emph{bounded polymorphism}, disjoint polymorphism
in \fnamee supports decidable type-checking.
\end{comment}

In summary the contributions of this paper are:
\begin{itemize}

\item {\bf The \fnamee calculus,} which is the first calculus to combine 
disjoint intersection types, BCD-style distributive subtyping and 
disjoint polymorphism. We show several meta-theoretical results, such as \emph{type-safety}, \emph{sound and complete algorithmic subtyping},
\emph{coherence} and \emph{decidability} of the type system.
\fnamee includes the \emph{bottom type}, which was considered to be a
significant challenge in previous work on disjoint polymorphism~\cite{alpuimdisjoint}.

\item {\bf An extension of the canonicity relation with polymorphism,}
  which enables the proof of coherence of \fnamee. We show that the ideas of
  System F's \emph{parametricity} cannot be ported to
  \fnamee. To overcome the problem we use a technique based on
  immediate substitutions and a predicativity restriction.

% \item {\bf Disjoint intersection types in the presence of bottom:}
%   Our calculus includes the bottom type, which was considered to be a
% significant challenge in previous work on disjoint polymorphism~\cite{alpuimdisjoint}.

\item {\bf Improved compositional designs:} We show that \fnamee's combination of features
enables improved
compositional programming designs and supports automated composition
of interpretations in programming techniques like object algebras and
finally tagless.

\item {\bf Implementation and proofs:} All of the metatheory
  of this paper, except some manual proofs of decidability, has been
  mechanically formalized in Coq. Furthermore, \fnamee is
  implemented and all code presented in the paper is available. The
  implementation, Coq proofs and extended version with appendices can be found in
  \url{https://github.com/bixuanzju/ESOP2019-artifact}.

\end{itemize}

% \bruno{
% Still need to figure out how to integrate row types in the intro story
% Furthermore, we provide a detailed
% comparison between \emph{distributive disjoint polymorphism} and
% \emph{row types}.
% }

% Compositionality is a desirable property in programming
% designs. Broadly defined, compositionality is the principle that a
% system should be built by composing smaller subsystems.
% In the area of programming languages compositionality is
% a key aspect of \emph{denotational semantics}~\cite{scott1971toward, scott1970outline}, where
% the denotation of a program is constructed from denotations of its parts.
% For example, the semantics for a language of simple arithmetic expressions
% is defined as:
% 
% \[\begin{array}{lcl}
% \llbracket n \rrbracket_{E} & = & n \\
% \llbracket e_1 + e_2 \rrbracket_{E} & = & \llbracket e_1 \rrbracket_E + \llbracket  e_2 \rrbracket_E \\
% \end{array}\]
% 
% \bruno{Replace E by fancier symbol?}
% Here there are two forms of expressions: numeric literals and
% additions. The semantics of a numeric literal is just the numeric
% value denoted by that literal. The semantics of addition is the
% addition of the values denoted by the two subexpressions.
% Compositional definitions have many benefits.
% One is ease of reasoning: since compositional
% definitions are recursively defined over smaller elements they
% can typically be reasoned about using induction. Another benefit
% of compositional definitions is that they are easy to extend,
% without modifying previous definitions.
% For example, if we also wanted to support multiplication,
% we could simply define an extra case:
% 
% \[\begin{array}{lcl}
% \llbracket e_1 * e_2 \rrbracket_E & = & \llbracket e_1 \rrbracket_E * \llbracket  e_2 \rrbracket_E \\
% \end{array}\]
% 
% Programming techniques that support compositional
% definitions include:
% \emph{shallow embeddings} of
% Domain Specific Languages (DSLs)~\cite{DBLP:conf/icfp/GibbonsW14}, \emph{finally
%   tagless}~\cite{CARETTE_2009}, \emph{polymorphic embeddings}~\cite{} or
% \emph{object algebras}~\cite{oliveira2012extensibility}. All those techniques allow us to easily create
% compositional definitions, which are easy to extend without
% modifications. Moreover both finally tagless and object algebras
% support \emph{multiple interpretations} (or denotations) of
% the syntax, thus offering a solution to the infamous \emph{Expression Problem}~\cite{wadler1998expression}.
% Because of these benefits they have become
% popular both in the functional and object-oriented
% programming communities.
% 
% However, programming languages often only support simple
% compositional designs well, while language support for more sophisticated
% compositional designs is lacking. Once we have multiple
% interpretations of syntax, then we may wish to compose those
% interpretations. In particular, when multiple interpretations exist, a useful operation
% is a \emph{merge} combinator ($\otimes$) that composes two
% interpretations~\cite{oliveira2012extensibility, oliveira2013feature, rendel14attributes}, forming a
% new interpretation that, when executed, returns the results of both
% interpretations. For example, consider another pretty printing interpretation (or
% semantics) $\llbracket \cdot \rrbracket_P$ for arithmetic expressions, which
% returns the string that denotes the concrete syntax of the
% expression. Using merge we can compose the two interpretations to
% obtain a new interpretation that executes both printing and evaluation:
% \jeremy{Explain what is $E\,\&\,P$?}
% 
% \[\begin{array}{lcl}
% \llbracket \cdot \rrbracket_E \otimes \llbracket \cdot \rrbracket_P & = & \llbracket \cdot \rrbracket_{E\,\&\,P} \\
% \end{array}\]
% 
% Such merge combinator can be manually defined in existing programming 
% The merge combinator can be manually defined in existing programming
% languages, and be used in combination with techniques such as finally
% tagless or object algebras. Furthermore variants of the
% merge combinator can help express more complex combinations of multiple
% interpretations. For example consider the following two
% interpretations ($\llbracket \cdot \rrbracket_{\mathsf{Odd}}$ and
% $\llbracket \cdot \rrbracket_{\mathsf{Even}}$) over Peano-style natural numbers:
% 
% \[\begin{array}{lclclcl}
% \llbracket 0 \rrbracket_{\mathsf{Even}}  & = & \mathsf{True} & ~~~~~~~~~~~~~~~~~~~~ & \llbracket 0 \rrbracket_{\mathsf{Odd}} & = & \mathsf{False} \\
% \llbracket S~e \rrbracket_{\mathsf{Even}} & = & \llbracket e \rrbracket_{\mathsf{Odd}} & ~~ & \llbracket S~e \rrbracket_{\mathsf{Odd}} & = & \llbracket e \rrbracket_{\mathsf{Even}}\\
% \end{array}\]
% 
% \emph{Are these interpretations compositional or not?} Under
% a strict definition of compositionality they are not because
% the interpretation of the parts does not depend \emph{only} on the
% interpretation being defined. Instead both interpretations also depend
% on the other interpretation of the parts. In general,
% definitions with dependencies are quite common in practice.
% In this paper we consider these
% interpretations compositional, and we
% would like to model such dependent (or even mutually dependent)
% interpretations in a modular and compositional style.
% 
% However defining the merge combinator in existing programming
% languages is verbose and cumbersome, and requires code for every new
% kind of syntax. Yet, that code is essentially mechanical and
% ought to be automated. While using advanced meta-programming
% techniques enables automating the merge combinator to a large extent
% in existing programming languages~\cite{oliveira2013feature, rendel14attributes}, those techniques have
% several problems. For example, error messages can be problematic, some
% techniques rely on type-unsafe reflection, while other techniques
% require highly advanced type-level features.
% 
% This paper presents a calculus and polymorphic type system with
% \emph{(disjoint) intersection types}~\cite{oliveira2016disjoint}, called \fnamee, that
% supports our broader notion of compositional designs, and enables
% the development of highly modular and reusable programs. \fnamee
% has a built-in merge operator and a powerful subtyping relation that
% are used to automate the composition of multiple interpretations
% (including dependent interpretations). \fnamee continues a line of
% research on disjoint intersection types. Previous work on
% \emph{disjoint polymorphism} (the \fname calculus) studied the
% combination between parametric polymorphism and disjoint intersection
% types, but the subtyping relation did not support
% BCD-style distributivity rules~\cite{Barendregt_1983}. More recently the \name
% calculus (or \namee) studied a system with \emph{disjoint
%   intersection types} and BCD-style distributivity rules, but did not
% account for parametric polymorphism. \fnamee is unique in that it
% allows the combination of three useful features in a single calculus:
% \emph{disjoint intersection types} and a \emph{merge operator};
% \emph{parametric (disjoint) polymorphism}; and a BCD-style subtyping
% relation with \emph{distributivity rules}. All three features are
% necessary to use improved versions of finally tagless or object
% algebras that support improved compositional designs.
% 
% Unfortunatelly the combination of the three features has non-trivial
% complications. The main technical challenge (as often is the case for
% calculi with disjoint intersection types) is the proof of coherence
% for \fnamee. Because of the presence BCD-style distributivity
% rules, the proof of coherence is based on the approach using a
% \emph{heterogeneous} logical relation employed in
% \namee~\cite{bi_et_al:LIPIcs:2018:9227}. However the logical relation in
% \namee, which we call here \emph{canonicity}, does not
% account for polymorphism. To account for polymorphism we originally
% expected to simply borrow ideas from \emph{parametricity}~\cite{reynolds1983types} in
% System F~\cite{reynolds1974towards} and adapt them to fit with the canonicity relation.
% However, this did not work. The problem is partly due to the fact that
% canonicity (unlike parametricity) is an heterogenous relation and
% needs to account for heterogeneous cases that are not considered in an
% homogeneous relation such as parametricity. Those heterogeneous cases, combined
% with \emph{impredicative polymorphism}, resulted in an ill-founded logical
% relation. Fortunatelly it turns out that
% restricting the calculus to \emph{predicative polymorphism} and using
% an approach based on substitutions is
% sufficient to recover a well-founded canonicity relation.
% Therefore we
% adopted this approach in \fnamee.
% We do not view
% the predicativity restriction as being very severe in practice, since many
% practical languages have such restriction as well. For example languages based
% on Hindley-Milner style type systems (such as Haskell, OCaml or ML)
% \ningning{it's hard to say this is true. When we say Hindley-Milner type system,
%   or Haskell, we are referring to the source language. However, the core
%   language for, for example Haskell, which is System FC, is impredicative.
%   \fnamee is more close to a core language (which usually has explicit type
%   abstractions/applications). In this sense it's unfair to compare it with other
%   source languages.} all use predicative polymorphism. Furthermore with the
% predicativity restriction, the canonicity relation and corresponding proofs
% remain relatively simple and do not require emplying more complex approaches
% such as \emph{step-indexed logical relations}. \ningning{we should emphasize
%   that predicativity is not a restriction, rather it's choice we made in order
%   to prove coherence in Coq. Step-indexed logical relation might work for
%   impredicativity; it's just we don't know.}
% 
% In summary the contributions of this paper are:
% 
% \begin{itemize}
% 
% \item {\bf The \fnamee calculus,} which integrates disjoint intersection types,
%     distributivity and disjoint polymorphism. \fnamee
%     is the first calculus puts all three features together. The
%     combination is non-trivial, expecially with respect to the
%     coherence proof.
% 
% \bruno{improve text}
% \item {\bf The canonicity logical relation,} which enables the proof
%     of coherence of \fnamee. We show that the ideas of
%   System F's \emph{parametricity} cannot be ported to
%   \fnamee. To overcome the problem we develop a canonicity
%   relation that enables a proof of coherence.
% 
% \item {\bf Disjoint intersection types in the presence of bottom:}
%   Our calculus includes a bottom type, which was considered to be a
% significant challenge in previous work.
% 
% \item {\bf Improved compositional designs:} We show how \fnamee has all the
% features that enable improved
% compositional programming designs and support automated composition
% of interpretations in programming techniques like object algebras and
% finally tagless.
% 
% \item {\bf Implementation and proofs:} All proofs
% (including type-safety, coherence and decidability of the type system)
% are proved in the Coq theorem prover. Furthermore \sedel \ningning{where comes the name \sedel?} and
% \fnamee are implemented and all code presented in the paper is
% available. The implementation, proofs and examples can be found in:
% 
% \url{MISSING}
% 
% \end{itemize}
% 
% \bruno{
% Still need to figure out how to integrate row types in the intro story
% Furthermore, we provide a detailed
% comparison between \emph{distributive disjoint polymorphism} and
% \emph{row types}.
% }

% Local Variables:
% org-ref-default-bibliography: "../paper.bib"
% End:

\section{Background}
\label{sec:background}

In this section we review a simple gradually typed language with
objects~\citep{siek2007gradual}, to introduce the concept of consistent
subtyping. We also briefly talk about the \citeauthor{odersky1996putting} type system for
higher-rank types~\citep{odersky1996putting}, which serves as the original
language on which our gradually typed calculus with implicit 
higher-rank polymorphism is based.


\subsection{Gradual Subtyping}

\begin{figure}[t]
  \begin{small}

  \begin{mathpar}
    \framebox{$A \tsub B$}\\
    \ObSInt \and \ObSBool \and \ObSFloat \and
    \ObSIntFloat \\ \ObFun \and
    \ObSRecord \and \ObSUnknown
  \end{mathpar}

  \begin{mathpar}
    \framebox{$A \sim B$}\\
    \ObCRefl \and \ObCUnknownR \and
    \ObCUnknownL \and \ObCC \and \ObCRecord
  \end{mathpar}

  \end{small}

  \caption{Subtyping and type consistency in \obb}
  \label{fig:objects}
\end{figure}

\citet{siek2007gradual} developed a gradual type system for object-oriented
languages that they call \obb. Central to gradual typing is the concept of
\emph{consistency} (written $\sim$) between gradual types, which are types
that may involve the unknown type $[[unknown]]$. The intuition is that consistency
relaxes the structure of a type system to tolerate unknown positions in a
gradual type. They also defined the subtyping relation in a way that static type
safety is preserved. Their key insight is that the unknown type $[[unknown]]$ is
neutral to subtyping, with only $[[ unknown <: unknown  ]]$. Both relations are
defined in \cref{fig:objects}.

A primary contribution of their work is to show that consistency and subtyping
are orthogonal. However, the orthogonality of consistency and subtyping does not
lead to a deterministic relation.
% To compose subtyping and consistency,
Thus \citeauthor{siek2007gradual} defined \emph{consistent subtyping} (written
$[[<~]]$) based on a \emph{restriction operator}, written $\mask A B$ that
``masks off'' the parts of type $A$ that are unknown in type $B$. For example,
$\mask {\nat \to \nat} {\bool \to \unknown} = \nat \to \unknown$, and $\mask
{\bool \to \unknown} {\nat \to \nat} = {\bool \to \unknown}$. The definition of
the restriction operator is given below:

\begin{small}
\begin{align*}
   \mask A B = & \kw{case}~(A, B)~\kw{of}  \\
               & \mid (\_, \unknown) \Rightarrow \unknown \\
               & \mid (A_1 \to A_2, B_1 \to B_2)  \Rightarrow  \mask {A_1} {B_1} \to \mask {A_2} {B_2} \\
               & \mid ([l_1: A_1,...,l_n:A_n], [l_1:B_1,...,l_m:B_m])~\kw{if}~n \leq m \Rightarrow
                [l_1 : \mask {A_1} {B_1}, ..., l_n : \mask {A_n} {B_n}]\\
               & \mid ([l_1: A_1,...,l_n:A_n], [l_1:B_1,...,l_m:B_m])~\kw{if}~n > m \Rightarrow  [l_1 : \mask {A_1} {B_1}, ..., l_m : \mask {A_m} {B_m},...,l_n:A_n ]\\
               & \mid (\_, \_) \Rightarrow A
\end{align*}
\end{small}

With the restriction operator, consistent subtyping is simply defined as:

\begin{definition}[Algorithmic Consistent Subtyping of \citet{siek2007gradual}] \label{def:algo-old-decl-conssub}
$A \tconssub B \equiv \mask A B \tsub \mask B A$.
\end{definition}

Later they show a proposition that consistent subtyping is equivalent to two
declarative definitions, which we refer to as the strawman for \emph{declarative}
consistent subtyping because it servers as a good guideline on superimposing
consistency and subtyping. Both definitions are non-deterministic because of
the intermediate type $C$.

\begin{definition}[Strawman for Declarative Consistent Subtyping] \label{def:old-decl-conssub}
The following two are equivalent:
\begin{enumerate}
\item $[[A <~ B]]$ if and only if $[[A ~ C]]$ and $[[C <: B]]$ for some $C$.
\item $[[A <~ B]]$ if and only if $[[A <: C]]$ and $[[C ~ B]]$ for some $C$.
\end{enumerate}
\end{definition}

In our later discussion, it will always be clear which definition we are referring
to. For example, we focus more on \cref{def:old-decl-conssub} in
\cref{subsec:towards-conssub}, and more on \cref{def:algo-old-decl-conssub} in
\cref{subsec:conssub-wo-exist}.


\subsection{The Odersky-L{\"a}ufer Type System}

\begin{figure}[t]
  \begin{small}
    \centering
    \begin{tabular}{lrcl} \toprule
      Types & $[[A]], [[B]]$ & $\Coloneqq$ & $[[int]] \mid [[a]] \mid [[A -> B]] \mid [[\/ a. A]]  $ \\
      Monotypes & $[[t]], [[s]]$ & $\Coloneqq$ & $ [[int]] \mid [[a]] \mid [[t -> s]] $ \\
      Terms & $[[e]]$ & $\Coloneqq$ & $ [[x]]  \mid [[n]]  \mid [[\x : A . e]] \mid [[ \x . e ]] \mid [[e1 e2]] \mid [[ let x = e1 in e2  ]] $ \\
      Contexts & $[[dd]]$ & $\Coloneqq$ & $ [[empty]]  \mid [[dd , x : A]]  \mid [[dd, a]]$ \\
      \bottomrule
    \end{tabular}

    \drules[u]{$ [[dd ||- e : A ]] $}{Typing}{var, int, lamann, lam, app, sub, gen, let}

    \drules[s]{$ [[dd |- A <: B ]] $}{Subtyping}{tvar, int, arrow, forallL, forallR}

  \end{small}
  \caption{Syntax and static semantics of the Odersky-L{\"a}ufer type system.}
  \label{fig:original-typing}
\end{figure}


The calculus we are combining gradual typing with is the well-established
predicative type system for higher-rank types proposed by
\citet{odersky1996putting}.

The syntax of the type system, along with the typing and subtyping judgments is
given in \cref{fig:original-typing}. An implicit assumption throughout the paper
is that variables in contexts are distinct. Most typing rules are standard.
The \rref{u-sub} (the subsumption rule) allows us to convert the type $[[A1]]$
of an expression $[[e]]$ to some 
supertype $[[A2]]$. The \rref{u-gen} generalizes type variables to polymorphic
types. These two rules can be applied anywhere.
Most subtyping rules are standard as well. \Rref{s-forallL} guesses a monotype
$[[t]]$ to instantiate the type variable $[[a]]$, and the subtyping relation
holds if the the instantiated type $[[ A[a ~> t] ]]$ is a subtype of $[[B]]$. In
\rref{s-forallR}, the type variable $[[a]]$ is put into the typing context and
subtyping continues with $[[A]]$ and $[[B]]$.
We save additional explanation about
the static semantics for \cref{sec:type-system}, where we present our gradually
typed version of the calculus.


%%% Local Variables:
%%% mode: latex
%%% TeX-master: "../paper"
%%% org-ref-default-bibliography: ../paper.bib
%%% End:

\section{Motivation and Applications}
\label{sec:motivation}

In this section we motivate why the combination of gradual typing and implicit
polymorphism is useful. We then illustrate two concrete applications related to
algebraic datatypes. The first application shows how gradual typing helps in
defining Scott encodings of algebraic datatypes~\citep{curry1958combinatory,
  parigot1992recursive}, which are impossible to encode in plain System F. The
second application shows how gradual typing makes it easy to model and use
heterogeneous containers.

% \subsection{Motivation}

% In this section we discuss why we are interested in integrating gradual
% typing and implicit polymorphism.

% \paragraph{Why bring gradual types to implicit polymorphism} The Glasgow
% Haskell Compiler (GHC) is a state-of-art compiler that is able to deal with many
% advanced features, including predicative implicit polymorphism. However,
% consider the program \lstinline{(\f. (f 1, f 'a')) id} that cannot type check in
% current GHC, where \lstinline{id} stands for the identity function. The problem
% with this expression is that type inference fails to infer a polymorphic type
% for $f$, given there is no explicit annotation for it. However, from dynamic
% point of view, this program would run smoothly. Also, requesting programmers to
% add annotations for each polymorphic types could become annoying when the
% program scales up and the annotation is long and complicated.

% Gradual types provides a simple solution for it. Rewriting the above expression
% to \lstinline{(\f: *. (f 1, f 'a')) id} produces the value \lstinline{(1, 'a')}.
% Namely, gradual typing provides an alternative that defers typing errors (if
% exists) into dynamic time. Also, without losing type safety, it allows
% programmer to be sloppy about annotations to be agile for program development,
% and later refine the typing annotations to regain the power of static typing.

% \paragraph{Why bring implicit polymorphism to gradual typing} There are several
% existing work about integrating explicit polymorphism into gradual type systems,
% with or without explicit casts~\citep{ahmed2011blame, yuu2017poly}, yet no work
% investigates to add the expressive power of implicit polymorphism into a
% source language.
% % In implicit
% % polymorphism, type applications can be reconstructed by the type checker, for
% % example, instead of \lstinline{id Int 3}, one can directly write \lstinline{id 3}.
% Implicit polymorphism is a hallmark of functional programming, and
% modern functional languages (such as Haskell) employ sophisticated
% type-inference algorithms that, aided by type annotations, can deal with
% higher-rank polymorphism. Therefore as a step towards gradualizing such type
% systems, this paper develops both declarative and algorithmic versions for a
% type system with implicit higher-rank polymorphism.

\subsection{Motivation: Gradually Typed Higher-Rank Polymorphism}

Our work combines implicit (higher-rank) polymorphism with gradual typing. As
is well known, a gradually typed language supports both fully static and fully
dynamic checking of program properties, as well as the continuum between these
two extremes. It also offers programmers fine-grained control over the
static-to-dynamic spectrum, i.e., a program can be evolved by introducing more
or less precise types as needed~\citep{garcia2016abstracting}. 

Haskell is a language renowned for its advanced type system, but it does not feature
gradual typing. Of particular interest to us is its support for implicit
higher-rank polymorphism, which is supported via explicit type
annotations. 
In Haskell some programs that are safe at run-time may
be rejected due to the conservativity of the type system. For example, consider
the following Haskell program adapted from \citet{jones2007practical}:
\begin{lstlisting}
foo :: ([Int], [Char])
foo = let f x = (x [1, 2] , x ['a', 'b']) in f reverse
\end{lstlisting}
This program is rejected by Haskell's type checker because Haskell implements
the Damas-Milner \citep{hindley69principal, damas1982principal} rule that a
lambda-bound argument (such as \lstinline{x}) can only have a monotype, i.e.,
the type checker can only assign \lstinline{x} the type
\lstinline{[Int] -> [Int]}, or \lstinline{[Char] -> [Char]},
but not \lstinline$foralla. [a] -> [a]$.
Finding such manual polymorphic annotations can be non-trivial, especially
when the program scales up and the annotation is long and complicated.

Instead
of rejecting the program outright, due to missing type annotations, gradual
typing provides a simple alternative by giving \lstinline$x$ the unknown type
$\unknown$. With this type the same program type-checks and produces
\lstinline$([2, 1], ['b', 'a'])$. By running the program, programmers can gain
more insight about its run-time behaviour. Then, with this insight, they can
also give \lstinline$x$ a more precise type (\lstinline$foralla. [a] -> [a]$) a
posteriori so that the program continues to type-check via implicit polymorphism
and also grants more static safety. In this paper, we envision such a language
that combines the benefits of both implicit higher-rank polymorphism and gradual
typing.

%-------------------------------------------------------------------------------
\subsection{Application: Efficient (Partly) Typed Encodings of ADTs}

Our calculus does not provide built-in support for algebraic datatypes (ADTs).
Nevertheless, the calculus is expressive enough to support efficient
function-based encodings of (optionally polymorphic) ADTs\footnote{In a type
  system with impure features, such as non-termination or exceptions, the encoded
  types can have valid inhabitants with side-effects, which means we only get
  the \textit{lazy} version of those datatypes.}.
This offers an immediate way to model algebraic
datatypes in our calculus without requiring extensions to our calculus or, more
importantly, to its target---the polymorphic blame calculus. While we believe
that such extensions are possible, they would likely require non-trivial
extensions to the polymorphic blame calculus. Thus the alternative of being able
to model algebraic datatypes without extending \pbc is appealing. The encoding
also paves the way to provide built-in support for algebraic datatypes in the
source language, while elaborating them via the encoding into \pbc.

\paragraph{Church and Scott Encodings}
It is well-known
that polymorphic calculi such as System F can encode datatypes via
Church encodings. However these encodings have well-known drawbacks. 
In particular, some operations are hard to define, and they can have a time
complexity that is greater than that of the corresponding functions for built-in
algebraic datatypes. A good example is the definition of
the predecessor function for Church numerals~\citep{church1941calculi}. Its
definition requires significant ingenuity (while it is trivial with 
built-in algebraic datatypes), and it has \emph{linear} time
complexity (versus the \emph{constant} time complexity for a definition 
using built-in algebraic datatypes). 

An alternative to Church encodings are the so-called Scott
encodings~\citep{curry1958combinatory}. They address the two drawbacks of Church
encodings: they allow simple definitions that directly correspond to programs
implemented with built-in algebraic datatypes, and those definitions have the same time
complexity to programs using algebraic datatypes.

Unfortunately, Scott encodings, or more precisely, their typed
variant~\citep{parigot1992recursive}, cannot be expressed in System F: in the
general case they require recursive types, which System F does not support.
However, with gradual typing, we can remove the need for recursive types, thus
enabling Scott encodings in our calculus.

\paragraph{A Scott Encoding of Parametric Lists}
Consider for instance the typed
Scott encoding of parametric lists in a system with implicit polymorphism:
\begin{align*}
   [[ List a ]] &\triangleq [[  mu L . \/b. b -> (a -> L -> b) -> b       ]] \\
   [[nil]] &\triangleq [[  fold [ List a ] (\n . \c . n ): \/ a . List a    ]] \\
   [[cons]] & \triangleq [[ \x . \xs . fold [List a]  (\n . \c. c x xs) :  \/a . a -> List a -> List a  ]]
\end{align*}
This encoding requires both polymorphic and recursive types\footnote{Here we use
iso-recursive types, but equi-recursive types can be used too.}. 
Like System F, our calculus 
only supports the former, but not the latter. Nevertheless, gradual types still
allow us to use the Scott encoding in a partially typed fashion.
The trick is to omit the recursive type binder $\mu L$ and replace the recursive
occurrence of $L$ by the unknown type $\unknown$:
\begin{align*}
   [[ Listu a  ]] &\triangleq [[\/ b. b -> (a -> unknown -> b) -> b]]
\end{align*}
As a consequence, we need to replace the term-level witnesses of the
iso-recursion by explicit type annotations to respectively forget or recover the type structure of
the recursive occurrences:
\begin{align*}
  [[ fold [Listu a] ]] & \triangleq [[\x . x : (\/b . b -> (a -> Listu a -> b) -> b) -> Listu a  ]] \\
  [[ unfold [Listu a] ]] & \triangleq [[ \x . x : Listu a -> (\/b . b -> (a -> Listu a -> b) -> b)     ]]
\end{align*}
With the reinterpretation of $[[fold]]$ and $[[unfold]]$ as functions instead of
built-in primitives, we have exactly the same definitions of $[[nilu]]$ and
$[[consu]]$.

Note that when we elaborate our calculus into the polymorphic blame calculus, the above
type annotations give rise to explicit casts. For
instance, after elaboration $[[ fold [Listu a] e   ]]$ results in the cast 
$ [[< (\/b . b -> (a -> Listu a -> b) -> b) `-> Listu a > pe]] $ where $[[pe]]$ is the elaboration of $[[e]]$.

In order to perform recursive traversals on lists, e.g., to compute their
length, we need a fixpoint combinator like the Y combinator. Unfortunately, this combinator
cannot be assigned a type in the simply typed lambda calculus or System F.
Yet, we can still provide a gradual typing for it in our system.
\begin{align*}
[[fix]] &\triangleq [[  \ f . (\x : unknown . f (x x)) (\x : unknown . f (x x)) : \/ a. (a -> a) -> a ]]
\end{align*}
This allows us for instance to compute the length of a list.
\begin{align*}
\mathsf{length} &\triangleq [[  fix ( \len . \l . zerou (\xs . succu (len xs)))  ]]
\end{align*}
Here $[[ zerou : natu  ]]$ and $[[ succu : natu -> natu    ]]$
are the encodings of the constructors for natural numbers $[[ natu
]]$. In practice, 
for performance reasons, we could extend our
language with a $\mathbf{letrec}$ construct in a standard way to
support general recursion, instead of defining a fixpoint combinator.

% length :: forall a. List a -> Nat
% length = fix (\len -> \l -> l zero (\xs -> succ (len xs)))

Observe that the gradual typing of lists still enforces that all
elements in the list are
of the same type. For instance, a heterogeneous list like
$[[  consu zerou (consu trueu nilu)    ]]$,
is rejected because $[[ zerou : natu    ]]$ and $[[ trueu : boolu  ]]$ have different types.

\paragraph{Heterogeneous Containers}
Heterogeneous containers are datatypes that can store data of different types,
which is very useful in various scenarios. One typical application is that an
XML element is heterogeneously typed. Moreover, the result of a SQL query
contains heterogeneous rows.

In statically typed languages, there are several ways to obtain heterogeneous lists. For example, in Haskell, one option is
to use \emph{dynamic types}. Haskell's library \textbf{Data.Dynamic} provides the
special type \textbf{Dynamic} along with its injection \textbf{toDyn} and
projection \textbf{fromDyn}. The drawback is that the code is littered with
\textbf{toDyn} and \textbf{fromDyn}, which obscures the program logic.
One can also use the \textsc{HList} library
\citep{kiselyov2004strongly}, which provides strongly typed data structures for
heterogeneous collections. The library requires several Haskell extensions, such as multi-parameter classes~\citep{jones1997type} and
functional dependencies~\citep{jones2000type}.
With fake dependent
types~\citep{mcbride2002faking}, heterogeneous vectors are also possible with
type-level constructors.

In our type system, with explicit type annotations that set the element types to
the unknown type we can disable the homogeneous typing discipline for the
elements and get gradually typed heterogeneous lists\footnote{This argument is
  based on the extended type system in \cref{sec:advanced-extension}.}. Such
gradually typed heterogeneous lists are akin to Haskell's approach with Dynamic
types, but much more convenient to use since no injections and projections are
needed, and the $[[unknown]]$ type is built-in and natural to use.

An example of such gradually typed heterogeneous collections is:
\[
l \triangleq [[consu (zerou : unknown) (consu (trueu : unknown) nilu)]]
\]
Here we annotate each element with type annotation $\unknown$ and the type
system is happy to type-check that $[[ l : Listu unknown ]]$.
Note that we are being meticulous about the syntax, but with proper
implementation of the source language, we could write more succinct programs
akin to Haskell's syntax, such as \lstinline{[0, True]}.

% \begin{itemize}
% \item Scott encodings of simple first-order ADTs (e.g. naturals)
% \item Parigot encodings improves Scott encodings with recursive schemes, but
%   occupies exponential space, whereas Church encoding only occupies linear
%   space.
% \item An alternative encoding which retains constant-time destructors but also
%   occupies linear space.
% \item Parametric ADTs also possible?
% \item Typing rules
% \end{itemize}
% 
% \begin{example}[Scott Encoding of Naturals]
% \begin{align*}
%   [[nat]] &\triangleq [[  \/a. a -> (unknown -> a) -> a ]] \\
%   \mathsf{zero} &\triangleq [[ \x . \f . x  ]] \\
%   \mathsf{succ} &\triangleq [[ \y : nat . \x . \f . f y ]]
% \end{align*}
% \end{example}
% Scott encodings give constant-time destructors (e.g., predecessor), but one has to
% get recursion somewhere. Since our calculus admits untyped lambda calculus, we
% could use a fixed point combinator.
% 
% \begin{example}[Parigot Encoding of Naturals]
% \begin{align*}
%   [[nat]] &\triangleq [[  \/a. a -> (unknown -> a -> a) -> a ]] \\
%   \mathsf{zero} &\triangleq [[ \x . \f . x  ]] \\
%   \mathsf{succ} &\triangleq [[ \y : nat . \x . \f . f y (y x f) ]]
% \end{align*}
% \end{example}
% Parigot encodings give primitive recursion, apart form constant-time
% destructors, but at the cost of exponential space complexity (notice in
% $\mathsf{succ}$ there are two occurances of $[[y]]$).
% 
% Both Scott and Parigot encodings are typable in System F with positive recursive
% types, which is strong normalizing.
% 
% \begin{example}[Alternative Encoding of Naturals]
% \begin{align*}
%   [[nat]] &\triangleq [[  \/a. a -> (unknown -> (unknown -> a) -> a) -> a ]] \\
%   \mathsf{zero} &\triangleq [[ \x . \f . x  ]] \\
%   \mathsf{succ} &\triangleq [[ \y : nat . \x . \f .  f y (\g . g x f) ]]
% \end{align*}
% \end{example}
% This encoding enjoys constant-time destructors, linear space complexity, and
% primitive recursion.
% The static version is $[[ mu b . \/ a . a -> (b -> (b -> a) -> a) -> a ]]$,
% which can only be expressed in System F with
% general recursive types (notice the second $[[b]]$ appears in a negative position).

%%% Local Variables:
%%% mode: latex
%%% TeX-master: "../paper"
%%% org-ref-default-bibliography: "../paper.bib"
%%% End:

\section{Revisiting Consistent Subtyping}
\label{sec:exploration}

In this section we explore the design space of consistent subtyping.
% Now
% metavariables $A,B$ in \cref{fig:original-typing} also range over the unknown type
% $\unknown$.
We start with the definitions of consistency and subtyping for
polymorphic types, and compare with some relevant work.
% (in particular the compatibility relation by \citet{ahmed2011blame} and type
% consistency by \citet{yuu2017poly}).
We then discuss the design decisions
involved towards our new definition of consistent subtyping, and justify the new
definition by demonstrating its equivalence with that of \citet{siek2007gradual}
and the AGT approach~\citep{garcia2016abstracting} on simple types.

% \begin{figure}[t]
%   \centering
%   \begin{small}
% \begin{tabular}{lrcl} \toprule
%   % Expressions & $e$ & \syndef & $x \mid n \mid
%   %                        \blam x A e \mid \erlam x e \mid e~e $ \\
% %%                         \mid \erlam x e \equiv \blam x \unknown e $ \\

%   Types & $A, B$ & \syndef & $ \nat \mid a \mid A \to B \mid \forall a. A \mid \unknown$ \\
%   Monotypes & $\tau, \sigma$ & \syndef & $ \nat \mid a \mid \tau \to \sigma$ \\

%   Contexts & $\dctx$ & \syndef & $\ctxinit \mid \dctx,x: A \mid \dctx, a$ \\
%   % Syntactic Sugar & $e : A$ & $\equiv$ & $(\blam x A x) ~ e$ \\
%   \bottomrule
% \end{tabular}
%   \end{small}
% \caption{Polymorphic types added by $\unknown$.}
% \label{fig:decl-types}
% \end{figure}

% \paragraph{Types}
The syntax of types is given at the top of
\cref{fig:decl:subtyping}.
% Meta-variable $e$ ranges over expressions.
% Expressions are either variables $x$, integers $n$, annotated lambda
% abstractions $\blam x A e$, un-annotated lambda abstractions $\erlam x e$ or
% applications $e_1 ~ e_2$.
We write $A$, $B$ for types. Types are either the integer type $\nat$, type
variables $a$, functions types $A \to B$, universal quantification $\forall a.
A$, or the unknown type $\unknown$. Though we only have one base type $\nat$, we
also use $\bool$ for the purpose of illustration. Note that monotypes $\tau$ contain all
types other than the universal quantifier and the unknown type $\unknown$. % Note that
% $\unknown$ is only added to the category of polymorphic types $A, B$.
We will discuss this restriction when we present the subtyping rules.
Contexts $\dctx$ are \textit{ordered} lists of type variable declarations and term variables.



\subsection{Consistency and Subtyping}
\label{subsec:consistency-subtyping}

We start by giving the definitions of consistency and subtyping for polymorphic
types, and comparing our definitions with the compatibility relation by
\citet{ahmed2011blame} and type consistency by \citet{yuu2017poly}.

\begin{figure}[t]
  \centering
  \begin{small}
\begin{tabular}{lrcl} \toprule
  % Expressions & $e$ & \syndef & $x \mid n \mid
  %                        \blam x A e \mid \erlam x e \mid e~e $ \\
%%                         \mid \erlam x e \equiv \blam x \unknown e $ \\

  Types & $A, B$ & \syndef & $ \nat \mid a \mid A \to B \mid \forall a. A \mid \unknown$ \\
  Monotypes & $\tau, \sigma$ & \syndef & $ \nat \mid a \mid \tau \to \sigma$ \\

  Contexts & $\dctx$ & \syndef & $\ctxinit \mid \dctx,x: A \mid \dctx, a$ \\
  % Syntactic Sugar & $e : A$ & $\equiv$ & $(\blam x A x) ~ e$ \\
  \bottomrule
\end{tabular}
  \begin{mathpar}
    \framebox{$A \sim B$} \\
    \CD \and \CA \and \CB \and \CC \and \CE
  \end{mathpar}
  % \begin{mathpar}
  %   \framebox{$\dctx \bywf A $} \\
  %   \DeclVarWF \and \DeclIntWF \and \DeclUnknownWF \\ \DeclFunWF \and \DeclForallWF
  % \end{mathpar}
  \begin{mathpar}
    \framebox{$\tpresub A \tsub B$} \\
    \HSForallR \quad \HSForallL  \quad \HSTVar \and
    \HSInt \and \HSFun  \and \HSUnknown
  \end{mathpar}
  \end{small}
  \caption{Syntax of types, consistency, and subtyping in the declarative system.}
  \label{fig:decl:subtyping}
\end{figure}

\paragraph{Consistency.}
The key observation here is that consistency is mostly a structural relation,
except that the unknown type $\unknown$ can be regarded as any type. Following
this observation, we naturally extend the definition from
\cref{fig:objects} with polymorphic types, as shown at the middle of
\cref{fig:decl:subtyping}. In particular a polymorphic type $\forall a. A$
is consistent with another polymorphic type $\forall a. B$ if $A$ is consistent
with $B$.

\paragraph{Subtyping.}

We express the fact that one type is a polymorphic generalization of another by
means of the subtyping judgment $\Psi \vdash A \tsub B$. Compared with the
subtyping rules of \citet{odersky1996putting} in
\cref{fig:original-typing}, the only addition is the neutral subtyping of
$\unknown$. Notice
that, in the rule
\rul{S-ForallL}, the universal quantifier is only allowed to be instantiated
with a \emph{monotype}.
The judgment $\dctx \bywf \tau$ checks all the type variables in $\tau$ are
bound in the context $\dctx$. For space reasons, we omit the definition.
According to the syntax in \cref{fig:decl:subtyping},
monotypes do not include the unknown type $\unknown$. This is because if we were
to allow the unknown type to be used for instantiation, we could have  $  \forall a . a \to a \tsub \unknown \to \unknown $
by instantiating $a$ with $\unknown$. Since $\unknown \to \unknown$ is
consistent with any functions $A \to B$, for instance, $\nat \to \bool$, this
means that we could provide an expression of type $\forall a. a \to a$ to a
function where the input type is supposed to be $\nat \to \bool$. However, as we
might expect, $\forall a. a \to a$ is definitely not compatible with $\nat \to
\bool$. This does not hold in any polymorphic type systems without gradual
typing. So the gradual type system should not accept it either. (This is the
so-called \textit{conservative extension} property that will be made precise in
\cref{sec:criteria}.)

Importantly there is a subtle but crucial distinction between a type variable
and the unknown type, although they all represent a kind of ``arbitrary'' type.
The unknown type stands for the absence of type information: it could be
\textit{any type} at \textit{any instance}. Therefore, the unknown type is
consistent with any type, and additional type-checks have to be performed at
runtime. On the other hand, a type variable indicates \textit{parametricity}.
% and is subject to global constraints
In other words, a
type variable can only be instantiated to a single type. For example, in the
type $\forall a. a \to a$, the two occurrences of $a$ represent an arbitrary but
single type (e.g., $\nat \to \nat$, $\bool \to \bool$), while $\unknown \to
\unknown$ could be an arbitrary function (e.g., $\nat \to \bool$) at runtime.

\paragraph{Comparison with Other Relations.}

In other polymorphic gradual calculi, consistency and subtyping are often mixed
up to some extent. In \pbc~\citep{ahmed2011blame}, the compatibility relation for polymorphic types
is defined as follows:
\begin{mathpar}
  \CompAllR \and \CompAllL
\end{mathpar}
Notice that, in rule \rul{Comp-AllL}, the universal quantifier is \textit{always}
instantiated to $\unknown$. However, this way, \pbc allows $\forall a. a \to a
\pbccons \nat \to \bool$, which as we discussed before might not be what we
expect. Indeed \pbc relies on sophisticated runtime checks to rule out such
instances of the compatibility relation a posteriori.

\citet{yuu2017poly} introduced the so-called
\textit{quasi-polymorphic} types for types that may be used where a
$\forall$-type is expected, which is important for their purpose of
conservativity over System F. Their type consistency relation, involving polymorphism, is
defined as follows\footnote{This is a simplified version.}:
\begin{mathpar}
  \inferrule{A \sim B }{\forall a. A \sim \forall a. B}
  \and
  \inferrule{A \sim B \\ B \neq \forall a. B' \\ \unknown \in \mathsf{Types}(B)}
  {\forall a. A \sim B}
\end{mathpar}
Compared with our consistency definition in \cref{fig:decl:subtyping},
their first rule is the same as ours. The second rule says that a non
$\forall$-type can be consistent with a $\forall$-type only if it contains
$\unknown$. In this way, their type system is able to reject $\forall a. a \to a
\sim \nat \to \bool$. However, in order to keep conservativity, they also reject
$\forall a. a \to a \sim \nat \to \nat$, which is perfectly sensible in their
setting (i.e., explicit polymorphism). However with implicit polymorphism, we
would expect $\forall a. a \to a$ to be related with $\nat \to \nat$, since $a$ can be instantiated to
$\nat$.

Nonetheless, when it comes to interactions between dynamically typed and
polymorphically typed terms, both relations allow $\forall a. a \to
\nat$ to be related with $\unknown \to \nat$ for example, which in our view, is some sort of
(implicit) polymorphic subtyping combined with type consistency, and
that should be derivable by the more primitive notions in the type system
(instead of inventing new relations). One of our design principles is that
subtyping and consistency is \textit{orthogonal}, and can be naturally
superimposed, echoing the same opinion of \citet{siek2007gradual}.

\subsection{Towards Consistent Subtyping}
\label{subsec:towards-conssub}

With the definitions of consistency and subtyping, the question now is how to
compose these two relations so that two types can be compared in a way that takes
these two relations into account.

Unfortunately, the original definition of \citeauthor{siek2007gradual}
(\cref{def:old-decl-conssub}) does not work well with our definitions of
consistency and subtyping for polymorphic types. Consider two types: $(\forall
a. a \to \nat) \to \nat$, and $(\unknown \to \nat) \to \nat$. The first type can only reach the
second type in one way (first by applying consistency, then subtyping), but not the
other way, as shown in \cref{fig:example:a}. We use $\bot$ to mean that we
cannot find such a type. Similarly, there are situations where the first type
can only reach the second type by the other way (first applying
subtyping, and then
consistency), as shown in \cref{fig:example:b}.

\begin{figure}[t]
  \begin{subfigure}[b]{.4\linewidth}
    \centering
      \begin{tikzpicture}
        \matrix (m) [matrix of math nodes,row sep=2em,column sep=4em,minimum width=2em]
        {
          \bot & (\unknown \to \nat) \to \nat \\
          (\forall a. a \to \nat) \to \nat & (\forall a. \unknown \to \nat) \to \nat \\};

        \path[-stealth]
        (m-2-1) edge node [left] {$\tsub$} (m-1-1)
        (m-2-2) edge node [left] {$\tsub$} (m-1-2);

        \draw
        (m-1-1) edge node [above] {$\sim$} (m-1-2)
        (m-2-1) edge node [below] {$\sim$} (m-2-2);
      \end{tikzpicture}
      \caption{}
      \label{fig:example:a}
  \end{subfigure}
  \begin{subfigure}[b]{.4\linewidth}
    \centering
    \begin{tikzpicture}
      \matrix (m) [matrix of math nodes,row sep=2em,column sep=4em,minimum width=2em]
      {
        \nat \to \nat & \nat \to \unknown \\
        \forall a. a & \bot \\};

      \path[-stealth]
      (m-2-1) edge node [left] {$\tsub$} (m-1-1)
      (m-2-2) edge node [left] {$\tsub$} (m-1-2);

      \draw
      (m-1-1) edge node [above] {$\sim$} (m-1-2)
      (m-2-1) edge node [below] {$\sim$} (m-2-2);
    \end{tikzpicture}
    \caption{}
    \label{fig:example:b}
  \end{subfigure}
  \begin{subfigure}[b]{.8\linewidth}
    \centering
    \begin{tikzpicture}
      \matrix (m) [matrix of math nodes,row sep=2em,column sep=1em,minimum width=2em]
      {
        \bot &
        (((\unknown \to \nat)\to \nat) \to \bool) \to (\nat \to \unknown)  \\
        (((\forall a. a \to \nat) \to \nat) \to \bool) \to (\forall a. a) &
        \bot \\};

      \path[-stealth]
      (m-2-1) edge node [left] {$\tsub$} (m-1-1)
      (m-2-2) edge node [left] {$\tsub$} (m-1-2);

      \draw
      (m-1-1) edge node [above] {$\sim$} (m-1-2)
      (m-2-1) edge node [below] {$\sim$} (m-2-2);
    \end{tikzpicture}
    \caption{}
    \label{fig:example:c}
  \end{subfigure}
  \caption{Examples that break the original definition of consistent subtyping.}
  \label{fig:example}
\end{figure}

What is worse, if those two examples are composed in a way that those types all
appear co-variantly, then the resulting types cannot reach each other
in either
way. For example, \cref{fig:example:c} shows such two types by putting a
$\bool$ type in the middle, and neither definition of consistent subtyping
works. % But these two types ought to be related somehow!

\paragraph{Observations on Consistent Subtyping Based on Information Propagation.}

In order to develop the correct definition of consistent subtyping for
polymorphic types, we need to understand how consistent subtyping works.
We first review two important properties of subtyping: (1) subtyping induces the
subsumption rule: if $A \tsub B$, then an expression of type $A$ can be used
where $B$ is expected; (2) subtyping is transitive: if $A \tsub B$, and $B \tsub
C$, then $A \tsub C$. Though consistent subtyping takes the unknown type into
consideration, the subsumption rule should also apply: if $A \tconssub B$, then
an expression of type $A$ can also be used where $B$ is expected, given that
there might be some information lost by consistency. A crucial difference from
subtyping is that consistent subtyping is \textit{not} transitive because
information can only be lost once (otherwise, any two types are a consistent
subtype of each other). Now consider a situation where we have both $A \tsub B$,
and $B \tconssub C$, this means that $A$ can be used where $B$ is expected, and
$B$ can be used where $C$ is expected, with possibly some loss of information. In
other words, we should expect that $A$ can be used where $C$ is expected, since
there is at most one-time loss of information.

\begin{observation}
  If $A \tsub B$, and $B \tconssub C$, then $A \tconssub C$.
\end{observation}

This is reflected in \cref{fig:obser:a}. A symmetrical
observation is given in \cref{fig:obser:b}:


\begin{observation}
  If $C \tconssub B$, and $B \tsub A$, then $C \tconssub A$.
\end{observation}

\begin{figure}[t]
  \centering
  \begin{subfigure}[b]{.4\linewidth}
    \centering
    \begin{tikzpicture}
      \matrix (m) [matrix of math nodes,row sep=2.5em,column sep=4em,minimum width=2em]
      {
        T_1 & C \\
        B   & T_2 \\
        A & \\};

      \path[-stealth]
      (m-3-1) edge node [left] {$\tsub$} (m-2-1)
      (m-2-2) edge node [left] {$\tsub$} (m-1-2)
      (m-2-1) edge node [left] {$\tsub$} (m-1-1);

      \draw
      (m-2-1) edge node [below] {$\sim$} (m-2-2)
      (m-1-1) edge node [above] {$\sim$} (m-1-2);

      \draw [dashed, ->]
      (m-2-1) edge node [above] {$\tconssub$} (m-1-2);

      \path [dashed, ->, out=0, in=0, looseness=2]
      (m-3-1) edge node [right] {$\tconssub$} (m-1-2);
    \end{tikzpicture}
    \caption{}
    \label{fig:obser:a}
  \end{subfigure}
  \centering
  \begin{subfigure}[b]{.4\linewidth}
    \centering
    \begin{tikzpicture}
      \matrix (m) [matrix of math nodes,row sep=2.5em,column sep=4em,minimum width=2em]
      {
        & A \\
        T_1 & B \\
        C   & T_2 \\};

      \path[-stealth]
      (m-3-1) edge node [left] {$\tsub$} (m-2-1)
      (m-3-2) edge node [left] {$\tsub$} (m-2-2)
      (m-2-2) edge node [left] {$\tsub$} (m-1-2);

      \draw
      (m-2-1) edge node [above] {$\sim$} (m-2-2)
      (m-3-1) edge node [below] {$\sim$} (m-3-2);

      \draw [dashed, ->]
      (m-3-1) edge node [above] {$\tconssub$} (m-2-2);

      \path [dashed, ->, out=135, in=180, looseness=2]
      (m-3-1) edge node [left] {$\tconssub$} (m-1-2);
    \end{tikzpicture}
    \caption{}
    \label{fig:obser:b}
  \end{subfigure}
  \caption{Observations of consistent subtyping}
  \label{fig:obser}
\end{figure}


From the above observations, we see what the problem is with the original
definition. In \cref{fig:obser:a}, if $B$ can reach $C$ by $T_1$, then by
subtyping transitivity, $A$ can reach $C$ by $T_1$. However, if $B$ can only reach $C$ by
$T_2$, then $A$ cannot reach $C$ through the original definition. A similar
problem is shown in \cref{fig:obser:b}.
% : if $C$ can only reach $B$ by $T_1$, then $C$ cannot reach $A$ through the original definition.

However, it turns out that those two problems can be fixed using the same strategy:
instead of taking one-step subtyping and one-step consistency, our definition of
consistent subtyping allows types to take \textit{one-step subtyping, one-step
consistency, and one more step subtyping}. Specifically, $A \tsub B \sim T_2
\tsub C$ (in \cref{fig:obser:a})
and $C \tsub T_1 \sim B \tsub A$ (in \cref{fig:obser:b}) have the same relation chain: subtyping,
consistency, and subtyping.

\paragraph{Definition of Consistent subtyping.} From the above discussion, we are
ready to modify \cref{def:old-decl-conssub}, and adapt it to our notation:

\begin{definition}[Consistent Subtyping]
  \label{def:decl-conssub}
  \[
    \inferrule{
       \tpresub A \tsub C
       \\ C \sim D
       \\ \tpresub D \tsub B
    }{
      \tpresub A \tconssub B
    }
  \]
  % $\tpresub A \tconssub B$, if and only if $\tpresub A \tsub C$, $C \sim D$, and
  % $\tpresub D \tsub B$ for
  % some $C, D$.
\end{definition}

\noindent With \cref{def:decl-conssub}, \Cref{fig:example:c:fix}
illustrates the correct relation chain for the broken example shown in
\cref{fig:example:c}.
At first sight, \cref{def:decl-conssub}
seems worse than the original: we need to guess \textit{two} types! It turns out
that \cref{def:decl-conssub} is a generalization of
\cref{def:old-decl-conssub}, and they are equivalent in the system of
\citet{siek2007gradual}. However, more generally, \cref{def:decl-conssub}
% We argue that this is a \textit{general} definition of
% consistent subtyping, and
is compatible with polymorphic types.

\begin{figure}[t]
  \centering
  \begin{subfigure}[b]{.4\linewidth}
  \begin{tikzpicture}
    \matrix (m) [matrix of math nodes,row sep=2.5em,column sep=6em,minimum width=2em]
    {
      A_2 &
      A_3
      \\
      A_1
      &
      A_4  \\
      };

    \path[-stealth]
    (m-2-1) edge node [left] {$\tsub$} (m-1-1)
    (m-1-2) edge node [left] {$\tsub$} (m-2-2);
    \path[dashed, ->, out=315, in=225, looseness=0.3]
    (m-2-1) edge node [above] {$\tconssub$} (m-2-2);

    \draw
    (m-1-1) edge node [above] {$\sim$} (m-1-2);
  \end{tikzpicture}
  \end{subfigure}
  \begin{subfigure}[b]{.4\linewidth}
  \begin{align*}
  A_1 &=(((\forall a. a \to \nat) \to \nat) \to \bool) \to (\forall a. a) \\
  A_2 &= ((\forall a. a \to \nat) \to \nat) \to \bool) \to (\nat \to \nat) \\
  A_3 &= ((\forall a. \unknown \to \nat) \to \nat) \to \bool) \to (\nat \to \unknown) \\
  A_4 &= (((\unknown \to \nat) \to \nat) \to \bool) \to (\nat \to \unknown)
  \end{align*}
  \end{subfigure}
  \caption{Example that is fixed by the new definition of consistent subtyping.}
  \label{fig:example:c:fix}
\end{figure}

\begin{proposition}[Generalization of Consistent Subtyping]\leavevmode
  \label{prop:subsumes}
\begin{itemize}
  \item \cref{def:decl-conssub} subsumes
    \cref{def:old-decl-conssub}.
    % In \cref{def:decl-conssub},
    % by choosing $D=B$, we have $A\tsub C$ and $C \sim B$; by choosing $C=A$, we have
    % $A \sim D$, and $D \tsub B$.
  \item \cref{def:old-decl-conssub} is equivalent to
    \cref{def:decl-conssub} in the system of \citeauthor{siek2007gradual}.
    % If $A \tsub C$, $C \sim D$, and $D \tsub
    % B$, by \cref{def:old-decl-conssub},
    % $A \sim C'$, $C' \tsub D$ for some $C'$. By subtyping
    % transitivity, $C' \tsub B$. So $A \tconssub B$ by $A \sim C'$ and $C'
    % \tsub B$.
  \end{itemize}
\end{proposition}


\subsection{Abstracting Gradual Typing}
\label{subsec:agt}

\citet{garcia2016abstracting} presented a new foundation for gradual typing that
they call the \textit{Abstracting Gradual Typing} (AGT) approach. In the AGT
approach, gradual types are interpreted as sets of static types, where static
types refer to types containing no unknown types. In this interpretation,
predicates and functions on static types can then be lifted to apply to gradual
types. Central to their approach is the so-called \textit{concretization}
function. For simple types, a concretization $\gamma$ from gradual types to a
set of static types\footnote{For simplification, we directly regard type
  constructor $\to$ as a set-level operator.} is defined as follows:

\begin{definition}[Concretization]
  \label{def:concret}
  \begin{mathpar}
    \gamma(\nat) = \{\nat\} \and \gamma(A \to B) = \gamma(A) \to \gamma(B) \and
    \gamma(\unknown) = \{\text{All static types}\}
  \end{mathpar}
\end{definition}

Based on the concretization function, subtyping between static types can be
lifted to gradual types, resulting in the consistent subtyping relation:
\begin{definition}[Consistent Subtyping in AGT]
  \label{def:agt-conssub}
  $A \agtconssub B$ if and only if $A_1 \tsub B_1$ for some $A_1 \in \gamma(A)$, $B_1 \in \gamma(B)$.
\end{definition}

Later they proved that this definition of consistent subtyping coincides with
that of \citet{siek2007gradual} (\cref{def:old-decl-conssub}). By
\cref{prop:subsumes}, we can directly conclude that our definition coincides with AGT:

\begin{proposition}[Equivalence to AGT on Simple Types]
  \label{lemma:coincide-agt}
  $A \tconssub B$ iff $A \agtconssub B$.
\end{proposition}

However, AGT does not show how to deal with polymorphism (e.g. the
interpretation of type variables) yet.
% As noted by \citet{garcia2016abstracting} in the conclusion, extending
% AGT to deal with polymorphism remains as an open question.
Still, as noted by \citet{garcia2016abstracting},
it is a promising line of future work for
AGT, and the question remains whether our definition would coincide with it.

Another note related to AGT is that the definition is later adopted by
\citet{castagna2017gradual}, where the static types $A_1, B_1$ in
Definition~\ref{def:agt-conssub} can be algorithmically computed by also
accounting for top and bottom types.

\subsection{Directed Consistency}

\textit{Directed consistency}~\citep{Jafery:2017:SUR:3093333.3009865} is defined
in terms of precision and static subtyping:
\[
  \inferrule{
       A' \lessp A
    \\ A \tsub B
    \\ B' \lessp B
  }{
    A' \tconssub B'
  }
\]
The judgment $A \lessp B$ is read ``$A$ is less precise than
$B$''. In their
setting, precision is defined for type constructors and subtyping for static
types. If we interpret this definition from AGT's point of view, finding a more
precise static type\footnote{The definition of precision of types is given in appendix. % It is a
  % partial order relation with $\unknown$ the least element. It is defined
  % structurally over type construction.
}
has the same effect as concretization. Namely, $A' \lessp A
$ implies $A \in \gamma(A')$ and $B' \lessp B$ implies $B \in \gamma(B')$.
Therefore we consider this definition as AGT-style. From this perspective, this
definition naturally coincides with \cref{def:decl-conssub}.

The value of their definition is that consistent subtyping is derived
compositionally from \textit{static subtyping} and \textit{precision}. These are
two more atomic relations. At first sight, their definition looks very similar
to \cref{def:decl-conssub} (replacing $\lessp$ by $<:$ and $<:$ by $\sim$). Then
a question arises as to \textit{which one is more fundamental}. To answer this,
we need to discuss the relation between consistency and precision.

\paragraph{Relating Consistency and Precision.}

Precision is a partial order (anti-symmetric and transitive), while
consistency is symmetric but not transitive. % One observation
% is that precision cares more about order (over precision), while this order can be mixed under
% the structure in consistency.
Nonetheless, precision and consistency are related by the following proposition:

\begin{proposition}[Consistency and Precision]\leavevmode
  \label{lemma:consistency-precision}
  \begin{itemize}
  \item If $A \sim B$,
    then there exists (static) $C$,
    such that $A \lessp C$,
    and $B \lessp C$.
  \item If for some (static) $C$,
    we have $A \lessp C$,
    and $B \lessp C$,
    then we have $A \sim B$.
  \end{itemize}
\end{proposition}

It may seem that precision is a more atomic relation, since consistency can be
derived from precision. However, recall that consistency is in fact an
equivalence relation lifted from static types to gradual types.
% , while precision
% is a subset relation over concretization of gradual
% types~\citep{garcia2016abstracting}.
Therefore defining consistency independently is straightforward, and it is
theoretically viable to validate the definition of consistency directly. On the
other hand, precision is usually connected with the gradual criteria
\citep{siek2015refined}, and finding a correct partial order that adheres to the
criteria is not always an easy task. For example, \citet{yuu2017poly} argued
that term precision for System $F_G$ is actually nontrivial, leaving the gradual
guarantee of the semantics as a conjecture. Thus precision can be difficult to
extend to more sophisticated type systems, e.g. dependent types.

Still, it is interesting that those two definitions illustrate the
correspondence of different foundations (on simple types): one is defined
directly on gradual types, and the other stems from AGT, which is based on
static subtyping.

\subsection{Consistent Subtyping Without Existentials}

\cref{def:decl-conssub} serves as a fine specification of how consistent
subtyping should behave in general. But it is inherently non-deterministic
because of the two intermediate types $C$ and $D$. As with
\cref{def:old-decl-conssub}, we need a combined relation to directly compare two
types. A natural attempt is to try to extend the restriction operator for
polymorphic types. Unfortunately, as we show below, this does not work. However
it is possible to devise an equivalent inductive definition instead.

\paragraph{Attempt to Extend the Restriction Operator.}
Suppose that we try to extend the restriction operator to account for polymorphic
types. The original restriction operator is structural, meaning that it works
for types of similar structures. But for polymorphic types, two input types
could have different structures due to universal quantifiers, e.g, $\forall a. a
\to \nat$ and $(\nat \to \unknown) \to \nat$. If we try to mask the first type
using the second, it seems hard to maintain the information that $a$ should be
instantiated to a function while ensuring that the return type is masked. There
seems to be no satisfactory way to extend the restriction operator in order to
support this kind of non-structural masking.

\paragraph{Interpretation of the Restriction Operator and Consistent Subtyping.}
If the restriction operator cannot be extended naturally, it is useful to
take a step back and revisit what the restriction operator actually does. For
consistent subtyping, two input types could have unknown types in different
positions, but we only care about the known parts. What the restriction
operator does is (1) erase the type information in one type if the corresponding
position in the other type is the unknown type; and (2) compare the resulting types
using the normal subtyping relation. The example below shows the
masking-off procedure for the types $\nat \to \unknown \to \bool$ and $\nat \to
\nat \to \unknown$. Since the known parts have the relation that $\nat \to
\unknown \to \unknown \tsub \nat \to \unknown \to \unknown$, we conclude that
$\nat \to \unknown \to \bool \tconssub \nat \to \nat \to \unknown$.
\begin{center}
  \begin{tikzpicture}
    \tikzstyle{column 5}=[anchor=base west, nodes={font=\tiny}]
    \matrix (m) [matrix of math nodes,row sep=0.5em,column sep=0em,minimum width=2em]
    {
      \nat \to & \unknown & \to & \bool & \mid \nat \to \nat \to \unknown &
      = \nat \to \unknown \to \unknown
      \\
       \nat \to & \nat & \to & \unknown & \mid \nat \to \unknown \to \bool &
      = \nat \to \unknown \to \unknown \\};

    \path[-stealth, ->, out=0, in=0]
    (m-1-6) edge node [right] {$\tsub$} (m-2-6);

    \draw
    (m-1-2.north west) rectangle (m-2-2.south east)
    (m-1-4.north west) rectangle (m-2-4.south east);
  \end{tikzpicture}
\end{center}
Here differences of the types in boxes are erased because of the restriction
operator. Now if we compare the types in boxes directly instead of through the
lens of the restriction operator, we can observe that the \textit{consistent
  subtyping relation always holds between the unknown type and an arbitrary
  type.} We can interpret this observation directly from
\cref{def:decl-conssub}: the unknown type is neutral to subtyping ($\unknown
\tsub \unknown$), the unknown type is consistent with any type ($\unknown \sim
A$), and subtyping is reflexive ($A \tsub A$). Therefore, \textit{the unknown
  type is a consistent subtype of any type ($\unknown \tconssub A$), and vice
  versa ($A \tconssub \unknown$).} Note that this interpretation provides a
general recipe on how to lift a (static) subtyping relation to a (gradual)
consistent subtyping relation, as discussed below.

\paragraph{Defining Consistent Subtyping Directly.}

From the above discussion, we can define the consistent subtyping relation
directly, \textit{without} resorting to subtyping or consistency at all. The key
idea is that we replace $\tsub$ with $\tconssub$ in
\cref{fig:decl:subtyping}, get rid of rule \rul{S-Unknown} and add two
extra rules concerning $\unknown$, resulting in the rules of consistent
subtyping in \cref{fig:decl:conssub}. Of particular interest are the rules
\rul{CS-UnknownL} and \rul{CS-UnknownR}, both of which correspond to what we
just said: the unknown type is a consistent subtype of any type, and vice versa.
\begin{figure}[t]
  \begin{small}
  \begin{mathpar}
    \framebox{$\tpresub A \tconssub B$} \\
    \CSForallR \and \CSForallL \and \CSFun \and
    \CSTVar \and \CSInt \and \CSUnknownL \and \CSUnknownR
  \end{mathpar}
  \end{small}
  \caption{Consistent Subtyping for implicit polymorphism.}
  \label{fig:decl:conssub}
\end{figure}
From now on, we use the symbol $\tconssub$ to refer to the consistent subtyping
relation in \cref{fig:decl:conssub}. What is more, we can prove that those two
are equivalent\footnote{Theorems with $\mathcal{T}$ are those
  proved in Coq. The same applies to $\mathcal{L}$emmas.}:

\begin{ctheorem}   \label{lemma:properties-conssub}
  $\tpreconssub A \tconssub B  \Leftrightarrow \tpresub A \tsub C$, $C \sim D$, $\tpresub D \tsub B$ for some $C, D$.
\end{ctheorem}

% \noindent Not surprisingly, consistent subtyping is reflexive:

% \begin{clemma}[Consistent Subtyping is Reflexive] \label{lemma:sub_refl}%
%   If $\Psi \vdash A$ then $\Psi \vdash A \tconssub A$.
% \end{clemma}





%%% Local Variables:
%%% mode: latex
%%% TeX-master: "../paper"
%%% org-ref-default-bibliography: ../paper.bib
%%% End:


\section{Declarative System}


\begin{center}
\begin{tabular}{lrcl} \toprule
  Types & $[[A]], [[B]]$ & \syndef & $[[int]] \mid [[a]] \mid [[A -> B]] \mid [[\/ a. A]] \mid [[unknown]] \mid [[static]] \mid [[gradual]] $ \\
  Monotypes & $[[t]], [[s]]$ & \syndef & $ [[int]] \mid [[a]] \mid [[t -> s]] \mid [[static]] \mid [[gradual]]$ \\
  Castable Types & $[[gc]]$ & \syndef & $ [[int]] \mid [[a]] \mid [[gc1 -> gc2]] \mid [[\/ a. gc]] \mid [[unknown]] \mid [[gradual]] $ \\
  Castable Monotypes & $[[tc]]$ & \syndef & $ [[int]] \mid [[a]] \mid [[tc1 -> tc2]] \mid [[gradual]]$ \\

  Contexts & $[[dd]]$ & \syndef & $[[empty]] \mid [[dd, x: A]] \mid [[dd, a]] $ \\
  Colored Types & $[[A]], [[B]]$ & \syndef & $ [[r@(int)]] \mid [[b@(int)]] \mid [[r@(a)]] \mid [[b@(a)]] \mid [[A -> B]] \mid [[r@ \/ a . A]] \mid [[b@ \/ a. A]] \mid [[b@(unknown)]] \mid [[r@(static)]] \mid [[r@(gradual)]] \mid [[b@(gradual)]]$\\
  Blue Castable Types & $[[b@(gc)]]$ & \syndef & $ [[b@(int)]] \mid [[b@(a)]] \mid [[b@(gc1) -> b@(gc2)]] \mid [[b@ \/ a. b@(gc)]] \mid [[b@(unknown)]] \mid [[b@(gradual)]] $ \\
  Blue Monotypes & $[[b@(t)]]$ & \syndef & $ [[b@(int)]] \mid [[b@(a)]] \mid [[b@(t -> s)]] \mid [[b@(gradual)]]$ \\
  Red Monotypes & $[[r@(t)]]$ & \syndef & $ [[r@(int)]] \mid [[r@(a)]] \mid [[ r@(t)  -> r@(s)]] \mid [[ r@(t) -> b@(s) ]] \mid [[ b@(t) ->  r@(s) ]] \mid [[r@(static)]] \mid [[r@(gradual)]]$ \\
  \bottomrule
\end{tabular}
\end{center}


\renewcommand\ottaltinferrule[4]{
  \inferrule*[narrower=0.7]
    {#3}
    {#4}
}

\drules[dconsist]{$ [[ A ~ B ]] $}{Type Consistent}{refl, unknownR, unknownL, arrow, forall}

\renewcommand\ottaltinferrule[4]{
  \inferrule*[narrower=0.7,right=\scriptsize{#1}]
    {#3}
    {#4}
}

\drules[s]{$ [[dd |- A <: B ]] $}{Subtyping}{forallR, forallLr, forallLb, tvarr, tvarb, intr, intb, arrow,
  unknown, spar, gparr, gparb}


% \begin{definition}[Specification of Consistent Subtyping]
%   \begin{mathpar}
%   \drule{cs-spec}
%   \end{mathpar}
% \end{definition}

\drules[cs]{$ [[dd |- A <~ B ]] $}{Consistent Subtyping}{forallR, forallL, arrow, tvar, int, unknownL, unknownR, spar, gpar}

\drules[]{$ [[dd |- e : A ~~> pe]] $}{Typing}{var, int, gen, lamann, lam, app}

\drules[m]{$ [[dd |- A |> A1 -> A2]] $}{Matching}{forall, arr, unknown}


\section{Target: PBC}

\begin{center}
\begin{tabular}{lrcl} \toprule
  Terms & $[[pe]]$ & \syndef & $[[x]] \mid [[n]] \mid [[\x : A. pe]] \mid [[/\a. pe]] \mid [[pe1 pe2]] \mid [[<A `-> B> pe]] $
  \\ \bottomrule
\end{tabular}
\end{center}


\clearpage
\section{Metatheory}

% \renewcommand{\hlmath}{}

\begin{definition}[Substitution]
  \begin{enumerate}
    \item Gradual type parameter substitution $\gsubst :: [[gradual]] \to [[tc]]$
    \item Static type parameter substitution $\ssubst :: [[static]] \to [[t]]$
    \item Type parameter Substitution $\psubst = \gsubst \cup \ssubst$
  \end{enumerate}
\end{definition}

\ningning{Note substitution ranges are monotypes.}

\begin{definition}[Translation Pre-order]
  Suppose $[[dd |- e : A ~~> pe1]]$ and $[[dd |- e : A ~~> pe2]]$,
  we define $[[pe1]] \leq [[pe2]]$ to mean $[[pe2]] = [[S(pe1)]]$ for
  some $[[S]]$.
\end{definition}


\begin{proposition}
  If $[[ pe1 ]] \leq [[pe2]]$ and $[[ pe2 ]] \leq [[pe1]]$, then $[[pe1]]$ and $[[pe2]]$
  are equal up to $\alpha$-renaming of type parameters.
\end{proposition}

 
\begin{definition}[Representative Translation]
  $[[pe]]$ is a representative translation of a typing derivation $[[dd |- e : A
  ~~> pe]]$ if and only if for any other translation $[[dd |- e : A ~~> pe']]$ such that $[[pe']]
  \leq [[pe]]$, we have $[[pe]] \leq [[pe']]$. From now on we use $[[rpe]]$ to
  denote a representative translation.
\end{definition}

\begin{definition}[Measurements of Translation]
  There are three measurements of a translation $[[pe]]$,
  \begin{enumerate}
  \item $[[ ||pe||e]]$, the size of the expression 
  \item $[[ ||pe||s ]]$, the number of distinct static type parameters in $[[pe]]$
  \item $[[ ||pe||g ]]$, the number of distinct gradual type parameters in $[[pe]]$
  \end{enumerate}
  We use $[[ ||pe|| ]]$ to denote the lexicographical order of the triple
  $([[ ||pe||e ]], -[[ ||pe||s ]], -[[ ||pe||g ]])$.
\end{definition}

\begin{definition}[Size of types]

  \begin{align*}
    [[ || int ||  ]] &= 1 \\
    [[ || a ||  ]] &= 1 \\
    [[ || A -> B  ||  ]] &= [[ || A || ]] + [[ || B || ]] + 1 \\
    [[ || \/a . A ||  ]] &= [[ || A || ]] + 1 \\
    [[ || unknown ||  ]] &= 1 \\
    [[ || static ||  ]] &= 1 \\
    [[ || gradual ||  ]] &= 1
  \end{align*}

\end{definition}


\begin{definition}[Size of expressions]

  \begin{align*}
    [[ || x ||e  ]] &= 1 \\
    [[ || n ||e  ]] &= 1 \\
    [[ || \x : A . pe ||e  ]] &= [[ || A || ]] + [[ || pe ||e ]] + 1 \\
    [[ || /\ a. pe ||e  ]] &= [[ || pe ||e ]] + 1 \\
    [[ || pe1 pe2 ||e  ]] &= [[ || pe1 ||e ]] + [[  || pe2 ||e ]] + 1 \\
    [[ || < A `-> B> pe ||e  ]] &= [[ || pe ||e ]] + [[  || A || ]] + [[  || B || ]] + 1 \\
  \end{align*}

\end{definition}


\begin{lemma} \label{lemma:size_e}
  If $[[dd |- e : A ~~> pe]]$ then $[[ || pe ||e    ]] \geq [[ || e ||e   ]]  $.
\end{lemma}
\begin{proof}
  Immediate by inspecting each typing rule.
\end{proof}

\begin{corollary} \label{lemma:decrease_stop}
  If $[[dd |- e : A ~~> pe]]$ then $[[ || pe ||   ]] > ([[ || e ||e ]], -[[ || e ||e ]], -[[ || e ||e ]] )  $.
\end{corollary}
\begin{proof}
  By \cref{lemma:size_e} and note that $ [[ || pe ||e   ]] > [[  || pe ||s  ]] $ and $ [[ || pe ||e   ]] > [[  || pe ||g  ]] $
\end{proof}


\begin{lemma} \label{lemma:type_decrease}
  $[[ || A || ]] \leq [[ || S(A) || ]]  $.
\end{lemma}
\begin{proof}
  By induction on the structure of $[[A]]$. The interesting cases are $[[ A ]] = [[static]]$ and
  $[[ A ]] = [[gradual]]$. When $[[ A ]] = [[static]]$, $[[ S(A) ]] = [[t]]$
  for some monotype $[[t]]$ and it is immediate that $[[ || static ||  ]]  \leq [[ || t || ]] $
  (note that $[[ || static ||  ]] < [[ || gradual ||  ]] $ by definition).
\end{proof}


\begin{lemma}[Substitution Decreases Measurement]
  \label{lemma:subst_dec_measure}
  If $[[pe1]] \leq [[pe2]]$, then $ {[[ ||pe1|| ]]} \leq [[ ||pe2|| ]]$; unless
  $[[pe2]] \leq [[pe1]]$ also holds, otherwise we have $[[ ||pe1|| ]] < [[ ||pe2|| ]]$.
\end{lemma}
\begin{proof}
  Since $[[ pe1  ]] \leq [[  pe2  ]]$, we know $[[ pe2  ]] = [[ S(pe1)  ]]$ for some $[[S]]$. By induction on
  the structure of $[[pe1]]$.

  \begin{itemize}
  \item Case $[[pe1]] = [[  \x : A . pe ]]$. We have
    $[[ pe2  ]] = [[  \x : S(A) . S(pe)  ]]$. By \cref{lemma:type_decrease} we have $[[ || A || ]] \leq [[ || S(A) || ]]$.
    By i.h., we have $[[ || pe ||  ]] \leq [[ || S(pe) ||  ]]$. Therefore $[[ || \x : A . pe ||    ]] \leq [[ || \x : S(A) . S(pe) ||  ]]$.
  \item Case $[[pe1]] = [[ < A `-> B > pe  ]]$. We have
    $[[pe2]] = [[ < S(A) `-> S(B) > S(pe)  ]]$.  By \cref{lemma:type_decrease} we have $[[ || A || ]] \leq [[ || S(A) || ]]$
    and $[[ || B || ]] \leq [[ || S(B) || ]]$. By i.h., we have $[[ || pe ||  ]] \leq [[ || S(pe) ||  ]]$.
    Therefore $[[  || < A `-> B > pe ||  ]] \leq [[ || < S(A) `-> S(B) > S(pe)  ||   ]]$.

  \item The rest of cases are immediate.
  \end{itemize}

\end{proof}


\begin{lemma}[Representative Translation for Typing]
  For any typing derivation that $[[dd |- e : A]]$, there exists at least one representative
  translation $r$ such that $[[dd |- e : A ~~> rpe]]$.
\end{lemma}
\begin{proof}
We already know that at least one translation $[[pe]] = [[pe1]]$ exists
for every typing derivation. If $[[pe1]]$ is a representative translation then we
are done. Otherwise there exists another translation $[[pe2]]$ such that
$[[pe2]] \leq [[pe1]]$ and $ [[pe1]] \not \leq [[pe2]]$. By
\cref{lemma:subst_dec_measure}, we have $[[||pe2||]] < [[ ||pe1|| ]]$. We continue
with $[[pe]] = [[pe2]]$, and get a strictly decreasing sequence $[[ || pe1 ||  ]], [[ || pe2 || ]], \dots$.
By \cref{lemma:decrease_stop}, we know this sequence cannot be infinite long. Suppose it ends at $[[ || pen || ]]$,
by the construction of the sequence, we know that $[[pen]]$ is a representative translation of $[[e]]$.
\end{proof}


\begin{conjecture}[Property of Representative Translation] \label{lemma:repr}
  If $[[empty |- e : A ~~> pe]]$, $\erasetp s \Downarrow v$, then we
  have $[[empty |- e : A ~~> rpe]]$, and $\erasetp r \Downarrow v'$.
\end{conjecture}

\ningning{shall we focus on values of type integer?}

\begin{definition}[Erasure of Type Parameters]
  \begin{center}
\begin{tabular}{p{5cm}l}
  $\erasetp \nat = \nat $ &
  $\erasetp a = a $ \\
  $\erasetp {A \to B} = \erasetp A \to \erasetp B $ &
  $\erasetp {\forall a. A} = \forall a. \erasetp A$ \\
  $\erasetp {\unknown} = \unknown  $&
  $\erasetp {\static} = \nat  $\\
  $\erasetp {\gradual} = \unknown  $\\
\end{tabular}

  \end{center}
\end{definition}


\begin{corollary}[Coherence up to cast errors]
  Suppose $[[ empty |- e : int ~~> pe1 ]]$ and $[[ empty |- e : int ~~> pe2 ]]$, if $| [[ pe1 ]] | \Downarrow [[n]]$
  then either $ | [[  pe2  ]] | \Downarrow n$ or $ | [[  pe2  ]] | \Downarrow \blamev$.
\end{corollary}
\jeremy{maybe Conjecture~\ref{lemma:repr} is enough to prove it? }


\begin{conjecture}[Dynamic Gradual Guarantee]
  Suppose $e' \lessp e$,
  \begin{enumerate}
  \item If $[[empty |- e : A ~~> rpe]]$, $\erasetp {r} \Downarrow v$,
    then for some $B$ and $r'$, we have $[[ empty |- e' : B ~~> rpe']]$,
    and $B \lessp A$,
    and $\erasetp {r'} \Downarrow v'$,
    and $v' \lessp v$.
  \item If $[[empty |- e' : B ~~> rpe']]$, $\erasetp {r'} \Downarrow v'$,
    then for some $A$ and $[[rpe]]$, we have $ [[empty |- e : A ~~> rpe]]$,
    and $B \lessp A$. Moreover,
    $\erasetp r \Downarrow v$ and $v' \lessp v$,
    or $\erasetp r \Downarrow \blamev$.
  \end{enumerate}
\end{conjecture}



\section{Efficient (Almost) Typed Encodings of ADTs}


\begin{itemize}
\item Scott encodings of simple first-order ADTs (e.g. naturals)
\item Parigot encodings improves Scott encodings with recursive schemes, but
  occupies exponential space, whereas Church encoding only occupies linear
  space.
\item An alternative encoding which retains constant-time destructors but also
  occupies linear space.
\item Parametric ADTs also possible?
\item Typing rules
\end{itemize}

\begin{example}[Scott Encoding of Naturals]
\begin{align*}
  [[nat]] &\triangleq [[  \/a. a -> (unknown -> a) -> a ]] \\
  \mathsf{zero} &\triangleq [[ \x . \f . x  ]] \\
  \mathsf{succ} &\triangleq [[ \y : nat . \x . \f . f y ]]
\end{align*}
\end{example}
Scott encodings give constant-time destructors (e.g., predecessor), but one has to
get recursion somewhere. Since our calculus admits untyped lambda calculus, we
could use a fixed point combinator.

\begin{example}[Parigot Encoding of Naturals]
\begin{align*}
  [[nat]] &\triangleq [[  \/a. a -> (unknown -> a -> a) -> a ]] \\
  \mathsf{zero} &\triangleq [[ \x . \f . x  ]] \\
  \mathsf{succ} &\triangleq [[ \y : nat . \x . \f . f y (y x f) ]]
\end{align*}
\end{example}
Parigot encodings give primitive recursion, apart form constant-time
destructors, but at the cost of exponential space complexity (notice in
$\mathsf{succ}$ there are two occurances of $[[y]]$).

Both Scott and Parigot encodings are typable in System F with positive recursive
types, which is strong normalizing.

\begin{example}[Alternative Encoding of Naturals]
\begin{align*}
  [[nat]] &\triangleq [[  \/a. a -> (unknown -> (unknown -> a) -> a) -> a ]] \\
  \mathsf{zero} &\triangleq [[ \x . \f . x  ]] \\
  \mathsf{succ} &\triangleq [[ \y : nat . \x . \f .  f y (\g . g x f) ]]
\end{align*}
\end{example}
This encoding enjoys constant-time destructors, linear space complexity, and
primitive recursion.
The static version is $[[ mu b . \/ a . a -> (b -> (b -> a) -> a) -> a ]]$,
which can only be expressed in System F with
general recursive types (notice the second $[[b]]$ appears in a negative position).





\section{Algorithmic System}

\begin{center}
\begin{tabular}{lrcl} \toprule
  Expressions & $[[ae]]$ & \syndef & $[[x]] \mid [[n]] \mid [[\x : aA . ae]] \mid [[\x . ae]] \mid [[ae1 ae2]] \mid [[ae : aA]] $ \\
  Existential variables & $[[evar]]$ & \syndef & $[[sa]]  \mid [[ga]]  $   \\
  Types & $[[aA]], [[aB]]$ & \syndef & $ [[int]] \mid [[a]] \mid [[evar]] \mid [[aA -> aB]] \mid [[\/ a. aA]] \mid [[unknown]] \mid [[static]] \mid [[gradual]] $ \\
  Static Types & $[[aT]]$ & \syndef & $ [[int]] \mid [[a]] \mid [[evar]] \mid [[aT1 -> aT2]] \mid [[\/ a. aT]] \mid [[static]] \mid [[gradual]] $ \\
  Monotypes & $[[at]], [[as]]$ & \syndef & $ [[int]] \mid [[a]] \mid [[evar]] \mid [[at -> as]] \mid [[static]] \mid [[gradual]]$ \\
  Castable Monotypes & $[[atc]]$ & \syndef & $ [[int]] \mid [[a]] \mid [[evar]] \mid [[atc1 -> atc2]] \mid [[gradual]]$ \\
  Castable Types & $[[agc]]$ & \syndef & $ [[int]] \mid [[a]] \mid [[evar]] \mid [[agc1 -> agc2]] \mid [[\/ a. agc]] \mid [[unknown]] \mid [[gradual]] $ \\
  Static Castable Types & $[[asc]]$ & \syndef & $ [[int]] \mid [[a]] \mid [[evar]] \mid [[asc1 -> asc2]] \mid [[\/ a. asc]] \mid [[gradual]] $ \\
  Contexts & $[[GG]], [[DD]], [[TT]]$ & \syndef & $[[empty]] \mid [[GG , x : aA]] \mid [[GG , a]] \mid [[GG , evar]] \mid [[GG, evar = at]] $ \\
  Complete Contexts & $[[OO]]$ & \syndef & $[[empty]] \mid [[OO , x : aA]] \mid [[OO , a]] \mid [[OO, evar = at]]$ \\ \bottomrule
\end{tabular}
\end{center}



\begin{definition}[Existential variable contamination] \label{def:contamination}
  \begin{align*}
    [[ [aA] empty    ]] &= [[empty]] \\
    [[ [aA] (GG, x : aA)  ]] &= [[ [aA] GG , x : aA     ]] \\
    [[ [aA] (GG, a)  ]] &= [[ [aA] GG , a     ]] \\
    [[ [aA] (GG, sa)  ]] &= [[ [aA] GG , ga , sa = ga  ]]  \quad \text{if $[[sa in fv(aA)]]$ }    \\
    [[ [aA] (GG, ga)  ]] &= [[ [aA] GG , ga     ]] \\
    [[ [aA] (GG, evar = at)  ]] &= [[ [aA] GG , evar = at     ]] \\
  \end{align*}
\end{definition}



\drules[ad]{$ [[GG |- aA ]] $}{Well-formedness of types}{int, unknown, static, gradual, tvar, evar, solvedEvar, arrow, forall}

\drules[wf]{$ [[ |- GG ]] $}{Well-formedness of algorithmic contexts}{empty, var, tvar, evar, solvedEvar}

\drules[as]{$ [[GG |- aA <~ aB -| DD ]] $}{Algorithmic Consistent Subtyping}{tvar, evar, int, arrow, forallR, forallL, spar, gpar, unknownL, unknownR, instL, instR}

\drules[instl]{$ [[ GG |- evar <~~ aA -| DD   ]] $}{Instantiation I}{solveS, solveG, solveUS, solveUG, reachSGOne, reachSGTwo, reachOtherwise, arr, forallR}

\drules[instr]{$ [[ GG |- aA <~~ evar -| DD   ]] $}{Instantiation II}{solveS, solveG, solveUS, solveUG, reachSGOne, reachSGTwo, reachOtherwise, arr, forallL}

\drules[inf]{$ [[ GG |- ae => aA -| DD ]] $}{Inference}{var, int, lamann, lam, anno, app}

\drules[chk]{$ [[ GG |- ae <= aA -| DD ]] $}{Checking}{lam, gen, sub}

\drules[am]{$ [[ GG |- aA |> aA1 -> aA2 -| DD ]] $}{Algorithmic Matching}{forall, arr, unknown, var}

\drules[ext]{$ [[ GG --> DD  ]] $}{Context extension}{id, var, tvar, evar, solvedEvar, solveS, solveG, add, addSolveS, addSolveG}



\clearpage


\section{Metatheory}

\begin{restatable}[Instantiation Soundness]{mtheorem}{instsoundness} \label{thm:inst_soundness}%
  Given $[[ DD --> OO ]]$ and $[[ [GG]aA = aA ]]$ and  $[[evar notin fv(aA)]]$:

  \begin{enumerate}
  \item If $[[GG |- evar <~~  aA -| DD ]]$ then $[[  [OO]DD |- [OO]evar <~ [OO]aA  ]] $.
  \item If $[[GG |- aA <~~ evar -| DD ]]$ then $[[  [OO]DD |- [OO]aA <~ [OO]evar  ]] $.
  \end{enumerate}
\end{restatable}


\begin{restatable}[Soundness of Algorithmic Consistent Subtyping]{mtheorem}{subsoundness} \label{thm:sub_soundness}%
  If $[[  GG |- aA <~ aB -| DD ]]$ where $[[ [GG]aA = aA  ]]$ and $[[  [GG] aB = aB  ]]$ and $[[  DD --> OO ]]$ then
  $[[  [OO]DD |- [OO]aA <~ [OO]aB   ]]$.
\end{restatable}



\begin{restatable}[Soundness of Algorithmic Typing]{mtheorem}{typingsoundness} \label{thm:type_sound}%
  Given $[[DD --> OO]]$:
  \begin{enumerate}
  \item If $[[  GG |- ae => aA -| DD  ]]$ then $\exists [[e']]$ such that $ [[  [OO]DD |- e' : [OO] aA  ]]   $ and $\erase{[[ae]]} = \erase{[[e']]}$.
  \item If $[[  GG |- ae <= aA -| DD  ]]$ then $\exists [[e']]$ such that $ [[  [OO]DD |- e' : [OO] aA  ]]   $ and $\erase{[[ae]]} = \erase{[[e']]}$.
  \end{enumerate}
\end{restatable}

\begin{restatable}[Instantiation Completeness]{mtheorem}{instcomplete} \label{thm:inst_complete}
  Given $[[GG --> OO]]$ and $[[aA = [GG]aA]]$ and $[[evar]] \notin \textsc{unsolved}([[GG]]) $ and $[[  evar notin fv(aA)  ]]$:
  \begin{enumerate}
  \item If $[[ [OO]GG |- [OO] evar <~ [OO]aA   ]]$ then there are $[[DD]], [[OO']]$ such that $[[OO --> OO']]$
    and $[[DD --> OO']]$ and $[[GG |- evar <~~ aA -| DD]]$.
  \item If $[[ [OO]GG |- [OO]aA  <~ [OO] evar  ]]$ then there are $[[DD]], [[OO']]$ such that $[[OO --> OO']]$
    and $[[DD --> OO']]$ and $[[GG |- aA <~~ evar -| DD]]$.
  \end{enumerate}

\end{restatable}


\begin{restatable}[Generalized Completeness of Consistent Subtyping]{mtheorem}{subcomplete} \label{thm:sub_completeness}
  If $[[ GG --> OO  ]]$ and $[[ GG |- aA  ]]$ and $[[ GG |- aB  ]]$ and $[[  [OO]GG |- [OO]aA <~ [OO]aB  ]]$ then
  there exist $[[DD]]$ and $[[OO']]$ such that $[[DD --> OO']]$ and $[[OO --> OO']]$ and $[[  GG |- [GG]aA <~ [GG]aB -| DD ]]$.
\end{restatable}


\begin{restatable}[Matching Completeness]{mtheorem}{matchcomplete} \label{thm:match_complete}%
  Given $[[ GG --> OO  ]]$ and $[[ GG |- aA  ]]$, if
  $[[ [OO]GG |- [OO]aA |> A1 -> A2  ]]$
  then there exist $[[DD]]$, $[[OO']]$, $[[aA1']]$ and $[[aA2']]$ such that $[[ GG |- [GG]aA |> aA1' -> aA2' -| DD   ]]$
  and $[[ DD --> OO'  ]]$ and $[[ OO --> OO'  ]]$ and $[[A1]] = [[ [OO']aA1'  ]]$ and $[[A2]] = [[ [OO']aA2'  ]]$.
\end{restatable}



\begin{restatable}[Completeness of Algorithmic Typing]{mtheorem}{typingcomplete}
  Given $[[GG --> OO]]$ and $[[GG |- aA]]$, if $[[ [OO]GG |- e : A ]]$ then there exist $[[DD]]$, $[[OO']]$, $[[aA']]$ and $[[ae']]$
  such that $[[DD --> OO']]$ and $[[OO --> OO']]$ and $[[  GG |- ae' => aA' -| DD  ]]$ and $[[A]] = [[ [OO']aA'  ]]$ and $\erase{[[e]]} = \erase{[[ae']]}$.
\end{restatable}


\section{Algorithmic Type System}
\label{sec:algorithm}

\begin{figure}[t]
  \centering
  \begin{small}
\begin{tabular}{lrcl} \toprule
  Expressions & $e$ & \syndef & $x \mid n \mid
                         \blam x A e \mid \erlam x e \mid e~e \mid e : A $ \\
  Types & $A, B$ & \syndef & $ \nat \mid a \mid \genA \mid A \to B \mid \forall a. A \mid \unknown$ \\
  Monotypes & $\tau, \sigma$ & \syndef & $ \nat \mid a \mid \genA \mid \tau \to \sigma$ \\
  Contexts & $\Gamma, \Delta, \Theta$ & \syndef & $\ctxinit \mid \tctx,x: A \mid \tctx, a \mid \tctx, \genA \mid \tctx, \genA = \tau$ \\
  Complete Contexts & $\Omega$ & \syndef & $\ctxinit \mid \Omega,x: A \mid \Omega, a \mid \Omega, \genA = \tau$ \\ \bottomrule
\end{tabular}
  \end{small}
\caption{Syntax of the algorithmic system}
\label{fig:algo-syntax}
\end{figure}


% The declarative type system in \cref{sec:type-system} serves as a good
% specification for how typing should behave. It remains to see whether this
% specification delivers an algorithm. The main challenge lies in the rules \rul{CS-ForallL} in
% \cref{fig:decl:conssub} and rule \rul{M-Forall} in
% \cref{fig:decl-typing}, which both need to guess a monotype.

% \bruno{why are we not highlightinh the differences in gray anymore?}
In this section we give a bidirectional account of the algorithmic type system
that implements the declarative specification. The algorithm is largely inspired
by the algorithmic bidirectional system of \citet{dunfield2013complete}
(henceforth DK system). However our algorithmic system differs from theirs in
three aspects: 1) the addition of the unknown type $\unknown$; 2) the use of the
matching judgment; and 3) the approach of \textit{gradual inference only
  producing static types}~\citep{garcia2015principal}. We then prove that our
algorithm is both sound and complete with respect to the declarative type
system. Full proofs can be found in the appendix.

\paragraph{Algorithmic Contexts.}

The algorithmic context $\Gamma$ is an
\textit{ordered} list containing declarations of type variables $a$ and term
variables $x : A$. Unlike declarative contexts, algorithmic contexts also
contain declarations of existential type variables $\genA$, which can be either
unsolved (written $\genA$) or solved to some monotype (written $\genA = \tau$).
Complete contexts $\Omega$ are those that contain no unsolved existential type
variables. \Cref{fig:algo-syntax} shows the syntax of the algorithmic system.
Apart from expressions in the declarative system, we have annotated expressions
$e : A$.

% \paragraph{Notational convenience}
% Following \citet{dunfield2013complete}, we use contexts as substitutions on
% types. We write $\ctxsubst{\Gamma}{A}$ to mean $\Gamma$ applied as a
% substitution to type $A$. We also use a hole notation, which is useful when
% manipulating contexts by inserting and replacing declarations in the middle. The
% hole notation is used extensively in proving soundness and completeness. For
% example, $\Gamma[\Theta]$ means $\Gamma$ has the form $\Gamma_L, \Theta,
% \Gamma_R$; if we have $\Gamma[\genA] = (\Gamma_L, \genA, \Gamma_R)$, then
% $\Gamma[\genA = \tau] = (\Gamma_L, \genA = \tau, \Gamma_R)$.

% \paragraph{Input and output contexts}
% The algorithmic system, compared with the declarative system, includes similar
% judgment forms, except that we replace the declarative context $\Psi$ with an
% algorithmic context $\Gamma$ (the \textit{input context}), and add an
% \textit{output context} $\Delta$ after a backward turnstile. For example,
% $\Gamma \vdash A \tconssub B \dashv \Delta$ is the judgment form for the
% algorithmic consistent subtyping, and so on. All rules manipulate input and
% output contexts in a way that is consistent with the notion of \textit{context
%   extension}, which is described in \cref{sec:ctxt:extension}.

% We start with the explanation of the algorithmic consistent subtyping as it
% involves manipulating existential type variables explicitly (and solving them if
% possible).

\subsection{Algorithmic Consistent Subtyping and Instantiation}
\label{sec:algo:subtype}

\begin{figure}[t]
  \centering
  \begin{small}
  %   \begin{mathpar}
  % \framebox{$\Gamma \vdash A$} \\
  % \VarWF \and \IntWF \and \UnknownWF \and \FunWF \and \ForallWF \and \EVarWF
  % \and \SolvedEVarWF
  %   \end{mathpar}

\begin{mathpar}
  \framebox{$\Gamma \vdash A \tconssub B \toctxr$} \\
  \ACSTVar \and \ACSExVar \and \ACSInt \quad \ACSUnknownL \quad \ACSUnknownR \and
  \ACSFun \and \ACSForallR \and \ACSForallL \and \AInstantiateL \quad \AInstantiateR
\end{mathpar}
  \end{small}
  \caption{Algorithmic consistent subtyping}
  \label{fig:algo:subtype}
\end{figure}

\Cref{fig:algo:subtype} shows the algorithmic consistent subtyping rules.
The first five rules do not manipulate contexts. % Rules \rul{ACS-TVar} and
% \rul{ACS-Int} do not involve existential variables, so the output context
% remains unchanged. Rule \rul{ACS-ExVar} says that any unsolved existential
% variable is a consistent subtype of itself. The output is still the same as the
% input context as this gives no clue as to what is the solution of that
% existential variable.
% Rules \rul{ACS-UnknownL} and \rul{ACS-UnknownR} are the verbatim
% correspondences of rule \rul{CS-UnknownL} and \rul{CS-UnknownR}.
Rule \rul{ACS-Fun} is a natural extension of its declarative counterpart. The
output context of the first premise is used by the second premise, and the
output context of the second premise is the output context of the conclusion.
Note that we do not simply check $A_2 \tconssub B_2$, but apply $\Theta$
% (the input context of the second premise)
to both types (e.g., $\ctxsubst{\Theta}{A_2} $). This is
to maintain an important invariant that types
% : whenever we try to derive $\Gamma \vdash A \tconssub B \dashv \Delta$, the types $A$ and $B$
are fully applied
under input context $\Gamma$ (they contain no existential variables already solved in
$\Gamma$). The same invariant applies to every algorithmic judgment.
Rule \rul{ACS-ForallR} looks similar to its declarative counterpart, except that
we need to drop the trailing context $a, \Theta$ from the concluding output
context since they become out of scope.
% again, bears a similarity with the declarative
% version. Note that the output context of its premise allows additional elements
% to appear after the type variable $a$, in a trailing context $\Theta$. Since $a$
% becomes out of scope in the conclusion, we need to drop the trailing context
% $\Theta$ together with $a$ from the concluding output context, resulting in
% $\Delta$.
% The next rule is essential to eliminating the guessing work, thus appears
% significantly different from its declarative version. Instead of guessing a
% monotype $\tau$ out of thin air,
Rule \rul{ACS-ForallL} generates a fresh
existential variable $\genA$, and replaces $a$ with $\genA$ in the body $A$. The
new existential variable $\genA$ is then added to the premise's input context.
% Unlike rule \rul{ACS-ForallR}, the output context $\Delta$ of the premise
% remains unchanged in the conclusion.
% A central idea behind this rule is that we
% defer the decision of choosing a monotype for a type variable, and hope that it
% could be solved later when we have more information at hand.
As a side note, when both types are quantifiers, then either \rul{ACS-ForallR}
or \rul{ACS-ForallR} could be tried. In practice, one can apply
\rul{ACS-ForallR} eagerly.
The last two rules % are specific to the algorithm, thus having no counterparts in
% the declarative version. They
together check consistent subtyping with an
unsolved existential variable on one side and an arbitrary type on the other
side by the help of the instantiation judgment. % Apart from checking that the existential variable does not occur in the
% type $A$, both of the rules do not directly solve the existential variables, but
% leave the real work to the instantiation judgment.

% \subsection{Instantiation}
% \label{sec:algo:instantiate}

\begin{figure}[t]
  \centering
  \begin{small}
\begin{mathpar}
  \framebox{$\tctx \vdash \genA \unif A \toctxr$} \\
  % {\quad \text{Under input context $\Gamma$, instantiate $\genA$ such that
  %     $\genA \tconssub A$, with output context $\Delta$ }} \\
  \InstLSolve \and \InstLReach \and \InstLSolveU   \and \InstLAllR \and \InstLArr
\end{mathpar}

% \begin{mathpar}
%   \framebox{$\tctx \vdash A \unif \genA  \toctxr$} \\
%   % {\quad \text{Under input context $\Gamma$, instantiate $\genA$ such that
%   %     $A \tconssub \genA$, with output context $\Delta$}} \\
%   \InstRSolve \and \InstRReach \and \InstRSolveU  \and \InstRAllL \and \InstRArr
% \end{mathpar}

  \end{small}
  \caption{Algorithmic instantiation}
  \label{fig:algo:instantiate}
\end{figure}

% A central idea of the algorithmic system is to defer the decision of picking a
% monotype to as late as possible.
The judgment $\Gamma \vdash \genA \unif A \dashv \Delta$ defined in
\cref{fig:algo:instantiate} instantiates unsolved existential variables.
Judgment $\genA \unif A$ reads ``instantiate $\genA$ to a consistent subtype of
$A$''. For space reasons, we omit its symmetric judgement $\Gamma \vdash A \unif
\genA \dashv \Delta$.
% Since these two are mutually defined, we
% discuss them together, and omit symmetric rules when convenient.
Rule \rul{InstLSolve} and rule \rul{InstLReach} set $\genA$ to
$\tau$ and $\genB$ in the output context, respectively.
% is the simplest
% one -- when an existential variable meets a monotype. In that case, we simply
% set the solution of $\genA$ to the monotype $\tau$ in the output context. We
% also need to check that the monotype $\tau$ is well-formed under the prefix
% context $\Gamma$.
Rule \rul{InstLSolveU} is similar to \rul{ACS-UnknownR} in that we put no
constraint on $\genA$ when it meets the unknown type $\unknown$. This design
decision reflects the point that type inference only produces static
types~\citep{garcia2015principal}. We will get back to this point in
\cref{subsec:algo:discuss}.
% Rule \rul{InstLReach} deals with the situation where two existential variables
% meet. Note that $\Gamma[\genA][\genB]$ denotes a context where some unsolved existential
% variable $\genA$ is declared before $\genB$. In this situation, the only logical
% thing we can do is to set the solution of one existential variable to the other
% one, depending on which is declared before which. For example, in the output
% context of rule \rul{InstLReach}, we have $\genB = \genA$ because in the input
% context, $\genA$ is declared before $\genB$.
Rule \rul{InstLAllR} is the instantiation version of rule \rul{ACS-ForallR}.
% Since our system is predicative, $\genA$ cannot be instantiated to $\forall b.
% B$, but we can decompose $\forall b. B$ in the same way as in \rul{ACS-ForallR}.
% Rule \rul{InstRAllL} is the instantiation version of rule \rul{ACS-ForallL}.
The last rule \rul{InstLArr} applies when $\genA$ meets a function type. It
follows that the solution must also be a function type.
% looks a bit complicated, but it is actually very
% intuitive: what does the solution of $\genA$ look like when $A$ is a function
% type? The solution must also be a function type!
That is why, in the first premise, we generate two fresh existential variables
$\genA_1$ and $\genA_2$, and insert them just before $\genA$ in the input
context, so that the solution of $\genA$ can mention them. Note that $A_1 \unif
\genA_1$ switches to the other instantiation judgment.


% \paragraph{Example}

% We show a derivation of $\Gamma[\genA] \vdash \forall b. b \to \unknown \unif
% \genA$ to demonstrate the interplay between instantiation, quantifiers and the
% unknown type:
% \[
%   \inferrule*[right=InstRAllL]
%       {
%         \inferrule*[right=InstRArr]
%         {
%           \inferrule*[right=InstLReach]{ }{\Gamma', \genB \vdash \genA_1 \unif \genB \dashv \Gamma' , \genB = \genA_1} \\
%           \inferrule*[right=InstRSolveU]{ }{\Gamma', \genB = \genA_1 \vdash \unknown \unif \genA_2 \dashv \Gamma', \genB = \genA_1}
%         }
%         {
%           \Gamma[\genA], \genB \vdash \genB \to \unknown \unif \genA \dashv \Gamma', \genB = \genA_1
%         }
%       }
%       {
%         \Gamma[\genA] \vdash \forall b. b \to \unknown \unif \genA \dashv \Gamma', \genB = \genA_1
%       }
% \]
% where $\Gamma' = \Gamma[\genA_2, \genA_1, \genA = \genA_1 \to \genA_2]$. Note
% that in the output context, $\genA$ is solved to $\genA_1 \to \genA_2$, and
% $\genA_2$ remains unsolved because the unknown type $\unknown$ puts no
% constraint on it. Essentially this means that the solution of $\genA$ can be any
% function, which is intuitively correct since $\forall b. b \to \unknown$ can be
% interpreted, from the parametricity point of view, as any function.

\subsection{Algorithmic Typing}
\label{sec:algo:typing}

\begin{figure}[t]
  \centering
  \begin{small}
\begin{mathpar}
  \framebox{$\Gamma \vdash e \Rightarrow A \toctxr $} \\
  % {\quad \text{Under input context $\Gamma$, $e$ synthesizes output type $A$,
  %     with output context $\Delta$}} \\
  \AVar \and \ANat \and \ALamU \and \ALamAnnA \and \AAnno \and \AApp
\end{mathpar}
\begin{mathpar}
  \framebox{$\Gamma \vdash e \Leftarrow A \toctxr $} \\
  % {\quad \text{Under input context $\Gamma$, $e$ synthesizes output type $A$,
  %     with output context $\Delta$}} \\
  \ALam \and \AGen \and \ASub
\end{mathpar}
\begin{mathpar}
  \framebox{$\Gamma \vdash A \match A_1 \to A_2 \toctxr$} \\
  % {\quad \text{Under input context $\Gamma$, $A$ synthesizes output type $A_1
  %     \to A_2$, with output context $\Delta$}} \\
  \AMMC \quad \AMMA \and \AMMB \and \AMMD
\end{mathpar}
  \end{small}
  \caption{Algorithmic typing}
  \label{fig:algo:typing}
\end{figure}

We now turn to the algorithmic typing rules in \cref{fig:algo:typing}. The
algorithmic system uses bidirectional type checking to accommodate polymorphism.
Most of them are quite standard.
% All of them are direct analogies of their declarative counterparts. Rules \rul{AVar}
% and \rul{ANat} do not generate any new information, thus the output context is
% the same as the input context. Rule \rul{ALamAnnA} infers the type of a lambda
% abstraction. It does so by pushing $x : A$ into the input context and continues
% to infer the type of the body $B$. The output context in the premise has
% additional declarations in the trailing context $\Theta$, which is discarded in
% the concluding output context.
Perhaps rule \rul{AApp} (which differs significantly from that in the DK system)
deserves attention. It relies on the algorithmic matching judgment $\Gamma
\vdash A \match A_1 \to A_2 \dashv \Delta$.
% The matching judgment
% algorithmically synthesizes a function type from an arbitrary type.
Rule
\rul{AM-ForallL} replaces $a$ with a fresh existential variable $\genA$, thus
eliminating guessing. Rule \rul{AM-Arr} and \rul{AM-Unknown} correspond
directly to the declarative rules.
% self-explanatory. Rule
% \rul{AM-Unknown} says that the unknown type $\unknown$ can be split into a
% function type $\unknown \to \unknown$.
Rule \rul{AM-Var}, which has no
corresponding declarative version, is similar to \rul{InstRArr}/\rul{InstLArr}:
we create $\genA$ and $\genB$ and add $\genC = \genA \to \genB$ to the context.

% Back to \rul{AApp}. This rule first infers the type of $e_1$, producing a output
% context $\Theta_1$. Then it applies $\Theta_1$ to $A$ and goes into the matching
% judgment, which delivers a function type $A_1 \to A_2$ and another output
% context $\Theta_2$. $\Theta_2$ is used as the input context when inferring the
% type of $e_2$. The last premise algorithmically checks if
% $\ctxsubst{\Theta_3}{A_3}$ is a consistent subtype of
% $\ctxsubst{\Theta_3}{A_1}$. $A_2$ and $\Delta$ are the concluding output type
% and the concluding output context, respectively.


% \section{Soundness and Completeness}
% \label{sec:sound:complete}

% To be confident that our algorithmic type system and the declarative type system
% accept exactly the same programs, we need to prove that the algorithmic rules
% are sound and complete with respect to the declarative specifications. Before we
% give the formal statements of the soundness and completeness theorems, we need a
% meta-theoretical device, called \textit{context extension}~\cite{dunfield2013complete}, to help capture a notion of
% information increase from input contexts to output contexts.

% \subsection{Context Extension}
% \label{sec:ctxt:extension}


% A context extension judgment $\Gamma \exto \Delta$ reads ``$\Gamma$ is extended
% by $\Delta$''. Intuitively, this judgment says that $\Delta$ has at least as
% much information as $\Gamma$: some unsolved existential variables in $\Gamma$
% may be solved in $\Delta$. (The full inductive definition can be found in the
% supplementary material. We refer the reader to \citet[][Section
% 4]{dunfield2013complete} for further explanations of context extension.)

\subsection{Completeness and Soundness}

We prove that the algorithmic rules are sound and complete with
respect to the declarative specifications. We need an auxiliary judgment
$\Gamma \exto \Delta$ that captures a notion of information increase from input
contexts $\Gamma$ to output contexts $\Delta$~\citep{dunfield2013complete}.

\paragraph{Soundness.} Roughly speaking, soundness of the algorithmic system says
that given an expression $e$ that type checks in the algorithmic system, there exists
a corresponding expression $e'$ that type checks in the declarative system.
However there is one complication: $e$ does not necessarily have more annotations
than $e'$. For example, by \rul{ALam} we have $\erlam{x}{x} \chkby (\forall a.
a) \rightarrow (\forall a . a)$, but $\erlam{x}{x}$ itself cannot have type
$(\forall a. a) \rightarrow (\forall a . a)$ in the declarative system. To
circumvent that, we add an annotation to the lambda abstraction, resulting in
$\blam{x}{(\forall a . a)}{x}$, which is typeable in the declarative system with
the same type. To relate $\erlam{x}{x}$ and $\blam{x}{(\forall a . a)}{x}$, we
erase all annotations on both expressions. The definition of erasure $\erase{\cdot}$ is
standard and thus omitted.

% \jeremy{mention erasure and why (talk about \rul{ALam} and \rul{ASub})}


% \begin{restatable}[Instantiation Soundness]{mtheorem}{instsoundness} \label{thm:inst_soundness}%
%   Given $\Delta \exto \Omega$ and $\ctxsubst{\Gamma}{A} = A$ and $\genA \notin \mathit{fv}(A)$:
%   \begin{itemize}
%   \item If $\Gamma \vdash \genA \unif A \dashv \Delta$ then $\ctxsubst{\Omega}{\Delta} \vdash \ctxsubst{\Omega}{\genA} \tconssub \ctxsubst{\Omega}{A}$.
%   \item If $\Gamma \vdash A \unif \genA \dashv \Delta$ then $\ctxsubst{\Omega}{\Delta} \vdash \ctxsubst{\Omega}{A} \tconssub \ctxsubst{\Omega}{\genA}$.
%   \end{itemize}
% \end{restatable}

% Notice that the declarative judgment uses $\ctxsubst{\Omega}{\Delta}$, a
% operation that applies a complete context $\Omega$ to the algorithmic context
% $\Delta$, essentially plugging in all known solutions and removing all
% declarations of existential variables (both solved and unsolved), resulting in a
% declarative context.

% With instantiation soundness, next we show that the algorithmic consistent
% subtyping is sound:

% \begin{restatable}[Soundness of Algorithmic Consistent Subtyping]{mtheorem}{subsoudness} \label{thm:sub_soundness}%
%   If $\Gamma \vdash A \tconssub B \toctxr$ where $\ctxsubst{\tctx}{A} = A$ and
%   $\ctxsubst{\tctx}{B} = B$ and $\ctxr \exto \cctx$ then
%   $\ctxsubst{\cctx}{\Delta} \vdash \ctxsubst{\cctx}{A} \tconssub
%   \ctxsubst{\cctx}{B}$.
% \end{restatable}

% At this point, we are ``two thirds of the way'' to proving the ultimate theorem.
% The remaining third concerns with the soundness of matching:

% \begin{restatable}[Matching Soundness]{mtheorem}{matchsoundness}  \label{thm:match_soundness}%
%   If $\Gamma \vdash A \match A_1 \to A_2 \dashv \Delta$ where
%   $\ctxsubst{\Gamma}{A} = A$ and $\Delta \exto \Omega$ then
%   $\ctxsubst{\Omega}{\Delta} \vdash \ctxsubst{\Omega}{A} \match
%   \ctxsubst{\Omega}{A_1} \to \ctxsubst{\Omega}{A_2}$.
% \end{restatable}


% Finally the soundness theorem of algorithmic typing is:

\begin{restatable}[Soundness of Algorithmic Typing]{mtheorem}{typingsoundness} \label{thm:type_sound}
  Given $\ctxr \exto \cctx$,

  \begin{enumerate}
  \item If $\Gamma \vdash e \infto A \toctxr$ then $\exists e'$ such
    that $\ctxsubst{\cctx}{\Delta} \vdash e' : \ctxsubst{\cctx}{A}$ and
    $\erase{e} = \erase{e'}$.
  \item If $\Gamma \vdash e \chkby A \toctxr$ then $\exists e'$ such
    that $\ctxsubst{\cctx}{\Delta} \vdash e' : \ctxsubst{\cctx}{A}$ and
    $\erase{e} = \erase{e'}$.
  \end{enumerate}


\end{restatable}


\paragraph{Completeness.}
Completeness of the algorithmic system is the reverse of soundness: given a
declarative judgment of the form $\ctxsubst{\Omega}{\Gamma} \vdash
\ctxsubst{\Omega} \dots $, we want to get an algorithmic derivation of $\Gamma
\vdash \dots \dashv \Delta$. It turns out that completeness is a bit trickier to
state in that the algorithmic rules generate existential variables on the fly,
so $\Delta$ could contain unsolved existential variables that are not found in
$\Gamma$, nor in $\Omega$. Therefore the completeness proof must produce another
complete context $\Omega'$ that extends both the output context $\Delta$, and
the given complete context $\Omega$. As with soundness, we need erasure to
relate both expressions.

% \jeremy{talk about \rul{Gen}}

% \begin{restatable}[Instantiation Completeness]{mtheorem}{instcomplete}  \label{thm:inst_complete}%
%   Given $\Gamma \exto \Omega$ and $A = \ctxsubst{\Gamma}{A}$ and $\genA \in
%   \mathit{unsolved}(\Gamma)$ and $\genA \notin \mathit{fv}(A)$:
%   \begin{enumerate}
%   \item If $\ctxsubst{\Omega}{\Gamma} \vdash \ctxsubst{\Omega}{\genA} \tconssub
%     \ctxsubst{\Omega}{A}$ then there exist $\Delta$, $\Omega'$ such that $\Omega \exto
%     \Omega'$ and $\Delta \exto \Omega'$ and $\Gamma \vdash \genA \unif A \dashv \Delta$.
%   \item If $\ctxsubst{\Omega}{\Gamma} \vdash \ctxsubst{\Omega}{A} \tconssub
%     \ctxsubst{\Omega}{\genA}$ then there exist $\Delta$, $\Omega'$ such that $\Omega \exto
%     \Omega'$ and $\Delta \exto \Omega'$ and $\Gamma \vdash A \unif \genA \dashv \Delta$.
%   \end{enumerate}
% \end{restatable}


% Next is the completeness of consistent subtyping:

% \begin{restatable}[Generalized Completeness of Subtyping]{mtheorem}{subcomplete}  \label{thm:sub_completeness}%
%   If $\Gamma \exto \Omega$ and $\Gamma \vdash A$ and $\Gamma \vdash B$ and
%   $\ctxsubst{\Omega}{\Gamma} \vdash \ctxsubst{\Omega}{A} \tconssub
%   \ctxsubst{\Omega}{B}$ then there exist $\Delta$, $\Omega'$ such that $\Delta
%   \exto \Omega'$ and $\Omega \exto \Omega'$ and $\Gamma \vdash
%   \ctxsubst{\Gamma}{A} \tconssub \ctxsubst{\Gamma}{B \dashv \Delta}$.
% \end{restatable}


% We prove that the algorithmic matching is complete with respect to the
% declarative matching:

% \begin{restatable}[Matching Completeness]{mtheorem}{matchcomplete} \label{thm:match_complete}%
%   Given $\Gamma \exto \Omega$ and $\Gamma \vdash A$, if
%   $\ctxsubst{\Omega}{\Gamma} \vdash \ctxsubst{\Omega}{A} \match A_1 \to A_2$
%   then there exist $\Delta$, $\Omega'$, $A_1'$ and $A_2'$ such that $\Gamma
%   \vdash \ctxsubst{\Gamma}{A} \match A_1' \to A_2' \dashv \Delta$ and $\Delta \exto \Omega'$ and
%   $\Omega \exto \Omega'$ and $A_1 = \ctxsubst{\Omega'}{A_1'}$ and $A_2 =
%   \ctxsubst{\Omega'}{A_2'}$.
% \end{restatable}


% Finally here is the completeness theorem of the algorithmic typing:

\begin{restatable}[Completeness of Algorithmic Typing]{mtheorem}{typingcomplete}  \label{thm:type_complete}
  Given $\Gamma \exto \Omega$ and $\Gamma \vdash A $, if
  $\ctxsubst{\Omega}{\Gamma} \vdash e : A$ then there exist $\Delta$,
  $\Omega'$, $A'$ and $e'$ such that $\Delta \exto \Omega'$ and $\Omega \exto \Omega'$
  and $\Gamma \vdash e' \infto A' \dashv \Delta$ and $A = \ctxsubst{\Omega'}{A'}$ and $\erase{e} = \erase{e'}$.
\end{restatable}





%%% Local Variables:
%%% mode: latex
%%% TeX-master: "../paper"
%%% org-ref-default-bibliography: "../paper.bib"
%%% End:


\section{Discussion}
\label{sec:discussion}

In this section we consider a simple extension to the language to further
demonstrate the applicability of our definition of consistent subtyping. We also
discuss the design decisions involved in the algorithmic system and show how it
helps address the issue arising from the ambiguity of the type-directed
translation.

\subsection{Extension with Top}
\label{subsec:extension-top}

We argued that our definition of consistent subtyping (\Cref{def:decl-conssub})
is a \textit{general} definition in that it is independent of language features.
We have shown its applicability to polymorphic types, for which neither
\citet{siek2007gradual} nor the AGT approach~\citep{garcia2016abstracting} can
be extended naturally. To strengthen our argument, we consider extending the
language with the $\tope$ type and show that our approach naturally embraces the
extension, with all the desired properties preserved. To aid comparison, we also
show how to adapt the AGT approach to support $\tope$ and verify that these two
approaches, though rooted in different foundations, coincide again on
\textit{simple types}. However, \Cref{def:old-decl-conssub} of
\citet{siek2007gradual} fails to support $\tope$.


\paragraph{Extending Definitions}

In order to preserve the orthogonality between subtyping and consistency, we
require $\top$ to be a common supertype of all static types, as shown in rule
\rul{S-Top}. This rule might seem strange at first glance, since even
if we remove
the requirement $A~static$, the rule seems reasonable.
However, the important point is that because of the orthogonality between
subtyping and consistency, subtyping itself should not contain a potential cast
in principle! Therefore, subtyping instances such as $\unknown \tsub \top$ are not allowed.
For consistency, we add the rule that $\top$ is consistent with $\top$, which is
actually included in the original reflexive rule $A \sim A$. For consistent
subtyping, every type is a consistent subtype of $\top$, for example, $\nat \to
\unknown \tconssub \top$.
\begin{mathpar}
  \SubTop \and \CTop \and \CSTop
\end{mathpar}
It is easy to verify that \Cref{def:decl-conssub} is still equivalent to that in
\Cref{fig:decl:conssub} extended with rule \rul{CS-Top}. That is,
\Cref{lemma:properties-conssub} holds:
\begin{mprop}[Extension with $\top$]
  The following are equivalent:
  \begin{itemize}
  \item  $\tpreconssub A \tconssub B$.
  \item  $\tpresub A \tsub C$, $C \sim D$, $\tpresub D \tsub B$, for some $C, D$.
  \end{itemize}
\end{mprop}
% \begin{proof}\leavevmode
%   \begin{itemize}
%   \item From first to second: By induction on the derivation of consistent
%     subtyping. We have extra case \rul{CS-Top} now, where $B = \top$.
%     We can choose $C = A$, and
%     $D$ by replacing the unknown types in $C$ by $\nat$. Namely, $D$ is a static
%     type, so by \rul{S-Top} we are done.
%   \item From second to first: By induction on the derivation of second
%     subtyping. We have extra case \rul{S-Top} now, where
%     $B = \top$, so $A \tconssub B$ holds by \rul{CS-Top}.
%   \end{itemize}
% \end{proof}

\paragraph{Extending AGT}

We now extend the definition of concretization (\Cref{def:concret}) with $\top$
by adding another equation:
\[
  \gamma(\top) = \{\top\}
\]
It is easy to verify that \Cref{lemma:coincide-agt} still holds:
\begin{mprop}[Equivalent to AGT Extended with $\top$ on Simple Types]
  \label{prop:agt-top}
  $A \tconssub B$ if only if $A \agtconssub B$.
\end{mprop}

% \begin{proof}\leavevmode
%   \begin{itemize}
%   \item From left to right: By induction on the derivation of consistent
%     subtyping. We have case \rul{CS-Top} now.
%     It follows that for
%     every static type $A_1 \in \gamma(A)$, we can derive $A_1 \tsub \top$ by
%     \rul{S-Top}.
%     We have $B_1 = B = \top$ and we are done.
%   \item From right to left: By induction on the derivation of subtyping and
%     inversion on the concretization. We have extra case \rul{S-Top} now, where
%     $B$ is $\top$. So
%     consistent subtyping directly holds.
%   \end{itemize}
% \end{proof}

\paragraph{\citeauthor{siek2007gradual}'s definition of consistent subtyping does not work for $\top$}

Similarly to the analysis in \Cref{subsec:towards-conssub}, $\nat \to \unknown
\tconssub \top$ only holds when we first apply consistency, then subtyping, as
shown in the following diagram. However we cannot find a type $A$ such that
$\nat \to \unknown \tsub A$ and $A \sim \top$. Also we have a similar problem in
extending the restriction operator: \textit{non-structural} masking between
$\nat \to \unknown$ and $\top$ cannot be easily achieved.
\begin{center}
  \begin{tikzpicture}
    \matrix (m) [matrix of math nodes,row sep=3em,column sep=4em,minimum width=2em]
    {
      \bot & \top \\
      \nat \to \unknown &
      \nat \to \nat \\};

    \path[-stealth]
    (m-2-1) edge node [left] {$\tsub$} (m-1-1)
    (m-2-2) edge node [left] {$\tsub$} (m-1-2);

    \draw
    (m-1-1) edge node [above] {$\sim$} (m-1-2)
    (m-2-1) edge node [below] {$\sim$} (m-2-2);
  \end{tikzpicture}
\end{center}


\subsection{Better to be Unknown}
\label{subsec:algo:discuss}

In \Cref{sec:type:trans} we have seen an example where a source expression could
produce two different target expressions with different runtime behaviour. As we
explained, this is due to the guessing nature of the declarative system, and
from the typing point of view, no type is particularly better than others.
However in practice, this is not desirable. Let us revisit the same example, now
from the algorithmic point of view (we omit the translation for space reasons,
the interested reader can try to write down the full derivation):
\[
  f: \forall a. a \to a \byinf (\blam x \unknown {f ~ x}) \infto \unknown \to \genA \dashv f : \forall a. a \to a, \genA
\]
Compared with declarative typing, which can produce as many types as
possible ($\unknown \to \nat$, $\unknown \to \bool$, and so on), the algorithm
computes the type $\unknown \to \genA$ with $\genA$ unsolved in the output
context. This is due to rule \rul{ACS-UnknownL}. What can we know from the
output context? The only thing we know is that $\genA$ is not constrained at
all! As we discussed, any monotype for $\genA$ is inappropriate. Instead, we
replace $\genA$ with the unknown type $\unknown$, which helps to avoid unnecessary
down-casts at runtime (any cast to $\unknown$ is safe), resulting in the final
type $\unknown \to \unknown$.


\paragraph{Do soundness and completeness still hold?}

The reader may ask if the declarative system produces types such as $\unknown
\to \nat$ and $\unknown \to \bool$, but the algorithmic system computes the type
$\unknown \to \unknown$, does it imply that the algorithmic system is no longer
sound and complete with respect to the declarative system? The answer is no.
First of all, note that \Cref{thm:type_complete} reads ``$\dots$ there exist
$\Delta$, $\Omega'$ and $A'$ such that $\dots$ $A = \ctxsubst{\Omega'}{A'}$''.
Now if $A'$ (which is produced by the algorithmic system) contains some unsolved
existential variables, it is up to us to pick solutions in $\Omega'$ to match up
with $A$ (which is produced by the declarative system). More concretely, let us
assume that the declarative system produces type $\unknown \to \nat$, and let
$\Omega = f : \forall a. a \to a$ and $\Delta = f : \forall a. a \to a, \genA$,
the completeness theorem asks if we can find a complete context $\Omega'$ that
extends both $\Delta$ and $\Omega$ such that $\ctxsubst{\Omega'}{(\unknown \to
  \genA)} = \unknown \to \nat$. It is obvious that $\Omega' = f : \forall a. a
\to a, \genA = \nat$ is one such complete context, and we have
$\ctxsubst{\Omega'}{(\unknown \to \genA)} = \unknown \to \nat$. So the
algorithmic system is still complete. Similar arguments apply to the soundness
theorem. 
Secondly, an observation (which follows from the soundness and
completeness theorems) is: if an
expression is typeable with many types in the declarative system, the same expression must be
typeable with a single type that contains some unsolved existential variables in the algorithmic
algorithmic system.
In that case, replacing them with $\unknown$ is the best
strategy for the sake of execution.

\paragraph{Existential variables do not indicate parametricity}

Another reading of the above example may suggest that the result type $\unknown
\to \genA$ implies parametricity, with the implication that $\genA$ can be
changed arbitrarily without affecting the runtime behaviour of the
program. A similar phenomenon is discussed by \citet{siek2008gradual}, where they
argue that we cannot simply ``ignore dynamic types during unification''. In
their view, a type signature with a type variable indicates parametricity, but
this type does not. We agree that type variables do indicate parametricity, but
\textit{existential variables} do not! An \textit{unsolved} existential variable
indicates that a value's only constraint is that it may be cast to and from
$\unknown$, thus may introduce runtime casts. A similar observation is also
found in \citet{garcia2015principal}, where they have to distinguish between
\textit{static polymorphism} and \textit{gradual polymorphism}, and in addition
to gradual type parameters, they have to introduce the so-called \textit{static
  type parameters}. We argue that our language design is much simpler, and our
algorithmic system naturally embraces this distinction.

Now the question is can we really do that? The syntax in \Cref{fig:algo-syntax}
specifies that the solution of a existential variable can only be a monotype.
This is true if the existential variable has a solution. Reading of the output
context reveals that $\genA$ does not have any solution at all. What is more,
our target language (i.e., \pbc) has a nice property, the so-called
``Jack-of-All-Trade Principle''~\cite{ahmed2011blame} that says if instantiating
a type parameter to any given type yields an answer then instantiating that type
parameter to $\unknown$ yields the same answer. In light of these, no type is
more suitable than $\unknown$.

We need to note that this does not mean we should always instantiate a type
parameter to $\unknown$, as is the case in \pbc. One of our design principles is
that we should extract as much information as possible from the static aspects of
the type system, until there is nothing more we can know, then leave the job to
the runtime checks.



%%% Local Variables:
%%% mode: latex
%%% TeX-master: "../paper"
%%% org-ref-default-bibliography: "../paper.bib"
%%% End:

\section{Restoring the Dynamic Gradual Guarantee with Type Parameters}
\label{sec:advanced-extension}

In \cref{sec:type:trans} we have seen an example where a single source expression could
produce two different target expressions with different runtime behaviors. As we
explained, this is due to the guessing nature of the declarative system, and,
from the (source) typing point of view, no guessed type is particularly better than 
any other. As a consequence, this breaks the dynamic gradual guarantee as discussed in \cref{sec:criteria}.

To alleviate this situation, we introduce \textit{static type parameters}, which
are placeholders for monotypes, and \textit{gradual type parameters}, which are
placeholders for monotypes that are consistent with the unknown type. The
concept of static type parameters and gradual type parameters in the context of
gradual typing was first introduced by \citet{garcia2015principal}, and later
played a central role in the work of \citet{yuu2017poly}. In our type system,
type parameters mainly help capture the notion of \textit{representative
  translations}, and should not appear in a source program.
% As far as we know,
% we are the first to employ type parameters in the (implicit) polymorphic
% setting.
With them we are able to recast the dynamic gradual guarantee in terms
of representative translations, and to prove that every well-typed source expression
possesses at least one representative translation. With a
coherence conjecture regarding representative translations, the dynamic gradual
guarantee of our extended source language now can be reduced to that of \pbc,
which, at the time of writing, is still an open question.


% \jeremy{emphasize type parameters are just analysis tool for the purpose of
%   discussing dynamic gradual guarantee, they don't actually appear in program text. }


% The crucial difference
% between them is that the former correspond to existential variables without any
% constraints, while the latter correspond to existential variables with the only
% constraint that they are compared with an unknown type. With static and gradual
% type parameters in place, we are able to reason about dynamic semantics in more
% depth.


\subsection{Declarative Type System}
\label{sec:type-param}

The new syntax of types is given at the top of \cref{fig:exd:type}, with the differences
highlighted. In addition to the types of \cref{fig:decl:subtyping}, we add \emph{static type parameters} $[[static]]$,
and \emph{gradual type parameters} $[[gradual]]$. Both kinds of type parameters are monotypes. The addition of type parameters,
however, leads to two new syntactic categories of types. \emph{Castable types} $[[gc]]$
represent types that can be cast from or to $[[unknown]]$. It includes all
types, except those that contain static type parameters. \emph{Castable monotypes}
$[[tc]]$ are those castable types that are also monotypes.

\begin{figure}[t]
  \centering
  \begin{small}
    \begin{tabular}{lrcl} \toprule
      Types & $[[A]], [[B]]$ & \syndef & $[[int]] \mid [[a]] \mid [[A -> B]] \mid [[\/ a. A]] \mid [[unknown]] \mid \hlmath{[[static]] \mid [[gradual]]} $ \\
      Monotypes & $[[t]], [[s]]$ & \syndef & $ [[int]] \mid [[a]] \mid [[t -> s]] \mid \hlmath{[[static]] \mid [[gradual]]}$ \\
      \hl{Castable Types} & $[[gc]]$ & \syndef & $ [[int]] \mid [[a]] \mid [[gc1 -> gc2]] \mid [[\/ a. gc]] \mid [[unknown]] \mid [[gradual]] $ \\
      \hl{Castable Monotypes} & $[[tc]]$ & \syndef & $ [[int]] \mid [[a]] \mid [[tc1 -> tc2]] \mid [[gradual]]$ \\

      \bottomrule
    \end{tabular}

    \begin{drulepar}[cs]{$ [[dd |- A <~ B ]] $}{Consistent Subtyping}{}
      \drule{tvar}
      \drule{int}
      \drule{arrow}
      \drule{forallR}
      \drule{forallL}
      \hlmath{\drule{unknownLL}} \and
      \hlmath{\drule{unknownRR}} \and
      \hlmath{\drule{spar}} \and
      \hlmath{\drule{gpar}}
    \end{drulepar}
  \end{small}
  \caption{Syntax of types, and consistent subtyping in the extended declarative
  system.}
  \label{fig:exd:type}
\end{figure}


\paragraph{Consistent Subtyping.}
The new definition of consistent subtyping is given at the bottom of
\cref{fig:exd:type}, again with the differences highlighted. Now the unknown type is only a consistent subtype of all
castable types, rather than of all types (\rref{cs-unknownLL}), and vice versa
(\rref{cs-unknownRR}). Moreover, the static type parameter $[[static]]$ is a consistent
subtype of itself (\rref{cs-spar}), and similarly for the gradual type parameter
(\rref{cs-gpar}). From this definition it follows immediately that 
$[[unknown]]$ is incomparable with types that contain static type parameters $[[static]]$,
such as $[[static -> int]]$.

\paragraph{Typing and Translation.}

Given these extensions to types and consistent subtyping, the typing process
remains the same as in \cref{fig:decl-typing}. To account for the
changes in the translation, if we extend \pbc with type parameters as in
\citet{garcia2015principal}, then the translation remains the same as well.

\subsection{Substitutions and Representative Translations}

As we mentioned, type parameters serve as placeholders for monotypes. As a
consequence, wherever a type parameter is used, any \emph{suitable} monotype
could appear just as well. To formalize this observation, we define substitutions for type
parameters as follows:

\begin{definition}[Substitution] Substitutions for type parameters are defined as:
  \begin{enumerate}
    \item Let $\ssubst : [[static]] \to [[t]]$ be a total function mapping static type
      parameters to monotypes. 
    \item Let $\gsubst : [[gradual]] \to [[tc]]$ be a total function mapping gradual type
      parameters to castable monotypes.
    \item Let $\psubst = \gsubst \cup \ssubst$ be a union of $\ssubst$ and $\gsubst$ mapping static and gradual
      type parameters accordingly.
  \end{enumerate}
\end{definition}
\noindent Note that since $[[gradual]]$ might be compared with $[[unknown]]$, only
castable monotypes are suitable substitutes, whereas $[[static]]$
can be replaced by any monotypes. Therefore, we can substitute $[[gradual]]$ for $[[static]]$,
but not the other way around.


Let us go back to our example and its two translations in \cref{sec:type:trans}. The
problem with those translations is that neither $[[int -> int]]$ nor $[[bool ->
int]]$ is general enough. With type parameters, however, we can state a more
\textit{general} translation that covers both through substitution:
\begin{align*}
  f: \forall a. a \to \nat &\byinf (\blam x \unknown {f ~ x})
                          : \unknown \to \nat \\
                          &\trto (\blam x \unknown (\cast {\forall a. a \to \nat} {\gradual \to \nat} f) ~
                          (\hlmath{\cast \unknown \gradual} x))
\end{align*}
The advantage of type parameters is that they help reasoning
about the dynamic semantics. Now we are not limited to a particular choice, such
as $[[int -> int]]$ or $[[bool -> int]]$, which might or might not emit a cast
error at runtime. Instead we have a general choice $[[gradual ->
int]]$. 

What does the more general choice with type parameters tell us? First, we know that in this case, there is no concrete
constraint on $[[a]]$, so we can instantiate it with a type parameter. Second,
the fact that the general choice uses $[[gradual]]$ rather than
$[[static]]$ indicates that any chosen instantiation needs to be a castable type.
It follows that any concrete instantiation will have an impact on the
runtime behavior; therefore it is best to instantiate $[[a]]$ with
$[[unknown]]$. However, type inference cannot instantiate $[[a]]$ with
$[[unknown]]$, and substitution cannot replace $[[gradual]]$ with $[[unknown]]$ either.
This means that we need a syntactic refinement process of the translated programs in
order to replace type parameters with allowed gradual types.

\paragraph{Syntactic Refinement.}

We define syntactic refinement of the translated expressions as follows. As
$[[static]]$ denotes no constraints at all, substituting it with any monotype
would work. Here we arbitrarily use $[[int]]$. We interpret
$[[gradual]]$ as $[[unknown]]$ since any monotype could possibly lead to a cast
error.

\begin{definition}[Syntactic Refinement] The syntactic refinement of a
  translated expression $[[pe]]$ is denoted by $\erasetp s$, and defined as follows:
  \begin{center}
\begin{tabular}{lllllll} \toprule
  $\erasetp{\nat}$ &$=$ & $ \nat$ &  &   $\erasetp{a} $ & $ = $ & $a $ \\
  $\erasetp{A \to B}$ &$=$ & $ \erasetp{A} \to \erasetp{B}$ &  &   $\erasetp {\forall a. A} $ & $ = $ & $ \forall a . \erasetp{A} $ \\
  $\erasetp{\unknown}$ &$=$ & $\unknown$ &  &   $\erasetp {\static} $ & $ = $ & $\nat$ \\
  $\erasetp{\gradual}$ &$=$ & $\unknown$ &  \\ \bottomrule
\end{tabular}

  \end{center}
\end{definition}
% \bruno{Can we align the ``='' and the types?}
\noindent Applying the syntactic refinement to the translated
expression, we get
  \[
    (\blam x \unknown (\cast {\forall a. a \to \nat} { \hlmath[blue!40]{\unknown} \to \nat} f) ~
    (\cast \unknown {\hlmath[blue!40]{\unknown}} x))
  \]
where two $[[gradual]]$ are refined by $[[unknown]]$ as highlighted.
It is easy to verify that both applying this expression to $3$ and to
$\mathit{true}$ now results in a translation that evaluates to
a value.

\paragraph{Representative Translations.}
To decide whether one translation is more general than the other, we define a preorder
between translations.

\begin{definition}[Translation Pre-order]
  Suppose $[[dd |- e : A ~~> pe1]]$ and $[[dd |- e : A ~~> pe2]]$,
  we define $[[pe1]] \leq [[pe2]]$ to mean $[[pe2]] \aeq [[S(pe1)]]$ for
  some $[[S]]$.
\end{definition}

\begin{restatable}[]{proposition}{propparalpha}
  \label{prop:parameter:alpha}
  If $[[ pe1 ]] \leq [[pe2]]$ and $[[ pe2 ]] \leq [[pe1]]$, then $[[pe1]]$ and
  $[[pe2]]$ are $\alpha$-equivalent (i.e., equivalent up to renaming of type parameters).
\end{restatable}

The preorder between translations gives rise to a notion of
what we call \textit{representative translations}:

\begin{definition}[Representative Translation]
  A translation $[[pe]]$ is said to be a representative translation of a typing
  derivation $[[dd |- e : A ~~> pe]]$ if and only if for any other translation
  $[[dd |- e : A ~~> pe']]$ such that $[[pe']] \leq [[pe]]$, we have $[[pe]]
  \leq [[pe']]$. From now on we use $[[rpe]]$ to denote a representative
  translation.
\end{definition}

An important property of representative translations, which we conjecture for
the lack of rigorous proof, is that if there exists any translation of an
expression that (after syntactic refinement) can reduce to a value, so can a
representative translation of that expression. Conversely, if a
representative translation runs into a blame, then no translation of that
expression can reduce to a value.

\begin{conjecture}[Property of Representative Translations]\label{lemma:repr}
  For any expression $[[e]]$ such that $[[ dd |- e : A ~~> pe ]]$ and $[[ dd |- e : A ~~> rpe ]]$ and
  $\forall [[CC]].\, [[CC : (dd |- A) ~~> (empty |- int) ]]   $, we have
  \begin{itemize}
  \item If $  [[CC]] \{  \erasetp{[[pe]]} \}  [[==>]] [[n]]$, then $ [[CC]] \{   \erasetp{[[rpe]]}   \} [[==>]] [[n]]$.
  \item If $[[CC]] \{ \erasetp {[[rpe]]}   \} [[==>]] [[blame]]$, then $ [[CC]] \{ \erasetp {[[pe]]} \}  [[==>]] [[blame]]$.
  \end{itemize}
\end{conjecture}

Given this conjecture, we can state a stricter coherence property (without the
``up to casts'' part) between any two representative translations. We first
strengthen \cref{conj:coher} following \citet{amal2017blame}:

\begin{definition}[Contextual Approximation \`a la \citet{amal2017blame}] \leavevmode
  \begin{center}
  \begin{tabular}{lll}
$[[dd]] \vdash \ctxappro{[[pe1]]}{[[pe2]]}{[[A]]}$ & $\defeq$ & $[[ dd |- pe1 : A  ]] \land [[dd |- pe2 : A ]] \ \land $ \\
                                                   & & for all $\mathcal{C}.\, [[ CC : (dd |- A) ~~> (empty |- int) ]] \Longrightarrow$ \\
                                                   & &  $\quad (\mathcal{C}\{ \erasetp{[[pe1]]} \}   \Downarrow [[n]] \Longrightarrow  \mathcal{C} \{ \erasetp{[[ pe2 ]]}  \}  \reduce [[n]]) \ \land$ \\
                                                   & & $\quad (\mathcal{C} \{ \erasetp{[[ pe1 ]]} \} \reduce \blamev \Longrightarrow \mathcal{C} \{ \erasetp{[[ pe2 ]]}  \}  \reduce \blamev)$

  \end{tabular}
  \end{center}
\end{definition}
The only difference is
that now when a program containing $[[pe1]]$ reduces to a value, so does one
containing $[[pe2]]$.


From \cref{lemma:repr}, it follows that coherence holds between
two representative translations of the same expression.

\begin{corollary}[Coherence for Representative Translations]
  For any expression $[[e]]$
  such that $[[ dd |- e : A ~~> rpe1    ]]$ and $[[ dd |- e : A ~~> rpe2    ]]$, we have
  $[[ dd ]] \vdash \ctxeq{[[rpe1]]}{[[rpe2]]}{[[A]]} $.
\end{corollary}

We have proved that for every typing derivation, at least one representative translation exists.

\begin{restatable}[Representative Translation for Typing]{lemma}{lemmareptyping}
  \label{lemma:rep:typing}
  For any typing derivation $[[dd |- e : A]]$ there exists at least one
  representative translation $r$ such that $[[dd |- e : A ~~> rpe]]$.
\end{restatable}

For our example, $(\blam x \unknown (\cast {\forall a. a \to \nat} {\gradual \to
  \nat} f) ~ (\cast \unknown \gradual x))$ is a representative translation,
while the other two are not.


\subsection{Dynamic Gradual Guarantee, Reloaded}

Given the above propositions, we are ready to revisit the dynamic gradual
guarantee. The nice thing about representative translations is that the
dynamic gradual guarantee of our source language is essentially that of \pbc,
our target language. However, the dynamic gradual guarantee for \pbc is still an
open question. According to \citet{yuu2017poly}, the difficulty lies in the
definition of term precision that preserves the semantics. We leave it here as a
conjecture as well. From a declarative point of view, we cannot prevent the
system from picking undesirable instantiations, but we know that some choices
are better than the others, so we can restrict the discussion of dynamic gradual
guarantee to representative translations.

\begin{conjecture}[Dynamic Gradual Guarantee in terms of Representative Translations]
  Suppose $e' \lessp e$,
  \begin{enumerate}
  \item If $[[empty |- e : A ~~> rpe]]$, $\erasetp {r} \Downarrow v$,
    then for some $B$ and $r'$, we have $[[ empty |- e' : B ~~> rpe']]$,
    and $B \lessp A$,
    and $\erasetp {r'} \Downarrow v'$,
    and $v' \lessp v$.
  \item If $[[empty |- e' : B ~~> rpe']]$, $\erasetp {r'} \Downarrow v'$,
    then for some $A$ and $[[rpe]]$, we have $ [[empty |- e : A ~~> rpe]]$,
    and $B \lessp A$. Moreover,
    $\erasetp r \Downarrow v$ and $v' \lessp v$,
    or $\erasetp r \Downarrow \blamev$.
  \end{enumerate}
\end{conjecture}

For the example in \cref{sec:criteria}, now we know that the representative
translation of the right one will evaluate to $1$ as well.
\begin{mathpar}
  (\blam{f}{\forall a. a \to \nat}{\blam{x}{\nat}{f~x}})~(\lambda x .\, 1)~3 \and
  (\blam{f}{\forall a. a \to \nat}{\blam{x}{\unknown}{f~x}})~(\lambda x .\, 1)~3
\end{mathpar}

More importantly, in what follows, we show that our extended algorithm is able to find those representative translations.


\subsection{Extended Algorithmic Type System}
\label{subsec:exd-algo}

\begin{figure}[t]
  \centering
  \begin{small}
    \begin{tabular}{lrcl} \toprule
      Types & $[[aA]], [[aB]]$ & \syndef & $ [[int]] \mid [[a]] \mid [[evar]] \mid [[aA -> aB]] \mid [[\/ a. aA]] \mid [[unknown]] \mid \hlmath{[[static]] \mid [[gradual]]} $ \\
      Monotypes & $[[at]], [[as]]$ & \syndef & $ [[int]] \mid [[a]] \mid [[evar]] \mid [[at -> as]] \mid \hlmath{[[static]] \mid [[gradual]]}$ \\
      \hl{Existential variables} & $[[evar]]$ & \syndef & $[[sa]]  \mid [[ga]]  $   \\
      \hl{Castable Types} & $[[agc]]$ & \syndef & $ [[int]] \mid [[a]] \mid [[evar]] \mid [[agc1 -> agc2]] \mid [[\/ a. agc]] \mid [[unknown]] \mid [[gradual]] $ \\
      \hl{Castable Monotypes} & $[[atc]]$ & \syndef & $ [[int]] \mid [[a]] \mid [[evar]] \mid [[atc1 -> atc2]] \mid [[gradual]]$ \\
      Algorithmic Contexts & $[[GG]], [[DD]], [[TT]]$ & \syndef & $[[empty]] \mid [[GG , x : aA]] \mid [[GG , a]] \mid [[GG , evar]]  \mid \hlmath{[[GG, sa = at]] \mid [[GG, ga = atc]]} \mid [[ GG, mevar ]] $ \\
      Complete Contexts & $[[OO]]$ & \syndef & $[[empty]] \mid [[OO , x : aA]] \mid [[OO , a]] \mid \hlmath{[[OO, sa = at]] \mid [[OO, ga = atc]]} \mid [[OO, mevar]] $ \\
      \bottomrule
    \end{tabular}
  \end{small}
  \caption{Syntax of types, contexts and consistent subtyping in the extended algorithmic system.}
  \label{fig:exd:algo:type}
\end{figure}


% \jeremy{the example is wrong, we need a new example to motivate}
To understand the design choices involved in the new algorithmic system, we
consider the following algorithmic typing example:
\[
  f: [[ \/a .  a -> int  ]], x : [[unknown]] \byinf [[ f x ]] \infto [[int]]  \dashv f : [[\/a . a -> int]], x : [[unknown]], \genA
\]
Compared with the declarative typing, where we have many choices (e.g., $[[int -> int]]$, $[[bool -> int]]$, and so on)
to instantiate $[[\/ a. a -> int]]$, the algorithm computes
the instantiation $[[ evar -> int ]]$ with $[[evar]]$ unsolved in the output context.
What can we know from the algorithmic typing? First we know that, here $[[evar]]$
is \textit{not constrained} by the typing problem. Second, and more importantly,
$[[evar]]$ has been compared with an unknown type (when typing $([[ f x ]])$).
Therefore, it is possible to make a more refined distinction
between different kinds of existential variables. The first
kind of existential variables are those that indeed have no constraints at all,
as they do not affect the dynamic semantics; while the second kind (as in this example) are
those where the only constraint is that
\textit{the variable was once compared with an unknown type}~\citep{garcia2015principal}.

The syntax of types is shown in \cref{fig:exd:algo:type}. A notable
difference, apart from the addition of static and gradual parameters, is that we
further split existential variables $[[evar]]$ into static existential variables
$[[ sa ]]$ and gradual existential variables $[[ga]]$.
Depending on whether an existential variable has been
compared with $[[unknown]]$ or not, its solution space changes. More
specifically, static existential variables can be solved to a monotype
$[[at]]$, whereas gradual existential variables can only be solved to a
castable monotype $[[atc]]$, as can be seen in the changes of algorithmic
contexts and complete contexts. As a result, the typing result for the above example
now becomes
\[
  f: [[ \/a .  a -> int  ]], x : [[unknown]] \byinf [[ f x ]] \infto [[int]]  \dashv f : [[\/a . a -> int]], x : [[unknown]], \hlmath{[[ga]]}
\]
since we can solve any unconstrained $[[ga]]$ to $[[gradual]]$, it is easy to
verify that the resulting translation is indeed a representative translation.

Our extended algorithm is novel in the following aspects. We naturally extend
the concept of existential variables~\citep{dunfield2013complete} to deal with
comparisons between existential variables and unknown types. Unlike
\citet{garcia2015principal}, where they use an extra set to store types that
have been compared with unknown types, our two kinds of existential variables emphasize
the type distinction better, and correspond more closely to the two kinds of type parameters,
as we can solve $[[sa]]$ to $ [[static]]$ and $[[ga]] $ to $ [[gradual]]$.

The implementation of the algorithm can be found in the supplementary materials.


\paragraph{Extended Algorithmic Consistent Subtyping}


\begin{figure}[t]
  \centering
  \begin{small}
   \begin{drulepar}[as]{$ [[GG |- aA <~ aB -| DD ]] $}{Algorithmic Consistent Subtyping}
     \drule{tvar}
     \drule{int}
     \drule{evar} \and
     \hlmath{\drule{spar}} \and
     \hlmath{\drule{gpar}} \and
     \hlmath{\drule{unknownLL}}
     \hlmath{\drule{unknownRR}} \and
     \drule{arrow}
     \drule{forallR} \and
     \hlmath{\drule{forallLL}} \and
     \drule{instL}
     \drule{instR}
   \end{drulepar}
  \end{small}
  \caption{Extended algorithmic consistent subtyping}
  \label{fig:exd:algo:sub}
\end{figure}

While the changes in the syntax seem negligible, the addition of static and
gradual type parameters changes the algorithmic judgments in a significant way.
We first discuss the algorithmic consistent subtyping, which is shown in \cref{fig:exd:algo:sub}.
For notational convenience, when static and
gradual existential variables have the same rule form, we compress them into one rule. For
example, \rref{as-evar} is really two rules $[[ GG[sa] |- sa <~ sa -| GG[sa] ]]$
and $[[ GG[ga] |- ga <~ ga -| GG[ga] ]]$; same for \rref{as-instL,as-instR}.

\Rref{as-spar,as-gpar} are direct analogies of \rref{cs-spar,cs-gpar}. Though
looking simple, \rref{as-unknownLL,as-unknownRR} deserve much explanation. To
understand what the output context $[[ [agc]GG ]]$ is for, let us first see why
this seemingly intuitive rule $[[ GG |- unknown <~ agc -| GG ]]$ (like
\rref{as-unknownL} in the original algorithmic system) is wrong. Consider the
judgment $[[ sa |- unknown <~ sa -> sa -| sa ]]$, which seems fine. If this
holds, then -- since $[[sa]]$ is unsolved in the output context -- we can solve
it to $[[ static ]]$ for example (recall that $[[sa]]$ can be solved to some
monotype), resulting in $[[ unknown <~ static -> static ]]$. However, this is in
direct conflict with \rref{cs-unknownLL} in the declarative system precisely
because $[[ static -> static ]]$ is not a castable type! A possible solution
would be to transform all static existential variables to gradual existential
variables within $[[agc]]$ whenever it is being compared to $[[ unknown ]]$:
while $[[ sa |- unknown <~ sa -> sa -| sa ]]$ does not hold, $[[ ga |- unknown
<~ ga -> ga -| ga ]]$ does. While substituting static existential variables with
gradual existential variables seems to be intuitively correct, it is rather hard
to formulate---not only do we need to perform substitution in $[[agc]]$, we also
need to substitute accordingly in both the input and output contexts in order to
ensure that no existential variables become unbound. However, making such changes is
at odds with the interpretation of input contexts: they are ``input'', which
evolve into output contexts with more variables solved. Therefore, in line with
the use of input contexts, a simple solution is to generate a
new gradual existential variable and solve the static existential variable to it
in the output context, without touching $[[agc]]$ at all. So we have $[[ sa |- unknown <~ sa -> sa -| ga, sa = ga ]]$.

Based on the above discussion, the following defines $[[ [aA]GG ]]$:
\begin{definition}$[[ [aA]GG ]]$ is defined inductively as follows  \label{def:contamination} %
  \begin{center}
    \begin{tabular}{llll} \toprule
     $[[ [aA] empty    ]]$ & = &  $[[empty]]$  & \\
    $[[ [aA] (GG, x : aA)  ]]$ &=& $[[ [aA] GG , x : aA     ]]$ & \\
    $[[ [aA] (GG, a)  ]]$ &=& $[[ [aA] GG , a     ]]$ & \\
    $[[ [aA] (GG, sa)  ]]$ &=& $[[ [aA] GG , ga , sa = ga  ]]$  & if $[[sa]]$ occurs in $[[aA]]$     \\
    $[[ [aA] (GG, sa)  ]]$ &=& $[[ [aA] GG , sa     ]]$     & if $[[sa]]$ does not occur in $[[aA]]$  \\
    $[[ [aA] (GG, ga)  ]]$ &=& $[[ [aA] GG , ga     ]]$ & \\
    $[[ [aA] (GG, evar = at)  ]]$ &=& $[[ [aA] GG , evar = at     ]]$ & \\
    $[[ [aA] (GG, mevar)  ]]$ &=& $[[ [aA] GG , mevar     ]]$ & \\ \bottomrule
    \end{tabular}
  \end{center}
\end{definition}
\noindent $[[ [aA]GG ]]$ solves all static existential variables found within $[[aA]]$ to fresh
gradual existential variables in $[[GG]]$. Notice the case for $[[ [aA] (GG, sa)]]$
is exactly what we have just described.

\Rref{as-forallLL} is slightly different from \rref{as-forallL} in the original
algorithmic system in that we replace $[[a]]$ with a new static existential
variable $[[sa]]$. Note that $[[sa]]$ might be solved to a gradual existential
variable later. The rest of the rules are the same as those in the original system.


\begin{figure}[t]
  \centering
  \begin{small}

   \begin{drulepar}[instl]{$ [[ GG |- evar <~~ aA -| DD   ]] $}{Instantiation I}
     \hlmath{\drule{solveS}} \and
     \hlmath{\drule{solveG}} \and
     \hlmath{\drule{solveUS}} \and
     \hlmath{\drule{solveUG}} \and
     \hlmath{\drule{reachSGOne}} \and
     \hlmath{\drule{reachSGTwo}} \and
     \hlmath{\drule{reachOther}} \and
     \drule{forallR}
     \drule{arr}
   \end{drulepar}

   \begin{drulepar}[instr]{$ [[ GG |- aA <~~ evar -| DD   ]] $}{Instantiation II, excerpt}
     \hlmath{\drule{forallLL}} \and
   \end{drulepar}
  \end{small}

  \caption{Instantiation in the extended algorithmic system}
  \label{fig:exd:inst}

\end{figure}

\paragraph{Extended Instantiation}

The instantiation judgments shown in \cref{fig:exd:inst} also change
significantly. The complication comes from the fact that now we have two different
kinds of existential variables, and the relative order they appear in the
context affects their solutions.


\Rref{instl-solveS, instl-solveG} are the refinement to \rref{instl-solve} in
the original system. The next two rules deal with situations where one side is
an existential variable and the other side is an unknown type.
\Rref{instl-solveUS} is a special case of \rref{as-unknownRR} where we create a
new gradual existential variable $[[ga]]$ and set the solution of $[[sa]]$ to be
$[[ga]]$ in the output context. \Rref{instl-solveUG} is the same as
\rref{instl-solveU} in the original system and simply propagates the input
context. The next two rules \rref*{instl-reachSG1,instl-reachSG2} are a bit
involved, but they both answer to the same question: how to solve a gradual
existential variable when it is declared after some static existential variable.
More concretely, in \rref{instl-reachSG1}, we feel that we need to solve
$[[gb]]$ to another existential variable. However, simply setting $[[ gb = sa]]$ and leaving $[[sa]]$ untouched
in the output context is wrong. The reason is that $[[gb]]$ could be a gradual existential
variable created by \rref{as-unknownLL}/\rref*{as-unknownRR} and solving $[[gb]]$ to a static existential
variable would result in the same problem as we have discussed. Instead, we create another new gradual
existential variable $[[ga]]$ and set the solutions of both $[[sa]]$ and $[[gb]]$ to it; similarly in \rref{instl-reachSG2}.
\Rref{instl-reachOther} deals with the other cases (e.g., $[[ sa <~~ sb  ]]$, $[[ ga <~~ gb  ]]$ and so on).
In those cases, we employ the same strategy as in the original system.

As for the other instantiation judgment, most of the rules are symmetric and thus omitted.
The only interesting rule is \rref*{instr-forallLL}, which is similar to what we did for \rref{as-forallLL}.



\paragraph{Algorithmic Typing and Metatheory}

Fortunately, the changes in the algorithmic bidirectional system are minimal: we replace
every existential variable with a static existential variable.
Furthermore, we proved that the extended
algorithmic system is sound and complete with respect to the extended
declarative system. The proofs can be found in the appendix.



\paragraph{Do We Really Need Type Parameters in the Algorithmic System?}

As we mentioned earlier, type parameters in the declarative system are merely an
analysis tool, and in practice, type parameters are inaccessible to
programmers. For the sake of proving soundness and completeness, we have to
endow the algorithmic system with type parameters. However, the algorithmic
system already has static and gradual existential variables, which can serve the same
purpose. In that regard, we could directly solve every \textit{unsolved} static and
gradual existential variable in the output context to $[[int]]$ and
$[[unknown]]$, respectively.


% \jeremy{example of showing finding the representative translation?}
% \ningning{Include a simple discussion?: since type parameters are used to help
%   with reasoning, in practice, programmers are actually not allowed to write
%   them. Therefore, the algorithm could directly set unsolved static existential
%   to integers and gradual existential to unknowns after type checking as
%   algorithmic refinement process, without even knowing type parameters. }

% \subsection{Discussion}

\subsection{Restricted Generalization}

In \cref{sec:type:trans}, we discussed the issue that the translation produces
multiple target expressions due to the different choices for instantiations, and
those translations have different dynamic semantics. Besides that, there is
another cause for multiple translations: redundant generalization during
translation by \rref{gen}. Consider the simple expression $[[(\x:int. x) 1]]$,
the following shows two possible translations:
\begin{align*}
  [[empty |- (\x : int . x) 1 : int ]] &[[~~>]] [[ (\x : int . x) (<int `-> int> 1)]]
  \\
  [[empty |- (\x : int . x) 1 : int ]] &[[~~>]]  [[ (\x : int . x) (<\/ a. int `-> int> (/\ a. 1))]]
\end{align*}
The difference comes from the fact that in the second translation, we apply
\rref{gen} while typing $1$ to get $[[empty |- 1 : \/ a. int]]$. As a consequence, the translation of $1$
is accompanied by a cast from $[[\/ a. int]]$ to $[[int]]$ since the former is a
consistent subtype of the latter. This difference is harmless, because obviously
these two expressions will reduce to the same value in \pbc, thus preserving
coherence (up to cast error). While it is not going to break coherence,
it does result in multiple representative translations for one
expression (e.g., the above two translations are both the representative translations).

There are several ways to make the translation process more deterministic. For
example, we can restrict generalization to happen only in let expressions and
require let expressions to include annotations, as $[[ let x : A = e1 in e2 ]]$.
Another feasible option would be to give a declarative, bidirectional system as
the specification (instead of the type assignment one), in the same spirit of
\citet{dunfield2013complete}. Then we can restrict generalization to be
performed through annotations in checking mode.

With restricted generalization, we hypothesize that now each expression has exactly
one representative translation (up to renaming of fresh type parameters).
Instead of calling it a \textit{representative} translation, we can say it is a
\textit{principal} translation. Of course the above is only a sketch; we have
not defined the corresponding rules, nor studied metatheory.


\begin{comment}
\subsubsection{Interpretation of Type Parameters}
\label{subsec:type-par}

% \jeremy{If I understand it correctly, we actually used these two interpretations
%   in the extended declarative system. Def 8.1 (substitutions) is the first
%   interpretation; and Def 8.2 (syntactic refinement) is the second
%   interpretation in that $[[static]]$ is irrelevant to program execution so we
%   can replace it with any type, whereas $[[gradual]]$ is relevant so we replace
%   it with unknown }

In \cref{sec:type-param}, we introduced type parameters into our type system. It turns
out that type parameters are a useful tool to help us identify
representative translations and reason about the dynamic semantics of the
type system. But what are type parameters, exactly? Below we provide two plausible
interpretations.

The first interpretation of type parameters (the one we adopted) is that they are placeholders for
monotypes. This is to say, their meaning is given by substitution, and replacing
type parameters with other monotypes should not break typing:

\begin{proposition}
  If $[[dd |- e : A]]$, then $\psubst ([[dd]]) \vdash \psubst ([[e]]) : \psubst ([[A]])$.
\end{proposition}

\jeremy{See Proposition 1 of Principle scheme for gradual programs, where they also have exactly the same proposition, but they call it
type polymorphism! how this compare to the second interpretation?}

In practice, we should not allow programmers to write type parameters explicitly
in a program, as type parameters are only generated during typing process, and
get refined before evaluation. As a result, we can hide all the details of type
parameters from programmers and internally replace them with suitable concrete
types when necessary. This also reflects the point we discussed in the end of
\cref{subsec:exd-algo}.

On the other hand, we can interpret type parameters using \textit{polymorphism}.
In this sense, both of them can be extracted to generate type abstractions.
However, there is one subtle difference. That is, static type parameter
indicates \textit{parametric polymorphism} in the traditional sense, which is
irrelevant to program execution; while gradual type parameter indicates
\textit{gradual polymorphism}, which means it has no typing constraints but is
relevant to program execution \citep{garcia2015principal}. This interpretation
suggests that we might need a more refined distinction between type
abstractions, such as \citet{yuu2017poly}.

We argue that the extension of type parameters is \textit{a} feasible way to
reason about the dynamic semantics in a implicit polymorphic language, but it is
not necessarily \textit{the} only way. Still, it remains to see if
our discussion sheds lights on the study of dynamic semantics for
gradual languages with implicit polymorphism.

\end{comment}



%%% Local Variables:
%%% mode: latex
%%% TeX-master: "../paper"
%%% org-ref-default-bibliography: ../paper.bib
%%% End:


\section{Related Work}
\label{sec:related}

Along the way we discussed some of the most relevant work to motivate,
compare and
promote our gradual typing design. In what follows, we briefly discuss related
work on gradual typing and polymorphism.


\paragraph{Gradual Typing}

The seminal paper by \citet{siek2006gradual} is the first to propose gradual
typing, which enables programmers to mix static and dynamic typing in a program
by providing a mechanism to control which parts of a program are statically
checked. The original proposal extends the simply typed lambda calculus by
introducing the unknown type $\unknown$ and replacing type equality with type
consistency. Casts are introduced to mediate between statically and dynamically
typed code. Later \citet{siek2007gradual} incorporated gradual typing into a
simple object oriented language, and showed that subtyping and consistency are
orthogonal -- an insight that partly inspired our work. We show that subtyping
and consistency are orthogonal in a much richer type system with higher-rank
polymorphism. \citet{siek2009exploring} explores the design space of different
dynamic semantics for simply typed lambda calculus with casts and unknown types.
In the light of the ever-growing popularity of gradual typing, and its somewhat
murky theoretical foundations, \citet{siek2015refined} felt the urge to have a
complete formal characterization of what it means to be gradually typed. They
proposed a set of criteria that provides important guidelines for designers of
gradually typed languages. \citet{cimini2016gradualizer} introduced the
\emph{Gradualizer}, a general methodology for generating gradual type systems
from static type systems. Later they also develop an algorithm to generate
dynamic semantics~\cite{CiminiPOPL}. \citet{garcia2016abstracting} introduced
the AGT approach based on abstract interpretation. As we discussed, none of
these approaches instructed us how to define consistent subtyping for
polymorphic types.

There is some work on integrating gradual typing with rich type disciplines.
\citet{Ba_ados_Schwerter_2014} establish a framework to combine gradual typing and
effects, with which a static effect system can be transformed to a dynamic
effect system or any intermediate blend. \citet{Jafery:2017:SUR:3093333.3009865}
present a type system with \emph{gradual sums}, which combines refinement and
imprecision. We have discussed the interesting definition of \emph{directed
  consistency} in Section~\ref{sec:exploration}. \citet{castagna2017gradual} develop a gradual type system with
intersection and union types, with consistent subtyping defined by following
the idea of \citet{garcia2016abstracting}.
TypeScript~\citep{typescript} has a distinguished dynamic type, written {\color{blue} any}, whose fundamental feature is that any type can be
implicitly converted to and from {\color{blue} any}.
% They prove that the conversion
% definition (called \emph{assignment compatibility}) coincides with the
% restriction operator from \citet{siek2007gradual}.
Our treatment of the unknown type in \cref{fig:decl:conssub} is similar to their
treatment of {\color{blue} any}. However, their type system does not have
polymorphic types. Also, Unlike our consistent subtyping which inserts runtime
casts, in TypeScript, type information is erased after compilation so there are
no runtime casts, which makes runtime type errors possible.
% dynamic checks does not contribute to type safety.


\paragraph{Gradual Type Systems with Explicit Polymorphism}

\citet{Morris:1973:TS:512927.512938} dynamically enforces
parametric polymorphism and uses \emph{sealing} functions as the
dynamic type mechanism. More recent works on integrating gradual typing with
parametric polymorphism include the dynamic type of \citet{abadi1995dynamic} and
the \emph{Sage} language of \citet{gronski2006sage}. None of these has carefully
studied the interaction between statically and dynamically typed code.
\citet{ahmed2011blame} proposed \pbc that extends the blame
calculus~\cite{Wadler_2009} to incorporate polymorphism. The key novelty of
their work is to use dynamic sealing to enforce parametricity. As such, they end
up with a sophisticated dynamic semantics. Later, \citet{amal2017blame} prove
that with more restrictions, \pbc satisfies parametricity. Compared to their
work, our type system can catch more errors earlier since, as we argued, 
their notion of \emph{compatibility} is too permissive. For example, the
following is rejected (more precisely, the corresponding source program never
gets elaborated) by our type system:
\[
  (\blam x \unknown x + 1) : \forall a. a \to a \rightsquigarrow \cast {\unknown \to \nat}
  {\forall a. a \to a} (\blam x \unknown x + 1)
\]
while the type system of \pbc would accept the translation, though at runtime,
the program would result in a cast error as it violates parametricity.
% This does not imply, in any regard that \pbc is not well-designed; there are
% circumstances where runtime checks are \emph{needed} to ensure
% parametricity.
We emphasize that it is the combination of our powerful type system together
with the powerful dynamic semantics of \pbc that makes it possible to have
implicit higher-rank polymorphism in a gradually typed setting.
% without compromising parametricity.
\citet{devriese2017parametricity} proved that
embedding of System F terms into \pbc is not fully abstract. \citet{yuu2017poly}
also studied integrating gradual typing with parametric polymorphism. They
proposed System F$_G$, a gradually typed extension of System F, and System
F$_C$, a new polymorphic blame calculus. As has been discussed extensively,
their definition of type consistency does not apply to our setting (implicit
polymorphism). All of these approaches mix consistency with subtyping to some
extent, which we argue should be orthogonal. On a side note, it seems that our
calculus can also be safely translated to System F$_C$. However we do not
understand all the tradeoffs involved in the choice between \pbc and System
F$_C$ as a target.



\paragraph{Gradual Type Inference}
\citet{siek2008gradual} studied unification-based type inference for gradual
typing, where they show why three straightforward approaches fail to meet their
design goals. One of their main observations is
that simply ignoring dynamic types during unification does not work. Therefore,
their type system assigns unknown types to type variables and infers gradual
types, which results in a complicated type system and inference algorithm. In
our algorithm presented in \cref{sec:advanced-extension}, comparisons between
existential variables and unknown types are emphasized by the distinction
between static existential variables and gradual existential variables. By
syntactically refining unsolved gradual existential variables with unknown types, we gain a
similar effect as assigning unknown types, while keeping the algorithm relatively
simple.
\citet{garcia2015principal} presented a new approach where gradual type
inference only produces static types, which is adopted in our type system. They
also deal with let-polymorphism (rank 1 types). They proposed the distinction
between static and gradual type parameters, which inspired our extension to
restore the dynamic gradual guarantee. Although those existing works all involve
gradual types and inference, none of these works deal with higher-rank
implicit polymorphism.


\paragraph{Higher-rank Implicit Polymorphism}

\citet{odersky1996putting} introduced a type system for higher-rank implicit
polymorphic types. Based on that, \citet{jones2007practical} developed an
approach for type checking higher-rank predicative polymorphism.
\citet{dunfield2013complete} proposed a bidirectional account of higher-rank
polymorphism, and an algorithm for implementing the declarative system, which
serves as the main inspiration for our algorithmic system. The key difference,
however, is the integration of gradual typing.
% \citet{vytiniotis2012defer}
% defers static type errors to runtime, which is fundamentally different from
% gradual typing, where programmers can control over static or runtime checks by
% precision of the annotations.
As our work, those works are in a
\emph{predicative} setting, since complete type inference for higher-rank
types in an impredicative setting is undecidable. Still, there are many type
systems trying to infer some impredicative types, such as
\texttt{$ML^F$}~\citep{le2014mlf,remy2008graphic,le2009recasting}, the HML
system~\citep{leijen2009flexible}, the FPH system~\citep{vytiniotis2008fph} and
so on. Those type systems usually end up with non-standard System F types, and
sophisticated forms of type inference.

%%% Local Variables:
%%% mode: latex
%%% TeX-master: "../paper"
%%% org-ref-default-bibliography: "../paper.bib"
%%% End:


\section{Conclusion and Future Work}
\label{sec:conclusion}

We have proposed \fnamee, a type-safe and coherent calculus with disjoint
intersection types, BCD subtyping and parametric polymorphism. \fnamee improves
the state-of-art of compositional designs, and enables the development of highly
modular and reusable programs. One interesting and useful further extension
would be implicit polymorphism. For that we want to combine
Dunfield and Krishnaswami's approach~\cite{dunfield2013complete} with our bidirectional type system.
We would also like to study the parametricity of \fnamee. As we have seen in
\cref{sec:failed:lr}, it is not at all obvious how to extend the standard
logical relation of System F to account for disjointness, and avoid potential
circularity due to impredicativity. A promising solution is to use step-indexed
logical relations~\cite{ahmed2006step}. 
% TOM: This sentence is broken. Do we even need it?
% We have yet investigated further on that direction.


\section*{Acknowledgments}

We thank the anonymous reviewers and Yaoda Zhou for their helpful comments.
This work has been sponsored by the Hong Kong Research Grant
Council projects number 17210617 and 17258816, and by the Research Foundation -
Flanders.



%%% Local Variables:
%%% mode: latex
%%% TeX-master: "../paper"
%%% org-ref-default-bibliography: "../paper.bib"
%%% End:


%% Acknowledgments

\section*{Acknowledgements}

We thank Ronald Garcia, Dustin Jamner, and the anonymous reviewers for their
helpful comments. This work has been sponsored by the Hong Kong Research Grant
Council projects number 17210617 and 17258816, and by the Research Foundation -
Flanders.

%%% Local Variables:
%%% mode: latex
%%% TeX-master: "../paper"
%%% org-ref-default-bibliography: "../paper.bib"
%%% End:


%% Bibliography

\bibliography{paper}

%% Appendix

\newpage
\appendix
\section{Specification and Metatheory of \ecore}
We give the specification and metatheory of \ecore in this section. We
have completely formalized proofs of metatheory in Coq based on
Chargu{\'e}raud's work~\citeapp{charcoq}. We also provide brief paper
proofs for \ecore for reference. Full proofs can be found in the Coq
scripts\footnote{\fullurl} \verb|WeakCast_*.v|. The corresponding name of
each lemma in Coq is marked at the beginning in brackets.

\subsection{Syntax}
\begin{center}
\begin{minipage}{0.55\textwidth}
\gram{\ottec}
\end{minipage}
\begin{minipage}{0.4\textwidth}
\gram{
  \ottGg\ottinterrule
  \ottv}
\end{minipage}
\\
\end{center}
\begin{tabular}{ll}
Syntactic Sugar \\
& $\ottcoresugar$ % defined in otthelper.mng.tex
\end{tabular}

\subsection{Operational Semantics}
\ottdefnstep{}
% \ottusedrule{\ottdruleSXXMu{}}

\subsection{Typing}
\ottdefnctx{}\ottinterrule
\ottdefnexpr{}
% \ottusedrule{\ottdruleTXXMu{}}

\subsection{Properties}
\begin{comment}
We follow the naming of lemmas and proofs of properties 
for Pure Type System from \citeapp{handbook}. Some lemmas have other well-known names, like
Lemma \ref{lem:appendix:thin} is often called \emph{Weakening} and 
Lemma \ref{lem:appendix:gen} is often called \emph{Inversion}.

\begin{lemma}[Free Variable]\label{lem:appendix:free}
    If $[[G |- e:t]]$, then $\FV(e) \subseteq \dom([[G]])$ and $\FV([[t]])
\subseteq \dom([[G]])$.
\end{lemma}

\begin{proof}
    By induction on the derivation of $[[G |- e:t]]$. We only treat cases
\ruleref{T\_Mu}, \ruleref{T\_CastUp} and \ruleref{T\_CastDown} (since proofs of
other cases are the same as \cc \citeapp{handbook}):
    \begin{description}
        \item[Case \ruleref{T\_Mu}:] From premises of $[[G |- (mu x:t.e1) :
t]]$, by the induction hypothesis, we have $\FV(e_1) \subseteq \dom([[G]]) \cup
\{[[x]]\}$ and $\FV(\tau) \subseteq \dom([[G]])$. Thus the result follows by
$\FV([[mu x:t.e1]])=\FV(e_1) \setminus \{[[x]]\} \subseteq \dom([[G]])$ and
$\FV(\tau) \subseteq \dom([[G]])$.
        \item[Case \ruleref{T\_CastUp}:] Since $\FV([[castup [t]
e1]])=\FV([[e1]])$, the result follows directly by the induction hypothesis.
        \item[Case \ruleref{T\_CastDown}:] Since $\FV([[castdown
e1]])=\FV([[e1]])$, the result follows directly by the induction hypothesis.
    \end{description}
\end{proof}

\begin{definition}[Multi-step reduction]
    The relation $[[->>]]$ is the transitive and reflexive closure of
$[[-->]]$.
\end{definition}

\begin{definition}[$n$-step reduction]
    The $n$-step reduction is denoted by $[[e0]] [[-->>]] [[en]]$, if
    there exists a sequence of one-step reductions $[[e0]] [[-->]]
    [[e1]] [[-->]] [[e2]] [[-->]] \dots [[-->]] [[en]]$, where $n$ is
    a positive integer and $[[ei]]\,(i=0,1,\dots,n)$ are valid
    expressions.
\end{definition}
\end{comment}

\begin{definition}[Notation of Alpha Equality \footnote{This notation
    is also applied to sections afterwards.}]
    The alpha equivalence between terms is denoted by notation $[[=a]]$.
\end{definition}

\begin{lemma}[Weakening] \label{lem:appendix:thin}
    \verb|[typing_weaken]|
    Let $[[G]]$ and $[[G']]$ be well-formed contexts such that $[[G]] \subseteq
[[G']]$. If $[[G |- e : t]]$ then $[[G' |- e : t]]$.
\end{lemma}

\begin{proof}
    By trivial induction on the derivation of $[[G |- e : t]]$.
\end{proof}

\begin{lemma}[Substitution]\label{lem:appendix:subst}
\verb|[typing_substitution]|
	If $[[G1, x:T, G2 |- e1:t]]$ and $[[G1 |- e2:T]]$, then $[[G1, G2 [x |-> e2]
|- e1[x |-> e2]  : t[x |-> e2] ]]$.
\end{lemma}

\begin{proof}
    By induction on the derivation of $[[G1, x:T, G2 |- e1:t]]$. We use the notation $[[e* == e
[x |-> e2] ]]$ to denote the substitution for short. Then the result can be written as \[ [[G1, G2* |- e1*  : t* ]]\]
We only treat cases \ruleref{T\_Mu}, \ruleref{T\_CastUp} and
\ruleref{T\_CastDown} since other cases can be easily followed by the proof for PTS in \citeapp{handbook}.
Consider the last step of derivation of the following
cases:
    \begin{description}
        \item[Case \ruleref{T\_Mu}:] $\inferrule{[[G1, x:T, G2, y:t |- e1:t]] \\
[[G1, x:T, G2 |- t:s]]}{[[G1, x:T, G2 |- (mu y:t.e1): t]]}$ 
        
        By the induction hypothesis, we have $[[G1, G2*, y:t* |- e1* : t*]]$ and $[[G1,
G2* |- t* : star]]$. Then by the derivation rule, $[[G1, G2* |- (mu
y:t*.e1*):t*]]$. Thus we can conclude $[[G1, G2* |- (mu y:t.e1)*:t*]]$.
        \item[Case \ruleref{T\_CastUp}:] $\inferrule{[[G1, x:T, G2 |- e1:t2]]
\\ [[G1, x:T, G2 |- t1:s]] \\ [[t1 --> t2]]}{[[G1, x:T, G2 |- (castup [t1]
e1):t1]]}$ 
        
        By the induction hypothesis, we have $[[G1, G2* |- e1*:t2*]]$, $[[G1, G2*
|- t1*:star]]$ and $[[t1 --> t2]]$. By the definition of substitution, we can
obtain $[[t1* --> t2*]]$ by $[[t1 --> t2]]$. Then by the derivation rule, $[[G1,
G2* |- (castup [t1*] e1*):t1*]]$. Thus we can conclude $[[G1, G2* |- (castup [t1]
e1)*:t1*]]$.
        \item[Case \ruleref{T\_CastDown}:] $\inferrule{[[G1, x:T, G2 |- e1:t1]]
\\ [[G1, x:T, G2 |- t2:s]] \\ [[t1 --> t2]]}{[[G1, x:T, G2 |- (castdown
e1):t2]]}$ 
        
        By the induction hypothesis, we have $[[G1, G2* |- e1*:t1*]]$, $[[G1, G2*
|- t2*:star]]$ and $[[t1 --> t2]]$ thus $[[t1* --> t2*]]$. Then by the
derivation rule, $[[G1, G2* |- (castdown e1*):t2*]]$. Thus we can conclude $[[G1, G2* |-
(castdown e1)*:t2*]]$.
    \end{description}
\end{proof}

\begin{lemma}[Inversion\footnote{We only formalized some necessary
    cases, i.e., the ones marked with Coq identifiers, since others
    could be easily derived by the \texttt{inversion}
    tactic.}]\label{lem:appendix:gen}
\begin{enumerate}[(1)]
	\item If $[[G |- x:T]]$, then there exists an expression $[[t]]$ such that $[[t
=a T]]$, $[[G |- t:s]]$ and $[[x:t elt G]]$.
	\item If $[[G |- e1 e2:T]]$, then there exist expressions $[[t1]]$ and
$[[t2]]$ such that $[[G |- e1 : (Pi x:t2.t1)]]$, $[[G |- e2:t2]]$ and $[[T =a
t1[x |-> e2] ]]$.
	\item \verb|[typing_abs_inv]| If $[[G |- (\x:t1.e):T]]$, then there exists an expression $[[t2]]$ such
that $[[T =a Pi x:t1.t2]]$ where $[[G |- (Pi x:t1.t2):s]]$ and $[[G,x:t1 |-
e:t2]]$.
    \item \verb|[typing_prod_inv]| If $[[G |- (Pi x:t1.t2):T]]$, then $[[T == s]]$, $[[G |- t1:s]]$ and
$[[G, x:t1 |- t2:s]]$.
	\item If $[[G |- (mu x:t.e):T]]$, then $[[G |- t:s]]$, $[[T =a t]]$ and $[[G,
x:t|-e:t]]$.
	\item \verb|[typing_castup_inv]| If $[[G |- (castup [t1] e):T]]$, then there exists an expression $[[t2]]$
such that $[[G |- e:t2]]$, $[[G |- t1:s]]$, $[[t1 --> t2]]$ and $[[T =a t1]]$.
	\item If $[[G |- (castdown e):T]]$, then there exist expressions
$[[t1]],[[t2]]$ such that $[[G |- e:t1]]$, $[[G |- t2:s]]$, $[[t1 --> t2]]$ and
$[[T =a t2]]$.
\end{enumerate}
\end{lemma}

\begin{proof}
    Consider a derivation of $[[G |- e:T]]$ for one of cases in the lemma. We
follow the process of derivation until expression $[[e]]$ is introduced the
first time. The last step of derivation can be done by
    \begin{itemize}
        \item rule \ruleref{T\_Var} for case 1;
        \item rule \ruleref{T\_App} for case 2;
        \item rule \ruleref{T\_Lam} for case 3;
        \item rule \ruleref{T\_Pi} for case 4;
        \item rule \ruleref{T\_Mu} for case 5;
        \item rule \ruleref{T\_CastUp} for case 6;
        \item rule \ruleref{T\_CastDown} for case 7.
    \end{itemize}
    In each case, assume the conclusion of the rule is $[[G' |- e : t']]$ where
$[[G']] \subseteq [[G]]$ and $[[t' =a T]]$. Then by inspection of used
derivation rules and Lemma \ref{lem:appendix:thin}, it can be shown that the
statement of the lemma holds and is the only possible case.
\end{proof}

\begin{lemma}[Determinacy of Reduction]\label{lem:appendix:determ}
\verb|[reduct_determ]|
    If $[[t --> t1]]$ and $[[t --> t2]]$, then $[[t1 == t2]]$.
\end{lemma}

\begin{proof}
    Trivial induction on the derivation of $[[t --> t1]]$.
\end{proof}

\begin{lemma}[Well-typedness of Reduction]\label{lem:appendix:wfreduct}
\verb|[typing_wf_from_reduct]|
    If $[[G |- t:s]]$ and $[[t --> t']]$, then $[[G |- t' : s]]$.
\end{lemma}

\begin{proof}
    Trivial induction on the derivation of $[[t --> t']]$.
\end{proof}

\begin{lemma}[Correctness of Types]\label{lem:appendix:corrtyp}
\verb|[typing_wf_from_typing]|
    If $[[G |- e:t]]$ then $[[G |- t : s]]$.
\end{lemma}

\begin{proof}
    Trivial induction on the derivation of $[[G |- e:t]]$ using Lemma
\ref{lem:appendix:gen}.
\end{proof}

\subsection{Decidability of Type Checking}
\begin{lemma}[Decidability of One-step Reduction]\label{lem:appendix:unired}
  \verb|[reduct_dec]| If there is a well-typed term $e$ such that
  $[[G |- e : t]]$, it is decidable to determine whether there exists
  $e'$ such that $[[e --> e']]$.
\end{lemma}

\begin{proof}
	By induction on the structure of $[[e]]$:
	\begin{description}
        \item[Case $[[e=x]]$:] $[[e]]$ is a variable which does not match any rules of $[[-->]]$. 
        Thus there is no $[[e]]'$ such that $[[e-->e']]$.
		\item[Case $[[e=v]]$:] $[[e]]$ is a value that has one of the following forms:
		\begin{inparaenum}[(1)]
		    \item $[[star]]$,
			\item $[[\x:t.e]]$,
			\item $[[Pi x:t1.t2]]$,
			\item $[[castup [t] v]]$.
		\end{inparaenum}
		Thus, it does not match any rules of $[[-->]]$. Then there is no $[[e]]'$ such that $[[e-->e']]$.
                \item[Case $[[e]]=[[mu x:t.e1]]$:] Only rule
                \ruleref{S\_Mu} can be applied. Thus, there exists
                $[[e]]'=[[e1[x|->mu x:t.e1] ]]$.
		\item[Case $[[e]]=[[(\x:t.e1) e2]]$:] Since the first
                  term $[[\x:t.e1]]$ is a value, rule \ruleref{S\_App}
                  does not apply to this case. Thus, only rule
                  \ruleref{S\_Beta} can be applied and there exists
                  $[[e']]=[[ e1[x|->e2] ]]$.
                
		\item[Case $[[e]]=[[e1 e2]]$ and $[[e1]]$ is not a
                  $\lambda$-term:] If $[[e1]]=v$ and is not a
                  $\lambda$-term, there is no rule to reduce $[[e]]$.
                  Then there is no $[[e1']]$ such that
                  $[[e1 --> e1']]$, which does not satisfy the premise
                  of rule \ruleref{S\_App}. Thus, there is no $[[e]]'$
                  such that $[[e-->e']]$.

                  Otherwise, if $[[e1]]$ is not a value, by IH it is
                  decidable to know if $[[e1 --> e1']]$. Suppose there
                  exists some $[[e1']]$ such that $[[e1 --> e1']]$. By
                  rule \ruleref{S\_App}, $[[e]]'=[[e1' e2]]$ is the
                  unique reduction of $[[e]]$. If there is no such
                  $[[e1']]$, then there does not exist $[[e']]$.
                \item[Case $[[e]]=[[castup [t] e1]]$ and $[[e1]]$ is
                  not a value:] By IH it is decidable to know if
                  $[[e1 --> e1']]$. Suppose there exists $[[e1']]$
                  such that $[[e1 --> e1']]$. Then by rule
                  \ruleref{S\_CastUp}, there exists
                  $[[e']]=[[castup [t] e1']]$. Otherwise, there is no
                  such $[[e']]$.
		\item[Case $[[e]]=[[castdown (castup [t] v)]]$:] Since
                  $[[castup [t] v]]$ is a value, rule
                  \ruleref{S\_CastDown} does not apply to this
                  case. Thus, only rule \ruleref{S\_CastElim} can be
                  applied and there exists $[[e']]=[[v]]$.
		\item[Case $[[e]]=[[castdown e1]]$ and $[[e1]]$ is not
                  $[[castup [t] v]]$:] If $[[e1]]=v$ and is not a
                  $[[castup [t] v]]$, there is no rule to reduce
                  $[[e1]]$.  Then there is no $[[e1']]$ such that
                  $[[e1 --> e1']]$, which does not satisfy the premise
                  of rule \ruleref{S\_CastDown}. Thus, there is no
                  $[[e]]'$ such that $[[e-->e']]$.

                  Otherwise, if $[[e1]]$ is not a value, by IH it is
                  decidable to know if $[[e1 --> e1']]$. Suppose there
                  exists some $[[e1']]$ such that $[[e1 --> e1']]$.
                  Thus, by rule \ruleref{S\_CastDown},
                  $[[e]]'=[[castdown e1']]$ is the unique reduction of
                  $[[e]]$. If there is no such $[[e1']]$, then there
                  does not exist $[[e']]$.
	\end{description}
\end{proof}

\begin{lemma}[Uniqueness of Typing]
\verb|[typing_unique]|
If $[[G |- e : t1]]$ and $[[G |- e : t2]]$, then $[[t1 == t2]]$.
\end{lemma}

\begin{proof}
  By trivial induction on the derivation of $[[G |- e : t1]]$. We only
  treat the interesting case \textsc{T\_CastDown}.

  Suppose $e = [[castdown e1]]$. By induction hypothesis, we have
  $[[G |- e1 : t1']]$, $[[G |- e1 : t2']]$ and $[[t1' == t2']]$. By
  inversion, we have $[[t1' --> t1]]$ and $[[t2' --> t2]]$. Thus, by
  Lemma \ref{lem:appendix:determ}, we have $[[t1 == t2]]$.
\end{proof}

\begin{theorem}[Decidability of Type Checking]
\verb|[typing_decidable]|
Given a well-formed context $[[G]]$ and a term $[[e]]$, it is decidable
to determine if there exists $[[t]]$ such that $[[G |- e : t]]$.
\end{theorem}

\begin{proof}
	By induction on the structure of $[[e]]$:
	\begin{description}
	    \item[Case $[[e=star]]$:] Trivial by applying \ruleref{T\_Ax} and $[[t ==
star]]$.
		\item[Case $[[e=x]]$:] Trivial by rule \ruleref{T\_Var}. If $[[x:t elt G]]$, then $[[t]]$ is the
unique type of $[[x]]$ such that $[[G |- x : t]]$. Otherwise, if $[[x]] \not \in \dom([[G]])$, there is no such $[[t]]$.
		\item[Case $[[e]]=[[e1 e2]]$:] By rule \ruleref{T\_App} and induction
hypothesis, there exist unique $[[t1]]$ and $[[t2]]$ such that $[[G
|- e1 : (Pi x:t1.t2)]]$, $[[G |- e2:t1]]$. Thus, $[[t2[x |-> e2] ]]$ is the unique type of $[[e]]$ such that $[[G |- e : t2[x |-> e2] ]]$.
		\item[Case $[[e=\x:t1.e1]]$:] By rule \ruleref{T\_Lam} and induction
hypothesis, there exist unique $[[t2]]$ such that $[[G |- (Pi
x:t1.t2):s]]$ and $[[G,x:t1 |- e:t2]]$. Thus, $[[Pi x:t1.t2 ]]$ is the unique type of $[[e]]$ such that $[[G |- e : Pi x:t1.t2  ]]$.
		\item[Case $[[e=Pi x:t1.t2]]$:] By rule \ruleref{T\_Pi} and induction
hypothesis, we have $[[G |- t1:s]]$ and $[[G, x:t1 |- t2:s]]$. Thus, $[[s]]$ is the unique type of $[[e]]$ such that $[[G |- e : s  ]]$.
		\item[Case $[[e=mu x:t.e1]]$:] By rule \ruleref{T\_Mu} and induction
hypothesis, we have $[[G |- t:s]]$ and $[[G, x:t|-e:t]]$. Thus, $[[t]]$ is the unique type of $[[e]]$ such that $[[G |- e : t]]$.
		\item[Case $[[e]]=[[castup [t1] e1]]$:] From the premises of rule
\ruleref{T\_CastUp}, by the induction hypothesis, we can derive the type of
$[[e1]]$ as $[[t2]]$ by $[[G |- e1:t2]]$, and check whether $[[t1]]$ is legal by $[[G |- t1:star]]$. 
For a legal $[[t1]]$, by Lemma \ref{lem:appendix:unired} and \ref{lem:appendix:determ}, there is
a unique $[[t1']]$ such that $[[t1 --> t1']]$ or there is no such $[[t1']]$. 
If such $[[t1']]$ does not exist, then we report type checking fails. 

Otherwise, we examine if $[[t1']]$ is syntactically equal to $[[t2]]$, 
i.e., $[[t1' =a t2]]$. If the equality
holds, we conclude the unique type of $[[e]]$ is $[[t1]]$, i.e., $[[G |- e:t1]]$. Otherwise, we
report $[[e]]$ fails to type check.
		\item[Case $[[e]]=[[castdown e1]]$:] From the premises of rule
\ruleref{T\_CastDown}, by the induction hypothesis, we can derive the type of
$[[e1]]$ as $[[t1]]$ by $[[G |- e1:t1]]$. By Lemma \ref{lem:appendix:unired} and \ref{lem:appendix:determ}, there is a unique
$[[t2]]$ such that $[[t1 --> t2]]$ or such $[[t2]]$ does not exist. 

If such $[[t2]]$ exists and its sorts is
$[[star]]$, we find the unique type of $[[e]]$ is $[[t2]]$ and can conclude $[[G |- e:t2]]$. Otherwise, we
report $[[e]]$ fails to type check.
	\end{description}
\end{proof}

\subsection{Type Safety}
\begin{theorem}[Subject Reduction]
\verb|[subject_reduction_result]|
If $[[G |- e:T]]$ and $[[e]] [[-->]] e'$ then $[[G |- e':T]]$.
\end{theorem}

\begin{proof}
%     We prove the case for one-step reduction, i.e., $[[e --> e']]$. The theorem
% follows by induction on the number of one-step reductions of $[[e]] [[->>]]
% [[e']]$.
    The proof is by induction with respect to the definition of one-step
reduction $[[-->]]$ as follows:
    \begin{description}
        \item[Case $\ottdruleSXXBeta{}$:] $\quad$ \\
        Suppose $[[G |- (\x:t1.e1)e2 :T]]$ and $[[G |- e1 [x |-> e2] :T']]$. By
Lemma \ref{lem:appendix:gen}(2), there exist expressions $[[t1']]$ and $[[t2]]$
such that 
        \begin{align}
            &[[G |- (\x:t1.e1):(Pi x:t1'.t2)]] \label{equ:lam} \\
            &[[G |- e2:t1']] \nonumber \\
            &[[T =a t2 [x |-> e2] ]] \nonumber
        \end{align}
        By Lemma \ref{lem:appendix:gen}(3), the judgment (\ref{equ:lam})
implies that there exists an expression $[[t2']]$ such that
        \begin{align}
            &[[Pi x:t1'.t2 =a Pi x:t1.t2']] \label{equ:lameq}\\
            &[[G, x:t1 |- e1:t2']] \nonumber
        \end{align}
        Hence, by (\ref{equ:lameq}) we have $[[t1 =a t1']]$ and $[[t2 =a
t2']]$. Then we can obtain $[[G, x:t1 |- e1:t2]]$ and $[[G |- e2:t1]]$. By
Lemma \ref{lem:appendix:subst}, we have $[[G |- e1[x |-> e2] : t2[x |-> e2]
]]$. Therefore, we conclude with $[[T' =a t2[x |-> e2] ]] [[=a]] [[T]]$.
        
        \item[Case $\ottdruleSXXApp{}$:] $\quad$ \\
        Suppose $[[G |- e1 e2 :T]]$ and $[[G |- e1' e2 :T']]$. By Lemma
\ref{lem:appendix:gen}(2), there exist expressions $[[t1]]$ and $[[t2]]$ such
that 
        \begin{align*}
            &[[G |- e1:(Pi x:t1.t2)]] \\
            &[[G |- e2:t1]]\\
            &[[T =a t2 [x |-> e2] ]]
        \end{align*}
        By the induction hypothesis, we have $[[G |- e1':(Pi x:t1.t2)]]$. By rule
\ruleref{T\_App}, we obtain $[[G |- e1' e2 : t2[x |-> e2] ]]$. Therefore, $[[T'
=a t2[x |-> e2] ]] [[=a]] [[T]]$.
        
        \item[Case $\ottdruleSXXCastDown{}$:] $\quad$ \\
        Suppose $[[G |- castdown e :T]]$ and $[[G |- castdown e' :T']]$. By
Lemma \ref{lem:appendix:gen}(7), there exist expressions $[[t1]], [[t2]]$ such
that 
        \begin{align*}
            &[[G |- e:t1]] \qquad [[G |- t2:s]] \\
            &[[t1 --> t2]] \qquad [[T =a t2 ]]
        \end{align*}
        By the induction hypothesis, we have $[[G |- e':t1]]$. By rule
\ruleref{T\_CastDown}, we obtain $[[G |- castdown e' : t2 ]]$. Therefore, $[[T'
=a t2]] [[=a]] [[T]]$.

\item[Case $\ottdruleSXXCastUp{}$:] $\quad$ \\
        Suppose $[[G |- castup [t1] e :T]]$ and $[[G |- castup [t1] e' :T']]$. By
Lemma \ref{lem:appendix:gen}(6), there exist expressions $[[t2]]$ such
that 
        \begin{align*}
            &[[G |- e:t2]] \qquad [[G |- t1:s]] \\
            &[[t1 --> t2]] \qquad [[T =a t1 ]]
        \end{align*}
        By the induction hypothesis, we have $[[G |- e':t2]]$. By rule
\ruleref{T\_CastUp}, we obtain $[[G |- castup [t1] e' : t1 ]]$. Therefore, $[[T'
=a t1]] [[=a]] [[T]]$.
        
        \item[Case $\ottdruleSXXCastElim{}$:] $\quad$ \\
        Suppose $[[G |- castdown (castup [t1] v) :T]]$ and $[[G |- v :T']]$. By
Lemma \ref{lem:appendix:gen}(7), there exist expressions $[[t1']], [[t2]]$ such
that 
        \begin{align}
            &[[G |- (castup [t1] v):t1']] \label{equ:fold} \\
            &[[t1' --> t2]] \label{equ:foldeq1} \\
            &[[T =a t2 ]] \label{equ:foldeq4}
        \end{align}
        By Lemma \ref{lem:appendix:gen}(6), the judgment (\ref{equ:fold})
implies that there exists an expression $[[t2']]$ such that
        \begin{align}
            &[[G |- v:t2']] \label{equ:foldr} \\
            &[[t1 --> t2']] \label{equ:foldeq2} \\
            &[[t1' =a t1]] \label{equ:foldeq3}
        \end{align}
        By (\ref{equ:foldeq1}, \ref{equ:foldeq2}, \ref{equ:foldeq3}) and Lemma
\ref{lem:appendix:unired} we obtain $[[t2 =a t2']]$. From (\ref{equ:foldr}) we
have $[[T' =a t2' ]]$. Therefore, by (\ref{equ:foldeq4}), $[[T' =a t2' ]]
[[=a]] [[t2 =a T]]$.
        
        \item[Case $\ottdruleSXXMu{}$:] $\quad$ \\
        Suppose $[[G |- (mu x:t.e) :T]]$ and $[[G |- e[x |-> mu x:t.e] :T']]$.
By Lemma \ref{lem:appendix:gen}(5), we have $[[T =a t]]$ and $[[G, x:t |-
e:t]]$. Then we obtain $[[G |- (mu x:t.e) : t]]$. Thus by Lemma
\ref{lem:appendix:subst}, we have $[[G |- e[x |-> mu x:t.e] : t[x |-> mu x:t.e]
]]$.
        
        Note that $[[x]]:[[t]]$, i.e., the type of $[[x]]$ is $[[t]]$, then
$[[x]] \notin \FV([[t]])$ holds implicitly. Hence, by the definition of
substitution, we obtain $[[t[x |-> mu x:t.e] == t]]$. Therefore, $[[T' =a t[x
|-> mu x:t.e] ]] [[==]] [[t =a T]]$.
    \end{description}
\end{proof}

\begin{theorem}[Progress]
\verb|[progress_result]|
If $[[empty |- e:T]]$ then either $[[e]]$ is a value $v$ or there exists $[[e]]'$
such that $[[e --> e']]$.
\end{theorem}

\begin{proof}
    By induction on the derivation of $[[empty |- e:T]]$ as follows:
    \begin{description}
        \item[Case $[[e=x]]$:] Impossible, because the context is empty.
        \item[Case $[[e=v]]$:] Trivial, since $[[e]]$ is already a value that
has one of the following forms:
		\begin{inparaenum}[(1)]
		    \item $[[star]]$,
			\item $[[\x:t.e]]$,
			\item $[[Pi x:t1.t2]]$,
			\item $[[castup [t] v]]$.
		\end{inparaenum}
		\item[Case $[[e]]=[[e1 e2]]$:] By Lemma \ref{lem:appendix:gen}(2), there
exist expressions $[[t1]]$ and $[[t2]]$ such that $[[empty |- e1:(Pi x:t1.t2)]]$ and
$[[empty |-e2:t1]]$. Consider whether $[[e1]]$ is a value:
    		\begin{itemize}
    		    \item If $[[e1]]=v$, by Lemma \ref{lem:appendix:gen}(3), it could be either $[[castup [Pi x : t1 . t2] e1']]$ , or a
$\lambda$-term such that $[[e1 == \x:t1.e1']]$ for some $[[e1']]$ satisfying
$[[empty |- e1':t2]]$. Note that $[[Pi x : t1 .t2]]$ is a value, thus there is no $[[t2']]$ such that $[[Pi x : t1 .t2 --> t2']]$ and $[[empty |- e1': t2']]$. Thus, $[[castup [Pi x : t1 . t2] e1']]$ is not well-typed and not a possible case. By rule \ruleref{S\_Beta}, we have $[[(\x:t1.e1') e2 -->
e1' [x |-> e2] ]]$. Thus, there exists $[[e' == e1' [x |-> e2] ]]$ such that
$[[e --> e']]$.
    		    \item Otherwise, by the induction hypothesis, there exists $[[e1']]$ such
that $[[e1 --> e1']]$. Then by rule \ruleref{S\_App}, we have $[[e1 e2 --> e1'
e2]]$. Thus, there exists $[[e' == e1' e2]]$ such that $[[e --> e']]$.
    		\end{itemize}
                \item[Case $[[e]]=[[castup [t] e1]]$ and $[[e1]]$ is
                  not a value:] By IH, $[[e1]]$ is well-typed. Then
                  there exists $[[e1']]$ such that $[[e1 -->
                  e1']]$. Thus there exists $[[e']]=[[castup [t] e1']]$.
		\item[Case $[[e]]=[[castdown e1]]$:] By Lemma \ref{lem:appendix:gen}(7),
there exist expressions $[[t1]]$ and $[[t2]]$ such that $[[empty |- e1:t1]]$ and
$[[t1 --> t2]]$. Consider whether $[[e1]]$ is a value:
		     \begin{itemize}
    		    \item If $[[e1]]=v$, by Lemma \ref{lem:appendix:gen}(6), it must be a
$[[castup]]$-term. Because the type of any other value is still a value and $[[t1 --> t2]]$ does not hold. Thus we have $[[e1 == castup [t1] e1']]$ for some $[[e1']]$
satisfying $[[empty |- e1':t2]]$. Then by rule \ruleref{S\_CastElim}, we can obtain
$[[castdown (castup [t1] e1') --> e1']]$. Thus, there exists $[[e' == e1']]$
such that $[[e --> e']]$.
    		    \item Otherwise, by the induction hypothesis, there exists $[[e1']]$ such
that $[[e1 --> e1']]$. Then by rule \ruleref{S\_CastDown}, we have $[[castdown
e1 --> castdown e1']]$. Thus, there exists $[[e' == castdown e1']]$ such that
$[[e --> e']]$.
    		\end{itemize}
		\item[Case $[[e]]=[[mu x:t.e1]]$:] By rule \ruleref{S\_Mu}, there always
exists $[[e' == e1[x |-> mu x:t.e1] ]]$.
    \end{description}
\end{proof}

\section{Specification and Metatheory of \namef}
Like \ecore, proofs for metatheory of \namef are completely formalized
in Coq. However, we only state lemmas \emph{without} paper proofs in
this section, since proofs for the erased system are very similar to
the ones for PTS~\cite{handbook}. Full proofs can be found in the
following Coq scripts\footnote{\fullurl}: \verb|FullCast_*.v| for the original system and
\verb|CoCMu_*.v| for the erased system. The corresponding name of each
lemma in Coq is marked at the beginning in brackets.

\subsection{Syntax}
\begin{center}
\begin{minipage}{0.55\textwidth}
\gram{\ottef}
\end{minipage}
\begin{minipage}{0.4\textwidth}
\gram{
  \ottGg\ottinterrule
  \ottvf}
\end{minipage}
\end{center}
    
\subsection{Erasure}
\erasuredef

\subsection{Typing}
\ottdefnctx{}\ottinterrule
\ottdefnexprfull{}

\subsection{Erased System}
\begin{description}
\item[Syntax]
\hfill \\[5pt]
\begin{center}
\gram{\otter\ottinterrule
        \ottGr\ottinterrule
\ottvu}
\end{center}
\hfill \\[5pt]

\item[Weak-head Reduction]
\hfill \\[5pt]
\ottdefnstepr{}

\item[Parallel Reduction]
\hfill \\[5pt]
\ottdefnstepp{}
\begin{definition}[Multi-step Parallel Reduction]
    The relation $[[-p*>]]$ is the transitive and reflexive closure of
    $[[-p*>]]$.
\end{definition}

\begin{definition}[Equality up to Parallel Reduction]
  The relation $[[=p]]$ is the reflexive, transitive and symmetric
  closure of $[[-p>]]$.
\end{definition}

\item[Typing]
\hfill \\[5pt]
\ottdefnctxr{}\ottinterrule
\ottdefnexprr{}
\end{description}

\subsection{Decidability of Type Checking}
\begin{lemma}[Uniqueness of Typing]
\verb|[typsrc_unique]|
If $[[G |- e : t1]]$ and $[[G |- e : t2]]$, then $[[t1 == t2]]$.
\end{lemma}

\begin{lemma}[Decidability of Type Checking]
\verb|[typsrc_decidable]|
Given a well-formed context $[[G]]$ and a term $[[e]]$, it is decidable
to determine if there exists $[[t]]$ such that $[[G |- e : t]]$.
\end{lemma}

\subsection{Correctness of Types}
\begin{lemma}[Weakening]
    \verb|[typsrc_weaken]|
    Let $[[G]]$ and $[[G']]$ be well-formed contexts such that $[[G]] \subseteq
[[G']]$. If $[[G |- e : t]]$ then $[[G' |- e : t]]$.
\end{lemma}

\begin{lemma}[Substitution]
\verb|[typsrc_substitution]|
	If $[[G1, x:T, G2 |- e1:t]]$ and $[[G1 |- e2:T]]$, then $[[G1, G2 [x |-> e2]
|- e1[x |-> e2]  : t[x |-> e2] ]]$.
\end{lemma}

\begin{lemma}[Inversion of Full Casts]\label{lem:appendix:gen}
\begin{enumerate}[(1)]
	\item \verb|[typsrc_castup_inv]| If $[[G |- (castupf [t1] e):T]]$, then there exists an expression $[[t2]]$
such that $[[G |- e:t2]]$, $[[G |- t1:s]]$, $[[|t1| -p> |t2|]]$ and $[[T =a t1]]$.
	\item \verb|[typsrc_castdn_inv]| If $[[G |- (castdownf [t2] e):T]]$, then there exists an expression $[[t1]]$
such that $[[G |- e:t1]]$, $[[G |- t2:s]]$, $[[|t1| -p> |t2|]]$ and $[[T =a t2]]$.
\end{enumerate}
\end{lemma}

\begin{lemma}[Correctness of Types]\label{lem:appendix:corrtyp}
\verb|[typsrc_wf_from_typsrc]|
    If $[[G |- e:t]]$ then $[[G |- t : s]]$.
\end{lemma}

\subsection{Soundness of Erasure}
\begin{lemma}[Substitution commutes with erasure]
\verb|[erasure_subst]|
We always have $[[|e [x |-> e']| = |e|[x |-> |e'|] ]]$.
\end{lemma}

\begin{lemma}[Substitution of Parallel Reduction after Erasure]
\verb|[erpared_red_out]| If $[[|e1| -p> |e2|]]$, then
  $[[|e1 [x |-> e]| -p> |e2 [x |-> e]| ]]$.
\end{lemma}

\begin{lemma}[Soundness of Erasure]
  \verb|[typsrc_to_typera]| If $[[G |- e : t]]$ then
  $[[|G| |- |e| : |t|]]$.
\end{lemma}

\subsection{Properties of Parallel Reduction}
\begin{lemma}[Reflexivity of $[[-p>]]$]
  \verb|[pared_red_refl]| If $[[r]]$ is a well-formed term, then
  $[[r -p> r]]$ holds.
\end{lemma}

\begin{lemma}[Substitution of $[[-p>]]$]
  \verb|[pared_red_out]| If $[[r1 -p> r2]]$, then
  $[[r1 [x |-> r] -p> r2 [x |-> r] ]]$.
\end{lemma}

\begin{lemma}[Confluence of $[[-p>]]$]
  \verb|[pared_iter_confluence]| If $[[r -p*> r1]]$ and
  $[[r -p*> r2]]$, then there exists $[[r']]$ such that
  $[[r1 -p*> r']]$ and $[[r2 -p*> r']]$.
\end{lemma}

\subsection{Type Safety of Erased System}
\begin{lemma}[Weakening]
    \verb|[typera_weaken]|
    Let $[[De]]$ and $[[De']]$ be well-formed contexts such that $[[De]] \subseteq
[[De']]$. If $[[De |- r : rh]]$ then $[[De' |- r : rh]]$.
\end{lemma}

\begin{lemma}[Correctness of Types]
\verb|[typera_wf_from_typera]|
    If $[[De |- r:rh]]$ then $[[De |- rh : s]]$.
\end{lemma}

\begin{lemma}[Substitution]
\verb|[typera_substitution]|
  If $[[De1, x:rh', De2 |- r1:rh]]$ and $[[De1 |- r2:rh']]$, then
  $[[De1, De2 [x |-> r2] |- r1[x |-> r2] : rh[x |-> r2] ]]$.
\end{lemma}

\begin{lemma}[Subject Reduction]
\verb|[subject_reduction_era]|
  If $[[De |- r:rh]]$ and $[[r -p> r']]$ then $[[De |- r':rh]]$.
\end{lemma}

\begin{lemma}[Progress]
\verb|[progress_era]|
  If $[[empty |- r:rh]]$ then either $[[r]]$ is a value $u$ or there
  exists $[[r']]$ such that $[[r --> r']]$.
\end{lemma}

\begin{comment}
\section{Full Specification of Surface Language}\label{sec:app:sufcc}
\subsection{Syntax}
See Figure \ref{fig:appendix:syntax}.
\begin{figure*}
\centering
\gram{\ottpgm\ottinterrule
\ottdecl\ottinterrule
\ottu\ottinterrule
\ottp\ottinterrule
\ottE\ottinterrule
\ottGs}
\begin{align*}
&\text{Syntactic Sugar} \\
&\ottsurfsugar % defined in otthelper.mng.tex
\end{align*}
\caption{Syntax of the surface language}
\label{fig:appendix:syntax}
\end{figure*}

\subsection{Expression Typing}
See Figure \ref{fig:appendix:typing}.

\subsection{Translation to \ecore}
See Figure \ref{fig:appendix:translate}.

\subsection{Type Safety of the Translation}

\begin{theorem}[Type Safety of Expression Translation]
Given a surface language expression $[[E]]$ and context $[[Gs]]$, 
if $[[Gs |- E:A ~> e]]$, $[[Gs |- A:star ~> t]]$ and $[[|- Gs ~> G]]$, then
$[[G |- e:t]]$.
\end{theorem}

\begin{proof}
    By induction on the derivation of $[[Gs |- E : A ~> e]]$. Suppose there is
a core language context $[[G]]$ such that $[[|- Gs ~> G]]$.
    \begin{description}
        \renewcommand{\hlmath}[1]{#1}
        \item[Case $\ottdruleTRXXAx{}$:] $\quad$ \\ Trivial. $[[e]] = [[t]] = [[star]]$ and
$[[G |- star:star]]$ holds by rule \ruleref{T\_Ax}.
        \item[Case $\ottdruleTRXXVar{}$:] $\quad$ \\ Trivial. By rule \ruleref{T\_Var}, we
have $[[|- Gs ~> G]]$, then $[[x]]:[[t]] [[elt]] [[G]]$ where $[[Gs |-
A:star~>t]]$.
        \item[Case $\ottdruleTRXXApp{}$:] $\quad$ \\ Suppose
            \[\begin{array}{l}
            [[Gs |- E1 E2 : A1[x |-> E2] ~> e1 e2]] \\
            [[Gs |- A1[x |-> E2] : star ~> t1 [x |-> e2] ]].
            \end{array} \]
            By induction
            hypothesis, we have 
            $
            [[G |- e1 : (Pi x:t2.t1)]],
            [[G |- e2:t2]],
            $
            where
            \[\begin{array}{l}
             [[Gs |- E1 : (Pi x:A2.A1) ~> e1]] \\
              [[Gs |- (Pi x:A2.A1) : star ~> (Pi x:t2.t1)]] \\
              [[Gs |- E2 : A2 ~> e2]] \\
              [[Gs |- A2 : star ~> t2]].
            \end{array}\] Thus by rule \ruleref{T\_App}, we can conclude $[[G |- e1 e2 : t1 [x |-> e2] ]]$.
        \item[Case $\ottdruleTRXXLam{}$:] $\quad$ \\ Suppose
            \[\begin{array}{l}
            [[Gs |- (\x:A1.E):(Pi x:A1.A2) ~> \x:t1.e]] \\ 
            [[Gs |- Pi x:A1.A2 : star ~> Pi x:t1.t2]].
            \end{array} \]
            By the induction hypothesis, we have 
            $
            [[G, x : t1 |- e:t2]],
            [[G |- Pi x:t1.t2 : star]]
            $
            where 
            \[
            \begin{array}{ll}
            [[Gs, x : A1 |- E : A2 ~> e]] & \\
            [[Gs |- A1 : star ~> t1]] & [[Gs |- A2 : star ~> t2]] \\
            [[Gs |- (Pi x:A1.A2) : s ~> Pi x:t1.t2]] &
            \end{array}
            \]
            Thus by rule \ruleref{T\_Lam}, we can conclude $[[G |- (\x:t1.e):(Pi x:t1.t2)]]$.
        \item[Case $\ottdruleTRXXPi{}$:] $\quad$ \\ Suppose 
                \[ [[Gs |- (Pi x:A1.A2):s ~> Pi x:t1.t2]]. \] 
            By the induction hypothesis, we have 
            $
                [[G |- t1 : star]], [[G, x : t1 |- t2 : star]]
            $
            where
            $
                [[Gs |- A1 : s ~> t1]], [[Gs, x: A1 |- A2 : s ~> t2]]
            $
            Thus by rule \ruleref{T\_Pi} we can conclude $[[G |- (Pi x:t1.t2) : star]]$.
        \item[Case $\ottdruleTRXXMu{}$:] $\quad$ \\ Suppose 
                \[\begin{array}{l}
                    [[Gs |- (mu x:A . E):A ~> mu x:t.e]] \\
                    [[Gs |- A : star ~> t]]. 
                \end{array}\]
            By the induction hypothesis, we have 
                \[ [[G, x : t |- e : t]],\text{ where }[[Gs, x:A |- E:A ~> e]]. \] 
            Thus by rule \ruleref{T\_Mu}, we can conclude $[[G |- (mu x:t.e) : t]]$.
        \item[Case $\ottdruleTRXXCase{}$:] $\quad$ \\ Suppose 
            \[\begin{array}{l}
                [[Gs |- case E1 of << p => E2>> : B ~> (unfoldnp e1) T <<e2>>]] \\
                [[Gs |- B : star ~> T]].
            \end{array}\]
            By the induction hypothesis, we have 
            \[\begin{array}{ll}
                [[Gs |- E1 : D@<<U>>n ~> e1]] &
                [[Gs |- D@<<U>>n : star ~> t1]] \\
                [[G |- e1 : t1]] &
                [[<< Gs |- p => E2 : D@<<U>>n -> B ~> e2 >>]]            
            \end{array}\]
            By rule \ruleref{TRpat\_Alt}, we have
            \begin{align*}
                [[p]] &[[==]] [[K <<x:A[<< u |-> U >>]>>]] \\
                [[<<e2>>]] &[[==]] [[<<\ <<x:t'>> .e>>]]
            \end{align*}
            where
            \[\begin{array}{ll}
                [[<<Gs |- E2 : B ~> e>>]] &
                [[<<G |- e : T>>]] \\
                [[<<Gs |- U : star ~> uu'>>]] &
                [[<<Gs |- A[<< u |-> U >>]:star ~> t[<<uu |-> uu'>>]>>]] \\
                [[t']] [[==]] [[ t[<<uu |-> uu'>>] ]]
            \end{array}\]
            By rule \ruleref{TRdecl\_Data}, we have $[[D]]  [[ == ]] \ottdeclD$. Thus,
            \[ [[t1]] [[==]] [[D]] [[<<uu'>>]]^n,\text{ where }[[<<G |- uu' : ro>>]].\] 
            Note that by operational semantics, the following reduction sequence follows for $[[t1]]$:
            \begin{align*}
                [[D]] [[<<uu'>>]]^n~
                &[[-->]]~ [[(\ <<u:ro>>n . (bb:star) -> << ((<<x : t[D |-> X][X |-> D]>>) -> bb) >> -> bb) ]][[<<uu'>>]]^n\\
                &[[-->>]]~ [[(bb:star) -> << (<<x:t'>>) -> bb >> -> bb]]
            \end{align*}
            Then by
            rule \ruleref{T\_CastDown} and the definition of $n$-step cast operator, the
            type of $[[unfoldnp e1]]$ is \[ [[(bb:star) -> << (<<x:t'>>) -> bb >> -> bb]].\] Note
            that by rule \ruleref{T\_Lam}, $[[G |- e2 : (<<x:t'>>) -> T]]$. Therefore, by rule
            \ruleref{T\_App}, we can conclude $[[G |- (unfoldnp e1) T <<e2>> : T]]$.
    \end{description}
\end{proof}

\begin{figure*}
\small
\begin{spacing}{0.8}
\renewcommand{\hlmath}[1]{}
\renewcommand{\ottdrulename}[1]{\textsc{\replace{#1}{TR}{TS}}}
\renewcommand{\ottcom}[1]{\text{\replace{#1}{translation}{typing}}}
\ottdefnctxtrans{}\ottinterrule
\ottdefnpgmtrans{}\ottinterrule
\ottdefndecltrans{}\ottinterrule % defined in otthelper.mng.tex
\ottdefnpattrans{}\ottinterrule
\ottdefnexprtrans{}
\end{spacing}
\caption{Typing rules of the surface language}
\label{fig:appendix:typing}
\end{figure*}

\begin{figure*}
\small
\begin{spacing}{0.8}
\ottdefnctxtrans{}\ottinterrule
\ottdefnpgmtrans{}\ottinterrule
\ottdefndecltrans{}
\[\hlmath{\ottdecltrans}\]\ottinterrule % defined in otthelper.mng.tex
\ottdefnpattrans{}\ottinterrule
\ottdefnexprtrans{}
\end{spacing}
\caption{Translation rules of the surface language}
\label{fig:appendix:translate}
\end{figure*}
\end{comment}


\end{document}
