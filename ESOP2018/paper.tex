\documentclass[oribibl]{llncs}
\usepackage{llncsdoc}
\usepackage{relsize}

% Basics
\usepackage{fixltx2e}
\usepackage{url}
\usepackage{fancyvrb}
\usepackage{mdwlist}  % Miscellaneous list-related commands
\usepackage{xspace}   % Smart spacing
\usepackage{supertabular}

% https://www.nesono.com/?q=book/export/html/347
% Package for inserting TODO statements in nice colorful boxes - so that you
% won’t forget to fix/remove them. To add a todo statement, use something like
% \todo{Find better wording here}.
\usepackage{todonotes}

%% Math
\usepackage{bm}       % Bold symbols in maths mode

% http://tex.stackexchange.com/questions/114151/how-do-i-reference-in-appendix-a-theorem-given-in-the-body
\usepackage{thmtools, thm-restate}

%% Theoretical computer science
\usepackage{stmaryrd}
\usepackage{mathtools}  % For "::=" ( \Coloneqq )

%% Font
% \usepackage[euler-digits,euler-hat-accent]{eulervm}


%% Some recommended packages.
\usepackage{booktabs}   %% For formal tables:
                        %% http://ctan.org/pkg/booktabs
\usepackage{subcaption} %% For complex figures with subfigures/subcaptions
                        %% http://ctan.org/pkg/subcaption


\usepackage{ottalt}

\usepackage{comment}

% Hyper links
\usepackage{url}
\usepackage{
  nameref,%\nameref
  hyperref,%\autoref
}
\usepackage[capitalise]{cleveref}
% \hypersetup{
%    colorlinks,
%    citecolor=black,
%    filecolor=black,
%    linkcolor=blue,
%    urlcolor=black
% }


% Code highlighting
\usepackage{listings}

\lstset{%
  backgroundcolor=\color{white},
  basicstyle=\small\ttfamily,
  keywordstyle=\sffamily\bfseries,
  captionpos=none,
  columns=flexible,
  lineskip=-1pt,
  keepspaces=true,
  showspaces=false,               % show spaces adding particular underscores
  showstringspaces=false,         % underline spaces within strings
  showtabs=false,                 % show tabs within strings adding particular underscores
  breaklines=true,                % sets automatic line breaking
  breakatwhitespace=true,         % sets if automatic breaks should only happen at whitespace
  escapeinside={(*}{*)},
  literate={->}{{$\rightarrow$}}1 {Top}{{$\top$}}1 {=>}{{$\Rightarrow$}}1 {/\\}{{$\Lambda$}}1,
  tabsize=2,
  commentstyle=\color{purple}\ttfamily,
  stringstyle=\color{red}\ttfamily,
  sensitive=false
}

\lstdefinelanguage{sedel}{
  keywords={Int, String, this, trait, inherits, super, type, Trait, override, self, new, if, then, else, let, in},
  identifierstyle=\color{black},
  morecomment=[l]{--},
  morecomment=[l]{//},
  morestring=[b]",
  xleftmargin  = 3mm,
  morestring=[b]'
}

\lstdefinelanguage{gbeta}{%
  language     = java,
  morekeywords = {virtual,refine},
  xleftmargin  = 3mm
}

\lstset{language=sedel}

\theoremstyle{remark}
\newtheorem{observation}{Observation}

% General
\newcommand{\code}[1]{\texttt {#1}}
\newcommand{\highlight}[1]{\colorbox{yellow}{#1}}

% Logic
\newcommand{\turns}{\vdash}

% Math
\newcommand{\im}[1]{\lvert #1 \rvert}

% PL
\newcommand{\subst}[2]{\lbrack #1 / #2 \rbrack}
\newcommand{\concatOp}{+\kern-1.3ex+\kern0.8ex}  % http://tex.stackexchange.com/a/4195/73122

% Constructors
\newcommand{\for}[2]{\forall #1. \, #2}
\newcommand{\lam}[2]{\lambda #1. \, #2}
\newcommand{\app}[2]{#1 \; #2}
\newcommand{\blam}[2]{\Lambda #1. #2}
\newcommand{\tapp}[2]{#1 \; #2}

\newcommand{\pair}[2]{\langle #1, #2 \rangle}
\newcommand{\inter}[2]{#1 \,\&\, #2}
\newcommand{\mer}[2]{#1 \, ,, \, #2}
\newcommand{\proj}[2]{{\code{proj}}_{#1} #2}
\newcommand{\ctx}[2]{#1\left\{#2\right\}}
\newcommand{\bra}[1]{\llbracket #1 \rrbracket}


\newcommand{\recordType}[2]{\{ #1 : #2 \}}
\newcommand{\recordCon}[2]{\{ #1 = #2 \}}

\newcommand{\ifThenElse}[3]{\code{if} \; #1 \; \code{then} \; #2 \; \code{else} \; #3}

\newcommand{\defeq}{\triangleq}

\newcommand{\logeq}[2]{#1 \backsimeq_{log} #2}
\newcommand{\kleq}[2]{#1 \backsimeq #2}
\newcommand{\ctxeq}[3]{#1 \backsimeq_{ctx} #2 : #3}

\newcommand{\stepn}{\longmapsto^*}
\newcommand{\step}{\longmapsto}


\usepackage[numbers]{natbib}
\bibliographystyle{abbrvnat}
% \setcitestyle{aysep={}}

% Author notes
\newcommand{\authorNote}[3]{{\color{#2} {\textsc{#1}}: #3}}
% \newcommand\jeremy[1]{\authorNote{Jeremy}{pink}{#1}}
% \newcommand\bruno[1]{\authorNote{Bruno}{red}{#1}}
% \newcommand\tom[1]{\authorNote{Tom}{orange}{#1}}
\newcommand\name{$\lambda_{i}^{\!+}$\xspace}
\newcommand\oname{$\lambda_{i}$\xspace}
\newcommand\fname{$F_{i}$\xspace}
\newcommand\tname{$\lambda_{c}$\xspace}

% Ott includes
\inputott{ott-rules}
% I prefer rulenames on the right
\renewcommand\ottaltinferrule[4]{
  \inferrule*[narrower=0.7,right=#1,#2]
    {#3}
    {#4}
}

\renewcommand{\ottnt}[1]{#1}
\renewcommand{\ottmv}[1]{#1}


% Logical equivalence related macros
\newcommand{\valR}[2]{\mathcal{V}\bra{#1 ; #2}}
\newcommand{\valRR}[1]{\mathcal{V}\bra{#1}}
\newcommand{\eeR}[2]{\mathcal{E}\bra{#1 ; #2}}
\newcommand{\eeRR}[1]{\mathcal{E}\bra{#1}}
\newcommand{\ggR}[2]{\mathcal{G}\bra{#1 ; #2}}

\newcommand{\hl}[2][gray!40]{\colorbox{#1}{#2}}
\newcommand{\hlmath}[2][gray!40]{%
  \colorbox{#1}{$\displaystyle#2$}}


\begin{document}


\title{Coherent Nested Composition with Disjoint Intersection Types}

\ifdefined\submitoption
\subtitle{Long version with appendix}
\fi


\author{Xuan Bi\inst{1}  \and Bruno C. d. S. Oliveira\inst{1}  \and Tom Schrijvers\inst{2}}

\institute{
  The University of Hong Kong, \email{\{xbi,bruno\}@cs.hku.hk}
  \and
  KU Leuven, \email{tom.schrijvers@cs.kuleuven.be}
}

\maketitle


\begin{abstract}
Calculi with \emph{disjoint intersection types} support an introduction form
for intersections called the
\emph{merge operator}, while retaining a \emph{coherent} semantics.
% The interesting feature of such calculi is that they
%retain a coherent semantics, which is known to be hard in the presence
%of the merge operator. 
Disjoint intersections types have great potential to
serve as a foundation for powerful, flexible and yet type-safe and
easy to reason OO languages. This paper shows how to significantly
increase the expressive power of disjoint intersection types by
adding support for \emph{nested subtyping and composition}, which 
enables simple forms of \emph{family polymorphism} to be expressed in the calculus. 
%The motivation for those features is Ernst's \emph{family polymorphism}: the idea 
%that inheritance can be extended from a single class, to a whole
%family of classes. Nested subtyping and composition enable simple forms 
%of family polymorphism to be expressed, while retaining type-safety 
%and coherence. 
The extension with nested subtyping and
composition is challenging, for two different reasons. Firstly, the
subtyping relation that supports these features is non-trivial,
especially when it comes to obtaining an algorithmic version. Secondly,
the syntactic method used to prove coherence for previous calculi with disjoint
intersection types is too inflexible, making it hard to
extend those calculi with new features (such as nested subtyping).
We show how to address the first problem by adapting and extending
 the Barendregt, Coppo and Dezani (BCD) subtyping rules for intersections
with records and coercions. A sound and complete algorithmic
system is obtained by using an approach inspired by Pierce's
work. To address the second
problem we replace the syntactic method to prove coherence,
by a semantic proof method based on \emph{logical relations}.
Our work has been fully formalized in Coq, and we have an implementation
of our calculus.
\end{abstract}


% Main meat

\section{Introduction}
\label{sec:introduction}

Compositionality is a desirable property in programming
designs. Broadly defined, it is the principle that a
system should be built by composing smaller subsystems. For instance,
in the area of programming languages, compositionality is
a key aspect of \emph{denotational semantics}~\cite{scott1971toward, scott1970outline}, where
the denotation of a program is constructed from the denotations of its parts.
% For example, the semantics for a language of simple arithmetic expressions
% is defined as:
% 
% \[\begin{array}{lcl}
% \llbracket n \rrbracket_{E} & = & n \\
% \llbracket e_1 + e_2 \rrbracket_{E} & = & \llbracket e_1 \rrbracket_E + \llbracket  e_2 \rrbracket_E \\
% \end{array}\]
% 
% \bruno{Replace E by fancier symbol?}
% Here there are two forms of expressions: numeric literals and
% additions. The semantics of a numeric literal is just the numeric
% value denoted by that literal. The semantics of addition is the
% addition of the values denoted by the two subexpressions.
Compositional definitions have many benefits.
One is ease of reasoning: since compositional
definitions are recursively defined over smaller elements they
can typically be reasoned about using induction. Another benefit
is that compositional definitions are easy to extend,
without modifying previous definitions.
% For example, if we also wanted to support multiplication,
% we could simply define an extra case:
% 
% \[\begin{array}{lcl}
% \llbracket e_1 * e_2 \rrbracket_E & = & \llbracket e_1 \rrbracket_E * \llbracket  e_2 \rrbracket_E \\
% \end{array}\]

Programming techniques that support compositional
definitions include:
\emph{shallow embeddings} of
Domain Specific Languages (DSLs)~\cite{DBLP:conf/icfp/GibbonsW14}, \emph{finally
  tagless}~\cite{CARETTE_2009}, \emph{polymorphic embeddings}~\cite{hofer_polymorphic_2008} or
\emph{object algebras}~\cite{oliveira2012extensibility}. These techniques allow us to create
compositional definitions, which are easy to extend without
modifications. Moreover, when modeling semantics, both finally tagless and object algebras
support \emph{multiple interpretations} (or denotations) of
syntax, thus offering a solution to the well-known \emph{Expression Problem}~\cite{wadler1998expression}.
Because of these benefits these techniques have become
popular both in the functional and object-oriented
programming communities.

However, programming languages often only support simple compositional designs
well, while support for more sophisticated compositional designs is lacking.
For instance, once we have multiple interpretations of syntax, we may wish to
compose them. Particularly useful is a \emph{merge} combinator,
which composes two interpretations~\cite{oliveira2012extensibility,
oliveira2013feature, rendel14attributes} to form a new interpretation that,
when executed, returns the results of both interpretations. 

% For example, consider another pretty printing interpretation (or
% semantics) $\llbracket \cdot \rrbracket_P$ for arithmetic expressions, which
% returns the string that denotes the concrete syntax of the
% expression. Using merge we can compose the two interpretations to
% obtain a new interpretation that executes both printing and evaluation:
% \jeremy{Explain what is $E\,\&\,P$?}
% 
% \[\begin{array}{lcl}
% \llbracket \cdot \rrbracket_E \otimes \llbracket \cdot \rrbracket_P & = & \llbracket \cdot \rrbracket_{E\,\&\,P} \\
% \end{array}\]

The merge combinator can be manually defined in existing programming languages,
and be used in combination with techniques such as finally tagless or object
algebras. Moreover variants of the merge combinator are useful to
model more complex combinations
of interpretations. A good example are so-called \emph{dependent} interpretations,
where an interpretation does not depend \emph{only} on itself, but also on 
a different interpretation. These definitions with dependencies are quite
common in practice, and, although they are not orthogonal to the interpretation they
depend on, we would like to model them (and also mutually dependent interpretations)
in a modular and compositional style.

% For example consider the following two
% interpretations ($\llbracket \cdot \rrbracket_{\mathsf{Odd}}$ and
% $\llbracket \cdot \rrbracket_{\mathsf{Even}}$) over Peano-style natural numbers:

% \[\begin{array}{lclclcl}
% \llbracket 0 \rrbracket_{\mathsf{Even}}  & = & \mathsf{True} & ~~~~~~~~~~~~~~~~~~~~ & \llbracket 0 \rrbracket_{\mathsf{Odd}} & = & \mathsf{False} \\
% \llbracket S~e \rrbracket_{\mathsf{Even}} & = & \llbracket e \rrbracket_{\mathsf{Odd}} & ~~ & \llbracket S~e \rrbracket_{\mathsf{Odd}} & = & \llbracket e \rrbracket_{\mathsf{Even}}\\
% \end{array}\]

% \emph{Are these interpretations compositional or not?} Under
% a strict definition of compositionality they are not because
% the interpretation of the parts does not depend \emph{only} on the
% interpretation being defined. Instead both interpretations also depend
% on the other interpretation of the parts. In general,
% definitions with dependencies are quite common in practice.
% In this paper we consider these
% interpretations compositional, and we
% would like to model such dependent (or even mutually dependent)
% interpretations in a modular and compositional style.

Defining the merge combinator in existing
programming languages is verbose and cumbersome, requiring code for every
new kind of syntax. Yet, that code is essentially mechanical and ought to be
automated. 
While using advanced meta-programming techniques enables automating
the merge combinator to a large extent in existing programming
languages~\cite{oliveira2013feature, rendel14attributes}, those techniques have
several problems: error messages can be problematic, type-unsafe reflection
is needed in some approaches~\cite{oliveira2013feature} and
advanced type-level features are required in others~\cite{rendel14attributes}.
An alternative to the merge combinator that supports modular multiple
interpretations and works in OO languages with
support for some form of multiple inheritance and covariant
type-refinement of fields has also been recently
proposed~\cite{zhang19shallow}. 
While this approach is relatively simple, it still
requires a lot of manual boilerplate code for composition of interpretations.

This paper presents a calculus and polymorphic type system with
\emph{(disjoint) intersection types}~\cite{oliveira2016disjoint},
called \fnamee. \fnamee
supports our broader notion of compositional designs, and enables
the development of highly modular and reusable programs. \fnamee
has a built-in merge operator and a powerful subtyping relation that
are used to automate the composition of multiple (possibly dependent)
interpretations. In \fnamee subtyping is coercive and enables the
automatic generation of coercions in a \emph{type-directed} fashion. 
This process is similar to that of other type-directed code generation mechanisms
such as 
\emph{type classes}~\cite{Wadler89typeclasses}, which eliminate 
boilerplate code associated to the \emph{dictionary translation}~\cite{Wadler89typeclasses}.

\fnamee continues a line of
research on disjoint intersection types.
 Previous work on
\emph{disjoint polymorphism} (the \fname calculus)~\cite{alpuimdisjoint} studied the
combination of parametric polymorphism and disjoint intersection
types, but its subtyping relation does not support
BCD-style distributivity rules~\cite{Barendregt_1983} and the type system
also prevents unrestricted intersections~\cite{dunfield2014elaborating}. More recently the \name
calculus (or \namee)~\cite{bi_et_al:LIPIcs:2018:9227} introduced a system with \emph{disjoint
  intersection types} and BCD-style distributivity rules, but did not
account for parametric polymorphism. \fnamee is unique in that it
combines all three features in a single calculus:
\emph{disjoint intersection types} and a \emph{merge operator};
\emph{parametric (disjoint) polymorphism}; and a BCD-style subtyping
relation with \emph{distributivity rules}. The three features together
allow us to improve upon the finally tagless and object
algebra approaches and support advanced compositional designs.
Moreover previous work on disjoint intersection types has shown 
various other applications that are also possible in \fnamee, including: \emph{first-class
  traits} and \emph{dynamic inheritance}~\cite{bi_et_al:LIPIcs:2018:9214}, \emph{extensible records} and \emph{dynamic
  mixins}~\cite{alpuimdisjoint}, and \emph{nested composition} and \emph{family polymorphism}~\cite{bi_et_al:LIPIcs:2018:9227}. 


Unfortunately the combination of the three features has non-trivial
complications. The main technical challenge (like for most other
calculi with disjoint intersection types) is the proof of coherence
for \fnamee. Because of the presence of BCD-style distributivity
rules, our coherence proof is based on the recent approach employed in
\namee~\cite{bi_et_al:LIPIcs:2018:9227}, which uses a
\emph{heterogeneous} logical relation called \emph{canonicity}. To account for polymorphism,
which \namee's canonicity does not support, we originally wanted
to incorporate the relevant parts of System~F's logical relation~\cite{reynolds1983types}.
However, due to a mismatch between the two relations, this did not work. The
parametricity relation has been carefully set up with a delayed type
substitution to avoid ill-foundedness due to its impredicative polymorphism.
Unfortunately, canonicity is a heterogeneous relation and needs to account for
cases that cannot be expressed with the delayed substitution setup of the
homogeneous parametricity relation. Therefore, to handle those heterogeneous
cases, we resorted to immediate substitutions and 
% restricted \fnamee to
\emph{predicative instantiations}.
%other
%measures to avoid the ill-foundedness of impredicative instantiation.
%We have settled on restricting \fnamee to \emph{predicative polymorphism} to
%keep the coherence proof manageable. 
We do not believe that predicativity is a severe restriction in practice, since many source
languages (e.g., those based on the Hindley-Milner type system like Haskell and
OCaml) are themselves predicative and do not require the full generality of an
impredicative core language. Should impredicative instantiation be required,
we expect that step-indexing~\cite{ahmed2006step} can be used to recover well-foundedness, though
at the cost of a much more complicated coherence proof.

The formalization and metatheory of \fnamee are a significant advance over that of
\fname. Besides the support for distributive subtyping, \fnamee removes 
several restrictions imposed by the syntactic coherence
proof in \fname. In particular \fnamee supports unrestricted
intersections, which are forbidden in \fname. Unrestricted
intersections enable, for example, encoding certain forms of 
bounded quantification~\cite{pierce1991programming}.
Moreover the new proof method is more robust
with respect to language extensions. For instance, \fnamee supports the bottom
type without significant complications in the proofs, while it was a challenging
open problem in \fname.
A final interesting aspect is that \fnamee's type-checking is decidable. In the
design space of languages with polymorphism and subtyping, similar mechanisms
have been known to lead to undecidability. Pierce's seminal paper
``\emph{Bounded quantification is undecidable}''~\cite{pierce1994bounded} shows
that the contravariant subtyping rule for bounded quantification in
\fsub leads to undecidability of subtyping.  In \fnamee the
contravariant rule for disjoint quantification retains decidability. 
Since with unrestricted intersections \fnamee can express several
use cases of bounded quantification, \fnamee could be an interesting and
decidable alternative to \fsub.

\begin{comment}
Besides coherence, we show
several other important meta-theoretical results, such as type-safety, 
sound and complete algorithmic subtyping, and
decidability of the type system. Remarkably, unlike 
\fsub's \emph{bounded polymorphism}, disjoint polymorphism
in \fnamee supports decidable type-checking.
\end{comment}

In summary the contributions of this paper are:
\begin{itemize}

\item {\bf The \fnamee calculus,} which is the first calculus to combine 
disjoint intersection types, BCD-style distributive subtyping and 
disjoint polymorphism. We show several meta-theoretical results, such as \emph{type-safety}, \emph{sound and complete algorithmic subtyping},
\emph{coherence} and \emph{decidability} of the type system.
\fnamee includes the \emph{bottom type}, which was considered to be a
significant challenge in previous work on disjoint polymorphism~\cite{alpuimdisjoint}.

\item {\bf An extension of the canonicity relation with polymorphism,}
  which enables the proof of coherence of \fnamee. We show that the ideas of
  System F's \emph{parametricity} cannot be ported to
  \fnamee. To overcome the problem we use a technique based on
  immediate substitutions and a predicativity restriction.

% \item {\bf Disjoint intersection types in the presence of bottom:}
%   Our calculus includes the bottom type, which was considered to be a
% significant challenge in previous work on disjoint polymorphism~\cite{alpuimdisjoint}.

\item {\bf Improved compositional designs:} We show that \fnamee's combination of features
enables improved
compositional programming designs and supports automated composition
of interpretations in programming techniques like object algebras and
finally tagless.

\item {\bf Implementation and proofs:} All of the metatheory
  of this paper, except some manual proofs of decidability, has been
  mechanically formalized in Coq. Furthermore, \fnamee is
  implemented and all code presented in the paper is available. The
  implementation, Coq proofs and extended version with appendices can be found in
  \url{https://github.com/bixuanzju/ESOP2019-artifact}.

\end{itemize}

% \bruno{
% Still need to figure out how to integrate row types in the intro story
% Furthermore, we provide a detailed
% comparison between \emph{distributive disjoint polymorphism} and
% \emph{row types}.
% }

% Compositionality is a desirable property in programming
% designs. Broadly defined, compositionality is the principle that a
% system should be built by composing smaller subsystems.
% In the area of programming languages compositionality is
% a key aspect of \emph{denotational semantics}~\cite{scott1971toward, scott1970outline}, where
% the denotation of a program is constructed from denotations of its parts.
% For example, the semantics for a language of simple arithmetic expressions
% is defined as:
% 
% \[\begin{array}{lcl}
% \llbracket n \rrbracket_{E} & = & n \\
% \llbracket e_1 + e_2 \rrbracket_{E} & = & \llbracket e_1 \rrbracket_E + \llbracket  e_2 \rrbracket_E \\
% \end{array}\]
% 
% \bruno{Replace E by fancier symbol?}
% Here there are two forms of expressions: numeric literals and
% additions. The semantics of a numeric literal is just the numeric
% value denoted by that literal. The semantics of addition is the
% addition of the values denoted by the two subexpressions.
% Compositional definitions have many benefits.
% One is ease of reasoning: since compositional
% definitions are recursively defined over smaller elements they
% can typically be reasoned about using induction. Another benefit
% of compositional definitions is that they are easy to extend,
% without modifying previous definitions.
% For example, if we also wanted to support multiplication,
% we could simply define an extra case:
% 
% \[\begin{array}{lcl}
% \llbracket e_1 * e_2 \rrbracket_E & = & \llbracket e_1 \rrbracket_E * \llbracket  e_2 \rrbracket_E \\
% \end{array}\]
% 
% Programming techniques that support compositional
% definitions include:
% \emph{shallow embeddings} of
% Domain Specific Languages (DSLs)~\cite{DBLP:conf/icfp/GibbonsW14}, \emph{finally
%   tagless}~\cite{CARETTE_2009}, \emph{polymorphic embeddings}~\cite{} or
% \emph{object algebras}~\cite{oliveira2012extensibility}. All those techniques allow us to easily create
% compositional definitions, which are easy to extend without
% modifications. Moreover both finally tagless and object algebras
% support \emph{multiple interpretations} (or denotations) of
% the syntax, thus offering a solution to the infamous \emph{Expression Problem}~\cite{wadler1998expression}.
% Because of these benefits they have become
% popular both in the functional and object-oriented
% programming communities.
% 
% However, programming languages often only support simple
% compositional designs well, while language support for more sophisticated
% compositional designs is lacking. Once we have multiple
% interpretations of syntax, then we may wish to compose those
% interpretations. In particular, when multiple interpretations exist, a useful operation
% is a \emph{merge} combinator ($\otimes$) that composes two
% interpretations~\cite{oliveira2012extensibility, oliveira2013feature, rendel14attributes}, forming a
% new interpretation that, when executed, returns the results of both
% interpretations. For example, consider another pretty printing interpretation (or
% semantics) $\llbracket \cdot \rrbracket_P$ for arithmetic expressions, which
% returns the string that denotes the concrete syntax of the
% expression. Using merge we can compose the two interpretations to
% obtain a new interpretation that executes both printing and evaluation:
% \jeremy{Explain what is $E\,\&\,P$?}
% 
% \[\begin{array}{lcl}
% \llbracket \cdot \rrbracket_E \otimes \llbracket \cdot \rrbracket_P & = & \llbracket \cdot \rrbracket_{E\,\&\,P} \\
% \end{array}\]
% 
% Such merge combinator can be manually defined in existing programming 
% The merge combinator can be manually defined in existing programming
% languages, and be used in combination with techniques such as finally
% tagless or object algebras. Furthermore variants of the
% merge combinator can help express more complex combinations of multiple
% interpretations. For example consider the following two
% interpretations ($\llbracket \cdot \rrbracket_{\mathsf{Odd}}$ and
% $\llbracket \cdot \rrbracket_{\mathsf{Even}}$) over Peano-style natural numbers:
% 
% \[\begin{array}{lclclcl}
% \llbracket 0 \rrbracket_{\mathsf{Even}}  & = & \mathsf{True} & ~~~~~~~~~~~~~~~~~~~~ & \llbracket 0 \rrbracket_{\mathsf{Odd}} & = & \mathsf{False} \\
% \llbracket S~e \rrbracket_{\mathsf{Even}} & = & \llbracket e \rrbracket_{\mathsf{Odd}} & ~~ & \llbracket S~e \rrbracket_{\mathsf{Odd}} & = & \llbracket e \rrbracket_{\mathsf{Even}}\\
% \end{array}\]
% 
% \emph{Are these interpretations compositional or not?} Under
% a strict definition of compositionality they are not because
% the interpretation of the parts does not depend \emph{only} on the
% interpretation being defined. Instead both interpretations also depend
% on the other interpretation of the parts. In general,
% definitions with dependencies are quite common in practice.
% In this paper we consider these
% interpretations compositional, and we
% would like to model such dependent (or even mutually dependent)
% interpretations in a modular and compositional style.
% 
% However defining the merge combinator in existing programming
% languages is verbose and cumbersome, and requires code for every new
% kind of syntax. Yet, that code is essentially mechanical and
% ought to be automated. While using advanced meta-programming
% techniques enables automating the merge combinator to a large extent
% in existing programming languages~\cite{oliveira2013feature, rendel14attributes}, those techniques have
% several problems. For example, error messages can be problematic, some
% techniques rely on type-unsafe reflection, while other techniques
% require highly advanced type-level features.
% 
% This paper presents a calculus and polymorphic type system with
% \emph{(disjoint) intersection types}~\cite{oliveira2016disjoint}, called \fnamee, that
% supports our broader notion of compositional designs, and enables
% the development of highly modular and reusable programs. \fnamee
% has a built-in merge operator and a powerful subtyping relation that
% are used to automate the composition of multiple interpretations
% (including dependent interpretations). \fnamee continues a line of
% research on disjoint intersection types. Previous work on
% \emph{disjoint polymorphism} (the \fname calculus) studied the
% combination between parametric polymorphism and disjoint intersection
% types, but the subtyping relation did not support
% BCD-style distributivity rules~\cite{Barendregt_1983}. More recently the \name
% calculus (or \namee) studied a system with \emph{disjoint
%   intersection types} and BCD-style distributivity rules, but did not
% account for parametric polymorphism. \fnamee is unique in that it
% allows the combination of three useful features in a single calculus:
% \emph{disjoint intersection types} and a \emph{merge operator};
% \emph{parametric (disjoint) polymorphism}; and a BCD-style subtyping
% relation with \emph{distributivity rules}. All three features are
% necessary to use improved versions of finally tagless or object
% algebras that support improved compositional designs.
% 
% Unfortunatelly the combination of the three features has non-trivial
% complications. The main technical challenge (as often is the case for
% calculi with disjoint intersection types) is the proof of coherence
% for \fnamee. Because of the presence BCD-style distributivity
% rules, the proof of coherence is based on the approach using a
% \emph{heterogeneous} logical relation employed in
% \namee~\cite{bi_et_al:LIPIcs:2018:9227}. However the logical relation in
% \namee, which we call here \emph{canonicity}, does not
% account for polymorphism. To account for polymorphism we originally
% expected to simply borrow ideas from \emph{parametricity}~\cite{reynolds1983types} in
% System F~\cite{reynolds1974towards} and adapt them to fit with the canonicity relation.
% However, this did not work. The problem is partly due to the fact that
% canonicity (unlike parametricity) is an heterogenous relation and
% needs to account for heterogeneous cases that are not considered in an
% homogeneous relation such as parametricity. Those heterogeneous cases, combined
% with \emph{impredicative polymorphism}, resulted in an ill-founded logical
% relation. Fortunatelly it turns out that
% restricting the calculus to \emph{predicative polymorphism} and using
% an approach based on substitutions is
% sufficient to recover a well-founded canonicity relation.
% Therefore we
% adopted this approach in \fnamee.
% We do not view
% the predicativity restriction as being very severe in practice, since many
% practical languages have such restriction as well. For example languages based
% on Hindley-Milner style type systems (such as Haskell, OCaml or ML)
% \ningning{it's hard to say this is true. When we say Hindley-Milner type system,
%   or Haskell, we are referring to the source language. However, the core
%   language for, for example Haskell, which is System FC, is impredicative.
%   \fnamee is more close to a core language (which usually has explicit type
%   abstractions/applications). In this sense it's unfair to compare it with other
%   source languages.} all use predicative polymorphism. Furthermore with the
% predicativity restriction, the canonicity relation and corresponding proofs
% remain relatively simple and do not require emplying more complex approaches
% such as \emph{step-indexed logical relations}. \ningning{we should emphasize
%   that predicativity is not a restriction, rather it's choice we made in order
%   to prove coherence in Coq. Step-indexed logical relation might work for
%   impredicativity; it's just we don't know.}
% 
% In summary the contributions of this paper are:
% 
% \begin{itemize}
% 
% \item {\bf The \fnamee calculus,} which integrates disjoint intersection types,
%     distributivity and disjoint polymorphism. \fnamee
%     is the first calculus puts all three features together. The
%     combination is non-trivial, expecially with respect to the
%     coherence proof.
% 
% \bruno{improve text}
% \item {\bf The canonicity logical relation,} which enables the proof
%     of coherence of \fnamee. We show that the ideas of
%   System F's \emph{parametricity} cannot be ported to
%   \fnamee. To overcome the problem we develop a canonicity
%   relation that enables a proof of coherence.
% 
% \item {\bf Disjoint intersection types in the presence of bottom:}
%   Our calculus includes a bottom type, which was considered to be a
% significant challenge in previous work.
% 
% \item {\bf Improved compositional designs:} We show how \fnamee has all the
% features that enable improved
% compositional programming designs and support automated composition
% of interpretations in programming techniques like object algebras and
% finally tagless.
% 
% \item {\bf Implementation and proofs:} All proofs
% (including type-safety, coherence and decidability of the type system)
% are proved in the Coq theorem prover. Furthermore \sedel \ningning{where comes the name \sedel?} and
% \fnamee are implemented and all code presented in the paper is
% available. The implementation, proofs and examples can be found in:
% 
% \url{MISSING}
% 
% \end{itemize}
% 
% \bruno{
% Still need to figure out how to integrate row types in the intro story
% Furthermore, we provide a detailed
% comparison between \emph{distributive disjoint polymorphism} and
% \emph{row types}.
% }

% Local Variables:
% org-ref-default-bibliography: "../paper.bib"
% End:


\section{Overview}
\label{sec:overview}

This section aims at introducing first-class classes and traits, their possible
uses and applications, as well as the typing challenges that arise
from their use.
We start by describing a hypothetical JavaScript library for text editing
widgets, inspired and adapted from Racket's GUI
toolkit~\cite{DBLP:conf/oopsla/TakikawaSDTF12}. The example is illustrative of
typical uses of dynamic inheritance/composition, and also the typing challenges
in the presence of first-class classes/traits. Without diving into
technical details, we then give the corresponding typed version in
\name, and informally presents its salient features.

\subsection{First-Class Classes in JavaScript}

A class construct was officially added to JavaScript in the ECMAScript
2015 Language Specification~\cite{EcmaScript:15}. One purpose of
adding classes to JavaScript was to support a construct that is more
familiar to programmers who come from mainstream class-based languages,
such as Java or C++. However classes in JavaScript are
\emph{first-class} and support functionality not easily mimicked in
statically-typed class-based languages.

\subparagraph{Conventional Classes.}
Before diving into the more advanced features of JavaScript classes, we first
review the more conventional class declarations supported in JavaScript as well
as many other languages. Even for conventional classes there are some
interesting points to note about JavaScript that will be important when we move
into a typed setting. An example of a JavaScript class declaration is:
\begin{lstlisting}[language=JavaScript]
class Editor {
  onKey(key) { return "Pressing " + key; }
  doCut()    { return this.onKey("C-x") + " for cutting text"; }
  showHelp() { return "Version: " + this.version() + " Basic usage..."; }
};
\end{lstlisting}
This form of class definition is standard and very similar to declarations in
class-based languages (for example Java). The \lstinline{Editor} class
defines three methods: \lstinline{onKey} for handling key events,
\lstinline{doCut} for cutting text and \lstinline{showHelp} for displaying help
message. For the purpose of demonstration, we elide the actual implementation,
and replace it with plain messages.

We wish to bring the readers' attention to two points in the above class.
Firstly, note that the \lstinline{doCut} method is defined in terms of the
\lstinline{onKey} method via the keyword
\lstinline[language=JavaScript]{this}. In other words the call to
\lstinline{onKey} is enabled by the \emph{self} reference and is
\emph{dynamically dispatched} (i.e., the particular implementation of
\lstinline{onKey} will only be determined when the class or subclass
is instantiated). % Typically an
% OO programmer seeing this definition would expect the \lstinline{doCut} method
% to call the \lstinline{onKey} method of a subclass of \lstinline{Editor}, even though
% the subclass does not exist when the superclass \lstinline{Editor} is being
% defined.
Secondly, notice that there is no definition of
the \lstinline{version} method in the class body, but such method is used inside the
\lstinline{showHelp} method. In a untyped language, such as JavaScript, using
undefined methods is error prone -- accidentally instantiating \lstinline{Editor}
and then calling \lstinline{showHelp} will cause a runtime error!
Statically-typed languages usually provide some means to protect us from this
situation. For example, in Java, we would need an \textit{abstract} \lstinline{version}
method, which effectively makes \lstinline{Editor} an abstract class and
prevents it from being instantiated. As we will see, \name's treatment of
abstract methods is quite different from mainstream languages. In fact, \name
has a unified (typing) mechanism for dealing with both dynamic dispatch and abstract
methods. We will describe \name's mechanism for dealing with both features and
justify our design in \cref{sec:traits}.

% A couple of things worth pointing out in the above code snippet: (1) the class
% \lstinline{Editor} has no definition of the method
% \lstinline{version}, but such method
% is used in the body of the method \lstinline{showHelp}. In a strongly-typed OO
% language, such as Java, we would need to define an abstract method for
% \lstinline{version}. (2) The \lstinline{Editor} class requires
% \emph{dynamic dispatching}.
%  In the body of the method \lstinline{doCut} we invoke
% the method \lstinline{onKey} defined in the same class through the keyword
% \lstinline[language=JavaScript]{this}. This has the implication that when a
% subclass of \lstinline{Editor} overrides the method \lstinline{onKey}, a call to
% \lstinline{doCut} should invoke \lstinline{onKey} defined in the subclass
% instead of the original one.\bruno{punchline?}
%As we will see later, the type system of \name correctly handles it.

\subparagraph{First-Class Classes and Class Expressions.}
Another way to define a class in JavaScript is via a \emph{class expression}. This is where the class
model in JavaScript is very different from the traditional class model found in
many mainstream OO languages, such as Java, where classes are second-class
(static) entities. JavaScript embraces a dynamic class model that treats classes
as \emph{first-class} expressions: a function can take classes as arguments,
or return them as a result. First-class classes enable programmers to
abstract over patterns in the class hierarchy and to experiment with new forms of OOP
such as mixins and traits. In particular, mixins become programmer-defined
constructs. We illustrate this by presenting a simple mixin that adds
spell checking to an editor:
\begin{lstlisting}[language=JavaScript]
const spellMixin = Base => {
  return class extends Base {
    check()    { return super.onKey("C-c") + " for spell checking"; }
    onKey(key) { return "Process " + key + " on spell editor"; }
  }
};
\end{lstlisting}
In JavaScript, a mixin is simply a function with a superclass as input and a
subclass extending that superclass as an output. Concretely, \lstinline{spellMixin}
adds a method \lstinline{check} for spell checking. It also provides
a method \lstinline{onKey}.
The function \lstinline{spellMixin} shows the typical use of what we call \emph{dynamic inheritance}.
Note that \lstinline{Base}, which is supposed to be a superclass being inherited, is \emph{parameterized}.
Therefore \lstinline{spellMixin} can be applied to any base class at
\emph{runtime}. This is impossible to do, in a type-safe way, in
conventional statically-typed class-based languages like Java or
C++.\footnote{With C++ templates, it is possible to
  implement a so-called mixin pattern~\cite{DBLP:conf/gcse/SmaragdakisB00}, which enables extending
a parameterized class. However C++ templates defer type-checking until
instantiation, and such pattern still does not allow selection of the
base class at runtime (only at up to class instantiation time).}

It is noteworthy that not all applications of \lstinline{spellMixin} to base
classes are successful. Notice the use of the \lstinline{super} keyword in the
\lstinline{check} method. If the base class does not implement the
\lstinline{onKey} method, then mixin application fails with a runtime error. In
a typed setting, a type system must express this requirement (i.e., the presence of
the \lstinline{onKey} method) on the (statically unknown) base class that is
being inherited.


% The class expression inside the function body has no
% definition of the method \lstinline{version}, but which is used in the body of
% the method \lstinline{showHelp}. In a statically-typed OO language, such as Java,
% we would need an \emph{abstract method} for
% \lstinline{version}.


We invite the readers to pause for a while and think about what the type of
\lstinline{spellMixin} would look like. Clearly our type system should be
flexible enough to express this kind of dynamic pattern of composition in order
to accommodate mixins (or traits), but also not too lenient to allow any
composition.


\subparagraph{Mixin Composition and Conflicts.}
The real power of mixins is that \lstinline{spellMixin}'s functionality is not
tied to a particular class hierarchy and is composable with other features. For
example, we can define another mixin that adds simple modal editing -- as in Vim
-- to an arbitrary editor:
\begin{lstlisting}[language=JavaScript]
const modalMixin = Base => {
  return class extends Base {
    constructor() {
      super();
      this.mode = "command";
    }
    toggleMode() { return "toggle succeeded"; }
    onKey(key)   { return "Process " + key + " on modal editor"; }
  };
};
\end{lstlisting}
\lstinline{modalMixin} adds a \lstinline{mode} field that controls which
keybindings are active, initially set to the command mode, and a method
\lstinline{toggleMode} that is used to switch between modes. It also provides a method \lstinline{onKey}.

Now we can compose \lstinline{spellMixin} with \lstinline{modalMixin} to produce
a combination of functionality, mimicking some form of multiple inheritance:
\begin{lstlisting}[language=JavaScript]
class IDEEditor extends modalMixin(spellMixin(Editor)) {
  version() { return 0.2; }
}
\end{lstlisting}
The class \lstinline{IDEEditor} extends the base class \lstinline{Editor} with
modal editing and spell checking capabilities. It also defines the missing
\lstinline{version} method.

At first glance, \lstinline{IDEEditor} looks quite fine, but it has a subtle
issue. Recall that two mixins \lstinline{modalMixin} and \lstinline{spellMixin}
both provide a method \lstinline{onKey}, and the \lstinline{Editor} class also
defines an \lstinline{onKey} method of its own. Now we have a name clash. A
question arises as to which one gets picked inside the \lstinline{IDEEditor}
class. A typical mixin model resolves this issue by looking at the order of mixin applications. Mixins appearing later in the order
overrides \emph{all} the identically named methods of earlier mixins. So in our
case, \lstinline{onKey} in \lstinline{modalMixin} gets picked. If we
change the order of application to \lstinline{spellMixin(modalMixin(Editor))},
then \lstinline{onKey} in \lstinline{spellMixin} is inherited.

\subparagraph{Problem of Mixin Composition.}
From the above discussion, we can see that mixin are composed linearly: all the
mixins used by a class must be applied one at a time. However, when we wish to
resolve conflicts by selecting features from different mixins, we may not be
able to find a suitable order. For example, when we compose the two mixins to
make the class \lstinline{IDEEditor}, we can choose which of them comes first,
but in either order, \lstinline{IDEEditor} cannot access to the \lstinline{onKey}
method in the \lstinline{Editor} class.

\subparagraph{Trait Model.}
Because of the total ordering and the limited means for resolving conflicts imposed by the mixin model,
researchers have proposed a simple compositional model called
traits~\cite{scharli2003traits, Ducasse_2006}. Traits are lightweight entities and serve as
the primitive units of code reuse. Among others, the key difference from
mixins is that the order of trait composition is irrelevant, and conflicting
methods must be resolved \emph{explicitly}. This gives programmers
fine-grained control, when conflicts arise, of selecting desired features from
different components. Thus we believe traits are a better model for multiple
inheritance in statically-typed OO languages, and in \name we realize this
vision by giving traits a first-class status in the language,
achieving more expressive power compared with traditional (second-class) traits.


\subparagraph{Summary of Typing Challenges.}
From our previous discussion, we can identify the following typing challenges
for a type system to accommodate the programming patterns (first-class classes/mixins)
we have just seen in a typed setting:
\begin{itemize}
\item How to account for, in a typed way, abstract methods and dynamic dispatch.
\item What are the types of first-class classes or mixins.
\item How to type dynamic inheritance.
\item How to express constraints on method presence and absence (the use of
  \lstinline{super} clearly demands that).
% \item How to ensure that composition of mixins is going to be valid, i.e., how
%   to reflect linearity in a type system.
\item In the presence of first-class traits, how to detect conflicts statically,
  even when the traits involved are not statically known.
\end{itemize}
\name elegantly solves the above challenges in a unified way, as
we will see next.


% From a pragmatic point of view, this implicit conflict resolution
% sometimes give programmers more surprises than convenience. What if the compiler can alarm us when a
% potential conflict may occur. Because of the dynamic nature of JavaScript, we
% would not know before actually running the code that there is a conflict. We
% miss the guarantee that a static type system can provide: such conflict can be
% detected at compile-time.

% Given the flexibility of first-class classes in dynamically-typed languages, we
% -- being advocates of statically-typed languages -- were wondering how to
% incorporate this same expressive power into statically-typed
% languages. As it
% turns out, designing a sound type system that fully supports first-class classes
% is notoriously hard; there are only a few, quite sophisticated, languages that
% manage this~\cite{DBLP:conf/oopsla/TakikawaSDTF12, DBLP:conf/ecoop/LeeASP15}. We
% pushed it further: \name has support for typed first-class
% traits.\bruno{Better to say there's no work on typed first-class
%   traits, and little work on first-class classes/mixins, despite
%  many dynamic languages prominently supporting such features.}

\subsection{A Glance at Typed First-Class Traits in \name}

We now rewrite the above library in \name, but this time with types. The resulting code has the same functionality as the dynamic version, but is
statically typed. All code snippets in this and later sections are runnable in
our prototype implementation. Before proceeding, we ask the readers to bear in mind that in this section we are not using traits
in the most canonical way, i.e., we use traits as if they are classes (but with
built-in conflict detection). This is because we are trying to stay as close as possible
to the structure of the JavaScript version for ease of comparison. In
\cref{sec:traits} we will remedy this to make better use of traits.

\subparagraph{Simple Traits.}
Below is a simple trait \lstinline{editor}, which corresponds to the JavaScript
class \lstinline{Editor}. The \lstinline{editor} trait defines the same set of
methods: \lstinline{on_key}, \lstinline{do_cut} and \lstinline{show_help}:
\lstinputlisting[linerange=14-18]{../../examples/overview2.sl}% APPLY:linerange=OVERVIEW_EDITOR
The first thing to notice is that \name uses a syntax (similar to Scala's
self type annotations~\cite{odersky2004overview}) where we can give a type annotation to the
\lstinline{self} reference. In the type of \lstinline{self} we use
\lstinline{&} construct to create intersection types. \lstinline{Editor} and \lstinline{Version} are two record types:
\lstinputlisting[linerange=7-8]{../../examples/overview2.sl}% APPLY:linerange=OVERVIEW_EDITOR_TYPES
For the sake of conciseness, \name uses \lstinline{type} aliases to abbreviate types.

\subparagraph{Self-Types Encode Abstract Methods.}
Recall that in the JavaScript class \lstinline{Editor}, the \lstinline{version}
method is undefined, but is used inside \lstinline{showHelp}. How can we express
this in the typed setting, if not with an abstract method? In \name, self-types
play the role of trait requirements. As the first approximation, we
can justify the use of \lstinline{self.version} by noticing that (part of) the
type of \lstinline{self} (i.e., \lstinline{Version}) contains the declaration of
\lstinline{version}. An interesting aspect of \name's trait model is that there
is no need for abstract methods. Instead, abstract methods can be simulated as
requirements of a trait. Later, when the trait is composed with other
traits, \emph{all} requirements on the self-types must be
satisfied and one of the traits in the composition must provide an
implementation of the method \lstinline{version}.
%to this point in \cref{sec:traits}.

As in the JavaScript version, the \lstinline{on_key} method is invoked on
\lstinline{self} in the body of \lstinline{do_cut}. This is allowed as (part of)
the type of \lstinline{self} (i.e., \lstinline{Editor}) contains the signature
of \lstinline{on_key}. Comparing \lstinline{editor} to the JavaScript class
\lstinline{Editor}, almost everything stays the same, except that we now have
the typed version. As a side note, since \name is currently a pure functional OO
language, there is no difference between fields and methods, so we can omit
empty arguments and parameter parentheses.

\subparagraph{First-Class Traits and Trait Expressions.}

\name treats traits as first-class expressions, putting them in the same
syntactic category as objects, functions, and other primitive forms. To
illustrate this, we give the \name version of \lstinline{spellMixin}:
\lstinputlisting[linerange=22-29]{../../examples/overview2.sl}% APPLY:linerange=OVERVIEW_HELP
This looks daunting at first, but \lstinline{spell_mixin} has almost the same structure as
its JavaScript cousin \lstinline{spellMixin}, albeit with
some type annotations. In \name, we use capital letters (\lstinline{A}, \lstinline{B}, $\dots$) to denote type variables, and trait
expressions \lstinline$trait [self : ...] inherits ... => {...}$ to create
first-class traits. Trait expressions have trait
types of the form \lstinline{Trait[T1, T2]} where \lstinline{T1} and \lstinline{T2} denote trait requirements and functionality respectively.
We will explain trait types in \cref{sec:traits}. Despite the structural similarities, there are several significant
features that are unique to \name (e.g., the disjointness operator \lstinline{*}).
We discuss these in the following.



\subparagraph{Disjoint Polymorphism and Conflict Detection.}

\name uses a type system based on \emph{disjoint intersection types}~\cite{oliveira2016disjoint} and
\emph{disjoint polymorphism}~\cite{alpuimdisjoint}. Disjoint intersections
empower \name to detect conflicts statically when trying to compose two
traits with identically named features. For example composing two traits
\lstinline{a} and \lstinline{b} that both provide \lstinline{foo} gives a
type error (the overloaded \lstinline{&} operator denotes trait composition):
\begin{lstlisting}
trait a => { foo = 1 };
trait b => { foo = 2 };
trait c inherits a & b => {}; -- type error!
\end{lstlisting}
Disjoint polymorphism, as a more advanced mechanism, allows detecting conflicts
even in the presence of polymorphism -- for example when a trait is parameterized and its
full set of methods is not statically known. As can be seen,
\lstinline{spell_mixin} is actually a polymorphic function. Unlike ordinary
parametric polymorphism, in \name, a type variable can also have a disjointness
constraint. For instance, \lstinline{A * Spelling & OnKey}
means that \lstinline{A} can be instantiated to any type as long as it \emph{does not}
contain \lstinline{check} and \lstinline{on_key}. To mimic mixins, the
argument \lstinline{base}, which is supposed to be some trait, serves as the
``base'' trait that is being inherited. Notice that the type variable
\lstinline{A} appears in the type of \lstinline{base}, which essentially states
that \lstinline{base} is a trait that contains at least those methods specified
by \lstinline{Editor}, and possibly more (which we do not know statically).
% In summary, \lstinline{Trait[Editor & Version, Editor & A]} (the assigned type
% of \lstinline{base}) specifies that both method \emph{presence} and \emph{absence}.
Also note that leaving out the \lstinline{override} keyword will result in a
type error. The type system is forcing us to be very specific as to what is the
intention of the \lstinline{on_key} method because it sees the same method is
also declared in \lstinline{base}, and blindly inheriting \lstinline{base}
will definitely cause a method conflict. As a final note, the use of \lstinline{super}
inside \lstinline{check} is allowed because the ``super'' trait \lstinline{base}
implements \lstinline{on_key}, as can be seen from its type.


\subparagraph{Dynamic Inheritance.}

Disjoint polymorphism enables us to correctly type dynamic inheritance:
\lstinline{spell_mixin} is able to take any trait that conforms with its
assigned type, equips it with the \lstinline{check} method and overrides its
old \lstinline{on_key} method. As a side note, the use of disjoint polymorphism
is essential to correctly model the mixin semantics. From the type we know
\lstinline{base} has some features specified by \lstinline{Editor}, plus
something more denoted by \lstinline{A}. By inheriting \lstinline{base}, we are
guaranteed that the result trait will have everything that is already contained
in \lstinline{base}, plus more features. This is in some sense similar to row
polymorphism~\cite{wand1994type} in that the result trait is prohibited from
forgetting methods from the argument trait. As we will discuss in
\cref{sec:related}, disjoint polymorphism is more expressive than row
polymorphism.


\subparagraph{Typing Mixin Composition.}
Next we give the typed version of \lstinline{modalMixin} as follows:
\lstinputlisting[linerange=34-41]{../../examples/overview2.sl}% APPLY:linerange=OVERVIEW_MODAL
Now the definition of \lstinline{modal_mixin} should be self-explanatory.
Finally we can apply both ``mixins'' one by one to \lstinline{editor} to create
a concrete editor:
\lstinputlisting[linerange=46-49]{../../examples/overview2.sl}% APPLY:linerange=OVERVIEW_LINE
As with the JavaScript version, we need to fill in the missing
\lstinline{version} method. It is easy to verify that the \lstinline{on_key} method
in \lstinline{modal_mixin} is inherited. Compared with the untyped version,
here this behaviour is reasonable because in each mixin we specifically tags the
\lstinline{on_key} method to be an overriding method. Let us take a close look
at the mixin applications. Since \name is currently explicitly typed, we need to
provide concrete types when using \lstinline{modal_mixin} and \lstinline{spell_mixin}.
In the inner application (\lstinline{spell_mixin Top editor}), we use the top
type \lstinline{Top} to instantiate \lstinline{A} because the \lstinline{editor} trait
provides exactly those method specified by \lstinline{Editor} and nothing more
(hence \lstinline{Top}). In the outer application, we use \lstinline{Spelling}
to instantiate \lstinline{A}. This is where implicit conflict resolution of
mixins happens. We know the result of the inner application actually forms a
trait that provides both \lstinline{check} and \lstinline{on_key}, but the
disjointness constraint of \lstinline{A} requires the absence of \lstinline{on_key},
thus we cannot instantiate \lstinline{A} to \lstinline{Spelling & OnKey} for example
when applying \lstinline{modal_mixin}. Therefore the outer application effectively excludes
\lstinline{on_key} from \lstinline{spell_mixin}.
In summary, the order of mixin applications is reflected by the order
of function applications, and conflict resolution code is implicitly embedded.
Of course changing the mixin application order to \lstinline{spell_mixin ModalEdit (modal_mixin Top editor)} gives the expected behaviour.


Admittedly the typed version is unnecessarily complicated as we were
mimicking mixins by functions over traits. The final editor
\lstinline{ide_editor} suffers from the same problem as the class
\lstinline{IDEEditor}, since there is no obvious way to access the
\lstinline{on_key} method in the \lstinline{editor} trait.\footnote{In fact, as
  we will see in \cref{sec:traits}, we can still access \lstinline{on_key} in
  \lstinline{editor} by the forwarding operator.} \cref{sec:traits}
makes better use of traits to simplify the editor code.



% Note that the use of \lstinline{override} is valid because the type system knows the inherit clause contains \lstinline{on_key}.
% As a bonus, since \name guarantees that there are no potential conflicts in a program,
% we can reason that the version number in \lstinline{modal_editor} is
% \lstinline{0.1}.

%%% Local Variables:
%%% mode: latex
%%% TeX-master: "../paper"
%%% org-ref-default-bibliography: ../paper.bib
%%% End:


\section{Declarative System}


\begin{center}
\begin{tabular}{lrcl} \toprule
  Types & $[[A]], [[B]]$ & \syndef & $[[int]] \mid [[a]] \mid [[A -> B]] \mid [[\/ a. A]] \mid [[unknown]] \mid [[static]] \mid [[gradual]] $ \\
  Monotypes & $[[t]], [[s]]$ & \syndef & $ [[int]] \mid [[a]] \mid [[t -> s]] \mid [[static]] \mid [[gradual]]$ \\
  Castable Types & $[[gc]]$ & \syndef & $ [[int]] \mid [[a]] \mid [[gc1 -> gc2]] \mid [[\/ a. gc]] \mid [[unknown]] \mid [[gradual]] $ \\
  Castable Monotypes & $[[tc]]$ & \syndef & $ [[int]] \mid [[a]] \mid [[tc1 -> tc2]] \mid [[gradual]]$ \\

  Contexts & $[[dd]]$ & \syndef & $[[empty]] \mid [[dd, x: A]] \mid [[dd, a]] $ \\
  Colored Types & $[[A]], [[B]]$ & \syndef & $ [[r@(int)]] \mid [[b@(int)]] \mid [[r@(a)]] \mid [[b@(a)]] \mid [[A -> B]] \mid [[r@ \/ a . A]] \mid [[b@ \/ a. A]] \mid [[b@(unknown)]] \mid [[r@(static)]] \mid [[r@(gradual)]] \mid [[b@(gradual)]]$\\
  Blue Castable Types & $[[b@(gc)]]$ & \syndef & $ [[b@(int)]] \mid [[b@(a)]] \mid [[b@(gc1) -> b@(gc2)]] \mid [[b@ \/ a. b@(gc)]] \mid [[b@(unknown)]] \mid [[b@(gradual)]] $ \\
  Blue Monotypes & $[[b@(t)]]$ & \syndef & $ [[b@(int)]] \mid [[b@(a)]] \mid [[b@(t -> s)]] \mid [[b@(gradual)]]$ \\
  Red Monotypes & $[[r@(t)]]$ & \syndef & $ [[r@(int)]] \mid [[r@(a)]] \mid [[ r@(t)  -> r@(s)]] \mid [[ r@(t) -> b@(s) ]] \mid [[ b@(t) ->  r@(s) ]] \mid [[r@(static)]] \mid [[r@(gradual)]]$ \\
  \bottomrule
\end{tabular}
\end{center}


\renewcommand\ottaltinferrule[4]{
  \inferrule*[narrower=0.7]
    {#3}
    {#4}
}

\drules[dconsist]{$ [[ A ~ B ]] $}{Type Consistent}{refl, unknownR, unknownL, arrow, forall}

\renewcommand\ottaltinferrule[4]{
  \inferrule*[narrower=0.7,right=\scriptsize{#1}]
    {#3}
    {#4}
}

\drules[s]{$ [[dd |- A <: B ]] $}{Subtyping}{forallR, forallLr, forallLb, tvarr, tvarb, intr, intb, arrow,
  unknown, spar, gparr, gparb}


% \begin{definition}[Specification of Consistent Subtyping]
%   \begin{mathpar}
%   \drule{cs-spec}
%   \end{mathpar}
% \end{definition}

\drules[cs]{$ [[dd |- A <~ B ]] $}{Consistent Subtyping}{forallR, forallL, arrow, tvar, int, unknownL, unknownR, spar, gpar}

\drules[]{$ [[dd |- e : A ~~> pe]] $}{Typing}{var, int, gen, lamann, lam, app}

\drules[m]{$ [[dd |- A |> A1 -> A2]] $}{Matching}{forall, arr, unknown}


\section{Target: PBC}

\begin{center}
\begin{tabular}{lrcl} \toprule
  Terms & $[[pe]]$ & \syndef & $[[x]] \mid [[n]] \mid [[\x : A. pe]] \mid [[/\a. pe]] \mid [[pe1 pe2]] \mid [[<A `-> B> pe]] $
  \\ \bottomrule
\end{tabular}
\end{center}


\clearpage
\section{Metatheory}

% \renewcommand{\hlmath}{}

\begin{definition}[Substitution]
  \begin{enumerate}
    \item Gradual type parameter substitution $\gsubst :: [[gradual]] \to [[tc]]$
    \item Static type parameter substitution $\ssubst :: [[static]] \to [[t]]$
    \item Type parameter Substitution $\psubst = \gsubst \cup \ssubst$
  \end{enumerate}
\end{definition}

\ningning{Note substitution ranges are monotypes.}

\begin{definition}[Translation Pre-order]
  Suppose $[[dd |- e : A ~~> pe1]]$ and $[[dd |- e : A ~~> pe2]]$,
  we define $[[pe1]] \leq [[pe2]]$ to mean $[[pe2]] = [[S(pe1)]]$ for
  some $[[S]]$.
\end{definition}


\begin{proposition}
  If $[[ pe1 ]] \leq [[pe2]]$ and $[[ pe2 ]] \leq [[pe1]]$, then $[[pe1]]$ and $[[pe2]]$
  are equal up to $\alpha$-renaming of type parameters.
\end{proposition}

 
\begin{definition}[Representative Translation]
  $[[pe]]$ is a representative translation of a typing derivation $[[dd |- e : A
  ~~> pe]]$ if and only if for any other translation $[[dd |- e : A ~~> pe']]$ such that $[[pe']]
  \leq [[pe]]$, we have $[[pe]] \leq [[pe']]$. From now on we use $[[rpe]]$ to
  denote a representative translation.
\end{definition}

\begin{definition}[Measurements of Translation]
  There are three measurements of a translation $[[pe]]$,
  \begin{enumerate}
  \item $[[ ||pe||e]]$, the size of the expression 
  \item $[[ ||pe||s ]]$, the number of distinct static type parameters in $[[pe]]$
  \item $[[ ||pe||g ]]$, the number of distinct gradual type parameters in $[[pe]]$
  \end{enumerate}
  We use $[[ ||pe|| ]]$ to denote the lexicographical order of the triple
  $([[ ||pe||e ]], -[[ ||pe||s ]], -[[ ||pe||g ]])$.
\end{definition}

\begin{definition}[Size of types]

  \begin{align*}
    [[ || int ||  ]] &= 1 \\
    [[ || a ||  ]] &= 1 \\
    [[ || A -> B  ||  ]] &= [[ || A || ]] + [[ || B || ]] + 1 \\
    [[ || \/a . A ||  ]] &= [[ || A || ]] + 1 \\
    [[ || unknown ||  ]] &= 1 \\
    [[ || static ||  ]] &= 1 \\
    [[ || gradual ||  ]] &= 1
  \end{align*}

\end{definition}


\begin{definition}[Size of expressions]

  \begin{align*}
    [[ || x ||e  ]] &= 1 \\
    [[ || n ||e  ]] &= 1 \\
    [[ || \x : A . pe ||e  ]] &= [[ || A || ]] + [[ || pe ||e ]] + 1 \\
    [[ || /\ a. pe ||e  ]] &= [[ || pe ||e ]] + 1 \\
    [[ || pe1 pe2 ||e  ]] &= [[ || pe1 ||e ]] + [[  || pe2 ||e ]] + 1 \\
    [[ || < A `-> B> pe ||e  ]] &= [[ || pe ||e ]] + [[  || A || ]] + [[  || B || ]] + 1 \\
  \end{align*}

\end{definition}


\begin{lemma} \label{lemma:size_e}
  If $[[dd |- e : A ~~> pe]]$ then $[[ || pe ||e    ]] \geq [[ || e ||e   ]]  $.
\end{lemma}
\begin{proof}
  Immediate by inspecting each typing rule.
\end{proof}

\begin{corollary} \label{lemma:decrease_stop}
  If $[[dd |- e : A ~~> pe]]$ then $[[ || pe ||   ]] > ([[ || e ||e ]], -[[ || e ||e ]], -[[ || e ||e ]] )  $.
\end{corollary}
\begin{proof}
  By \cref{lemma:size_e} and note that $ [[ || pe ||e   ]] > [[  || pe ||s  ]] $ and $ [[ || pe ||e   ]] > [[  || pe ||g  ]] $
\end{proof}


\begin{lemma} \label{lemma:type_decrease}
  $[[ || A || ]] \leq [[ || S(A) || ]]  $.
\end{lemma}
\begin{proof}
  By induction on the structure of $[[A]]$. The interesting cases are $[[ A ]] = [[static]]$ and
  $[[ A ]] = [[gradual]]$. When $[[ A ]] = [[static]]$, $[[ S(A) ]] = [[t]]$
  for some monotype $[[t]]$ and it is immediate that $[[ || static ||  ]]  \leq [[ || t || ]] $
  (note that $[[ || static ||  ]] < [[ || gradual ||  ]] $ by definition).
\end{proof}


\begin{lemma}[Substitution Decreases Measurement]
  \label{lemma:subst_dec_measure}
  If $[[pe1]] \leq [[pe2]]$, then $ {[[ ||pe1|| ]]} \leq [[ ||pe2|| ]]$; unless
  $[[pe2]] \leq [[pe1]]$ also holds, otherwise we have $[[ ||pe1|| ]] < [[ ||pe2|| ]]$.
\end{lemma}
\begin{proof}
  Since $[[ pe1  ]] \leq [[  pe2  ]]$, we know $[[ pe2  ]] = [[ S(pe1)  ]]$ for some $[[S]]$. By induction on
  the structure of $[[pe1]]$.

  \begin{itemize}
  \item Case $[[pe1]] = [[  \x : A . pe ]]$. We have
    $[[ pe2  ]] = [[  \x : S(A) . S(pe)  ]]$. By \cref{lemma:type_decrease} we have $[[ || A || ]] \leq [[ || S(A) || ]]$.
    By i.h., we have $[[ || pe ||  ]] \leq [[ || S(pe) ||  ]]$. Therefore $[[ || \x : A . pe ||    ]] \leq [[ || \x : S(A) . S(pe) ||  ]]$.
  \item Case $[[pe1]] = [[ < A `-> B > pe  ]]$. We have
    $[[pe2]] = [[ < S(A) `-> S(B) > S(pe)  ]]$.  By \cref{lemma:type_decrease} we have $[[ || A || ]] \leq [[ || S(A) || ]]$
    and $[[ || B || ]] \leq [[ || S(B) || ]]$. By i.h., we have $[[ || pe ||  ]] \leq [[ || S(pe) ||  ]]$.
    Therefore $[[  || < A `-> B > pe ||  ]] \leq [[ || < S(A) `-> S(B) > S(pe)  ||   ]]$.

  \item The rest of cases are immediate.
  \end{itemize}

\end{proof}


\begin{lemma}[Representative Translation for Typing]
  For any typing derivation that $[[dd |- e : A]]$, there exists at least one representative
  translation $r$ such that $[[dd |- e : A ~~> rpe]]$.
\end{lemma}
\begin{proof}
We already know that at least one translation $[[pe]] = [[pe1]]$ exists
for every typing derivation. If $[[pe1]]$ is a representative translation then we
are done. Otherwise there exists another translation $[[pe2]]$ such that
$[[pe2]] \leq [[pe1]]$ and $ [[pe1]] \not \leq [[pe2]]$. By
\cref{lemma:subst_dec_measure}, we have $[[||pe2||]] < [[ ||pe1|| ]]$. We continue
with $[[pe]] = [[pe2]]$, and get a strictly decreasing sequence $[[ || pe1 ||  ]], [[ || pe2 || ]], \dots$.
By \cref{lemma:decrease_stop}, we know this sequence cannot be infinite long. Suppose it ends at $[[ || pen || ]]$,
by the construction of the sequence, we know that $[[pen]]$ is a representative translation of $[[e]]$.
\end{proof}


\begin{conjecture}[Property of Representative Translation] \label{lemma:repr}
  If $[[empty |- e : A ~~> pe]]$, $\erasetp s \Downarrow v$, then we
  have $[[empty |- e : A ~~> rpe]]$, and $\erasetp r \Downarrow v'$.
\end{conjecture}

\ningning{shall we focus on values of type integer?}

\begin{definition}[Erasure of Type Parameters]
  \begin{center}
\begin{tabular}{p{5cm}l}
  $\erasetp \nat = \nat $ &
  $\erasetp a = a $ \\
  $\erasetp {A \to B} = \erasetp A \to \erasetp B $ &
  $\erasetp {\forall a. A} = \forall a. \erasetp A$ \\
  $\erasetp {\unknown} = \unknown  $&
  $\erasetp {\static} = \nat  $\\
  $\erasetp {\gradual} = \unknown  $\\
\end{tabular}

  \end{center}
\end{definition}


\begin{corollary}[Coherence up to cast errors]
  Suppose $[[ empty |- e : int ~~> pe1 ]]$ and $[[ empty |- e : int ~~> pe2 ]]$, if $| [[ pe1 ]] | \Downarrow [[n]]$
  then either $ | [[  pe2  ]] | \Downarrow n$ or $ | [[  pe2  ]] | \Downarrow \blamev$.
\end{corollary}
\jeremy{maybe Conjecture~\ref{lemma:repr} is enough to prove it? }


\begin{conjecture}[Dynamic Gradual Guarantee]
  Suppose $e' \lessp e$,
  \begin{enumerate}
  \item If $[[empty |- e : A ~~> rpe]]$, $\erasetp {r} \Downarrow v$,
    then for some $B$ and $r'$, we have $[[ empty |- e' : B ~~> rpe']]$,
    and $B \lessp A$,
    and $\erasetp {r'} \Downarrow v'$,
    and $v' \lessp v$.
  \item If $[[empty |- e' : B ~~> rpe']]$, $\erasetp {r'} \Downarrow v'$,
    then for some $A$ and $[[rpe]]$, we have $ [[empty |- e : A ~~> rpe]]$,
    and $B \lessp A$. Moreover,
    $\erasetp r \Downarrow v$ and $v' \lessp v$,
    or $\erasetp r \Downarrow \blamev$.
  \end{enumerate}
\end{conjecture}



\section{Efficient (Almost) Typed Encodings of ADTs}


\begin{itemize}
\item Scott encodings of simple first-order ADTs (e.g. naturals)
\item Parigot encodings improves Scott encodings with recursive schemes, but
  occupies exponential space, whereas Church encoding only occupies linear
  space.
\item An alternative encoding which retains constant-time destructors but also
  occupies linear space.
\item Parametric ADTs also possible?
\item Typing rules
\end{itemize}

\begin{example}[Scott Encoding of Naturals]
\begin{align*}
  [[nat]] &\triangleq [[  \/a. a -> (unknown -> a) -> a ]] \\
  \mathsf{zero} &\triangleq [[ \x . \f . x  ]] \\
  \mathsf{succ} &\triangleq [[ \y : nat . \x . \f . f y ]]
\end{align*}
\end{example}
Scott encodings give constant-time destructors (e.g., predecessor), but one has to
get recursion somewhere. Since our calculus admits untyped lambda calculus, we
could use a fixed point combinator.

\begin{example}[Parigot Encoding of Naturals]
\begin{align*}
  [[nat]] &\triangleq [[  \/a. a -> (unknown -> a -> a) -> a ]] \\
  \mathsf{zero} &\triangleq [[ \x . \f . x  ]] \\
  \mathsf{succ} &\triangleq [[ \y : nat . \x . \f . f y (y x f) ]]
\end{align*}
\end{example}
Parigot encodings give primitive recursion, apart form constant-time
destructors, but at the cost of exponential space complexity (notice in
$\mathsf{succ}$ there are two occurances of $[[y]]$).

Both Scott and Parigot encodings are typable in System F with positive recursive
types, which is strong normalizing.

\begin{example}[Alternative Encoding of Naturals]
\begin{align*}
  [[nat]] &\triangleq [[  \/a. a -> (unknown -> (unknown -> a) -> a) -> a ]] \\
  \mathsf{zero} &\triangleq [[ \x . \f . x  ]] \\
  \mathsf{succ} &\triangleq [[ \y : nat . \x . \f .  f y (\g . g x f) ]]
\end{align*}
\end{example}
This encoding enjoys constant-time destructors, linear space complexity, and
primitive recursion.
The static version is $[[ mu b . \/ a . a -> (b -> (b -> a) -> a) -> a ]]$,
which can only be expressed in System F with
general recursive types (notice the second $[[b]]$ appears in a negative position).





\section{Algorithmic System}

\begin{center}
\begin{tabular}{lrcl} \toprule
  Expressions & $[[ae]]$ & \syndef & $[[x]] \mid [[n]] \mid [[\x : aA . ae]] \mid [[\x . ae]] \mid [[ae1 ae2]] \mid [[ae : aA]] $ \\
  Existential variables & $[[evar]]$ & \syndef & $[[sa]]  \mid [[ga]]  $   \\
  Types & $[[aA]], [[aB]]$ & \syndef & $ [[int]] \mid [[a]] \mid [[evar]] \mid [[aA -> aB]] \mid [[\/ a. aA]] \mid [[unknown]] \mid [[static]] \mid [[gradual]] $ \\
  Static Types & $[[aT]]$ & \syndef & $ [[int]] \mid [[a]] \mid [[evar]] \mid [[aT1 -> aT2]] \mid [[\/ a. aT]] \mid [[static]] \mid [[gradual]] $ \\
  Monotypes & $[[at]], [[as]]$ & \syndef & $ [[int]] \mid [[a]] \mid [[evar]] \mid [[at -> as]] \mid [[static]] \mid [[gradual]]$ \\
  Castable Monotypes & $[[atc]]$ & \syndef & $ [[int]] \mid [[a]] \mid [[evar]] \mid [[atc1 -> atc2]] \mid [[gradual]]$ \\
  Castable Types & $[[agc]]$ & \syndef & $ [[int]] \mid [[a]] \mid [[evar]] \mid [[agc1 -> agc2]] \mid [[\/ a. agc]] \mid [[unknown]] \mid [[gradual]] $ \\
  Static Castable Types & $[[asc]]$ & \syndef & $ [[int]] \mid [[a]] \mid [[evar]] \mid [[asc1 -> asc2]] \mid [[\/ a. asc]] \mid [[gradual]] $ \\
  Contexts & $[[GG]], [[DD]], [[TT]]$ & \syndef & $[[empty]] \mid [[GG , x : aA]] \mid [[GG , a]] \mid [[GG , evar]] \mid [[GG, evar = at]] $ \\
  Complete Contexts & $[[OO]]$ & \syndef & $[[empty]] \mid [[OO , x : aA]] \mid [[OO , a]] \mid [[OO, evar = at]]$ \\ \bottomrule
\end{tabular}
\end{center}



\begin{definition}[Existential variable contamination] \label{def:contamination}
  \begin{align*}
    [[ [aA] empty    ]] &= [[empty]] \\
    [[ [aA] (GG, x : aA)  ]] &= [[ [aA] GG , x : aA     ]] \\
    [[ [aA] (GG, a)  ]] &= [[ [aA] GG , a     ]] \\
    [[ [aA] (GG, sa)  ]] &= [[ [aA] GG , ga , sa = ga  ]]  \quad \text{if $[[sa in fv(aA)]]$ }    \\
    [[ [aA] (GG, ga)  ]] &= [[ [aA] GG , ga     ]] \\
    [[ [aA] (GG, evar = at)  ]] &= [[ [aA] GG , evar = at     ]] \\
  \end{align*}
\end{definition}



\drules[ad]{$ [[GG |- aA ]] $}{Well-formedness of types}{int, unknown, static, gradual, tvar, evar, solvedEvar, arrow, forall}

\drules[wf]{$ [[ |- GG ]] $}{Well-formedness of algorithmic contexts}{empty, var, tvar, evar, solvedEvar}

\drules[as]{$ [[GG |- aA <~ aB -| DD ]] $}{Algorithmic Consistent Subtyping}{tvar, evar, int, arrow, forallR, forallL, spar, gpar, unknownL, unknownR, instL, instR}

\drules[instl]{$ [[ GG |- evar <~~ aA -| DD   ]] $}{Instantiation I}{solveS, solveG, solveUS, solveUG, reachSGOne, reachSGTwo, reachOtherwise, arr, forallR}

\drules[instr]{$ [[ GG |- aA <~~ evar -| DD   ]] $}{Instantiation II}{solveS, solveG, solveUS, solveUG, reachSGOne, reachSGTwo, reachOtherwise, arr, forallL}

\drules[inf]{$ [[ GG |- ae => aA -| DD ]] $}{Inference}{var, int, lamann, lam, anno, app}

\drules[chk]{$ [[ GG |- ae <= aA -| DD ]] $}{Checking}{lam, gen, sub}

\drules[am]{$ [[ GG |- aA |> aA1 -> aA2 -| DD ]] $}{Algorithmic Matching}{forall, arr, unknown, var}

\drules[ext]{$ [[ GG --> DD  ]] $}{Context extension}{id, var, tvar, evar, solvedEvar, solveS, solveG, add, addSolveS, addSolveG}



\clearpage


\section{Metatheory}

\begin{restatable}[Instantiation Soundness]{mtheorem}{instsoundness} \label{thm:inst_soundness}%
  Given $[[ DD --> OO ]]$ and $[[ [GG]aA = aA ]]$ and  $[[evar notin fv(aA)]]$:

  \begin{enumerate}
  \item If $[[GG |- evar <~~  aA -| DD ]]$ then $[[  [OO]DD |- [OO]evar <~ [OO]aA  ]] $.
  \item If $[[GG |- aA <~~ evar -| DD ]]$ then $[[  [OO]DD |- [OO]aA <~ [OO]evar  ]] $.
  \end{enumerate}
\end{restatable}


\begin{restatable}[Soundness of Algorithmic Consistent Subtyping]{mtheorem}{subsoundness} \label{thm:sub_soundness}%
  If $[[  GG |- aA <~ aB -| DD ]]$ where $[[ [GG]aA = aA  ]]$ and $[[  [GG] aB = aB  ]]$ and $[[  DD --> OO ]]$ then
  $[[  [OO]DD |- [OO]aA <~ [OO]aB   ]]$.
\end{restatable}



\begin{restatable}[Soundness of Algorithmic Typing]{mtheorem}{typingsoundness} \label{thm:type_sound}%
  Given $[[DD --> OO]]$:
  \begin{enumerate}
  \item If $[[  GG |- ae => aA -| DD  ]]$ then $\exists [[e']]$ such that $ [[  [OO]DD |- e' : [OO] aA  ]]   $ and $\erase{[[ae]]} = \erase{[[e']]}$.
  \item If $[[  GG |- ae <= aA -| DD  ]]$ then $\exists [[e']]$ such that $ [[  [OO]DD |- e' : [OO] aA  ]]   $ and $\erase{[[ae]]} = \erase{[[e']]}$.
  \end{enumerate}
\end{restatable}

\begin{restatable}[Instantiation Completeness]{mtheorem}{instcomplete} \label{thm:inst_complete}
  Given $[[GG --> OO]]$ and $[[aA = [GG]aA]]$ and $[[evar]] \notin \textsc{unsolved}([[GG]]) $ and $[[  evar notin fv(aA)  ]]$:
  \begin{enumerate}
  \item If $[[ [OO]GG |- [OO] evar <~ [OO]aA   ]]$ then there are $[[DD]], [[OO']]$ such that $[[OO --> OO']]$
    and $[[DD --> OO']]$ and $[[GG |- evar <~~ aA -| DD]]$.
  \item If $[[ [OO]GG |- [OO]aA  <~ [OO] evar  ]]$ then there are $[[DD]], [[OO']]$ such that $[[OO --> OO']]$
    and $[[DD --> OO']]$ and $[[GG |- aA <~~ evar -| DD]]$.
  \end{enumerate}

\end{restatable}


\begin{restatable}[Generalized Completeness of Consistent Subtyping]{mtheorem}{subcomplete} \label{thm:sub_completeness}
  If $[[ GG --> OO  ]]$ and $[[ GG |- aA  ]]$ and $[[ GG |- aB  ]]$ and $[[  [OO]GG |- [OO]aA <~ [OO]aB  ]]$ then
  there exist $[[DD]]$ and $[[OO']]$ such that $[[DD --> OO']]$ and $[[OO --> OO']]$ and $[[  GG |- [GG]aA <~ [GG]aB -| DD ]]$.
\end{restatable}


\begin{restatable}[Matching Completeness]{mtheorem}{matchcomplete} \label{thm:match_complete}%
  Given $[[ GG --> OO  ]]$ and $[[ GG |- aA  ]]$, if
  $[[ [OO]GG |- [OO]aA |> A1 -> A2  ]]$
  then there exist $[[DD]]$, $[[OO']]$, $[[aA1']]$ and $[[aA2']]$ such that $[[ GG |- [GG]aA |> aA1' -> aA2' -| DD   ]]$
  and $[[ DD --> OO'  ]]$ and $[[ OO --> OO'  ]]$ and $[[A1]] = [[ [OO']aA1'  ]]$ and $[[A2]] = [[ [OO']aA2'  ]]$.
\end{restatable}



\begin{restatable}[Completeness of Algorithmic Typing]{mtheorem}{typingcomplete}
  Given $[[GG --> OO]]$ and $[[GG |- aA]]$, if $[[ [OO]GG |- e : A ]]$ then there exist $[[DD]]$, $[[OO']]$, $[[aA']]$ and $[[ae']]$
  such that $[[DD --> OO']]$ and $[[OO --> OO']]$ and $[[  GG |- ae' => aA' -| DD  ]]$ and $[[A]] = [[ [OO']aA'  ]]$ and $\erase{[[e]]} = \erase{[[ae']]}$.
\end{restatable}



%%%%%%%%%%%%%%%%%%%%%%%%%%%%%%%%%%%%%%%%%%%%%%%%%%%%%%%%%%%%%%%%%%%%%%%%
\section{Establishing Coherence for \fnamee}
\label{sec:coherence:poly}
%%%%%%%%%%%%%%%%%%%%%%%%%%%%%%%%%%%%%%%%%%%%%%%%%%%%%%%%%%%%%%%%%%%%%%%%

In this section, we establish the coherence property for \fnamee. The proof
strategy mostly follows that of \namee, but the construction of the
heterogeneous logical relation is significantly more complicated. Firstly in
\cref{sec:para:intuition} we discuss why adding BCD subtyping to disjoint
polymorphism introduces significant complications. In
\cref{sec:failed:lr}, we discuss why a natural extension of
System F's logical relation to deal with disjoint polymorphism fails. The technical
difficulty is \emph{well-foundedness}, stemming from the interaction between
impredicativity and disjointness. Finally in \cref{sec:succeed:lr}, we present
our (predicative) logical relation that is specially crafted to prove coherence
for \fnamee.
% and allude to a potential solution to lift the predicativity restriction.

\subsection{The Challenge}
\label{sec:para:intuition}

Before we tackle the coherence of \fnamee, let us first consider how \fname
(and its predecessor \oname) enforces coherence. Its essentially syntactic
approach is to make sure that there is at most one subtyping derivation for any
two types. As an immediate consequence, the produced coercions are uniquely determined and thus
the calculus is clearly coherent. Key to this approach is the invariant that
the type system only produces \emph{disjoint} intersection types. As we
mentioned in \cref{sec:typesystem}, this invariant complicates the calculus
and its metatheory, and leads to a weaker substitution lemma.
% To see this, consider the judgment $[[ X ** nat |- X & nat ]]$. 
% Clearly $[[X]]$ cannot be instantiated to an arbitrary type. For
% instance, substituting $[[X]]$ with $[[nat]]$ would lead to an ill-formed
% intersection type $[[nat & nat]]$ in \fname. 
% Therefore in the
% substitution lemma, the range of substituted types is narrowed down to those
% that respect the disjointness constraints.
% The motivation of maintaining this invariant was to enable
% Generally speaking, in \fname all meta-theoretic properties are weakened to
% account for disjointness pre-conditions. All of these contribute
Moreover, the syntactic coherence approach is incompatible with BCD subtyping,
which leads to multiple subtyping derivations with different coercions and
requires a more general substitution lemma. For example, consider the
coercions produced by $[[ \X ** nat . X & X <: \X ** nat & nat . X ]]$ (neither
type is ``well-formed'' in the sense of \fname). Two possible ones are
$[[ \f . \X . pp1 (f X) ]]$ and $[[ \f . \X . pp2 (f X) ]]$. It is not at all
obvious that they should be equivalent in an appropriate sense.
To accommodate BCD into \oname, Bi et al.~\cite{bi_et_al:LIPIcs:2018:9227}
have created the \namee calculus and
developed a semantically-founded proof method based on logical relations.
Because \namee does not feature polymorphism, the problem at hand is to
incorporate support for polymorphism in this semantic approach to coherence,
which turns out to be more challenging than is apparent.

% preclude the possibility of adding BCD
% subtyping, which requires a general substitution lemma. This implies that the
% avenue taken by Alpuim et al.~\cite{alpuimdisjoint} to prove coherence does not
% work for \fnamee anymore. In particular, subtyping does not necessarily produces unique
% coercions. For example, consider the possible coercions generated by $[[ \X ** nat . X & X <: \X ** nat & nat . X ]]$ (neither of which is ``well-formed''
% in the sense of \fname). Two possible coercions are $[[ \f . \X . pp1 (f X) ]]$
% and $[[ \f . \X . pp2 (f X) ]]$. It is not at all obvious that these two
% coercions are equivalent in an appropriate sense. Moreover, the addition of BCD subtyping
% aggravates the matter even more---the subtyping relation can produce additional
% syntactically different coercions that are harder to argue to be equivalent.
% Inspired by Bi et al.~\cite{bi_et_al:LIPIcs:2018:9227}, a new semantically-founded
% proof method is called for. Logical relations \`a la System F might shed some
% light, as we will discuss next.

\begin{figure}[t]
  \centering
  \begin{tabular}{rll}
    $[[(v1 , v2) in V ( nat ; nat ) ]]$  & $\defeq$ & $\exists [[i]].\, [[v1]] = [[v2]] = [[i]]$ \\
    $[[(v1, v2)  in V(T1 -> T2; T1' -> T2') ]]$ &$\defeq$ & $\forall [[(v, v') in V (T1; T1')   ]].\, [[  (v1 v , v2 v') in E (T2 ; T2') ]]$ \\
    $[[( < v1 , v2 > , v3  )  in V ( T1 * T2 ;  T3  )  ]]$  &$\defeq$& $[[ (v1, v3)  in V (T1 ; T3)  ]] \land [[ (v2, v3)  in V (T2 ; T3)  ]]$ \\
    $[[( v3 , < v1 , v2 >  )  in V ( T3 ; T1 * T2  )  ]]$  &$\defeq$& $[[ (v3, v1)  in V (T3 ; T1)  ]] \land [[ (v3, v2)  in V (T3 ; T2)  ]]$
  \end{tabular}
  \caption{Selected cases from \namee's canonicity relation}
  \label{fig:logical:necolus}
\end{figure}

\subsection{Impredicativity and Disjointness at Odds}
\label{sec:failed:lr}

\Cref{fig:logical:necolus} shows selected cases of \emph{canonicity},
which is \namee's (heterogeneous) logical relation used
in the coherence proof. The definition captures that two values
$[[v1]]$ and $[[v2]]$ of types $[[ T1 ]]$ and $[[T2]]$ are in $\valR{[[T1]]}{[[T2]]}$ iff
either the types are disjoint or the types are equal and the values are
semantically equivalent. Because both alternatives entail coherence, 
canonicity is key to \namee's coherence proof.

\paragraph{Well-foundedness issues.}
For \fnamee, we need to extend canonicity with additional cases to
account for universally quantified types. For reasons that will become clear in
\cref{sec:succeed:lr}, the type indices become source types (rather than target types as in \cref{fig:logical:necolus}).
A naive formulation of one case rule is:
{\small
\begin{align*}
    &[[(v1, v2)  in V(\X ** A1 . B1; \X ** A2 . B2) ]] \defeq  \\
    &\qquad \forall [[C1 ** A1]], [[C2 ** A2]].\ [[( v1 | C1 | , v2 | C2 | ) in E ( B1 [X ~> C1]; B2 [X ~> C2]) ]]
\end{align*}
}%
This case is problematic because it destroys the well-foundedness of \namee's
logical relation, which is based on structural induction on the type indices.
Indeed, the type $[[ B1 [X ~> C1] ]]$ may well be larger than $[[ \X ** A1 . B1 ]]$.


% \begin{verbatim}
% Further outline
% - show System F-style case with deferred substitions
% - introduce variable case
% - show well-foundedness problem with variable case (also present in System F)
% - show System F solution for the problem by adding a relation parameter R
% - introduce problem with heterogeneous case
% \end{verbatim}

However, System F's well-known parametricity logical
relation~\cite{reynolds1983types} provides us with a means to avoid this
problem.  Rather than performing the type substitution immediately as in the
above rule, we can defer it to a later point by adding it to an extra parameter
$[[pq]]$ of the relation, which accumulates the deferred substitutions. This yields a modified rule where the type indices in the recursive occurrences are indeed smaller:
{\small
\begin{align*}
  &[[(v1, v2)  in V(\X ** A1 . B1; \X ** A2 . B2) with pq ]]  \defeq  \\
  &\qquad \forall [[C1 ** A1]], [[C2 ** A2]]. ([[v1 | C1 | ]] ,  [[v2 | C2 |]]) \in \eeR{[[B1]]}{{[[B2]]}}_{[[pq]] [ [[X]] \mapsto ([[C1]], [[C2]])]}
\end{align*}
}%
Of course, the deferred substitution has to be performed eventually, to be precise when the type indices are type variables.
\[
    [[(v1, v2)  in V(X ; X) with pq ]] \defeq [[ (v1, v2) in V(pq1 (X); pq2 (X)) with emp  ]]
\]
Unfortunately, this way we have not only moved the type substitution to the type variable case, but also the ill-foundedness problem. Indeed, this problem is also
present in System F. The standard solution is to not fix the relation $[[Rel]]$ by which values
at type $[[X]]$ are related to $\valR{[[pq1 (X)]]}{[[pq2 (X)]]}$, but instead to make it a parameter that is tracked by $[[pq]]$.
This yields the following two rules for disjoint quantification and type variables:
{\small
\begin{align*}
  [[(v1, v2)  in V(\X ** A1 . B1; \X ** A2 . B2) with pq ]] &\defeq \forall [[C1 ** A1]], [[C2 ** A2]], [[Rel]] \subseteq [[C1]] \times [[C2]]. \\
                                                            & ([[v1 | C1 | ]] ,  [[v2 | C2 |]]) \in \eeR{[[B1]]}{{[[B2]]}}_{[[pq]] [ [[X]] \mapsto ([[C1]], [[C2]], [[Rel]])]} \\
    [[(v1, v2)  in V(X; X) with pq ]] & \defeq ([[v1]], [[v2]]) \in [[pq]]_{[[Rel]]}([[X]])
\end{align*}
}%
Now we have finally recovered the well-foundedness of the relation. It is again
structurally inductive on the size of the type indexes.


\paragraph{Heterogeneous issues.}

We have not yet accounted for one major difference between the parametricity relation, from which we have borrowed ideas, and the canonicity relation, to which we have been adding. The former is homogeneous (i.e., the types of the two values is the same) and therefore has one type index, while the latter is heterogeneous (i.e., the two values may have different types) and therefore has two type indices. Thus we must also consider cases like
$\valR{[[X]]}{[[nat]]}$. A definition that seems to handle this case
appropriately is:
{\small
  \begin{align} \label{eq:var}
    [[(v1, v2)  in V(X; nat) with pq ]] \defeq [[ (v1, v2) in V(pq1 (X); nat) with emp  ]]
  \end{align}
}%
Here is an example to motivate this case.
Let  $  [[ee]] = [[\ X ** Top . (\x . x) : X & nat -> X & nat]] $.
We expect that $[[ee nat 1 -->> <1 , 1> ]]$, which
%%\footnote{The reader is advised to try it out in our prototype interpreter.}
boils down to showing $  (1 , 1)   \in \valR{[[X]]}{[[nat]]}_{[ [[X]] \mapsto ([[nat]], [[nat]], [[Rel]])   ]}  $.
According to \cref{eq:var}, this is indeed the case. However, we run into ill-foundedness issue again, because
$[[pq1 (X)]]$ could be larger than $[[X]]$. Alas, this time the parametricity relation has no solution for us.


\subsection{The Canonicity Relation for \fnamee}
\label{sec:succeed:lr}

% \bruno{Perhaps we are still showing too many auxiliary lemmas here? We
% could cut on some of these if we are looking for space.}
In light of the fact that substitution in the logical relation seems unavoidable
in our setting, and that impredicativity is at odds with substitution, we turn
to \emph{predicativity}: we change \rref{T-tapp} to its predicative version:
{\small
\[
  \drule{T-tappMono}
\]
}%
where metavariable $[[t]]$ ranges over monotypes (types minus disjoint quantification).
We do not believe that predicativity is a severe restriction in practice, since many source
languages (e.g., those based on the Hindley-Milner type system~\cite{milner1978theory, hindley1969principal} like Haskell and
OCaml) are themselves predicative and do not require the full generality of an
impredicative core language.

% The restriction to
% predicative polymorphism, though reducing expressiveness in theory, does not seem to cost much
% in practice. Languages based on the Hindley–Milner type
% system~\cite{milner1978theory, hindley1969principal}, such as Haskell and ML,
% have such restriction. We also plan to study a variant of \fnamee with implicit
% polymorphism in the future, where a predicativity restriction is
% likely to be required anyway.

\begin{figure}[t]
  \centering
  \begin{tabular}{rll}
    $[[(v1 , v2) in V ( nat ; nat ) ]]$  & $\defeq$ & $\exists [[i]].\, [[v1]] = [[v2]] = [[i]]$ \\
    $[[(v1, v2) in V ( {l : A}  ; {l : B} ) ]]$ & $\defeq$ & $[[ (v1, v2) in V ( A ; B ) ]]$\\
    $[[(v1 , v2) in V ( A1 -> B1 ; A2 -> B2 ) ]]$  & $\defeq$ & $\forall [[(v2' , v1') in V ( A2 ; A1 ) ]].\, [[ (v1 v1' , v2 v2') in E ( B1 ; B2 ) ]]$ \\
    $[[( < v1 , v2 > , v3  )  in V ( A & B ;  C  ) ]]$  & $\defeq$ & $[[ (v1, v3)  in V (A ; C) ]] \land [[ (v2, v3)  in V (B ; C) ]]$  \\
    $[[( v3 , < v1 , v2 >  )  in V ( C; A & B  ) ]]$  & $\defeq$ & $[[ (v3, v1)  in V (C ; A) ]] \land [[ (v3, v2)  in V (C ; B) ]]$  \\
    $[[(v1, v2)  in V ( \ X ** A1 . B1; \ X ** A2 . B2 ) ]]$  &$\defeq$ & $\forall [[empty |- t ** A1 & A2 ]].\ [[  (v1 |t| , v2 |t|) in E ( B1 [X ~> t] ;  B2 [ X ~> t]) ]]$ \\
  % $[[(v1, v2) in V ( A  ; B ) ]]$ & $\defeq$ & $[[A top]] \, \lor \, [[B top]]    $ \\
    $[[(v1 , v2) in V (A; B)]] $  &$\defeq$ & $\mathsf{true} \quad \text{otherwise} $ \\
    $[[(e1, e2) in E (A; B)]]$ & $\defeq$ & $\exists [[v1]], [[v2]].\, [[e1 -->> v1]] \land [[e2 -->> v2]] \ \land [[(v1, v2) in V (A; B)]]$ \\ \\
  \end{tabular}

  \begin{tabular}{rrll}
    $[[p in  DD]]$ & $\defeq$ &  $\ottaltinferrule{}{}{  }{ [[empp in empty]] }$ &     $\ottaltinferrule{}{}{ [[p in DD]] \\ [[empty |- t ** p(B)]] \\  }{ [[p [ X -> t ] in DD , X ** B]]  }$ \\ \\
    $[[  (g1, g2)  in GG with p ]]$ & $\defeq$ &  $\ottaltinferrule{}{}{  }{ [[(empg, empg) in empty with p ]]  }$ & $\ottaltinferrule{}{}{ [[(g1, g2) in GG with p ]] \\ [[(v1, v2) in V (p(A) ; p(A)) ]] }{ [[(g1 [ x -> v1 ] , g2 [ x -> v2 ]  )  in GG , x : A with p ]] }$
  \end{tabular}
  \caption{The canonicity relation for \fnamee}
  \label{fig:logical:fi}
\end{figure}

Luckily, substitution with monotypes does not prevent well-foundedness.
\Cref{fig:logical:fi} defines the \emph{canonicity} relation for
\fnamee. The canonicity relation is a family of binary relations over \tnamee
values that are \emph{heterogeneous}, i.e., indexed by two \fnamee types. Two
points are worth mentioning. (1) An apparent difference from \namee's logical
relation is that our relation is now indexed by \emph{source types}. The reason is that
the type translation function (\cref{def:type:translate:fi}) discards disjointness
constraints, which are crucial in our setting, whereas \namee's
type translation does not have information loss. (2) Heterogeneity
allows relating values of different types, and in particular values whose types are
disjoint. The rationale behind the canonicity relation is to combine equality
checking from traditional (homogeneous) logical relations with disjointness
checking. It consists of two relations: the value relation $\valR{[[A]]}{[[B]]}$
relates \emph{closed} values; and the expression relation
$\eeR{[[A]]}{[[B]]}$---defined in terms of the value relation---relates closed
expressions.

% \paragraph{Value relation.}

The relation $\valR{[[A]]}{[[B]]}$ is defined by induction on the structures of $[[A]]$ and
$[[B]]$. For integers, it requires the two values to be literally the same. For
two records to behave the same, their fields must behave the same. For two
functions to behave the same, they are required to produce outputs related at
$[[B1]]$ and $[[B2]]$ when given related inputs at $[[A1]]$ and $[[A2]]$. For
the next two cases regarding intersection types, the relation distributes
over intersection constructor $[[&]]$. Of particular interest is the case for
disjoint quantification. Notice that it \emph{does not} quantify over arbitrary
relations, but directly substitutes $[[X]]$ with monotype $[[t]]$ in $[[B1]]$ and
$[[B2]]$. This means that our canonicity relation \emph{does not} entail
parametricity. % , and as such, the free theorem in \cref{sec:failed:lr}
% cannot be proved using the canonicity relation.
However, it suffices for our
purposes to prove coherence. Another noticeable thing is that we keep the
invariant that $[[A]]$ and $[[B]]$ are closed types throughout the relation, so
we no longer need to consider type variables. This simplifies things a lot. % The
% other cases are quite standard.
Note that when one type is $[[Bot]]$, two
values are vacuously related because there simply are no values of type $[[Bot]]$.
% We refer to Bi et al.~\cite{bi_et_al:LIPIcs:2018:9227} for more explanations of
% the canonicity relation.
We need to show that the relation is indeed well-founded:

\begin{restatable}[Well-foundedness]{lemma}{wellfounded}\label{lemma:well-founded}
  The canonicity relation of \fnamee is well-founded.
\end{restatable}
\proof
  Let $| \cdot |_{\forall}$ and $| \cdot |_s$ be the number of
  $\forall$-quantifies and the size of types, respectively. We consider the measure $\langle
  | \cdot |_{\forall} , | \cdot |_s \rangle$,
  where $\langle \dots \rangle$ denotes lexicographic order. For the case of
  disjoint quantification, the number of $\forall$-quantifiers decreases because monotype $[[t]]$ does not contain $\forall$-quantifiers.
  For the other cases, the measure of $| \cdot |_{\forall}$ does not increase, and
  the measure of $| \cdot |_s$ strictly decreases.
\qed

% \begin{lemma}[Symmetry]
%   If $[[ (v1, v2) in V ( A ; B ) ]]$ then $[[ (v2, v1) in V ( B ; A ) ]]$.
% \end{lemma}
% \begin{proof}
%   The proof proceeds by first induction on $ | [[A]] |_{\forall} $, then
%   simultaneous induction on the structures of $[[A]]$ and $[[B]]$.
% \end{proof}

% We give the logical interpretations of type and term contexts ($[[p]]$ is a mapping
% from type variables to monotypes, $[[g]]$ is a mapping from variables to values).

% The canonicity relation is so constructed to contain values of disjoint types:
% We need to first show an auxiliary lemma regarding top-like types:

% \begin{lemma}
%   If $[[  empty ; empty |-  v1 : |A|  ]]$,
%   $[[  empty ; empty |-  v2 : |B|  ]]$ and
%   $[[ A top  ]]$,
%   then $[[   (v1, v2) in V ( A ; B  )  ]]$.
% \end{lemma}
% \begin{proof}
%   By simultaneous induction on $[[t1]]$ and $[[t2]]$.
% \end{proof}

% \begin{lemma}[Disjoint values are related]
%   If $[[DD |- A ** B]]$, $[[ p in DD  ]]$, $[[  empty ; empty |-  v1 : |p (A)|  ]]$ and $[[  empty ; empty |-  v2 : |p (B)|  ]]$
%   then $[[   (v1, v2) in V ( p(A) ; p(B)  )    ]]$.
% \end{lemma}


\subsection{Establishing Coherence}

\paragraph{Logical equivalence.}

The canonicity relation can be lifted to open expressions in the standard way,
i.e., by considering all possible interpretations of free type and term variables.
The logical interpretations of type and term contexts are found in the bottom
half of \cref{fig:logical:fi}.
\begin{definition}[Logical equivalence $\backsimeq_{log}$]
  {\small
  \begin{align*}
    &[[DD ; GG |- e1 == e2 : A ; B]]   \defeq  [[|DD| ; |GG| |- e1 : |A|]] \land [[ |DD | ; |GG| |- e2 : | B | ]] \ \land \\
    &\qquad (\forall [[p]], [[g1]], [[g2]]. \ [[p in DD]] \land [[(g1, g2) in GG with p ]] \Longrightarrow [[(g1 (p1 (e1)), g2 (p2 (e2)))  in E (p(A) ; p(B)) ]])
  \end{align*}
  }%
\end{definition}
For conciseness, we write $[[DD ; GG |- e1 == e2 : A]]$ to mean $[[DD ; GG |- e1 == e2 : A ; A]]$.

\paragraph{Contextual equivalence.}

\begin{figure}[t]
  \centering
\begin{tabular}{llll}\toprule
  \tnamee contexts & $[[cc]]$ & $\Coloneqq$ &  $[[__]] \mid [[\ x . cc]] \mid [[\ X . cc]]  \mid [[ cc T  ]] \mid [[cc e]] \mid [[e cc]] \mid [[< cc , e>]] \mid [[<e , cc>]] \mid [[c cc]] $ \\
  \fnamee contexts & $[[CC]]$ & $\Coloneqq$ &  $[[__]] \mid [[\ x . CC]] \mid [[\ X ** A. CC]] \mid [[ CC A  ]] \mid [[CC ee]] \mid [[ee CC]] \mid [[ CC ,, ee  ]] \mid [[ ee ,, CC  ]] \mid [[ { l = CC}  ]]  \mid [[ CC . l]] \mid [[ CC : A ]] $ \\ \bottomrule
\end{tabular}
  \caption{Expression contexts}
  \label{fig:contexts:fi}
\end{figure}

Following \namee, the notion of coherence is based on \emph{contextual
  equivalence}. The intuition is that two programs are equivalent if we
\emph{cannot} tell them apart in any context. More formally, we introduce
\emph{expression contexts}, whose syntax is shown in \cref{fig:contexts:fi}. Due
to the bidirectional nature of the type system, the typing judgment of $[[C]]$
features 4 different forms (see \cref{appendix:fnamee}),
e.g., $[[CC : (DD; GG => A) ~> (DD'; GG' => A') ~~> cc]]$ reads if $[[DD ; GG |- ee => A]]$
then $[[DD' ; GG' |- CC { ee } => A']]$. The judgment also generates a well-typed \tnamee context $[[cc]]$. The
following two definitions capture the notion of contextual equivalence:

\begin{definition}[Kleene Equality $\backsimeq$]
  Two complete programs, $[[e]]$ and $[[e']]$, are Kleene equal, written
  $\kleq{[[e]]}{[[e']]}$, iff there exists $[[ii]]$ such that $[[e -->> ii]]$ and
  $[[e' -->> ii]]$.
\end{definition}

\begin{definition}[Contextual Equivalence $\backsimeq_{ctx}$] \label{def:cxtx2}
  {\small
  \begin{align*}
    &[[DD ; GG |- ee1 ~= ee2 : A]]  \defeq \forall [[e1]], [[e2]].\  [[DD ; GG |- ee1 => A ~~> e1]] \land [[DD ; GG |- ee2 => A ~~> e2]] \ \land   \\
    &\qquad (\forall [[C]], [[cc]].\ [[CC : (DD; GG => A) ~> (empty ; empty => nat) ~~> cc]] \Longrightarrow \kleq{[[cc{e1}]]}{[[cc{e2}]]})
  \end{align*}
  }%
\end{definition}

% \noindent In other words, for all possible experiments $[[ cc ]]$, the outcome of an
% experiment on $[[e1]]$ is the same as the outcome on $[[e2]]$
% (i.e., $\kleq{[[cc{e1}]]}{[[cc{e2}]]}$).

% \begin{proof}
%   By induction on the derivation of disjointness. The most interesting case is the variable rule:
%   \[
%     \drule{D-tvarL}
%   \]
%   By the definition of $[[p]]$, we know $[[p(X)]]$ is a monotype. If $[[B]]$ is
%   a polytype, then it follows easily from the definition of logical relation. If
%   $[[B]]$ is also a monotype, we know $[[p(X)]]$ and $[[p(A)]]$ are disjoint by
%   definition. Then by \cref{lemma:covariance:disjoint} and $[[A <: B]]$,
%   we have $[[p(X)]]$ and $[[p(B)]]$ are also disjoint. Finally we apply
%   \cref{lemma:disjoint:mono}.
% \end{proof}

% \paragraph{Compatibility.}

% Firstly we need the compatibility lemmas. Most of them are standard and are thus
% omitted. We show only two compatibility lemmas that are specific to our setting:

% \begin{lemma}[Coercion compatibility] \label{lemma:co-compa} % APPLYCOQ=COERCION_COMPAT
%   Suppose that $[[A1 <: A2 ~~> c]]$,
%   \begin{itemize}
%   \item If $[[DD ; GG |- e1 == e2 : A1 ; A0]]$ then $[[DD ; GG |- c e1 == e2 : A2 ; A0]]$.
%   \item If $[[DD ; GG |- e1 == e2 : A0 ; A1]]$ then $[[DD ; GG |- e1 == c e2 : A0 ; A2]]$.
%   \end{itemize}
% \end{lemma}
% % \begin{proof}
% %   By induction on the subtyping derivation.
% % \end{proof}

% \begin{lemma}[Merge compatibility] % APPLYCOQ=MERGE_COMPAT
%   If $[[ DD ;   GG |- e1 == e1' : A ]]$, $[[  DD ; GG |- e2 == e2' : B ]]$ and $[[ DD |- A ** B ]]$,
%   then $[[ DD ;  GG |- < e1, e2 > == <e1', e2'> : A & B ]]$.
% \end{lemma}
% \begin{proof}
%   By the definition of logical relation and \cref{lemma:disjoint}.
% \end{proof}


% \paragraph{Fundamental property.}

% The ``Fundamental Property'' states that any well-typed expression is related to
% itself by the logical relation. In our elaboration setting, we rephrase it so
% that any two \tnamee terms elaborated from the \emph{same} \fnamee expression
% are related To prove it, we require \cref{thm:uniq}.

% \begin{theorem} \label{thm:uniq}
%   If $[[DD ; GG |- ee => A1]]$ and $[[DD ; GG |- ee => A2]]$, then $[[A1]] \equiv_\alpha [[A2]]$.
% \end{theorem}

% \begin{theorem}[Fundamental property] We have that:
%   \begin{itemize}
%   \item If $[[DD; GG |- ee => A ~~> e]]$ and $[[DD; GG |- ee => A ~~> e']]$, then $[[DD; GG |- e == e' : A ]]$.
%   \item If $[[DD ; GG |- ee <= A ~~> e]]$ and $[[DD ; GG |- ee <= A ~~> e']]$, then $[[DD; GG |- e == e' : A ]]$.
%   \end{itemize}
% \end{theorem}


% We show that logical equivalence is preserved by \fnamee contexts:

% \begin{theorem}[Congruence]
%  If $[[CC : (DD ; GG dirflag A) ~> (DD' ; GG' dirflag' A') ~~> cc]]$, $[[DD ; GG |- ee1 dirflag A ~~> e1]]$, $[[DD ; GG |- ee2 dirflag A ~~> e2]]$
%  and $[[DD ; GG |- e1 == e2 : A ]]$, then $[[DD' ; GG' |- cc{e1} == cc{e2} : A']]$.
% \end{theorem}

\paragraph{Coherence.}

For space reasons, we directly show the coherence statement of \fnamee.
We need several technical lemmas such as compatibility lemmas, fundamental property, etc.
The interested reader can refer to our Coq formalization.

\begin{theorem}[Coherence] \label{thm:coherence:fi}
  We have that
  \begin{itemize}
  \item If $[[DD ; GG |- ee => A ]]$ then $[[DD ; GG |- ee ~= ee : A]]$.
  \item If $[[DD ; GG |- ee <= A ]]$ then $[[DD ; GG |- ee ~= ee : A]]$.
  \end{itemize}
\end{theorem}
\noindent That is, coherence is a special case of \cref{def:cxtx2} where
$[[ee1]]$ and $[[ee2]]$ are the same. At first glance, this
appears underwhelming: of course $[[ee]]$ behaves the same as itself! The tricky
part is that, if we expand it according to \cref{def:cxtx2}, it is not $[[ee]]$
itself but all its translations $[[e1]]$ and $[[e2]]$ that behave the same!




% Local Variables:
% org-ref-default-bibliography: "../paper.bib"
% End:

\section{Algorithmic Type System}
\label{sec:algorithm}

\begin{figure}[t]
  \centering
  \begin{small}
\begin{tabular}{lrcl} \toprule
  Expressions & $e$ & \syndef & $x \mid n \mid
                         \blam x A e \mid \erlam x e \mid e~e \mid e : A $ \\
  Types & $A, B$ & \syndef & $ \nat \mid a \mid \genA \mid A \to B \mid \forall a. A \mid \unknown$ \\
  Monotypes & $\tau, \sigma$ & \syndef & $ \nat \mid a \mid \genA \mid \tau \to \sigma$ \\
  Contexts & $\Gamma, \Delta, \Theta$ & \syndef & $\ctxinit \mid \tctx,x: A \mid \tctx, a \mid \tctx, \genA \mid \tctx, \genA = \tau$ \\
  Complete Contexts & $\Omega$ & \syndef & $\ctxinit \mid \Omega,x: A \mid \Omega, a \mid \Omega, \genA = \tau$ \\ \bottomrule
\end{tabular}
  \end{small}
\caption{Syntax of the algorithmic system}
\label{fig:algo-syntax}
\end{figure}


% The declarative type system in \cref{sec:type-system} serves as a good
% specification for how typing should behave. It remains to see whether this
% specification delivers an algorithm. The main challenge lies in the rules \rul{CS-ForallL} in
% \cref{fig:decl:conssub} and rule \rul{M-Forall} in
% \cref{fig:decl-typing}, which both need to guess a monotype.

% \bruno{why are we not highlightinh the differences in gray anymore?}
In this section we give a bidirectional account of the algorithmic type system
that implements the declarative specification. The algorithm is largely inspired
by the algorithmic bidirectional system of \citet{dunfield2013complete}
(henceforth DK system). However our algorithmic system differs from theirs in
three aspects: 1) the addition of the unknown type $\unknown$; 2) the use of the
matching judgment; and 3) the approach of \textit{gradual inference only
  producing static types}~\citep{garcia2015principal}. We then prove that our
algorithm is both sound and complete with respect to the declarative type
system. Full proofs can be found in the appendix.

\paragraph{Algorithmic Contexts.}

The algorithmic context $\Gamma$ is an
\textit{ordered} list containing declarations of type variables $a$ and term
variables $x : A$. Unlike declarative contexts, algorithmic contexts also
contain declarations of existential type variables $\genA$, which can be either
unsolved (written $\genA$) or solved to some monotype (written $\genA = \tau$).
Complete contexts $\Omega$ are those that contain no unsolved existential type
variables. \Cref{fig:algo-syntax} shows the syntax of the algorithmic system.
Apart from expressions in the declarative system, we have annotated expressions
$e : A$.

% \paragraph{Notational convenience}
% Following \citet{dunfield2013complete}, we use contexts as substitutions on
% types. We write $\ctxsubst{\Gamma}{A}$ to mean $\Gamma$ applied as a
% substitution to type $A$. We also use a hole notation, which is useful when
% manipulating contexts by inserting and replacing declarations in the middle. The
% hole notation is used extensively in proving soundness and completeness. For
% example, $\Gamma[\Theta]$ means $\Gamma$ has the form $\Gamma_L, \Theta,
% \Gamma_R$; if we have $\Gamma[\genA] = (\Gamma_L, \genA, \Gamma_R)$, then
% $\Gamma[\genA = \tau] = (\Gamma_L, \genA = \tau, \Gamma_R)$.

% \paragraph{Input and output contexts}
% The algorithmic system, compared with the declarative system, includes similar
% judgment forms, except that we replace the declarative context $\Psi$ with an
% algorithmic context $\Gamma$ (the \textit{input context}), and add an
% \textit{output context} $\Delta$ after a backward turnstile. For example,
% $\Gamma \vdash A \tconssub B \dashv \Delta$ is the judgment form for the
% algorithmic consistent subtyping, and so on. All rules manipulate input and
% output contexts in a way that is consistent with the notion of \textit{context
%   extension}, which is described in \cref{sec:ctxt:extension}.

% We start with the explanation of the algorithmic consistent subtyping as it
% involves manipulating existential type variables explicitly (and solving them if
% possible).

\subsection{Algorithmic Consistent Subtyping and Instantiation}
\label{sec:algo:subtype}

\begin{figure}[t]
  \centering
  \begin{small}
  %   \begin{mathpar}
  % \framebox{$\Gamma \vdash A$} \\
  % \VarWF \and \IntWF \and \UnknownWF \and \FunWF \and \ForallWF \and \EVarWF
  % \and \SolvedEVarWF
  %   \end{mathpar}

\begin{mathpar}
  \framebox{$\Gamma \vdash A \tconssub B \toctxr$} \\
  \ACSTVar \and \ACSExVar \and \ACSInt \quad \ACSUnknownL \quad \ACSUnknownR \and
  \ACSFun \and \ACSForallR \and \ACSForallL \and \AInstantiateL \quad \AInstantiateR
\end{mathpar}
  \end{small}
  \caption{Algorithmic consistent subtyping}
  \label{fig:algo:subtype}
\end{figure}

\Cref{fig:algo:subtype} shows the algorithmic consistent subtyping rules.
The first five rules do not manipulate contexts. % Rules \rul{ACS-TVar} and
% \rul{ACS-Int} do not involve existential variables, so the output context
% remains unchanged. Rule \rul{ACS-ExVar} says that any unsolved existential
% variable is a consistent subtype of itself. The output is still the same as the
% input context as this gives no clue as to what is the solution of that
% existential variable.
% Rules \rul{ACS-UnknownL} and \rul{ACS-UnknownR} are the verbatim
% correspondences of rule \rul{CS-UnknownL} and \rul{CS-UnknownR}.
Rule \rul{ACS-Fun} is a natural extension of its declarative counterpart. The
output context of the first premise is used by the second premise, and the
output context of the second premise is the output context of the conclusion.
Note that we do not simply check $A_2 \tconssub B_2$, but apply $\Theta$
% (the input context of the second premise)
to both types (e.g., $\ctxsubst{\Theta}{A_2} $). This is
to maintain an important invariant that types
% : whenever we try to derive $\Gamma \vdash A \tconssub B \dashv \Delta$, the types $A$ and $B$
are fully applied
under input context $\Gamma$ (they contain no existential variables already solved in
$\Gamma$). The same invariant applies to every algorithmic judgment.
Rule \rul{ACS-ForallR} looks similar to its declarative counterpart, except that
we need to drop the trailing context $a, \Theta$ from the concluding output
context since they become out of scope.
% again, bears a similarity with the declarative
% version. Note that the output context of its premise allows additional elements
% to appear after the type variable $a$, in a trailing context $\Theta$. Since $a$
% becomes out of scope in the conclusion, we need to drop the trailing context
% $\Theta$ together with $a$ from the concluding output context, resulting in
% $\Delta$.
% The next rule is essential to eliminating the guessing work, thus appears
% significantly different from its declarative version. Instead of guessing a
% monotype $\tau$ out of thin air,
Rule \rul{ACS-ForallL} generates a fresh
existential variable $\genA$, and replaces $a$ with $\genA$ in the body $A$. The
new existential variable $\genA$ is then added to the premise's input context.
% Unlike rule \rul{ACS-ForallR}, the output context $\Delta$ of the premise
% remains unchanged in the conclusion.
% A central idea behind this rule is that we
% defer the decision of choosing a monotype for a type variable, and hope that it
% could be solved later when we have more information at hand.
As a side note, when both types are quantifiers, then either \rul{ACS-ForallR}
or \rul{ACS-ForallR} could be tried. In practice, one can apply
\rul{ACS-ForallR} eagerly.
The last two rules % are specific to the algorithm, thus having no counterparts in
% the declarative version. They
together check consistent subtyping with an
unsolved existential variable on one side and an arbitrary type on the other
side by the help of the instantiation judgment. % Apart from checking that the existential variable does not occur in the
% type $A$, both of the rules do not directly solve the existential variables, but
% leave the real work to the instantiation judgment.

% \subsection{Instantiation}
% \label{sec:algo:instantiate}

\begin{figure}[t]
  \centering
  \begin{small}
\begin{mathpar}
  \framebox{$\tctx \vdash \genA \unif A \toctxr$} \\
  % {\quad \text{Under input context $\Gamma$, instantiate $\genA$ such that
  %     $\genA \tconssub A$, with output context $\Delta$ }} \\
  \InstLSolve \and \InstLReach \and \InstLSolveU   \and \InstLAllR \and \InstLArr
\end{mathpar}

% \begin{mathpar}
%   \framebox{$\tctx \vdash A \unif \genA  \toctxr$} \\
%   % {\quad \text{Under input context $\Gamma$, instantiate $\genA$ such that
%   %     $A \tconssub \genA$, with output context $\Delta$}} \\
%   \InstRSolve \and \InstRReach \and \InstRSolveU  \and \InstRAllL \and \InstRArr
% \end{mathpar}

  \end{small}
  \caption{Algorithmic instantiation}
  \label{fig:algo:instantiate}
\end{figure}

% A central idea of the algorithmic system is to defer the decision of picking a
% monotype to as late as possible.
The judgment $\Gamma \vdash \genA \unif A \dashv \Delta$ defined in
\cref{fig:algo:instantiate} instantiates unsolved existential variables.
Judgment $\genA \unif A$ reads ``instantiate $\genA$ to a consistent subtype of
$A$''. For space reasons, we omit its symmetric judgement $\Gamma \vdash A \unif
\genA \dashv \Delta$.
% Since these two are mutually defined, we
% discuss them together, and omit symmetric rules when convenient.
Rule \rul{InstLSolve} and rule \rul{InstLReach} set $\genA$ to
$\tau$ and $\genB$ in the output context, respectively.
% is the simplest
% one -- when an existential variable meets a monotype. In that case, we simply
% set the solution of $\genA$ to the monotype $\tau$ in the output context. We
% also need to check that the monotype $\tau$ is well-formed under the prefix
% context $\Gamma$.
Rule \rul{InstLSolveU} is similar to \rul{ACS-UnknownR} in that we put no
constraint on $\genA$ when it meets the unknown type $\unknown$. This design
decision reflects the point that type inference only produces static
types~\citep{garcia2015principal}. We will get back to this point in
\cref{subsec:algo:discuss}.
% Rule \rul{InstLReach} deals with the situation where two existential variables
% meet. Note that $\Gamma[\genA][\genB]$ denotes a context where some unsolved existential
% variable $\genA$ is declared before $\genB$. In this situation, the only logical
% thing we can do is to set the solution of one existential variable to the other
% one, depending on which is declared before which. For example, in the output
% context of rule \rul{InstLReach}, we have $\genB = \genA$ because in the input
% context, $\genA$ is declared before $\genB$.
Rule \rul{InstLAllR} is the instantiation version of rule \rul{ACS-ForallR}.
% Since our system is predicative, $\genA$ cannot be instantiated to $\forall b.
% B$, but we can decompose $\forall b. B$ in the same way as in \rul{ACS-ForallR}.
% Rule \rul{InstRAllL} is the instantiation version of rule \rul{ACS-ForallL}.
The last rule \rul{InstLArr} applies when $\genA$ meets a function type. It
follows that the solution must also be a function type.
% looks a bit complicated, but it is actually very
% intuitive: what does the solution of $\genA$ look like when $A$ is a function
% type? The solution must also be a function type!
That is why, in the first premise, we generate two fresh existential variables
$\genA_1$ and $\genA_2$, and insert them just before $\genA$ in the input
context, so that the solution of $\genA$ can mention them. Note that $A_1 \unif
\genA_1$ switches to the other instantiation judgment.


% \paragraph{Example}

% We show a derivation of $\Gamma[\genA] \vdash \forall b. b \to \unknown \unif
% \genA$ to demonstrate the interplay between instantiation, quantifiers and the
% unknown type:
% \[
%   \inferrule*[right=InstRAllL]
%       {
%         \inferrule*[right=InstRArr]
%         {
%           \inferrule*[right=InstLReach]{ }{\Gamma', \genB \vdash \genA_1 \unif \genB \dashv \Gamma' , \genB = \genA_1} \\
%           \inferrule*[right=InstRSolveU]{ }{\Gamma', \genB = \genA_1 \vdash \unknown \unif \genA_2 \dashv \Gamma', \genB = \genA_1}
%         }
%         {
%           \Gamma[\genA], \genB \vdash \genB \to \unknown \unif \genA \dashv \Gamma', \genB = \genA_1
%         }
%       }
%       {
%         \Gamma[\genA] \vdash \forall b. b \to \unknown \unif \genA \dashv \Gamma', \genB = \genA_1
%       }
% \]
% where $\Gamma' = \Gamma[\genA_2, \genA_1, \genA = \genA_1 \to \genA_2]$. Note
% that in the output context, $\genA$ is solved to $\genA_1 \to \genA_2$, and
% $\genA_2$ remains unsolved because the unknown type $\unknown$ puts no
% constraint on it. Essentially this means that the solution of $\genA$ can be any
% function, which is intuitively correct since $\forall b. b \to \unknown$ can be
% interpreted, from the parametricity point of view, as any function.

\subsection{Algorithmic Typing}
\label{sec:algo:typing}

\begin{figure}[t]
  \centering
  \begin{small}
\begin{mathpar}
  \framebox{$\Gamma \vdash e \Rightarrow A \toctxr $} \\
  % {\quad \text{Under input context $\Gamma$, $e$ synthesizes output type $A$,
  %     with output context $\Delta$}} \\
  \AVar \and \ANat \and \ALamU \and \ALamAnnA \and \AAnno \and \AApp
\end{mathpar}
\begin{mathpar}
  \framebox{$\Gamma \vdash e \Leftarrow A \toctxr $} \\
  % {\quad \text{Under input context $\Gamma$, $e$ synthesizes output type $A$,
  %     with output context $\Delta$}} \\
  \ALam \and \AGen \and \ASub
\end{mathpar}
\begin{mathpar}
  \framebox{$\Gamma \vdash A \match A_1 \to A_2 \toctxr$} \\
  % {\quad \text{Under input context $\Gamma$, $A$ synthesizes output type $A_1
  %     \to A_2$, with output context $\Delta$}} \\
  \AMMC \quad \AMMA \and \AMMB \and \AMMD
\end{mathpar}
  \end{small}
  \caption{Algorithmic typing}
  \label{fig:algo:typing}
\end{figure}

We now turn to the algorithmic typing rules in \cref{fig:algo:typing}. The
algorithmic system uses bidirectional type checking to accommodate polymorphism.
Most of them are quite standard.
% All of them are direct analogies of their declarative counterparts. Rules \rul{AVar}
% and \rul{ANat} do not generate any new information, thus the output context is
% the same as the input context. Rule \rul{ALamAnnA} infers the type of a lambda
% abstraction. It does so by pushing $x : A$ into the input context and continues
% to infer the type of the body $B$. The output context in the premise has
% additional declarations in the trailing context $\Theta$, which is discarded in
% the concluding output context.
Perhaps rule \rul{AApp} (which differs significantly from that in the DK system)
deserves attention. It relies on the algorithmic matching judgment $\Gamma
\vdash A \match A_1 \to A_2 \dashv \Delta$.
% The matching judgment
% algorithmically synthesizes a function type from an arbitrary type.
Rule
\rul{AM-ForallL} replaces $a$ with a fresh existential variable $\genA$, thus
eliminating guessing. Rule \rul{AM-Arr} and \rul{AM-Unknown} correspond
directly to the declarative rules.
% self-explanatory. Rule
% \rul{AM-Unknown} says that the unknown type $\unknown$ can be split into a
% function type $\unknown \to \unknown$.
Rule \rul{AM-Var}, which has no
corresponding declarative version, is similar to \rul{InstRArr}/\rul{InstLArr}:
we create $\genA$ and $\genB$ and add $\genC = \genA \to \genB$ to the context.

% Back to \rul{AApp}. This rule first infers the type of $e_1$, producing a output
% context $\Theta_1$. Then it applies $\Theta_1$ to $A$ and goes into the matching
% judgment, which delivers a function type $A_1 \to A_2$ and another output
% context $\Theta_2$. $\Theta_2$ is used as the input context when inferring the
% type of $e_2$. The last premise algorithmically checks if
% $\ctxsubst{\Theta_3}{A_3}$ is a consistent subtype of
% $\ctxsubst{\Theta_3}{A_1}$. $A_2$ and $\Delta$ are the concluding output type
% and the concluding output context, respectively.


% \section{Soundness and Completeness}
% \label{sec:sound:complete}

% To be confident that our algorithmic type system and the declarative type system
% accept exactly the same programs, we need to prove that the algorithmic rules
% are sound and complete with respect to the declarative specifications. Before we
% give the formal statements of the soundness and completeness theorems, we need a
% meta-theoretical device, called \textit{context extension}~\cite{dunfield2013complete}, to help capture a notion of
% information increase from input contexts to output contexts.

% \subsection{Context Extension}
% \label{sec:ctxt:extension}


% A context extension judgment $\Gamma \exto \Delta$ reads ``$\Gamma$ is extended
% by $\Delta$''. Intuitively, this judgment says that $\Delta$ has at least as
% much information as $\Gamma$: some unsolved existential variables in $\Gamma$
% may be solved in $\Delta$. (The full inductive definition can be found in the
% supplementary material. We refer the reader to \citet[][Section
% 4]{dunfield2013complete} for further explanations of context extension.)

\subsection{Completeness and Soundness}

We prove that the algorithmic rules are sound and complete with
respect to the declarative specifications. We need an auxiliary judgment
$\Gamma \exto \Delta$ that captures a notion of information increase from input
contexts $\Gamma$ to output contexts $\Delta$~\citep{dunfield2013complete}.

\paragraph{Soundness.} Roughly speaking, soundness of the algorithmic system says
that given an expression $e$ that type checks in the algorithmic system, there exists
a corresponding expression $e'$ that type checks in the declarative system.
However there is one complication: $e$ does not necessarily have more annotations
than $e'$. For example, by \rul{ALam} we have $\erlam{x}{x} \chkby (\forall a.
a) \rightarrow (\forall a . a)$, but $\erlam{x}{x}$ itself cannot have type
$(\forall a. a) \rightarrow (\forall a . a)$ in the declarative system. To
circumvent that, we add an annotation to the lambda abstraction, resulting in
$\blam{x}{(\forall a . a)}{x}$, which is typeable in the declarative system with
the same type. To relate $\erlam{x}{x}$ and $\blam{x}{(\forall a . a)}{x}$, we
erase all annotations on both expressions. The definition of erasure $\erase{\cdot}$ is
standard and thus omitted.

% \jeremy{mention erasure and why (talk about \rul{ALam} and \rul{ASub})}


% \begin{restatable}[Instantiation Soundness]{mtheorem}{instsoundness} \label{thm:inst_soundness}%
%   Given $\Delta \exto \Omega$ and $\ctxsubst{\Gamma}{A} = A$ and $\genA \notin \mathit{fv}(A)$:
%   \begin{itemize}
%   \item If $\Gamma \vdash \genA \unif A \dashv \Delta$ then $\ctxsubst{\Omega}{\Delta} \vdash \ctxsubst{\Omega}{\genA} \tconssub \ctxsubst{\Omega}{A}$.
%   \item If $\Gamma \vdash A \unif \genA \dashv \Delta$ then $\ctxsubst{\Omega}{\Delta} \vdash \ctxsubst{\Omega}{A} \tconssub \ctxsubst{\Omega}{\genA}$.
%   \end{itemize}
% \end{restatable}

% Notice that the declarative judgment uses $\ctxsubst{\Omega}{\Delta}$, a
% operation that applies a complete context $\Omega$ to the algorithmic context
% $\Delta$, essentially plugging in all known solutions and removing all
% declarations of existential variables (both solved and unsolved), resulting in a
% declarative context.

% With instantiation soundness, next we show that the algorithmic consistent
% subtyping is sound:

% \begin{restatable}[Soundness of Algorithmic Consistent Subtyping]{mtheorem}{subsoudness} \label{thm:sub_soundness}%
%   If $\Gamma \vdash A \tconssub B \toctxr$ where $\ctxsubst{\tctx}{A} = A$ and
%   $\ctxsubst{\tctx}{B} = B$ and $\ctxr \exto \cctx$ then
%   $\ctxsubst{\cctx}{\Delta} \vdash \ctxsubst{\cctx}{A} \tconssub
%   \ctxsubst{\cctx}{B}$.
% \end{restatable}

% At this point, we are ``two thirds of the way'' to proving the ultimate theorem.
% The remaining third concerns with the soundness of matching:

% \begin{restatable}[Matching Soundness]{mtheorem}{matchsoundness}  \label{thm:match_soundness}%
%   If $\Gamma \vdash A \match A_1 \to A_2 \dashv \Delta$ where
%   $\ctxsubst{\Gamma}{A} = A$ and $\Delta \exto \Omega$ then
%   $\ctxsubst{\Omega}{\Delta} \vdash \ctxsubst{\Omega}{A} \match
%   \ctxsubst{\Omega}{A_1} \to \ctxsubst{\Omega}{A_2}$.
% \end{restatable}


% Finally the soundness theorem of algorithmic typing is:

\begin{restatable}[Soundness of Algorithmic Typing]{mtheorem}{typingsoundness} \label{thm:type_sound}
  Given $\ctxr \exto \cctx$,

  \begin{enumerate}
  \item If $\Gamma \vdash e \infto A \toctxr$ then $\exists e'$ such
    that $\ctxsubst{\cctx}{\Delta} \vdash e' : \ctxsubst{\cctx}{A}$ and
    $\erase{e} = \erase{e'}$.
  \item If $\Gamma \vdash e \chkby A \toctxr$ then $\exists e'$ such
    that $\ctxsubst{\cctx}{\Delta} \vdash e' : \ctxsubst{\cctx}{A}$ and
    $\erase{e} = \erase{e'}$.
  \end{enumerate}


\end{restatable}


\paragraph{Completeness.}
Completeness of the algorithmic system is the reverse of soundness: given a
declarative judgment of the form $\ctxsubst{\Omega}{\Gamma} \vdash
\ctxsubst{\Omega} \dots $, we want to get an algorithmic derivation of $\Gamma
\vdash \dots \dashv \Delta$. It turns out that completeness is a bit trickier to
state in that the algorithmic rules generate existential variables on the fly,
so $\Delta$ could contain unsolved existential variables that are not found in
$\Gamma$, nor in $\Omega$. Therefore the completeness proof must produce another
complete context $\Omega'$ that extends both the output context $\Delta$, and
the given complete context $\Omega$. As with soundness, we need erasure to
relate both expressions.

% \jeremy{talk about \rul{Gen}}

% \begin{restatable}[Instantiation Completeness]{mtheorem}{instcomplete}  \label{thm:inst_complete}%
%   Given $\Gamma \exto \Omega$ and $A = \ctxsubst{\Gamma}{A}$ and $\genA \in
%   \mathit{unsolved}(\Gamma)$ and $\genA \notin \mathit{fv}(A)$:
%   \begin{enumerate}
%   \item If $\ctxsubst{\Omega}{\Gamma} \vdash \ctxsubst{\Omega}{\genA} \tconssub
%     \ctxsubst{\Omega}{A}$ then there exist $\Delta$, $\Omega'$ such that $\Omega \exto
%     \Omega'$ and $\Delta \exto \Omega'$ and $\Gamma \vdash \genA \unif A \dashv \Delta$.
%   \item If $\ctxsubst{\Omega}{\Gamma} \vdash \ctxsubst{\Omega}{A} \tconssub
%     \ctxsubst{\Omega}{\genA}$ then there exist $\Delta$, $\Omega'$ such that $\Omega \exto
%     \Omega'$ and $\Delta \exto \Omega'$ and $\Gamma \vdash A \unif \genA \dashv \Delta$.
%   \end{enumerate}
% \end{restatable}


% Next is the completeness of consistent subtyping:

% \begin{restatable}[Generalized Completeness of Subtyping]{mtheorem}{subcomplete}  \label{thm:sub_completeness}%
%   If $\Gamma \exto \Omega$ and $\Gamma \vdash A$ and $\Gamma \vdash B$ and
%   $\ctxsubst{\Omega}{\Gamma} \vdash \ctxsubst{\Omega}{A} \tconssub
%   \ctxsubst{\Omega}{B}$ then there exist $\Delta$, $\Omega'$ such that $\Delta
%   \exto \Omega'$ and $\Omega \exto \Omega'$ and $\Gamma \vdash
%   \ctxsubst{\Gamma}{A} \tconssub \ctxsubst{\Gamma}{B \dashv \Delta}$.
% \end{restatable}


% We prove that the algorithmic matching is complete with respect to the
% declarative matching:

% \begin{restatable}[Matching Completeness]{mtheorem}{matchcomplete} \label{thm:match_complete}%
%   Given $\Gamma \exto \Omega$ and $\Gamma \vdash A$, if
%   $\ctxsubst{\Omega}{\Gamma} \vdash \ctxsubst{\Omega}{A} \match A_1 \to A_2$
%   then there exist $\Delta$, $\Omega'$, $A_1'$ and $A_2'$ such that $\Gamma
%   \vdash \ctxsubst{\Gamma}{A} \match A_1' \to A_2' \dashv \Delta$ and $\Delta \exto \Omega'$ and
%   $\Omega \exto \Omega'$ and $A_1 = \ctxsubst{\Omega'}{A_1'}$ and $A_2 =
%   \ctxsubst{\Omega'}{A_2'}$.
% \end{restatable}


% Finally here is the completeness theorem of the algorithmic typing:

\begin{restatable}[Completeness of Algorithmic Typing]{mtheorem}{typingcomplete}  \label{thm:type_complete}
  Given $\Gamma \exto \Omega$ and $\Gamma \vdash A $, if
  $\ctxsubst{\Omega}{\Gamma} \vdash e : A$ then there exist $\Delta$,
  $\Omega'$, $A'$ and $e'$ such that $\Delta \exto \Omega'$ and $\Omega \exto \Omega'$
  and $\Gamma \vdash e' \infto A' \dashv \Delta$ and $A = \ctxsubst{\Omega'}{A'}$ and $\erase{e} = \erase{e'}$.
\end{restatable}





%%% Local Variables:
%%% mode: latex
%%% TeX-master: "../paper"
%%% org-ref-default-bibliography: "../paper.bib"
%%% End:

% 
\section{Discussion}
\label{sec:discussion}

In this section we consider a simple extension to the language to further
demonstrate the applicability of our definition of consistent subtyping. We also
discuss the design decisions involved in the algorithmic system and show how it
helps address the issue arising from the ambiguity of the type-directed
translation.

\subsection{Extension with Top}
\label{subsec:extension-top}

We argued that our definition of consistent subtyping (\Cref{def:decl-conssub})
is a \textit{general} definition in that it is independent of language features.
We have shown its applicability to polymorphic types, for which neither
\citet{siek2007gradual} nor the AGT approach~\citep{garcia2016abstracting} can
be extended naturally. To strengthen our argument, we consider extending the
language with the $\tope$ type and show that our approach naturally embraces the
extension, with all the desired properties preserved. To aid comparison, we also
show how to adapt the AGT approach to support $\tope$ and verify that these two
approaches, though rooted in different foundations, coincide again on
\textit{simple types}. However, \Cref{def:old-decl-conssub} of
\citet{siek2007gradual} fails to support $\tope$.


\paragraph{Extending Definitions}

In order to preserve the orthogonality between subtyping and consistency, we
require $\top$ to be a common supertype of all static types, as shown in rule
\rul{S-Top}. This rule might seem strange at first glance, since even
if we remove
the requirement $A~static$, the rule seems reasonable.
However, the important point is that because of the orthogonality between
subtyping and consistency, subtyping itself should not contain a potential cast
in principle! Therefore, subtyping instances such as $\unknown \tsub \top$ are not allowed.
For consistency, we add the rule that $\top$ is consistent with $\top$, which is
actually included in the original reflexive rule $A \sim A$. For consistent
subtyping, every type is a consistent subtype of $\top$, for example, $\nat \to
\unknown \tconssub \top$.
\begin{mathpar}
  \SubTop \and \CTop \and \CSTop
\end{mathpar}
It is easy to verify that \Cref{def:decl-conssub} is still equivalent to that in
\Cref{fig:decl:conssub} extended with rule \rul{CS-Top}. That is,
\Cref{lemma:properties-conssub} holds:
\begin{mprop}[Extension with $\top$]
  The following are equivalent:
  \begin{itemize}
  \item  $\tpreconssub A \tconssub B$.
  \item  $\tpresub A \tsub C$, $C \sim D$, $\tpresub D \tsub B$, for some $C, D$.
  \end{itemize}
\end{mprop}
% \begin{proof}\leavevmode
%   \begin{itemize}
%   \item From first to second: By induction on the derivation of consistent
%     subtyping. We have extra case \rul{CS-Top} now, where $B = \top$.
%     We can choose $C = A$, and
%     $D$ by replacing the unknown types in $C$ by $\nat$. Namely, $D$ is a static
%     type, so by \rul{S-Top} we are done.
%   \item From second to first: By induction on the derivation of second
%     subtyping. We have extra case \rul{S-Top} now, where
%     $B = \top$, so $A \tconssub B$ holds by \rul{CS-Top}.
%   \end{itemize}
% \end{proof}

\paragraph{Extending AGT}

We now extend the definition of concretization (\Cref{def:concret}) with $\top$
by adding another equation:
\[
  \gamma(\top) = \{\top\}
\]
It is easy to verify that \Cref{lemma:coincide-agt} still holds:
\begin{mprop}[Equivalent to AGT Extended with $\top$ on Simple Types]
  \label{prop:agt-top}
  $A \tconssub B$ if only if $A \agtconssub B$.
\end{mprop}

% \begin{proof}\leavevmode
%   \begin{itemize}
%   \item From left to right: By induction on the derivation of consistent
%     subtyping. We have case \rul{CS-Top} now.
%     It follows that for
%     every static type $A_1 \in \gamma(A)$, we can derive $A_1 \tsub \top$ by
%     \rul{S-Top}.
%     We have $B_1 = B = \top$ and we are done.
%   \item From right to left: By induction on the derivation of subtyping and
%     inversion on the concretization. We have extra case \rul{S-Top} now, where
%     $B$ is $\top$. So
%     consistent subtyping directly holds.
%   \end{itemize}
% \end{proof}

\paragraph{\citeauthor{siek2007gradual}'s definition of consistent subtyping does not work for $\top$}

Similarly to the analysis in \Cref{subsec:towards-conssub}, $\nat \to \unknown
\tconssub \top$ only holds when we first apply consistency, then subtyping, as
shown in the following diagram. However we cannot find a type $A$ such that
$\nat \to \unknown \tsub A$ and $A \sim \top$. Also we have a similar problem in
extending the restriction operator: \textit{non-structural} masking between
$\nat \to \unknown$ and $\top$ cannot be easily achieved.
\begin{center}
  \begin{tikzpicture}
    \matrix (m) [matrix of math nodes,row sep=3em,column sep=4em,minimum width=2em]
    {
      \bot & \top \\
      \nat \to \unknown &
      \nat \to \nat \\};

    \path[-stealth]
    (m-2-1) edge node [left] {$\tsub$} (m-1-1)
    (m-2-2) edge node [left] {$\tsub$} (m-1-2);

    \draw
    (m-1-1) edge node [above] {$\sim$} (m-1-2)
    (m-2-1) edge node [below] {$\sim$} (m-2-2);
  \end{tikzpicture}
\end{center}


\subsection{Better to be Unknown}
\label{subsec:algo:discuss}

In \Cref{sec:type:trans} we have seen an example where a source expression could
produce two different target expressions with different runtime behaviour. As we
explained, this is due to the guessing nature of the declarative system, and
from the typing point of view, no type is particularly better than others.
However in practice, this is not desirable. Let us revisit the same example, now
from the algorithmic point of view (we omit the translation for space reasons,
the interested reader can try to write down the full derivation):
\[
  f: \forall a. a \to a \byinf (\blam x \unknown {f ~ x}) \infto \unknown \to \genA \dashv f : \forall a. a \to a, \genA
\]
Compared with declarative typing, which can produce as many types as
possible ($\unknown \to \nat$, $\unknown \to \bool$, and so on), the algorithm
computes the type $\unknown \to \genA$ with $\genA$ unsolved in the output
context. This is due to rule \rul{ACS-UnknownL}. What can we know from the
output context? The only thing we know is that $\genA$ is not constrained at
all! As we discussed, any monotype for $\genA$ is inappropriate. Instead, we
replace $\genA$ with the unknown type $\unknown$, which helps to avoid unnecessary
down-casts at runtime (any cast to $\unknown$ is safe), resulting in the final
type $\unknown \to \unknown$.


\paragraph{Do soundness and completeness still hold?}

The reader may ask if the declarative system produces types such as $\unknown
\to \nat$ and $\unknown \to \bool$, but the algorithmic system computes the type
$\unknown \to \unknown$, does it imply that the algorithmic system is no longer
sound and complete with respect to the declarative system? The answer is no.
First of all, note that \Cref{thm:type_complete} reads ``$\dots$ there exist
$\Delta$, $\Omega'$ and $A'$ such that $\dots$ $A = \ctxsubst{\Omega'}{A'}$''.
Now if $A'$ (which is produced by the algorithmic system) contains some unsolved
existential variables, it is up to us to pick solutions in $\Omega'$ to match up
with $A$ (which is produced by the declarative system). More concretely, let us
assume that the declarative system produces type $\unknown \to \nat$, and let
$\Omega = f : \forall a. a \to a$ and $\Delta = f : \forall a. a \to a, \genA$,
the completeness theorem asks if we can find a complete context $\Omega'$ that
extends both $\Delta$ and $\Omega$ such that $\ctxsubst{\Omega'}{(\unknown \to
  \genA)} = \unknown \to \nat$. It is obvious that $\Omega' = f : \forall a. a
\to a, \genA = \nat$ is one such complete context, and we have
$\ctxsubst{\Omega'}{(\unknown \to \genA)} = \unknown \to \nat$. So the
algorithmic system is still complete. Similar arguments apply to the soundness
theorem. 
Secondly, an observation (which follows from the soundness and
completeness theorems) is: if an
expression is typeable with many types in the declarative system, the same expression must be
typeable with a single type that contains some unsolved existential variables in the algorithmic
algorithmic system.
In that case, replacing them with $\unknown$ is the best
strategy for the sake of execution.

\paragraph{Existential variables do not indicate parametricity}

Another reading of the above example may suggest that the result type $\unknown
\to \genA$ implies parametricity, with the implication that $\genA$ can be
changed arbitrarily without affecting the runtime behaviour of the
program. A similar phenomenon is discussed by \citet{siek2008gradual}, where they
argue that we cannot simply ``ignore dynamic types during unification''. In
their view, a type signature with a type variable indicates parametricity, but
this type does not. We agree that type variables do indicate parametricity, but
\textit{existential variables} do not! An \textit{unsolved} existential variable
indicates that a value's only constraint is that it may be cast to and from
$\unknown$, thus may introduce runtime casts. A similar observation is also
found in \citet{garcia2015principal}, where they have to distinguish between
\textit{static polymorphism} and \textit{gradual polymorphism}, and in addition
to gradual type parameters, they have to introduce the so-called \textit{static
  type parameters}. We argue that our language design is much simpler, and our
algorithmic system naturally embraces this distinction.

Now the question is can we really do that? The syntax in \Cref{fig:algo-syntax}
specifies that the solution of a existential variable can only be a monotype.
This is true if the existential variable has a solution. Reading of the output
context reveals that $\genA$ does not have any solution at all. What is more,
our target language (i.e., \pbc) has a nice property, the so-called
``Jack-of-All-Trade Principle''~\cite{ahmed2011blame} that says if instantiating
a type parameter to any given type yields an answer then instantiating that type
parameter to $\unknown$ yields the same answer. In light of these, no type is
more suitable than $\unknown$.

We need to note that this does not mean we should always instantiate a type
parameter to $\unknown$, as is the case in \pbc. One of our design principles is
that we should extract as much information as possible from the static aspects of
the type system, until there is nothing more we can know, then leave the job to
the runtime checks.



%%% Local Variables:
%%% mode: latex
%%% TeX-master: "../paper"
%%% org-ref-default-bibliography: "../paper.bib"
%%% End:


\section{Related Work}
\label{sec:related}

Along the way we discussed some of the most relevant work to motivate,
compare and
promote our gradual typing design. In what follows, we briefly discuss related
work on gradual typing and polymorphism.


\paragraph{Gradual Typing}

The seminal paper by \citet{siek2006gradual} is the first to propose gradual
typing, which enables programmers to mix static and dynamic typing in a program
by providing a mechanism to control which parts of a program are statically
checked. The original proposal extends the simply typed lambda calculus by
introducing the unknown type $\unknown$ and replacing type equality with type
consistency. Casts are introduced to mediate between statically and dynamically
typed code. Later \citet{siek2007gradual} incorporated gradual typing into a
simple object oriented language, and showed that subtyping and consistency are
orthogonal -- an insight that partly inspired our work. We show that subtyping
and consistency are orthogonal in a much richer type system with higher-rank
polymorphism. \citet{siek2009exploring} explores the design space of different
dynamic semantics for simply typed lambda calculus with casts and unknown types.
In the light of the ever-growing popularity of gradual typing, and its somewhat
murky theoretical foundations, \citet{siek2015refined} felt the urge to have a
complete formal characterization of what it means to be gradually typed. They
proposed a set of criteria that provides important guidelines for designers of
gradually typed languages. \citet{cimini2016gradualizer} introduced the
\emph{Gradualizer}, a general methodology for generating gradual type systems
from static type systems. Later they also develop an algorithm to generate
dynamic semantics~\cite{CiminiPOPL}. \citet{garcia2016abstracting} introduced
the AGT approach based on abstract interpretation. As we discussed, none of
these approaches instructed us how to define consistent subtyping for
polymorphic types.

There is some work on integrating gradual typing with rich type disciplines.
\citet{Ba_ados_Schwerter_2014} establish a framework to combine gradual typing and
effects, with which a static effect system can be transformed to a dynamic
effect system or any intermediate blend. \citet{Jafery:2017:SUR:3093333.3009865}
present a type system with \emph{gradual sums}, which combines refinement and
imprecision. We have discussed the interesting definition of \emph{directed
  consistency} in Section~\ref{sec:exploration}. \citet{castagna2017gradual} develop a gradual type system with
intersection and union types, with consistent subtyping defined by following
the idea of \citet{garcia2016abstracting}.
TypeScript~\citep{typescript} has a distinguished dynamic type, written {\color{blue} any}, whose fundamental feature is that any type can be
implicitly converted to and from {\color{blue} any}.
% They prove that the conversion
% definition (called \emph{assignment compatibility}) coincides with the
% restriction operator from \citet{siek2007gradual}.
Our treatment of the unknown type in \cref{fig:decl:conssub} is similar to their
treatment of {\color{blue} any}. However, their type system does not have
polymorphic types. Also, Unlike our consistent subtyping which inserts runtime
casts, in TypeScript, type information is erased after compilation so there are
no runtime casts, which makes runtime type errors possible.
% dynamic checks does not contribute to type safety.


\paragraph{Gradual Type Systems with Explicit Polymorphism}

\citet{Morris:1973:TS:512927.512938} dynamically enforces
parametric polymorphism and uses \emph{sealing} functions as the
dynamic type mechanism. More recent works on integrating gradual typing with
parametric polymorphism include the dynamic type of \citet{abadi1995dynamic} and
the \emph{Sage} language of \citet{gronski2006sage}. None of these has carefully
studied the interaction between statically and dynamically typed code.
\citet{ahmed2011blame} proposed \pbc that extends the blame
calculus~\cite{Wadler_2009} to incorporate polymorphism. The key novelty of
their work is to use dynamic sealing to enforce parametricity. As such, they end
up with a sophisticated dynamic semantics. Later, \citet{amal2017blame} prove
that with more restrictions, \pbc satisfies parametricity. Compared to their
work, our type system can catch more errors earlier since, as we argued, 
their notion of \emph{compatibility} is too permissive. For example, the
following is rejected (more precisely, the corresponding source program never
gets elaborated) by our type system:
\[
  (\blam x \unknown x + 1) : \forall a. a \to a \rightsquigarrow \cast {\unknown \to \nat}
  {\forall a. a \to a} (\blam x \unknown x + 1)
\]
while the type system of \pbc would accept the translation, though at runtime,
the program would result in a cast error as it violates parametricity.
% This does not imply, in any regard that \pbc is not well-designed; there are
% circumstances where runtime checks are \emph{needed} to ensure
% parametricity.
We emphasize that it is the combination of our powerful type system together
with the powerful dynamic semantics of \pbc that makes it possible to have
implicit higher-rank polymorphism in a gradually typed setting.
% without compromising parametricity.
\citet{devriese2017parametricity} proved that
embedding of System F terms into \pbc is not fully abstract. \citet{yuu2017poly}
also studied integrating gradual typing with parametric polymorphism. They
proposed System F$_G$, a gradually typed extension of System F, and System
F$_C$, a new polymorphic blame calculus. As has been discussed extensively,
their definition of type consistency does not apply to our setting (implicit
polymorphism). All of these approaches mix consistency with subtyping to some
extent, which we argue should be orthogonal. On a side note, it seems that our
calculus can also be safely translated to System F$_C$. However we do not
understand all the tradeoffs involved in the choice between \pbc and System
F$_C$ as a target.



\paragraph{Gradual Type Inference}
\citet{siek2008gradual} studied unification-based type inference for gradual
typing, where they show why three straightforward approaches fail to meet their
design goals. One of their main observations is
that simply ignoring dynamic types during unification does not work. Therefore,
their type system assigns unknown types to type variables and infers gradual
types, which results in a complicated type system and inference algorithm. In
our algorithm presented in \cref{sec:advanced-extension}, comparisons between
existential variables and unknown types are emphasized by the distinction
between static existential variables and gradual existential variables. By
syntactically refining unsolved gradual existential variables with unknown types, we gain a
similar effect as assigning unknown types, while keeping the algorithm relatively
simple.
\citet{garcia2015principal} presented a new approach where gradual type
inference only produces static types, which is adopted in our type system. They
also deal with let-polymorphism (rank 1 types). They proposed the distinction
between static and gradual type parameters, which inspired our extension to
restore the dynamic gradual guarantee. Although those existing works all involve
gradual types and inference, none of these works deal with higher-rank
implicit polymorphism.


\paragraph{Higher-rank Implicit Polymorphism}

\citet{odersky1996putting} introduced a type system for higher-rank implicit
polymorphic types. Based on that, \citet{jones2007practical} developed an
approach for type checking higher-rank predicative polymorphism.
\citet{dunfield2013complete} proposed a bidirectional account of higher-rank
polymorphism, and an algorithm for implementing the declarative system, which
serves as the main inspiration for our algorithmic system. The key difference,
however, is the integration of gradual typing.
% \citet{vytiniotis2012defer}
% defers static type errors to runtime, which is fundamentally different from
% gradual typing, where programmers can control over static or runtime checks by
% precision of the annotations.
As our work, those works are in a
\emph{predicative} setting, since complete type inference for higher-rank
types in an impredicative setting is undecidable. Still, there are many type
systems trying to infer some impredicative types, such as
\texttt{$ML^F$}~\citep{le2014mlf,remy2008graphic,le2009recasting}, the HML
system~\citep{leijen2009flexible}, the FPH system~\citep{vytiniotis2008fph} and
so on. Those type systems usually end up with non-standard System F types, and
sophisticated forms of type inference.

%%% Local Variables:
%%% mode: latex
%%% TeX-master: "../paper"
%%% org-ref-default-bibliography: "../paper.bib"
%%% End:


\section{Conclusion and Future Work}
\label{sec:conclusion}

We have proposed \fnamee, a type-safe and coherent calculus with disjoint
intersection types, BCD subtyping and parametric polymorphism. \fnamee improves
the state-of-art of compositional designs, and enables the development of highly
modular and reusable programs. One interesting and useful further extension
would be implicit polymorphism. For that we want to combine
Dunfield and Krishnaswami's approach~\cite{dunfield2013complete} with our bidirectional type system.
We would also like to study the parametricity of \fnamee. As we have seen in
\cref{sec:failed:lr}, it is not at all obvious how to extend the standard
logical relation of System F to account for disjointness, and avoid potential
circularity due to impredicativity. A promising solution is to use step-indexed
logical relations~\cite{ahmed2006step}. 
% TOM: This sentence is broken. Do we even need it?
% We have yet investigated further on that direction.


\section*{Acknowledgments}

We thank the anonymous reviewers and Yaoda Zhou for their helpful comments.
This work has been sponsored by the Hong Kong Research Grant
Council projects number 17210617 and 17258816, and by the Research Foundation -
Flanders.



%%% Local Variables:
%%% mode: latex
%%% TeX-master: "../paper"
%%% org-ref-default-bibliography: "../paper.bib"
%%% End:



\newpage

\def\bibfont{\small}
\bibliography{paper}


\ifdefined\submitoption
\newpage
\appendix
\section{Specification and Metatheory of \ecore}
We give the specification and metatheory of \ecore in this section. We
have completely formalized proofs of metatheory in Coq based on
Chargu{\'e}raud's work~\citeapp{charcoq}. We also provide brief paper
proofs for \ecore for reference. Full proofs can be found in the Coq
scripts\footnote{\fullurl} \verb|WeakCast_*.v|. The corresponding name of
each lemma in Coq is marked at the beginning in brackets.

\subsection{Syntax}
\begin{center}
\begin{minipage}{0.55\textwidth}
\gram{\ottec}
\end{minipage}
\begin{minipage}{0.4\textwidth}
\gram{
  \ottGg\ottinterrule
  \ottv}
\end{minipage}
\\
\end{center}
\begin{tabular}{ll}
Syntactic Sugar \\
& $\ottcoresugar$ % defined in otthelper.mng.tex
\end{tabular}

\subsection{Operational Semantics}
\ottdefnstep{}
% \ottusedrule{\ottdruleSXXMu{}}

\subsection{Typing}
\ottdefnctx{}\ottinterrule
\ottdefnexpr{}
% \ottusedrule{\ottdruleTXXMu{}}

\subsection{Properties}
\begin{comment}
We follow the naming of lemmas and proofs of properties 
for Pure Type System from \citeapp{handbook}. Some lemmas have other well-known names, like
Lemma \ref{lem:appendix:thin} is often called \emph{Weakening} and 
Lemma \ref{lem:appendix:gen} is often called \emph{Inversion}.

\begin{lemma}[Free Variable]\label{lem:appendix:free}
    If $[[G |- e:t]]$, then $\FV(e) \subseteq \dom([[G]])$ and $\FV([[t]])
\subseteq \dom([[G]])$.
\end{lemma}

\begin{proof}
    By induction on the derivation of $[[G |- e:t]]$. We only treat cases
\ruleref{T\_Mu}, \ruleref{T\_CastUp} and \ruleref{T\_CastDown} (since proofs of
other cases are the same as \cc \citeapp{handbook}):
    \begin{description}
        \item[Case \ruleref{T\_Mu}:] From premises of $[[G |- (mu x:t.e1) :
t]]$, by the induction hypothesis, we have $\FV(e_1) \subseteq \dom([[G]]) \cup
\{[[x]]\}$ and $\FV(\tau) \subseteq \dom([[G]])$. Thus the result follows by
$\FV([[mu x:t.e1]])=\FV(e_1) \setminus \{[[x]]\} \subseteq \dom([[G]])$ and
$\FV(\tau) \subseteq \dom([[G]])$.
        \item[Case \ruleref{T\_CastUp}:] Since $\FV([[castup [t]
e1]])=\FV([[e1]])$, the result follows directly by the induction hypothesis.
        \item[Case \ruleref{T\_CastDown}:] Since $\FV([[castdown
e1]])=\FV([[e1]])$, the result follows directly by the induction hypothesis.
    \end{description}
\end{proof}

\begin{definition}[Multi-step reduction]
    The relation $[[->>]]$ is the transitive and reflexive closure of
$[[-->]]$.
\end{definition}

\begin{definition}[$n$-step reduction]
    The $n$-step reduction is denoted by $[[e0]] [[-->>]] [[en]]$, if
    there exists a sequence of one-step reductions $[[e0]] [[-->]]
    [[e1]] [[-->]] [[e2]] [[-->]] \dots [[-->]] [[en]]$, where $n$ is
    a positive integer and $[[ei]]\,(i=0,1,\dots,n)$ are valid
    expressions.
\end{definition}
\end{comment}

\begin{definition}[Notation of Alpha Equality \footnote{This notation
    is also applied to sections afterwards.}]
    The alpha equivalence between terms is denoted by notation $[[=a]]$.
\end{definition}

\begin{lemma}[Weakening] \label{lem:appendix:thin}
    \verb|[typing_weaken]|
    Let $[[G]]$ and $[[G']]$ be well-formed contexts such that $[[G]] \subseteq
[[G']]$. If $[[G |- e : t]]$ then $[[G' |- e : t]]$.
\end{lemma}

\begin{proof}
    By trivial induction on the derivation of $[[G |- e : t]]$.
\end{proof}

\begin{lemma}[Substitution]\label{lem:appendix:subst}
\verb|[typing_substitution]|
	If $[[G1, x:T, G2 |- e1:t]]$ and $[[G1 |- e2:T]]$, then $[[G1, G2 [x |-> e2]
|- e1[x |-> e2]  : t[x |-> e2] ]]$.
\end{lemma}

\begin{proof}
    By induction on the derivation of $[[G1, x:T, G2 |- e1:t]]$. We use the notation $[[e* == e
[x |-> e2] ]]$ to denote the substitution for short. Then the result can be written as \[ [[G1, G2* |- e1*  : t* ]]\]
We only treat cases \ruleref{T\_Mu}, \ruleref{T\_CastUp} and
\ruleref{T\_CastDown} since other cases can be easily followed by the proof for PTS in \citeapp{handbook}.
Consider the last step of derivation of the following
cases:
    \begin{description}
        \item[Case \ruleref{T\_Mu}:] $\inferrule{[[G1, x:T, G2, y:t |- e1:t]] \\
[[G1, x:T, G2 |- t:s]]}{[[G1, x:T, G2 |- (mu y:t.e1): t]]}$ 
        
        By the induction hypothesis, we have $[[G1, G2*, y:t* |- e1* : t*]]$ and $[[G1,
G2* |- t* : star]]$. Then by the derivation rule, $[[G1, G2* |- (mu
y:t*.e1*):t*]]$. Thus we can conclude $[[G1, G2* |- (mu y:t.e1)*:t*]]$.
        \item[Case \ruleref{T\_CastUp}:] $\inferrule{[[G1, x:T, G2 |- e1:t2]]
\\ [[G1, x:T, G2 |- t1:s]] \\ [[t1 --> t2]]}{[[G1, x:T, G2 |- (castup [t1]
e1):t1]]}$ 
        
        By the induction hypothesis, we have $[[G1, G2* |- e1*:t2*]]$, $[[G1, G2*
|- t1*:star]]$ and $[[t1 --> t2]]$. By the definition of substitution, we can
obtain $[[t1* --> t2*]]$ by $[[t1 --> t2]]$. Then by the derivation rule, $[[G1,
G2* |- (castup [t1*] e1*):t1*]]$. Thus we can conclude $[[G1, G2* |- (castup [t1]
e1)*:t1*]]$.
        \item[Case \ruleref{T\_CastDown}:] $\inferrule{[[G1, x:T, G2 |- e1:t1]]
\\ [[G1, x:T, G2 |- t2:s]] \\ [[t1 --> t2]]}{[[G1, x:T, G2 |- (castdown
e1):t2]]}$ 
        
        By the induction hypothesis, we have $[[G1, G2* |- e1*:t1*]]$, $[[G1, G2*
|- t2*:star]]$ and $[[t1 --> t2]]$ thus $[[t1* --> t2*]]$. Then by the
derivation rule, $[[G1, G2* |- (castdown e1*):t2*]]$. Thus we can conclude $[[G1, G2* |-
(castdown e1)*:t2*]]$.
    \end{description}
\end{proof}

\begin{lemma}[Inversion\footnote{We only formalized some necessary
    cases, i.e., the ones marked with Coq identifiers, since others
    could be easily derived by the \texttt{inversion}
    tactic.}]\label{lem:appendix:gen}
\begin{enumerate}[(1)]
	\item If $[[G |- x:T]]$, then there exists an expression $[[t]]$ such that $[[t
=a T]]$, $[[G |- t:s]]$ and $[[x:t elt G]]$.
	\item If $[[G |- e1 e2:T]]$, then there exist expressions $[[t1]]$ and
$[[t2]]$ such that $[[G |- e1 : (Pi x:t2.t1)]]$, $[[G |- e2:t2]]$ and $[[T =a
t1[x |-> e2] ]]$.
	\item \verb|[typing_abs_inv]| If $[[G |- (\x:t1.e):T]]$, then there exists an expression $[[t2]]$ such
that $[[T =a Pi x:t1.t2]]$ where $[[G |- (Pi x:t1.t2):s]]$ and $[[G,x:t1 |-
e:t2]]$.
    \item \verb|[typing_prod_inv]| If $[[G |- (Pi x:t1.t2):T]]$, then $[[T == s]]$, $[[G |- t1:s]]$ and
$[[G, x:t1 |- t2:s]]$.
	\item If $[[G |- (mu x:t.e):T]]$, then $[[G |- t:s]]$, $[[T =a t]]$ and $[[G,
x:t|-e:t]]$.
	\item \verb|[typing_castup_inv]| If $[[G |- (castup [t1] e):T]]$, then there exists an expression $[[t2]]$
such that $[[G |- e:t2]]$, $[[G |- t1:s]]$, $[[t1 --> t2]]$ and $[[T =a t1]]$.
	\item If $[[G |- (castdown e):T]]$, then there exist expressions
$[[t1]],[[t2]]$ such that $[[G |- e:t1]]$, $[[G |- t2:s]]$, $[[t1 --> t2]]$ and
$[[T =a t2]]$.
\end{enumerate}
\end{lemma}

\begin{proof}
    Consider a derivation of $[[G |- e:T]]$ for one of cases in the lemma. We
follow the process of derivation until expression $[[e]]$ is introduced the
first time. The last step of derivation can be done by
    \begin{itemize}
        \item rule \ruleref{T\_Var} for case 1;
        \item rule \ruleref{T\_App} for case 2;
        \item rule \ruleref{T\_Lam} for case 3;
        \item rule \ruleref{T\_Pi} for case 4;
        \item rule \ruleref{T\_Mu} for case 5;
        \item rule \ruleref{T\_CastUp} for case 6;
        \item rule \ruleref{T\_CastDown} for case 7.
    \end{itemize}
    In each case, assume the conclusion of the rule is $[[G' |- e : t']]$ where
$[[G']] \subseteq [[G]]$ and $[[t' =a T]]$. Then by inspection of used
derivation rules and Lemma \ref{lem:appendix:thin}, it can be shown that the
statement of the lemma holds and is the only possible case.
\end{proof}

\begin{lemma}[Determinacy of Reduction]\label{lem:appendix:determ}
\verb|[reduct_determ]|
    If $[[t --> t1]]$ and $[[t --> t2]]$, then $[[t1 == t2]]$.
\end{lemma}

\begin{proof}
    Trivial induction on the derivation of $[[t --> t1]]$.
\end{proof}

\begin{lemma}[Well-typedness of Reduction]\label{lem:appendix:wfreduct}
\verb|[typing_wf_from_reduct]|
    If $[[G |- t:s]]$ and $[[t --> t']]$, then $[[G |- t' : s]]$.
\end{lemma}

\begin{proof}
    Trivial induction on the derivation of $[[t --> t']]$.
\end{proof}

\begin{lemma}[Correctness of Types]\label{lem:appendix:corrtyp}
\verb|[typing_wf_from_typing]|
    If $[[G |- e:t]]$ then $[[G |- t : s]]$.
\end{lemma}

\begin{proof}
    Trivial induction on the derivation of $[[G |- e:t]]$ using Lemma
\ref{lem:appendix:gen}.
\end{proof}

\subsection{Decidability of Type Checking}
\begin{lemma}[Decidability of One-step Reduction]\label{lem:appendix:unired}
  \verb|[reduct_dec]| If there is a well-typed term $e$ such that
  $[[G |- e : t]]$, it is decidable to determine whether there exists
  $e'$ such that $[[e --> e']]$.
\end{lemma}

\begin{proof}
	By induction on the structure of $[[e]]$:
	\begin{description}
        \item[Case $[[e=x]]$:] $[[e]]$ is a variable which does not match any rules of $[[-->]]$. 
        Thus there is no $[[e]]'$ such that $[[e-->e']]$.
		\item[Case $[[e=v]]$:] $[[e]]$ is a value that has one of the following forms:
		\begin{inparaenum}[(1)]
		    \item $[[star]]$,
			\item $[[\x:t.e]]$,
			\item $[[Pi x:t1.t2]]$,
			\item $[[castup [t] v]]$.
		\end{inparaenum}
		Thus, it does not match any rules of $[[-->]]$. Then there is no $[[e]]'$ such that $[[e-->e']]$.
                \item[Case $[[e]]=[[mu x:t.e1]]$:] Only rule
                \ruleref{S\_Mu} can be applied. Thus, there exists
                $[[e]]'=[[e1[x|->mu x:t.e1] ]]$.
		\item[Case $[[e]]=[[(\x:t.e1) e2]]$:] Since the first
                  term $[[\x:t.e1]]$ is a value, rule \ruleref{S\_App}
                  does not apply to this case. Thus, only rule
                  \ruleref{S\_Beta} can be applied and there exists
                  $[[e']]=[[ e1[x|->e2] ]]$.
                
		\item[Case $[[e]]=[[e1 e2]]$ and $[[e1]]$ is not a
                  $\lambda$-term:] If $[[e1]]=v$ and is not a
                  $\lambda$-term, there is no rule to reduce $[[e]]$.
                  Then there is no $[[e1']]$ such that
                  $[[e1 --> e1']]$, which does not satisfy the premise
                  of rule \ruleref{S\_App}. Thus, there is no $[[e]]'$
                  such that $[[e-->e']]$.

                  Otherwise, if $[[e1]]$ is not a value, by IH it is
                  decidable to know if $[[e1 --> e1']]$. Suppose there
                  exists some $[[e1']]$ such that $[[e1 --> e1']]$. By
                  rule \ruleref{S\_App}, $[[e]]'=[[e1' e2]]$ is the
                  unique reduction of $[[e]]$. If there is no such
                  $[[e1']]$, then there does not exist $[[e']]$.
                \item[Case $[[e]]=[[castup [t] e1]]$ and $[[e1]]$ is
                  not a value:] By IH it is decidable to know if
                  $[[e1 --> e1']]$. Suppose there exists $[[e1']]$
                  such that $[[e1 --> e1']]$. Then by rule
                  \ruleref{S\_CastUp}, there exists
                  $[[e']]=[[castup [t] e1']]$. Otherwise, there is no
                  such $[[e']]$.
		\item[Case $[[e]]=[[castdown (castup [t] v)]]$:] Since
                  $[[castup [t] v]]$ is a value, rule
                  \ruleref{S\_CastDown} does not apply to this
                  case. Thus, only rule \ruleref{S\_CastElim} can be
                  applied and there exists $[[e']]=[[v]]$.
		\item[Case $[[e]]=[[castdown e1]]$ and $[[e1]]$ is not
                  $[[castup [t] v]]$:] If $[[e1]]=v$ and is not a
                  $[[castup [t] v]]$, there is no rule to reduce
                  $[[e1]]$.  Then there is no $[[e1']]$ such that
                  $[[e1 --> e1']]$, which does not satisfy the premise
                  of rule \ruleref{S\_CastDown}. Thus, there is no
                  $[[e]]'$ such that $[[e-->e']]$.

                  Otherwise, if $[[e1]]$ is not a value, by IH it is
                  decidable to know if $[[e1 --> e1']]$. Suppose there
                  exists some $[[e1']]$ such that $[[e1 --> e1']]$.
                  Thus, by rule \ruleref{S\_CastDown},
                  $[[e]]'=[[castdown e1']]$ is the unique reduction of
                  $[[e]]$. If there is no such $[[e1']]$, then there
                  does not exist $[[e']]$.
	\end{description}
\end{proof}

\begin{lemma}[Uniqueness of Typing]
\verb|[typing_unique]|
If $[[G |- e : t1]]$ and $[[G |- e : t2]]$, then $[[t1 == t2]]$.
\end{lemma}

\begin{proof}
  By trivial induction on the derivation of $[[G |- e : t1]]$. We only
  treat the interesting case \textsc{T\_CastDown}.

  Suppose $e = [[castdown e1]]$. By induction hypothesis, we have
  $[[G |- e1 : t1']]$, $[[G |- e1 : t2']]$ and $[[t1' == t2']]$. By
  inversion, we have $[[t1' --> t1]]$ and $[[t2' --> t2]]$. Thus, by
  Lemma \ref{lem:appendix:determ}, we have $[[t1 == t2]]$.
\end{proof}

\begin{theorem}[Decidability of Type Checking]
\verb|[typing_decidable]|
Given a well-formed context $[[G]]$ and a term $[[e]]$, it is decidable
to determine if there exists $[[t]]$ such that $[[G |- e : t]]$.
\end{theorem}

\begin{proof}
	By induction on the structure of $[[e]]$:
	\begin{description}
	    \item[Case $[[e=star]]$:] Trivial by applying \ruleref{T\_Ax} and $[[t ==
star]]$.
		\item[Case $[[e=x]]$:] Trivial by rule \ruleref{T\_Var}. If $[[x:t elt G]]$, then $[[t]]$ is the
unique type of $[[x]]$ such that $[[G |- x : t]]$. Otherwise, if $[[x]] \not \in \dom([[G]])$, there is no such $[[t]]$.
		\item[Case $[[e]]=[[e1 e2]]$:] By rule \ruleref{T\_App} and induction
hypothesis, there exist unique $[[t1]]$ and $[[t2]]$ such that $[[G
|- e1 : (Pi x:t1.t2)]]$, $[[G |- e2:t1]]$. Thus, $[[t2[x |-> e2] ]]$ is the unique type of $[[e]]$ such that $[[G |- e : t2[x |-> e2] ]]$.
		\item[Case $[[e=\x:t1.e1]]$:] By rule \ruleref{T\_Lam} and induction
hypothesis, there exist unique $[[t2]]$ such that $[[G |- (Pi
x:t1.t2):s]]$ and $[[G,x:t1 |- e:t2]]$. Thus, $[[Pi x:t1.t2 ]]$ is the unique type of $[[e]]$ such that $[[G |- e : Pi x:t1.t2  ]]$.
		\item[Case $[[e=Pi x:t1.t2]]$:] By rule \ruleref{T\_Pi} and induction
hypothesis, we have $[[G |- t1:s]]$ and $[[G, x:t1 |- t2:s]]$. Thus, $[[s]]$ is the unique type of $[[e]]$ such that $[[G |- e : s  ]]$.
		\item[Case $[[e=mu x:t.e1]]$:] By rule \ruleref{T\_Mu} and induction
hypothesis, we have $[[G |- t:s]]$ and $[[G, x:t|-e:t]]$. Thus, $[[t]]$ is the unique type of $[[e]]$ such that $[[G |- e : t]]$.
		\item[Case $[[e]]=[[castup [t1] e1]]$:] From the premises of rule
\ruleref{T\_CastUp}, by the induction hypothesis, we can derive the type of
$[[e1]]$ as $[[t2]]$ by $[[G |- e1:t2]]$, and check whether $[[t1]]$ is legal by $[[G |- t1:star]]$. 
For a legal $[[t1]]$, by Lemma \ref{lem:appendix:unired} and \ref{lem:appendix:determ}, there is
a unique $[[t1']]$ such that $[[t1 --> t1']]$ or there is no such $[[t1']]$. 
If such $[[t1']]$ does not exist, then we report type checking fails. 

Otherwise, we examine if $[[t1']]$ is syntactically equal to $[[t2]]$, 
i.e., $[[t1' =a t2]]$. If the equality
holds, we conclude the unique type of $[[e]]$ is $[[t1]]$, i.e., $[[G |- e:t1]]$. Otherwise, we
report $[[e]]$ fails to type check.
		\item[Case $[[e]]=[[castdown e1]]$:] From the premises of rule
\ruleref{T\_CastDown}, by the induction hypothesis, we can derive the type of
$[[e1]]$ as $[[t1]]$ by $[[G |- e1:t1]]$. By Lemma \ref{lem:appendix:unired} and \ref{lem:appendix:determ}, there is a unique
$[[t2]]$ such that $[[t1 --> t2]]$ or such $[[t2]]$ does not exist. 

If such $[[t2]]$ exists and its sorts is
$[[star]]$, we find the unique type of $[[e]]$ is $[[t2]]$ and can conclude $[[G |- e:t2]]$. Otherwise, we
report $[[e]]$ fails to type check.
	\end{description}
\end{proof}

\subsection{Type Safety}
\begin{theorem}[Subject Reduction]
\verb|[subject_reduction_result]|
If $[[G |- e:T]]$ and $[[e]] [[-->]] e'$ then $[[G |- e':T]]$.
\end{theorem}

\begin{proof}
%     We prove the case for one-step reduction, i.e., $[[e --> e']]$. The theorem
% follows by induction on the number of one-step reductions of $[[e]] [[->>]]
% [[e']]$.
    The proof is by induction with respect to the definition of one-step
reduction $[[-->]]$ as follows:
    \begin{description}
        \item[Case $\ottdruleSXXBeta{}$:] $\quad$ \\
        Suppose $[[G |- (\x:t1.e1)e2 :T]]$ and $[[G |- e1 [x |-> e2] :T']]$. By
Lemma \ref{lem:appendix:gen}(2), there exist expressions $[[t1']]$ and $[[t2]]$
such that 
        \begin{align}
            &[[G |- (\x:t1.e1):(Pi x:t1'.t2)]] \label{equ:lam} \\
            &[[G |- e2:t1']] \nonumber \\
            &[[T =a t2 [x |-> e2] ]] \nonumber
        \end{align}
        By Lemma \ref{lem:appendix:gen}(3), the judgment (\ref{equ:lam})
implies that there exists an expression $[[t2']]$ such that
        \begin{align}
            &[[Pi x:t1'.t2 =a Pi x:t1.t2']] \label{equ:lameq}\\
            &[[G, x:t1 |- e1:t2']] \nonumber
        \end{align}
        Hence, by (\ref{equ:lameq}) we have $[[t1 =a t1']]$ and $[[t2 =a
t2']]$. Then we can obtain $[[G, x:t1 |- e1:t2]]$ and $[[G |- e2:t1]]$. By
Lemma \ref{lem:appendix:subst}, we have $[[G |- e1[x |-> e2] : t2[x |-> e2]
]]$. Therefore, we conclude with $[[T' =a t2[x |-> e2] ]] [[=a]] [[T]]$.
        
        \item[Case $\ottdruleSXXApp{}$:] $\quad$ \\
        Suppose $[[G |- e1 e2 :T]]$ and $[[G |- e1' e2 :T']]$. By Lemma
\ref{lem:appendix:gen}(2), there exist expressions $[[t1]]$ and $[[t2]]$ such
that 
        \begin{align*}
            &[[G |- e1:(Pi x:t1.t2)]] \\
            &[[G |- e2:t1]]\\
            &[[T =a t2 [x |-> e2] ]]
        \end{align*}
        By the induction hypothesis, we have $[[G |- e1':(Pi x:t1.t2)]]$. By rule
\ruleref{T\_App}, we obtain $[[G |- e1' e2 : t2[x |-> e2] ]]$. Therefore, $[[T'
=a t2[x |-> e2] ]] [[=a]] [[T]]$.
        
        \item[Case $\ottdruleSXXCastDown{}$:] $\quad$ \\
        Suppose $[[G |- castdown e :T]]$ and $[[G |- castdown e' :T']]$. By
Lemma \ref{lem:appendix:gen}(7), there exist expressions $[[t1]], [[t2]]$ such
that 
        \begin{align*}
            &[[G |- e:t1]] \qquad [[G |- t2:s]] \\
            &[[t1 --> t2]] \qquad [[T =a t2 ]]
        \end{align*}
        By the induction hypothesis, we have $[[G |- e':t1]]$. By rule
\ruleref{T\_CastDown}, we obtain $[[G |- castdown e' : t2 ]]$. Therefore, $[[T'
=a t2]] [[=a]] [[T]]$.

\item[Case $\ottdruleSXXCastUp{}$:] $\quad$ \\
        Suppose $[[G |- castup [t1] e :T]]$ and $[[G |- castup [t1] e' :T']]$. By
Lemma \ref{lem:appendix:gen}(6), there exist expressions $[[t2]]$ such
that 
        \begin{align*}
            &[[G |- e:t2]] \qquad [[G |- t1:s]] \\
            &[[t1 --> t2]] \qquad [[T =a t1 ]]
        \end{align*}
        By the induction hypothesis, we have $[[G |- e':t2]]$. By rule
\ruleref{T\_CastUp}, we obtain $[[G |- castup [t1] e' : t1 ]]$. Therefore, $[[T'
=a t1]] [[=a]] [[T]]$.
        
        \item[Case $\ottdruleSXXCastElim{}$:] $\quad$ \\
        Suppose $[[G |- castdown (castup [t1] v) :T]]$ and $[[G |- v :T']]$. By
Lemma \ref{lem:appendix:gen}(7), there exist expressions $[[t1']], [[t2]]$ such
that 
        \begin{align}
            &[[G |- (castup [t1] v):t1']] \label{equ:fold} \\
            &[[t1' --> t2]] \label{equ:foldeq1} \\
            &[[T =a t2 ]] \label{equ:foldeq4}
        \end{align}
        By Lemma \ref{lem:appendix:gen}(6), the judgment (\ref{equ:fold})
implies that there exists an expression $[[t2']]$ such that
        \begin{align}
            &[[G |- v:t2']] \label{equ:foldr} \\
            &[[t1 --> t2']] \label{equ:foldeq2} \\
            &[[t1' =a t1]] \label{equ:foldeq3}
        \end{align}
        By (\ref{equ:foldeq1}, \ref{equ:foldeq2}, \ref{equ:foldeq3}) and Lemma
\ref{lem:appendix:unired} we obtain $[[t2 =a t2']]$. From (\ref{equ:foldr}) we
have $[[T' =a t2' ]]$. Therefore, by (\ref{equ:foldeq4}), $[[T' =a t2' ]]
[[=a]] [[t2 =a T]]$.
        
        \item[Case $\ottdruleSXXMu{}$:] $\quad$ \\
        Suppose $[[G |- (mu x:t.e) :T]]$ and $[[G |- e[x |-> mu x:t.e] :T']]$.
By Lemma \ref{lem:appendix:gen}(5), we have $[[T =a t]]$ and $[[G, x:t |-
e:t]]$. Then we obtain $[[G |- (mu x:t.e) : t]]$. Thus by Lemma
\ref{lem:appendix:subst}, we have $[[G |- e[x |-> mu x:t.e] : t[x |-> mu x:t.e]
]]$.
        
        Note that $[[x]]:[[t]]$, i.e., the type of $[[x]]$ is $[[t]]$, then
$[[x]] \notin \FV([[t]])$ holds implicitly. Hence, by the definition of
substitution, we obtain $[[t[x |-> mu x:t.e] == t]]$. Therefore, $[[T' =a t[x
|-> mu x:t.e] ]] [[==]] [[t =a T]]$.
    \end{description}
\end{proof}

\begin{theorem}[Progress]
\verb|[progress_result]|
If $[[empty |- e:T]]$ then either $[[e]]$ is a value $v$ or there exists $[[e]]'$
such that $[[e --> e']]$.
\end{theorem}

\begin{proof}
    By induction on the derivation of $[[empty |- e:T]]$ as follows:
    \begin{description}
        \item[Case $[[e=x]]$:] Impossible, because the context is empty.
        \item[Case $[[e=v]]$:] Trivial, since $[[e]]$ is already a value that
has one of the following forms:
		\begin{inparaenum}[(1)]
		    \item $[[star]]$,
			\item $[[\x:t.e]]$,
			\item $[[Pi x:t1.t2]]$,
			\item $[[castup [t] v]]$.
		\end{inparaenum}
		\item[Case $[[e]]=[[e1 e2]]$:] By Lemma \ref{lem:appendix:gen}(2), there
exist expressions $[[t1]]$ and $[[t2]]$ such that $[[empty |- e1:(Pi x:t1.t2)]]$ and
$[[empty |-e2:t1]]$. Consider whether $[[e1]]$ is a value:
    		\begin{itemize}
    		    \item If $[[e1]]=v$, by Lemma \ref{lem:appendix:gen}(3), it could be either $[[castup [Pi x : t1 . t2] e1']]$ , or a
$\lambda$-term such that $[[e1 == \x:t1.e1']]$ for some $[[e1']]$ satisfying
$[[empty |- e1':t2]]$. Note that $[[Pi x : t1 .t2]]$ is a value, thus there is no $[[t2']]$ such that $[[Pi x : t1 .t2 --> t2']]$ and $[[empty |- e1': t2']]$. Thus, $[[castup [Pi x : t1 . t2] e1']]$ is not well-typed and not a possible case. By rule \ruleref{S\_Beta}, we have $[[(\x:t1.e1') e2 -->
e1' [x |-> e2] ]]$. Thus, there exists $[[e' == e1' [x |-> e2] ]]$ such that
$[[e --> e']]$.
    		    \item Otherwise, by the induction hypothesis, there exists $[[e1']]$ such
that $[[e1 --> e1']]$. Then by rule \ruleref{S\_App}, we have $[[e1 e2 --> e1'
e2]]$. Thus, there exists $[[e' == e1' e2]]$ such that $[[e --> e']]$.
    		\end{itemize}
                \item[Case $[[e]]=[[castup [t] e1]]$ and $[[e1]]$ is
                  not a value:] By IH, $[[e1]]$ is well-typed. Then
                  there exists $[[e1']]$ such that $[[e1 -->
                  e1']]$. Thus there exists $[[e']]=[[castup [t] e1']]$.
		\item[Case $[[e]]=[[castdown e1]]$:] By Lemma \ref{lem:appendix:gen}(7),
there exist expressions $[[t1]]$ and $[[t2]]$ such that $[[empty |- e1:t1]]$ and
$[[t1 --> t2]]$. Consider whether $[[e1]]$ is a value:
		     \begin{itemize}
    		    \item If $[[e1]]=v$, by Lemma \ref{lem:appendix:gen}(6), it must be a
$[[castup]]$-term. Because the type of any other value is still a value and $[[t1 --> t2]]$ does not hold. Thus we have $[[e1 == castup [t1] e1']]$ for some $[[e1']]$
satisfying $[[empty |- e1':t2]]$. Then by rule \ruleref{S\_CastElim}, we can obtain
$[[castdown (castup [t1] e1') --> e1']]$. Thus, there exists $[[e' == e1']]$
such that $[[e --> e']]$.
    		    \item Otherwise, by the induction hypothesis, there exists $[[e1']]$ such
that $[[e1 --> e1']]$. Then by rule \ruleref{S\_CastDown}, we have $[[castdown
e1 --> castdown e1']]$. Thus, there exists $[[e' == castdown e1']]$ such that
$[[e --> e']]$.
    		\end{itemize}
		\item[Case $[[e]]=[[mu x:t.e1]]$:] By rule \ruleref{S\_Mu}, there always
exists $[[e' == e1[x |-> mu x:t.e1] ]]$.
    \end{description}
\end{proof}

\section{Specification and Metatheory of \namef}
Like \ecore, proofs for metatheory of \namef are completely formalized
in Coq. However, we only state lemmas \emph{without} paper proofs in
this section, since proofs for the erased system are very similar to
the ones for PTS~\cite{handbook}. Full proofs can be found in the
following Coq scripts\footnote{\fullurl}: \verb|FullCast_*.v| for the original system and
\verb|CoCMu_*.v| for the erased system. The corresponding name of each
lemma in Coq is marked at the beginning in brackets.

\subsection{Syntax}
\begin{center}
\begin{minipage}{0.55\textwidth}
\gram{\ottef}
\end{minipage}
\begin{minipage}{0.4\textwidth}
\gram{
  \ottGg\ottinterrule
  \ottvf}
\end{minipage}
\end{center}
    
\subsection{Erasure}
\erasuredef

\subsection{Typing}
\ottdefnctx{}\ottinterrule
\ottdefnexprfull{}

\subsection{Erased System}
\begin{description}
\item[Syntax]
\hfill \\[5pt]
\begin{center}
\gram{\otter\ottinterrule
        \ottGr\ottinterrule
\ottvu}
\end{center}
\hfill \\[5pt]

\item[Weak-head Reduction]
\hfill \\[5pt]
\ottdefnstepr{}

\item[Parallel Reduction]
\hfill \\[5pt]
\ottdefnstepp{}
\begin{definition}[Multi-step Parallel Reduction]
    The relation $[[-p*>]]$ is the transitive and reflexive closure of
    $[[-p*>]]$.
\end{definition}

\begin{definition}[Equality up to Parallel Reduction]
  The relation $[[=p]]$ is the reflexive, transitive and symmetric
  closure of $[[-p>]]$.
\end{definition}

\item[Typing]
\hfill \\[5pt]
\ottdefnctxr{}\ottinterrule
\ottdefnexprr{}
\end{description}

\subsection{Decidability of Type Checking}
\begin{lemma}[Uniqueness of Typing]
\verb|[typsrc_unique]|
If $[[G |- e : t1]]$ and $[[G |- e : t2]]$, then $[[t1 == t2]]$.
\end{lemma}

\begin{lemma}[Decidability of Type Checking]
\verb|[typsrc_decidable]|
Given a well-formed context $[[G]]$ and a term $[[e]]$, it is decidable
to determine if there exists $[[t]]$ such that $[[G |- e : t]]$.
\end{lemma}

\subsection{Correctness of Types}
\begin{lemma}[Weakening]
    \verb|[typsrc_weaken]|
    Let $[[G]]$ and $[[G']]$ be well-formed contexts such that $[[G]] \subseteq
[[G']]$. If $[[G |- e : t]]$ then $[[G' |- e : t]]$.
\end{lemma}

\begin{lemma}[Substitution]
\verb|[typsrc_substitution]|
	If $[[G1, x:T, G2 |- e1:t]]$ and $[[G1 |- e2:T]]$, then $[[G1, G2 [x |-> e2]
|- e1[x |-> e2]  : t[x |-> e2] ]]$.
\end{lemma}

\begin{lemma}[Inversion of Full Casts]\label{lem:appendix:gen}
\begin{enumerate}[(1)]
	\item \verb|[typsrc_castup_inv]| If $[[G |- (castupf [t1] e):T]]$, then there exists an expression $[[t2]]$
such that $[[G |- e:t2]]$, $[[G |- t1:s]]$, $[[|t1| -p> |t2|]]$ and $[[T =a t1]]$.
	\item \verb|[typsrc_castdn_inv]| If $[[G |- (castdownf [t2] e):T]]$, then there exists an expression $[[t1]]$
such that $[[G |- e:t1]]$, $[[G |- t2:s]]$, $[[|t1| -p> |t2|]]$ and $[[T =a t2]]$.
\end{enumerate}
\end{lemma}

\begin{lemma}[Correctness of Types]\label{lem:appendix:corrtyp}
\verb|[typsrc_wf_from_typsrc]|
    If $[[G |- e:t]]$ then $[[G |- t : s]]$.
\end{lemma}

\subsection{Soundness of Erasure}
\begin{lemma}[Substitution commutes with erasure]
\verb|[erasure_subst]|
We always have $[[|e [x |-> e']| = |e|[x |-> |e'|] ]]$.
\end{lemma}

\begin{lemma}[Substitution of Parallel Reduction after Erasure]
\verb|[erpared_red_out]| If $[[|e1| -p> |e2|]]$, then
  $[[|e1 [x |-> e]| -p> |e2 [x |-> e]| ]]$.
\end{lemma}

\begin{lemma}[Soundness of Erasure]
  \verb|[typsrc_to_typera]| If $[[G |- e : t]]$ then
  $[[|G| |- |e| : |t|]]$.
\end{lemma}

\subsection{Properties of Parallel Reduction}
\begin{lemma}[Reflexivity of $[[-p>]]$]
  \verb|[pared_red_refl]| If $[[r]]$ is a well-formed term, then
  $[[r -p> r]]$ holds.
\end{lemma}

\begin{lemma}[Substitution of $[[-p>]]$]
  \verb|[pared_red_out]| If $[[r1 -p> r2]]$, then
  $[[r1 [x |-> r] -p> r2 [x |-> r] ]]$.
\end{lemma}

\begin{lemma}[Confluence of $[[-p>]]$]
  \verb|[pared_iter_confluence]| If $[[r -p*> r1]]$ and
  $[[r -p*> r2]]$, then there exists $[[r']]$ such that
  $[[r1 -p*> r']]$ and $[[r2 -p*> r']]$.
\end{lemma}

\subsection{Type Safety of Erased System}
\begin{lemma}[Weakening]
    \verb|[typera_weaken]|
    Let $[[De]]$ and $[[De']]$ be well-formed contexts such that $[[De]] \subseteq
[[De']]$. If $[[De |- r : rh]]$ then $[[De' |- r : rh]]$.
\end{lemma}

\begin{lemma}[Correctness of Types]
\verb|[typera_wf_from_typera]|
    If $[[De |- r:rh]]$ then $[[De |- rh : s]]$.
\end{lemma}

\begin{lemma}[Substitution]
\verb|[typera_substitution]|
  If $[[De1, x:rh', De2 |- r1:rh]]$ and $[[De1 |- r2:rh']]$, then
  $[[De1, De2 [x |-> r2] |- r1[x |-> r2] : rh[x |-> r2] ]]$.
\end{lemma}

\begin{lemma}[Subject Reduction]
\verb|[subject_reduction_era]|
  If $[[De |- r:rh]]$ and $[[r -p> r']]$ then $[[De |- r':rh]]$.
\end{lemma}

\begin{lemma}[Progress]
\verb|[progress_era]|
  If $[[empty |- r:rh]]$ then either $[[r]]$ is a value $u$ or there
  exists $[[r']]$ such that $[[r --> r']]$.
\end{lemma}

\begin{comment}
\section{Full Specification of Surface Language}\label{sec:app:sufcc}
\subsection{Syntax}
See Figure \ref{fig:appendix:syntax}.
\begin{figure*}
\centering
\gram{\ottpgm\ottinterrule
\ottdecl\ottinterrule
\ottu\ottinterrule
\ottp\ottinterrule
\ottE\ottinterrule
\ottGs}
\begin{align*}
&\text{Syntactic Sugar} \\
&\ottsurfsugar % defined in otthelper.mng.tex
\end{align*}
\caption{Syntax of the surface language}
\label{fig:appendix:syntax}
\end{figure*}

\subsection{Expression Typing}
See Figure \ref{fig:appendix:typing}.

\subsection{Translation to \ecore}
See Figure \ref{fig:appendix:translate}.

\subsection{Type Safety of the Translation}

\begin{theorem}[Type Safety of Expression Translation]
Given a surface language expression $[[E]]$ and context $[[Gs]]$, 
if $[[Gs |- E:A ~> e]]$, $[[Gs |- A:star ~> t]]$ and $[[|- Gs ~> G]]$, then
$[[G |- e:t]]$.
\end{theorem}

\begin{proof}
    By induction on the derivation of $[[Gs |- E : A ~> e]]$. Suppose there is
a core language context $[[G]]$ such that $[[|- Gs ~> G]]$.
    \begin{description}
        \renewcommand{\hlmath}[1]{#1}
        \item[Case $\ottdruleTRXXAx{}$:] $\quad$ \\ Trivial. $[[e]] = [[t]] = [[star]]$ and
$[[G |- star:star]]$ holds by rule \ruleref{T\_Ax}.
        \item[Case $\ottdruleTRXXVar{}$:] $\quad$ \\ Trivial. By rule \ruleref{T\_Var}, we
have $[[|- Gs ~> G]]$, then $[[x]]:[[t]] [[elt]] [[G]]$ where $[[Gs |-
A:star~>t]]$.
        \item[Case $\ottdruleTRXXApp{}$:] $\quad$ \\ Suppose
            \[\begin{array}{l}
            [[Gs |- E1 E2 : A1[x |-> E2] ~> e1 e2]] \\
            [[Gs |- A1[x |-> E2] : star ~> t1 [x |-> e2] ]].
            \end{array} \]
            By induction
            hypothesis, we have 
            $
            [[G |- e1 : (Pi x:t2.t1)]],
            [[G |- e2:t2]],
            $
            where
            \[\begin{array}{l}
             [[Gs |- E1 : (Pi x:A2.A1) ~> e1]] \\
              [[Gs |- (Pi x:A2.A1) : star ~> (Pi x:t2.t1)]] \\
              [[Gs |- E2 : A2 ~> e2]] \\
              [[Gs |- A2 : star ~> t2]].
            \end{array}\] Thus by rule \ruleref{T\_App}, we can conclude $[[G |- e1 e2 : t1 [x |-> e2] ]]$.
        \item[Case $\ottdruleTRXXLam{}$:] $\quad$ \\ Suppose
            \[\begin{array}{l}
            [[Gs |- (\x:A1.E):(Pi x:A1.A2) ~> \x:t1.e]] \\ 
            [[Gs |- Pi x:A1.A2 : star ~> Pi x:t1.t2]].
            \end{array} \]
            By the induction hypothesis, we have 
            $
            [[G, x : t1 |- e:t2]],
            [[G |- Pi x:t1.t2 : star]]
            $
            where 
            \[
            \begin{array}{ll}
            [[Gs, x : A1 |- E : A2 ~> e]] & \\
            [[Gs |- A1 : star ~> t1]] & [[Gs |- A2 : star ~> t2]] \\
            [[Gs |- (Pi x:A1.A2) : s ~> Pi x:t1.t2]] &
            \end{array}
            \]
            Thus by rule \ruleref{T\_Lam}, we can conclude $[[G |- (\x:t1.e):(Pi x:t1.t2)]]$.
        \item[Case $\ottdruleTRXXPi{}$:] $\quad$ \\ Suppose 
                \[ [[Gs |- (Pi x:A1.A2):s ~> Pi x:t1.t2]]. \] 
            By the induction hypothesis, we have 
            $
                [[G |- t1 : star]], [[G, x : t1 |- t2 : star]]
            $
            where
            $
                [[Gs |- A1 : s ~> t1]], [[Gs, x: A1 |- A2 : s ~> t2]]
            $
            Thus by rule \ruleref{T\_Pi} we can conclude $[[G |- (Pi x:t1.t2) : star]]$.
        \item[Case $\ottdruleTRXXMu{}$:] $\quad$ \\ Suppose 
                \[\begin{array}{l}
                    [[Gs |- (mu x:A . E):A ~> mu x:t.e]] \\
                    [[Gs |- A : star ~> t]]. 
                \end{array}\]
            By the induction hypothesis, we have 
                \[ [[G, x : t |- e : t]],\text{ where }[[Gs, x:A |- E:A ~> e]]. \] 
            Thus by rule \ruleref{T\_Mu}, we can conclude $[[G |- (mu x:t.e) : t]]$.
        \item[Case $\ottdruleTRXXCase{}$:] $\quad$ \\ Suppose 
            \[\begin{array}{l}
                [[Gs |- case E1 of << p => E2>> : B ~> (unfoldnp e1) T <<e2>>]] \\
                [[Gs |- B : star ~> T]].
            \end{array}\]
            By the induction hypothesis, we have 
            \[\begin{array}{ll}
                [[Gs |- E1 : D@<<U>>n ~> e1]] &
                [[Gs |- D@<<U>>n : star ~> t1]] \\
                [[G |- e1 : t1]] &
                [[<< Gs |- p => E2 : D@<<U>>n -> B ~> e2 >>]]            
            \end{array}\]
            By rule \ruleref{TRpat\_Alt}, we have
            \begin{align*}
                [[p]] &[[==]] [[K <<x:A[<< u |-> U >>]>>]] \\
                [[<<e2>>]] &[[==]] [[<<\ <<x:t'>> .e>>]]
            \end{align*}
            where
            \[\begin{array}{ll}
                [[<<Gs |- E2 : B ~> e>>]] &
                [[<<G |- e : T>>]] \\
                [[<<Gs |- U : star ~> uu'>>]] &
                [[<<Gs |- A[<< u |-> U >>]:star ~> t[<<uu |-> uu'>>]>>]] \\
                [[t']] [[==]] [[ t[<<uu |-> uu'>>] ]]
            \end{array}\]
            By rule \ruleref{TRdecl\_Data}, we have $[[D]]  [[ == ]] \ottdeclD$. Thus,
            \[ [[t1]] [[==]] [[D]] [[<<uu'>>]]^n,\text{ where }[[<<G |- uu' : ro>>]].\] 
            Note that by operational semantics, the following reduction sequence follows for $[[t1]]$:
            \begin{align*}
                [[D]] [[<<uu'>>]]^n~
                &[[-->]]~ [[(\ <<u:ro>>n . (bb:star) -> << ((<<x : t[D |-> X][X |-> D]>>) -> bb) >> -> bb) ]][[<<uu'>>]]^n\\
                &[[-->>]]~ [[(bb:star) -> << (<<x:t'>>) -> bb >> -> bb]]
            \end{align*}
            Then by
            rule \ruleref{T\_CastDown} and the definition of $n$-step cast operator, the
            type of $[[unfoldnp e1]]$ is \[ [[(bb:star) -> << (<<x:t'>>) -> bb >> -> bb]].\] Note
            that by rule \ruleref{T\_Lam}, $[[G |- e2 : (<<x:t'>>) -> T]]$. Therefore, by rule
            \ruleref{T\_App}, we can conclude $[[G |- (unfoldnp e1) T <<e2>> : T]]$.
    \end{description}
\end{proof}

\begin{figure*}
\small
\begin{spacing}{0.8}
\renewcommand{\hlmath}[1]{}
\renewcommand{\ottdrulename}[1]{\textsc{\replace{#1}{TR}{TS}}}
\renewcommand{\ottcom}[1]{\text{\replace{#1}{translation}{typing}}}
\ottdefnctxtrans{}\ottinterrule
\ottdefnpgmtrans{}\ottinterrule
\ottdefndecltrans{}\ottinterrule % defined in otthelper.mng.tex
\ottdefnpattrans{}\ottinterrule
\ottdefnexprtrans{}
\end{spacing}
\caption{Typing rules of the surface language}
\label{fig:appendix:typing}
\end{figure*}

\begin{figure*}
\small
\begin{spacing}{0.8}
\ottdefnctxtrans{}\ottinterrule
\ottdefnpgmtrans{}\ottinterrule
\ottdefndecltrans{}
\[\hlmath{\ottdecltrans}\]\ottinterrule % defined in otthelper.mng.tex
\ottdefnpattrans{}\ottinterrule
\ottdefnexprtrans{}
\end{spacing}
\caption{Translation rules of the surface language}
\label{fig:appendix:translate}
\end{figure*}
\end{comment}

\fi


\end{document}
