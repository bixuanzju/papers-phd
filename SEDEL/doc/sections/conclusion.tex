\section{Conclusion}

This paper presents \name: the first design for a polymorphic statically-typed
language with first-class traits, supporting dynamic inheritance as well as
conventional OO features such as dynamic dispatching and abstract methods. The
paper also shows how high-level source language constructs can be elaborated
into a core record calculus with disjoint polymorphism. Finally the paper
illustrates the applicability of \name by showing greatly improved design
patterns such as Object Algebras and Extensible \textsc{Visitor}s, leveraging
first-class traits. As for future work, we are interested to study how
first-class traits interacts with features such as mutable state and recursive
types. For mutable state, one immediate issue of supporting mutation is how it
affects the coherence property of \bname, and we foresee major
technical challenges to adjust the previous coherence proof. A more powerful proof method such as
logical relations~\cite{xuan_nested, ahmed2004semantics} may be needed.

% \jeremy{future work}

% This paper describes \name: a polymorphic, statically-typed and delegation-based
% programming language. \name provides a powerful form of conflict-free multiple
% inheritance mechanism called dynamically composable traits. \name is both safe
% and flexible. Throughout the paper, we have shown how the mechanisms of \name
% improve extensibility design pattern such as Extensible Visitors and Object
% Algebras.

% There are many avenues for future work. On one hand, we intend to port the core
% functionality of \name into a JVM-based compiler. One the other hand, \name is
% still very simple, lacking interesting features such as state variables,
% recursive types, \textbf{super} keyword, etc. So we are interested to see how
% our trait model interacts with those features. Finally we plan to further study
% the formal meta-theory of the extended version of \bname.

%%% Local Variables:
%%% mode: latex
%%% TeX-master: "../paper"
%%% org-ref-default-bibliography: ../paper.bib
%%% End:
