\section{Introduction}
\label{sec:intro}

Many dynamically typed-languages (including JavaScript~\cite{javascript}, Ruby~\cite{ruby}, Python~\cite{python}
or Racket~\cite{racket}) support \emph{first-class classes}~\cite{DBLP:conf/aplas/FlattFF06}, or related concepts
such as first-class mixins and/or traits. In those languages classes
are first-class values and, like any other values, they can be
passed as an argument, or returned from a function. Furthermore
first-class classes support \emph{dynamic inheritance}: i.e., they
can inherit from other classes at \emph{runtime}, enabling
programmers to abstract over the inheritance hierarchy. 
Those features make first-class classes very powerful and expressive,
and enable highly modular and reusable pieces of code, such as:
\begin{lstlisting}[language=JavaScript]
const mixin = Base => { return class extends Base { ... } };
\end{lstlisting}
In this piece of JavaScript code, \lstinline{mixin} is
parameterized by a class \lstinline{Base}. Note that the concrete
implementation of \lstinline{Base} can be 
even dynamically determined at runtime, for example 
after reading a configuration file to decide which 
class to use as the base class.  When applied to an argument, 
\lstinline{mixin} will create a new class on-the-fly and return that
as a result. Later that class can be instantiated and used to create 
new objects, as any other classes.

In contrast, most statically-typed
languages do not have first-class classes and dynamic
inheritance. While all statically-typed OO languages allow first-class
\emph{objects} (i.e. objects can be passed as arguments and returned
as results), the same is not true for classes. Classes in languages such
Scala, Java or C++ are typically a second-class construct, and the
inheritance hierarchy is \emph{statically determined}. The closest thing
to first-class classes in 
languages like Java or Scala are classes such as 
\lstinline[language=java]{java.lang.Class} that enable representing classes and
interfaces as part of their reflective framework. \lstinline[language=java]{java.lang.Class} can be used to
mimic some of the uses of first-class classes, but in an essentially
dynamically-typed way. Furthermore simulating first-class classes
using such mechanisms is highly cumbersome because classes need to be
manipulated programmatically. For example instantiating a new class
cannot be done using the standard \lstinline{new} construct, but
rather requires going through API methods of
\lstinline[language=java]{java.lang.Class}, such as \lstinline{newInstance}, for
creating a new instance of a class.

Despite the popularity and expressive power of first-class classes in dynamically-typed
languages, there is surprisingly little work on typing of first-class
classes (or related concepts such as first-class mixins or traits).
First-class classes and dynamic inheritance pose well-known
difficulties in terms of typing. For example, in his thesis,
Bracha~\cite{bracha1992programming} comments several times on the difficulties of typing
dynamic inheritance and first-class mixins, and proposes the
restriction to static inheritance that is also common in
statically-typed languages. He also observes that such restriction
poses severe limitations in terms of expressiveness, but that appeared
(at the time)
to be a necessary compromise when typing was also desired.
Only recently some progress has been made in statically typing 
first-class classes and dynamic inheritance. In particular there are
two works in this area: Racket's gradually
typed first-class classes~\cite{DBLP:conf/oopsla/TakikawaSDTF12}; and Lee et al.'s model of
typed first-class classes~\cite{DBLP:conf/ecoop/LeeASP15}. Both works provide typed models of
first-class classes, and they enable encodings of mixins~\cite{bracha1990mixin}
similar to those employed in dynamically-typed languages.

However, as far as we known no previous work supports statically-typed
\emph{first-class traits}. Traits~\cite{scharli2003traits} are an alternative to
mixins, and other models of (multiple) inheritance. The key difference between traits and mixins lies on the
treatment of conflicts when
composing multiple traits/mixins. Mixins adopt an
\emph{implicit} resolution strategy for conflicts, where the 
compiler automatically picks one implementation in case of conflicts.
For example, Scala uses the order of mixin composition to determine which 
implementation to pick in case of conflicts.
Traits, on the other hand, employ an \emph{explicit} resolution
strategy, where the compositions with conflicts are rejected, and the
conflicts are explicitly resolved by programmers. 

Sch\"arli et al.~\cite{scharli2003traits} make a good case for the advantages of the trait
model. In particular, traits avoid bugs that could arise from
accidental conflicts that were not detected by programmers. With the
mixin model, such conflicts would be silently resolved, possibly
resulting in unexpected runtime behaviour due to a wrong method
implementation choice. In a setting with dynamic inheritance and
first-class classes this problem
is exacerbated by not knowing all components being composed
statically, greatly increasing the possibility of accidental
conflicts. From a modularity point-of-view,
the trait model also ensures that composition is \emph{commutative}, thus the order
of composition is irrelevant and does not affect the
semantics. Bracha~\cite{bracha1992programming} claims that ``\emph{The only modular
  solution is to treat the name collisions as errors...}'', strengthening the case for the use of a trait model of
composition. Otherwise, if the semantics is affected by the order of
composition, global knowledge about the full inheritance graph is
required to determine which implementations are chosen. 
Sch\"arli et al. discuss several other issues with mixins, which
can be improved by traits. We refer to their paper for further details.

%% \footnote{\name stands for \textbf{S}afe and \textbf{E}xpressive \textbf{DEL}egation}
This paper presents the design of \name: a polymorphic statically-typed
(pure) language with \emph{first-class traits}, supporting \emph{dynamic
  inheritance} as well as conventional OO features such as
\emph{dynamic dispatching} and \emph{abstract methods}.
Traits pose additional challenges when compared to
models with first-class classes or mixins, because method conflicts
should be detected \emph{statically}, even in the presence of features
such as dynamic inheritance and composition and \emph{parametric polymorphism}.
To address the challenges of typing first-class traits and detecting conflicts statically,
\name adopts a polymorphic structural type system based on
\emph{disjoint polymorphism}~\cite{alpuimdisjoint}.
The choice of structural typing is due to its simplicity, but we think
similar ideas should also work in a nominal type system.

The main contribution of this paper is to show how to model
source language constructs for first-class traits and
dynamic inheritance,
supporting standard OO features such as \emph{dynamic dispatching} 
and \emph{abstract methods}. Previous work on disjoint intersection types is
aimed at core record calculi, and omits important features
for practical OO languages, including (dynamic) inheritance, dynamic
dispatching and abstract methods. Based on Cook and Palsberg's work on the
denotational semantics for inheritance~\cite{cook1989denotational}, we show how to design a source 
language that can be elaborated into Alpuim et al.'s
\bname~\cite{alpuimdisjoint}, a polymorphic calculus with records supporting disjoint polymorphism.
\name's elaboration into \bname is proved to be both type-safe and
coherent. Coherence ensures that the semantics of
\name is unambiguous. In particular this property is useful to ensure that
programs using traits are free of conflicts/ambiguities
(even when the types of the object parts being composed are not fully statically
know).

We illustrate the applicability of \name with several example uses for
first-class traits. Furthermore we conduct a case study that modularizes programming 
language interpreters using a highly modular form of
Object Algebras~\cite{oliveira2012extensibility} and \textsc{Visitor}s.
In particular we show how \name can easily compose multiple object
algebras into a single object algebra. Such composition operation has
previously been shown to be highly challenging in languages like Java
or Scala~\cite{oliveira2013feature, rendel14attributes}. The previous state-of-the-art implementations for such
operation require employing type-unsafe reflective techniques
% (including \lstinline[language=java]{java.lang.Class})
to simulate the features of first-class classes. Moreover conflicts are not
statically detected. In contrast the approach in this paper is fully
type-safe, convenient to use and conflicts are statically detected.    

% \begin{comment}
% Mainstream statically-typed Object-Oriented Programming (OOP) languages (such as Java,
% C++, C\# or Scala) all use a similar programming model based on
% classes. This programming model has its roots on the
% origins of OOP in the 1960s in the Simula language~\cite{dahl1967simula}.
% We will refer to this model as the \emph{covariant model} for the
% remainder of this paper, because in this model inheritance and
% subtyping vary in the same way. More concretely
% the following is expected in the covariant model:

% \begin{itemize}

% \item {\bf Extensions always produce subtypes:} In the covariant model, when a
% subclass \emph{extends} a class it automatically becomes a
% \emph{subtype} of the superclass.

% \item{\bf Inheritance and subtyping go along together:}
% Class extension does two things at once: it inherits code from the
% superclass; and it creates a subtype.

% \end{itemize}

% The covariant model has been successfully used for over 50 years,
% so it clearly has demonstrated its value in practice.
% Part of this success can probably be attributed
% to its relative simplicity. In particular, programmers do not have to think carefully
% about the difference between subtyping and inheritance (indeed many
% programmers confuse the two concepts).

% Nevertheless the study of the theoretical foundations of OOP has taught us that
% the story is not quite so simple. Since the earliest works on the theory of OOP
% and subtyping, we have known that the covariant view of objects is somewhat
% simplified. Already in \cite{cardelli1984semantics}'s earliest work on
% the theoretic foundations of OOP~\cite{cardelli1984semantics}, we knew that functions do not behave in a
% strictly covariant way. However it was only until \cite{cook1989inheritance}'s
% famous paper~\cite{cook1989inheritance} on ``\emph{Inheritance is not Subtyping}'' that the issues were
% discussed in more detail. As \cite{cook1989inheritance} argued inheritance and subtyping are
% different relations: subtyping being a relation on types and inheritance being a
% relation on objects. In the covariant model the subtype relation is based on the
% inheritance hierarchy. This works very well if extensions produce subtypes.
% However, as \cite{cook1989inheritance}'s work famously demonstrated this is not always the
% case. The essential implication of this is that the covariant model cannot
% express well programs where inheritance and subtyping do not go along together.
% Following their observations about inheritance and subtyping, \cite{cook1989inheritance}
% suggest a more general and flexible programming model (which we call the
% \emph{flexible model}) with the following properties:

% \begin{itemize}

% \item {\bf Inheritance and subtyping should be decoupled:}
% That is, there should be different mechanisms for class inheritance
% and class/interface subtyping.

% \item {\bf Extensions do not always produce subtypes:}
% There are cases where classes can inherit from other classes without
% producing subtypes.

% \end{itemize}

% Despite being proposed almost 30 years ago, and one of the most famous papers in
% OOP, \cite{cook1989inheritance}'s paper has not had much impact on the design of mainstream OOP
% languages (although it has influenced the design of several academic
% languages~\cite{america1991designing,graver1989type,chambers1992object,bruce1995polytoil}).
% %Certainly this is not because
% %researchers or designer of OOP language are unaware of the subtleties
% %of covariance and contravariance. Indeed over the years, and for other
% %reasons various features have been added to programming languages to
% We believe that there are two primary reasons for the lack of adoption
% of their model.  Firstly, the mental programming model is not
% as simple as the covariant model. In the flexible model programmers have to
% think more carefully on whether extensions produce subtypes or not,
% for example.  Thus, it is crucial for programmers to understand the
% difference between subtyping and inheritance.
% Secondly, and perhaps more importantly, there are not many compelling applications in
% the literature where the need for a more flexible OOP model is
% necessary. Thus language designers may argue that the costs outweigh
% the benefits, and may decide to stick instead to the covariant
% model, which is simple, and well-understood by programmers.
% %Indeed a famous instance of this is the design of DART

% This paper has three primary goals. The first goal is to argue that the
% covariant model significantly restricts statically-typed OOP in terms of
% modularity and reuse for important practical applications. The second goal is to
% identify additional desirable features that improve flexibility of OOP and are
% needed in practice. In particular we argue that supporting a more \emph{dynamic}
% form of inheritance (where concrete implementations of the inherited code are
% possibly unknown) is highly desirable in practice. Thus we are naturally led to
% a OO language design using \emph{delegation} (or \emph{dynamic
%   inheritance})~\cite{ungar1988self,chambers1992object}. Finally, we present
% \name\footnote{\name stands for: {\bf S}afe and {\bf E}xpressive {\bf
%     DEL}egation}: a purely functional OO language that puts these ideas into
% practice using a polymorphic structural type system based on \emph{disjoint
%   intersection types}~\cite{oliveira2016disjoint} and \emph{disjoint
%   polymorphism}~\cite{alpuimdisjoint}. The choice of structural typing is due to
% its simplicity, but we think similar ideas should also work in a nominal type
% system.

% Regarding the first goal, we show that the inflexibility of type systems of
% mainstream statically-typed OOP languages is problematic for \emph{extensible
%   designs}. There has been a remarkable number of works aimed at improving
% support for extensibility in programming languages
% ~\cite{Prehofer97,Tarr99ndegrees,Harrison93subject,McDirmid01Jiazzi,Aracic06CaesarJ,Smaragdakis98mixin,nystrom2006j,togersen:2004,Zenger-Odersky2005,oliveira09modular,oliveira2012extensibility}.
% Some of the more recent work on extensibility is focused on design patterns such
% as Extensible Visitors~\cite{togersen:2004,oliveira09modular} or Object Algebras
% \cite{oliveira2012extensibility}. Although such design patterns give practical
% benefits in terms of extensibility, they also expose limitations in existing
% mainstream OOP languages. In particular there are two main issues:

% \begin{enumerate}

% \item {\bf Visitor/Object-Algebra extensions produce supertypes:}
%   Since in the covariant model extensions always produce subtypes, it
%   is very hard to correctly express the subtyping relations between
%   Visitor and Object Algebra extensions and the original types.

% \item {\bf Object Algebra combinators require a very flexible form of
%     dynamic inheritance:} As shown by \cite{oliveira2013feature,rendel14attributes}, Object Algebra
%   combinators, which allow a very flexible form of composition for
%   Object Algebras, requires features not available in languages like
%   Java or Scala.

% \end{enumerate}

% It is clear that an obvious way to solve the first issue is to move away from
% the covariant model, and this is precisely what \name does. For the second
% issue, \cite{oliveira2013feature,rendel14attributes} do show how to encode
% Object Algebra combinators in Scala. However, this requires the use of low-level
% (type-unsafe) reflection techniques, such as dynamic proxies, reflection or
% other forms of meta-programming. Better language support is desirable. \name
% does this by embracing delegation, and having a powerful polymorphic type system
% with intersection types and disjoint polymorphism. \name's type system is
% capable of statically type checking the code for delegation-based Object Algebra
% combinators and their uses, while at the same time ensuring that the composition
% is \emph{conflict-free}. The mechanism that enables this is called
% \emph{dynamically composable traits}, since the model is quite close to
% traits~\cite{scharli2003traits}, but based on dynamic inheritance.
% \end{comment}


% \begin{comment}
% However
% \cite{oliveira2016disjoint} and \cite{alpuimdisjoint}
%  do not show how to build high-level OOP constructs for
% delegation-based programming, discuss the applications to extensible designs, or
% provide a language implementation and conduct case studies to evaluate a
% language design. What their work does is to develop core calculi that support an
% expressive form of \emph{intersection
%   types}~\cite{coppo1981functional,pottinger1980type} with a \emph{merge
%   construct}~\cite{dunfield2014elaborating}. In \name subtyping is modelled with
% disjoint intersection types, and inheritance is modelled with the merge
% construct. \name inherits the key properties of these calculi, which are
% type-safety and coherence.
% \end{comment}


% \begin{comment}
% The
% novelty of the work in this paper is a three-fold. Firstly we show how
% to develop and implement a statically typed, delegation-based OOP
% source language on top of core language constructs provided by
% disjoint intersection types. Secondly we illustrate the applications
% of those high-level constructs to solve issues that show up in
% extensibility designs. Finally, we provide a case study on
% modularization of language components.
% \end{comment}

In summary the contributions of this paper are:
\begin{itemize}

\item {\bf Typed first-class traits}: We present \name: a statically-typed
  language design that supports first-class traits, dynamic
  inheritance, as well as standard high-level OO constructs such as
  dynamic dispatching and abstract methods.


\item {\bf Elaboration of first-class traits into disjoint
    intersection types/polymorphism:} We show how the semantics of
  \name can be defined by elaboration into Alpuim et al.'s
  \bname~\cite{alpuimdisjoint}. The elaboration is inspired by the
  work of Cook and Palsberg~\cite{cook1989denotational} to model inheritance.

%\item {\bf Case Study and Improved variants of extensible designs:} We present
%  improved variants of \emph{Object
%    Algebras} and \emph{Extensible Visitors} in \name.

\item {\bf Implementation and modularization case study:} \name is implemented
  and available.\footnote{The implementation, case study code and proofs are available at \url{https://goo.gl/uFrWkr}.} To
  evaluate \name we conduct a case study. The case study shows that support for
  composition of Object Algebras and \textsc{Visitor}s is greatly improved in \name.
  Using such improved design patterns we re-code the interpreters in
  Cook's undergraduate Programming Languages book~\cite{poplcook} in
  a modular way in \name.

\end{itemize}

% \begin{comment}

% \paragraph{\bf JavaScript-style Mixin-Based Programming}
% A common programming pattern in JavaScript is based on a variant of
% Mixins. This programming style is very flexible and enables forms
% of reuse not usually available in more statically typed languages like Java.
% However, mixins in JavaScript fundamentally rely on an \emph{object-level composition}
% operator for inheritance~\cite{}. This requires a very dynamic form of
% inheritance/delegation that is not available in most class-based
% statically-typed OO languages. Ideally the essence of such
% form of mixins should be capturable in statically-typed languages.
% Languages such as TypeScript do attempt to provide better static
% type-checking support for those patterns. However, as recently illustrated
% by the work of \cite{alpuimdisjoint}, there are several issues with such an
% approach, including type-unsoundness!

% \end{comment}


%%% Local Variables:
%%% mode: latex
%%% TeX-master: "../paper"
%%% org-ref-default-bibliography: ../paper.bib
%%% End:
