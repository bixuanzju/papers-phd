
\section{Background}
\label{sec:background}

This section recaps the terminologies~\cite{oliveira2016disjoint,
  alpuimdisjoint} that are used throughout the paper in case the readers are not
familiar with the type system (\bname with some extensions) behind \name. In
particular, we show that \name supports (disjoint) intersection types, the merge
operator, parametric polymorphism and extensible records.

\subsection{(Disjoint) Intersection Types and the Merge Operator}
\label{sec:intersection}

Intersection types date back as early as Coppo et al.'s
work~\cite{coppo1981functional}. Since then researchers have studied
intersection types, and some languages have adopted them in one form or another.

In Java, for examples,
\begin{lstlisting}[language=java]
  interface AwithB extends A, B {}
\end{lstlisting}
introduces a new interface \lstinline$AwithB$ that satisfies the interface of
both \lstinline{A} and \lstinline{B}. In Scala, given two concrete traits, it is
possible to use \textit{mixin composition} to create an object that implements
both traits.
\begin{lstlisting}[language=scala]
  trait A
  trait B
  val newAB : A with B = new A with B
\end{lstlisting}
Scala also allows intersection of type parameters. For example,
\begin{lstlisting}[language=scala]
  def merge[A,B] (x: A) (y: B) : A with B = ...
\end{lstlisting}
uses the anonymous intersection of two type parameters \lstinline{A} and
\lstinline{B}.

However, in Scala it is not possible to dynamically compose two objects. For
example, the following code:
\begin{lstlisting}[language=scala]
  // Invalid Scala code:
  def merge[A,B] (x: A) (y: B) : A with B = x with y
\end{lstlisting}
is rejected by the Scala compiler. The problem is that the \lstinline$with$
construct for Scala expression can only be used to mixin traits or classes, not
arbitrary objects. This limitation essentially puts intersection types in Scala
in a second-class status. Although \lstinline{merge} returns an intersection
type, it is hard to build values with such types.

To address the limitation of intersection types in language like Scala, \name
takes the approach of a particular formulation, where intersection types are
introduced by a \textit{merge operator} (denoted by \lstinline{,,}). As
~\citet{dunfield2014elaborating} has argued, a merge operator adds considerable
expressiveness to a calculus. With the merge operator, it is trivial to
implement the \lstinline{merge} function in \name\footnote{Note that this is not
  the correct definition in \name, see Section~\ref{sec:polymorphism} for
  disjoint quantification.}:
\begin{lstlisting}
  def merge A B (x: A) (y: B) : A & B = x ,, y
\end{lstlisting}
In \name, type variables use capitalized names, while term variables use
lowercase names. In contrast to Scala's expression-level \lstinline{with}
construct, the merge operator \lstinline{,,} allows two arbitrary values to be
merged. The resulting type is an intersection type (\lstinline{A & B} in this
case).

\subsubsection{Incoherence and Disjointness}

Unfortunately the implicit nature of elimination for intersection types built
with a merge operator can lead to incoherence. For a language to be coherent, it
is required that any \textit{valid program} has exactly one
meaning~\cite{reynolds1991coherence}. The naive addition of a merge operator
would allow values of overlapping types to be merged. For example, what should
be the result of the following program, which asks for an integer out of a merge
of two integers:
\begin{lstlisting}
  (1 ,, 2) : Int
\end{lstlisting}
Should the result be 1 or 2?

Following the work of~\citet{oliveira2016disjoint}, in \name, two types can be
merged if and only if they are \textit{disjoint}. Disjointness, in its simplest
form, means that the set of values of both types are disjoint. With the
disjointness requirement, \lstinline{1 ,, 2} is rejected by the type system,
while \lstinline{1 ,, 'c'} is accepted.

As for the above \lstinline{merge} function, since it is a polymorphic function,
it is unknown whether the instantiated types of \lstinline{A} and \lstinline{B}
are disjoint or not. It is still possible that incoherence may occur, as shown
in the following program:
\begin{lstlisting}
  (merge Int Int 1 2) : Int
\end{lstlisting}
To avoid incoherence in such circumstances, \name also employs the notion of
disjoint quantification, as explained in Section~\ref{sec:polymorphism}.

\subsubsection{Parametric Polymorphism and Intersection Types}
\label{sec:polymorphism}

Inspired by the work of~\citet{alpuimdisjoint}, \name uses an extension to
universal quantification called \textit{disjoint quantification}, where a type
variable can be constrained so that it is disjoint with a given type. With
disjoint quantification, the correct version of \lstinline{merge}, which is
accepted in \name, is written as:
\begin{lstlisting}
  def merge A [B * A] (x: A) (y: B) : A & B = x ,, y
\end{lstlisting}

The only difference with the previous version lies in the declaration of the
type parameter \lstinline{B}. The notation \lstinline{B * A} means that the type
variable \lstinline{B} is constrained so that it can only be instantiated with
any type disjoint to \lstinline{A}. Thus the problematic use of
\lstinline{merge} as in \lstinline$merge Int Int 1 2$ is rejected because
\lstinline{Int} is not disjoint with \lstinline{Int}.


\subsection{Extensible Records}
\label{sec:records}

Following~\citet{reynolds1997design} and~\citet{castagna1995calculus}, \name
leverages intersection types to type extensible records. The idea is that a
multi-field record can be encoded as merges of single-field records, and
multi-record types as intersections. Therefore in \name, there are only
single-field record constructs. As such, record operations in \name can occur on
\textit{any} type.

\subsubsection{Record Operations}

To illustrate the various operations on records, we consider a record with three
fields:
\begin{lstlisting}
  {open : Int, high : Int, low : Int}
\end{lstlisting}
Note that this type is just syntactic sugar for:
\begin{lstlisting}
  {open : Int} & {high : Int} & {low : Int}
\end{lstlisting}
That is, a multi-field record type is desugared as intersections of single-field
record types.

\name supports three primitive operations related to records:
\textit{construction} \textit{selection} and \textit{restriction}.
\textit{Extension}, described in many other record systems, is delegated to the
merge operator. Working with records is type-safe: the type system prevents
accessing a field that does not exist.

\paragraph{Record construction.} The usual notation for constructing records
\begin{lstlisting}
  {open = 192, high = 195, low = 189}
\end{lstlisting}
is a shorthand for merges of single-field records
\begin{lstlisting}
  {open = 192} ,, {high = 195} ,, {low = 189}
\end{lstlisting}

\paragraph{Record selection.} Fields are extracted using the dot notation. For
example,
\begin{lstlisting}
  {open = 192, high=195, low= 189}.open
\end{lstlisting}
selects the value of the field labelled \lstinline{open} from the record. Since
records are just merges of terms, we can even select a field that is buried deep
inside a merge, so long as it is present.
\begin{lstlisting}
  ({open = 192} ,, 3 ,, {low = 189}).open
\end{lstlisting}

\jeremy{say something about record restriction}

\paragraph{Record extension} Extension, just like construction, is implemented
with the merge operator. The following example adds a \lstinline{close} field to
the record:
\begin{lstlisting}
  {open = 192, high=195, low=189} ,, {close = 195}
\end{lstlisting}


\begin{comment}
\subsubsection{Restriction via Subtyping}

Unlike most record systems, restriction is not a primitive operation in \name.
Instead, \name uses subtyping for restriction. Combined with disjoint
quantification, we can encode a \lstinline{remove} function that removes a given
field from a record:
\lstinputlisting[linerange=-]{}% APPLY:linerange=RCD_DEF
\lstinline{remove} takes a value x which contains a record of type
\lstinline${low : Int}$ as well as some extra information of type \lstinline{B}.
The disjointness constraint ensures that the value of type \lstinline{B} does
not contain a record with type \lstinline${low : Int}$. The following examples
shows removing the \lstinline{low} field:
\lstinputlisting[linerange=-]{}% APPLY:linerange=RCD_EG
\end{comment}


\subsubsection{Disjointness of Records}

Most record calculi forbid duplicate labels in the declarations of record types.
Some allow labels coincide but the last field overrides the previous ones.
Records in \name allow duplicate labels. This is because we adopt a lenient
approach to record disjointness~\cite{alpuimdisjoint}. Of course records with
distinct fields are disjoint naturally. \name accepts duplicate labels as long
as the types of the overlapping fields are disjoint. For example,
\begin{lstlisting}
  {open = 192, high = 195, open = true}
\end{lstlisting}
is allowed in \name. An interesting question arises when we try to select a
duplicate label, say \lstinline{open}. What should be the result? Should it be
192 or \lstinline{true}? Neither choice is satisfying, as with the coherence
problem for merges. Instead \name rejects such expression, and asks for more
type information from the context. Thus the following is accepted because there
is only one \lstinline{open} associated with \lstinline{Int}.
\begin{lstlisting}
  {open = 192, high = 195, open = true}.open + 3
\end{lstlisting}
