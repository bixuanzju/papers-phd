\section{Typed First-Class Traits}
\label{sec:traits}

In \cref{sec:overview} we have seen some examples of first-class traits at work
in \name. In this section we give a detailed account of \name's support for
typed first-class traits, to complement what has been presented so far. In doing so,
we simplify the examples in \cref{sec:overview} to make better use of traits.
\Cref{sec:typesystem} presents the formal type system of first-class traits.

\subsection{Traits in \name}

\name supports a simple, yet expressive form of traits~\cite{scharli2003traits}.
Traits provide a simple mechanism for find-grained code reuse, which
can be regarded as a disciplined form of multiple inheritance. A trait is
similar to a mixin in that it encapsulates a collection of related methods to be
added to a class. The
practical difference between traits and mixins is the way conflicting features
that typically arise in multiple inheritance are dealt with. Instead of
automatically resolved by scoping rules, conflicts are, in \name,
detected by the type system, and explicitly resolved by the programmer. Compared
with traditional trait models, there are three interesting points about
\name's traits: (1) they are \emph{statically typed}; (2) they are
\emph{first-class} values; and (3) they support \emph{dynamic inheritance}. The
support for such combination of features is one of the key novelties of \name.
Another minor difference from traditional traits (e.g., in Scala) is that,
due to the use of structural types, a trait name is not a type.
%We have seen some examples of trait declarations in
%\cref{sec:overview}. The following discussions illustrate some
%fundamental features of \name's traits.

\subsection{Two Roles of Traits in \name}

\subparagraph{Traits as Templates for Creating Objects.} An obvious difference between
traits in \name and many other models of
traits~\cite{scharli2003traits,fisher2004typed,odersky2005scalable} is that they
directly serve as templates for objects. In many other trait models, traits are
complemented by classes, which take the responsibility for object creation. In
particular, most models of traits do not allow constructors for traits. However,
a trait in \name has a single constructor of the same name. Take our last trait
\lstinline{ide_editor} in \cref{sec:overview} for example:
\lstinputlisting[linerange=64-64]{../../examples/overview2.sl}% APPLY:linerange=EDITOR_INST
As with conventional OO languages, the keyword \lstinline{new} is used to create
an object. A difference to other OO languages is that the keyword
\lstinline{new} also specifies the intended type of the object. We
instantiate the \lstinline{ide_editor} trait and create an object
\lstinline{a_editor1} of type \lstinline{IDEEditor}. As we will see in
\cref{subsec:cons}, constructors with parameters can also be expressed.

It is tempting to try to instantiate the \lstinline{editor} trait such as
\lstinline{new[Editor] editor}. However this results in a type error, because as
we discussed, \lstinline{editor} has no definition of \lstinline{version}, and
blindly instantiating it would cause runtime error. This behaviour is on a par
with Java's abstract classes -- traits with undefined methods cannot be
instantiated on their own.

\subparagraph{Traits as Units of Code Reuse.}
The traditional role of traits is to serve as units of code reuse. \name's traits
can have this role as well.
Our \lstinline{spell_mixin} function in \cref{sec:overview} is more complicated than it should be.
This is because we were mimicking classes as traits, and
mixins as functions over traits. Instead, traits already provide a mechanism of
code reuse. To illustrate this, we simplify \lstinline{spell_mixin} as follows:
\lstinputlisting[linerange=87-90]{../../examples/overview2.sl}% APPLY:linerange=HELP
This is much cleaner. The trait \lstinline{spell} adds a method
\lstinline{check}. It also defines a method \lstinline{on_key}.
A key difference with \lstinline{spell_mixin} is that \lstinline{on_key} is invoked on the \lstinline{self}
parameter instead of \lstinline{super}. Note that this does not necessarily mean \lstinline{check} will call \lstinline{on_key}
defined in the same trait. As we will see, the actual behaviour entirely depends on how we compose \lstinline{spell}
with other traits. One minor difference is that we do not need to tag \lstinline{on_key}
with the \lstinline{override} keyword, because \lstinline{spell} stands as a standalone entity.
Another interesting point is that the self-type \lstinline{OnKey}
is not the same as that of the trait body, which also contains the \lstinline{check} method.
In \name, self-types of traits are known as trait \emph{requirements}.


\subparagraph{Classes and/or Traits}

In the literature on traits~\cite{Ducasse_2006, scharli2003traits}, the
aforementioned two roles are considered as competing. One reason of the two
roles conflicting in class-based languages is because a class must adopt a fixed
position in the class hierarchy and therefore it can be difficult to reuse and
resolve conflicts, whereas in \name, a trait is a standalone entity and is not
tied to any particular hierarchy. Therefore we can view our traits either as
generators of instances, or units of reuse. Another important reason why our
model can do just with traits is because we have a pure language. Mutable state
can often only appear in classes in imperative models of traits, which is a good
reason for having both classes and traits.



% \lstinline{version} method is not defined. Like mixins, \lstinline{help} can be
% combined with other traits to produce several combinations of functionality. For
% instance, we create another editor that inherits two traits \lstinline{editor}
% and \lstinline{help}.
% \lstinputlisting[linerange=75-76]{../../examples/overview2.sl}% APPLY:linerange=HELP2
% Due to the lack of multiple inheritance in JavaScript, we were forced to use
% mixins. In \name, this can be easily achieved because traits support multiple
% inheritance. In general, \lstinline{inherits ...} can take one or more trait
% expressions (delimited by \lstinline{&}).


\subsection{Trait Types and Trait Requirements}

\subparagraph{Object Types and Trait Types.}
\name adopts a relatively standard foundational model of object-oriented
constructs~\cite{DBLP:conf/ecoop/LeeASP15} where objects are encoded as records
with a structural type. This is why the type of the object \lstinline{a_editor1}
is the record type \lstinline{IDEEditor}. In \name, an object type is
different from a trait type. A trait type is specified with the keyword
\lstinline{Trait}. For example, the type of the \lstinline{spell} trait is
\lstinline{Trait[OnKey, OnKey & Spelling]}. 

\subparagraph{Trait Requirements and Functionality.}
In general, a trait type
\lstinline{Trait[T1, T2]} specifies both the \emph{requirements} \lstinline{T1}
and the \emph{functionality} \lstinline{T2} of a trait. The requirements of a trait denote the types/methods that the
trait needs to support for defining the functionality it provides. Both are
reflected in the trait type. For example, \lstinline{spell} has type
\lstinline{Trait[OnKey, OnKey & Spelling]}, which means that \lstinline{spell}
requires some implementation of the \lstinline{on_key} method, and it provides
implementations for the \lstinline{on_key} and \lstinline{check} methods.
When a trait
has no requirements, the absence of a requirement is denoted by using
the top type (\lstinline{Top}). A simplified sugar \lstinline{Trait[T]} is
used to denote a trait without requirements, but providing functionality \lstinline{T}.

\subparagraph{Trait Requirements as Abstract Methods.}
Let us go back to our very first trait \lstinline{editor}. Note how in
\lstinline{editor} the type of the \lstinline{self} parameter is
\lstinline{Editor & Version}, where \lstinline{Version} contains a declaration of the
\lstinline{version} method that is needed for the definition of
\lstinline{show_help}. Note also that the trait itself does not actually contain
a \lstinline{version} definition. In many other OO models a similar program
could be achieved by having an \emph{abstract} definition of
\lstinline{version}. In \name there are no abstract definitions (methods or
fields), but a similar result can be achieved via trait requirements.
Requirements of a trait are met at the object creation point. For example, as we
mentioned before, the \lstinline{editor} trait alone cannot be instantiated
since it lacks \lstinline{version}. However, when it is composed with a trait that
provides \lstinline{version}, the composition can be instantiated, as shown below:
\lstinputlisting[linerange=75-76]{../../examples/overview2.sl}% APPLY:linerange=HELP2
\name uses a syntax where the self parameter
can be explicitly named (not necessarily named \lstinline{self}) with a
type annotation. When the self parameter is omitted (for example in the
\lstinline{foo} trait above), its type defaults to \lstinline{Top}. This
is different from typical OO languages, where the default type of the self
parameter is the same as the class being defined.



\subparagraph{Intersection Types Model Subtyping.}
\lstinline{IDEEditor} is defined as an intersection type (\lstinline{Editor & Version & Spelling & ModalEdit}).
Intersection types~\cite{coppo1981functional,pottinger1980type} have been woven
into many modern languages these days. A notable example is Scala, which makes
fundamental use of intersection types to express a class/trait that extends
multiple other traits. An intersection type such as \lstinline{T1 & T2} contains
exactly those values which can be used as values of type \lstinline{T1} and of
type \lstinline{T2}, and as such, \lstinline{T1 & T2} immediately introduces a
subtyping relation between itself and its two constituent types \lstinline{T1}
and \lstinline{T2}. Unsurprisingly, \lstinline{IDEEditor} is a subtype of
\lstinline{Editor}.


% \subparagraph{Composition of traits}


% The definition of the object \lstinline{my_editor2} also shows the second way to
% introduce inheritance, namely by \emph{composition} of traits. Composition of
% traits is denoted by the operator \lstinline{&}. Thus \name offers two options
% when it comes to inheritance: we can either compose beforehand when declaring
% traits (using \lstinline{inherits}), or compose at the object creation point
% (using \lstinline{new} and the \lstinline{&} operator).

%Under the hood, inheritance is accomplished by using the \emph{merge operator}
%(denoted by \lstinline{,,}). The merge operator~\cite{dunfield2014elaborating}
%allows two arbitrary values to be merged, with the resulting type being an
%intersection type.
%For example the type of \lstinline{2 ,, true} is
%\lstinline{Int & Bool}.
\begin{comment}
\subparagraph{Mutually dependent traits} When two traits are composed, any two
methods in those two traits can refer to each other via the self-reference. We
say these two traits are \emph{mutually dependent}. The next example, though a
bit contrived, illustrates this point.
\lstinputlisting[linerange=-]{}% APPLY:linerange=EVEN_ODD

\noindent By utilizing trait requirements, the \lstinline{isEven} and
\lstinline{isOdd} methods can refer to each other in two different traits.
\end{comment}


\subsection{Traits with Parameters and First-Class Traits}\label{subsec:cons}

So far our uses of traits involve no parameters. Instead of inventing another trait
syntax with parameters, a trait with parameters is just a function that produces
a trait expression, since functions already have parameters of their own. This
is one benefit of having first-class traits in terms of language economy. To
illustrate, let us simplify \lstinline{modal_mixin} in a similar way as in \lstinline{spell_mixin}:
\lstinputlisting[linerange=94-98]{../../examples/overview2.sl}% APPLY:linerange=MODAL2
The first thing to notice is that \lstinline{modal} is a function with one
argument, and returns a trait expression, which essentially makes
\lstinline{modal} a trait with one parameter.
Now it is easy to see that a trait declaration
\lstinline$trait name [self : ...] => {...}$ is just syntactic sugar for
function definition \lstinline$name = trait [self : ...] => {...}$. The body of
the \lstinline{modal} trait is straightforward. We initialize the
\lstinline{mode} field to \lstinline{init_mode}.
The \lstinline{modal} trait also comes with a constructor with one parameter,
so we can do \lstinline{new[ModalEdit] (modal "insert")} for example.

\subsection{Detecting and Resolving Conflicts in Trait Composition}

A common problem in multiple inheritance is how to detect and/or resolve conflicts. For example, when
inheriting from two traits that have the same field, then it is unclear which
implementation to choose. There are various approaches to dealing with
conflicts. The trait-based approach requires conflicts to be resolved at the
level of the composition, otherwise the program is rejected by
the type system. \name provides a means to help resolve conflicts.

We start by assembling all the traits defined in this section
to create the final editor with the same functionality as
\lstinline{ide_editor} in \cref{sec:overview}. Our first try is as follows:
\lstinputlisting[linerange=108-110]{../../examples/overview2.sl}% APPLY:linerange=MODAL_CONFLICT
Unfortunately the above trait gets rejected by \name because
\lstinline{editor}, \lstinline{spell} and \lstinline{modal} all define an \lstinline{on_key} method.
Recall that in \cref{sec:overview}, when we use a mixin-style composition,
the conflict resolution code has been hardwired in the definition.
However, in a trait-style composition, this is not the case: conflicts must be resolved \emph{explicitly}.
The
above definition is ill-typed precisely because there is a conflicting
method \lstinline{on_key}, thus violating the disjointness conditions
of disjoint intersection types.

\subparagraph{Resolving Conflicts.}
To resolve the conflict, we need to explicitly state which
\lstinline{on_key} gets to stay. \name provides such a means, the so-called
\emph{exclusion} operator (denoted by \lstinline$\$), which allows one to
exclude a field/method from a given trait. The following matches the behaviour
in \cref{sec:overview} where \lstinline{on_key} in the \lstinline{modal} trait
is selected:
\lstinputlisting[linerange=117-120]{../../examples/overview2.sl}% APPLY:linerange=MODAL_OK
Now the above code type checks. We can also select \lstinline{on_key} in the \lstinline{spell} trait as easily:
\lstinputlisting[linerange=125-128]{../../examples/overview2.sl}% APPLY:linerange=MODAL_OK2
In \cref{sec:overview} we mentioned that in the mixin style, it is impossible
to select \lstinline{on_key} in the \lstinline{editor} trait, but this is not a problem here:
\lstinputlisting[linerange=132-135]{../../examples/overview2.sl}% APPLY:linerange=MODAL_OK3


% Moreover, we can reason that the version number
% in \lstinline{modal_editor} is \lstinline{0.2}.


% On a side note, leaving out the keyword \lstinline{override}
% also results in a conflict because \lstinline{editor} also contains
% \lstinline{on_key}. Whenever the type system sees \lstinline{override}, it will
% first try to check if the inherit clause contains the method to be overridden,
% and if so, insert the exclusion operator, otherwise a type error would
% occur.

% \subparagraph{The Keyword Override}

% Since it is very common that we want to exclude some methods from the
% inherited traits, and redefine them in the body, \name provides the keyword
% \lstinline{override} to achieve this. For example, suppose in
% \lstinline{editor3} we want to define a new
% \lstinline{version_num} in the body instead of the one from
% \lstinline{modal_edit}, we can of course exclude
% \lstinline{version_num} from \lstinline{modal_edit} and define a new
% \lstinline{version_num} in the body. We can also use the keyword
% \lstinline{override} to achieve the same:
% \lstinputlisting[linerange=125-128]{../../examples/overview2.sl}% APPLY:linerange=MODAL_OK2
% That is, by labeling a method as a overriding method, the type system will
% automatically exclude this method from the inherited traits. Note that if the
% overriding method is not present in the inherited traits, a type error would
% occur.

% \subparagraph{The Keyword Super}

% What if we want to invoke the overridden method in the overriding method. As
% with convectional OO languages, \name also provides the keyword
% \lstinline{super} to achieve this. In the following, we access the overridden
% field \lstinline{version_num} from \lstinline{modal_edit} through the use of
% the keyword \lstinline{super}:
% \lstinputlisting[linerange=132-135]{../../examples/overview2.sl}% APPLY:linerange=MODAL_OK3

\subparagraph{The Forwarding Operator.}
Another operator that \name provides is the so-called \emph{forwarding}
operator, which can be useful when we want to access some method that has been
explicitly excluded in the \lstinline{inherits} clause. This is a common scenario in
diamond inheritance, where \lstinline{super} is not enough. Below we show a
variant of \lstinline{ide_editor}:
\lstinputlisting[linerange=140-147]{../../examples/overview2.sl}% APPLY:linerange=MODAL_WIRE
Notice that \lstinline{on_key} in \lstinline{spell} has been
excluded. However, we can
still access it by using the forwarding operator as in \lstinline{spell ^ self},
which gives full access to all the methods in \lstinline{spell}. Also note that
using \lstinline{super} only gives us access to \lstinline{on_key} in the
\lstinline{modal} trait. To see \lstinline{ide_editor4} in action, we create a
small test:
\lstinputlisting[linerange=151-153]{../../examples/overview2.sl}% APPLY:linerange=MODAL_USE
% \jeremy{Compare this to trait alias operator?}
% \jeremy{would it be better to use forwarding operator in mix-style \lstinline{ide_editor} in
%   \cref{sec:overview}, where we show how to access \lstinline{on_key} from editor, which is impossible in JavaScript?}

% Since the result editor trait has such an exciting feature, we increment the version number to \lstinline{0.2}!



\subsection{Disjoint Polymorphism and Dynamic Composition}
\label{sec:merge}

\begin{comment}
One important difference to traditional traits or classes is that traits in
\name are quite dynamic: we are able to compose traits \emph{dynamically} and
then instantiate them later. We have seen an example of dynamic inheritance in
\cref{sec:overview} to mimic mixins, where disjoint polymorphism plays an
important role. This is impossible in traditional OO languages, such as Scala,
since classes being inherited/instantiated must be known statically.

In Scala, given two concrete traits, it is possible to use \emph{mixin composition} to
create an object that implements both traits.
\begin{lstlisting}[language=scala]
trait A
trait B
val newAB : A with B = new A with B
\end{lstlisting}
Here the type \lstinline[language=scala]{A with B} is an intersection type,
and the expression \lstinline[language=scala]{new A with B} allows creating a new
object from the combination of the traits \lstinline{A} and
\lstinline{B}.
However, in Scala it is not possible to dynamically compose two
(statically unknown) \emph{objects}. For example, the following code: \jeremy{how is this useful to have?}
\begin{lstlisting}[language=scala]
// Invalid Scala code:
def merge[A,B] (x: A) (y: B) : A with B = new x with y
\end{lstlisting}
is rejected by the Scala compiler. The problem is that the
\lstinline[language=scala]{with} construct in Scala can only be used to mixin
(statically known) traits or classes, not arbitrary objects. This limitation essentially puts
intersection types in Scala in a second-class status.
\end{comment}

\name supports disjoint polymorphism. The combination of disjoint
polymorphism and first-class traits enables the highly modular code
where traits with \emph{statically unknown} types can be instantiated
and composed in a type-safe way! The following is illustrative of this:
%However, this is not a problem in \name. Thanks to disjoint polymorphism and
%disjoint intersection types, we can define the same \lstinline{merge} that is able to take two
%traits (where the full set of the members may not be known statically), combine and instantiate them.
\lstinputlisting[linerange=76-76]{../../examples/overview.sl}% APPLY:linerange=MERGE
The \lstinline{merge} function takes two traits \lstinline{x} and \lstinline{y} of
some arbitrary types \lstinline{A} and \lstinline{B}, composes them,
and instantiates an object with the resulting composed trait. Clearly
such composition cannot always work if \lstinline{A} and
\lstinline{B} can have conflicts. However, \lstinline{merge} has a
constraint \lstinline{B * A} that ensures that whatever types are used
to instantiate \lstinline{A} and \lstinline{B} they must be disjoint.
Thus, under the assumption that \lstinline{A} and \lstinline{B} are
disjoint the code type-checks. We want to emphasize that row polymorphism is unable to express
this kind of disjointness of two polymorphic types, thus systems using row polymorphism is unable to define
the \lstinline{merge} function, which plays an essential role in \cref{sec:application}.

\begin{comment}
Notice the disjoint constraint on the type variable \lstinline{B} (\lstinline{B * A}), which is crucial
to ensure the composition of two traits \lstinline{x} and \lstinline{y} is conflict-free.
As far as we are aware, no mainstream statically typed OO languages can do this.
\end{comment}


%%% Local Variables:
%%% mode: latex
%%% TeX-master: "../paper"
%%% org-ref-default-bibliography: ../paper.bib
%%% End:
