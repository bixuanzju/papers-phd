
\documentclass[a4paper,UKenglish]{lipics-v2018}
%This is a template for producing LIPIcs articles. 
%See lipics-manual.pdf for further information.
%for A4 paper format use option "a4paper", for US-letter use option "letterpaper"
%for british hyphenation rules use option "UKenglish", for american hyphenation rules use option "USenglish"
% for section-numbered lemmas etc., use "numberwithinsect"


\nolinenumbers

\usepackage{microtype}%if unwanted, comment out or use option "draft"


% AMS packages
\usepackage{amsmath}
\usepackage{amssymb}

\usepackage{mathtools}
\usepackage{mdwlist}
\usepackage{pifont}


% Miscellaneous
\usepackage{paralist}
\usepackage{graphicx}
\usepackage{epstopdf}
\usepackage{float}
\usepackage{longtable}
\usepackage{multirow}
\usepackage{lscape}


% Revision tools
\usepackage{xspace}
\usepackage{comment}
\newcommand\mynote[3]{}

\newcommand{\hl}[2][gray!40]{\colorbox{#1}{#2}}
\newcommand{\hlmath}[2][gray!40]{\colorbox{#1}{$\displaystyle#2$}}
\newcommand{\otthl}[2][gray!40]{ \colorbox{#1}{$\displaystyle#2$}}


% Graphs
\usepackage{tikz}
\usetikzlibrary{matrix}
\usetikzlibrary{arrows,automata}
\usetikzlibrary{positioning}


% Hyper links
\usepackage{url}
\usepackage{
  nameref,%\nameref
  hyperref,%\autoref
}
\usepackage[capitalise]{cleveref}

\usepackage{thmtools, thm-restate}

\usepackage[misc]{ifsym}

\declaretheorem[name=$\mathcal{L}$emma,
  % numberwithin=section,
  refname={$\mathcal{L}$emma,$\mathcal{L}$emmas},
  Refname={$\mathcal{L}$emma,$\mathcal{L}$emmas}]{clemma}

\declaretheorem[name=$\mathcal{T}$heorem,
  % numberwithin=section,
  refname={$\mathcal{T}$heorem,$\mathcal{T}$heorems},
  Refname={$\mathcal{T}$heorem,$\mathcal{T}$heorems}]{ctheorem}

\declaretheorem[name=Observation]{observation}

\declaretheorem[name=Proof]{Proof}


\usepackage{rotating}

\usepackage{ottalt}

\renewcommand\ottaltinferrule[4]{
  \inferrule*[right=\scriptsize{#1}]
    {#3}
    {#4}
}

\input{macros/pl-theory.tex}

\newcommand{\name}{\textsf{NeColus}\xspace}
\newcommand{\namee}{$\lambda_{i}^{+}$\xspace}
\newcommand\oname{$\lambda_{i}$\xspace}
\newcommand\fname{$F_{i}$\xspace}
\newcommand\tname{$\lambda_{c}$\xspace}

% Ott includes
\inputott{ott-rules}
% I prefer rulenames on the right
% \renewcommand\ottaltinferrule[4]{
%   \inferrule*[narrower=0.7,right=#1,#2]
%     {#3}
%     {#4}
% }
\renewcommand\ottaltinferrule[4]{
  \inferrule*[narrower=0.5,lab=#1,#2]
    {#3}
    {#4}
}

% \renewcommand{\ottnt}[1]{#1}
% \renewcommand{\ottmv}[1]{#1}
% \renewcommand{\subparagraph}[1]{\vspace{3pt}\noindent{\bf #1}}

% Logical equivalence related macros
\newcommand{\valR}[2]{\mathcal{V}\bra{#1 ; #2}}
\newcommand{\valRR}[1]{\mathcal{V}\bra{#1}}
\newcommand{\eeR}[2]{\mathcal{E}\bra{#1 ; #2}}
\newcommand{\eeRR}[1]{\mathcal{E}\bra{#1}}
\newcommand{\ggR}[2]{\mathcal{G}\bra{#1 ; #2}}

\newcommand{\hll}[2][gray!40]{\colorbox{#1}{#2}}
\newcommand{\hlmath}[2][gray!40]{%
  \colorbox{#1}{$\displaystyle#2$}}


%\graphicspath{{./graphics/}}%helpful if your graphic files are in another directory

\bibliographystyle{plainurl}% the recommnded bibstyle

\title{The Essence of Nested Composition}

% \titlerunning{Dummy short title}%optional, please use if title is longer than one line

\author{Xuan Bi$^1$}{The University of Hong Kong, Hong Kong, China}{xbi@cs.hku.hk}{}{}%mandatory, please use full name; only 1 author per \author macro; first two parameters are mandatory, other parameters can be empty.

\author{Bruno C. d. S. Oliveira}{The University of Hong Kong, Hong Kong, China}{bruno@cs.hku.hk}{}{Funded by Hong Kong Research Grant Council projects number 17210617 and 17258816}

\author{Tom Schrijvers}{KU Leuven, Belgium}{tom.schrijvers@cs.kuleuven.be}{}{Funded by The Research Foundation - Flanders}

\authorrunning{X.\,Bi, B.\,C.\,d.\,S.\,Oliveira and T.\,Schrijvers} %mandatory. First: Use abbreviated first/middle names. Second (only in severe cases): Use first author plus 'et. al.'

\Copyright{Xuan Bi, Bruno C. d. S. Oliveira and Tom Schrijvers}%mandatory, please use full first names. LIPIcs license is "CC-BY";  http://creativecommons.org/licenses/by/3.0/


\subjclass{Software and its engineering $\rightarrow$ Object oriented languages}% mandatory: Please choose ACM 2012 classifications from https://www.acm.org/publications/class-2012 or https://dl.acm.org/ccs/ccs_flat.cfm . E.g., cite as "General and reference $\rightarrow$ General literature" or \ccsdesc[100]{General and reference~General literature}.

\keywords{nested composition, family polymorphism, intersection types, coherence}%mandatory

\category{}%optional, e.g. invited paper

\relatedversion{}%optional, e.g. full version hosted on arXiv, HAL, or other respository/website

\supplement{}%optional, e.g. related research data, source code, ... hosted on a repository like zenodo, figshare, GitHub, ...

\funding{}%optional, to capture a funding statement, which applies to all authors. Please enter author specific funding statements as fifth argument of the \author macro.

\acknowledgements{We thank the anonymous reviewers for their helpful comments.}%optional

%Editor-only macros:: begin (do not touch as author)%%%%%%%%%%%%%%%%%%%%%%%%%%%%%%%%%%
\EventEditors{Todd Millstein}
\EventNoEds{1}
\EventLongTitle{32nd European Conference on Object-Oriented Programming (ECOOP 2018)}
\EventShortTitle{ECOOP 2018}
\EventAcronym{ECOOP}
\EventYear{2018}
\EventDate{July 16--21, 2018}
\EventLocation{Amsterdam, Netherlands}
\EventLogo{}
\SeriesVolume{109}
\ArticleNo{22} % “New number” (=<article-no>) goes here!%\nolinenumbers %uncomment to disable line numbering
%\hideLIPIcs  %uncomment to remove references to LIPIcs series (logo, DOI, ...), e.g. when preparing a pre-final version to be uploaded to arXiv or another public repository
%%%%%%%%%%%%%%%%%%%%%%%%%%%%%%%%%%%%%%%%%%%%%%%%%%%%%%

\begin{document}

\maketitle

\begin{abstract}
Calculi with \emph{disjoint intersection types} support an introduction form
for intersections called the
\emph{merge operator}, while retaining a \emph{coherent} semantics.
% The interesting feature of such calculi is that they
%retain a coherent semantics, which is known to be hard in the presence
%of the merge operator.
Disjoint intersections types have great potential to
serve as a foundation for powerful, flexible and yet type-safe and
easy to reason OO languages. This paper shows how to significantly
increase the expressive power of disjoint intersection types by
adding support for \emph{nested subtyping and composition}, which
enables simple forms of \emph{family polymorphism} to be expressed in the calculus.
%The motivation for those features is Ernst's \emph{family polymorphism}: the idea
%that inheritance can be extended from a single class, to a whole
%family of classes. Nested subtyping and composition enable simple forms
%of family polymorphism to be expressed, while retaining type-safety
%and coherence.
The extension with nested subtyping and
composition is challenging, for two different reasons. Firstly, the
subtyping relation that supports these features is non-trivial,
especially when it comes to obtaining an algorithmic version. Secondly,
the syntactic method used to prove coherence for previous calculi with disjoint
intersection types is too inflexible, making it hard to
extend those calculi with new features (such as nested subtyping).
We show how to address the first problem by adapting and extending
 the Barendregt, Coppo and Dezani (BCD) subtyping rules for intersections
with records and coercions. A sound and complete algorithmic
system is obtained by using an approach inspired by Pierce's
work. To address the second
problem we replace the syntactic method to prove coherence,
by a semantic proof method based on \emph{logical relations}.
Our work has been fully formalized in Coq, and we have an implementation
of our calculus.
\end{abstract}


% Main meat
\section{Introduction}

Modern statically typed functional languages (such as ML, Haskell,
Scala or OCaml) have increasingly expressive type systems. Often these
large source languages are translated into a much smaller typed core
language. The choice of the core language is essential to ensure that
all the features of the source language can be encoded. For a simple
polymorphic functional language it is possible to pick a
variant of System $F$~\cite{systemfw,Reynolds:1974} as a core
language. However, the desire for more expressive type system features
puts pressure on the core languages, often requiring them to be
extended to support new features.  For example, if the source language
supports \emph{higher-kinded types} or \emph{type-level functions}
then System $F$ is not expressive enough and can no longer be used as
the core language. Instead another core language that does provide
support for higher-kinded types, such as
System~$F_{\omega}$~\cite{systemfw}, needs to be used. Of course the
drive to add more and more advanced type-level features means that
eventually the core language needs to be extended again. Indeed modern
functional languages like Haskell use specially crafted core
languages, such as System $F_{C}$~\cite{fc}, that provide support for all
modern features of Haskell. Although \emph{extensions} of System
$F_{C}$~\cite{fc:pro,Eisenberg:2014} satisfy the current needs of
modern Haskell, it is very likely to be extended again in the
future~\cite{fc:kind}. Moreover System $F_{C}$ has grown to be a relatively
large and complex language, with multiple syntactic levels, and dozens
of language constructs.

\begin{comment}
However System~$F_{\omega}$ is
significantly more complex than System F and thus harder to
maintain. If later a new feature, such as \emph{kind polymorphism}, is
desired the core language may need to be changed again to account for
the new feature, introducing at the same time new sources of
complexity. Indeed the core language for modern versions of 
functional languages are quite complex, having multiple syntactic 
sorts (such as terms, types and kinds), as well as dozens of 
language constructs~\cite{}\bruno{$F_{C}$}. 
\end{comment}

The more expressive type (and kind) systems become, the more types become similar
to the terms. Therefore a natural idea is to unify terms and
types. There are obvious benefits in this approach: only one syntactic
level (terms) is needed; and there are much less language constructs,
making the core language easier to reason, implement and maintain. At the same
time the core language becomes more expressive, giving us for free
many useful language features. Moreover, due to the inherent
expressiveness, extensions are less likely to be required.
\emph{Pure type systems} (PTS)~\cite{handbook} build
on such observations and show how a whole family of type systems
(including System $F$ and System $F_{\omega}$) can be implemented
using just a single syntactic form. With the added expressiveness it
is even possible to have type-level programs expressed using the same
syntax as terms, as well as dependently typed programs~\cite{coc}.
Because the idea of using a unified syntax is so appealing several
researchers have in the past considered such an
option for implementing functional languages~\cite{cayenne, typeintype, pts:henk}.

However having the same syntax for types and terms can also be
problematic. Usually type systems based on PTS have a conversion rule
to support type-level computation.  In such type systems ensuring the
\emph{decidability} of type checking requires type-level computation
to terminate. When the syntax of types and terms is the same, the
decidability of type checking is usually dependent on the strong
normalization of the calculus. An example is the proof of decidability
of type checking for the \emph{calculus of constructions}~\cite{coc}
(and other normalizing PTS), which depends on strong normalization
~\cite{pts:normalize}. Modern dependently
typed languages such as Idris~\cite{idris} and Agda~\cite{agda}, which are also
built on a unified syntax for types and terms, require strong
normalization as well: all recursive programs must pass a termination
checker.  An unfortunate consequence of coupling
decidability of type checking and strong normalization is that adding
(unrestricted) general recursion to such calculi is difficult. Indeed
past work on using a simple PTS-like calculi to model functional languages
with unrestricted general recursion, had to give up on decidability of
type-checking~\cite{cayenne, typeintype}.
%There
%is a clear tension between decidability of type checking and allowing
%general recursion in calculi with unified syntax.

This paper proposes \name: a simple yet expressive call-by-name
variant of the calculus of constructions, which has a fraction of the
language constructs of existing core languages. The key challenge
solved in this work is how to define a calculus comparable in
simplicity to the calculus of constructions, while featuring both
general recursion and decidable type checking. The main idea, 
inspired by the traditional treatment of \emph{iso-recursive
  types}~\cite{tapl}, is to recover decidable type-checking by making each
type-level computation step explicit, i.e., each beta reduction or
expansion at the type level is controlled by a \emph{type-safe}
cast. Since single computation steps are trivially terminating, decidability
of type checking is possible even in the presence of non-terminating
programs at the type level.  At the same time term-level programs
using general recursion work as in any conventional functional
languages, and can even be non-terminating.

\begin{comment}
For example, if a type-level program requires two beta reductions to
reach normal form, then two casts are needed in the program. If a
non-terminating program is used at the type level, it is not possible
to cause non-termination in the type checker, because that would
require a program with an infinite number of casts. Therefore, since
single beta-steps are trivially terminating, decidability of type
checking is possible even in the presence of non-terminating programs
at the type level.  At the same time term-level programs using general
recursion work as in any conventional functional languages, and can
even be non-terminating.
\end{comment}

Our motivation to develop \name is to use it as a simpler alternative
to existing core languages for functional programming. We focus on traditional
functional languages like ML or Haskell extended with many interesting
type-level features, but perhaps not the \emph{full power} of
dependent types.  The paper shows how many of programming language
features of Haskell, including some of the latest extensions, can be
encoded in \name via a surface language. The surface
language supports \emph{algebraic datatypes}, \emph{higher-kinded
  types}, \emph{nested datatypes}~\cite{nesteddt}, \emph{kind
  polymorphism}~\cite{fc:pro} and \emph{datatype
  promotion}~\cite{fc:pro}.  This result is interesting because \name
is a minimal calculus with only 8 language constructs and a single
syntactic sort. In contrast the latest versions of System $F_{C}$
(Haskell's core language) have multiple syntactic sorts and dozens of
language constructs.
%Even if support for equality and
%coercions, which constitutes a significant part of System $F_{C}$,
%would be removed the resulting language would still be significantly
%larger and more complex than \name.

It is worth emphasizing that \name does sacrifice having an expressive form
of type equality to gain the ability of doing arbitrary general
recursion at the term level.  Nevertheless, 
the core language (System $F_{C}$) of Haskell also comes with a similarly weak
notion of type equality.  In both System $F_{C}$ and \name, type
equality in \name is purely syntactic (modulo alpha-conversion).

A non-goal of the current work (although a worthy avenue for future
work) is to use \name as a core language for modern dependently typed
languages like Agda or Idris. In contrast to \name, those languages
use a more powerful notion of equality. In particular \name
currently lacks full-reduction and it is unable to exploit injectivity 
properties when comparing two types for equality. Moreover,
\name (and also System $F_{C}$) lack \emph{logical consistency}:
that is ensuring the soundness of proofs written as programs.
This is in contrast to dependently typed languages, where logical
consistency is typically ensured.
Various researchers~\cite{zombie:popl14,zombie:thesis,Swamy2011} have been investigating how to combine logical
consistency, general recursion and dependent types. However, this is
usually done by having the type system carefully control the total and
partial parts of computation, making those calculi significantly more
complex than \name or the calculus of constructions. In
\name, logical consistency is traded by the simplicity of the system.

\begin{comment}
In particular
the treatment of type-level computation in \name shares similar ideas
with Haskell. Although Haskell's surface language provides a rich set
of mechanisms to do type-level computation~\cite{}, the core language
lacks fundamental mechanisms todo type-level computation. Type
equality in System $F_{C}$ is, like in \name, purely syntactic (modulo
alpha-conversion).
\end{comment}

\begin{comment}
 and there is no type-level
abstraction. In other words in Haskell, mechanisms such as type
classes and type families

Although it may seem that forcing each step of computation 
at the type-level to be explicit will prevent convinient use of 
type-level computation.

Point about the treatment of type-level computation in Haskell. Haskell's
core language has type applications, but no type-level lambda. Equality 
is syntactic modulo alpha-conversion. This design choice was rooted in the 
desire to support Hindley-Milner type-inference... 
\end{comment}

In summary, the contributions of this work are:

\begin{itemize}
\item {\bf The \name calculus:} A simple core calculus for functional programming, that collapses terms, types and
  kinds into the same hierarchy and supports general recursion. \name
  is type-safe and the type system is decidable.

\item {\bf One-step casts and a generalization of iso-recursive types:} \name 
 generalizes iso-recursive types by making all type-level computation
 steps explicit via \emph{one-step casts}. In \name the combination of
  one-step casts and recursion subsumes iso-recursive types.

\item {\bf An expressive surface language}, built on top of \name,
  that supports datatypes, pattern matching and various advanced
  language extensions of Haskell. The type safety of the type-directed
  translation to \name is proved.

\item {\bf A prototype implementation:} The implementation of \name is
  available\footnote{\url{https://github.com/bixuanzju/full-version}}.
\end{itemize}

\begin{comment}
\begin{enumerate}[a)]
\item Motivations:

\begin{itemize}

\item Because of the reluctance to introduce dependent
  types\footnote{This might be changed in the near future. See
    \url{https://ghc.haskell.org/trac/ghc/wiki/DependentHaskell/Phase1}.},
  the current intermediate language of Haskell, namely System $F_C$
  \cite{fc}, separates expressions as terms, types and kinds, which
  brings complexity to the implementation as well as further
  extensions \cite{fc:pro,fc:kind}.

\item Popular full-spectrum dependently typed languages, like Agda,
  Coq, Idris, have to ensure the termination of functions for the
  decidability of proofs. No general recursion and the limitation of
  enforcing termination checking make such languages impractical for
  general-purpose programming.

\item We would like to introduce a simple and compiler-friendly
  dependently typed core language with only one hierarchy, which
  supports general recursion at the same time.

\end{itemize}

\item Contribution:

\begin{itemize}

\item A core language based on Calculus of Constructions (CoC) that
  collapses terms, types and kinds into the same hierarchy.

\item General recursion by introducing recursive types for both terms
  and types by the same $\mu$ primitive.

\item Decidable type checking and managed type-level computation by
  replacing implicit conversion rule of CoC with generalized
  \textsf{fold}/\textsf{unfold} semantics.

\item First-class equality by coercion, which is used for encoding
  GADTs or newtypes without runtime overhead.

\item Surface language that supports datatypes, pattern matching and
  other language extensions for Haskell, and can be encoded into the
  core language.

\end{itemize}


\end{enumerate}
\end{comment}


% \section{Overview}


% - Shallow embedding in Haskell (2 interpretations);
% 
% - How to compose? possible but lots of boilerplate;
% 
% - Finally tagless solves some problems, but how about dependencies?
% Still some boilerplate needed. 
% 
% - Introduce the solution in our calculus. Show that we can do
% everything finally tagless can + more because we have 
% distributivity in the type system.

%-------------------------------------------------------------------------------
\section{Compositional Programming}
\label{sec:overview}

% \bruno{Do we need something about easily adding new cases? Although
% this is a solved problem, people may wonder about this? Perhaps
% we need some text (1 or 2 sentences) at least. }


To demonstrate the compositional properties of \fnamee we use Gibbons and Wu's shallow embeddings of
parallel prefix circuits~\cite{DBLP:conf/icfp/GibbonsW14}. By means of several different shallow
embeddings, we first illustrate the short-comings of a state-of-the-art
compositional approach, popularly known as a \emph{finally tagless}
encoding~\cite{CARETTE_2009}, in Haskell.
Next we show how parametric polymorphism and distributive intersection types provide
a more elegant and compact solution in \sedel~\cite{bi_et_al:LIPIcs:2018:9214}, a source language built on top of
our \fnamee calculus.


%- - - - - - - - - - - - - - - - - - - - - - - - - - - - - - - - - - - - - - - - 
\subsection{A Finally Tagless Encoding in Haskell}

The circuit DSL represents networks that map a number of inputs (known as the width) of some type $A$ onto
the same number of outputs of the same type. The outputs combine (with repetitions) one or more
inputs using a binary associative operator $\oplus : A \times A \to A$.
A particularly interesting class of circuits that can be expressed in the DSL are
\emph{parallel prefix circuits}. These represent computations that take $n > 0$
inputs $x_1, \ldots, x_n$ and produce $n$ outputs $y_1, \ldots, y_n$, where
$y_i = x_1 \oplus x_2 \oplus \ldots \oplus x_i$.

The DSL features 5 language primitives: two basic circuit constructors and
three circuit combinators. These are captured in the Haskell type class \lstinline[language=haskell]{Circuit}:
\lstinputlisting[language=haskell,linerange=5-10]{./examples/Scan.hs}% APPLY:linerange=DSL_DEF
An \lstinline[language=haskell]{identity} circuit with $n$ inputs $x_i$,  has
$n$ outputs $y_i = x_i$. A \lstinline[language=haskell]{fan} circuit has $n$ inputs $x_i$ and $n$
outputs $y_i$, where $y_1 = x_1$ and $y_j = x_1 \oplus x_j$ ($j > 1)$.
The binary \lstinline[language=haskell]{beside} combinator puts two circuits in parallel; the combined circuit
takes the inputs of both circuits to the outputs of both circuits.
The binary \lstinline[language=haskell]{above} combinator connects the outputs of the first circuit to
the inputs of the second; the width of both circuits has to be same.
Finally,
\lstinline[language=haskell]{stretch ws c} interleaves the wires of circuit \lstinline[language=haskell]{c} with
bundles of additional wires that map their input straight on their output.
The \lstinline[language=haskell]{ws} parameter specifies the width of the consecutive bundles;
the $i$th wire of \lstinline[language=haskell]{c} is preceded by a bundle of width $\textit{ws}_i - 1$.

%- - - - - - - - - - - - - - - - - - - - - - - - - - - - - - - - - - - - - - - -

\begin{figure}[t]
  \begin{subfigure}[b]{0.44\textwidth}
    \lstinputlisting[language=haskell,linerange=15-22]{./examples/Scan.hs}% APPLY:linerange=DSL_WIDTH
    \subcaption{Width embedding}
  \end{subfigure} ~
  \begin{subfigure}[b]{0.57\textwidth}
    \lstinputlisting[language=haskell,linerange=27-34]{./examples/Scan.hs}% APPLY:linerange=DSL_DEPTH
    \subcaption{Depth embedding}
  \end{subfigure}
  \caption{Two finally tagless embeddings of circuits.}\label{fig:finally-tagless}
\end{figure}


\paragraph{Basic width and depth embeddings.}

\Cref{fig:finally-tagless} shows two simple shallow embeddings, which represent a circuit
respectively in terms of its width and its depth. The former denotes the number
of inputs/outputs of a circuit, while the latter is the maximal number of
$\oplus$ operators between any input and output.
Both definitions follow the same setup: a new Haskell datatype
(\lstinline[language=haskell]{Width}/\lstinline[language=haskell]{Depth}) wraps the primitive result value and provides an
instance of the \lstinline[language=haskell]{Circuit} type class that interprets the 5 DSL primitives
accordingly.
The following code creates a so-called Brent-Kung parallel prefix circuit~\cite{brent1980chip}:
\lstinputlisting[language=haskell,linerange=39-42]{./examples/Scan.hs}% APPLY:linerange=DSL_E1
Here \lstinline[language=haskell]{e1} evaluates to \lstinline[language=haskell]$W {width = 4}$. If we want to know the
depth of the circuit, we have to change type signature to \lstinline[language=haskell]{Depth}.

%- - - - - - - - - - - - - - - - - - - - - - - - - - - - - - - - - - - - - - - - 
\paragraph{Interpreting multiple ways.}

Fortunately, with the help of polymorphism we can define a type
of circuits that support multiple interpretations at once.
\lstinputlisting[language=haskell,linerange=47-47]{./examples/Scan.hs}% APPLY:linerange=DSL_FORALL
This way we can provide a single Brent-Kung parallel prefix circuit definition that can be reused
for different interpretations.
\lstinputlisting[language=haskell,linerange=51-54]{./examples/Scan.hs}% APPLY:linerange=DSL_BRENT
A type annotation then selects the desired interpretation.
For instance, \lstinline[language=haskell]{brentKung :: Width} yields the width and
\lstinline[language=haskell]{brentKung :: Depth} the depth.

%- - - - - - - - - - - - - - - - - - - - - - - - - - - - - - - - - - - - - - - - 
\paragraph{Composition of embeddings.}

What is not ideal in the above code is that the same \lstinline[language=haskell]{brentKung}
circuit is processed twice, if we want to execute both interpretations. We can do 
better by processing the circuit only once, computing both interpretations simultaneously.
The finally tagless encoding achieves this with a boilerplate instance
for tuples of interpretations.
\lstinputlisting[language=haskell,linerange=59-64]{./examples/Scan.hs}% APPLY:linerange=DSL_TUPLE
Now we can get both embeddings simultaneously as follows:
\lstinputlisting[language=haskell,linerange=68-69]{./examples/Scan.hs}% APPLY:linerange=DSL_E12
This evaluates to \lstinline[language=haskell]$(W {width = 4}, D {depth = 2})$.

%- - - - - - - - - - - - - - - - - - - - - - - - - - - - - - - - - - - - - - - - 
\paragraph{Composition of dependent interpretations.}

The composition above is easy because the two embeddings are
orthogonal. In contrast, the composition of dependent interpretations is
rather cumbersome in the standard finally tagless setup. An example of the
latter is the interpretation of circuits as their well-sizedness, which
captures whether circuits are well-formed. This interpretation depends on the
interpretation of circuits as their width.\footnote{Dependent recursion schemes
are also known as \emph{zygomorphism}~\cite{fokkinga1989tupling} after the ancient Greek word \emph{\textzeta\textupsilon\textgamma\textomikron\textnu}
for yoke. We have labeled the \lstinline{Width} field with \lstinline{ox} because it is pulling the yoke.}
\lstinputlisting[language=haskell,linerange=74-81]{./examples/Scan.hs}% APPLY:linerange=DSL_WS
The \lstinline[language=haskell]{WellSized} datatype represents the well-sizedness of a circuit with
a Boolean, and also keeps track of the circuit's width. The 5 primitives
compute the well-sizedness in terms of both the width and well-sizedness of the subcomponents.
What makes the code cumbersome is that it has to explicitly delegate to the \lstinline[language=haskell]{Width}
interpretation to collect this additional information.

With the help of a substantially more complicated setup that features a dozen
Haskell language extensions, and advanced programming techniques, we can make
the explicit delegation implicit (see the appendix). Nevertheless,
that approach still requires \emph{a lot of boilerplate} that needs to be repeated for
each DSL, as well as explicit projections that need to be written in each
interpretation. Another alternative Haskell encoding that also enables
multiple dependent interpretations is proposed by Zhang and Oliveira~\cite{zhang19shallow},
but it does not eliminate the explicit delegation and still requires
substantial amounts of boilerplate.
A final remark is that adding new primitives (e.g.,
a ``right stretch'' \lstinline{rstretch}
combinator~\cite{hinze2004algebra}) can also be easily 
achieved~\cite{emgm}.

 
%- - - - - - - - - - - - - - - - - - - - - - - - - - - - - - - - - - - - - - - - 
\subsection{The \sedel Encoding}

\sedel is a source language that elaborates to \fnamee, adding
a few convenient source level constructs.
The \sedel setup of the circuit DSL is similar to the finally tagless
approach. Instead of a \lstinline[language=haskell]{Circuit c} type class, there is a \lstinline{Circuit[C]}
type that gathers the 5 circuit primitives in a record. Like in Haskell, the type
parameter \lstinline{C} expresses that the interpretation of circuits
is a parameter.
\lstinputlisting[linerange=42-44]{./examples/scan.sl}% APPLY:linerange=SEDEL_DEF
As a side note if a new constructor (e.g., \lstinline{rstretch}) is
needed, then this is done by means of
intersection types (\lstinline{&} creates an intersection type) in \sedel:
\lstinputlisting[linerange=49-49]{./examples/scan.sl}% APPLY:linerange=SEDEL_DEF2

%- - - - - - - - - - - - - - - - - - - - - - - - - - - - - - - - - - - - - - - - 
% \paragraph{Basic width and depth embeddings.}

\begin{figure}[t]
\lstinputlisting[linerange=59-65]{./examples/scan.sl}% APPLY:linerange=SEDEL_WIDTH
\hrule
\lstinputlisting[linerange=74-80]{./examples/scan.sl}% APPLY:linerange=SEDEL_DEPTH
\caption{Two \sedel embeddings of circuits.}
\label{fig:sedel}
\end{figure}

\Cref{fig:sedel} shows the two basic shallow embeddings for width and
depth. In both cases, a named \sedel definition
replaces the corresponding unnamed
Haskell type class instance in providing the implementations of the 5 language
primitives for a particular interpretation.


The use of the \sedel embeddings is different from that of their Haskell
counterparts. Where Haskell implicitly selects the appropriate type class
instance based on the available type information, in \sedel the programmer
explicitly selects the implementation following the style used by
object algebras.
The following code does this by
% creating an object \lstinline{l1} out of the \lstinline{language1}
% trait and then
building a circuit with \lstinline{l1} (short for \lstinline{language1}).
\lstinputlisting[linerange=85-88]{./examples/scan.sl}% APPLY:linerange=SEDEL_E1
Here \lstinline{e1} evaluates to \lstinline${width = 4}$. If we want to know the
depth of the circuit, we have to replicate the code with \lstinline{language2}.

%- - - - - - - - - - - - - - - - - - - - - - - - - - - - - - - - - - - - - - - - 
\paragraph{Dynamically reusable circuits.}

Just like in Haskell, we can use polymorphism to define a type
of circuits that can be interpreted with different languages.
\lstinputlisting[linerange=93-93]{./examples/scan.sl}% APPLY:linerange=SEDEL_FORALL
In contrast to the Haskell solution, this implementation explicitly accepts
the implementation.
\lstinputlisting[linerange=99-104]{./examples/scan.sl}% APPLY:linerange=SEDEL_BRENT

%- - - - - - - - - - - - - - - - - - - - - - - - - - - - - - - - - - - - - - - - 
\paragraph{Automatic composition of languages.}

Of course, like in Haskell we can also compute both results simultaneously.
However, unlike in Haskell, the composition of the two interpretation requires
no boilerplate whatsoever---in particular, there is no \sedel counterpart of the
\lstinline[language=haskell]{Circuit (c1, c2)} instance. Instead, we can just compose the two interpretations
with the term-level merge operator (\lstinline{,,}) and specify the desired type \lstinline{Circuit[Width & Depth]}.
\lstinputlisting[linerange=109-110]{./examples/scan.sl}% APPLY:linerange=SEDEL_E3
Here the use of the merge operator creates a term with the intersection type
\lstinline{Circuit[Width] & Circuit[Depth]}. Implicitly, the \sedel type system
takes care of the details, turning this intersection type into
\lstinline{Circuit[Width & Depth]}. This is possible because intersection (\lstinline{&}) distributes over function and record types (a distinctive feature of BCD-style subtyping).

%- - - - - - - - - - - - - - - - - - - - - - - - - - - - - - - - - - - - - - - - 
\paragraph{Composition of dependent interpretations.}

In \sedel the composition scales nicely to dependent interpretations.
For instance, the well-sizedness interpretation can be expressed without
explicit projections.
\lstinputlisting[linerange=119-127]{./examples/scan.sl}% APPLY:linerange=SEDEL_WS
% It may be instructive to show the type of \lstinline{language4}:
% \begin{lstlisting}
% { identity : Int -> WellSized
% , fan      : Int -> WellSized
% , above    : WellSized & Width -> WellSized & Width -> WellSized
% , beside   : WellSized -> WellSized -> WellSized
% , stretch  : List[Int] -> WellSized & Width -> WellSized
% }
% \end{lstlisting}
Here the \lstinline{WellSized & Width} type in the \lstinline{above} and \lstinline{stretch} cases
expresses that both the well-sizedness and width of subcircuits must be given,
and that the width implementation is left as a dependency---when \lstinline{language4} is used,
then the width implementation must be provided.
Again, the distributive properties of \lstinline{&} in the type system take care of
merging the two interpretations.
\lstinputlisting[linerange=142-143]{./examples/scan.sl}% APPLY:linerange=SEDEL_E4

%- - - - - - - - - - - - - - - - - - - - - - - - - - - - - - - - - - - - - - - - 
\paragraph{Disjoint polymorphism and dynamic merges.}

While it may seem from the above examples that definitions have to be merged
statically, \sedel in fact supports dynamic merges. For instance, we can
encapsulate the merge operator in the \lstinline{combine} function while
abstracting over the two components \lstinline{x} and \lstinline{y} that are merged
as well as over their types \lstinline{A} and \lstinline{B}.
\lstinputlisting[linerange=132-132]{./examples/scan.sl}% APPLY:linerange=SEDEL_COMBINE
This way the components \lstinline{x} and \lstinline{y} are only known at runtime and
thus the merge can only happen at that time.
The types \lstinline{A} and \lstinline{B} cannot be chosen entirely freely. For
instance, if both components would contribute an implementation for the same
method, which implementation is provided by the combination would be ambiguous.
To avoid this problem the two types \lstinline{A} and \lstinline{B} have to be
\emph{disjoint}. This is expressed in the disjointness constraint \lstinline{* A}
on the quantifier of the type variable \lstinline{B}. If a quantifier mentions
no disjointness constraint, like that of \lstinline{A}, it defaults to the
trivial \lstinline{* Top} constraint which implies no restriction.
% With \lstinline{combine},
% we can rewrite \lstinline{l3} as follows (note that \lstinline{Width} and \lstinline{Depth} are disjoint):
% \lstinputlisting[linerange=137-137]{./examples/scan.sl}% APPLY:linerange=SEDEL_L3


% We can extend our circuit DSL with additional features.
% Suppose we 
% \begin{Verbatim}[fontsize=\small]
%   addBelow[C,S,R * {below: C -> C -> C, above : C -> C -> C}](lang: Trait[S,{above : C -> C -> C} & R])
%     = trait inherits lang { below(c1 : C, c2 : C) = super.above(c2,c2) }
% \end{Verbatim}


% Local Variables:
% org-ref-default-bibliography: "../paper.bib"
% TeX-master: "../paper"
% End:

\section{A Type System with Gradually Typed Implicit Polymorphism}
\label{sec:type-system}

In \Cref{sec:exploration} we have introduced the consistent
subtyping relation that naturally extends to polymorphic types. In
this section we continue with the development by giving a declarative
type system for implicit polymorphism that employs the consistent
subtyping relation. The declarative system itself is already quite
interesting as it is equipped with both higher-rank polymorphism and
the unknown type. Moreover, unlike non-gradual type systems with
higher-rank polymorphism, guessed types affect runtime behaviour if
used by the implicit casts, which raises concerns with respect to
coherency. Our response to those concerns is given in \Cref{subsec:algo:discuss},
after we give a simple
algorithm that implements the declarative system
(\Cref{sec:algorithm}) and discuss soundness and completeness.

% Later in \Cref{sec:algorithm} we give a simple
%algorithm that implements the declarative system.

\subsection{Language Overview}

\begin{figure}[t]
  \centering
  \begin{small}
\begin{tabular}{lrcl} \toprule
  Expressions & $e$ & \syndef & $x \mid n \mid
                         \blam x A e \mid e~e$ \\
%%                         \mid \erlam x e \equiv \blam x \unknown e $ \\

  Types & $A, B$ & \syndef & $ \nat \mid a \mid A \to B \mid \forall a. A \mid \unknown$ \\
  Monotypes & $\tau, \sigma$ & \syndef & $ \nat \mid a \mid \tau \to \sigma$ \\

  Contexts & $\dctx$ & \syndef & $\ctxinit \mid \dctx,x: A \mid \dctx, a$ \\
  Syntactic Sugar & $\erlam x e$ & $\equiv$ & $\blam x \unknown e$ \\
              & $e : A$ & $\equiv$ & $(\blam x A x) ~ e$ \\ \bottomrule
\end{tabular}
  \end{small}
\caption{Syntax of the declarative type system}
\label{fig:decl-syntax}
\end{figure}

The complete syntax of the declarative system is given in
\Cref{fig:decl-syntax}. We use the meta-variable $e$ to range over expressions.
Expressions are either variables $x$, integers $n$, annotated lambda
abstractions $\blam x A e$, or applications $e_1 ~ e_2$. We write $A$, $B$ for
types. Types are either the integer type $\nat$, type variables $a$, functions
types $A \to B$, universal quantification $\forall a. A$, or the unknown type
$\unknown$. Though we only have one base type $\nat$, we also use $\bool$ for
the purpose of illustration. Monotypes $\tau$ contain all types other than the
universal quantifier and the unknown type. Contexts $\dctx$ map term variables
to their types, and record all type variables with the expected well-formedness
condition. Following \citet{siek2006gradual}, if a lambda binder does not have
an annotation, it is automatically annotated with $\unknown$. As a convenience,
the language also provides type ascription $e : A$, which is simulated by
$(\blam x A x) ~ e$.

\subsection{Typing in Detail}

\Cref{fig:decl-typing} gives the typing rules for our declarative system
(the reader is advised to ignore the gray-shaded parts for now). Rule \rul{Var}
extracts the type of the variable from the typing context. Rule \rul{Nat} always
infers integer types. Rule \rul{LamAnn} puts $x$ with type annotation $A$ into
the context, and continues type checking the body $e$. Rule \rul{App} first
infers the type of $e_1$, then the matching judgment $\tprematch A \match A_1
\to A_2$ extracts the domain type $A_1$ and the codomain type $A_2$ from type
$A$. The type $A_3$ of the argument $e_2$ is then compared with $A_1$ using the
consistent subtyping judgment.

\renewcommand{\trto}[1]{\hlmath{\rightsquigarrow{#1}}}
\begin{figure}[t]
  \begin{small}
  \begin{mathpar}
    \framebox{$\tpreinf e : A \trto s$} \\
    \DVar \and \DNat \and \DLamAnnA \and \DApp
  \end{mathpar}

  \begin{mathpar}
    \framebox{$\tprematch A \match A_1 \to A_2$} \\
    \MMC \\ \MMA \and \MMB
  \end{mathpar}

  \end{small}
  \caption{Declarative typing}
  \label{fig:decl-typing}
\end{figure}

\paragraph{Matching} It turns out that matching~\cite{siek2015refined} can be
extended to polymorphic types naturally. In \rul{M-Forall}, a monotype $\tau$ is
guessed to instantiate the universal quantifier $a$. This natural extension is
also inspired by the \textit{application judgment} $\tpreinf A \bullet e \infto
C$ by \citet{dunfield2013complete}, which says that if we apply a term of type
$A$ to an argument $e$, we get something of type $C$. If $A$ is a polymorphic
type, the judgment works by guessing instantiations of polymorphic quantifiers
until it reaches an arrow type. Rule \rul{M-Arr} and \rul{M-Unknown} are the
same as \citet{siek2015refined}.


\renewcommand{\trto}[1]{\rightsquigarrow{#1}}
\subsection{Type-directed Translation}
\label{sec:type:trans}

We give the dynamic semantics of our language by translating it to
\pbc~\cite{ahmed2011blame}. Below we show a subset of the terms in \pbc that are
used in the translation:
\[
  \text{Terms}\quad s ::= x \mid n \mid \blam x A s \mid s~s \mid \cast A B s
\]
A cast $\cast A B {s}$ converts the value of term $s$ from type $A$ to type $B$.
A cast from $A$ to $B$ is permitted only if the types are \textit{compatible},
written $A \pbccons B$, as briefly mentioned in
\Cref{subsec:consistency-subtyping}. The syntax of types in \pbc is the
same as ours.

The translation is given in the gray-shaded parts in \Cref{fig:decl-typing}. The
only interesting case here is to insert explicit casts in the application rule.
Note that there is no need to translate matching or consistent subtyping,
instead we insert the source and target types of a cast directly in the
translated expressions, thanks to the following two lemmas:

\begin{clemma}[Compatibility of Matching]
  \label{lemma:comp-match}
  If $\tprematch A \match A_1 \to A_2$, then $A \pbccons A_1 \to A_2$.
\end{clemma}

\begin{clemma}[Compatibility of Consistent Subtyping]
  \label{lemma:comp-conssub}
  If $\tpreconssub A \tconssub B$, then $A \pbccons B$.
\end{clemma}

In order to show the correctness of the translation, we prove that our
translation always produces well-typed expressions in \pbc. By
\Cref{lemma:comp-match,lemma:comp-conssub}, we have the following theorem:

\begin{ctheorem}[Type Safety]
  \label{lemma:type-safety}
  If $\tpreinf e : A \trto s$, then $\dctx \bypinf s : A$.
\end{ctheorem}

\paragraph{Parametricity} An important semantic property of polymorphic types is
\textit{relational parametricity}~\cite{reynolds1983types}. The parametricity
property says that all instances of a parametrically polymorphic function should
behave \textit{uniformly}. In other words, functions cannot inspect into a type
variable, and act differently for different instances of the type variable. A
classic example is a function with the type $\forall a . a \to a$. The
parametricity property guarantees that a value of this type must be either the
identity function (i.e., $\lambda x . x$) or the undefined function (one which
never returns a value). However, with the addition of the unknown type
$\unknown$, careful measures are to be taken to ensure parametricity. This is
exactly the circumstance that \pbc was designed to address. \citet{amal2017blame}
proved that \pbc satisfies relational parametricity. Based on their result, and
by \Cref{lemma:type-safety}, parametricity is preserved in our system.

\paragraph{Guessed types affect runtime behaviour}

However, the translation does not always produce a unique target expression.
This is because when we guess a monotype $\tau$ in rule \rul{M-Forall} and
\rul{CS-ForallL}, we could have different choices, which inevitably leads to
different types. Unlike (non-gradual) polymorphic type systems
\citep{jones2007practical, dunfield2013complete}, the guessed types affect
runtime behaviour of the translated programs, since they could appear inside the
explicit casts. For example, the following shows two possible translations for
the same source expression $\blam x \unknown {f ~ x}$, where $f$ is
instantiated to $\nat \to \nat$ and $\bool \to \bool$, respectively:
\begin{align*}
  f: \forall a. a \to a &\byinf (\blam x \unknown {f ~ x})
                          : \unknown \to \nat \\
                          &\trto (\blam x \unknown (\cast {\forall a. a \to a} {\nat \to \nat} f) ~
                          (\hlmath{\cast \unknown \nat} x))
  \\
  f: \forall a. a \to a &\byinf (\blam x \unknown {f ~ x})
                          : \unknown \to \bool \\
                          &\trto (\blam x \unknown (\cast {\forall a. a \to a} {\bool \to \bool} f) ~
                          (\hlmath{\cast \unknown \bool} x))
\end{align*}
If we apply $\blam x \unknown {f ~ x}$ to $3$ for example, which should be fine
since the function can take any input, the first translation runs smoothly in
\pbc, while the second one will raise a cast error ($\nat$ cannot be cast to
$\bool$). Similarly, if we apply it to $\truee$, then the second succeeds while
the first fails. The culprit lies in the highlighted parts where any
instantiation of $a$ would be put inside the explicit cast. More generally, any
choice introduces an explicit cast to that type in the translation, which causes
a runtime cast error if the function is applied to a value whose type does not
match the guessed type. Note that this does not compromise the type safety of
the translated expressions, since cast errors are part of the type safety
guarantees.

\paragraph{Coherency}

The ambiguity of translation seems to imply that the
declarative is \textit{incoherent}. Coherence is a desired
property for a semantics. A semantic is coherent if any \textit{valid program}
has exactly one meaning~\cite{Reynolds_coherence}. We argue that the declarative
system is still coherent in the sense that if a program produces a value, this
value is unique. In the above example, whatever the translation might be,
applying $\blam x \unknown {f ~ x}$ to $3$ either results in a cast error, or
produces $3$, and not any other values.

This discrepancy is due to the guessing nature of the \textit{declarative}
system. As far as the declarative system is concerned, both $\nat \to \nat$ and
$\bool \to \bool$ are equally acceptable. But this is not the case at runtime.
The acute reader may have found that the \textit{only} appropriate choice is to
instantiate $f$ to $\unknown \to \unknown$. However, as specified by rule
\rul{M-Forall} in \Cref{fig:decl-typing}, we can only instantiate type variables
to monotypes, but $\unknown$ is \textit{not} a monotype! We will get back to
this issue in \Cref{subsec:algo:discuss} after we present the corresponding
algorithmic system in \Cref{sec:algorithm}.


\subsection{Correctness Criteria}
\label{sec:criteria}

\citet{siek2015refined} present a set of properties that a well-designed gradual
typing calculus must have, which they call refined criteria. Among all the
criteria, those related to the static aspects of gradual typing are well
summarized by \citet{cimini2016gradualizer}. Here we review those criteria and
adapt them to our notation. We have proved in Coq that our type system satisfies
all of these criteria.

\begin{clemma}[Correctness Criteria]\leavevmode
  \begin{itemize}
  \item \textbf{Conservative extension:}
    for all static $\dctx$, $e$, and $A$,
    $\dctx \byhinf e : A $ if and only if $\dctx \byinf e : A$.
  \item \textbf{Monotonicity w.r.t. precision:}
    for all $\dctx, e, e', A$,
    if $\dctx \byinf e : A$,
    and $e' \lessp e$,
    then $\dctx \byinf e' : B$,
    and $B \lessp A$ for some B.
  \item \textbf{Type Preservation of cast insertion:}
    for all $\dctx, e, A$,
    if $\dctx \byinf e : A$,
    then $\dctx \byinf e : A \trto s$,
    and $\dctx \bypinf s : A$ for some $s$.
  \item \textbf{Monotonicity of cast insertion:}
    for all $\dctx, e_1, e_2, e_1', e_2', A$,
    if $\dctx \byinf e_1 : A \trto e_1'$,
    and $\dctx \byinf e_2 : A \trto e_2'$,
    and $e_1 \lessp e_2$,
    then $\dctx \ctxsplit \dctx \bylessp e_1' \lesspp e_2'$.
  \end{itemize}
\end{clemma}

\begin{figure}[t]
  \begin{small}
  \begin{mathpar}
    \framebox{$A \lessp B$}{\quad \text{Type precision}} \\
    \LUnknown \and \LNat \and \LArrow \and \LTVar
    \and \LForall
  \end{mathpar}

  \begin{mathpar}
    \framebox{$e_1 \lessp e_2$}{\quad \text{Term precision}} \\
    \LRefl \and \LAbsAnn \and \LApp
  \end{mathpar}

  \begin{mathpar}
    \framebox{$\dctx_1 \ctxsplit \dctx_2 \bylessp e_1 \lesspp e_2$}
    {\quad \text{Term less precision in \pbc}} \\
    \LVar \and \LNatP \and \LAbsAnnP \and
    \LAppP \and \LCast \and \LCastL \and
    \LCastR
  \end{mathpar}
  \end{small}
  \caption{Less Precision}
  \label{fig:lessp}
\end{figure}


The first criterion states that the gradual type system should be a conservative
extension of the original system (i.e., the Odersky-L{\"a}ufer type system in
our case). In other words, a \textit{static} program that is typeable in the
original type system should remain typeable in the gradual type
system. A static program is one that does not contain any type $\unknown$. It also
ensures that ill-typed programs of the original language remain so in the
gradual type system.

The second criterion states that if a typeable expression loses some type
information, it remains typeable. This criterion depends on the definition of
the precision relation, written $A \lessp B$, which is given in the top of
\Cref{fig:lessp}. The relation intuitively captures a notion of types containing
more or less unknown types ($\unknown$). The precision relation over types lifts
to programs, i.e., $e_1 \lessp e_2$ means that $e_1$ and $e_2$ are the same
program except that $e_2$ has more unknown types.

The first two criteria are fundamental to gradual typing. They explain for
example why these two programs $(\blam x \nat {x + 1})$ and $(\blam x \unknown
{x + 1})$ are typeable, as the former is typeable in the Odersky-L{\"a}ufer type
system and the latter is a less-precise version of it.

The last two criteria relate to the compilation to the cast calculus. The
third criterion is essentially the same as \Cref{lemma:type-safety}, given that
a target expression should always exist, which can be easily seen from
\Cref{fig:decl-typing}. The last criterion ensures that the translation
must be monotonic over the precision relation $\lessp$. (The definition of the
precision relation $\lesspp$ for \pbc is found in the bottom of
\Cref{fig:lessp}.)


%%% Local Variables:
%%% mode: latex
%%% TeX-master: "../paper"
%%% org-ref-default-bibliography: "../paper.bib"
%%% End:

%%%%%%%%%%%%%%%%%%%%%%%%%%%%%%%%%%%%%%%%%%%%%%%%%%%%%%%%%%%%%%%%%%%%%%%%
\section{Establishing Coherence for \fnamee}
\label{sec:coherence:poly}
%%%%%%%%%%%%%%%%%%%%%%%%%%%%%%%%%%%%%%%%%%%%%%%%%%%%%%%%%%%%%%%%%%%%%%%%

In this section, we establish the coherence property for \fnamee. The proof
strategy mostly follows that of \namee, but the construction of the
heterogeneous logical relation is significantly more complicated. Firstly in
\cref{sec:para:intuition} we discuss why adding BCD subtyping to disjoint
polymorphism introduces significant complications. In
\cref{sec:failed:lr}, we discuss why a natural extension of
System F's logical relation to deal with disjoint polymorphism fails. The technical
difficulty is \emph{well-foundedness}, stemming from the interaction between
impredicativity and disjointness. Finally in \cref{sec:succeed:lr}, we present
our (predicative) logical relation that is specially crafted to prove coherence
for \fnamee.
% and allude to a potential solution to lift the predicativity restriction.

\subsection{The Challenge}
\label{sec:para:intuition}

Before we tackle the coherence of \fnamee, let us first consider how \fname
(and its predecessor \oname) enforces coherence. Its essentially syntactic
approach is to make sure that there is at most one subtyping derivation for any
two types. As an immediate consequence, the produced coercions are uniquely determined and thus
the calculus is clearly coherent. Key to this approach is the invariant that
the type system only produces \emph{disjoint} intersection types. As we
mentioned in \cref{sec:typesystem}, this invariant complicates the calculus
and its metatheory, and leads to a weaker substitution lemma.
% To see this, consider the judgment $[[ X ** nat |- X & nat ]]$. 
% Clearly $[[X]]$ cannot be instantiated to an arbitrary type. For
% instance, substituting $[[X]]$ with $[[nat]]$ would lead to an ill-formed
% intersection type $[[nat & nat]]$ in \fname. 
% Therefore in the
% substitution lemma, the range of substituted types is narrowed down to those
% that respect the disjointness constraints.
% The motivation of maintaining this invariant was to enable
% Generally speaking, in \fname all meta-theoretic properties are weakened to
% account for disjointness pre-conditions. All of these contribute
Moreover, the syntactic coherence approach is incompatible with BCD subtyping,
which leads to multiple subtyping derivations with different coercions and
requires a more general substitution lemma.
% For example, consider the
% coercions produced by $[[ \X ** nat . X & X <: \X ** nat & nat . X ]]$ (neither
% type is ``well-formed'' in the sense of \fname). Two possible ones are
% $[[ \f . \X . pp1 (f X) ]]$ and $[[ \f . \X . pp2 (f X) ]]$. It is not at all
% obvious that they should be equivalent in an appropriate sense.
To accommodate BCD into \oname, Bi et al.~\cite{bi_et_al:LIPIcs:2018:9227}
have created the \namee calculus and
developed a semantically-founded proof method based on logical relations.
Because \namee does not feature polymorphism, the problem at hand is to
incorporate support for polymorphism in this semantic approach to coherence,
which turns out to be more challenging than is apparent.

% preclude the possibility of adding BCD
% subtyping, which requires a general substitution lemma. This implies that the
% avenue taken by Alpuim et al.~\cite{alpuimdisjoint} to prove coherence does not
% work for \fnamee anymore. In particular, subtyping does not necessarily produces unique
% coercions. For example, consider the possible coercions generated by $[[ \X ** nat . X & X <: \X ** nat & nat . X ]]$ (neither of which is ``well-formed''
% in the sense of \fname). Two possible coercions are $[[ \f . \X . pp1 (f X) ]]$
% and $[[ \f . \X . pp2 (f X) ]]$. It is not at all obvious that these two
% coercions are equivalent in an appropriate sense. Moreover, the addition of BCD subtyping
% aggravates the matter even more---the subtyping relation can produce additional
% syntactically different coercions that are harder to argue to be equivalent.
% Inspired by Bi et al.~\cite{bi_et_al:LIPIcs:2018:9227}, a new semantically-founded
% proof method is called for. Logical relations \`a la System F might shed some
% light, as we will discuss next.

\begin{figure}[t]
  \centering
  \begin{tabular}{rll}
    $[[(v1 , v2) in V ( nat ; nat ) ]]$  & $\defeq$ & $\exists [[i]].\, [[v1]] = [[v2]] = [[i]]$ \\
    $[[(v1, v2)  in V(T1 -> T2; T1' -> T2') ]]$ &$\defeq$ & $\forall [[(v, v') in V (T1; T1')   ]].\, [[  (v1 v , v2 v') in E (T2 ; T2') ]]$ \\
    $[[( < v1 , v2 > , v3  )  in V ( T1 * T2 ;  T3  )  ]]$  &$\defeq$& $[[ (v1, v3)  in V (T1 ; T3)  ]] \land [[ (v2, v3)  in V (T2 ; T3)  ]]$ \\
    $[[( v3 , < v1 , v2 >  )  in V ( T3 ; T1 * T2  )  ]]$  &$\defeq$& $[[ (v3, v1)  in V (T3 ; T1)  ]] \land [[ (v3, v2)  in V (T3 ; T2)  ]]$
  \end{tabular}
  \caption{Selected cases from \namee's canonicity relation}
  \label{fig:logical:necolus}
\end{figure}

\subsection{Impredicativity and Disjointness at Odds}
\label{sec:failed:lr}

\Cref{fig:logical:necolus} shows selected cases of \emph{canonicity},
which is \namee's (heterogeneous) logical relation used
in the coherence proof. The definition captures that two values
$[[v1]]$ and $[[v2]]$ of types $[[ T1 ]]$ and $[[T2]]$ are in $\valR{[[T1]]}{[[T2]]}$ iff
either the types are disjoint or the types are equal and the values are
semantically equivalent. Because both alternatives entail coherence, 
canonicity is key to \namee's coherence proof.

\paragraph{Well-foundedness issues.}
For \fnamee, we need to extend canonicity with additional cases to
account for universally quantified types. For reasons that will become clear in
\cref{sec:succeed:lr}, the type indices become source types (rather than target types as in \cref{fig:logical:necolus}).
A naive formulation of one case rule is:
{\small
\begin{align*}
    &[[(v1, v2)  in V(\X ** A1 . B1; \X ** A2 . B2) ]] \defeq  \\
    &\qquad \forall [[C1 ** A1]], [[C2 ** A2]].\ [[( v1 | C1 | , v2 | C2 | ) in E ( B1 [X ~> C1]; B2 [X ~> C2]) ]]
\end{align*}
}%
This case is problematic because it destroys the well-foundedness of \namee's
logical relation, which is based on structural induction on the type indices.
Indeed, the type $[[ B1 [X ~> C1] ]]$ may well be larger than $[[ \X ** A1 . B1 ]]$.


% \begin{verbatim}
% Further outline
% - show System F-style case with deferred substitions
% - introduce variable case
% - show well-foundedness problem with variable case (also present in System F)
% - show System F solution for the problem by adding a relation parameter R
% - introduce problem with heterogeneous case
% \end{verbatim}

However, System F's well-known parametricity logical
relation~\cite{reynolds1983types} provides us with a means to avoid this
problem.  Rather than performing the type substitution immediately as in the
above rule, we can defer it to a later point by adding it to an extra parameter
$[[pq]]$ of the relation, which accumulates the deferred substitutions. This yields a modified rule where the type indices in the recursive occurrences are indeed smaller:
{\small
\begin{align*}
  &[[(v1, v2)  in V(\X ** A1 . B1; \X ** A2 . B2) with pq ]]  \defeq  \\
  &\qquad \forall [[C1 ** A1]], [[C2 ** A2]]. ([[v1 | C1 | ]] ,  [[v2 | C2 |]]) \in \eeR{[[B1]]}{{[[B2]]}}_{[[pq]] [ [[X]] \mapsto ([[C1]], [[C2]])]}
\end{align*}
}%
Of course, the deferred substitution has to be performed eventually, to be precise when the type indices are type variables.
\[
    [[(v1, v2)  in V(X ; X) with pq ]] \defeq [[ (v1, v2) in V(pq1 (X); pq2 (X)) with emp  ]]
\]
Unfortunately, this way we have not only moved the type substitution to the type variable case, but also the ill-foundedness problem. Indeed, this problem is also
present in System F. The standard solution is to not fix the relation $[[Rel]]$ by which values
at type $[[X]]$ are related to $\valR{[[pq1 (X)]]}{[[pq2 (X)]]}$, but instead to make it a parameter that is tracked by $[[pq]]$.
This yields the following two rules for disjoint quantification and type variables:
{\small
\begin{align*}
  [[(v1, v2)  in V(\X ** A1 . B1; \X ** A2 . B2) with pq ]] &\defeq \forall [[C1 ** A1]], [[C2 ** A2]], [[Rel]] \subseteq [[C1]] \times [[C2]]. \\
                                                            & ([[v1 | C1 | ]] ,  [[v2 | C2 |]]) \in \eeR{[[B1]]}{{[[B2]]}}_{[[pq]] [ [[X]] \mapsto ([[C1]], [[C2]], [[Rel]])]} \\
    [[(v1, v2)  in V(X; X) with pq ]] & \defeq ([[v1]], [[v2]]) \in [[pq]]_{[[Rel]]}([[X]])
\end{align*}
}%
Now we have finally recovered the well-foundedness of the relation. It is again
structurally inductive on the size of the type indexes.


\paragraph{Heterogeneous issues.}

We have not yet accounted for one major difference between the parametricity relation, from which we have borrowed ideas, and the canonicity relation, to which we have been adding. The former is homogeneous (i.e., the types of the two values is the same) and therefore has one type index, while the latter is heterogeneous (i.e., the two values may have different types) and therefore has two type indices. Thus we must also consider cases like
$\valR{[[X]]}{[[nat]]}$. A definition that seems to handle this case
appropriately is:
{\small
  \begin{align} \label{eq:var}
    [[(v1, v2)  in V(X; nat) with pq ]] \defeq [[ (v1, v2) in V(pq1 (X); nat) with emp  ]]
  \end{align}
}%
Here is an example to motivate it.
Let  $  [[ee]] = [[\ X ** Top . ((\x . x) : X & nat -> X & nat)]] $.
We expect that $[[ee nat 1 ]]$ evaluates to $[[ <1 , 1> ]]$. To prove that,
%%\footnote{The reader is advised to try it out in our prototype interpreter.}
we need to show $  (1 , 1)   \in \valR{[[X]]}{[[nat]]}_{[ [[X]] \mapsto ([[nat]], [[nat]], [[Rel]])   ]}  $.
According to \cref{eq:var}, this is indeed the case. However, we run into ill-foundedness issue again, because
$[[pq1 (X)]]$ could be larger than $[[X]]$. Alas, this time the parametricity relation has no solution for us.


\subsection{The Canonicity Relation for \fnamee}
\label{sec:succeed:lr}

% \bruno{Perhaps we are still showing too many auxiliary lemmas here? We
% could cut on some of these if we are looking for space.}

\renewcommand\ottaltinferrule[4]{
  \inferrule*[narrower=0.8,right=#1,#2]
    {#3}
    {#4}
}

In light of the fact that substitution in the logical relation seems unavoidable
in our setting, and that impredicativity is at odds with substitution, we turn
to \emph{predicativity}: we change \rref{T-tapp} to its predicative version:
{\small
\[
  \drule{T-tappMono}
\]
}%
where metavariable $[[t]]$ ranges over monotypes (types minus disjoint quantification).
We do not believe that predicativity is a severe restriction in practice, since many source
languages (e.g., those based on the Hindley-Milner type system~\cite{milner1978theory, hindley1969principal} like Haskell and
OCaml) are themselves predicative and do not require the full generality of an
impredicative core language.

\renewcommand\ottaltinferrule[4]{
  \inferrule*[narrower=0.6,lab=#1,#2]
    {#3}
    {#4}
}


% The restriction to
% predicative polymorphism, though reducing expressiveness in theory, does not seem to cost much
% in practice. Languages based on the Hindley–Milner type
% system~\cite{milner1978theory, hindley1969principal}, such as Haskell and ML,
% have such restriction. We also plan to study a variant of \fnamee with implicit
% polymorphism in the future, where a predicativity restriction is
% likely to be required anyway.

\begin{figure}[t]
  \centering
  \begin{tabular}{rll}
    $[[(v1 , v2) in V ( nat ; nat ) ]]$  & $\defeq$ & $\exists [[i]].\, [[v1]] = [[v2]] = [[i]]$ \\
    $[[(v1, v2) in V ( {l : A}  ; {l : B} ) ]]$ & $\defeq$ & $[[ (v1, v2) in V ( A ; B ) ]]$\\
    $[[(v1 , v2) in V ( A1 -> B1 ; A2 -> B2 ) ]]$  & $\defeq$ & $\forall [[(v2' , v1') in V ( A2 ; A1 ) ]].\, [[ (v1 v1' , v2 v2') in E ( B1 ; B2 ) ]]$ \\
    $[[( < v1 , v2 > , v3  )  in V ( A & B ;  C  ) ]]$  & $\defeq$ & $[[ (v1, v3)  in V (A ; C) ]] \land [[ (v2, v3)  in V (B ; C) ]]$  \\
    $[[( v3 , < v1 , v2 >  )  in V ( C; A & B  ) ]]$  & $\defeq$ & $[[ (v3, v1)  in V (C ; A) ]] \land [[ (v3, v2)  in V (C ; B) ]]$  \\
    $[[(v1, v2)  in V ( \ X ** A1 . B1; \ X ** A2 . B2 ) ]]$  &$\defeq$ & $\forall [[empty |- t ** A1 & A2 ]].\ [[  (v1 |t| , v2 |t|) in E ( B1 [X ~> t] ;  B2 [ X ~> t]) ]]$ \\
  % $[[(v1, v2) in V ( A  ; B ) ]]$ & $\defeq$ & $[[A top]] \, \lor \, [[B top]]    $ \\
    $[[(v1 , v2) in V (A; B)]] $  &$\defeq$ & $\mathsf{true} \quad \text{otherwise} $ \\
    $[[(e1, e2) in E (A; B)]]$ & $\defeq$ & $\exists [[v1]], [[v2]].\, [[e1 -->> v1]] \land [[e2 -->> v2]] \ \land [[(v1, v2) in V (A; B)]]$ \\ \\
  \end{tabular}

  \begin{tabular}{rrll}
    $[[p in  DD]]$ & $\defeq$ &  $\ottaltinferrule{}{}{  }{ [[empp in empty]] }$ &     $\ottaltinferrule{}{}{ [[p in DD]] \\ [[empty |- t ** p(B)]] \\  }{ [[p [ X -> t ] in DD , X ** B]]  }$ \\ \\
    $[[  (g1, g2)  in GG with p ]]$ & $\defeq$ &  $\ottaltinferrule{}{}{  }{ [[(empg, empg) in empty with p ]]  }$ & $\ottaltinferrule{}{}{ [[(g1, g2) in GG with p ]] \\ [[(v1, v2) in V (p(A) ; p(A)) ]] }{ [[(g1 [ x -> v1 ] , g2 [ x -> v2 ]  )  in GG , x : A with p ]] }$
  \end{tabular}
  \caption{The canonicity relation for \fnamee}
  \label{fig:logical:fi}
\end{figure}

Luckily, substitution with monotypes does not prevent well-foundedness.
\Cref{fig:logical:fi} defines the \emph{canonicity} relation for
\fnamee. The canonicity relation is a family of binary relations over \tnamee
values that are \emph{heterogeneous}, i.e., indexed by two \fnamee types. Two
points are worth mentioning. (1) An apparent difference from \namee's logical
relation is that our relation is now indexed by \emph{source types}. The reason is that
the type translation function (\cref{def:type:translate:fi}) discards disjointness
constraints, which are crucial in our setting, whereas \namee's
type translation does not have information loss. (2) Heterogeneity
allows relating values of different types, and in particular values whose types are
disjoint. The rationale behind the canonicity relation is to combine equality
checking from traditional (homogeneous) logical relations with disjointness
checking. It consists of two relations: the value relation $\valR{[[A]]}{[[B]]}$
relates \emph{closed} values; and the expression relation
$\eeR{[[A]]}{[[B]]}$---defined in terms of the value relation---relates closed
expressions.

% \paragraph{Value relation.}

The relation $\valR{[[A]]}{[[B]]}$ is defined by induction on the structures of $[[A]]$ and
$[[B]]$. For integers, it requires the two values to be literally the same. For
two records to behave the same, their fields must behave the same. For two
functions to behave the same, they are required to produce outputs related at
$[[B1]]$ and $[[B2]]$ when given related inputs at $[[A1]]$ and $[[A2]]$. For
the next two cases regarding intersection types, the relation distributes
over intersection constructor $[[&]]$. Of particular interest is the case for
disjoint quantification. Notice that it \emph{does not} quantify over arbitrary
relations, but directly substitutes $[[X]]$ with monotype $[[t]]$ in $[[B1]]$ and
$[[B2]]$. This means that our canonicity relation \emph{does not} entail
parametricity. % , and as such, the free theorem in \cref{sec:failed:lr}
% cannot be proved using the canonicity relation.
However, it suffices for our
purposes to prove coherence. Another noticeable thing is that we keep the
invariant that $[[A]]$ and $[[B]]$ are closed types throughout the relation, so
we no longer need to consider type variables. This simplifies things a lot. % The
% other cases are quite standard.
Note that when one type is $[[Bot]]$, two
values are vacuously related because there simply are no values of type $[[Bot]]$.
% We refer to Bi et al.~\cite{bi_et_al:LIPIcs:2018:9227} for more explanations of
% the canonicity relation.
We need to show that the relation is indeed well-founded:

\begin{restatable}[Well-foundedness]{lemma}{wellfounded}\label{lemma:well-founded}
  The canonicity relation of \fnamee is well-founded.
\end{restatable}
\proof
  Let $| \cdot |_{\forall}$ and $| \cdot |_s$ be the number of
  $\forall$-quantifies and the size of types, respectively. Consider the measure $\langle
  | \cdot |_{\forall} , | \cdot |_s \rangle$,
  where $\langle \dots \rangle$ denotes lexicographic order. For the case of
  disjoint quantification, the number of $\forall$-quantifiers decreases.
  For the other cases, the measure of $| \cdot |_{\forall}$ does not increase, and
  the measure of $| \cdot |_s$ strictly decreases.
\qed

% \begin{lemma}[Symmetry]
%   If $[[ (v1, v2) in V ( A ; B ) ]]$ then $[[ (v2, v1) in V ( B ; A ) ]]$.
% \end{lemma}
% \begin{proof}
%   The proof proceeds by first induction on $ | [[A]] |_{\forall} $, then
%   simultaneous induction on the structures of $[[A]]$ and $[[B]]$.
% \end{proof}

% We give the logical interpretations of type and term contexts ($[[p]]$ is a mapping
% from type variables to monotypes, $[[g]]$ is a mapping from variables to values).

% The canonicity relation is so constructed to contain values of disjoint types:
% We need to first show an auxiliary lemma regarding top-like types:

% \begin{lemma}
%   If $[[  empty ; empty |-  v1 : |A|  ]]$,
%   $[[  empty ; empty |-  v2 : |B|  ]]$ and
%   $[[ A top  ]]$,
%   then $[[   (v1, v2) in V ( A ; B  )  ]]$.
% \end{lemma}
% \begin{proof}
%   By simultaneous induction on $[[t1]]$ and $[[t2]]$.
% \end{proof}

% \begin{lemma}[Disjoint values are related]
%   If $[[DD |- A ** B]]$, $[[ p in DD  ]]$, $[[  empty ; empty |-  v1 : |p (A)|  ]]$ and $[[  empty ; empty |-  v2 : |p (B)|  ]]$
%   then $[[   (v1, v2) in V ( p(A) ; p(B)  )    ]]$.
% \end{lemma}


\subsection{Establishing Coherence}

\paragraph{Logical equivalence.}

The canonicity relation can be lifted to open expressions in the standard way,
i.e., by considering all possible interpretations of free type and term variables.
The logical interpretations of type and term contexts are found in the bottom
half of \cref{fig:logical:fi}.
\begin{definition}[Logical equivalence $\backsimeq_{log}$]
  {\small
  \begin{align*}
    &[[DD ; GG |- e1 == e2 : A ; B]]   \defeq  [[|DD| ; |GG| |- e1 : |A|]] \land [[ |DD | ; |GG| |- e2 : | B | ]] \ \land \\
    &\quad (\forall [[p]], [[g1]], [[g2]]. \ [[p in DD]] \land [[(g1, g2) in GG with p ]] \Longrightarrow [[(g1 (p1 (e1)), g2 (p2 (e2)))  in E (p(A) ; p(B)) ]])
  \end{align*}
  }%
\end{definition}
For conciseness, we write $[[DD ; GG |- e1 == e2 : A]]$ to mean $[[DD ; GG |- e1 == e2 : A ; A]]$.

\paragraph{Contextual equivalence.}

% \begin{figure}[t]
%   \centering
% \begin{tabular}{llll}\toprule
%   \tnamee contexts & $[[cc]]$ & $\Coloneqq$ &  $[[__]] \mid [[\ x . cc]] \mid [[\ X . cc]]  \mid [[ cc T  ]] \mid [[cc e]] \mid [[e cc]] \mid [[< cc , e>]] \mid [[<e , cc>]] \mid [[c cc]] $ \\
%   \fnamee contexts & $[[CC]]$ & $\Coloneqq$ &  $[[__]] \mid [[\ x . CC]] \mid [[\ X ** A. CC]] \mid [[ CC A  ]] \mid [[CC ee]] \mid [[ee CC]] \mid [[ CC ,, ee  ]] \mid [[ ee ,, CC  ]] \mid [[ { l = CC}  ]]  \mid [[ CC . l]] \mid [[ CC : A ]] $ \\ \bottomrule
% \end{tabular}
%   \caption{Expression contexts}
%   \label{fig:contexts:fi}
% \end{figure}

Following \namee, the notion of coherence is based on \emph{contextual
  equivalence}. The intuition is that two programs are equivalent if we
\emph{cannot} tell them apart in any context. As usual, contextual
equivalence is expressed using \emph{expression contexts} ($[[CC]]$ and $[[cc]]$ denote \fnamee and \tnamee expression contexts, respectively),
Due to the bidirectional nature of the type system, the typing judgment of $[[CC]]$
features 4 different forms (full rules are in the appendix),
e.g., $[[CC : (DD; GG => A) ~> (DD'; GG' => A') ~~> cc]]$ reads if $[[DD ; GG |- ee => A]]$
then $[[DD' ; GG' |- CC { ee } => A']]$. The judgment also generates a well-typed \tnamee context $[[cc]]$. The
following two definitions capture the notion of contextual equivalence:

\begin{definition}[Kleene Equality $\backsimeq$]
  Two complete programs (i.e., closed terms of type $[[nat]]$), $[[e]]$ and $[[e']]$, are Kleene equal, written
  $\kleq{[[e]]}{[[e']]}$, iff there exists an integer $[[ii]]$ such that $[[e -->> ii]]$ and
  $[[e' -->> ii]]$.
\end{definition}

\begin{definition}[Contextual Equivalence $\backsimeq_{ctx}$] \label{def:cxtx2}
  {\small
  \begin{align*}
    &[[DD ; GG |- ee1 ~= ee2 : A]]  \defeq \forall [[e1]], [[e2]].\  [[DD ; GG |- ee1 => A ~~> e1]] \land [[DD ; GG |- ee2 => A ~~> e2]] \ \land   \\
    &\qquad (\forall [[C]], [[cc]].\ [[CC : (DD; GG => A) ~> (empty ; empty => nat) ~~> cc]] \Longrightarrow \kleq{[[cc{e1}]]}{[[cc{e2}]]})
  \end{align*}
  }%
\end{definition}

% \noindent In other words, for all possible experiments $[[ cc ]]$, the outcome of an
% experiment on $[[e1]]$ is the same as the outcome on $[[e2]]$
% (i.e., $\kleq{[[cc{e1}]]}{[[cc{e2}]]}$).

% \begin{proof}
%   By induction on the derivation of disjointness. The most interesting case is the variable rule:
%   \[
%     \drule{D-tvarL}
%   \]
%   By the definition of $[[p]]$, we know $[[p(X)]]$ is a monotype. If $[[B]]$ is
%   a polytype, then it follows easily from the definition of logical relation. If
%   $[[B]]$ is also a monotype, we know $[[p(X)]]$ and $[[p(A)]]$ are disjoint by
%   definition. Then by \cref{lemma:covariance:disjoint} and $[[A <: B]]$,
%   we have $[[p(X)]]$ and $[[p(B)]]$ are also disjoint. Finally we apply
%   \cref{lemma:disjoint:mono}.
% \end{proof}

% \paragraph{Compatibility.}

% Firstly we need the compatibility lemmas. Most of them are standard and are thus
% omitted. We show only two compatibility lemmas that are specific to our setting:

% \begin{lemma}[Coercion compatibility] \label{lemma:co-compa} % APPLYCOQ=COERCION_COMPAT
%   Suppose that $[[A1 <: A2 ~~> c]]$,
%   \begin{itemize}
%   \item If $[[DD ; GG |- e1 == e2 : A1 ; A0]]$ then $[[DD ; GG |- c e1 == e2 : A2 ; A0]]$.
%   \item If $[[DD ; GG |- e1 == e2 : A0 ; A1]]$ then $[[DD ; GG |- e1 == c e2 : A0 ; A2]]$.
%   \end{itemize}
% \end{lemma}
% % \begin{proof}
% %   By induction on the subtyping derivation.
% % \end{proof}

% \begin{lemma}[Merge compatibility] % APPLYCOQ=MERGE_COMPAT
%   If $[[ DD ;   GG |- e1 == e1' : A ]]$, $[[  DD ; GG |- e2 == e2' : B ]]$ and $[[ DD |- A ** B ]]$,
%   then $[[ DD ;  GG |- < e1, e2 > == <e1', e2'> : A & B ]]$.
% \end{lemma}
% \begin{proof}
%   By the definition of logical relation and \cref{lemma:disjoint}.
% \end{proof}


% \paragraph{Fundamental property.}

% The ``Fundamental Property'' states that any well-typed expression is related to
% itself by the logical relation. In our elaboration setting, we rephrase it so
% that any two \tnamee terms elaborated from the \emph{same} \fnamee expression
% are related To prove it, we require \cref{thm:uniq}.

% \begin{theorem} \label{thm:uniq}
%   If $[[DD ; GG |- ee => A1]]$ and $[[DD ; GG |- ee => A2]]$, then $[[A1]] \equiv_\alpha [[A2]]$.
% \end{theorem}

% \begin{theorem}[Fundamental property] We have that:
%   \begin{itemize}
%   \item If $[[DD; GG |- ee => A ~~> e]]$ and $[[DD; GG |- ee => A ~~> e']]$, then $[[DD; GG |- e == e' : A ]]$.
%   \item If $[[DD ; GG |- ee <= A ~~> e]]$ and $[[DD ; GG |- ee <= A ~~> e']]$, then $[[DD; GG |- e == e' : A ]]$.
%   \end{itemize}
% \end{theorem}


% We show that logical equivalence is preserved by \fnamee contexts:

% \begin{theorem}[Congruence]
%  If $[[CC : (DD ; GG dirflag A) ~> (DD' ; GG' dirflag' A') ~~> cc]]$, $[[DD ; GG |- ee1 dirflag A ~~> e1]]$, $[[DD ; GG |- ee2 dirflag A ~~> e2]]$
%  and $[[DD ; GG |- e1 == e2 : A ]]$, then $[[DD' ; GG' |- cc{e1} == cc{e2} : A']]$.
% \end{theorem}

\paragraph{Coherence.}

For space reasons, we directly show the coherence statement of \fnamee.
We need several technical lemmas such as compatibility lemmas, fundamental property, etc.
The interested reader can refer to our Coq formalization.

\begin{theorem}[Coherence] \label{thm:coherence:fi}
  We have that
  \begin{itemize}
  \item If $[[DD ; GG |- ee => A ]]$ then $[[DD ; GG |- ee ~= ee : A]]$.
  \item If $[[DD ; GG |- ee <= A ]]$ then $[[DD ; GG |- ee ~= ee : A]]$.
  \end{itemize}
\end{theorem}
\noindent That is, coherence is a special case of \cref{def:cxtx2} where
$[[ee1]]$ and $[[ee2]]$ are the same. At first glance, this
appears underwhelming: of course $[[ee]]$ behaves the same as itself! The tricky
part is that, if we expand it according to \cref{def:cxtx2}, it is not $[[ee]]$
itself but all its translations $[[e1]]$ and $[[e2]]$ that behave the same!




% Local Variables:
% org-ref-default-bibliography: "../paper.bib"
% End:

\section{Algorithmic Type System}
\label{sec:algorithm}

\begin{figure}[t]
  \centering
  \begin{small}
\begin{tabular}{lrcl} \toprule
  Expressions & $e$ & \syndef & $x \mid n \mid
                         \blam x A e \mid \erlam x e \mid e~e \mid e : A $ \\
  Types & $A, B$ & \syndef & $ \nat \mid a \mid \genA \mid A \to B \mid \forall a. A \mid \unknown$ \\
  Monotypes & $\tau, \sigma$ & \syndef & $ \nat \mid a \mid \genA \mid \tau \to \sigma$ \\
  Contexts & $\Gamma, \Delta, \Theta$ & \syndef & $\ctxinit \mid \tctx,x: A \mid \tctx, a \mid \tctx, \genA \mid \tctx, \genA = \tau$ \\
  Complete Contexts & $\Omega$ & \syndef & $\ctxinit \mid \Omega,x: A \mid \Omega, a \mid \Omega, \genA = \tau$ \\ \bottomrule
\end{tabular}
  \end{small}
\caption{Syntax of the algorithmic system}
\label{fig:algo-syntax}
\end{figure}


% The declarative type system in \cref{sec:type-system} serves as a good
% specification for how typing should behave. It remains to see whether this
% specification delivers an algorithm. The main challenge lies in the rules \rul{CS-ForallL} in
% \cref{fig:decl:conssub} and rule \rul{M-Forall} in
% \cref{fig:decl-typing}, which both need to guess a monotype.

% \bruno{why are we not highlightinh the differences in gray anymore?}
In this section we give a bidirectional account of the algorithmic type system
that implements the declarative specification. The algorithm is largely inspired
by the algorithmic bidirectional system of \citet{dunfield2013complete}
(henceforth DK system). However our algorithmic system differs from theirs in
three aspects: 1) the addition of the unknown type $\unknown$; 2) the use of the
matching judgment; and 3) the approach of \textit{gradual inference only
  producing static types}~\citep{garcia2015principal}. We then prove that our
algorithm is both sound and complete with respect to the declarative type
system. Full proofs can be found in the appendix.

\paragraph{Algorithmic Contexts.}

The algorithmic context $\Gamma$ is an
\textit{ordered} list containing declarations of type variables $a$ and term
variables $x : A$. Unlike declarative contexts, algorithmic contexts also
contain declarations of existential type variables $\genA$, which can be either
unsolved (written $\genA$) or solved to some monotype (written $\genA = \tau$).
Complete contexts $\Omega$ are those that contain no unsolved existential type
variables. \Cref{fig:algo-syntax} shows the syntax of the algorithmic system.
Apart from expressions in the declarative system, we have annotated expressions
$e : A$.

% \paragraph{Notational convenience}
% Following \citet{dunfield2013complete}, we use contexts as substitutions on
% types. We write $\ctxsubst{\Gamma}{A}$ to mean $\Gamma$ applied as a
% substitution to type $A$. We also use a hole notation, which is useful when
% manipulating contexts by inserting and replacing declarations in the middle. The
% hole notation is used extensively in proving soundness and completeness. For
% example, $\Gamma[\Theta]$ means $\Gamma$ has the form $\Gamma_L, \Theta,
% \Gamma_R$; if we have $\Gamma[\genA] = (\Gamma_L, \genA, \Gamma_R)$, then
% $\Gamma[\genA = \tau] = (\Gamma_L, \genA = \tau, \Gamma_R)$.

% \paragraph{Input and output contexts}
% The algorithmic system, compared with the declarative system, includes similar
% judgment forms, except that we replace the declarative context $\Psi$ with an
% algorithmic context $\Gamma$ (the \textit{input context}), and add an
% \textit{output context} $\Delta$ after a backward turnstile. For example,
% $\Gamma \vdash A \tconssub B \dashv \Delta$ is the judgment form for the
% algorithmic consistent subtyping, and so on. All rules manipulate input and
% output contexts in a way that is consistent with the notion of \textit{context
%   extension}, which is described in \cref{sec:ctxt:extension}.

% We start with the explanation of the algorithmic consistent subtyping as it
% involves manipulating existential type variables explicitly (and solving them if
% possible).

\subsection{Algorithmic Consistent Subtyping and Instantiation}
\label{sec:algo:subtype}

\begin{figure}[t]
  \centering
  \begin{small}
  %   \begin{mathpar}
  % \framebox{$\Gamma \vdash A$} \\
  % \VarWF \and \IntWF \and \UnknownWF \and \FunWF \and \ForallWF \and \EVarWF
  % \and \SolvedEVarWF
  %   \end{mathpar}

\begin{mathpar}
  \framebox{$\Gamma \vdash A \tconssub B \toctxr$} \\
  \ACSTVar \and \ACSExVar \and \ACSInt \quad \ACSUnknownL \quad \ACSUnknownR \and
  \ACSFun \and \ACSForallR \and \ACSForallL \and \AInstantiateL \quad \AInstantiateR
\end{mathpar}
  \end{small}
  \caption{Algorithmic consistent subtyping}
  \label{fig:algo:subtype}
\end{figure}

\Cref{fig:algo:subtype} shows the algorithmic consistent subtyping rules.
The first five rules do not manipulate contexts. % Rules \rul{ACS-TVar} and
% \rul{ACS-Int} do not involve existential variables, so the output context
% remains unchanged. Rule \rul{ACS-ExVar} says that any unsolved existential
% variable is a consistent subtype of itself. The output is still the same as the
% input context as this gives no clue as to what is the solution of that
% existential variable.
% Rules \rul{ACS-UnknownL} and \rul{ACS-UnknownR} are the verbatim
% correspondences of rule \rul{CS-UnknownL} and \rul{CS-UnknownR}.
Rule \rul{ACS-Fun} is a natural extension of its declarative counterpart. The
output context of the first premise is used by the second premise, and the
output context of the second premise is the output context of the conclusion.
Note that we do not simply check $A_2 \tconssub B_2$, but apply $\Theta$
% (the input context of the second premise)
to both types (e.g., $\ctxsubst{\Theta}{A_2} $). This is
to maintain an important invariant that types
% : whenever we try to derive $\Gamma \vdash A \tconssub B \dashv \Delta$, the types $A$ and $B$
are fully applied
under input context $\Gamma$ (they contain no existential variables already solved in
$\Gamma$). The same invariant applies to every algorithmic judgment.
Rule \rul{ACS-ForallR} looks similar to its declarative counterpart, except that
we need to drop the trailing context $a, \Theta$ from the concluding output
context since they become out of scope.
% again, bears a similarity with the declarative
% version. Note that the output context of its premise allows additional elements
% to appear after the type variable $a$, in a trailing context $\Theta$. Since $a$
% becomes out of scope in the conclusion, we need to drop the trailing context
% $\Theta$ together with $a$ from the concluding output context, resulting in
% $\Delta$.
% The next rule is essential to eliminating the guessing work, thus appears
% significantly different from its declarative version. Instead of guessing a
% monotype $\tau$ out of thin air,
Rule \rul{ACS-ForallL} generates a fresh
existential variable $\genA$, and replaces $a$ with $\genA$ in the body $A$. The
new existential variable $\genA$ is then added to the premise's input context.
% Unlike rule \rul{ACS-ForallR}, the output context $\Delta$ of the premise
% remains unchanged in the conclusion.
% A central idea behind this rule is that we
% defer the decision of choosing a monotype for a type variable, and hope that it
% could be solved later when we have more information at hand.
As a side note, when both types are quantifiers, then either \rul{ACS-ForallR}
or \rul{ACS-ForallR} could be tried. In practice, one can apply
\rul{ACS-ForallR} eagerly.
The last two rules % are specific to the algorithm, thus having no counterparts in
% the declarative version. They
together check consistent subtyping with an
unsolved existential variable on one side and an arbitrary type on the other
side by the help of the instantiation judgment. % Apart from checking that the existential variable does not occur in the
% type $A$, both of the rules do not directly solve the existential variables, but
% leave the real work to the instantiation judgment.

% \subsection{Instantiation}
% \label{sec:algo:instantiate}

\begin{figure}[t]
  \centering
  \begin{small}
\begin{mathpar}
  \framebox{$\tctx \vdash \genA \unif A \toctxr$} \\
  % {\quad \text{Under input context $\Gamma$, instantiate $\genA$ such that
  %     $\genA \tconssub A$, with output context $\Delta$ }} \\
  \InstLSolve \and \InstLReach \and \InstLSolveU   \and \InstLAllR \and \InstLArr
\end{mathpar}

% \begin{mathpar}
%   \framebox{$\tctx \vdash A \unif \genA  \toctxr$} \\
%   % {\quad \text{Under input context $\Gamma$, instantiate $\genA$ such that
%   %     $A \tconssub \genA$, with output context $\Delta$}} \\
%   \InstRSolve \and \InstRReach \and \InstRSolveU  \and \InstRAllL \and \InstRArr
% \end{mathpar}

  \end{small}
  \caption{Algorithmic instantiation}
  \label{fig:algo:instantiate}
\end{figure}

% A central idea of the algorithmic system is to defer the decision of picking a
% monotype to as late as possible.
The judgment $\Gamma \vdash \genA \unif A \dashv \Delta$ defined in
\cref{fig:algo:instantiate} instantiates unsolved existential variables.
Judgment $\genA \unif A$ reads ``instantiate $\genA$ to a consistent subtype of
$A$''. For space reasons, we omit its symmetric judgement $\Gamma \vdash A \unif
\genA \dashv \Delta$.
% Since these two are mutually defined, we
% discuss them together, and omit symmetric rules when convenient.
Rule \rul{InstLSolve} and rule \rul{InstLReach} set $\genA$ to
$\tau$ and $\genB$ in the output context, respectively.
% is the simplest
% one -- when an existential variable meets a monotype. In that case, we simply
% set the solution of $\genA$ to the monotype $\tau$ in the output context. We
% also need to check that the monotype $\tau$ is well-formed under the prefix
% context $\Gamma$.
Rule \rul{InstLSolveU} is similar to \rul{ACS-UnknownR} in that we put no
constraint on $\genA$ when it meets the unknown type $\unknown$. This design
decision reflects the point that type inference only produces static
types~\citep{garcia2015principal}. We will get back to this point in
\cref{subsec:algo:discuss}.
% Rule \rul{InstLReach} deals with the situation where two existential variables
% meet. Note that $\Gamma[\genA][\genB]$ denotes a context where some unsolved existential
% variable $\genA$ is declared before $\genB$. In this situation, the only logical
% thing we can do is to set the solution of one existential variable to the other
% one, depending on which is declared before which. For example, in the output
% context of rule \rul{InstLReach}, we have $\genB = \genA$ because in the input
% context, $\genA$ is declared before $\genB$.
Rule \rul{InstLAllR} is the instantiation version of rule \rul{ACS-ForallR}.
% Since our system is predicative, $\genA$ cannot be instantiated to $\forall b.
% B$, but we can decompose $\forall b. B$ in the same way as in \rul{ACS-ForallR}.
% Rule \rul{InstRAllL} is the instantiation version of rule \rul{ACS-ForallL}.
The last rule \rul{InstLArr} applies when $\genA$ meets a function type. It
follows that the solution must also be a function type.
% looks a bit complicated, but it is actually very
% intuitive: what does the solution of $\genA$ look like when $A$ is a function
% type? The solution must also be a function type!
That is why, in the first premise, we generate two fresh existential variables
$\genA_1$ and $\genA_2$, and insert them just before $\genA$ in the input
context, so that the solution of $\genA$ can mention them. Note that $A_1 \unif
\genA_1$ switches to the other instantiation judgment.


% \paragraph{Example}

% We show a derivation of $\Gamma[\genA] \vdash \forall b. b \to \unknown \unif
% \genA$ to demonstrate the interplay between instantiation, quantifiers and the
% unknown type:
% \[
%   \inferrule*[right=InstRAllL]
%       {
%         \inferrule*[right=InstRArr]
%         {
%           \inferrule*[right=InstLReach]{ }{\Gamma', \genB \vdash \genA_1 \unif \genB \dashv \Gamma' , \genB = \genA_1} \\
%           \inferrule*[right=InstRSolveU]{ }{\Gamma', \genB = \genA_1 \vdash \unknown \unif \genA_2 \dashv \Gamma', \genB = \genA_1}
%         }
%         {
%           \Gamma[\genA], \genB \vdash \genB \to \unknown \unif \genA \dashv \Gamma', \genB = \genA_1
%         }
%       }
%       {
%         \Gamma[\genA] \vdash \forall b. b \to \unknown \unif \genA \dashv \Gamma', \genB = \genA_1
%       }
% \]
% where $\Gamma' = \Gamma[\genA_2, \genA_1, \genA = \genA_1 \to \genA_2]$. Note
% that in the output context, $\genA$ is solved to $\genA_1 \to \genA_2$, and
% $\genA_2$ remains unsolved because the unknown type $\unknown$ puts no
% constraint on it. Essentially this means that the solution of $\genA$ can be any
% function, which is intuitively correct since $\forall b. b \to \unknown$ can be
% interpreted, from the parametricity point of view, as any function.

\subsection{Algorithmic Typing}
\label{sec:algo:typing}

\begin{figure}[t]
  \centering
  \begin{small}
\begin{mathpar}
  \framebox{$\Gamma \vdash e \Rightarrow A \toctxr $} \\
  % {\quad \text{Under input context $\Gamma$, $e$ synthesizes output type $A$,
  %     with output context $\Delta$}} \\
  \AVar \and \ANat \and \ALamU \and \ALamAnnA \and \AAnno \and \AApp
\end{mathpar}
\begin{mathpar}
  \framebox{$\Gamma \vdash e \Leftarrow A \toctxr $} \\
  % {\quad \text{Under input context $\Gamma$, $e$ synthesizes output type $A$,
  %     with output context $\Delta$}} \\
  \ALam \and \AGen \and \ASub
\end{mathpar}
\begin{mathpar}
  \framebox{$\Gamma \vdash A \match A_1 \to A_2 \toctxr$} \\
  % {\quad \text{Under input context $\Gamma$, $A$ synthesizes output type $A_1
  %     \to A_2$, with output context $\Delta$}} \\
  \AMMC \quad \AMMA \and \AMMB \and \AMMD
\end{mathpar}
  \end{small}
  \caption{Algorithmic typing}
  \label{fig:algo:typing}
\end{figure}

We now turn to the algorithmic typing rules in \cref{fig:algo:typing}. The
algorithmic system uses bidirectional type checking to accommodate polymorphism.
Most of them are quite standard.
% All of them are direct analogies of their declarative counterparts. Rules \rul{AVar}
% and \rul{ANat} do not generate any new information, thus the output context is
% the same as the input context. Rule \rul{ALamAnnA} infers the type of a lambda
% abstraction. It does so by pushing $x : A$ into the input context and continues
% to infer the type of the body $B$. The output context in the premise has
% additional declarations in the trailing context $\Theta$, which is discarded in
% the concluding output context.
Perhaps rule \rul{AApp} (which differs significantly from that in the DK system)
deserves attention. It relies on the algorithmic matching judgment $\Gamma
\vdash A \match A_1 \to A_2 \dashv \Delta$.
% The matching judgment
% algorithmically synthesizes a function type from an arbitrary type.
Rule
\rul{AM-ForallL} replaces $a$ with a fresh existential variable $\genA$, thus
eliminating guessing. Rule \rul{AM-Arr} and \rul{AM-Unknown} correspond
directly to the declarative rules.
% self-explanatory. Rule
% \rul{AM-Unknown} says that the unknown type $\unknown$ can be split into a
% function type $\unknown \to \unknown$.
Rule \rul{AM-Var}, which has no
corresponding declarative version, is similar to \rul{InstRArr}/\rul{InstLArr}:
we create $\genA$ and $\genB$ and add $\genC = \genA \to \genB$ to the context.

% Back to \rul{AApp}. This rule first infers the type of $e_1$, producing a output
% context $\Theta_1$. Then it applies $\Theta_1$ to $A$ and goes into the matching
% judgment, which delivers a function type $A_1 \to A_2$ and another output
% context $\Theta_2$. $\Theta_2$ is used as the input context when inferring the
% type of $e_2$. The last premise algorithmically checks if
% $\ctxsubst{\Theta_3}{A_3}$ is a consistent subtype of
% $\ctxsubst{\Theta_3}{A_1}$. $A_2$ and $\Delta$ are the concluding output type
% and the concluding output context, respectively.


% \section{Soundness and Completeness}
% \label{sec:sound:complete}

% To be confident that our algorithmic type system and the declarative type system
% accept exactly the same programs, we need to prove that the algorithmic rules
% are sound and complete with respect to the declarative specifications. Before we
% give the formal statements of the soundness and completeness theorems, we need a
% meta-theoretical device, called \textit{context extension}~\cite{dunfield2013complete}, to help capture a notion of
% information increase from input contexts to output contexts.

% \subsection{Context Extension}
% \label{sec:ctxt:extension}


% A context extension judgment $\Gamma \exto \Delta$ reads ``$\Gamma$ is extended
% by $\Delta$''. Intuitively, this judgment says that $\Delta$ has at least as
% much information as $\Gamma$: some unsolved existential variables in $\Gamma$
% may be solved in $\Delta$. (The full inductive definition can be found in the
% supplementary material. We refer the reader to \citet[][Section
% 4]{dunfield2013complete} for further explanations of context extension.)

\subsection{Completeness and Soundness}

We prove that the algorithmic rules are sound and complete with
respect to the declarative specifications. We need an auxiliary judgment
$\Gamma \exto \Delta$ that captures a notion of information increase from input
contexts $\Gamma$ to output contexts $\Delta$~\citep{dunfield2013complete}.

\paragraph{Soundness.} Roughly speaking, soundness of the algorithmic system says
that given an expression $e$ that type checks in the algorithmic system, there exists
a corresponding expression $e'$ that type checks in the declarative system.
However there is one complication: $e$ does not necessarily have more annotations
than $e'$. For example, by \rul{ALam} we have $\erlam{x}{x} \chkby (\forall a.
a) \rightarrow (\forall a . a)$, but $\erlam{x}{x}$ itself cannot have type
$(\forall a. a) \rightarrow (\forall a . a)$ in the declarative system. To
circumvent that, we add an annotation to the lambda abstraction, resulting in
$\blam{x}{(\forall a . a)}{x}$, which is typeable in the declarative system with
the same type. To relate $\erlam{x}{x}$ and $\blam{x}{(\forall a . a)}{x}$, we
erase all annotations on both expressions. The definition of erasure $\erase{\cdot}$ is
standard and thus omitted.

% \jeremy{mention erasure and why (talk about \rul{ALam} and \rul{ASub})}


% \begin{restatable}[Instantiation Soundness]{mtheorem}{instsoundness} \label{thm:inst_soundness}%
%   Given $\Delta \exto \Omega$ and $\ctxsubst{\Gamma}{A} = A$ and $\genA \notin \mathit{fv}(A)$:
%   \begin{itemize}
%   \item If $\Gamma \vdash \genA \unif A \dashv \Delta$ then $\ctxsubst{\Omega}{\Delta} \vdash \ctxsubst{\Omega}{\genA} \tconssub \ctxsubst{\Omega}{A}$.
%   \item If $\Gamma \vdash A \unif \genA \dashv \Delta$ then $\ctxsubst{\Omega}{\Delta} \vdash \ctxsubst{\Omega}{A} \tconssub \ctxsubst{\Omega}{\genA}$.
%   \end{itemize}
% \end{restatable}

% Notice that the declarative judgment uses $\ctxsubst{\Omega}{\Delta}$, a
% operation that applies a complete context $\Omega$ to the algorithmic context
% $\Delta$, essentially plugging in all known solutions and removing all
% declarations of existential variables (both solved and unsolved), resulting in a
% declarative context.

% With instantiation soundness, next we show that the algorithmic consistent
% subtyping is sound:

% \begin{restatable}[Soundness of Algorithmic Consistent Subtyping]{mtheorem}{subsoudness} \label{thm:sub_soundness}%
%   If $\Gamma \vdash A \tconssub B \toctxr$ where $\ctxsubst{\tctx}{A} = A$ and
%   $\ctxsubst{\tctx}{B} = B$ and $\ctxr \exto \cctx$ then
%   $\ctxsubst{\cctx}{\Delta} \vdash \ctxsubst{\cctx}{A} \tconssub
%   \ctxsubst{\cctx}{B}$.
% \end{restatable}

% At this point, we are ``two thirds of the way'' to proving the ultimate theorem.
% The remaining third concerns with the soundness of matching:

% \begin{restatable}[Matching Soundness]{mtheorem}{matchsoundness}  \label{thm:match_soundness}%
%   If $\Gamma \vdash A \match A_1 \to A_2 \dashv \Delta$ where
%   $\ctxsubst{\Gamma}{A} = A$ and $\Delta \exto \Omega$ then
%   $\ctxsubst{\Omega}{\Delta} \vdash \ctxsubst{\Omega}{A} \match
%   \ctxsubst{\Omega}{A_1} \to \ctxsubst{\Omega}{A_2}$.
% \end{restatable}


% Finally the soundness theorem of algorithmic typing is:

\begin{restatable}[Soundness of Algorithmic Typing]{mtheorem}{typingsoundness} \label{thm:type_sound}
  Given $\ctxr \exto \cctx$,

  \begin{enumerate}
  \item If $\Gamma \vdash e \infto A \toctxr$ then $\exists e'$ such
    that $\ctxsubst{\cctx}{\Delta} \vdash e' : \ctxsubst{\cctx}{A}$ and
    $\erase{e} = \erase{e'}$.
  \item If $\Gamma \vdash e \chkby A \toctxr$ then $\exists e'$ such
    that $\ctxsubst{\cctx}{\Delta} \vdash e' : \ctxsubst{\cctx}{A}$ and
    $\erase{e} = \erase{e'}$.
  \end{enumerate}


\end{restatable}


\paragraph{Completeness.}
Completeness of the algorithmic system is the reverse of soundness: given a
declarative judgment of the form $\ctxsubst{\Omega}{\Gamma} \vdash
\ctxsubst{\Omega} \dots $, we want to get an algorithmic derivation of $\Gamma
\vdash \dots \dashv \Delta$. It turns out that completeness is a bit trickier to
state in that the algorithmic rules generate existential variables on the fly,
so $\Delta$ could contain unsolved existential variables that are not found in
$\Gamma$, nor in $\Omega$. Therefore the completeness proof must produce another
complete context $\Omega'$ that extends both the output context $\Delta$, and
the given complete context $\Omega$. As with soundness, we need erasure to
relate both expressions.

% \jeremy{talk about \rul{Gen}}

% \begin{restatable}[Instantiation Completeness]{mtheorem}{instcomplete}  \label{thm:inst_complete}%
%   Given $\Gamma \exto \Omega$ and $A = \ctxsubst{\Gamma}{A}$ and $\genA \in
%   \mathit{unsolved}(\Gamma)$ and $\genA \notin \mathit{fv}(A)$:
%   \begin{enumerate}
%   \item If $\ctxsubst{\Omega}{\Gamma} \vdash \ctxsubst{\Omega}{\genA} \tconssub
%     \ctxsubst{\Omega}{A}$ then there exist $\Delta$, $\Omega'$ such that $\Omega \exto
%     \Omega'$ and $\Delta \exto \Omega'$ and $\Gamma \vdash \genA \unif A \dashv \Delta$.
%   \item If $\ctxsubst{\Omega}{\Gamma} \vdash \ctxsubst{\Omega}{A} \tconssub
%     \ctxsubst{\Omega}{\genA}$ then there exist $\Delta$, $\Omega'$ such that $\Omega \exto
%     \Omega'$ and $\Delta \exto \Omega'$ and $\Gamma \vdash A \unif \genA \dashv \Delta$.
%   \end{enumerate}
% \end{restatable}


% Next is the completeness of consistent subtyping:

% \begin{restatable}[Generalized Completeness of Subtyping]{mtheorem}{subcomplete}  \label{thm:sub_completeness}%
%   If $\Gamma \exto \Omega$ and $\Gamma \vdash A$ and $\Gamma \vdash B$ and
%   $\ctxsubst{\Omega}{\Gamma} \vdash \ctxsubst{\Omega}{A} \tconssub
%   \ctxsubst{\Omega}{B}$ then there exist $\Delta$, $\Omega'$ such that $\Delta
%   \exto \Omega'$ and $\Omega \exto \Omega'$ and $\Gamma \vdash
%   \ctxsubst{\Gamma}{A} \tconssub \ctxsubst{\Gamma}{B \dashv \Delta}$.
% \end{restatable}


% We prove that the algorithmic matching is complete with respect to the
% declarative matching:

% \begin{restatable}[Matching Completeness]{mtheorem}{matchcomplete} \label{thm:match_complete}%
%   Given $\Gamma \exto \Omega$ and $\Gamma \vdash A$, if
%   $\ctxsubst{\Omega}{\Gamma} \vdash \ctxsubst{\Omega}{A} \match A_1 \to A_2$
%   then there exist $\Delta$, $\Omega'$, $A_1'$ and $A_2'$ such that $\Gamma
%   \vdash \ctxsubst{\Gamma}{A} \match A_1' \to A_2' \dashv \Delta$ and $\Delta \exto \Omega'$ and
%   $\Omega \exto \Omega'$ and $A_1 = \ctxsubst{\Omega'}{A_1'}$ and $A_2 =
%   \ctxsubst{\Omega'}{A_2'}$.
% \end{restatable}


% Finally here is the completeness theorem of the algorithmic typing:

\begin{restatable}[Completeness of Algorithmic Typing]{mtheorem}{typingcomplete}  \label{thm:type_complete}
  Given $\Gamma \exto \Omega$ and $\Gamma \vdash A $, if
  $\ctxsubst{\Omega}{\Gamma} \vdash e : A$ then there exist $\Delta$,
  $\Omega'$, $A'$ and $e'$ such that $\Delta \exto \Omega'$ and $\Omega \exto \Omega'$
  and $\Gamma \vdash e' \infto A' \dashv \Delta$ and $A = \ctxsubst{\Omega'}{A'}$ and $\erase{e} = \erase{e'}$.
\end{restatable}





%%% Local Variables:
%%% mode: latex
%%% TeX-master: "../paper"
%%% org-ref-default-bibliography: "../paper.bib"
%%% End:

% 
\section{Discussion}
\label{sec:discuss}

Discuss the limitations/issues of \name. The differences from the original trait model.

\begin{itemize}
\item Override: Form a new trait by layering additional methods over an existing
  trait. This operation is an asymmetric sum.
\item Exclusion: forms a new trait by removing a method from an existing trait
\end{itemize}


\begin{itemize}
\item Traits have no proper notation of inheritance relationship (no super keyword)
\item Traits have no nice syntax of redefining
\item traits allow annotation of type, then term declarations don't need to
\end{itemize}
\section{Related Work}
\label{sec:related}

% \bruno{I think (part of) this text can be discussed in here instead:


There are multiple flavours of inheritance. To avoid confusion, since the same
terminology is often used in the literature to mean different things, we use the
following 3 terms when comparing related work with ours.

\begin{itemize}
\item{{\bf Static inheritance:}} Static inheritance refers to what the typical
  model of inheritance in class-based languages. The inheritance model is said
  to be static because when using class extension, the extended classes are
  statically known at compile-time.
\item{{\bf Mutable Inheritance:}} Prototype-based languages allow another model
  of inheritance, which we call \emph{mutable inheritance}. In this inheritance
  model, self-references are mutable and changeable at any point.
\item{{\bf Dynamic Inheritance:}} Dynamic inheritance is a less well-known model
  which stands in between static and mutable inheritance. Unlike the static
  inheritance model, with dynamic inheritance objects can inherit from other
  objects which are not statically known. However, unlike mutable inheritance,
  the self-reference is not mutable and cannot be arbitrarily changed at
  run-time.
\end{itemize}

Figure~\ref{fig:comparision} shows the comparison between \name and various
similar languages that follow \citeauthor{cook1989inheritance}'s ``Inheritance is not
Subtyping'' (i.e. the flexible model), as we will explain below.

\begin{figure}[t]
  \centering
  \begin{small}
  \begin{tabular}{|l||c|c|c|c|}
    \hline
    & \bf{Statically typed} & \bf{Polymorphism} & \bf{Meta-theory} & \bf{Inheritance}  \\
    \hline
    \name & \cmark & \cmark & \cmark & Dynamic \\
    \hline
    \textsc{Self} & \xmark & \xmark & \xmark & Mutable \\
    \hline
    Cecil & \cmark & \cmark & \xmark & Static \\
    \hline
    Cook's Modula-3 & \cmark & \xmark & \xmark & Static \\
    \hline
    IFJ & \cmark & \xmark & \cmark & Dynamic \\
    \hline
    \textsc{Darwin} & \cmark & \xmark & \xmark & Dynamic \\
    \hline
  \end{tabular}
  \end{small}
  \caption{Comparison between \name and various similar languages that
  adopt the \emph{flexible model}.}
  \label{fig:comparision}
\end{figure}



% \paragraph{Dynamically-typed Languages with Delegation Mechanism}

% \begin{itemize}
% \item Clojure Protocols
%   % http://www.ibm.com/developerworks/library/j-clojure-protocols/
% \item Ruby mixin
% \item JS mixin
% \end{itemize}

% They are all dynamically typed.


\paragraph{Delegation-based languages}

\citet{lieberman1986using} is the first to promote the use of prototypes and
delegation as the mechanism to code sharing between objects. Since then many
researchers have studied the mechanisms of
delegation~\cite{wegner1987dimensions,malenfant1995semantic,goldberg1989smalltalk}.
\textsc{Self}~\cite{ungar1988self} is a dynamically typed, prototype-based
language with a simple and uniform object model. \textsc{Self}'s inheritance
model is typical of what we call mutable inheritance, because an object's parent
slots may be assigned new values at run-time. Mutable inheritance is rather
unstructured, and oftentimes access to any clashing methods will generate a
``messageAmbiguous'' error at run-time. Although \name's dynamic inheritance is
not as powerful as mutable inheritance, its static type system can guarantee
that no such errors occur at run-time.

There is not much work on statically-typed, delegation-based languages.
\citet{kniesel1999type} provides a good overview of problems when combining
delegation with a static type discipline. Cecil~\cite{chambers1992object,
  chambers1993cecil} is a prototype-based language, where delegation is the
mechanism for method call and code reuse. Cecil supports a polymorphic static
type system, although no meta-theory of any kind is given. Its type system is
able to detect statically when a message might be ambiguously defined as a
result of multiple inheritance or multiple dispatching. However, one major
omission of Cecil, which is also one of the interesting features of \name, is
dynamic inheritance. There are other
works~\cite{fisher1995delegation,anderson2003can} on delegation in a
statically-typed setting, but none of them provide means (such as the merge
construct, disjointness constraints, etc.) that are needed for extensible
designs.

\citet{cook1989inheritance} were the first to propose a typed model of
inheritance where subtyping and inheritance are two separate concepts. In
particular, they introduce the notion of \textit{type inheritance} and show that
inherited objects have inherited types, not subtypes. An interesting aspect of
their calculus is the \textbf{with} construct, used to join two records. This is
somewhat similar to our merge construct. However two major differences are worth
pointing out: 1) the \textbf{with} construct operates only on records; and 2) it
is a biased operator, favoring values from its right argument. This biased
operator is good for modelling mixins, but not traits. The
\textbf{with} construct seems to be unable to merge two arbitrary (and possible
polymorphic) values, since this seems to require something like
\emph{row polymorphism}~\cite{wand1987complete,wand1989type}, which is not available in their language.
The \textit{onion} construct in the Big Bang
language~\cite{palmer2015building,menon2012big} has a similar bias problem -- it is a
left-associative operator which gives rightmost precedence to one
implementation when conflicts exist.

\paragraph{Mixin-based inheritance}

Mixins have become very popular in many OO languages
~\cite{flatt1998classes,bono1999core, ancona2003jam}. \citeauthor{bracha1990mixin}'s
seminal paper~\citep{bracha1990mixin} extends Modula-3 with mixins. Mixins are subclasses parameterized
over a superclass, and used to produce a variety of classes with the same
functionality and behaviour. Mixin-based inheritance requires that mixins be
composed linearly, and as such, conflicts are resolved implicitly (mixins
appearing later overwrite all the identically named features of earlier mixins).
In comparison, the trait model in \name requires conflicts be resolved
explicitly. We want to emphasize that this conflict detection is essential in
expressing composition operators for Object Algebras, without running
into ambiguities.


\paragraph{Trait-based inheritance}

The seminar paper by \citet{scharli2003traits} introduced the ideas behind
traits, where they also documented an implementation of the trait
mechanism in a dynamically typed version of Smalltalk. Since then many
formalizations of traits have been
proposed~\cite{scharli2003traitsformal,ducasse2006traits,bettini2010prototypical}.
For example \citet{fisher2004typed} presented a statically-typed calculus that
models traits. Conflict detection is the hallmark of trait-based
inheritance, compared with mixin-based inheritance. One important difference
with \name is that those systems support \textit{classes} in addition to traits,
and consider the interaction between them, whereas \name is 
delegation based and the mechanism for code reuse is purely traits
(i.e., there are no classes in \name). The
deviation from traditional class-based models is not only because of its
simplicity, but also because we need a very \textit{dynamic} form of
inheritance, as has been elaborated throughout the paper.

Compared to the traditional trait mode, traits in \name have the following
differences: 1) traditional traits cannot be instantiated but only composed with
a class, whereas traits in \name can be instantiated directly; 2) traditional
traits cannot take constructor parameters whereas ours can; 3) the trait system
in \name lacks a proper notation of inheritance relationship. For example in the
traditional trait model, if the same method (i.e., from the same trait) is
obtained more than once via different paths, there is no conflict. This is not
the case in \name; and 4) traits in \name support dynamic
inheritance. 
%In the
%traditional trait model, when it comes to inheritance, the traits being
%inherited must be statically known.




% \citet{flatt1998classes} proposed MIXEDJAVA, an extension to a subset of
% sequential Java called CLASSICJAVA with mixins. In their model, mixins
% completely subsume the role of classes (classes are mixins that do not inherit
% any services). One interesting aspect in their system is that two identically
% named methods are allowed to coexist, and are resolved at run-time with run-time
% context information provided by the current \textit{view} of an object. In
% comparison, conflicts in \name are detected statically, and resolved by the
% programmers. Like \name, their model also enforces the distinction between
% implementation inheritance and subtyping.

% \citet{bono1999core} develop an imperative class-based calculus that provides a
% formal model for both single and mixin inheritance. Objects are represented by
% records and produced by instantiating classes. In their calculus, the class
% construct is extensible but not subtypable, while objects are subtypable but not
% extensible. Like \name, their system has a clean separation between subtyping
% and inheritance. Also, their type system does not have polymorphism.

% \citet{ancona2003jam} extends the Java language to support mixins, called Jam.
% Since Jam is an upward-compatible extension of Java 1.0, it is inheritantly a
% covariant mode. Unlike MIXEDJAVA, mixins can be only instantiated on classes,
% and there is no notion of mixin composition.


\begin{comment}

\begin{itemize}


\item ``Object-Oriented Multi-Methods in Cecil''

\item ``Dimensions of Object-Based Language Design''

\item ``On the Semantic Diversity of Delegation-Based Programming Languages''

\item ``Self: The power of simplicity''

\item ``Type-safe delegation for run-time component adaptation''

\item ``A delegation-based object calculus with subtyping''

\item ``Can Addresses be Types? (a case study: objects with delegation)''

\item ``Inheritance is not subtyping''


Mixins

\item ``mixin-based inheritance''

\item ``Classes and mixins''

\item ``A core calculus of classes and mixins''

\item ``A core calculus of higher-order mixins and classes''

\item ``Jam—Designing a Java Extension with Mixins''



\end{itemize}

Do they have polymorphic type systems? Do they support mutable self reference?

\end{comment}


\paragraph{Class-based languages with more advanced forms of inheritance}

Incomplete Featherweight Java (IFJ), proposed by \citet{bettini2008type}, is a
conservative extension of Featherweight Java with incomplete objects. Besides
standard classes, programmers can also define incomplete classes, whose
instances are incomplete objects. Incomplete objects can be composed (by object
composition) with complete objects, yielding new complete objects at run-time,
while ensuring statically that the composition is type-safe. Incomplete objects
are quite flexible, and support dynamic inheritance. However, object composition
in IFJ is quite restrictive, compared to \name, in that it can only compose an
incomplete object with a complete object. In that regard, and also because IFJ's
type system is not polymorphic, IFJ is unable to encode composition operators of
Object Algebras. \citet{kniesel1999type} showed that type-safe integration of
delegation with subtyping into a class-based model is possible, resulting in the
\textsc{Darwin} model. In \textsc{Darwin}, the type of the parent object must be
a declared class and this limits the flexibility of dynamic composition.
\citeauthor{ostermann2002dynamically}'s delegation
layers~\citep{ostermann2002dynamically} use delegation for doing dynamic
composition in a system with virtual classes. This is in contrast with most
other approaches that use class-based composition, but closer to the dynamic
composition that we use in \name.

There are many other class-based OO languages that are equipped with more
advanced forms of
inheritance~\cite{meyer1987eiffel,buchi2000generic,ostermann2001object}. Most of
them are heavyweight and are specific to classes. \name is object-centered, more
lightweight, and is dedicated to express extensible designs in a simpler way.


% Eiffel~\cite{meyer1987eiffel} is a class-based language that is based on the
% identification of classes with types and of inheritance with subtyping. Eiffel
% supports multiple inheritance, with the restriction that name collisions are
% considered programming errors, and ambiguities must be resolved explicitly by
% the programmer (by means of renaming). In this regard, \name is quite like
% Eiffel. However, the type system in \name is more lenient in that two
% identically named methods with different signatures can coexist without any
% problems.

% \citet{kniesel1999type} is the first to show that type-safe integration of
% delegation with subtyping into a class-based model is possible, resulting in the
% DARWIN model. In the DARWIN model, the type of the parent object must be a
% declared class and this limits the flexibility of dynamic composition, whereas
% in \name, the merge operator can merge/compose any objects. Another difference
% with \name lies in the conflict resolution, where DARWIN relies on method
% overriding with the assumption that the author of the overriding method is aware
% of the effect.

% Generic wrappers~\cite{buchi2000generic} supports aggregating objects at
% run-time. In their model, once a ``wrappee'' is assigned to a ``wrapper'', the
% wrappee is fixed. GBETA~\cite{ernst2000gbeta} has some dynamic features that are
% related to delegation. Like Generic wrappers, parents in GBETA are fixed at
% run-time.

% \citet{ostermann2001object} proposed compound references (CR) as a abstraction
% for object references, which provides explicit linguistic support for combining
% different composition properties on-demand. The model is statically typed, and
% decouples subtype declaration from implementation reuse.


% \citet{ostermann2002dynamically} proposed delegation layers as an approach to
% decompose a collaboration into layers and compose these layers dynamically at
% run-time. This combines and generalizes delegation and virtual classes concepts.

% \citet{ostermann2008nominal} compared the nominal and structural subtyping
% mechanisms. They argue nominal subtyping gives more safety guarantee, whereas
% structural subtyping is more flexible from a component-based perspective. The
% type system of \name chooses structural subtyping.

\paragraph{Intersection types, polymorphism and the merge construct}

There is a large body of work on intersection types. Here we only talk about
work that have direct influences on ours. \citet{dunfield2014elaborating} shows
significant expressiveness of type systems with intersection types and a merge
construct. However his calculus lacks coherence. The limitation was addressed
by~\citet{oliveira2016disjoint}, where they introduced the notion of
disjointness to ensure coherence. The combination of intersection types, a merge
construct and parametric polymorphism, while achieving coherence was first
studied in the \bname calculus~\cite{alpuimdisjoint}, where they proposed the
notion of disjoint polymorphism. \bname serves as the theoretical foundation of
\name.


\begin{comment}

\begin{itemize}

\item Eiffel

\item ``Delegation by object composition'' (IFJ) and ``Type safe dynamic object
  delegation in class-based languages''

\item ``Dynamically composable collaborations with delegation layers''

\item ``Generic wrappers''

\item ``Object-Oriented Composition Untangled''

\item ``gbeta - a language with virtual attributes, Block Structure, and Propagating, Dynamic Inheritance''

\item ``Nominal and Structural Subtyping in Component-Based Programming''

\item ``Engineering a programming language: The type and class system of Sather ''

\item ``Big Bang Designing a Statically-Typed Scripting Language''

\item ``Building a Typed Scripting Language''



\end{itemize}

\end{comment}


\section{Conclusions and Future Work}
\label{sec:conclusion}

We have proposed \name, a type-safe and coherent calculus with disjoint
intersection types, and support for nested composition/subtyping. \name
improves upon earlier work with a more
flexible notion of disjoint intersection types, which leads to
a clean and elegant formulation of the type system. Due to the added
flexibility we have had to employ a more powerful proof method based on logical
relations to rigorously prove coherence.
We also show how \name supports essential features of family
polymorphism, such as nested composition. We believe \name provides insights into family polymorphism, and
has potential for practical applications for extensible software designs.

A natural direction for future work is to enrich \name with parametric
polymorphism. There is abundant literature on logical relations for parametric
polymorphism~\citep{reynolds1983types} and we foresee no fundamental
difficulties in extending our proof method.\footnote{
Our prototype
  implementation already supports polymorphism, but we
  are still in the process of extending our Coq development with polymorphism. } The resulting calculus will be
more expressive than \fname. An interesting application that we intend to investigate
is native support for \textit{object algebras}~\citep{oliveira2012extensibility}
(or the finally tagless approach~\citep{CARETTE_2009}). For example, we can
define the object algebra interfaces for the Expression Problem example in
\cref{sec:overview} as follows:
\lstinputlisting[linerange=75-76]{../../impl/examples/overview.sl}% APPLY:linerange=LANG_EXT_INTER
By instantiating \lstinline{E} with \lstinline{IPrint}, i.e.,
\lstinline{ExpAlg[IPrint]}, we get the interface of the \lstinline{Lang} family.
In that sense, object algebra interfaces can be viewed as family interfaces.
Moreover, combing algebras implementing \lstinline{ExpAlg[IPrint]} and
\lstinline{ExpAlg[IEval]} to form \lstinline{ExpAlg[IPrint & IEval]} is trivial
with nested composition. Polymorphism also improves code reuse across expressions in the
base and extended languages. For example, the following creates two expressions,
one in the base language, the other in the extended language:
\lstinputlisting[linerange=81-82]{../../impl/examples/overview.sl}% APPLY:linerange=LANG_EXT
Notice how we can  reuse \lstinline{e1} of the base language in the definition
of \lstinline{e2}.



% \jeremy{creating expressions using base and extended expressions, and show more reuse}

% \jeremy{future work} \jeremy{mention in passing this rule is unsound with
%   effects, see ``Intersection types and computational effects''}

% Local Variables:
% mode: latex
% TeX-master: "../paper"
% End:


%%
%% Bibliography
%%

%% Please use bibtex,

\bibliography{paper}


% \ifdefined\submitoption
\newpage
\appendix
\section{Full Specification of Core Language}

\subsection{Syntax}
\gram{\otte\ottinterrule
        \ottG\ottinterrule
        \ottv}
\\[2.0mm]
Syntactic Sugar\\
\resizebox{\columnwidth}{!}{$\ottcoresugar$} % defined in otthelper.mng.tex

\subsection{Operational Semantics}
\ottdefnstep{}
\ottusedrule{\ottdruleSXXMu{}}

\subsection{Typing}
\ottdefnctx{}\ottinterrule
\ottdefnexpr{}
\ottusedrule{\ottdruleTXXMu{}}

\section{Proofs about Core Language}
\subsection{Properties}
We follow the naming of lemmas and proofs of properties 
for Pure Type System from \cite{handbook}. Some lemmas have other well-known names, like
Lemma \ref{lem:appendix:thin} is often called \emph{Weakening} and 
Lemma \ref{lem:appendix:gen} is often called \emph{Inversion}.

\begin{comment}
\begin{lem}[Free Variable]\label{lem:appendix:free}
    If $[[G |- e:t]]$, then $\FV(e) \subseteq \dom([[G]])$ and $\FV([[t]])
\subseteq \dom([[G]])$.
\end{lem}

\begin{proof}
    By induction on the derivation of $[[G |- e:t]]$. We only treat cases
\ruleref{T\_Mu}, \ruleref{T\_CastUp} and \ruleref{T\_CastDown} (since proofs of
other cases are the same as \cc \cite{handbook}):
    \begin{description}
        \item[Case \ruleref{T\_Mu}:] From premises of $[[G |- (mu x:t.e1) :
t]]$, by the induction hypothesis, we have $\FV(e_1) \subseteq \dom([[G]]) \cup
\{[[x]]\}$ and $\FV(\tau) \subseteq \dom([[G]])$. Thus the result follows by
$\FV([[mu x:t.e1]])=\FV(e_1) \setminus \{[[x]]\} \subseteq \dom([[G]])$ and
$\FV(\tau) \subseteq \dom([[G]])$.
        \item[Case \ruleref{T\_CastUp}:] Since $\FV([[castup [t]
e1]])=\FV([[e1]])$, the result follows directly by the induction hypothesis.
        \item[Case \ruleref{T\_CastDown}:] Since $\FV([[castdown
e1]])=\FV([[e1]])$, the result follows directly by the induction hypothesis.
    \end{description}
\end{proof}
\end{comment}

\begin{lem}[Thinning]\label{lem:appendix:thin}
    Let $[[G]]$ and $[[G']]$ be legal contexts such that $[[G]] \subseteq
[[G']]$. If $[[G |- e : t]]$ then $[[G' |- e : t]]$.
\end{lem}

\begin{proof}
    By trivial induction on the derivation of $[[G |- e : t]]$.
\end{proof}

\begin{lem}[Substitution]\label{lem:appendix:subst}
	If $[[G1, x:T, G2 |- e1:t]]$ and $[[G1 |- e2:T]]$, then $[[G1, G2 [x |-> e2]
|- e1[x |-> e2]  : t[x |-> e2] ]]$.
\end{lem}

\begin{proof}
    By induction on the derivation of $[[G1, x:T, G2 |- e1:t]]$. We use the notation $[[e* == e
[x |-> e2] ]]$ to denote the substitution for short. Then the result can be written as \[ [[G1, G2* |- e1*  : t* ]]\]
We only treat cases \ruleref{T\_Mu}, \ruleref{T\_CastUp} and
\ruleref{T\_CastDown} since other cases can be easily followed by the proof for PTS in \cite{handbook}.
Consider the last step of derivation of the following
cases:
    \begin{description}
        \item[Case \ruleref{T\_Mu}:] $\inferrule{[[G1, x:T, G2, y:t |- e1:t]] \\
[[G1, x:T, G2 |- t:s]]}{[[G1, x:T, G2 |- (mu y:t.e1): t]]}$ 
        
        By the induction hypothesis, we have $[[G1, G2*, y:t* |- e1* : t*]]$ and $[[G1,
G2* |- t* : star]]$. Then by the derivation rule, $[[G1, G2* |- (mu
y:t*.e1*):t*]]$. Thus we can conclude $[[G1, G2* |- (mu y:t.e1)*:t*]]$.
        \item[Case \ruleref{T\_CastUp}:] $\inferrule{[[G1, x:T, G2 |- e1:t2]]
\\ [[G1, x:T, G2 |- t1:s]] \\ [[t1 --> t2]]}{[[G1, x:T, G2 |- (castup [t1]
e1):t1]]}$ 
        
        By the induction hypothesis, we have $[[G1, G2* |- e1*:t2*]]$, $[[G1, G2*
|- t1*:star]]$ and $[[t1 --> t2]]$. By the definition of substitution, we can
obtain $[[t1* --> t2*]]$ by $[[t1 --> t2]]$. Then by the derivation rule, $[[G1,
G2* |- (castup [t1*] e1*):t1*]]$. Thus we can conclude $[[G1, G2* |- (castup [t1]
e1)*:t1*]]$.
        \item[Case \ruleref{T\_CastDown}:] $\inferrule{[[G1, x:T, G2 |- e1:t1]]
\\ [[G1, x:T, G2 |- t2:s]] \\ [[t1 --> t2]]}{[[G1, x:T, G2 |- (castdown
e1):t2]]}$ 
        
        By the induction hypothesis, we have $[[G1, G2* |- e1*:t1*]]$, $[[G1, G2*
|- t2*:star]]$ and $[[t1 --> t2]]$ thus $[[t1* --> t2*]]$. Then by the
derivation rule, $[[G1, G2* |- (castdown e1*):t2*]]$. Thus we can conclude $[[G1, G2* |-
(castdown e1)*:t2*]]$.
    \end{description}
\end{proof}

\begin{lem}[Generation]\label{lem:appendix:gen}
If the alpha equivalence is witnessed by notation $[[=a]]$, we have the following results:
\begin{enumerate}[(1)]
	\item If $[[G |- x:T]]$, then there exist an expression $[[t]]$ such that $[[t
=a T]]$, $[[G |- t:s]]$ and $[[x:t elt G]]$.
	\item If $[[G |- e1 e2:T]]$, then there exist expressions $[[t1]]$ and
$[[t2]]$ such that $[[G |- e1 : (Pi x:t2.t1)]]$, $[[G |- e2:t2]]$ and $[[T =a
t1[x |-> e2] ]]$.
	\item If $[[G |- (\x:t1.e):T]]$, then there exist an expression $[[t2]]$ such
that $[[T =a Pi x:t1.t2]]$ where $[[G |- (Pi x:t1.t2):s]]$ and $[[G,x:t1 |-
e:t2]]$.
    \item If $[[G |- (Pi x:t1.t2):T]]$, then $[[T == s]]$, $[[G |- t1:s]]$ and
$[[G, x:t1 |- t2:s]]$.
	\item If $[[G |- (mu x:t.e):T]]$, then $[[G |- t:s]]$, $[[T =a t]]$ and $[[G,
x:t|-e:t]]$.
	\item If $[[G |- (castup [t1] e):T]]$, then there exist an expression $[[t2]]$
such that $[[G |- e:t2]]$, $[[G |- t1:s]]$, $[[t1 --> t2]]$ and $[[T =a t1]]$.
	\item If $[[G |- (castdown e):T]]$, then there exist expressions
$[[t1]],[[t2]]$ such that $[[G |- e:t1]]$, $[[G |- t2:s]]$, $[[t1 --> t2]]$ and
$[[T =a t2]]$.
\end{enumerate}
\end{lem}

\begin{proof}
    Consider a derivation of $[[G |- e:T]]$ for one of cases in the lemma. We
follow the process of derivation until expression $[[e]]$ is introduced the
first time. The last step of derivation can be done by
    \begin{itemize}
        \item rule \ruleref{T\_Var} for case 1;
        \item rule \ruleref{T\_App} for case 2;
        \item rule \ruleref{T\_Lam} for case 3;
        \item rule \ruleref{T\_Pi} for case 4;
        \item rule \ruleref{T\_Mu} for case 5;
        \item rule \ruleref{T\_CastUp} for case 6;
        \item rule \ruleref{T\_CastDown} for case 7.
    \end{itemize}
    In each case, assume the conclusion of the rule is $[[G' |- e : t']]$ where
$[[G']] \subseteq [[G]]$ and $[[t' =a T]]$. Then by inspection of used
derivation rules and Lemma \ref{lem:appendix:thin}, it can be shown that the
statement of the lemma holds and is the only possible case.
\end{proof}

\begin{lem}[Correctness of Types]\label{lem:appendix:corrtyp}
    If $[[G |- e:t]]$ then $[[t == s]]$ or $[[G |- t : s]]$.
\end{lem}

\begin{proof}
    Trivial induction on the derivation of $[[G |- e:t]]$ using Lemma
\ref{lem:appendix:gen}.
\end{proof}

\subsection{Decidability of Type Checking}
\begin{lem}[Decidability of One-step Reduction]\label{lem:appendix:unired}
	The one-step reduction $[[-->]]$ is called decidable if 
given $[[e]]$ there is a unique $[[e']]$ such that $[[e --> e']]$ or there is no such $[[e']]$.
\end{lem}

\begin{proof}
	By induction on the structure of $[[e]]$:
	\begin{description}
        \item[Case $[[e=x]]$:] $[[e]]$ is a variable which does not match any rules of $[[-->]]$. 
        Thus there is no $[[e]]'$ such that $[[e-->e']]$.
		\item[Case $[[e=v]]$:] $[[e]]$ is a value that has one of the following forms:
		\begin{inparaenum}[(1)]
		    \item $[[star]]$,
			\item $[[\x:t.e]]$,
			\item $[[Pi x:t1.t2]]$,
			\item $[[castup [t] e]]$.
		\end{inparaenum}
		Thus, it does not match any rules of $[[-->]]$. Then there is no $[[e]]'$ such that $[[e-->e']]$.
		\item[Case $[[e]]=[[(\x:t.e1) e2]]$:] Since the first term $[[\x:t.e1]]$ is a value, rule \ruleref{S\_App} does not apply to this case. Thus, only rule \ruleref{S\_Beta} can be applied and there is a unique $[[e']]=[[ e1[x|->e2] ]]$.
		\item[Case $[[e]]=[[castdown (castup [t] e1)]]$:] Since the inner term $[[castup [t] e1]]$ is a value, rule \ruleref{S\_CastDown} does not apply to this case. Thus, only rule \ruleref{S\_CastDownUp} can be applied and there is a unique $[[e']]=[[e1]]$.
		\item[Case $[[e]]=[[mu x:t.e1]]$:] Only rule \ruleref{S\_Mu} can be applied. Thus, there is a unique $[[e]]'=[[e1[x|->mu x:t.e1] ]]$.
		\item[Case $[[e]]=[[e1 e2]]$ and $[[e1]]$ is not a $\lambda$-term:] If
$[[e1]]=v$ and is not a $\lambda$-term, there is no rule to reduce $[[e]]$. 
Then there is no $[[e1']]$ such that $[[e1 --> e1']]$, which does not satisfy the premise of 
rule \ruleref{S\_App}. Thus, there is no $[[e]]'$ such that $[[e-->e']]$.

		Otherwise, if $[[e1]]$ is not a value, there exists some $[[e1']]$ such that $[[e1 --> e1']]$. By the
induction hypothesis, $[[e1']]$ is the unique reduction of $[[e1]]$. Thus, by rule
\ruleref{S\_App}, $[[e]]'=[[e1' e2]]$ is the unique reduction of $[[e]]$.
		\item[Case $[[e]]=[[castdown e1]]$ and $[[e1]]$ is not a $[[castup]]$-term:] If
$[[e1]]=v$ and is not a $[[castup]]$-term, there is no rule to reduce $[[e]]$. 
Then there is no $[[e1']]$ such that $[[e1 --> e1']]$, which does not satisfy the premise of 
rule \ruleref{S\_CastDown}. Thus, there is no $[[e]]'$ such that $[[e-->e']]$.

        Otherwise, if $[[e1]]$ is not a value, there exists some $[[e1']]$ such that $[[e1 --> e1']]$. By the
induction hypothesis, $[[e1']]$ is the unique reduction of $[[e1]]$. Thus, by rule
\ruleref{S\_CastDown}, $[[e]]'=[[castdown e1']]$ is the unique reduction of $[[e]]$.
	\end{description}
\end{proof}

\begin{thm}[Decidability of Type Checking]
	There is an algorithm which given $[[G]], [[e]]$ computes the unique
$[[t]]$ such that $[[G |- e:t]]$ or reports there is no such $[[t]]$.
\end{thm}

\begin{proof}
	By induction on the structure of $[[e]]$:
	\begin{description}
	    \item[Case $[[e=star]]$:] Trivial by applying \ruleref{T\_Ax} and $[[t ==
star]]$.
		\item[Case $[[e=x]]$:] Trivial by rule \ruleref{T\_Var}. If $[[x:t elt G]]$, then $[[t]]$ is the
unique type of $[[x]]$ such that $[[G |- x : t]]$. Otherwise, if $[[x]] \not \in \dom([[G]])$, there is no such $[[t]]$.
		\item[Case $[[e]]=[[e1 e2]]$:] By rule \ruleref{T\_App} and induction
hypothesis, there exist unique $[[t1]]$ and $[[t2]]$ such that $[[G
|- e1 : (Pi x:t1.t2)]]$, $[[G |- e2:t1]]$. Thus, $[[t2[x |-> e2] ]]$ is the unique type of $[[e]]$ such that $[[G |- e : t2[x |-> e2] ]]$.
		\item[Case $[[e=\x:t1.e1]]$:] By rule \ruleref{T\_Lam} and induction
hypothesis, there exist unique $[[t2]]$ such that $[[G |- (Pi
x:t1.t2):s]]$ and $[[G,x:t1 |- e:t2]]$. Thus, $[[Pi x:t1.t2 ]]$ is the unique type of $[[e]]$ such that $[[G |- e : Pi x:t1.t2  ]]$.
		\item[Case $[[e=Pi x:t1.t2]]$:] By rule \ruleref{T\_Pi} and induction
hypothesis, we have $[[G |- t1:s]]$ and $[[G, x:t1 |- t2:s]]$. Thus, $[[s]]$ is the unique type of $[[e]]$ such that $[[G |- e : s  ]]$.
		\item[Case $[[e=mu x:t.e1]]$:] By rule \ruleref{T\_Mu} and induction
hypothesis, we have $[[G |- t:s]]$ and $[[G, x:t|-e:t]]$. Thus, $[[t]]$ is the unique type of $[[e]]$ such that $[[G |- e : t]]$.
		\item[Case $[[e]]=[[castup [t1] e1]]$:] From the premises of rule
\ruleref{T\_CastUp}, by the induction hypothesis, we can derive the type of
$[[e1]]$ as $[[t2]]$ by $[[G |- e1:t2]]$, and check whether $[[t1]]$ is legal by $[[G |- t1:star]]$. 
For a legal $[[t1]]$, by Lemma \ref{lem:appendix:unired}, there is
a unique $[[t1']]$ such that $[[t1 --> t1']]$ or there is no such $[[t1']]$. 
If such $[[t1']]$ does not exist, then we report type checking fails. 

Otherwise, we examine if $[[t1']]$ is syntactically equal to $[[t2]]$, 
i.e., $[[t1' =a t2]]$. If the equality
holds, we conclude the unique type of $[[e]]$ is $[[t1]]$, i.e., $[[G |- e:t1]]$. Otherwise, we
report $[[e]]$ fails to type check.
		\item[Case $[[e]]=[[castdown e1]]$:] From the premises of rule
\ruleref{T\_CastDown}, by the induction hypothesis, we can derive the type of
$[[e1]]$ as $[[t1]]$ by $[[G |- e1:t1]]$. By Lemma \ref{lem:appendix:unired}, there is a unique
$[[t2]]$ such that $[[t1 --> t2]]$ or such $[[t2]]$ does not exist. 

If such $[[t2]]$ exists and its sorts is
$[[star]]$, we find the unique type of $[[e]]$ is $[[t2]]$ and can conclude $[[G |- e:t2]]$. Otherwise, we
report $[[e]]$ fails to type check.
	\end{description}
\end{proof}

\subsection{Type Safety}
\begin{dfn}[Multi-step reduction]
    The relation $[[->>]]$ is the transitive and reflexive closure of
$[[-->]]$.
\end{dfn}

\begin{dfn}[$n$-step reduction]
    The $n$-step reduction is denoted by $[[e0]] [[-->>]] [[en]]$, if
    there exists a sequence of one-step reductions $[[e0]] [[-->]]
    [[e1]] [[-->]] [[e2]] [[-->]] \dots [[-->]] [[en]]$, where $n$ is
    a positive integer and $[[ei]]\,(i=0,1,\dots,n)$ are valid
    expressions.
\end{dfn}

\begin{thm}[Subject Reduction]
If $[[G |- e:T]]$ and $[[e]] [[->>]] e'$ then $[[G |- e':T]]$.
\end{thm}

\begin{proof}
    We prove the case for one-step reduction, i.e., $[[e --> e']]$. The theorem
follows by induction on the number of one-step reductions of $[[e]] [[->>]]
[[e']]$.
    The proof is by induction with respect to the definition of one-step
reduction $[[-->]]$ as follows:
    \begin{description}
        \item[Case $\ottdruleSXXBeta{}$:] $\quad$ \\
        Suppose $[[G |- (\x:t1.e1)e2 :T]]$ and $[[G |- e1 [x |-> e2] :T']]$. By
Lemma \ref{lem:appendix:gen}(2), there exist expressions $[[t1']]$ and $[[t2]]$
such that 
        \begin{align}
            &[[G |- (\x:t1.e1):(Pi x:t1'.t2)]] \label{equ:lam} \\
            &[[G |- e2:t1']] \nonumber \\
            &[[T =a t2 [x |-> e2] ]] \nonumber
        \end{align}
        By Lemma \ref{lem:appendix:gen}(3), the judgement (\ref{equ:lam})
implies that there exists an expression $[[t2']]$ such that
        \begin{align}
            &[[Pi x:t1'.t2 =a Pi x:t1.t2']] \label{equ:lameq}\\
            &[[G, x:t1 |- e1:t2']] \nonumber
        \end{align}
        Hence, by (\ref{equ:lameq}) we have $[[t1 =a t1']]$ and $[[t2 =a
t2']]$. Then we can obtain $[[G, x:t1 |- e1:t2]]$ and $[[G |- e2:t1]]$. By
Lemma \ref{lem:appendix:subst}, we have $[[G |- e1[x |-> e2] : t2[x |-> e2]
]]$. Therefore, we conclude with $[[T' =a t2[x |-> e2] ]] [[=a]] [[T]]$.
        
        \item[Case $\ottdruleSXXApp{}$:] $\quad$ \\
        Suppose $[[G |- e1 e2 :T]]$ and $[[G |- e1' e2 :T']]$. By Lemma
\ref{lem:appendix:gen}(2), there exist expressions $[[t1]]$ and $[[t2]]$ such
that 
        \begin{align*}
            &[[G |- e1:(Pi x:t1.t2)]] \\
            &[[G |- e2:t1]]\\
            &[[T =a t2 [x |-> e2] ]]
        \end{align*}
        By the induction hypothesis, we have $[[G |- e1':(Pi x:t1.t2)]]$. By rule
\ruleref{T\_App}, we obtain $[[G |- e1' e2 : t2[x |-> e2] ]]$. Therefore, $[[T'
=a t2[x |-> e2] ]] [[=a]] [[T]]$.
        
        \item[Case $\ottdruleSXXCastDown{}$:] $\quad$ \\
        Suppose $[[G |- castdown e :T]]$ and $[[G |- castdown e' :T']]$. By
Lemma \ref{lem:appendix:gen}(7), there exist expressions $[[t1]], [[t2]]$ such
that 
        \begin{align*}
            &[[G |- e:t1]] \qquad [[G |- t2:s]] \\
            &[[t1 --> t2]] \qquad [[T =a t2 ]]
        \end{align*}
        By the induction hypothesis, we have $[[G |- e':t1]]$. By rule
\ruleref{T\_CastDown}, we obtain $[[G |- castdown e' : t2 ]]$. Therefore, $[[T'
=a t2]] [[=a]] [[T]]$.
        
        \item[Case $\ottdruleSXXCastDownUp{}$:] $\quad$ \\
        Suppose $[[G |- castdown (castup [t1] e) :T]]$ and $[[G |- e :T']]$. By
Lemma \ref{lem:appendix:gen}(7), there exist expressions $[[t1']], [[t2]]$ such
that 
        \begin{align}
            &[[G |- (castup [t1] e):t1']] \label{equ:fold} \\
            &[[t1' --> t2]] \label{equ:foldeq1} \\
            &[[T =a t2 ]] \label{equ:foldeq4}
        \end{align}
        By Lemma \ref{lem:appendix:gen}(6), the judgement (\ref{equ:fold})
implies that there exists an expression $[[t2']]$ such that
        \begin{align}
            &[[G |- e:t2']] \label{equ:foldr} \\
            &[[t1 --> t2']] \label{equ:foldeq2} \\
            &[[t1' =a t1]] \label{equ:foldeq3}
        \end{align}
        By (\ref{equ:foldeq1}, \ref{equ:foldeq2}, \ref{equ:foldeq3}) and Lemma
\ref{lem:appendix:unired} we obtain $[[t2 =a t2']]$. From (\ref{equ:foldr}) we
have $[[T' =a t2' ]]$. Therefore, by (\ref{equ:foldeq4}), $[[T' =a t2' ]]
[[=a]] [[t2 =a T]]$.
        
        \item[Case $\ottdruleSXXMu{}$:] $\quad$ \\
        Suppose $[[G |- (mu x:t.e) :T]]$ and $[[G |- e[x |-> mu x:t.e] :T']]$.
By Lemma \ref{lem:appendix:gen}(5), we have $[[T =a t]]$ and $[[G, x:t |-
e:t]]$. Then we obtain $[[G |- (mu x:t.e) : t]]$. Thus by Lemma
\ref{lem:appendix:subst}, we have $[[G |- e[x |-> mu x:t.e] : t[x |-> mu x:t.e]
]]$.
        
        Note that $[[x]]:[[t]]$, i.e., the type of $[[x]]$ is $[[t]]$, then
$[[x]] \notin \FV([[t]])$ holds implicitly. Hence, by the definition of
substitution, we obtain $[[t[x |-> mu x:t.e] == t]]$. Therefore, $[[T' =a t[x
|-> mu x:t.e] ]] [[==]] [[t =a T]]$.
    \end{description}
\end{proof}

\begin{thm}[Progress]
If $[[empty |- e:T]]$ then either $[[e]]$ is a value $v$ or there exists $[[e]]'$
such that $[[e --> e']]$.
\end{thm}

\begin{proof}
    By induction on the derivation of $[[empty |- e:T]]$ as follows:
    \begin{description}
        \item[Case $[[e=x]]$:] Impossible, because the context is empty.
        \item[Case $[[e=v]]$:] Trivial, since $[[e]]$ is already a value that
has one of the following forms:
		\begin{inparaenum}[(1)]
		    \item $[[star]]$,
			\item $[[\x:t.e]]$,
			\item $[[Pi x:t1.t2]]$,
			\item $[[castup [t] e]]$.
		\end{inparaenum}
		\item[Case $[[e]]=[[e1 e2]]$:] By Lemma \ref{lem:appendix:gen}(2), there
exist expressions $[[t1]]$ and $[[t2]]$ such that $[[empty |- e1:(Pi x:t1.t2)]]$ and
$[[empty |-e2:t1]]$. Consider whether $[[e1]]$ is a value:
    		\begin{itemize}
    		    \item If $[[e1]]=v$, by Lemma \ref{lem:appendix:gen}(3), it must be a
$\lambda$-term such that $[[e1 == \x:t1.e1']]$ for some $[[e1']]$ satisfying
$[[empty |- e1':t2]]$. Then by rule \ruleref{S\_Beta}, we have $[[(\x:t1.e1') e2 -->
e1' [x |-> e2] ]]$. Thus, there exists $[[e' == e1' [x |-> e2] ]]$ such that
$[[e --> e']]$.
    		    \item Otherwise, by the induction hypothesis, there exists $[[e1']]$ such
that $[[e1 --> e1']]$. Then by rule \ruleref{S\_App}, we have $[[e1 e2 --> e1'
e2]]$. Thus, there exists $[[e' == e1' e2]]$ such that $[[e --> e']]$.
    		\end{itemize}
		\item[Case $[[e]]=[[castdown e1]]$:] By Lemma \ref{lem:appendix:gen}(7),
there exist expressions $[[t1]]$ and $[[t2]]$ such that $[[empty |- e1:t1]]$ and
$[[t1 --> t2]]$. Consider whether $[[e1]]$ is a value:
		     \begin{itemize}
    		    \item If $[[e1]]=v$, by Lemma \ref{lem:appendix:gen}(6), it must be a
$[[castup]]$-term such that $[[e1 == castup [t1] e1']]$ for some $[[e1']]$
satisfying $[[empty |- e1':t2]]$. Then by rule \ruleref{S\_CastDownUp}, we can obtain
$[[castdown (castup [t1] e1') --> e1']]$. Thus, there exists $[[e' == e1']]$
such that $[[e --> e']]$.
    		    \item Otherwise, by the induction hypothesis, there exists $[[e1']]$ such
that $[[e1 --> e1']]$. Then by rule \ruleref{S\_CastDown}, we have $[[castdown
e1 --> castdown e1']]$. Thus, there exists $[[e' == castdown e1']]$ such that
$[[e --> e']]$.
    		\end{itemize}
		\item[Case $[[e]]=[[mu x:t.e1]]$:] By rule \ruleref{S\_Mu}, there always
exists $[[e' == e1[x |-> mu x:t.e1] ]]$.
    \end{description}
\end{proof}

\section{Full Specification of Surface Language}
\subsection{Syntax}
See Figure \ref{fig:appendix:syntax}.
\begin{figure*}
\centering
\gram{\ottpgm\ottinterrule
\ottdecl\ottinterrule
\ottu\ottinterrule
\ottp\ottinterrule
\ottE\ottinterrule
\ottGs}
\begin{align*}
&\text{Syntactic Sugar} \\
&\ottsurfsugar % defined in otthelper.mng.tex
\end{align*}
\caption{Syntax of the surface language}
\label{fig:appendix:syntax}
\end{figure*}

\subsection{Expression Typing}
See Figure \ref{fig:appendix:typing}.

\subsection{Translation to the Core}
See Figure \ref{fig:appendix:translate}.

\section{Proofs about Surface Language}
\subsection{Type Safety of the Translation}

\begin{thm}[Type Safety of Expression Translation]
Given a surface language expression $[[E]]$ and context $[[Gs]]$, 
if $[[Gs |- E:A ~> e]]$, $[[Gs |- A:star ~> t]]$ and $[[|- Gs ~> G]]$, then
$[[G |- e:t]]$.
\end{thm}

\begin{proof}
    By induction on the derivation of $[[Gs |- E : A ~> e]]$. Suppose there is
a core language context $[[G]]$ such that $[[|- Gs ~> G]]$.
    \begin{description}
        \renewcommand{\hlmath}[1]{#1}
        \item[Case $\ottdruleTRXXAx{}$:] $\quad$ \\ Trivial. $[[e]] = [[t]] = [[star]]$ and
$[[G |- star:star]]$ holds by rule \ruleref{T\_Ax}.
        \item[Case $\ottdruleTRXXVar{}$:] $\quad$ \\ Trivial. By rule \ruleref{T\_Var}, we
have $[[|- Gs ~> G]]$, then $[[x]]:[[t]] [[elt]] [[G]]$ where $[[Gs |-
A:star~>t]]$.
        \item[Case \resizebox{.9\columnwidth}{!}{$\ottdruleTRXXApp{}$}:] $\quad$ \\ Suppose
            \[\begin{array}{l}
            [[Gs |- E1 E2 : A1[x |-> E2] ~> e1 e2]] \\
            [[Gs |- A1[x |-> E2] : star ~> t1 [x |-> e2] ]].
            \end{array} \]
            By induction
            hypothesis, we have 
            $
            [[G |- e1 : (Pi x:t2.t1)]],
            [[G |- e2:t2]],
            $
            where
            \[\begin{array}{l}
             [[Gs |- E1 : (Pi x:A2.A1) ~> e1]] \\
              [[Gs |- (Pi x:A2.A1) : star ~> (Pi x:t2.t1)]] \\
              [[Gs |- E2 : A2 ~> e2]] \\
              [[Gs |- A2 : star ~> t2]].
            \end{array}\] Thus by rule \ruleref{T\_App}, we can conclude $[[G |- e1 e2 : t1 [x |-> e2] ]]$.
        \item[Case $\ottdruleTRXXLam{}$:] $\quad$ \\ Suppose
            \[\begin{array}{l}
            [[Gs |- (\x:A1.E):(Pi x:A1.A2) ~> \x:t1.e]] \\ 
            [[Gs |- Pi x:A1.A2 : star ~> Pi x:t1.t2]].
            \end{array} \]
            By the induction hypothesis, we have 
            $
            [[G, x : t1 |- e:t2]],
            [[G |- Pi x:t1.t2 : star]]
            $
            where 
            \[
            \begin{array}{ll}
            [[Gs, x : A1 |- E : A2 ~> e]] & \\
            [[Gs |- A1 : star ~> t1]] & [[Gs |- A2 : star ~> t2]] \\
            [[Gs |- (Pi x:A1.A2) : s ~> Pi x:t1.t2]] &
            \end{array}
            \]
            Thus by rule \ruleref{T\_Lam}, we can conclude $[[G |- (\x:t1.e):(Pi x:t1.t2)]]$.
        \item[Case $\ottdruleTRXXPi{}$:] $\quad$ \\ Suppose 
                \[ [[Gs |- (Pi x:A1.A2):r ~> Pi x:t1.t2]]. \] 
            By the induction hypothesis, we have 
            $
                [[G |- t1 : star]], [[G, x : t1 |- t2 : star]]
            $
            where
            $
                [[Gs |- A1 : s ~> t1]], [[Gs, x: A1 |- A2 : r ~> t2]]
            $
            Thus by rule \ruleref{T\_Pi} we can conclude $[[G |- (Pi x:t1.t2) : star]]$.
        \item[Case $\ottdruleTRXXMu{}$:] $\quad$ \\ Suppose 
                \[\begin{array}{l}
                    [[Gs |- (mu x:A . E):A ~> mu x:t.e]] \\
                    [[Gs |- A : star ~> t]]. 
                \end{array}\]
            By the induction hypothesis, we have 
                \[ [[G, x : t |- e : t]],\text{ where }[[Gs, x:A |- E:A ~> e]]. \] 
            Thus by rule \ruleref{T\_Mu}, we can conclude $[[G |- (mu x:t.e) : t]]$.
        \item[Case \resizebox{.9\columnwidth}{!}{$\ottdruleTRXXCase{}$}:] $\quad$ \\ Suppose 
            \[\begin{array}{l}
                [[Gs |- case E1 of << p => E2>> : B ~> (unfoldnp e1) T <<e2>>]] \\
                [[Gs |- B : star ~> T]].
            \end{array}\]
            By the induction hypothesis, we have 
            \[\begin{array}{ll}
                [[Gs |- E1 : D@<<U>>n ~> e1]] &
                [[Gs |- D@<<U>>n : star ~> t1]] \\
                [[G |- e1 : t1]] &
                [[<< Gs |- p => E2 : D@<<U>>n -> B ~> e2 >>]]            
            \end{array}\]
            By rule \ruleref{TRpat\_Alt}, we have
            \begin{align*}
                [[p]] &[[==]] [[K <<x:A[<< u |-> U >>]>>]] \\
                [[<<e2>>]] &[[==]] [[<<\ <<x:t'>> .e>>]]
            \end{align*}
            where
            \[\begin{array}{ll}
                [[<<Gs |- E2 : B ~> e>>]] &
                [[<<G |- e : T>>]] \\
                [[<<Gs |- U : star ~> uu'>>]] &
                [[<<Gs |- A[<< u |-> U >>]:star ~> t[<<uu |-> uu'>>]>>]] \\
                [[t']] [[==]] [[ t[<<uu |-> uu'>>] ]]
            \end{array}\]
            By rule \ruleref{TRdecl\_Data}, we have $[[D]]  [[ == ]] \ottdeclD$. Thus,
            \[ [[t1]] [[==]] [[D]] [[<<uu'>>]]^n,\text{ where }[[<<G |- uu' : ro>>]].\] 
            Note that by operational semantics of \name, the following reduction sequence follows for $[[t1]]$:
            \begin{align*}
                [[D]] [[<<uu'>>]]^n~
                &[[-->]]~ \mathscale[0.7]{[[(\ <<u:ro>>n . (bb:star) -> << ((<<x : t[D |-> X][X |-> D]>>) -> bb) >> -> bb) ]][[<<uu'>>]]^n}\\
                &[[-->>]]~ [[(bb:star) -> << (<<x:t'>>) -> bb >> -> bb]]
            \end{align*}
            Then by
            rule \ruleref{T\_CastDown} and the definition of $n$-step cast operator, the
            type of $[[unfoldnp e1]]$ is \[ [[(bb:star) -> << (<<x:t'>>) -> bb >> -> bb]].\] Note
            that by rule \ruleref{T\_Lam}, $[[G |- e2 : (<<x:t'>>) -> T]]$. Therefore, by rule
            \ruleref{T\_App}, we can conclude $[[G |- (unfoldnp e1) T <<e2>> : T]]$.
    \end{description}
\end{proof}

\begin{figure*}
\renewcommand{\hlmath}[1]{}
\renewcommand{\ottdrulename}[1]{\textsc{\replace{#1}{TR}{TS}}}
\renewcommand{\ottcom}[1]{\text{\replace{#1}{translation}{typing}}}
\ottdefnctxtrans{}\ottinterrule
\ottdefnpgmtrans{}\ottinterrule
\ottdefndecltrans{}\ottinterrule % defined in otthelper.mng.tex
\ottdefnpattrans{}\ottinterrule
\ottdefnexprtrans{}
\caption{Typing rules of the surface language}
\label{fig:appendix:typing}
\end{figure*}

\begin{figure*}
\ottdefnctxtrans{}\ottinterrule
\ottdefnpgmtrans{}\ottinterrule
\ottdefndecltrans{}
\[\hlmath{\ottdecltrans}\]\ottinterrule % defined in otthelper.mng.tex
\ottdefnpattrans{}\ottinterrule
\ottdefnexprtrans{}
\caption{Translation rules of the surface language}
\label{fig:appendix:translate}
\end{figure*}


% \fi

\end{document}
