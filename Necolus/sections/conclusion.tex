
\section{Conclusions and Future Work}
\label{sec:conclusion}

We have proposed \name, a type-safe and coherent calculus with disjoint
intersection types, and support for nested composition/subtyping. It
improves upon earlier work with a more
flexible notion of disjoint intersection types, which leads to
a clean and elegant formulation of the type system. Due to the added
flexibility we have had to employ a more powerful proof method based on logical
relations to rigorously prove coherence.
We also show how \name supports essential features of family
polymorphism, such as nested composition. We believe \name provides insights into family polymorphism, and
has potential for practical applications for extensible software designs.

A natural direction for future work is to enrich \name with parametric
polymorphism. There is abundant literature on logical relations for parametric
polymorphism~\cite{reynolds1983types}. % and we foresee no fundamental difficulties
% in extending our proof method .
The main challenge in the definition of the
logical relation is the clause that relates type variables with arbitrary types. Careful
measures are to be taken to avoid potential circularity due to
impredicativity.\footnote{ Our prototype implementation already
  supports polymorphism, but we are still in the process of extending our Coq
  development with polymorphism.} With the combination of parametric polymorphism and nested
composition, an interesting application that we intend to investigate is native
support for a highly modular form of \textit{Object Algebras}~\cite{oliveira2012extensibility, xuan_traits} and \textsc{Visitor}s
(or the finally tagless approach~\cite{CARETTE_2009}).

Another direction for future work is to add mutable references, which would
affect two aspects of our metatheory: type safety and coherence. For type safety,
we expect that lessons learned from previous work on family polymorphism and
mutability on OO apply to our work. For example, it is well-known that
subtyping in the presence of mutable state often needs restrictions. Given such
suitable restrictions we expect that type-safety in the presence of mutability
still holds. For coherence, it would be a major technical challenge to adjust
our coherence proof and its Coq mechanization: logical relations that account
for mutable state (e.g., see Ahmed's thesis~\cite{ahmed2004semantics}) introduce significant complexity.





% For example, we can
% define the object algebra interfaces for the Expression Problem example in
% \cref{sec:overview} as follows:
% \lstinputlisting[linerange=77-78]{./examples/overview.sl}% APPLY:linerange=LANG_EXT_INTER
% By instantiating \lstinline{E} with \lstinline{IPrint}, i.e.,
% \lstinline{ExpAlg[IPrint]}, we get the interface of the \lstinline{Lang} family.
% In that sense, object algebra interfaces can be viewed as family interfaces.
% Moreover, combining algebras implementing \lstinline{ExpAlg[IPrint]} and
% \lstinline{ExpAlg[IEval]} to form \lstinline{ExpAlg[IPrint & IEval]} is trivial
% with nested composition. Polymorphism also improves code reuse across expressions in the
% base and extended languages. For example, the following creates two expressions,
% one in the base language, the other in the extended language:
% \lstinputlisting[linerange=83-84]{./examples/overview.sl}% APPLY:linerange=LANG_EXT
% Notice how we can  reuse \lstinline{e1} of the base language in the definition
% of \lstinline{e2}.



% \jeremy{creating expressions using base and extended expressions, and show more reuse}

% \jeremy{future work} \jeremy{mention in passing this rule is unsound with
%   effects, see ``Intersection types and computational effects''}

% Local Variables:
% mode: latex
% TeX-master: "../paper"
% org-ref-default-bibliography: ../paper.bib
% End:
