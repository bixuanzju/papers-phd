

\section{\name and IFOs, Informally}\label{sec:overview}

This section informally presents \name programs and their IFO-based
encoding. It also shows how to deal with tail-call elimination.
Sections~\ref{sec:fcore} and \ref{sec:tce} present a formalized
compilation method for a subset of \name (System F) into Middleweight Java, 
a formalized subset of Java, based on the ideas from this section.
Note that, for purposes of presentation, we simplify the encodings shown in this
section slightly compared to what we generate by our formal compilation
method.


\begin{comment}
\subsection{Imperative Functional Objects}

\begin{figure}[t]

% use scala
\begin{lstlisting}
def tfact : Int => Int => Int = acc => n => 
  if (n == 0) acc else tfact(acc*n)(n-1)

def even : Int => Boolean = n =>
    if (n == 0) true else odd(n-1)
    
def odd : Int => Boolean = n =>
    if (n == 0) false else even(n-1)
\end{lstlisting}

\caption{Scala function definitions for computing factorial, even and odd.}

\label{fig:scala_defs}

\end{figure}

Currently, functional programming in the JVM requires some compromises, especially 
when performance or memory are primary considerations. For example, consider the 
Scala functions in Figure~\ref{fig:scala_defs}.
The function \lstinline{tfact} computes the factorial of some number
\lstinline{n} using a standard tail-recursive implementation with an accumulator.  The
functions \lstinline{even} and \lstinline{odd} define a naive
algorithm for detecting whether a number is even or
odd. We define the later two functions using \emph{mutually recursive tail calls}. 
The code uses Scala first-class functions (\lstinline{A => B}), represented by the following trait:
%\footnote{For the purposes of this
%  paper a trait can be interpreted as a Java interface.}:

\begin{lstlisting}
// slightly simplified for presentation
trait Function[A,B] { 
  def apply(x : A) : B
}
\end{lstlisting}

\noindent This trait is a variant of FAOs. A small differerence to 
the \lstinline{FAO} interface presented in Section~\ref{sec:intro}
is that the argument and returned values are typed.
Scala provides a syntactic sugar that makes defining and
applying functions much more convenient than using the
\lstinline{Function} interface directly. The type \lstinline{Int =>Int=> Int} 
corresponds to
\lstinline{Function[Int,Function[Int,Int]]}. In other words, a function
such as \lstinline{tfact} is a function that returns another function. 

%In the code, the function
%\lstinline{tfact} provides a simple recursive implementation of the
%factorial function using an accumulator parameter. Thus computing the
%factorial of a number amounts to calling \lstinline{tfact(0)}. 

While the definitions of the functions in Figure~\ref{fig:scala_defs}
are standard in ML-like languages, they pose problems
in Scala. 

\noindent {\bf Problem 1: Multiple representations of functions} 
The first problem is that, when efficiency is a concern, we need multiple
function representations. JVM
methods are typically a better choice to achieve
performance. Nevertheless, functions as objects are still needed to encode currying
and partial application. 
We can illustrate the performance problem with the definition of
\lstinline{tfact}. In \lstinline{tfact}, every recursive call creates
one new instance of \lstinline{Function} (to encode the second
argument of the function). 

Naive encoding using \lstinline{Function} objects to represent multi-argument
recursive functions can be inefficient. We require an amount of
memory proportional to the number of recursive calls; allocating and
garbage collecting \lstinline{Function} objects incurs a
performance penalty; and invoking methods for each
\lstinline{Function} object adds extra performance penalties.

Alternatively, Scala programmers can use standard 
JVM methods to define \lstinline{tfact}:

\begin{lstlisting}
def tfact2(acc : Int, n : Int) = acc => n => 
  if (n == 0) acc else tfact(acc*n,n-1)
\end{lstlisting}

\noindent This definition is more efficient than 
the one using functions as objects. Indeed in Scala, this is
probably the recommended way to define \lstinline{tfact}. 
However, if we needed to compute, for
example, a list of factorial numbers up to \lstinline{n}, a simple 
functional definition would be:

\begin{lstlisting}
 def facts(n : Int) : List[Int] = 
    List.range(0,n,(x : Int) => x + 1)
      .map(tfact(1))
\end{lstlisting}

\noindent Unfortunately, this definition would not work if we replaced 
\lstinline{tfact} with \lstinline{tfact2}. JVM methods cannot be curried.
Thus to achieve the same effect, we need to change the argument 
of \lstinline{map} to \lstinline{x => tfact(1,x)}, using a functional
object to encode the partial application. 


At the very best, compilers need to be aware of the two possible
representations and try to automatically choose the best representation
for a particular definition or expression. However, it may be
difficult to automatically choose the best representation, since it
depends on how we use the functions. Furthermore, the compilation
process becomes more complicated. 
Some languages, such as Scala,
avoid this problem by offering programmers control over which function
representation to use. However, this makes programming less
transparent: the user now has to decide which representation to
use; and programmers also need to explicitly convert between
these representations. Furthermore, it also makes the language itself 
more complex, since the language constructs that deal 
with we need both representations.



\noindent {\bf Problem 2: Eliminating general tail-calls efficiently
  on the JVM is hard.} We expose the second problem by
\lstinline{even} and \lstinline{odd}. When we call these functions 
on large inputs (typically larger than 10000), the JVM throws a
\lstinline{StackOverflow} exception. In a functional language, we might 
hope that tail-call elimination would prevent \lstinline{StackOverflow}
exceptions and allow tail calls on arbitrarily large
inputs. Unfortunately, the JVM does not provide support for proper 
tail-calls and eliminating tail-calls via other techniques is hard. 
Existing techniques can only deal with direct tail-recursive calls (such as \lstinline{tfact}) effectively on the JVM. However, this excludes 
other classes of useful tail-call functions, including
mutually tail-recursive functions such as \lstinline{even} and \lstinline{odd}.
Although there are some techniques for dealing with general tail calls
in the JVM, these solutions come with important drawbacks. 
For example, trampolines~\cite{} (which provide a general
solution for tail-calls in the JVM) require more time and
memory. Programs using trampolines are usually orders of magnitude times slower
than programs without trampolines.

Generally speaking, tail-call elimination/optimization is essential
for functional programming, for three reasons:

\begin{itemize}

\item {\bf Correctness:} (Proper) tail-call elimination enables programs with tail calls and 
deep call stacks to run without stack overflows. Therefore, 
those programs can terminate normally, instead of with exceptional
behavior.

\item {\bf Space optimization:} (Proper) tail-call elimination removes tail
  calls and removes all stack-space required by these
  calls. Therefore, in contrast to regular calls, tail-calls consume no
  memory.

\item {\bf Time optimization:} When we replace tail calls with jumps, 
the performance of programs greatly improves. Steele's famous 
paper introducing tail call optimization~\cite{Steele1977} illustrated how purely
functional programs written in a functional language can beat
imperative programs written in Fortran using tail-call optimization.

\end{itemize}


The absence of an effective technique for dealing with tail-call
elimination in the JVM means that idiomatic functional programs
have to be rewritten in a different style to achieve correctness and performance.

\end{comment}

\begin{comment}
If we try to compute the
factorial of \lstinline{n}, we need to create \lstinline{n}
\lstinline{Function} objects. Moreover, invoking \lstinline{tfact}
with two arguments will also require \lstinline{2*n} method calls. We
have to invoke the \lstinline{apply} method of the first function and
then the \lstinline{apply} method of the returned (and freshly
created) \lstinline{Function} object.

Generally speaking, a recursive function \lstinline{f} with the form

\begin{lstlisting}
def f : T1 => (*\ldots *) => Tm => Tm1 = 
   a1 => (*\ldots *) => am => (*\ldots *) f(a1',(*\ldots*),am') (*\ldots*)
\end{lstlisting}

\noindent that takes n recursive iterations to terminate requires the
creation of \lstinline{(m-1) * n + 1} function objects and (at least)
\lstinline{m*n} method calls. 
\end{comment}

\begin{comment}
{\bf Summary:}
Although Functional Programming is possible on the JVM
Indeed the Scala language offers various other alternatives for defining
multi-argument functions like \lstinline{tfact} (for example using
standard JVM methods). However these alternatives require programmers 
to change the style in which programs are written, and often 
\end{comment}

\begin{comment}

The Java interface \lstinline{Function} represents a function with input 
\lstinline{x} and some resulting output. With this representation,
we can encode curried functions, partial application and 
pass functions as arguments. However, compared with JVM methods, this
representation can be quite costly in terms of time and memory. Every
time a function (or closure) is needed, we need to allocated a new object.
For recursive functions, this representation can be
particularly costly, since we may need a number of object allocations 
(for functions) proportial to the number of recursive calls.
\bruno{Pointer to Section 2, with examples?} Therefore, programmers
that care about performance, usually try to avoid using this representation
when possible. 

\end{comment}


\subsection{Encoding Functions with IFOs}\label{subsec:ifos}

\begin{comment}
\begin{figure*}
\begin{tabular}[t]{l l}
\begin{lstlisting}
//Naive factorial
Closure tfact = new Closure() {
		void apply() {
			out = new Closure() {
				void apply() {
					Integer n = (Integer) this.x;
					Integer acc = (Integer) tfact.x;
					if (n == 0) {
						out = acc;  
					} else {
						tfact.x = n * acc;
						tfact.apply(); // call first apply
						Closure r = (Closure) tfact.out;
						r.x = n-1;
						r.apply();     // call second apply
						out = r.out;
					}
        }};	
		}
}; 
\end{lstlisting}&
\hspace{10pt}\begin{lstlisting}
  // Multi-argument optimization
	Closure tfact2 = new Closure() {
		{ // Initialization block
			out = new Closure() {
				void apply() {
					Integer n = (Integer) this.x;
					Integer acc = (Integer) tfact2.x;
					if (n == 0) {
						out = acc;
					} else {
						tfact2.x = n * acc;
						//tfact2.apply(); // no call here
						Closure r = (Closure) tfact2.out;
						r.x = n-1;
						r.apply(); // call apply 1 time
						out = r.out;
					}}};
		} 	
		void apply() {} // empty apply method
	};
\end{lstlisting}
\end{tabular}

\caption{Factorial function using IFOs naively (left) and with
  multi-argument optimization (right).}\bruno{Using Closure instead of
Function!}

\label{fig:multi}

\end{figure*}
\end{comment}



%%Using a function representation based on IFOs, it is possible to 
%%have a more efficient representation of functions and tail-call elimination.
%%\paragraph{Naively encoding IFOs}
%Recall the IFOs \lstinline{Function} interface presented in
%Section~\ref{sec:intro}:
%
%\begin{lstlisting}
%abstract class Function { 
%   Object arg, res;
%   abstract void apply();
%}
%\end{lstlisting}

\noindent In \Name, we compile all functions to objects that are
instances of the \lstinline{Function} class presented in 
Section~\ref{sec:intro}. For example, consider a simple identity
function on integers. In \name or System F (extended with integers),
we represent such function as follows:

\vspace{5pt}
$id \equiv (\lambda x : Int).~x$
\vspace{5pt}

\noindent We can manually encode this definition with an IFO in Java as follows: 

\begin{lstlisting}
class Id extends Function
{
    Function x = this;
    public void apply ()
    {
        final Integer y = (Integer) x.arg;
        res = y;
    }
}
\end{lstlisting}

\noindent The \lstinline{arg} field encodes the argument of the
function, whereas the \lstinline{res} field encodes the result. Thus, 
to create the identity function, all we need to do is to copy the
argument to the result. A function invocation such as 
$id~3$ is encoded as follows:

\begin{lstlisting}
Function id = new Id();
id.arg = 3; // setting argument
id.apply();  // invoking apply()
\end{lstlisting}

\noindent The function application goes in two steps:
it first sets the \lstinline{arg} field to \lstinline{3} and then 
invokes the \lstinline{apply()} method. 

\paragraph{Curried Functions.} Of course, a 
fundamental feature in functional programming is currying. 
Therefore in \Name, it is also possible to define 
curried functions, such as:

\vspace{5pt}
$constant \equiv (\lambda x : Int) .~(\lambda y : Int) .~x$
\vspace{5pt}

\noindent Given two integer arguments, this function will always return
the first one. Using IFOs, we can encode $constant$ in Java as follows:

\begin{lstlisting}
class Constant extends Function
{
    Function x = this;
    public void apply ()
    {
        final Integer y = (Integer) x.arg;
        class IConstant extends Function
        {
            Function x1 = this;
            public void apply()
            {
              final Integer y1 = (Integer) x1.arg;
              res = y; // overwrite if: res = x.arg;
            }
        }
        res = new IConstant();
    }
}
\end{lstlisting}

\noindent Here, the first lambda function sets the second one as its result.
The definition of the second \lstinline{apply} method
sets the result of the function to the argument of the first lambda
function. The use of inner classes enforces the lexical scoping of functions.
 We encode an application such as $constant~3~4$ as:

\begin{lstlisting}
Function constant = new Constant();
constant.arg = 3;
constant.apply();
Function f = (Function) constant.res;
f.arg = 4;
f.apply();
\end{lstlisting}

\noindent We first set the argument of the \lstinline{constant} function to $3$. Then,
we invoke the \lstinline{apply} method and store the
resulting function to a variable \lstinline{f}. Finally, we set the
argument of \lstinline{f} to $4$ and invoke \lstinline{f}'s
\lstinline{apply} method.
Note that the alias \lstinline{y} for \lstinline{x.arg} is needed to prevent accidental
overwriting of arguments in partial applications. For example in 
$constant~3~(constant~4~5)$, the inner application $constant~4~5$ would overwrite
$3$ to $4$ if \lstinline{x.arg} was used instead of \lstinline{y},
 and the outer one would incorrectly return $4$ instead of $3$.
\begin{comment}
An alternative way to encode $constant$ is:

\begin{lstlisting}
static class Constant2 extends Function
{
    Function x = this;
    {
        res = new Function() {
            void apply() { res = x.arg; }
        };
    }
    public void apply () {}
}
\end{lstlisting}

\noindent Differently from the previous definition, we set the first
\lstinline{res} field when
initializing \lstinline{constant2}, instead of only after we invoke the
\lstinline{apply} method. This
approach has two benefits: 1) when we create \lstinline{constant2}, the 
second lambda is also initialized; 2) because the
\lstinline{apply} method of the first lambda function becomes 
empty, it is redundant to call it. Therefore, as a consequence of 2), 
the invocation $constant~3~4$ becomes:

\begin{lstlisting}
Function constant2 = new Constant2();
constant2.arg = 3;
// no call needed here
Function g = (Function) constant2.res;
g.arg = 4;
g.apply();
\end{lstlisting}

\noindent Instead of two \lstinline{apply} method calls, we only need one call.
This alternative encoding can improve 
memory and time performance of multi-argument functions. It is especially important
in recursive multi-argument functions. 
With the first encoding, those functions create new
\lstinline{Function} objects on every recursive call and require
multiple \lstinline{apply} methods. This is exactly the same
deficiency as of the FAO encoding. However, the alternative encoding 
avoids these problems.
\end{comment}

\begin{figure*}
\begin{tabular}[t]{l l}
\hspace{30pt}\begin{lstlisting}
//Naive even and odd
class Mutual {
	Function even;
	Function odd;
	class Even extends Function {
		Function x = this;
		public void apply () {
			final Integer n = (Integer) x.arg;
			if (n == 0) {
				res = true;
			}
			else {
				odd.arg = n - 1;
				odd.apply();
				res = odd.res;
			}
		}
	}
	class Odd extends Function {
		Function x = this;
		public void apply () {
			final Integer n = (Integer) x.arg;
			if (n == 0) {
				res = false;
			}
			else {
				even.arg = n - 1;
				even.apply();
				res = even.res;
			}
		}
	}
	{ // initialization block
		odd = new Odd();
		even = new Even();
	}
}
\end{lstlisting}&
\hspace{70pt}\begin{lstlisting}
// tail call elimination
class Mutual {
	Function teven;
	Function todd;
	class TEven extends Function {
		Function x = this;
		public void apply () {
			final Integer n = (Integer) x.arg;
			if (n == 0) {
				res = true;
			}
			else {
				todd.arg = n - 1;
				// tail call
				Next.next = todd;
			}
		}
	}
	class TOdd extends Function {
		Function x = this;
		public void apply () {
			final Integer n = (Integer) x.arg;
			if (n == 0) {
				res = false;
			}
			else {
				teven.arg = n - 1;
				// tail call
				Next.next = teven;
			}
		}
	}
	{ // initialization block
		todd = new TOdd();
		teven = new TEven();
	}
}
\end{lstlisting}
\end{tabular}

\caption{Functions \lstinline{even} and \lstinline{odd} using IFOs naively (left) and with
  tail-call elimination (right).}

\label{fig:even}

\end{figure*}

\paragraph{Partial Function Application.} With curried functions, we can encode partial application easily.
For example, consider the following 
expression:

\vspace{5pt}
$three \equiv constant~3$
\vspace{5pt}

\noindent The code for this partial application is simply:

\begin{lstlisting}
Function constant = new Constant();
constant.arg = 3;
constant.apply();
\end{lstlisting}

\begin{comment}
\noindent If we use the alternative encoding, it is even simpler, since
we do not need to invoke the \lstinline{apply} method:

\begin{lstlisting}
Function constant2 = new Constant2();
constant2.arg = 3;
\end{lstlisting}
\end{comment}

\begin{comment}
We begin by illustrating how it is possible to encode functional
definitions using IFOs without any optimizations. This is helpful to
introduce the key ideas of IFOs. In Figure~\ref{fig:even}, there are
two versions of an encoding of \lstinline{even} and \lstinline{odd}
using IFOs. The encoding on the left is naive and it disallows
tail-call elimination. Each function is encoded as an IFO function
object. The bodies of \lstinline{apply} show how the definitions 
of the functions are encoded. The argument \lstinline{n} of 
the functions is recovered from the field \lstinline{arg} in the 
function object. The last statement in each path of the programs 
sets the \lstinline{res} field.  
\end{comment}

\begin{comment}
While direct encodings of functions using IFOs do not by
themselfves immediatly solve the two problems, it is possible to create
simple optimizations

We illustrate how both of these goals are achieved using the 
the functions 
\end{comment}

\begin{comment}
\paragraph{Multi-argument Optimization}
To remove the inefficiencies that occur with multi-argument functions,
we propose a new optimisation technique. This optimisation technique
ensures that: multi-argument functions require only a constant amount
of memory; and only one \lstinline{apply} method call is needed for
calling multi-argument functions. 

To illustrate the idea of the optimisation, consider the definition of
\lstinline{tfact} on the right-side of Figure \ref{fig:multi}. 
There are two key differences to the definition on the left-side: 
1) the \lstinline{out} field in the first function is set
only once, at the initialisation of the object; 
2) there is only one call to \lstinline{apply()} in each recursive
iteration. The first difference builds on the observation that all 
the objects that are constructed in the recursive calls are the same. So we 
can effectively build a single object instead at the initialisation 
of the object. Since the second function object is now built at the
initialisation of the first function, the \lstinline{apply()} method
of the first function no longer does anything. Therefore, the second 
difference is to remove spurious \lstinline{apply()}
calls to the empty apply methods.
\end{comment}

\paragraph{Recursion.} \name supports simple recursion, as well as
mutual recursion. For example, consider the functions $even$ and $odd$
defined to be mutually recursive:

\vspace{5pt}
\noindent$even \equiv \lambda (n : Int).~{\bf if}~(n=0)~{\bf then}~true~{\bf  else}~odd(n-1)$ \\
\noindent$odd \equiv \lambda (n : Int).~{\bf if}~(n=0)~{\bf then}~false~{\bf else}~even(n-1)$
\vspace{5pt}

\noindent These two functions define a naive algorithm for detecting
whether a number is even or odd. The left-side of
Figure~\ref{fig:even} shows a naive encoding of these two
functions using IFOs in Java. Recursion is encoded using Java's own
recursion: the Java references \lstinline{even} and
\lstinline{odd} are themselves mutually recursive.

\subsection{Tail-call Elimination}

The recursive calls in $even$ and $odd$ are tail calls. 
IFOs present new ways for doing tail-call elimination in the JVM.
The key idea, inspired by Steele's work on encoding tail-call
elimination~\cite{Steele1978}, is to use a simple
auxiliary structure

\begin{lstlisting}
class Next {static Function next = null;}
\end{lstlisting}

\noindent that keeps track of the next call to be executed.  The
right-side of Figure~\ref{fig:even} illustrates the use of the
\lstinline{Next} structure. The code differs from the code on the
left-side on the recursive (tail) calls of \lstinline{even}
and \lstinline{odd}. This is where we make a fundamental use of the fact
that function application is divided into two parts with IFOs.
In tail calls, we set the arguments of the
function, but we delay the \lstinline{apply} method calls.
Instead, the \lstinline{next} field of \lstinline{Next} is set
to the function with the \lstinline{apply} method. The \lstinline{apply}
method is then invoked at the call-site of the functions. For example, 
the following code illustrates the call \lstinline{even 10}:

\begin{lstlisting}
teven.arg = 10;
Next.next = teven;
Function c;
Boolean res;
do {
  c = Next.next;
  Next.next = null;
  c.apply();
} while (Next.next != null);
res = (Boolean) c.res;
\end{lstlisting}

\noindent The idea is that a function call (which is not a tail-call) 
has a loop that jumps back-and-forth into functions. The technique 
is similar to some trampoline approaches in C-like languages. However, 
an important difference to JVM-style trampolines is that utilization of heap space is not growing.
In other words, tail-calls do not create new objects for their execution, which
improves memory and time performance.
Note that this method is \emph{general}: it works for \emph{simple recursive
tail calls}, \emph{mutually recursive tail calls}, and \emph{non-recursive tail calls}.

\begin{comment}
\subsection{Thread Safety}\label{sec:thread-safe}
System F is not a concurrent calculus and the presented technique focuses on sequential
functional-style code compilation. Nevertheless, JVM is inherently a concurrent platform
and the use of mutable fields may seem to prevent thread safety. This would be a major
drawback of IFOs, because, for instance, the generated code could not be safely used from
multi-threading Java applications. Here, we briefly illustrate this is not the case
with two trivial changes.

\noindent Consider the following program: $constant~3~4~\|~constant~5~6$ where the 
$\|$ operator denotes parallel execution (each invocation runs in its own thread).
If the \lstinline{Function} instance representing $constant$ is shared, there is a race
condition and results of function invocations may be incorrect. Depending on the interleaving
of the assignments, one of the threads could overwrite the argument or the result field
of the other. This, however, can be easily fixed: we would enforce a separate instance 
for each thread by allocating the object at its call site rather than at its definition site.
This is illustrated below:

\begingroup
\setlength{\tabcolsep}{-10pt} % Default value: 6pt
\begin{tabular}[t]{l l}
\begin{lstlisting}

//thread unsafe
...
Function constant = 
  new Constant();
...
Thread thread1 = 
  new Thread() {
		public void run() {
			constant.arg = 3;
			constant.apply();
			...
		}
	};
		
Thread thread2 = 
  new Thread() {
		public void run() {
			constant.arg = 5;
			constant.apply();
			...
		}
	};
...
\end{lstlisting}&
\hspace{45pt}\begin{lstlisting}
//thread safe
...
Thread thread1 = 
  new Thread() {
		public void run() {
			Function constant = 
			  new Constant();
			constant.arg = 3;
			constant.apply();
			...
		}
	};
		
Thread thread2 =
  new Thread() {
		public void run() {
			Function constant = 
			  new Constant();
			constant.arg = 5;
			constant.apply();
			...
		}
	};
...
\end{lstlisting}
\end{tabular}
\endgroup

\noindent In the TCE version of the code, there would be an additional race condition
due to the static \lstinline{next} field. Again, we only need a small modification:
the \lstinline{next} field should not be static and \lstinline{Next} must have a local
instance in each thread. Function applications would be done via this local
instance of \lstinline{Next} rather than the class.

For the matter of implementation, instead of creating instances
immediately after function definitions, we would return \emph{thunk}s
to delay memory allocations. Section~\ref{sec:implementation} presents
some implementation details.
\end{comment}
