\section{Introduction}\label{sec:intro}

% Motivation

Efficient compilers for functional languages in platforms such as the JVM
are attractive to both compiler writers and programmers: 
it enables cross-platform development and comes with a
large collection of libraries and tools that took many
man-years to develop. Moreover, programmers can
benefit from functional programming (FP) languages on the
JVM. Functional languages provide
simple, concise and elegant ways to write different algorithms and achieve
code reuse via higher-order functions. They also
offer more opportunities for parallelism, by avoiding the overuse
of side-effects (and shared mutable state) \cite{erlang}.

%There are two main defining characteristics in functional programming.
%Functional Programming promotes a programming style where functions
%are first-class values and can be partially applied and curried. FP
%also promotes the use of recursion instead of mutable state and loops
%to define algorithms.

% What is the Problem?

%The JVM is not transparent in current 

Unfortunately, compilers for functional languages are hard to
implement efficiently in the JVM. Functional Programming promotes a
programming style where \emph{functions are first-class values} and are often
\emph{partially applied} and \emph{curried}. FP also promotes the use of \emph{recursion}
instead of mutable state and loops to define algorithms. The JVM is
not designed to deal with programs that make intensive use of a
functional programming style.

The difficulty in optimizing FP in the JVM means that: 
\emph{while functional programming in the JVM is possible today, 
some compromises are still necessary for writing efficient programs.}
Existing JVM functional languages, including
Scala~\cite{Odersky2014b} 
and Clojure~\cite{Hickey2008}, usually
work around the challenges imposed by the JVM. Those languages give
programmers alternatives to a FP
style. Therefore, performance-aware programmers avoid certain idiomatic
functional styles, which may be costly in those languages, and use
the available alternatives instead.
 
%\paragraph{JVM Challenges}
In particular, two important challenges when writing a
compiler for a functional language targeting the JVM are:

\begin{enumerate}

\item How to efficiently represent first-class functions, currying, and 
partial function application.

\item How to eliminate and/or optimize tail calls.

\end{enumerate}

For the first problem,
there are two standard options to represent functions in the JVM: \emph{JVM methods} and 
\emph{functions as objects} (FAOs).
Encoding first-class functions using only JVM methods directly is
limiting: JVM methods do not support currying and partial function
application directly. 
To support these features, the majority of functional languages or extensions (including Scala,
Clojure, and Java 8) also adopt variants of the functions-as-objects
approach: 
%\footnote{In statically typed languages (like Scala or Java 8)
%an interface like \lstinline{FAO} is typically parametrized by the input and
%output types.}:

\begin{lstlisting}
interface FAO { Object apply(Object arg);}
\end{lstlisting}

\noindent With this representation,
we can encode curried functions, partial application and 
pass functions as arguments.

Nevertheless, encoding functions as objects can have substantial time and memory overheads,
especially when defining multi-argument
functions. Every time a function is needed, we need to
allocate a new object. For recursive functions, this can be
particularly costly, since we may need a number of object allocations
(for functions) proportional to the number of recursive calls.
Therefore, programmers (and compiler writers) that care about
performance try to avoid using this representation when possible.  In
other words, we face a dilemma: FAOs have severe performance
penalties, but they seem unavoidable to support currying and partial function
applications. As a result, languages that care about performance have to
use both representations: JVM methods when possible; and FAOs when
necessary.

For the second problem, neither FAOs nor JVM methods offer a good solution to
deal with \emph{general tail-call elimination}
(TCE)~\cite{Steele1977}. The JVM does not support proper tail calls.
In particular scenarios, such as single, tail-recursive calls, we can
easily achieve an optimal solution in the JVM.  Both Scala and Clojure
provide some support for tail-recursion~\cite{Odersky2014b,recur}.
However, for more general tail calls (such as mutually recursive
functions or non-recursive tail calls), existing solutions can worsen
the overall performance. For example, JVM-style trampolines~\cite{Schinz2001}
(which provide a general solution for tail calls) are significantly
slower than normal calls and consume heap memory for every tail call.
%Programs using trampolines
%are usually several times slower than programs without trampolines.

\begin{comment}

There are many good solutions for solving the two problems 
and compiling FP languages into native code. However, often these
solutions cannot be directly ported to higher-level target platforms,
such as the JVM. The JVM bytecode provides a relatively high-level set
of instructions, which was originally developed with the Java language
in mind. For non-Java languages, and especially languages that use a
different computational model, the available bytecodes can be quite
restrictive.


Existing functional programming languages or extensions on the JVM --
including Scala, Clojure or Java 8 -- provide 

{\bf Existing Approaches to First-Class Functions}
While the JVM already support methods, methods are not first-class
\bruno{Need to be careful here because of JVM 8 and lambdas}. That 
is they cannot directly passed as arguments or support currying. 
In Java 8 ...
Therefore, the majority of existing JVM functional languages~\cite{}
adopt (variations of) the following representation of first-class
functions (written as plain Java code):


First-class functions: Consume more memory \& time compared to native
methods. 

% some old text

To build an
efficient compiler for a functional language requires compiler writers
to implement a number of of non-trivial optimizations. In some
settings, such as the JVM, an additional problem is that restrictions
on the supported bytecode makes it very hard to implement standard
(functional) optimizations such as proper tail calls.

For a compiler writer, the challenge is how to efficiently support
that programming style.

in a platform that did not account for such features in the first place. 

Two Problems with the JVM as a target for functional languages:

\begin{itemize}

\item How to represent and implement first-class functions, currying and 
partial function application?

Methods calls do not provide good support for currying.

\item How to eliminate and/or optimize tail calls?

\end{itemize}

% What are the existing solutions?

What do current functional languages do? Essentially they implement 
a number of compiler optimizations (such as tail-recursion optimization). 
They 

% What is our solution and how does it improve on the state-of-the-art?

\end{comment}

\paragraph{Contributions.} 
This paper presents a new JVM compilation technique for
functional programs, and creates an efficient implementation of System
F~\cite{girard72dissertation,reynolds74towards} using the new
technique. The compilation technique builds on a new representation of
first-class functions in the JVM: \emph{imperative functional
  objects} (IFOs). IFOs are representations of
functions that support currying, partial function applications, and TCE. 
\emph{With IFOs it is possible to use a single
 representation of functions in the JVM and still achieve reasonable
efficiency for FP}. Thus, IFOs provide a good solution for the two problems
mentioned above.

We represent an IFO by the following abstract class:

\begin{lstlisting}
abstract class Function { 
   Object arg, res;
   abstract void apply();
}
\end{lstlisting}

\noindent With IFOs, we encode both the argument
(\lstinline{arg}) and the result of the functions (\lstinline{res})
as mutable fields. 
We set the argument field before
invoking the \lstinline{apply()} method. At the end of the \lstinline{apply()} method, we set  the result field. 
An important difference between the IFOs and FAOs encoding of 
first-class functions is that, in IFOs, \emph{function application is
divided in two parts}: \emph{setting the argument field}; and \emph{invoking the
apply method}.
%\emph{function application
%  is no longer an atomic operation} 
For example, if we have a function call
\lstinline{factorial 10}, the corresponding Java code using IFOs
is:

\begin{lstlisting}
factorial.arg = 10;  // setting argument
factorial.apply();      // invoking function
\end{lstlisting} 

The fact that we can split function application into two parts is key
to enable new optimizations related to functions
in the JVM. In particular, this paper focuses on showing how this
allows a new and more efficient way to do TCE. The new TCE
approach does not require memory allocation for each tail
call and is faster than the JVM-style trampolines used in languages
such as Clojure and Scala. Essentially, with IFOs, it is possible to
provide a straightforward TCE implementation, resembling Steele's ``UUO
handler''~\cite{Steele1978}, in the JVM.
% Moreover it consistently beats trampoline-based implementations 
%of functional programs in terms of time performance. 
 
Using IFOs and the new TCE technique, we created
\Name: an efficient JVM implementation of an extension of \emph{System
  F}. 
\name aims to serve as an intermediate functional layer on top
of the JVM, which ML-style languages can target. 
Compiler writers can simply translate from their source
functional language into \name and get a number of FP 
optimizations for free!  This liberates compiler
writers from implementing a number of tedious and difficult compiler
optimizations by themselves, allowing them to focus on the development
of their source languages.


Our experimental results show that \name programs perform
competitively against programs using regular JVM methods,
while still supporting TCE. Programs in \name are also
faster and use less memory than programs using conventional 
JVM trampolines. 
%Moreover, programs written in \name are free 
%of the compromises needed by other functional languages on the JVM.
Therefore, \name has the performance needed to serve as a target for a 
source language.
%So far, the only complete implementation of System F in a
%JVM-like platform that we know of was done in the .Net
%platform~\cite{Kennedy2004}. This implementation accounts for full
%System F, but due to the use of FAOs, it
%is too inefficient to be used as a target language of a
%compiler. Furthermore TCE is not supported.  In
%contrast, functional programs using \name provide competitive
%performance and support TCE. 


%Moreover, the essential parts of the translation process of
%\name into Java are formalized and implemented.

In summary, the contributions of this paper are:

\begin{itemize}

\item \Name: An efficient implementation of a System F-based
  intermediate language that can be used to target the JVM by FP
  compilers.

\item {\bf Imperative Functional Objects:} A new representation of
  first-class functions in the JVM, offering new ways to
  optimize functional programs.

\item {\bf A new approach to tail-call elimination:} A new way to
  implement TCE in the JVM using IFOs. 

\item {\bf Formalization and empirical results:} Our compilation 
  method from a subset of \name into Java is formalized. Our empirical
  results indicate that the performance of \name is competitive with regular JVM methods 
  and significantly better than existing TCE approaches.

 
%%\item {\bf A new optimization for multi-argument (first-class)
%%    functions in the JVM:}

\end{itemize}
