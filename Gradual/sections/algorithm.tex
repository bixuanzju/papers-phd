\section{Algorithmic Type System}
\label{sec:algorithm}

\begin{figure}[t]
  \centering
  \begin{small}
\begin{tabular}{lrcl} \toprule
  Expressions & $e$ & \syndef & $x \mid n \mid
                         \blam x A e \mid \erlam x e \mid e~e \mid e : A $ \\
  Types & $A, B$ & \syndef & $ \nat \mid a \mid \genA \mid A \to B \mid \forall a. A \mid \unknown$ \\
  Monotypes & $\tau, \sigma$ & \syndef & $ \nat \mid a \mid \genA \mid \tau \to \sigma$ \\
  Contexts & $\Gamma, \Delta, \Theta$ & \syndef & $\ctxinit \mid \tctx,x: A \mid \tctx, a \mid \tctx, \genA \mid \tctx, \genA = \tau$ \\
  Complete Contexts & $\Omega$ & \syndef & $\ctxinit \mid \Omega,x: A \mid \Omega, a \mid \Omega, \genA = \tau$ \\ \bottomrule
\end{tabular}
  \end{small}
\caption{Syntax of the algorithmic system}
\label{fig:algo-syntax}
\end{figure}


% The declarative type system in \cref{sec:type-system} serves as a good
% specification for how typing should behave. It remains to see whether this
% specification delivers an algorithm. The main challenge lies in the rules \rul{CS-ForallL} in
% \cref{fig:decl:conssub} and rule \rul{M-Forall} in
% \cref{fig:decl-typing}, which both need to guess a monotype.

% \bruno{why are we not highlightinh the differences in gray anymore?}
In this section we give a bidirectional account of the algorithmic type system
that implements the declarative specification. The algorithm is largely inspired
by the algorithmic bidirectional system of \citet{dunfield2013complete}
(henceforth DK system). However our algorithmic system differs from theirs in
three aspects: 1) the addition of the unknown type $\unknown$; 2) the use of the
matching judgment; and 3) the approach of \textit{gradual inference only
  producing static types}~\citep{garcia2015principal}. We then prove that our
algorithm is both sound and complete with respect to the declarative type
system. Full proofs can be found in the appendix.

\paragraph{Algorithmic Contexts.}

The algorithmic context $\Gamma$ is an
\textit{ordered} list containing declarations of type variables $a$ and term
variables $x : A$. Unlike declarative contexts, algorithmic contexts also
contain declarations of existential type variables $\genA$, which can be either
unsolved (written $\genA$) or solved to some monotype (written $\genA = \tau$).
Complete contexts $\Omega$ are those that contain no unsolved existential type
variables. \Cref{fig:algo-syntax} shows the syntax of the algorithmic system.
Apart from expressions in the declarative system, we have annotated expressions
$e : A$.

% \paragraph{Notational convenience}
% Following \citet{dunfield2013complete}, we use contexts as substitutions on
% types. We write $\ctxsubst{\Gamma}{A}$ to mean $\Gamma$ applied as a
% substitution to type $A$. We also use a hole notation, which is useful when
% manipulating contexts by inserting and replacing declarations in the middle. The
% hole notation is used extensively in proving soundness and completeness. For
% example, $\Gamma[\Theta]$ means $\Gamma$ has the form $\Gamma_L, \Theta,
% \Gamma_R$; if we have $\Gamma[\genA] = (\Gamma_L, \genA, \Gamma_R)$, then
% $\Gamma[\genA = \tau] = (\Gamma_L, \genA = \tau, \Gamma_R)$.

% \paragraph{Input and output contexts}
% The algorithmic system, compared with the declarative system, includes similar
% judgment forms, except that we replace the declarative context $\Psi$ with an
% algorithmic context $\Gamma$ (the \textit{input context}), and add an
% \textit{output context} $\Delta$ after a backward turnstile. For example,
% $\Gamma \vdash A \tconssub B \dashv \Delta$ is the judgment form for the
% algorithmic consistent subtyping, and so on. All rules manipulate input and
% output contexts in a way that is consistent with the notion of \textit{context
%   extension}, which is described in \cref{sec:ctxt:extension}.

% We start with the explanation of the algorithmic consistent subtyping as it
% involves manipulating existential type variables explicitly (and solving them if
% possible).

\subsection{Algorithmic Consistent Subtyping and Instantiation}
\label{sec:algo:subtype}

\begin{figure}[t]
  \centering
  \begin{small}
  %   \begin{mathpar}
  % \framebox{$\Gamma \vdash A$} \\
  % \VarWF \and \IntWF \and \UnknownWF \and \FunWF \and \ForallWF \and \EVarWF
  % \and \SolvedEVarWF
  %   \end{mathpar}

\begin{mathpar}
  \framebox{$\Gamma \vdash A \tconssub B \toctxr$} \\
  \ACSTVar \and \ACSExVar \and \ACSInt \quad \ACSUnknownL \quad \ACSUnknownR \and
  \ACSFun \and \ACSForallR \and \ACSForallL \and \AInstantiateL \quad \AInstantiateR
\end{mathpar}
  \end{small}
  \caption{Algorithmic consistent subtyping}
  \label{fig:algo:subtype}
\end{figure}

\Cref{fig:algo:subtype} shows the algorithmic consistent subtyping rules.
The first five rules do not manipulate contexts. % Rules \rul{ACS-TVar} and
% \rul{ACS-Int} do not involve existential variables, so the output context
% remains unchanged. Rule \rul{ACS-ExVar} says that any unsolved existential
% variable is a consistent subtype of itself. The output is still the same as the
% input context as this gives no clue as to what is the solution of that
% existential variable.
% Rules \rul{ACS-UnknownL} and \rul{ACS-UnknownR} are the verbatim
% correspondences of rule \rul{CS-UnknownL} and \rul{CS-UnknownR}.
Rule \rul{ACS-Fun} is a natural extension of its declarative counterpart. The
output context of the first premise is used by the second premise, and the
output context of the second premise is the output context of the conclusion.
Note that we do not simply check $A_2 \tconssub B_2$, but apply $\Theta$
% (the input context of the second premise)
to both types (e.g., $\ctxsubst{\Theta}{A_2} $). This is
to maintain an important invariant that types
% : whenever we try to derive $\Gamma \vdash A \tconssub B \dashv \Delta$, the types $A$ and $B$
are fully applied
under input context $\Gamma$ (they contain no existential variables already solved in
$\Gamma$). The same invariant applies to every algorithmic judgment.
Rule \rul{ACS-ForallR} looks similar to its declarative counterpart, except that
we need to drop the trailing context $a, \Theta$ from the concluding output
context since they become out of scope.
% again, bears a similarity with the declarative
% version. Note that the output context of its premise allows additional elements
% to appear after the type variable $a$, in a trailing context $\Theta$. Since $a$
% becomes out of scope in the conclusion, we need to drop the trailing context
% $\Theta$ together with $a$ from the concluding output context, resulting in
% $\Delta$.
% The next rule is essential to eliminating the guessing work, thus appears
% significantly different from its declarative version. Instead of guessing a
% monotype $\tau$ out of thin air,
Rule \rul{ACS-ForallL} generates a fresh
existential variable $\genA$, and replaces $a$ with $\genA$ in the body $A$. The
new existential variable $\genA$ is then added to the premise's input context.
% Unlike rule \rul{ACS-ForallR}, the output context $\Delta$ of the premise
% remains unchanged in the conclusion.
% A central idea behind this rule is that we
% defer the decision of choosing a monotype for a type variable, and hope that it
% could be solved later when we have more information at hand.
As a side note, when both types are quantifiers, then either \rul{ACS-ForallR}
or \rul{ACS-ForallR} could be tried. In practice, one can apply
\rul{ACS-ForallR} eagerly.
The last two rules % are specific to the algorithm, thus having no counterparts in
% the declarative version. They
together check consistent subtyping with an
unsolved existential variable on one side and an arbitrary type on the other
side by the help of the instantiation judgment. % Apart from checking that the existential variable does not occur in the
% type $A$, both of the rules do not directly solve the existential variables, but
% leave the real work to the instantiation judgment.

% \subsection{Instantiation}
% \label{sec:algo:instantiate}

\begin{figure}[t]
  \centering
  \begin{small}
\begin{mathpar}
  \framebox{$\tctx \vdash \genA \unif A \toctxr$} \\
  % {\quad \text{Under input context $\Gamma$, instantiate $\genA$ such that
  %     $\genA \tconssub A$, with output context $\Delta$ }} \\
  \InstLSolve \and \InstLReach \and \InstLSolveU   \and \InstLAllR \and \InstLArr
\end{mathpar}

% \begin{mathpar}
%   \framebox{$\tctx \vdash A \unif \genA  \toctxr$} \\
%   % {\quad \text{Under input context $\Gamma$, instantiate $\genA$ such that
%   %     $A \tconssub \genA$, with output context $\Delta$}} \\
%   \InstRSolve \and \InstRReach \and \InstRSolveU  \and \InstRAllL \and \InstRArr
% \end{mathpar}

  \end{small}
  \caption{Algorithmic instantiation}
  \label{fig:algo:instantiate}
\end{figure}

% A central idea of the algorithmic system is to defer the decision of picking a
% monotype to as late as possible.
The judgment $\Gamma \vdash \genA \unif A \dashv \Delta$ defined in
\cref{fig:algo:instantiate} instantiates unsolved existential variables.
Judgment $\genA \unif A$ reads ``instantiate $\genA$ to a consistent subtype of
$A$''. For space reasons, we omit its symmetric judgement $\Gamma \vdash A \unif
\genA \dashv \Delta$.
% Since these two are mutually defined, we
% discuss them together, and omit symmetric rules when convenient.
Rule \rul{InstLSolve} and rule \rul{InstLReach} set $\genA$ to
$\tau$ and $\genB$ in the output context, respectively.
% is the simplest
% one -- when an existential variable meets a monotype. In that case, we simply
% set the solution of $\genA$ to the monotype $\tau$ in the output context. We
% also need to check that the monotype $\tau$ is well-formed under the prefix
% context $\Gamma$.
Rule \rul{InstLSolveU} is similar to \rul{ACS-UnknownR} in that we put no
constraint on $\genA$ when it meets the unknown type $\unknown$. This design
decision reflects the point that type inference only produces static
types~\citep{garcia2015principal}. We will get back to this point in
\cref{subsec:algo:discuss}.
% Rule \rul{InstLReach} deals with the situation where two existential variables
% meet. Note that $\Gamma[\genA][\genB]$ denotes a context where some unsolved existential
% variable $\genA$ is declared before $\genB$. In this situation, the only logical
% thing we can do is to set the solution of one existential variable to the other
% one, depending on which is declared before which. For example, in the output
% context of rule \rul{InstLReach}, we have $\genB = \genA$ because in the input
% context, $\genA$ is declared before $\genB$.
Rule \rul{InstLAllR} is the instantiation version of rule \rul{ACS-ForallR}.
% Since our system is predicative, $\genA$ cannot be instantiated to $\forall b.
% B$, but we can decompose $\forall b. B$ in the same way as in \rul{ACS-ForallR}.
% Rule \rul{InstRAllL} is the instantiation version of rule \rul{ACS-ForallL}.
The last rule \rul{InstLArr} applies when $\genA$ meets a function type. It
follows that the solution must also be a function type.
% looks a bit complicated, but it is actually very
% intuitive: what does the solution of $\genA$ look like when $A$ is a function
% type? The solution must also be a function type!
That is why, in the first premise, we generate two fresh existential variables
$\genA_1$ and $\genA_2$, and insert them just before $\genA$ in the input
context, so that the solution of $\genA$ can mention them. Note that $A_1 \unif
\genA_1$ switches to the other instantiation judgment.


% \paragraph{Example}

% We show a derivation of $\Gamma[\genA] \vdash \forall b. b \to \unknown \unif
% \genA$ to demonstrate the interplay between instantiation, quantifiers and the
% unknown type:
% \[
%   \inferrule*[right=InstRAllL]
%       {
%         \inferrule*[right=InstRArr]
%         {
%           \inferrule*[right=InstLReach]{ }{\Gamma', \genB \vdash \genA_1 \unif \genB \dashv \Gamma' , \genB = \genA_1} \\
%           \inferrule*[right=InstRSolveU]{ }{\Gamma', \genB = \genA_1 \vdash \unknown \unif \genA_2 \dashv \Gamma', \genB = \genA_1}
%         }
%         {
%           \Gamma[\genA], \genB \vdash \genB \to \unknown \unif \genA \dashv \Gamma', \genB = \genA_1
%         }
%       }
%       {
%         \Gamma[\genA] \vdash \forall b. b \to \unknown \unif \genA \dashv \Gamma', \genB = \genA_1
%       }
% \]
% where $\Gamma' = \Gamma[\genA_2, \genA_1, \genA = \genA_1 \to \genA_2]$. Note
% that in the output context, $\genA$ is solved to $\genA_1 \to \genA_2$, and
% $\genA_2$ remains unsolved because the unknown type $\unknown$ puts no
% constraint on it. Essentially this means that the solution of $\genA$ can be any
% function, which is intuitively correct since $\forall b. b \to \unknown$ can be
% interpreted, from the parametricity point of view, as any function.

\subsection{Algorithmic Typing}
\label{sec:algo:typing}

\begin{figure}[t]
  \centering
  \begin{small}
\begin{mathpar}
  \framebox{$\Gamma \vdash e \Rightarrow A \toctxr $} \\
  % {\quad \text{Under input context $\Gamma$, $e$ synthesizes output type $A$,
  %     with output context $\Delta$}} \\
  \AVar \and \ANat \and \ALamU \and \ALamAnnA \and \AAnno \and \AApp
\end{mathpar}
\begin{mathpar}
  \framebox{$\Gamma \vdash e \Leftarrow A \toctxr $} \\
  % {\quad \text{Under input context $\Gamma$, $e$ synthesizes output type $A$,
  %     with output context $\Delta$}} \\
  \ALam \and \AGen \and \ASub
\end{mathpar}
\begin{mathpar}
  \framebox{$\Gamma \vdash A \match A_1 \to A_2 \toctxr$} \\
  % {\quad \text{Under input context $\Gamma$, $A$ synthesizes output type $A_1
  %     \to A_2$, with output context $\Delta$}} \\
  \AMMC \quad \AMMA \and \AMMB \and \AMMD
\end{mathpar}
  \end{small}
  \caption{Algorithmic typing}
  \label{fig:algo:typing}
\end{figure}

We now turn to the algorithmic typing rules in \cref{fig:algo:typing}. The
algorithmic system uses bidirectional type checking to accommodate polymorphism.
Most of them are quite standard.
% All of them are direct analogies of their declarative counterparts. Rules \rul{AVar}
% and \rul{ANat} do not generate any new information, thus the output context is
% the same as the input context. Rule \rul{ALamAnnA} infers the type of a lambda
% abstraction. It does so by pushing $x : A$ into the input context and continues
% to infer the type of the body $B$. The output context in the premise has
% additional declarations in the trailing context $\Theta$, which is discarded in
% the concluding output context.
Perhaps rule \rul{AApp} (which differs significantly from that in the DK system)
deserves attention. It relies on the algorithmic matching judgment $\Gamma
\vdash A \match A_1 \to A_2 \dashv \Delta$.
% The matching judgment
% algorithmically synthesizes a function type from an arbitrary type.
Rule
\rul{AM-ForallL} replaces $a$ with a fresh existential variable $\genA$, thus
eliminating guessing. Rule \rul{AM-Arr} and \rul{AM-Unknown} correspond
directly to the declarative rules.
% self-explanatory. Rule
% \rul{AM-Unknown} says that the unknown type $\unknown$ can be split into a
% function type $\unknown \to \unknown$.
Rule \rul{AM-Var}, which has no
corresponding declarative version, is similar to \rul{InstRArr}/\rul{InstLArr}:
we create $\genA$ and $\genB$ and add $\genC = \genA \to \genB$ to the context.

% Back to \rul{AApp}. This rule first infers the type of $e_1$, producing a output
% context $\Theta_1$. Then it applies $\Theta_1$ to $A$ and goes into the matching
% judgment, which delivers a function type $A_1 \to A_2$ and another output
% context $\Theta_2$. $\Theta_2$ is used as the input context when inferring the
% type of $e_2$. The last premise algorithmically checks if
% $\ctxsubst{\Theta_3}{A_3}$ is a consistent subtype of
% $\ctxsubst{\Theta_3}{A_1}$. $A_2$ and $\Delta$ are the concluding output type
% and the concluding output context, respectively.


% \section{Soundness and Completeness}
% \label{sec:sound:complete}

% To be confident that our algorithmic type system and the declarative type system
% accept exactly the same programs, we need to prove that the algorithmic rules
% are sound and complete with respect to the declarative specifications. Before we
% give the formal statements of the soundness and completeness theorems, we need a
% meta-theoretical device, called \textit{context extension}~\cite{dunfield2013complete}, to help capture a notion of
% information increase from input contexts to output contexts.

% \subsection{Context Extension}
% \label{sec:ctxt:extension}


% A context extension judgment $\Gamma \exto \Delta$ reads ``$\Gamma$ is extended
% by $\Delta$''. Intuitively, this judgment says that $\Delta$ has at least as
% much information as $\Gamma$: some unsolved existential variables in $\Gamma$
% may be solved in $\Delta$. (The full inductive definition can be found in the
% supplementary material. We refer the reader to \citet[][Section
% 4]{dunfield2013complete} for further explanations of context extension.)

\subsection{Completeness and Soundness}

We prove that the algorithmic rules are sound and complete with
respect to the declarative specifications. We need an auxiliary judgment
$\Gamma \exto \Delta$ that captures a notion of information increase from input
contexts $\Gamma$ to output contexts $\Delta$~\citep{dunfield2013complete}.

\paragraph{Soundness.} Roughly speaking, soundness of the algorithmic system says
that given an expression $e$ that type checks in the algorithmic system, there exists
a corresponding expression $e'$ that type checks in the declarative system.
However there is one complication: $e$ does not necessarily have more annotations
than $e'$. For example, by \rul{ALam} we have $\erlam{x}{x} \chkby (\forall a.
a) \rightarrow (\forall a . a)$, but $\erlam{x}{x}$ itself cannot have type
$(\forall a. a) \rightarrow (\forall a . a)$ in the declarative system. To
circumvent that, we add an annotation to the lambda abstraction, resulting in
$\blam{x}{(\forall a . a)}{x}$, which is typeable in the declarative system with
the same type. To relate $\erlam{x}{x}$ and $\blam{x}{(\forall a . a)}{x}$, we
erase all annotations on both expressions. The definition of erasure $\erase{\cdot}$ is
standard and thus omitted.

% \jeremy{mention erasure and why (talk about \rul{ALam} and \rul{ASub})}


% \begin{restatable}[Instantiation Soundness]{mtheorem}{instsoundness} \label{thm:inst_soundness}%
%   Given $\Delta \exto \Omega$ and $\ctxsubst{\Gamma}{A} = A$ and $\genA \notin \mathit{fv}(A)$:
%   \begin{itemize}
%   \item If $\Gamma \vdash \genA \unif A \dashv \Delta$ then $\ctxsubst{\Omega}{\Delta} \vdash \ctxsubst{\Omega}{\genA} \tconssub \ctxsubst{\Omega}{A}$.
%   \item If $\Gamma \vdash A \unif \genA \dashv \Delta$ then $\ctxsubst{\Omega}{\Delta} \vdash \ctxsubst{\Omega}{A} \tconssub \ctxsubst{\Omega}{\genA}$.
%   \end{itemize}
% \end{restatable}

% Notice that the declarative judgment uses $\ctxsubst{\Omega}{\Delta}$, a
% operation that applies a complete context $\Omega$ to the algorithmic context
% $\Delta$, essentially plugging in all known solutions and removing all
% declarations of existential variables (both solved and unsolved), resulting in a
% declarative context.

% With instantiation soundness, next we show that the algorithmic consistent
% subtyping is sound:

% \begin{restatable}[Soundness of Algorithmic Consistent Subtyping]{mtheorem}{subsoudness} \label{thm:sub_soundness}%
%   If $\Gamma \vdash A \tconssub B \toctxr$ where $\ctxsubst{\tctx}{A} = A$ and
%   $\ctxsubst{\tctx}{B} = B$ and $\ctxr \exto \cctx$ then
%   $\ctxsubst{\cctx}{\Delta} \vdash \ctxsubst{\cctx}{A} \tconssub
%   \ctxsubst{\cctx}{B}$.
% \end{restatable}

% At this point, we are ``two thirds of the way'' to proving the ultimate theorem.
% The remaining third concerns with the soundness of matching:

% \begin{restatable}[Matching Soundness]{mtheorem}{matchsoundness}  \label{thm:match_soundness}%
%   If $\Gamma \vdash A \match A_1 \to A_2 \dashv \Delta$ where
%   $\ctxsubst{\Gamma}{A} = A$ and $\Delta \exto \Omega$ then
%   $\ctxsubst{\Omega}{\Delta} \vdash \ctxsubst{\Omega}{A} \match
%   \ctxsubst{\Omega}{A_1} \to \ctxsubst{\Omega}{A_2}$.
% \end{restatable}


% Finally the soundness theorem of algorithmic typing is:

\begin{restatable}[Soundness of Algorithmic Typing]{mtheorem}{typingsoundness} \label{thm:type_sound}
  Given $\ctxr \exto \cctx$,

  \begin{enumerate}
  \item If $\Gamma \vdash e \infto A \toctxr$ then $\exists e'$ such
    that $\ctxsubst{\cctx}{\Delta} \vdash e' : \ctxsubst{\cctx}{A}$ and
    $\erase{e} = \erase{e'}$.
  \item If $\Gamma \vdash e \chkby A \toctxr$ then $\exists e'$ such
    that $\ctxsubst{\cctx}{\Delta} \vdash e' : \ctxsubst{\cctx}{A}$ and
    $\erase{e} = \erase{e'}$.
  \end{enumerate}


\end{restatable}


\paragraph{Completeness.}
Completeness of the algorithmic system is the reverse of soundness: given a
declarative judgment of the form $\ctxsubst{\Omega}{\Gamma} \vdash
\ctxsubst{\Omega} \dots $, we want to get an algorithmic derivation of $\Gamma
\vdash \dots \dashv \Delta$. It turns out that completeness is a bit trickier to
state in that the algorithmic rules generate existential variables on the fly,
so $\Delta$ could contain unsolved existential variables that are not found in
$\Gamma$, nor in $\Omega$. Therefore the completeness proof must produce another
complete context $\Omega'$ that extends both the output context $\Delta$, and
the given complete context $\Omega$. As with soundness, we need erasure to
relate both expressions.

% \jeremy{talk about \rul{Gen}}

% \begin{restatable}[Instantiation Completeness]{mtheorem}{instcomplete}  \label{thm:inst_complete}%
%   Given $\Gamma \exto \Omega$ and $A = \ctxsubst{\Gamma}{A}$ and $\genA \in
%   \mathit{unsolved}(\Gamma)$ and $\genA \notin \mathit{fv}(A)$:
%   \begin{enumerate}
%   \item If $\ctxsubst{\Omega}{\Gamma} \vdash \ctxsubst{\Omega}{\genA} \tconssub
%     \ctxsubst{\Omega}{A}$ then there exist $\Delta$, $\Omega'$ such that $\Omega \exto
%     \Omega'$ and $\Delta \exto \Omega'$ and $\Gamma \vdash \genA \unif A \dashv \Delta$.
%   \item If $\ctxsubst{\Omega}{\Gamma} \vdash \ctxsubst{\Omega}{A} \tconssub
%     \ctxsubst{\Omega}{\genA}$ then there exist $\Delta$, $\Omega'$ such that $\Omega \exto
%     \Omega'$ and $\Delta \exto \Omega'$ and $\Gamma \vdash A \unif \genA \dashv \Delta$.
%   \end{enumerate}
% \end{restatable}


% Next is the completeness of consistent subtyping:

% \begin{restatable}[Generalized Completeness of Subtyping]{mtheorem}{subcomplete}  \label{thm:sub_completeness}%
%   If $\Gamma \exto \Omega$ and $\Gamma \vdash A$ and $\Gamma \vdash B$ and
%   $\ctxsubst{\Omega}{\Gamma} \vdash \ctxsubst{\Omega}{A} \tconssub
%   \ctxsubst{\Omega}{B}$ then there exist $\Delta$, $\Omega'$ such that $\Delta
%   \exto \Omega'$ and $\Omega \exto \Omega'$ and $\Gamma \vdash
%   \ctxsubst{\Gamma}{A} \tconssub \ctxsubst{\Gamma}{B \dashv \Delta}$.
% \end{restatable}


% We prove that the algorithmic matching is complete with respect to the
% declarative matching:

% \begin{restatable}[Matching Completeness]{mtheorem}{matchcomplete} \label{thm:match_complete}%
%   Given $\Gamma \exto \Omega$ and $\Gamma \vdash A$, if
%   $\ctxsubst{\Omega}{\Gamma} \vdash \ctxsubst{\Omega}{A} \match A_1 \to A_2$
%   then there exist $\Delta$, $\Omega'$, $A_1'$ and $A_2'$ such that $\Gamma
%   \vdash \ctxsubst{\Gamma}{A} \match A_1' \to A_2' \dashv \Delta$ and $\Delta \exto \Omega'$ and
%   $\Omega \exto \Omega'$ and $A_1 = \ctxsubst{\Omega'}{A_1'}$ and $A_2 =
%   \ctxsubst{\Omega'}{A_2'}$.
% \end{restatable}


% Finally here is the completeness theorem of the algorithmic typing:

\begin{restatable}[Completeness of Algorithmic Typing]{mtheorem}{typingcomplete}  \label{thm:type_complete}
  Given $\Gamma \exto \Omega$ and $\Gamma \vdash A $, if
  $\ctxsubst{\Omega}{\Gamma} \vdash e : A$ then there exist $\Delta$,
  $\Omega'$, $A'$ and $e'$ such that $\Delta \exto \Omega'$ and $\Omega \exto \Omega'$
  and $\Gamma \vdash e' \infto A' \dashv \Delta$ and $A = \ctxsubst{\Omega'}{A'}$ and $\erase{e} = \erase{e'}$.
\end{restatable}





%%% Local Variables:
%%% mode: latex
%%% TeX-master: "../paper"
%%% org-ref-default-bibliography: "../paper.bib"
%%% End:
