\section{Discussion of Variants of the Type System}

In \cref{sec:type-system} we present our declarative system in terms of the
matching and consistent subtyping judgments. The rationale behind this design
choice is that the result declarative system is syntax-directed, thus making it
easier to design an algorithmic system accordingly. Still, one may wonder if it
is possible to design a more declarative specification. For example, even though
we mentioned that the subsumption rule is incompatible with consistent
subtyping, it might be possible that accommodating a subsumption rule for normal subtyping
(instead of consistent subtyping) would work. In this section, we discuss two
alternatives for the design of the declarative system.

\subsection{Variant I: The Naive Design}

A first obvious attempt is to replace the rule \rul{App} in \cref{fig:decl-typing} with the
following two rules:
\begin{mathpar}
  \inferrule{
          \tctx \byinf e : A 
       \\ A \tsub B
          }{
          \tctx \byinf e : B
          }\rname{V-Sub}

  \inferrule{
          \tctx \byinf e_1 : A 
       \\ \tctx \byinf e_2 : A_1
       \\ A \sim A_1 \to A_2
          }{
          \tctx \byinf e_1 ~ e_2 : A_2
          }\rname{V-App1}
\end{mathpar}
Rule \rul{V-Sub} is the standard subsumption rule: if an expression $e$ has type
$A$, then it can be assigned some type $B$ that is a supertype of $A$. Rule
\rul{V-App1} first infers that $e_1$ has type $A$, and $e_2$ has type $A_1$, then
it finds some $A_2$ so that $A$ is consistent with $A_1 \to A_2$.

There would be two obvious benefits of this variant if it did work:
firstly this approach closely resembles
the traditional declarative type systems for calculi with subtyping; secondly it
saves us from discussing various forms of $A$ in \rul{V-App1}, leaving the job
to the consistency judgment.

As intuitive as this variant may seem, it actually does not work! The reason lies
in the information loss caused by the choice of $A_2$ in \rul{V-App1}. Suppose
we have $\mathsf{succ} : \nat \to \nat$, and we apply it to $1$ to get
$\mathsf{succ}~{1}$. According to \rul{V-App1}, we have $\nat \to \nat \sim \nat
\to \bool$, which means that we have $\mathsf{succ}~1 : \bool$. This is obviously
wrong! The moral of the story is that we should be very careful when using type consistency.
We hypothesize that it is inevitable to do case analysis for the type of the
function in an application (i.e., $A$ in \rul{V-App1}).


\subsection{Variant II}

The second variant refines the first variant by carefully separating
\rul{V-App1} into two cases based on the typing result of $e_1$: (1) when it is
an arrow type, and (2) when it is an unknown type. In this variant, we replace
the rule \rul{App} in \cref{fig:decl-typing} with the following three rules:
\begin{mathpar}
  \inferrule{
          \tctx \byinf e : A 
       \\ A \tsub B
          }{
          \tctx \byinf e : B
          }\rname{V-Sub}

  \inferrule{
          \tctx \byinf e_1 : \unknown
       \\ \tctx \byinf e_2 : A
          }{
          \tctx \byinf e_1 ~ e_2 : \unknown
          }\rname{V-App2}

  \inferrule{
          \tctx \byinf e_1 : A_1 \to A_2 
       \\ \tctx \byinf e_2 : A_3
       \\ A_1 \sim A_3
          }{
          \tctx \byinf e_1 ~ e_2 : A_2
          }\rname{V-App3}
\end{mathpar}
Rule \rul{V-Sub} is the same as the first variant. In rule \rul{V-App2}, we
infer that $e_1$ has type $\unknown$. Under this situation, $e_2$ can have any
type $A$, and the application result is of type $\unknown$. In rule
\rul{V-App3}, we infer that $e_1$ has an arrow type $A_1 \to A_2$, then we need to
ensure that the argument type $A_3$ is consistent with $A_1$, and use $A_2$ as
the result type of the application. These two application rules are closely
related to the ones in \citet{siek2006gradual} and \citet{siek2007gradual}.

It turns out that the new system with these rules actually works. In the
following we discuss the relation between this new type system and the one we
presented in \cref{sec:type-system}.

Firstly, this system indeed provides a more declarative specification without
using the consistent subtyping or matching judgment. However, the more
declarative nature of this system also implies that it is highly non-syntax
directed, in the sense that rule \rul{V-Sub} can be applied anywhere.

% Meanwhile, the declarative system in \cref{sec:type-system}, making use of
% consistent subtyping and matching, is syntax-directed, and is an important step
% towards an algorithm of the type system.

Secondly, we have proved in Coq the following lemmas to establish soundness and
completeness of this system with respect to our original system (to avoid
ambiguity, we use the notation $\byminf$ to indicate the more declarative
version):
\begin{clemma}[Completeness of $\byminf$]
  If $\tctx \byinf e : A$, then $\tctx \byminf e : A$.
\end{clemma}
\begin{clemma}[Soundness of $\byminf$]
  If $\tctx \byminf e : A$, then there exists some $B$, such that $\tctx \byinf e : B$ and $\tctx \bysub B \tsub A$.
\end{clemma}

%%% Local Variables:
%%% mode: latex
%%% TeX-master: "../paper"
%%% org-ref-default-bibliography: ../paper.bib
%%% End: